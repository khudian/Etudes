\magnification=1200
\baselineskip-14pt

\def\e{{\bf e}}
\def\f{{\bf f}}
\def\y{{\bf y}}
\def\p{\partial}
\def\a{\alpha}
\def\t{\tilde}
\def\G {{\cal G}}
\def\C {{\bf C}}
\def\vare{\varepsilon}
\def\t {{\bf t}}
\def\Tr {{\rm Tr\,}}
\def\G {{\cal G}}

   \centerline {\bf On Kac Peterson Formula; calculation of volumes of groups
   corresponding to algebras $sl(n)$}

\medskip

   Let $\G$ be simple Lie algebra. Then
Kac Peterson formula tells that
        volume $V(G)$ of compact connected simple connected group  
$G(\G)$ is defined by the formula,
      $$
V^2(G)=(8\pi)^2J(4\pi i\rho)\,,
       $$
where $\rho={1\over 2}\sum_{\a\in\triangle_+}\a$
is Weyl vector,and $J$ Jacobian of the map $\G\to G$:
      $$
J(x)=\det\left(1-e^{-ad_x}\over ad_x\right)
      $$

(We suppose that covectors and vectors are identified
via  Killing Cartan metric
       $$
 \phi(x,y)=-{\rm Tr}\left(ad_x ad_y\right)
       $$ 
$\phi=\langle\,,\,\rangle$)

 This implies that

          $$
V=(2\pi\sqrt 2)^{{ dim\,}\G}
\prod_{\a\in\triangle_+}
f(2\pi\phi(\rho,\a))
\,,\qquad {\rm where}\,\quad f(x)={\sin x\over x}\,. 
          $$
(See equation (4.32.1) in
 V.Kac, D.Peterson ``Infinite-dimensioanal Lie algebras, Theta functions
and modular forms'', Advances in Math,. {\bf 13}, pp.125---264 (1984))


Let us apply this formula to the most simple case $su(2)$,
then to $sl(n,C)$ (we mean calculate volume of corresponding simple
simply connected Lie groups.


     \centerline {volume of $SU(2)$})

Consider generators $\e_1,\e_2,\e_3$:
          $$
   [\e_i,\e_k]=\vare_{ikm}\e_m\,.
          $$
One can see that
         $$
   {\rm Tr}(ad_{e_i}ad_{e_k})=-2\delta_{ik}\,,\hbox
{this means that $\phi(\e_i,\e_k)=2\delta_{ik}$}
        $$
 Consider  Cartan algebra $h$ spanned by vector $\e_3$.
    We have that
         $$
  [i\e_3,\e_\pm]=\pm \e_\pm\,,\quad {\rm where}\,\, \e_\pm=\e_1\pm i\e_2\,. 
         $$
Thus we have two roots  $\a_+,\a_-$:
            $$
     \a_\pm\in h^*\,\colon \a_\pm(\e_1)=\pm {1\over i}\, 
        \, {\rm i.e.}\,\,                
\cases
             {
\a_+\in h^*\,,\quad \hbox{such that $\a_+(\e_1)= -i$}\cr 
\a_-\in h^*\,,\quad \hbox{such that $\a_+(\e_1)=i$}\cr
          }\,.
         $$
  We see vectors $\e_1,\e_2,\e_3$ have length $\sqrt 2$ 
and  covectors $\a_+, \a_-$ have length $1\over \sqrt 2$
(matrix of the Cartan-Killing metric in the basis 
  ${\e_1,\e_2,\e_3}$ is $\pmatrix{2 &0 & 0\cr 0&2 &0\cr 0 &0 &2 }$,)
Weyl vector $\rho={\a\over 2}$. Hence
        $$
  2\pi \phi(\rho,\a_+)=\pi|\a_+|^2 ={\pi\over 2}\,.
        $$ 
We come to
         $$
Vol(su(2))=(2\pi\sqrt 2)^{{ dim\,}\G}
\prod_{\a\in\triangle_+}
{\sin (2\pi\phi(\rho,\a))
      \over
2\pi\phi(\rho,\a)
       }=
        $$
        $$
=(2\pi\sqrt 2)^3
{\sin (2\pi\phi(\rho,\a))
      \over
2\pi\phi(\rho,\a)
       }=(2\pi\sqrt 2)^3
{\sin {\pi\over 2}
      \over
    {\pi\over 2}
           }=32\sqrt 2\pi^2
        $$
Notice that $su(2)$ is three-dimensional sphere
and volume of three-dimensional sphere is proportional to
  $\pi^2$: volume of sphere of radius $R$ is equal
to $2\pi R^3$: volume of $3$-dimensional sphere of radius $1$ is equal to
          $$
    Vol(S^3)={2\pi^{k+1\over 2}\over \Gamma\left(k+1\over2\right)}
\big\vert_{k=3}=2\pi^2
          $$ 
One can say that  volume of $su(2)$ is equal to the volume of $S^3$ with
radius $R=2\sqrt 2$. (it has to be clarified.)


  \bigskip

  First  we will study roots of algebra $sl(n,C)$.

            \centerline   Lie algebras $sl(n,C)$

  Denote by  $E_{ij}$ $n\times n$ matrix such that its all entries vanish except
 the intry in $i$-th column  amd $j$-th row which is equal to $1$.
   These matrices span over $C$ $gl(n,\C)$ Lie algebra.
Traceless matrices span Lie algebra $sl(n,\C)$.
   Denote by $\t H$  Cartan algebra of diagonal matrices in $gl(n,C)$
and respectively  by $H$ Cartan algebra of traceless diagonal
matrices in $sl(n,C)$.  
  Denote by $\a^{ij}, (i\not=j)$
linear functions on Lie algebra $\t H$ (i.e. elements of $\t H^*$)
such that 
                     $$
        \a^{ij}(t^m E_{mm}a=t^i-t^j\,.
                     $$

One can see  $\{\a^{ij}\}$ are roots --- they are equal to values of
observables (Cartan subalgebra $\t H$) on root vectors  $\{E_{ij}\}$:
                          $$
\forall \t\in  H, \t=t^mE_{mm}, \,\, \hat \t E_{ik}=[\t, E_{ik}]=
            (t^i-t^k)E_{ik}
                          $$
               $$
             \pmatrix
                   {
 \hbox{matrices $E_{ij}, (i>j)$ are positive weight vectors}\cr
 \hbox{matrices $E_{ij}, (i<j)$ are negative weight vectors}\cr
                  }
              $$ 
Teperj poschitajem metriku Cartana Killinga na algebre $gl(n,\C)$
ad its subalgebra $sl(n,\C)$.
                 $$
    \phi(X,Y)=\Tr (\hat X\circ \hat Y)
                 $$
where $\hat X=ad_X\colon \hat X Y=[X,Y]$.
           Notice that for every matrix $X$ for
coefficients of expansion we have:
                     $$
                X=X^{\pi\rho}E_{\pi\rho}\Rightarrow X^{\pi\rho}=
\Tr(XE_{\rho\pi})\,.
                     $$
   Hence for metric coefficients we have
                       $$
g_{ik|pq}=\Tr(\hat E_{ik}\circ \hat E_{pq})=
\Tr                   \left(
                      \left(
                      \left[E_{ik},
                      \left[E_{pq},E_{\a\beta}
                      \right]
                      \right]      
                      \right)E_{\beta\a}
                     \right)=
                       $$
               $$
        Tr
                 \left(
           E_{ik}E_{pq}E_{\a\beta}E_{\beta_\alpha}-
           E_{ik}E_{\a\beta}E_{pq}E_{\beta_\alpha}-
           E_{pq}E_{\a\beta}E_{ik}E_{\beta_\alpha}+
           E_{\a\beta}E_{pq}E_{ik}E_{\beta_\alpha}
                     \right)=
               $$
                $$
    2N\delta_{iq}\delta_{kp}-2\delta_{ik}\delta_{pq}\,.
                $$

We see that this metric is degenerate: identity matrix is zeroeigenvector,
in other words algebra $gl(n,\C)$ is not semisimple, it possesses
the centre. The corank of the metric is just one---the algebra
  $sl(n,\C)$ is semisimple.  Calculate Kartan-Killing on $sl(n,\C)$.
  For every $X\in sl(n,\C)$
                  $$
    \Tr X\big\vert_{gl(N,\C)}=\Tr X\big\vert_{sl(N,\C)}
                  $$
since $\hat X I=[X,I]=0$. Hence
metric is defined by the same formula. 

  Choose the basis in $gl(n,\C)$:
                  $$
        E_{pq}, p\not=q\,\, T_i=E_{ii}-E_{nn}, (i=1,\dots,n-1)\,.
                  $$
  Our next step to calculate scalar products of roots.
        We see from previous calculations that
non-zero metric entries are only the following:
       $$
\hbox {for every two indices $p,q$ such that $p\not=q$, 
$\phi(E_{pq},E_{qp})=2n$}\,
       $$
        $$
\hbox {for every two indices $i,j$}
\cases { 
\hbox {$\phi(T_i,T_j)=2n$ if $i\not=j$}\cr
\hbox {$\phi(T_i,T_j)=4n$ if $i=j$}\cr
       }
       $$
and all other entries vanish.

Choose the ordering 
              $$
  \{E_{21}, E_{31},E_{32},\dots,E_{n1},\dots,E_{n\, n-1},|
          T_1,T_2,\dots, T_{n-1}, 
      E_{12},\dots,E_{1n},E_{23},\dots,E_{2n},\dots, E_{n-1\, n}\}
                   $$
of basic vectors.
Then    we see that $(n^2-1)\times (n^2-1)$ matrix of
    Cartan-Killing metric has the following appearance 
                       $$
||G||\vert_{sl(n,\C)}=2n\pmatrix {0 & 0 & I\cr 
                               0 &K & 0\cr 
                              I & 0& 0}\,,
                       $$
where $I$ is ${n^2-n\over 2}\times {n^2-n\over 2}$
 unity matrix,a and
 $K$ is $n-1\times n-1$ matrix such that 
                   $$
             K=\pmatrix {
              2 &1 &1\dots &1 &1\cr
              1 &2 &1\dots &1 &1\cr
                \dots\cr
    1&1 &1 \dots    &1 &2\cr
                      }
                   $$
The inverse matrix (we need it to calculate  the scalar product of 
covectors (roots)) has the appearance                       
                     $$
||G^{-1}||\vert_{sl(n,\C)}={1\over 2n}
                 \pmatrix {0 & 0 & I\cr 
                               0 &K^{-1} & 0\cr 
                              I & 0& 0}\,,
                       $$
where $n-1\times n-1$ matrix $K^{-1}$ has the appearance: 
                   $$
             K^{-1}={1\over n}\pmatrix {
              n-1 &-1 
            &-1\dots &-1 &-1\cr
              -1 &n-1 &-1\dots 
             &-1 &-1\cr
                \dots\cr
    -1&-1 &-1 \dots  &-1
                     &n-1\cr
                      }
                   $$
Thus we will be able to calculate scalar producst of roots.


\bigskip


  \centerline    {\bf Roots, coroots}

  Calculate the components of roots $\a^{ik}$ in the basis $T_i$
(Recall that $T_i=E_{ii}-E_{nn}$).
Calculating $\a^{ik}(T_i)$ we come to components of roots, covectors:
               $$
\a^{12}=\pmatrix{\a^{ik}(T_1)\cr\a^{ik}(T_2)\cr
                       \a^{ik}(T_3\cr
                       \a^{ik}(T_4)\cr
                          \dots\cr
        \a^{ik}(T_{n-1})\cr\a^{ik}(T_n)\cr}=
             \pmatrix{1\cr-1\cr 0\cr 0\cr
                   \dots\cr
        0\cr0\cr}\,,
        $$
  For example for simple roots $\a^{ i,i+1}$ we have:
          $$
                     \a^{23}=
             \pmatrix{0\cr 1\cr-1\cr 0\cr
                   \dots\cr
                    0\cr0\cr}\,,
              \dots,
                \a^{n-2,n-1}=
             \pmatrix{0\cr 0\cr \dots \cr 0\cr
                   1\cr
                    -1 \cr 0\cr}\,,
               \a^{n-1,n}=
             \pmatrix{1\cr 1
\cr \dots\cr 1\cr
                   1\cr
        1 \cr 2\cr}\,,
              $$
Every positive (negative) root is combination of simple roots 
with positive (negative )integers, e.g.
                     $$
   \a^{36}=\a^{34}+\a^{45}+\a^{56}\,.
                     $$   
Cartan-Killing metric defines isomorphism between vectors and covectrs:
To every covector $f\in \G^*$ corresponds the vector $\xi_f\in \G$
such that
          $$
   \phi\left(\xi_f,x\right)=f(x)\,,\quad
\hbox{for every vector $x$}\,.
          $$ 
In particular define coroots $\a_{pq}$ dula to roots.
From the explicit expression for Cartan-Killing metric in the basis
   $\{E_{21},\dots,E_{n,n-1} T_{1}\dots T_{n-1},E_{12},\dots E_{n-1,n} \}$
we have that  
            $$
       \a_{pq}={1\over 2n}K^{-1}\a^{pq}\,.
            $$
In particular for simple roots we have
            $$
\a_{12}={1\over 2n^2}\pmatrix {
              n-1 &-1 
            &-1\dots &-1 &-1\cr
              -1 &n-1 &-1\dots 
             &-1 &-1\cr
                \dots\cr
    -1&-1&-1 \dots    &-1
                     &n-1\cr
                      }
             \pmatrix{1\cr -1\cr 0\cr 0\cr
                   \dots\cr
                    0\cr0\cr}=
{1\over 2n}\pmatrix{1, -1, 0, \dots, 0}\,..
            $$
     $$
\a_{23}={1\over 2n^2}\pmatrix {
              n-1 &-1 
            &-1\dots &-1 &-1\cr
              -1 &n-1 &-1\dots 
             &-1 &-1\cr
                \dots\cr
    -1&-1&-1 \dots    &-1
                     &n-1\cr
                      }
                         \pmatrix{0\cr 1\cr -1\cr 0\cr
                   \dots\cr
                    0\cr0\cr}=
{1\over 2n}\pmatrix{0,1, -1, 0, \dots, 0}\,..
            $$
and so on
  $$
\a_{n-2 n-1}={1\over 2n^2}\pmatrix {
              n-1 &-1 
            &-1\dots &-1 &-1\cr
              -1 &n-1 &-1\dots 
             &-1 &-1\cr
                \dots\cr
    -1&-1&-1 \dots    &-1
                     &n-1\cr
                      }
             \pmatrix{0\cr 0\cr \dots\cr 0\cr
                   1\cr
                    -1 \cr0\cr}=
{1\over 2n}\pmatrix{0,\dots, 0, 1, -1}\,..
            $$
and
  $$
\a_{n-1 n2}={1\over 2n^2}\pmatrix {
              n-1 &-1 
            &-1\dots &-1 &-1\cr
              -1 &n-1 &-1\dots 
             &-1 &-1\cr
                \dots\cr
    -1&-1&-1 \dots    &-1
                     &n-1\cr
                      }
             \pmatrix{1\cr 1\cr 1\cr 1\cr
                   \dots\cr
                    1 \cr 2\cr}=
{1\over 2n}\pmatrix{0, 0, \dots, 0, 1}\,..
            $$
Now we can calulate all scalar products and all lengths.

For example:

It is useful to calculate coroots $\a_{pq}$ which are 
Now we can perfoirm calculations, for example
                 $$
|\a^{12}|^2==\langle\a^{12},\a^{12}\rangle_{CK}=
\phi(\a^{12},\a^{12})=\a^{12}(\a_{12})=
   {1\over 2n}\pmatrix {1, -1,0,0\dots,0,0}
             \pmatrix{1\cr -1\cr 0\cr 0\cr
                   \dots\cr
                    0\cr0\cr}={1\over n}\,,        
                     $$ 
                      $$
\langle\a^{12},\a^{23}\rangle_{CK}=\phi(\a^{12},\a^{23})=
   {1\over 2n}\pmatrix {1, -1,0,0\dots,0,0}
             \pmatrix{0\cr 1\cr 0\cr 0\cr
                   \dots\cr
                    0\cr0\cr}=-{1\over 2n}         
                     $$
%i.e, root $\a^{12}$ has the length ${1\over \sqrt n}$ 
and angle between roots is eqaul to ${2\pi\over 3}$.
              
One can see that all the simple roots have the same length
   $1\over \sqrt n$.


Now caculate the Weyl covector  $\rho$ which is equal to halph of the
sum of positive roots:
                $$
\rho={1\over 2}\sum_{p<q}\a^{pq}={1\over 2}\sum_{p<q}
\left(\sum_{i=p}^{q-1}\a^{i,i+1}\right)=\sum_{k=1}^{n-1} k(n-k)\a_{k,k+1}\,.
                $$
    The vector $\rho'$ which is dual to covector  $\rho$
is given by the formula                 
                $$
\rho'={1\over 2}
\left((n-1)\a_{12}+2(n-2)\a_{23}+3(n-3)\a_{34}+\dots+(n-2)2\a_{n-2,n-1}+
              (n-1)\a_{n-1,n}\right)=
                 $$
                 $$
    {1\over 4n} \pmatrix{n-1, n-3,n-5,\dots,-n+3}
                $$
Now we are able to caclulate the volume.

Later Cartan-Killing scalar product we will denote just by 
$\langle\,\,,\,\,\rangle$.

We have
             $$
\langle\rho,\a^{12}\rangle=\a^{12}(\rho')=
\pmatrix{1\cr -1\cr 0\cr\dots\cr 0\cr}
    {1\over 4n} \pmatrix{n-1, n-3,n-5,\dots,-n+3}={1\over 2n}
             $$
The same is for all other simple roots including the root $\a^{n-1,n}$:
       $$
\langle\rho,\a^{12}\rangle=
\langle\rho,\a^{23}\rangle=\dots
\langle\rho,\a^{n-1,n}\rangle={1\over n}
       $$
Hence  we have that for an arbitrary root $\a^{pq}$
                $$
\langle\rho,\a_{pq}\rangle=
\langle\rho,\a^{p,p+1}+\dots +\a^{q-1,q}={q-p\over 2n}\,.
                $$
Now return to the formula for volume of the group
$G_n$ which is compact connected simply connected Lie group
such that complexification of its Lie algebra is equal to $gl(n,\C)$. 

Kac-Peterson formula and our caculationn give
        $$
   Vol(G_n)=(2\pi\sqrt 2)^{dim G_n}\prod_{\a^{pq}\colon p<q}
      {\sin\left(2\pi \langle\rho,\a^{pq}\rangle\right)\over
        2\pi \langle\rho,\a^{pq}\rangle}
        $$
            $$
Vol G=(2\pi\sqrt 2)^{n^2-1}\prod_{k=1}^{n-1}
    \left({\sin\left(k\pi\over n\right)\over {k\pi\over n}}\right)^{n-k}\,.
            $$

\bye
