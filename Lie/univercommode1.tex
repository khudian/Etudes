
\def\w {\omega}
\def \R {{\cal R}}
\def \x {{\bf x}}
\magnification=1200
\def \y {{\bf y}}
\def \a   {{\bf \alpha}}
\def \y {{\bf y}}


The regular representation of group contains all irreducible representations


   The well-known (Burnside?)  lemma states that for finite group $G$
                   $$
   \sum n_i^2=|G|\,,
                 $$
where $n_i$ is a dimension of irreducible representation. E.g.
the group $S_3$ has two $1$-dimensional irreducible representations:
(identical and alternating) and one $2$-dimensional representation
(symmetry of the triangle in the plane) and $1^2+2^2+1^2=6$.

  There are many different proofs of this lemma. I will present here the proof
which makes me to remember the story of the commode which is full of the dishes
(this proof ) but these dishes never  breake.....


  \bigskip


   The proof is founded on the following construction:

  Let $\rho$ be a representation of a group $G$ in finite-dimensional space $V$.
Then arbitrary element  $\w$ in the dual space $V^*$ defines the map
$\iota_{\rho,\w}$ of $V$ in the space $\R_G$ of of functions on $G$:
          $$
\iota_{\rho,\w}(\x)\colon\,\,\,
=\iota_{\rho,\w}(\x|g)=
 \langle \rho_g(\x),\w\rangle\,.
      \eqno (1)
          $$
The kernel of this map, the subspace $V'\subseteq V$  
is invariant with respect to the action of the group $G$. Hence
$V'=0$ if $\w\not=0$ and the representation 
$\rho$ is irreducible representation. Hence  
map (1) of $V$ into regular representation $\R$ is injection map in the case 
if representation $\rho$ is irreducible and $\w\not=0$. 
 %  If $V$ is the space of irreducible representation $\rho$, then we denote
% by  $V^{(\rho,\w)}$ the image of the space $V$ under the embedding map (1).


 {\bf Observation}.
  Let $(\rho_1,V_1)$ and $(\rho_2,V_2)$ be two irreducible representations
of group $G$, and $\w_1,\w_2$ be two non-zero covectors,
$\w_1\in V^*_1$, $\w_2\in V^*_2$. Consider the reciprocal position of
images of spaces $V_1$ and $V_2$
under injection (1) under the injection (1). The following statement is valid:

{\it if representations $\rho_1,\rho_2$ are not equivalent and then
the images intersect only by zero element. If  these
representations are equivalent, $\rho_1\approx \rho_2$ 
i.e. there exists interwining
bijection  $T\colon V_1\to V_2, \rho_2\circ T=T\circ \rho_1$,
then the images intersect only by zero element if and only if
covectors $\w_1$ and $T^*\w_2$ are linear independent. In the
case if these covectors are proportional then the images coicide.
}

Indeed suppose that there exist $\x\in V_1,\y\in V_2$ such that
                $$
\forall g\,\, \langle \rho_1(g)(\x), \w_1\rangle=
\langle \rho_2(g)(\y), \w_2\rangle\,,\quad (\w_1\not=0, \x\not=0)\,.
                     \eqno (2)
                $$ 
This condition defines invariant subspaces in $V_1$ and $V_2$.
Since representations are irreducible and $\x\not=0$ we see that
this condition holds for all $\x\in V_1$. Since the injectivity of maps
(1) we see that condition (2) defines interwinning operator $T$
on $V_1$ such that $T(\x)=\y$. Hence by Schur lemma these representations
are equvalent. We may identify spaces $V_1$ and $V_2$ by the
interwinning operator. In this case $T(\x)=\lambda \x$, i.e. $\w_1=\lambda \w_1$.

Using this observations perform the following operation: 

Let $\{\rho_a, V_\a\}$ be a set of {\it all} irreducible  
representations of the group $G$.  They all are finite-dimnesional. For every
representation $(\rho_\a,V_\a$ consider an arbitrary basis
$\{\w_{i_\a}\}=\{\w_1,\dots,\w_{n_\a}\}$
in $V_\a$ ($n_\a$ is dimension of the space $V_\a$). Then
an arbitrary irreducible representation $(\rho_\a,V_\a)$  defines $n_\a$
embeddings $\iota_{\w_{i_\a},\rho_\a}$ such that all images of these embeddings
intersect by zero element in the ``universal commode'', the regular representation
$\R$. Two {\rm non-equivalent representations} 
$(\rho_\a, V_\a), (\rho_\beta, V_\beta)$, $\rho_\a\not\approx\rho_\beta$  define
 two collections of subspaces in $\R$, the collection of $n_\a$ subspaces,
and the collection of 
second $n_\beta$ subspaces. All these subspaces will intersect by zero element.
Consider the union of the embeddings in the ``universal commode''
$\R$ of {\rm all} pairwise non-equivalent representation; every
representation $\rho_\a$ is embedded $n_\a$ times: 
                  $$
         \oplus_{_{\rho_\a\colon \rho_\a\not\approx \rho_{\a'}}} 
        \oplus_{i_\a}  \iota_{\w_{i_\a},\rho_\a}(V_\a)\subseteq \R,
                 \eqno (3)
                  $$
i.e.                    
                  $$
   \sum_\a ({\rm dim\,} V_\a)^2=\sum n^2_\a \leq {\rm dim\,}\R=|G|\,.
                  \eqno (3a)  
                  $$
(The number relation 3a is the corollary of the relation (3))
It remains to prove that order of $G$ is not bigger than $\sum_\a n^2_\a$.

Consider in $\R$ the subspace $W$ which is orthogonal to all subspaces dual to 
images of subspaces $V_\a$. This subspace contains some irreducible
subspace of functions $(\rho_0, V_0)$.  This subspace of functions
is equivalent to one of irreducible representation
  $(\rho_{\a_0}, V_{\a_0})$ and hence it is equivalent to its image
(1) to the subspace in the ``universal commode'' $\R$.
Two euivalent irreducible subspaces in the same space have to coincide
by Schur Lemme. Hence $W=0$. We proved that  
                  $$
         \oplus_{\rho_\a\colon \rho_\a\not\approx \rho_{\a'}} 
        \oplus_{i_\a}  \iota_{\w_{i_\a},\rho_\a}(V_\a)= \R,
                 \eqno (3)
                  $$
i.e.                    
                  $$
   \sum_\a ({\rm dim\,} V_\a)^2=\sum n^2_\a = {\rm dim\,}\R=|G|\,.
                  \eqno (3a)  
                  $$
  
The proof is finished.

{\it Many years ago giving this proof to students in Yerevan University
I told them the story about ``universal commode'': you buy after marriage
the commode in two three months it becomes full ofthings, there is not free space,
but everything is put there safely. This is exactly what happens here:
all irreducible representations are packed in the regular representation.}


\bye
 
