\magnification=1200

\def\e{{\bf e}}
\def\f{{\bf f}}
\def\y{{\bf y}}

\def\G {{\cal G}}
\def\H {{\cal H}}
\def\ad {{\rm ad\,}}
\def\m {\medskip}

       \centerline {\bf Cartan algebras}

  We say that $\H\subseteq \G$ is Cartan subalgebra in $\G$
if $\H$ is nilpotent subalgebra and it coicides with tis normaliser.

  Consider polynomial 

          $$
     P_x(z)=\det (z-\ad x)=a_0(x)+a_1(x)z+a_2(x)z^2+\dots+a_N(x)z^N\,,
           $$
where $N={\rm dim\,}\G$. It is evident that $a_0\equiv 0$.

\m


   We say that algebra $\G$ has rank $l$ if
   there exist $x\in \G$ such that
this polynomial has non-zero coefficient $a_l$,
but all $a_k$ identically vanish for $k<l$.

\m

   We say that $x\in \G$ is {\it regular} element in $\G$
if $a_l(x)\not=0$, where $l$ is a rank of the algebra $\G$


\m

    {\bf Theorem}  Set of regualr elements in algebra Lie is
open, dense and simply connected.


  \m 

  {\bf Theorem}  Let $x$ be an arbitrary regular element in Lie algebra
$\G$. Then  
        $$
\G_x^0=\{\xi\colon\,\, (\ad x)^n\xi=0\}
        $$ 
is subalgebra, and this subalgebra is Cartan subalgebra.

\m

    {\bf Theorem} The group $G$ of inner automorphisms of Lie algebra
   $\G$ acts transitevely on set of Cartan algebras.

\m

   {\sl Corollary}

   Every Cartan subalgebra has dimension $l$, where $l$ is a rank 
of the algebra $\G$. 

          If $\H$ is the Cartan subalgebra, then there exists $x\in \G$
such that $\H=\G_x^0$.




\bye
