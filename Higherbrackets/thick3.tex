\magnification=1200

\baselineskip=17pt



\def\vare {\varepsilon}
\def\A {{\bf A}}
\def\t {\tilde}
\def\a {\alpha}
\def\K {{\bf K}}
\def\N {{\bf N}}
\def\V {{\cal V}}
\def\s {{\sigma}}
\def\S {{\bf S}}
\def\s {{\sigma}}
\def\bs {{\bf s}}
\def\p{\partial}
\def\vare{{\varepsilon}}
\def\Q {{\bf Q}}
\def\D {{\cal D}}
\def\L {{\cal L}}
\def\G {{\Gamma}}
\def\C {{\bf C}}
\def\M {{\cal M}}
\def\Z {{\bf Z}}
\def\U  {{\cal U}}
\def\H {{\cal H}}
\def\R  {{\bf R}}
\def\E  {{\bf E}}
\def\l {\lambda}
\def\degree {{\bf {\rm degree}\,\,}}
\def \finish {${\,\,\vrule height1mm depth2mm width 8pt}$}
\def \m {\medskip}
\def\p {\partial}
\def\r {{\bf r}}
\def\v {{\bf v}}
\def\n {{\bf n}}
\def\t {{\bf t}}
\def\b {{\bf b}}
\def\e{{\bf e}}
\def\f{{\bf f}}
\def\ac {{\bf a}}
\def \X   {{\bf X}}
\def \Y   {{\bf Y}}
\def\diag {\rm diag\,\,}
\def\pt {{\bf p}}
\def\w {\omega}
\def\la{\langle}
\def\ra{\rangle}
\def\x{{\bf x}}
\def\m {\medskip}
\def\thick {{\buildrel \to\over \to}}

  \centerline{\bf Thick morphism in the third order}

{\it This is an attempt of straightforward calculations for
Voronov's  thick morphism $M_1\to M_2$}.


We consider manifolds $M_1$ and $M_2$.
Let $(x^i)$ be local coordinates on $M_1$
and let $(y^\a)$ be local coordinates on $M_2$.
Respectively let $(p_i,x^i)$ be local coordinates on $T^*M_1$
and let $(q_\a,y^\beta)$ be local coordinates on $T^*M_2$





Let  
       $$
S(x,q)=S(x)+S^\a(x)q_\a+{1\over 2}S^{\a\beta}q_\beta q_\a
+{1\over 6}S^{\a\beta\gamma}q_\gamma q_\beta q_a
+{1\over 24}S^{\a\beta\gamma\pi}q_\pi q_\gamma q_\beta q_\a+\dots
       $$
be a function which defines Lagrangian surface 
$\Lambda_S$ in $T^*M_1\times (-T^*M_2)$:
            $$
\Lambda_S=\left\{(p_i,x^j,q_\a,y^\beta)\colon\, 
p_i={\p S(x,q)\over \p x^i}\,,y^\a={\p S(x,q)\over \p q_\a}\right\}
            $$
($\Lambda_S$ is Lagrangian with respect to canonical symplectic form 
$dp_i\wedge dx^i-dq_\a dy^\a$ on $T^*M_1\times T^*M_2$.)


  This Lagrangian surface define V-thick morphism $\varphi_S$ 
$M_1\thick M_2$
in the following way:
                $$
C(M_2) \ni g=g(y)\to f=f(x)
                $$
        such that Lagrangian surfaces defined by graphs
of functions $f$ and $g$ are related with Lagrangian 
surface $\Lambda_S$, i.e.
          $$
p_i={\p f(x)\over \p x^i}={\p S(x,q)\over \p x^i}\,,\quad
q_\a={\p g(y)\over \p y^a}\,,\quad y^\a={\p S(x,q)\over \p q^\a}\,.
 $$


One can see that in local coordinates
      $$
f(x)=S(x,q)+g(y)-y^\a q_\a
      $$
(see T.Voronov [1].)

We suppose that $g\to \vare g$ is infinitesimal function.
Thick morphism defines non-linear map of infinitesimal functions
$\vare g$ on $M_2$ to infinitesimal functions on $M_1$.
        
Solve these equations indictively up to an arbitrary order.
Since $y^\a={\p S(x,q)\over \p q_\a}$ 
and $q_\a={\p g(y)\over \p y^\a}$ hence we
see that $q_\a=q_\a(x)$ and $y^a=y^\a(x)$ can be 
recurrently expressed
due to the relation:
         $$
q_\a(x)={\p g(y)\over \p y^\a}
        \big\vert_{y^\beta=S^{\beta\pi}q_\pi+\dots}
         $$
We have
       $$
y^\a={\p S(x,q)\over \p q_\a}=
 S^\a(x)+S^{\a\beta}q_\beta 
+{1\over 2}S^{\a\beta\gamma}q_\gamma q_\beta 
+{1\over 6}S^{\a\beta\gamma\pi}q_\pi q_\gamma q_\beta +\dots
       $$
and expanding in Taylor series we come to
       $$
q_\a=\vare{\p g(y)\over \p y^\a}\big\vert_{y^a=
S^\a+S^{\a\beta}q_\beta 
+{1\over 2}S^{\a\beta\gamma}q_\gamma q_\beta+\dots}=
     $$ 
        $$
\vare{\p g(y)\over 
\p y^\a}\left(
   S^\a(x)+S^{\a\beta}(x)q_\beta 
+{1\over 2}S^{\a\beta\gamma}(x)q_\gamma q_\beta+\dots
        \right)=
     $$
        $$
\vare\sum {1\over n!}{\p^{n+1} g(y)\over 
\p y^\a\p y^{\sigma_1}\dots \p y^{\sigma_n}}\left(
   S^\a(x)\right)T^{\sigma_1}\dots T^{\sigma_n}\,,
     $$
where
     $$
 T^\a=S^{\a\beta}q_\beta 
+{1\over 2}S^{\a\beta\gamma}q_\gamma q_\beta 
+{1\over 6}S^{\a\beta\gamma\pi}q_\pi q_\gamma q_\beta +\dots
     $$
Hence we come to
            $$
q_\a=\vare q^{(1)}_\a+
          \vare^2 q^{(2)}_\a+
          \vare^3 q^{(3)}_\a+\dots
         \eqno (*)
       $$
where recurrently:
       $$
    q^{(1)}_\a=q^{(1)}_\a(x)=
     {\p g(y)\over \p y^\a}\left(
   S^\a(x)\right)=l_\a\,,
       $$
       $$
    q^{(2)}_\a=q^{(2)}_\a(x)=
    l_{\a\sigma}S^{\sigma\beta}l_\beta\,,\quad
     l_{\a\sigma}={\p^2 g(y)\over \p y^\a \p y^\sigma}(y)
\big\vert_{y^\a=S^\a(x)}  
       $$
       $$
    q^{(3)}_\a=q^{(3)}_\a(x)=
    l_{\a\sigma}S^{\sigma\beta}l_{\beta \omega}S^{\omega\pi}l_\pi
   +{1\over 2}l_{\a\sigma}S^{\sigma\beta\gamma}l_\beta l_\gamma
    +{1\over 2}l_{\a\sigma}S^{\a\beta}S^{\sigma\omega}l_\beta l_\omega\,,
          \quad
     l_{\a\sigma}={\p^2 g(y)\over \p y^\a \p y^\sigma}(y)
\big\vert_{y^\a=S^\a(x)}  
       $$
One can express all answers in terms of series (*).
  We use notation:
          $$
   S(x,q)=S(x)+S^\a(x)q_\a+{1\over 2}S^{\a\beta}q_\beta q_\a
+{1\over 6}S^{\a\beta\gamma}q_\gamma q_\beta q_a
+{1\over 24}S^{\a\beta\gamma\pi}q_\pi q_\gamma q_\beta q_\a+
+{1\over 120}S^{\a\beta\gamma\pi\rho}q_\rho q_\pi q_\gamma q_\beta q_\a
+\dots=
      $$
       $$
=\sum_{r\geq 0} S_r(x,q)=S_0+\sum_{r\geq 0} V^\a_{r}(x,q)q_\a
          $$
   We have
      $$
f(x)=S(x,q)+g(y)-y^\a q_\a=
        $$
         $$
       \sum_r S_r(x,q)+
         g\left( {\p S(x,q)\over \p q^\a}=V^\a(x)+
                \sum_{r\geq 1}V(x,q)q_\a\right)-
     \left({\p S(x,q)\over \p q^\a}q_\a=\sum_r r S_r(x,q)\right)\,.
      $$
    $$
S_0(x)+\sum_{r\geq 2} S_r\left(x,l_\a+\sum_{k\geq 2} q^{(k)}_\a\right)+
   g\left(
S^\a+\sum_{r\geq 2} S_r\left(x,l_\a+\sum_{k\geq 2} q^{(k)}_\a\right)\right)
     $$

  $$ $$

{\bf Example}
   $$
   g=c+\vare y^r_\a\Rightarrow f=f(x)=c+S(x,r) (\vare?????)
   $$
     $$
   g=c+\vare y^r_\a+{1\over 2}t_{\a\beta}y^\a y^\beta\Rightarrow f=f(x)=?????
   $$


\centerline {\bf Example}
Let $M_1=M_2=\R$ and
        $$
       S(x,q)=a(x)+\varphi(x)q+{1\over 2}bq^2
        $$
Then
     $$
      \cases
        {
   y=\varphi(x)+b(x)q\cr
      q=g'(y)\cr
        }`
   \Rightarrow
       \cases
        {
   y=\varphi(x)+b(x)g'(y)\cr
      q=g'(\varphi(x)+b(x)q)\cr
        }\Rightarrow
        $$
         $$
    \cases
        {
   y=\varphi+bg'(\varphi+bg')=
   \varphi+bl'+bl''b(l'+bl''bl')+{b\over 2}l'''bl'bl'+\dots\cr
      q=g'(\varphi(x)+b(x)q)=l'+l''b(l'+l''bl')+{1\over 2}l'''bl'bl'+\cr
        }``
     $$
where $l=g(\varphi)$, $l'=g'(y)|big\vert_{y=\varphi(x)}$,
$l''=g''(y)|big\vert_{y=\varphi(x)}$,\dots
 We have
      $$
g(y)=g(\varphi+bq)=\dots
      $$
and
     $$
f(x)=a+\varphi q+{1\over 2}bq^2+g(y)-yq=
     a+\varphi q+{1\over 2}bq^2+g(y)-(\varphi +bq)q=
    a+{1\over 2}bq^2+g(\varphi+bq)-bq^2
          $$
          $$
     a+{1\over 2}b(l'+bl'l''+b^2l''l''l'+{1\over 2}b^2l'l'l''')^2
     $$
\bye
