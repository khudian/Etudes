





\magnification=1200

\baselineskip=17pt



\def\vare {\varepsilon}
\def\A {{\bf A}}
\def\t {\tilde}
\def\a {\alpha}
\def\K {{\bf K}}
\def\N {{\bf N}}
\def\V {{\cal V}}
\def\s {{\sigma}}
\def\S {{\bf S}}
\def\s {{\sigma}}
\def\bs {{\bf s}}
\def\p{\partial}
\def\vare{{\varepsilon}}
\def\Q {{\bf Q}}
\def\D {{\cal D}}
\def\L {{\cal L}}
\def\G {{\Gamma}}
\def\C {{\bf C}}
\def\M {{\cal M}}
\def\Z {{\bf Z}}
\def\U  {{\cal U}}
\def\H {{\cal H}}
\def\R  {{\bf R}}
\def\E  {{\bf E}}
\def\l {\lambda}
\def\degree {{\bf {\rm degree}\,\,}}
\def \finish {${\,\,\vrule height1mm depth2mm width 8pt}$}
\def \m {\medskip}
\def\p {\partial}
\def\r {{\bf r}}
\def\v {{\bf v}}
\def\n {{\bf n}}
\def\t {{\bf t}}
\def\b {{\bf b}}
\def\e{{\bf e}}
\def\f{{\bf f}}
\def\ac {{\bf a}}
\def \X   {{\bf X}}
\def \Y   {{\bf Y}}
\def\diag {\rm diag\,\,}
\def\pt {{\bf p}}
\def\w {\omega}
\def\la{\langle}
\def\ra{\rangle}
\def\x{{\bf x}}
\def\m {\medskip}
\def\thick {{\buildrel \to\over \to}}

  \centerline{\bf Thick morphisms of vector spaces.}

 Let $U,W$ be a vector spaces, and $L\colon U\to W$ linear map from
   $U$ to $W$. One can consider its adjont $L^*\colon V^*\to U^*$:
                      $$
L^*\colon W^*\to U^*\,\, \hbox {such that for arbitr.}\,\,
     \Y\in U, \langle\Y, L^*(\w) \rangle
              =\langle L(\Y), \w \rangle\,.
            \eqno (0.1)
                      $$
  In the case if $W=U^*$ both  $L$ and $L^*$
map $U\to U^*$.
  Map $L$ is self-adjoint (anti-self adjoint if $L=L^*$.
   This is standard text-book stuff. Using
   Vornov's thick mrophisms in fact generalise
this statement for non-linear maps.

   All the constructions will be based on
   Voronov's thick morphisms.


  For linear spaces $U,W$ consider linear space
          $$
T=U\oplus U^*\oplus W\oplus W^*
\eqno (1.1)
          $$

If $V$ is an arbitrary vector space then denote by $\w_{_V}$
$2$-form on vector space $V\oplus V^*$ such that
                $$
    \cases
     {
\w_{_V}(\X,\Y)=0\,{\rm if}\, \X\,,\Y\in V\,,\cr
\w_{_V}(t,\X)=t(\X)\,{\rm if}\,\X\in V \in\,,t \in V^*\cr
\w_{_V}(t,s)=0\,{\rm if}\,t\,,s \in V^* \cr
     }\,, \hbox{in coordinates}\,
  \w_{_U}=dp_i\wedge du^i\,
  \eqno (1.1a)
                $$
if $p_i$ coordinates on $V^*$ dual to coordinates to $u^j$.

Spaces $U\oplus U^*$ and $W\oplus W^*$can be provided with 
canonical sympletic structures  $\w_{_U},\w_{_W}$,
and the space
$T=U\otimes U^*\otimes W\otimes W^*$ with sympelctic structure
     $$
\Omega=\w_{_U}-\w_{_V}=dp_i du^i-dq_a dw^a\,
  \eqno (1.1b)
      $$
where we denote by $u^i$ some coordinates on $U$,
$p_i$ coordinates on $U^*$ dual to $u^i$,
and
$w^a$ some coordinates on $W$,
$q_a$ coordinates on $W^*$ dual to $w^a$,

\medskip

{\bf Definition} Let $S=S(u,q)$ be a smooth function on $V\times W^*$.
  This function defines Lagrangian surface
          $$
\L_S=   \left\{(u^i,p_j,w^a,q_b)\colon\,\, 
    p_j={\p S(u,q)\over \p u^i}, w^a={\p S(u,q)\over \p q_a}
        \right\}
      \eqno (1.2a)
          $$
 {\bf Definition} The function $S$ and Lagrangian  surface
 $\L_S$ define thich morhism
        $$
 \Phi_S\colon \,  U\rightarrow W\,.
   \eqno (1.2b)
        $$
Thick morphism $\Phi_S$ defines pull-back of functions
       $$
 \Phi^*_S\colon \,  U\leftarrow W\,.
\eqno (1.2c)
       $$
which is the following non-linear map:
      $$
C(W)\ni g\to f=f(u)=g(w)+S(u,q)=q(w)\,,
 \eqno (1.3) 
      $$
Geometrical meaning of this map:  Function $g$ defines Lagrangian
surface 
            $$
      \L_g=\left\{(w^a,q_b)\colon\,\, 
     q_a={\p g(w)\over \p w^a}
        \right\}
        \eqno (1.4a)
       $$
This is -graph of the function  $dg$ in the symplectic space 
$W\oplus W^*$. 
Respectively, function $f$ defines Lagrangian
surface $\L_f=\left\{(x^i,p_j)\colon\,\, 
     p_j={\p f(x)\over \p x^i}
        \right\}$.
This is -graph of the function  $dg$ in the symplectic space 
$W\oplus W^*$.  The ansatz (1.3) means that Lagrangian
surfaces $\L_F,\L_g$ are related via the Lagrangian surface $\L_S$:
         $$
\L_f=\L_S\circ \L_g
     \eqno (1.5)
      $$
Note that the relations (1.5) define pullback up to a constant.
 ({ecrire mieux})


In the special case if
       $$
   S=\Psi^a(x)q_a
      $$
then thick morphism $\Phi_S$ is nothing but the morphism
   $w^a=\Psi^a(x)$.
\bye
