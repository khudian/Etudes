\magnification=1200

\baselineskip=17pt



\def\vare {\varepsilon}
\def\A {{\bf A}}
\def\t {\tilde}
\def\a {\alpha}
\def\K {{\bf K}}
\def\N {{\bf N}}
\def\V {{\cal V}}
\def\s {{\sigma}}
\def\S {{\bf S}}
\def\s {{\sigma}}
\def\bs {{\bf s}}
\def\p{\partial}
\def\vare{{\varepsilon}}
\def\Q {{\bf Q}}
\def\D {{\cal D}}
\def\L {{\Lambda}}
\def\LL {{\cal L}}
\def\G {{\Gamma}}
\def\C {{\bf C}}
\def\M {{\cal M}}
\def\Z {{\bf Z}}
\def\U  {{\cal U}}
\def\H {{\cal H}}
\def\R  {{\bf R}}
\def\E  {{\bf E}}
\def\l {\lambda}
\def\degree {{\bf {\rm degree}\,\,}}
\def \finish {${\,\,\vrule height1mm depth2mm width 8pt}$}
\def \m {\medskip}
\def\p {\partial}
\def\r {{\bf r}}
\def\v {{\bf v}}
\def\n {{\bf n}}
\def\t {{\bf t}}
\def\b {{\bf b}}
\def\e{{\bf e}}
\def\f{{\bf f}}
\def\ac {{\bf a}}
\def \X   {{\bf X}}
\def \Y   {{\bf Y}}
\def\diag {\rm diag\,\,}
\def\pt {{\bf p}}
\def\w {\omega}
\def\la{\langle}
\def\ra{\rangle}
\def\x{{\bf x}}
\def\m {\medskip}
\def\thick {{\buildrel \to\over \to}}

  \centerline{\bf Toy example of thick morphism}

Let
  $V$ be a vector space.

Let  $u^i$ be coordinates on $V$.

Let  $w_i$ be arbitrary coordinates on $V^*$.


 Consider on vector space $N=V\oplus V^*$
canonical symplectic structure
             $$
   dp_i\wedge du^i=dw_i\wedge dq^i\,.
        \eqno (1a)
             $$     
where $p_i, q^j$ are coordinates on $V^*, V$ dual to coordinates $u^i,w_j$.

Consider on $N\times N$ the symplectic structure
       $$
     \Omega=\w_1-\w_2
\eqno (1b)
       $$
If $(u,v)$ are coordinates on the first examplaire of $N$
and $(q,w)$ are coordinates on the second exemplaire of $N$
then
       $$
\Omega=dp_i\wedge du^i-dw_i\wedge dq^i\,.
      \eqno (1c)
       $$
 Let $S(u,q)$ be an arbitrary (smooth) function.
It defines Lagrangian surface
          $$
\L_S=\left\{(u,p,q,w)\colon 
  p_i={\p S(u,q)\over \p u^i}\,,\,
  w_i={\p S(u,q)\over \p q^i}\right\}
\eqno (2a)
         $$
This is Lagrangian since its dimension is equal to $n$ ($n=\dim V$)
and due to the construction
   $$
 \left(dS=p_idu^i+w_mdq^m\right)\big\vert_{\L_S}
  \Rightarrow
   (d^2S=0=d\left(p_idu^i+w_mdq^m\right)=dp_i\wedge du^i+dw_m\wedge dq^m=
   \Omega)\big\vert_{\L_S}\,.
\eqno (2b)
    $$
This Lagrangian surface defines anticanonical relation $\sim_S$:
on sympletic vector space $N=V\oplus V^*$:
              $$
        (u,p)\sim_S (q,u) \,\,{\rm if}\,\,
   (u,p,q,w)\in\L_S\,,{\rm i.e.}\,\,
          p_i={\p S(u,q\p u^i)}\,,w_i={\p S(u,q\p q^i)}
            \eqno (3a)
              $$      
For example bilinear form $S=u^iS_{ik}q^k$ defines canonical relation
              $$
(u,p)\sim_S (q,w) \,\,{\rm if}\,\,
   \cases{p_i=S_{ik}q^k\cr w_i=u^mS_{mi}}
    \eqno (3b)
              $$
 This canonical relation may define antisymplectic  transformation
if it defines bijection of $N$ on $N$. E,g, if bikinear form $S$
is non-degenerate, then
      canonical relation (3b) defines anticanonical transformation 
        $$
      \cases
        {
    u^m=w_kS^{km}\cr  p_i=S_{ik}q^k\cr
        }
      \eqno (3c)
        $$

Now let  $S(u,q)$ be an arbitrary smooth function. It defines
  lagrangian surface $\L_S$.

  Consider the action of sympelctomorphism
        $$
\kappa \colo u^i\leftrightarrwo  q^i, p_j\leftrightarrow w_j, 
        $$
Under the action of this sympelctorphsim lagrangian surface
$\L_=\L_S$ transforms to Lgrangian surface $\L^*=\L_{S^*}$, where
           $$
S^*(u,q)=S(q,u)
           $$
  These two anticanonical realtions are related with
 symplectic transformation $\kappa$


Now define thick morphisms
\bye
