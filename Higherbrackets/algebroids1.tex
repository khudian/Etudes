





% Here is the first attempt to write down the lectures.
% on $L_\infty$ algebroids. Text is based on my inter[pretation
%of TRed's Voronov two lectures    lectures
% on 25 26 November after one year of worki he comes to new understanding....
%of Ted and my understnding of these lectures.
\def\vare {\varepsilon}
\def\A {{\bf A}}
\def\t {\tilde}
\def\a {\alpha}
\def\K {{\bf K}}
\def\N {{\bf N}}
\def\V {{\cal V}}
\def\s {{\sigma}}
\def\S {{\bf S}}
\def\s {{\sigma}}
\def\bs {{\bf s}}
\def\p{\partial}
\def\vare{{\varepsilon}}
\def\Q {{\bf Q}}
\def\D {{\cal D}}
\def\L {{\cal L}}
\def\G {{\Gamma}}
\def\C {{\bf C}}
\def\M {{\cal M}}
\def\Z {{\bf Z}}
\def\U  {{\cal U}}
\def\H {{\cal H}}
\def\R  {{\bf R}}
\def\E  {{\bf E}}
\def\l {\lambda}
\def\degree {{\bf {\rm degree}\,\,}}
\def \finish {${\,\,\vrule height1mm depth2mm width 8pt}$}
\def \m {\medskip}
\def\p {\partial}
\def\r {{\bf r}}
\def\v {{\bf v}}
\def\n {{\bf n}}
\def\t {{\bf t}}
\def\b {{\bf b}}
\def\e{{\bf e}}
\def\f{{\bf f}}
\def\ac {{\bf a}}
\def \X   {{\bf X}}
\def \Y   {{\bf Y}}
\def\diag {\rm diag\,\,}
\def\pt {{\bf p}}
\def\w {\omega}
\def\la{\langle}
\def\ra{\rangle}
\def\x{{\bf x}}
\def\m {\medskip}
\documentclass[12pt]{article}
\usepackage{amsmath,amsthm}


\usepackage{amsmath,amssymb,amsfonts,amsthm}


\theoremstyle{theorem}
\newtheorem{thm}{Khimera}

\numberwithin{equation}{section}


\title{$L_\infty$ algeborids and $L_\infty$ morphisms}
\date{}
\begin{document}
\maketitle

  \centerline {H.M.Khudaverdian\footnote{The content of these notes is
based on my discussions  with Ted Voronov
about algebroids  mainly under the influence of his notes and remarks.}}

   \centerline { I began to write it in Manchester, 26 November  2015}





\tableofcontents
\pagenumbering{roman}
\newpage
\pagenumbering{arabic}

 Let $(M,P)$ be a Poisson manifold. It defines
Lie algebroid $T^*M\to M$ of $1$-forms. It will be
 our most inspiring example. This is fundamental object. 
During these lectures we often return to it,


\section {Lie algebroids. Definition of all its manifestations.}

\subsection {Lie algebroid}

  We give a definition of Lie algebroid. 
As usual in these lectures we will consider 
four different manifestations of Lie algebroid.
m

{\sl I-st manifestation.}  

Let $E\to M$ be a vector bundle,
$[[\,,\,]]$ commutator of sections and
   $\ac$--anchor map: linear map  of $E\to M$
to tangent bundle $TM$:
            $$
[[\bs_1,\bs_2]]=-[[\bs_2,\bs_1]]\,,
 [[\bs_1,f\bs_2]]=(-1)^{...}f[[\bs_1,f\bs_2]]+
    \ac(\bs_1)f\bs_2\,\hbox{(Leibnitz rule)}
           $$
and Jacobi identity is obeyed
           \begin{equation*}\label{jacobi}
  [[\,[[\bs_1,\bs_2]],\bs_3]]+\hbox{cyclic permutation}=0
            \end{equation*}
It is useful to write local formulae for commutator and anchor.

  Let $x^\mu$ be local coordinates on the base $M$. Then
 set of linearly independent local sections $\{\e_a(x)\}$
($a=1,\dots,n$, where $n$ is a dimension of fibre) define
local coordinates $(x^\mu.s^a)$ on $\E$: $(x^\mu,y^a)\mapsto y^a\e_a(x)$.    
   We denote by
                  \begin{equation}\label{defofalgebroid1}
      c^a_{bc}\colon\quad        c^a_{bc}(x)\e_a(x)=[[\e_b(x),\e_c(x)\,\,
  {\rm and}\,\, \a^\mu_a\colon\quad \a^\mu_a\p_\mu=\ac(\e_a)
                    \end{equation}
Then we have that for two arbitrary sections  
$\bs_1(x)=y^a_1(x)\e_a(x)$,$\bs_2(x)=y^a_2(x)\e_a(x)$
                  $$
[[\bs_1,\bs_2]]=[[y^a_1(x)\e_a(x),y^b_2(x)\e_b(x)]]=
                   $$
                   $$
y^a_1(x)y^b_2(x)c_{ba}^d(x)+
    y^a_1(x)\a_a^\mu(x)\p_\mu y^b_2(x)-y^a_2(x)\a_a^\mu(x)\p_\mu y^b_1(x)
                $$
                  
{\bf Exercise}  Show that  
       \begin{equation*}\label{anchore1}
 \ac \left([[\bs_1,\bs_2]]\right)=[\ac(\bs_1),\ac(\bs_2)]
            \end{equation*}
 This is the condition
of morphisms of algebroid to the tangent bundle algeborid
(see alter).  Later we also will give a conceptual proof of this statement.

\m

  Let $E\to M$ be an algebroid with commutator $[[\,,\,]]$ and anchor map $a$.

  Now we consider other three manifestations of an alfgebroid.

{\sl II-nd manifestation of algebroid}  For an algebroid $E\to M$
consider fibbre bundle $\Pi E\to M$ with opposite parities of fibres
($\Pi$ parity reversing functor).   One can define
  an homological vector field $Q$ on $\Pi E$ of weight $\s=1$
defined by commutator relations and anchor in the following way:
 In local coordinates $(x^\mu,y^a)$ (see \eqref{defofalgebroid1}) 
                \begin{equation}\label{hofieldforalgebroid1}
            Q=\xi^a\xi^b c_{ba}^d(x){\p\over \p \xi^d}+
            \xi^a\a^\mu_a(x){\p\over \p x^\mu}\,.
                 \end{equation}
Here $(x^\mu,\xi^a)$ are local coordinates on $\Pi E$ corresponding to 
local coordinates $(x^\mu,y^a)$ on $ E$.

  If $\bs_1(x)$, $\bs_2(x)$ are two sections, then
              $$
[[\bs_1,\bs_2]]=\Pi\left(\left[\left[Q,\Pi \bs_1\right],\Pi \bs_2
         \right]\right)\,.
              $$


  The condition that $Q$ defines the algebroid is equivalent to the condition
that $Q^2=0$.
 

\subsection{Examples of algebroid}

  \subsubsection{Tangent bundle algebroid in all its manifestations}

Let $M$ be a manifold. Consider tangent bundle algebroid
 $TM\to M$ with $[[\,\,,\,\,]]$ be equal to usual commutator 
$[\,\,,\,\,]$ of vector and anchore--identity map.

Consider all other manifestations of this algebroid.

  II-nd manifestation: tangent bundle $\Pi TM\to M$ with homological
vector field, de Rham differential
       \begin{equation*}
  Q=dx^m{\p\over \ x^m}
        \end{equation*}
($x^m$-local coordinates on $M$.)
If $\bs_1,\bs_2$ two sections of tangent bundle $TM\to M$
        $$
[[]]
        $$

III-rd manifestation: contangent bundle $T*M$ with canonical symplectic 
structure.
IY-th manifestation: contangent bundle $\Pi T*M$ with canonical odd
symplectic structure, Schouten commutator.


  \subsubsection {Algebroid $T^*M\to M$ in all its manifestations
for Poisson manifold $(M,P)$}

  We define commutator and anchor by relations:
      $$
[[df,dg]]=d\{f,g\}\,, \ac(df)=D_f
      $$


{II-nd manifestation}: 
Fibre bundle $\Pi T^*M$ with homological vector field
        \begin{equation*}\label{}         
  Q=\theta_i\theta_j\p_r P^{ij}{\p\over \p \theta_\r}
   \pm\theta_iP^{ij}{\p\over \p x^j}       
 \end{equation*}


{III-nd manifestation}: Linear (in fibres) even Poisson bracket on $TM$,
i.e. Poisson bracket defined by relations
       $$
   \{v^i,v^j\}_0=v^r\p_r P^{ij}\,,\,\, \{v^i,x^j\}=P^{ij}
       $$ 

{II-nd manifestation}:  Linear (in fibres) odd Poisson bracket on
          $\Pi TM$, Koszul bracket, odd Poisson 
bracket defined by relations
       $$
   \{\theta^i,\theta^j\}_0=\theta^r\p_r P^{ij}\,,\,\, 
      \{\theta^i,x^j\}=P^{ij}
       $$ 

\subsection {Morphisms of algebroids}
  Let $E_1\to M$, $E_2\to M$ be two algebroids on the same base.
We define morphisms of these algebroids in all manifestations.


    {\sl Note that it is very inmprotant the special case: the morphism
of aarbitrary algebroid $E\to M$ on the tangent algebroid $TM\to M$.
}
 {I-st manifestation}


\section {$L_\infty$ algeborids. Definition in all its manifestations.}
`
 \end{document}


