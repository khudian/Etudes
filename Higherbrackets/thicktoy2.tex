




\magnification=1200

\baselineskip=17pt



\def\vare {\varepsilon}
\def\A {{\bf A}}
\def\t {\tilde}
\def\a {\alpha}
\def\K {{\bf K}}
\def\N {{\bf N}}
\def\V {{\cal V}}
\def\s {{\sigma}}
\def\S {{\bf S}}
\def\s {{\sigma}}
\def\bs {{\bf s}}
\def\p{\partial}
\def\vare{{\varepsilon}}
\def\Q {{\bf Q}}
\def\D {{\cal D}}
\def\L {{\cal L}}
\def\G {{\Gamma}}
\def\C {{\bf C}}
\def\M {{\cal M}}
\def\Z {{\bf Z}}
\def\U  {{\cal U}}
\def\H {{\cal H}}
\def\R  {{\bf R}}
\def\E  {{\bf E}}
\def\l {\lambda}
\def\degree {{\bf {\rm degree}\,\,}}
\def \finish {${\,\,\vrule height1mm depth2mm width 8pt}$}
\def \m {\medskip}
\def\p {\partial}
\def\r {{\bf r}}
\def\v {{\bf v}}
\def\n {{\bf n}}
\def\t {{\bf t}}
\def\b {{\bf b}}
\def\e{{\bf e}}
\def\f{{\bf f}}
\def\ac {{\bf a}}
\def \X   {{\bf X}}
\def \Y   {{\bf Y}}
\def\diag {\rm diag\,\,}
\def\pt {{\bf p}}
\def\w {\omega}
\def\la{\langle}
\def\ra{\rangle}
\def\x{{\bf x}}
\def\m {\medskip}
\def\thick {{\buildrel \to\over \to}}

  \centerline{\bf Two toy examples.}

 Let $U,V$ be a vector spaces, and $L\colon U\to V$ linear map from
   $V$ to $U$. One can consider its adjont $L^*\colon V^*\to U^*$:
                      $$
L^*\colon V^*\to U^*\,\, \hbox {such that for arbitr.}\,\,
     \Y\in U, \langle\Y, L^*(\w) \rangle
              =\langle L(\Y), \w \rangle\,.
                      $$
  In the case if $V=U^*$ both  $L$ and $L^*$
map $U\to U^*$.
  Map $L$ is self-adjoint (anti-self adjoint if $L=L^*$.
   This is standard text-book stuff. Using
   Vornov's thick mrophisms in fact generalise
this statement for non-linear maps.

   All the constructions will be based on
   Voronov's thick morphisms.

        For vector spaces
 $U,V$ consider in a symplectic space $T^*U\times (-T^*V)$
   Lagrangian surface $\Lambda_S$ defined by the function  $S(x,q)$:
                  $$
 \Lambda_S=\{(x^i,p_j,y^a,q_b)\colon\,\, p_i={\p S(x,q)\over \p x^i}\,\,
    y^a={\p S(x,q)\over \p q_a},\}
                 $$
($x^i$ are coordinates in $U$, $p_j$ are coordinates in $U^*$, 
respectively$y^a$ are coordinates in $V$, $q_b$ are coordinates in $V^*$.
Symplectic form is $dp_i\wedge dx^i-dq_a\wedge dy^a$)

  Lagrangian surface $S=S_\Phi$ defines thick morphism $\Phi\colon U\to V$.
    (This is definition.)

{\bf Core of construction}:

  Thick morphism  $\Phi$ defines pull-back
$\Phi^*$  from space  $C(V)$ of functions on $V$ to 
space of functions $C(U)$ on $U$ in the following way:
               $$
C(V)\ni g(y)\to \Phi_S^*(g)(x)=f(x)=g(y)+S(x,q)-y^aq_a\,,\,\,
                           \eqno (defintion)
\,\,
               $$
where $y^a=y^a(x), q_b=q_b(x)$ are defined by the conditions
                    $$
y^a={\p S(x,q)\over \p q_a}\,,    q_b={\p g(y)\over y^b}\,.
\eqno {definition1}
                    $$
Functions $g$ and $f=\Phi^*g$ define Lagrangian
surfaces in $T^*V$ and $T^*U$ respectively and
they are realted by lLagrangian surface $\L_S$.

Note that the map $\Phi^*$ in general is non-linear.

  Two reamkrs which explain the definitions above:

 {\bf Remark} Let $\phi$ be a usual (may be non-linear) map
   $y^a=F^a(x^i)$. Assign to this map Lagrangian surface
            $$
            S_\varphi(x,q)=F^a(x)q_a\,.
            $$
Then it is easy to see that
  the pull-back $\phi^*$ is nothing but usual pull-back,
which is a usual homomorphism of functions:
    If $S=F^a(x)q_a$, then equations
   (definit 1) imply that $y^a=F^a(x)$, hence it follows from
equation (definition) that
          $$
f(x)=(g(y)+F^a(x)q_a-y^aq_a)\big\vert_{y^a=F^a(x)}=g(F(x))\,.
          $$
 
{\bf Remark}
Notice that the non-linear pull-back can be defined as `classical limit'
of the following integral:
               $$
f(x)\colon \,\,
  e^{f(x)}\approx
 \int e^{i\tau\left(S(x,q)-y^aq_a+g(y)\right)} DyDq\,, 
       \tau\to \infty\,.
               $$

   What for is it useful?

  {\bf Theorem}(Voronov)

Every Hamiltonian $H=H(x,p)$ on $T^*V$
defines Hamilton-Jacobi vector field $\X_H$ on the space of functions on $U$:
                   $$
\X_H\colon f\to f+\varepsilon H\left(x, p={\p f\over \p x}\right)
                   $$

Let $\Phi$ be an arbitrary thick morphism  $\Phi\colon U\to V$.

 

  Let $H(x,p)$, $h(y,q)$ be Hamiltonians on
    $T^*U$ and $T^*V$ respectively.

   We say that these Hamiltonians are $\Phi$-related if
                       $$
                H\left(x,{\p S(x,q)\over \p x^a}\right)\equiv
                h\left({\p S(x,q)\over \p q^a}, q\right)
                       $$
where $S$ is a function generating the thick morphism.



  Consider on the spaces $C(U), C(V)$ (infinite-dimensional space!)
vector fields
    $\X_H, \Y_h$. Then these vector fields
are $\Phi$-related if their Hamiltonians are $\Phi$-related, i.e.
              $$
 \Phi^*\left(g+\varepsilon h\left(y, q={\p g\over \p y}\right)\right)=
 \Phi^*g+\varepsilon H\left(x, p={\p \Phi^*g\over \p x}\right)\,.
              $$
if Hamitonians are $\Phi$-related.



Now it is time to consider thick morphisms generalising adjoint maps.

  Let $\Phi=\Phi_S$ be a thick morphism from $U$ to $V$ defined by the
 Lagrangian surface $\L_S$, which in its turn
is defined by generatig function  $S=S(x^a,q_b)$, as above.

   An arbitrary thick morphism from $V^*$ to $U^*$
has to be defined by generating function $S'(q_a,x^b)$.
   We define a morphism  $\Phi^*$
adjoint to the morphism $\Phi=\Phi_S$, as a morphism defined
by the `same' generating function
             $$
      S'(q,x)=S(x,q)
                  $$

{\bf Exercise} Consider morphism $\Phi\colon U\to V$ 
with generating function
   $S(x,q)=A_{i}^a x^iq_a$.  This is usual linear map  $y^a=A_i^ax^i$.
  Its adjoint is generated by the `same' function
  $S(q,x)=A_i^a q_a x^i$ (we change in purpose the order 
of arguments, emphasizing that the first argument is the argument of the map,
and the second is conjugate to its value.)
It is nothing but adjoint map: $p_i=A_i^aq_a$.


We see that in the special case
if $V=U^*$ every thick morphism has its pair. 


 


we consider its adjoint: $\Phi^*=\Phi^*_{S^*}$. where  



  Consider the map from
   $V$ to $V^*$, i.e.  collection
   $\{F_0,  F_1, F_2,\dots, F_i,\dots\}$ of maps, where
   $F_k$ is $k$-linear map from $V$, to $V^*$,
which is symmetrical with respect to 
$k-1$ arguments:  
              $$
F_0\in V^*\,   F_1\in V^*\otimes V^*\,,
 F_2\in V^*\otimes  V^*\otimes_S V^*\,,
            F_k\in V^*\otimes \underbrace{V^*\otimes_S\dots\otimes+S V^*}
                            _{\hbox {$k$ times}}
              $$
This collection of maps defines non-linear map $F\colon V\to V^*$
such that
      $$
    F(\X)=F_0+F_1(\X)+F_2(\X,\X)+\dots
      $$
This non-linear map defines pull-back of functions:
                   $$
V^*\ni g(\w)\to F^*g=f(\X)=g(F(\X))\qquad
   (g \hbox {is a function on dual space $V^*$})
                   $$

One can define the function $f$ as 
               $$
f=F^*g\colon \quad  f(\X)=
           g(\w)+S(\Y,\X)-\w(\Y)\, 
               $$
where
                    $$
       S(\Y,\X)=\langle\Y, F(\X)\rangle
                    $$
and  $\X$ is chosen such that {\it 
RHS does not depend on $\Y$ and $\w$}

  Now consider thick morphism $F_{\rm th}$ adjusted to $F$.
  We consider instead function $S(\X,\Y)$ the function
$S(\Y,\X)$, then
Thick morphism defines non-linear pull-back 
                $$
f=F_{\rm thick}^*g\colon \quad  f(\X)=
           g(\w)+S(\X,\Y)-\w(\Y)\, \,\quad
      S(\X,\Y)=\langle \X,F(\Y)\rangle
               $$
where  RHS  does not depend on $\Y$ and  $\w$:
                $$
       f(\X)=g(w_a)+F_a(y)x^a- w_a y^a\,,
                   \cases
                        {   
        \w=\langle\X,\p_\Y F(\Y)\rangle\cr
              \Y=\p_\w g(\w)\cr
                     }
              $$
   In components 
              $$
       f(x^a)=g(w_a)+F_a(y)x^a- w_a y^a\,,
                   \cases
                        {   
        w_a={\p F_b(y)x^b\over \p y^a}\cr
              y^a={\p g(w)\over \p w^a}\cr
                     }
              $$
        Thus we come to the following iteration formula
                 $$
   f(\X)=g(F'(\Y)\X)\big\vert_{\Y=}
                 $$
We denote by $\X,\Y$ vectors on $V$ and $\w,\sigma$ covectors on $V^*$

  Consider a function $w=F(x)$ of one-variable.
 It is useful to ebar in mind that
   $x$ is a vector and $w$-covector. This is a map $V\to V^*$ for $V=\R$.

Consider thick  morphism adjoint to this map.
  The map $w=F(x)\colon \R\to \R^*$ can be defined as 
  









\bye
