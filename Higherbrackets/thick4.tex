\magnification=1200

\baselineskip=14pt



\def\vare {\varepsilon}
\def\A {{\bf A}}
\def\t {\tilde}
\def\a {\alpha}
\def\K {{\bf K}}
\def\N {{\bf N}}
\def\V {{\cal V}}
\def\s {{\sigma}}
\def\S {{\bf S}}
\def\s {{\sigma}}
\def\bs {{\bf s}}
\def\p{\partial}
\def\vare{{\varepsilon}}
\def\Q {{\bf Q}}
\def\D {{\cal D}}
\def\L {{\cal L}}
\def\G {{\Gamma}}
\def\C {{\bf C}}
\def\M {{\cal M}}
\def\Z {{\bf Z}}
\def\U  {{\cal U}}
\def\H {{\cal H}}
\def\R  {{\bf R}}
\def\E  {{\bf E}}
\def\l {\lambda}
\def\degree {{\bf {\rm degree}\,\,}}
\def \finish {${\,\,\vrule height1mm depth2mm width 8pt}$}
\def \m {\medskip}
\def\p {\partial}
\def\r {{\bf r}}
\def\v {{\bf v}}
\def\n {{\bf n}}
\def\t {{\bf t}}
\def\b {{\bf b}}
\def\e{{\bf e}}
\def\f{{\bf f}}
\def\ac {{\bf a}}
\def \X   {{\bf X}}
\def \Y   {{\bf Y}}
\def\diag {\rm diag\,\,}
\def\pt {{\bf p}}
\def\w {\omega}
\def\la{\langle}
\def\ra{\rangle}
\def\x{{\bf x}}
\def\m {\medskip}
\def\thick {{\buildrel \to\over \to}}

  \centerline{\bf One construction}

We consider two fibre bundles 
 $E_1=\Pi T^*M\to M$ and $E_2=\Pi TM\to M$.

  Coordinates on $E_1$ are $(x^a,\theta_b)$, coordinates on $E_2$ are 
$(y^a,\xi^b)$.

  We consider also $T^*E_1=T^*(\Pi T^*M)$ with coordinates
            $$
           (x^a,\theta_b; p_a,\eta^b)
        $$
and $T^*E_2=T^*(\Pi TM)$ with coordinates
            $$
           (y^a,\xi^b; q_a,\pi_b)
        $$
We consider canonical Hamiltonian
      $$
   h=\xi^aq_a\quad\hbox{on $\Pi T^*M$}
      $$
This is linear Hamiltonian, and
        $
   (h,h)=0
       $
where $(\,,\,)$ is canonical Poisson bracket in $T^*(\Pi TM)$.

  The image of this linear Hamiltonian under  MX symplectomorphism
             $\kappa\colon  T^*(\Pi T^*M)\to T^*(\Pi TM)$ is 
quadratic Hamiltonian
           $$
        H=k^*h=p_a\eta^a
           $$
Hamiltonian $h$ generates degenerate
  homotopy Schouten bracket on $\Pi T^*M$---de Rham differential
and Hamiltonian $H$ generates canonical Schoutten bracket on $\Pi T^*M$:
            $$
       \matrix
         {
\hbox{for arbitrary function $F$ on $\Pi T^*M$}\,, (h,F)=dF\cr    
\hbox{for arbitrary functions $G,H$ on $\Pi TM$}\,, ((H,G),H)=[G,H]\cr    
         }   
         $$
all other brackets vanish.

Let as usual $P$ be a function on $\Pi T^*M$ such that
        $$
    [P,P]=((H,P),P)=0\,.
        $$
This master-function defines on $\Pi T^*M$ Lichnerowicz differential
        $$
   d_P F=[P,F]=((H,P),F)
        $$
Consider on $\Pi T^*M$
{\it linear Hamiltonian $H_P$ such that the Lichneroitch differential
 is just its homotopy Schouten bracket}, i.e.
      $$
  (H_P, F)=d_PF
      $$
  Hence
      $$
   H_P=(P,H)
 =(P,H)=(P,p_a\eta^a)=
 p_a{\p P(x,\theta)\over \p \theta_a}+
    \eta^a{\p P(x,\theta)\over \p x^a}\,.
     $$
Now consider the Hamiltonian $h_P$ in $\Pi TM$, which is
 image of  $H_P$ under MX symplectomorphism:
          $$
    h_p={(\kappa^{-1})}^*H_p
=q_a{\p P(x,\pi)\over \p_a}+
    \xi^a{\p P(x,\pi)\over \p x^a}
          $$  
(znaki ne proverial)

We just come to Hamiltonian which generates
  homotopy Schoutem nracket on $\Pi TM$---higher Koszul bracket.

  We have pair of Hamiltonians $(H,H_P)$ on $T^*(\Pi T^*M)$,
           $H_P=(H,P)$
  and
   pair of Hamiltonians $(h,h_P)$
  on $T^*(\Pi TM)$ which are their MX image.


The quadratic Hamiltonian $H$ (linear Hamiltonian $h$)
generates canonical odd Poisson bracket on $\Pi T^*M$
   (de Rham differential on $\Pi TM$).

The linear Hamiltonian $H_P$ (Hamiltonian $h_P$)
generates Lichnerowicz differential  on $\Pi T^*M$
   (Higher Koszul bracket on $\Pi TM$)

\medskip

Now go to the map $\varphi_P\colon \Pi T^*M\to \Pi TM$
            $$
\phi_P\colon x^a=y^a, \xi^a={\p P(x,\theta)\over \p x^a}.
            $$ 

       We know that 
   $$
`   d_P(\varphi^*\w)=(H_P,\varphi^*\w)=\varphi^*(d\w)=
\varphi^*(h,\w)
           $$
in other words
{\it morphism  $\varphi_P\colon \Pi T^*M\to \Pi TM$
(more precisely, lifting of this morphism, canonical
  transformation of $T^*(\Pi TM)$
  $\hat {\varphi_P}$)   
interwins Hamiltonians $H_P$ and $h$:



{\it Conjugate thick morphism $\Phi^{(thick)}_P$
interwins Hamiltonians $H$ and $h_P$:
      $$
      $$

We see that we have {\bf three symplectomorphisms}
One MX symp,ectomrophism and two symoplectomorphisms
 thick and usual related with the map $P$.
  The later both are conjugate with each other via MX symplectomorpshim.


   
\bye
            

