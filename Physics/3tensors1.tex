
\magnification=1200 \baselineskip=14pt

\font\magnifiedfiverm=cmr10 at 10 pt

\centerline{Irreducible components of third rank tensors}

\def\vare{\varepsilon}
\def\a{\alpha}


  

  Let $T^i_{km}$ be a tensor in $E^3$
   such that $T^i_{km}=T^i_{mk}$, i.e. $T^i(x)=T^i_{km}x^kx^m$. 
What are its irreducible components\footnote{$^1$}
{(We need it considering analogues of quadropole expansion for
current $J^i(x)$)}(with respect to $SO(3)$)?

        $$
   T_{ikm}=T^n_{i(km)}+T^s_{(ikm)}\,,\qquad (18=10+8)
        $$
where $T^s$ is totally symmetric and $T^n$ is normal:
        $$
    T^s_{ikm}={1\over 3}(T_{ikm}+T_{kim}+T_{mki})\,,
    T^n_{ikm}={2\over 3}T_{ikm}-{1\over 3}T_{kim}-{1\over 3}T_{mki})\,,
         $$
        $$ 
T^n_{ikm}x^ix^kx^m=0\,, {\rm i.e.}\,\,   
T^n_{ikm}+T^n_{kim}+T^n_{mki}=0, \,\, (T^n_{ikm}=T^n_{imk})\,.
        \eqno (1.1c)
        $$  
Now consider restriction on the group $SO(3)$.

 Space of tensors $T^s_{(ikm)}$ is irreducible with respect to the group
$GL(3)$, but it is reducible with respect to $SO(3)$.
Tensor $T^s_{ikm}$ is a sum of $l=3$ (symmetric traceless tensor) 
and of $l=1$:
          $$
   T^s_{imk}=T^{l=2}_{(imk)}+\left(a_i\delta_{mk}+
      a_m\delta_{ik}+a_k\delta_{mi}\right)\,, \quad (10=7+3)\,,
          \eqno (1.2)
          $$
where  $T^{l=2}_{imk}$ is symmetric traceless tensor and
         $
  a_i={1\over 5}T^s_{irr}
         $,

  Little bit more care about the $8$-dimensional
 space of tensors $T^n_{i(mk)}$, obeying conditions (1.1c) It is one of
equivalent irreducible spaces with respect to $GL(3)$
($27=10+8+8+1$), and with respect
to $SO(3)$ it is a sum of $l=2$ and $l=1$ subspaces ($8=5+3$):
 Indeed 
tensor $T^n_{i(km)}$ is the following sum
          $$
   T^n_{i(mk)}=T^{l=2}_{i(mk)}+\left(2b_i\delta_{mk}-
      b_m\delta_{ik}-b_k\delta_{mi}\right)\,,
            (1.3a)
          $$
where $b_i={1\over 4}T_{irr}$
and tensor $T^{l=2}_{ikm}$ obeys conditions:
             $$
    T^{l=2}_{irr}=0\,,\quad
             T^{l=2}_{ikm}=T^{l=2}_{imk}\,,\quad
          T^{l=2}_{ikm}x^ix^kx^m=0\,,\quad
             (1.3b)
                      $$
One can see that it is pseudotensor of angular momentum  $l=2$
          $$
T^{l=2}_{ikm}=\vare_{ikr}t_{rm}+\vare_{imr}t_{rk}\,,\quad 
 t_{mn}={1\over 3}\vare_{ikm}T^{l=2}_{ikn} 
          $$
where $t_{rk}$ is traceless symmetric tensor ($l=2$).

{\bf Exrcise} Analyze these maps in detail.
Show that map above is one-one correspondence.

  Finally collecting these formulae
we see that tensor $T_{imk}$ which is symmetric with 
second and third index possesses one field of $l=3$, 
one field of $l=2$ and two vector fields:
                $$
   T_{ikm}=\underbrace {t_{ikm}}_{l=3}+
          \underbrace{\vare_{ikr}t_{rm}+\vare_{imr}t_{rk}}_{l=2}+
            \underbrace {a_i\delta_{mk}+
      a_m\delta_{ik}+a_k\delta_{mi}}_{l=1}+
        \underbrace{2b_i\delta_{mk}-
      b_m\delta_{ik}-b_k\delta_{mi}}_{l=1}
            \eqno (1.4)
                $$

{\bf Remark} Notice that one can consider instead decomposition
(1.3) the decomposition:
          $$
   T^n_{i(mk)}=T^{l=2}_{i(mk)}+\left(2b_i\delta_{mk}-
      b_m\delta_{ik}-b_k\delta_{mi}\right)\,,
            (1.3'a)
          $$
where instead conditions (1.3b) we have  
for  the tensor $T^{l=2}_{ikm}$ the  conditions:
             $$
    T^{l=2}_{rri}=0\,,\quad
             T^{l=2}_{ikm}=T^{l=2}_{imk}\,,\quad
          T^{l=2}_{ikm}x^ix^kx^m=0\,,\quad
             (1.3'b)
                      $$
In this case  condition 
$b_i={1\over 4}T_{irr}$ for (1.3a) transforms to condition
`$b_i=-{1\over 2}T_{rri}$.

iBoth conditions define the same subsapces. This follows from
the uniqueness of decomposition of space on irreducible components.
Onwe can see it by brute force: $T^{rri}=0\leftrightarrow T_{iir}=0$
for tensors obeu=ying condition (1.1c).


  \bigskip 



  What happens in the case of general tensor. In this case
first two words about decomposition on irreducible subspaces
with respect to group $GL(n)$ (We temporarlly consider arbitrary dimension)

Consider first decompostion  on tensors
symmetric and antisymmetric with respect to $k,m$:
          $$
   T_{ikm}=T_{i(km)}+T_{i[km]}
          $$ 

We decompose $T_{i(km)}$ on the sum of symmetric and normal tensor
as in (1.1) and $T_{i[mk]}$ on a sum of antisymmetric and normal
   



Then for general symmetric group we will have
         $$
T_{ikm}=\underbrace{t_{ikm}}_{\hbox{symmetric}}+
              \underbrace{ u^n_{i(km)}+v^n_{i[km]}}_
        {\hbox {two equival. irreduc. repres.}}+
    \underbrace{s_{[ikm]}}_{\hbox{antisymmetric}}\,,
         $$
where $u^n_{(ikm)}=v^n_{[ikm]}=0$:
      $$
t_{ikm}={1\over 6}(T_{ikm}+T_{imk}+T_{kmi}+T_{kim}+T_{mik}+T_{mki})\,,
      $$
      $$
u^n_{ikm}={1\over 6}(2T_{ikm}+2T_{imk}-T_{kmi}-T_{kim}-T_{mik}-T_{mki})\,,
      $$
($u^n_{ikm}=u^n_{imk},  u^n_{ikm}x^i x^m x^k=0)$.)
 
    $$
s_{ikm}={1\over 6}(T_{ikm}-T_{imk}+T_{kmi}-T_{kim}+T_{mik}-T_{mki})\,,
      $$
      $$
v^n_{ikm}={1\over 6}(2T_{ikm}+2T_{imk}-T_{kmi}+T_{kim}-T_{mik}+T_{mki})\,,
      $$
($v^n_{ikm}=-v^n_{imk},  v^n_{ikm}\xi^i\xi^m\xi^k=0)$,
$\xi$ is odd variable)).

     For dimensions we have
             $$
n^3=\underbrace{{n(n+1)(n+2)\over 6}}_{\hbox{symmetric}}+
              \underbrace{2\times {n(n+1)(n-1)\over 3}}_
        {\hbox {two equival. irreduc. repres.}}+
    \underbrace{{n(n-1)(n-2)\over 6}}_{\hbox{antisymmetric}}\,,
             $$
In particular for $n=3$    
     $$
   27=10+8+8+1
         $$

Restrict on the group $SO(3)$. We already did  for symmetric part
  $T_{i(mk)}=t_{imk}+u^n_{imk}$ (see (1.4))  Analogously we do for
for antisymmetric part. 
For $n=3$ $s_{imk}=s\vare_{imk}$ defines pseudoscalar.
For tensor $v^n_{imk}$:
      $$
v^n_{imk}=-v^n_{ikm}, \quad v^n_{imk}\xi^i\xi^m\xi^k=0),\qquad {\rm i.e.}\,
\vare_{imk}v_{imk}=0
     $$
we come to decomposition like in (1.3'a):
           $$
v^n_{imk}=\tau_{ij}\vare_{imk}+(\delta_{im}\beta_k-\delta_{ik}\beta_m)\,.
           $$
            
where $\tau_{ij}$ is symmetric traceless tensor ($l=2$) and $\beta_i$
is a vector ($l=1$)
We come to decomposition:
  $$
        T_{ikm}=\underbrace {t_{ikm}}_{l=3}+
              \underbrace {a_i\delta_{mk}+
      \a_m\delta_{ik}+\a_k\delta_{mi}}_{l=1}+
\underbrace{\vare_{ikr}t_{rm}+\vare_{imr}t_{rk}}_{l=2}+
        \underbrace{2b_i\delta_{mk}-
      b_m\delta_{ik}-b_k\delta_{mi}}_{l=1}+
              $$
                 $$
        \underbrace{\tau_{ij}\vare_{jkm}}_{l=2}+
        \underbrace{2\beta_i\delta_{mk}-
      \beta_m\delta_{ik}-\beta_k\delta_{mi}}_{l=1}+
     \underbrace{v\vare_{ikm}}_{l=0}
                 $$

We will have one particle with $l=3$, two with $l=2$, three vector
fields, $l=1$
and one pseudoscalar: 
              $$
27=10+8+8+1=(7+3)+(5+3)+(5+3)+1\,
              $$

\bye

