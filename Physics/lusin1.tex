
   \documentclass[12pt,reqno,a4paper]{amsart}
%\documentclass[12pt,reqno]{article}
\usepackage[russian]{babel}
\usepackage[utf8]{inputenc}
\usepackage[T2A]{fontenc}

\usepackage{amsmath,amssymb,amsfonts,amsthm}

\def\w{\omega}





\begin{document}

 Несколько дней тому назад мне позвонил мой приятель, Айк Басеян, 
и спросил у меня  знаю ли я, что Луна притягивается к Солнцу 
раза два-три больше чем  к Земле.  Вопрос заставил меня врасплох. 
Конечно же я быстро понял как это оценить, 
не заглядывая в никакие справочники:
                         $$
{\text{сила притяжения Луны к Земле}
         \over
  \text{сила притяжения Луны к Солнцу}
                 } \approx{\w^2 r\over \Omega^2 R} \,,
                         $$
где $\w$ и $\Omega$ это периоды обращения Луны вокруг Земли и
 Солнца, а
    $r$ и  $R$ это расстояния от Луны до Земли и до Солнца.
  Любой культурный человек знает, что
                      $$
         {\w\over \Omega}={{1\over 28\, \text {дней}}
              \over {1\over 1\,\text {год} }}=
                {365\over 28}\approx 13\,,
                      $$
и                      $$
         {r\over R}=
             {1{1\over 4}\, \text {световых секунд }\over 8\,
                \text {световых минут} }\approx {1\over 400}\,.
                      $$
Значит                     $$
{\text{сила притяжения Луны к Земле}
         \over
  \text{сила притяжения Луны к Солнцу}
                 } \approx{\w^2 r\over \Omega^2 R}\approx
\left(\w\over \Omega\right)^2{r\over R}\approx {{13^2\over 400}}\approx
              {1\over 2.4}
                         $$
Этот ответ вызывает некоторое неприятие, 
ведь Луна всё-таки двигается вокруг Земли!
 Кстати человека три которым я задал этот вопрос 
отреагировали на него так же как и я: 
они достаточно легко оценили это отношение, и 
немного удивились полученному ответу.

...В чём же всё-таки дело?  Не мне и не тут
 стараться  ответить на этот вопрос. Движение Луны вокруг Земли и Солнца
достаточно серьёзная задача.  Я б хотел 
всё-таки привести некоторые элементарные
доводы, которые `защищают'  истинное положение дел.





\end{document}
