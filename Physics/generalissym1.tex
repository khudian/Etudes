


\magnification=1200 %\baselineskip=14pt
\def\vare {\varepsilon}
\def\A {{\bf A}}
\def\t {\tilde}
\def\a {\alpha}
\def\d {\delta}
\def\K {{\bf K}}
\def\N {{\bf N}}
\def\V {{\cal V}}
\def\s {{\sigma}}
\def\S {{\Sigma}}
\def\s {{\sigma}}
\def\p{\partial}
\def\vare{{\varepsilon}}
\def\Q {{\bf Q}}
\def\O {{\bf O}}
\def\D {{\cal D}}
\def\G {{\Gamma}}
\def\C {{\bf C}}
\def\M {{\cal M}}
\def\hM {{\hat M}}
\def\Z {{\bf Z}}
\def\L  {{\cal L}}
\def\H {{\cal H}}
\def\R  {{\bf R}}
\def\S  {{\bf S}}
\def\E  {{\bf E}}
\def\l {\lambda}
\def\degree {{\bf {\rm degree}\,\,}}
\def \finish {${\,\,\vrule height1mm depth2mm width 8pt}$}
\def \m {\medskip}
\def\p {\partial}
\def\r {{\bf r}}
\def\v {{\bf v}}
\def\n {{\bf n}}
\def\t {{\bf t}}
\def\b {{\bf b}}
\def\c {{\bf c }}
\def\e{{\bf e}}
\def\ac {{\bf a}}
\def \X   {{\bf X}}
\def \Y   {{\bf Y}}
\def \x   {{\bf x}}
\def \y   {{\bf y}}
\def \G{{\cal G}}
\def\w{\omega}
\def\finish {${\,\,\vrule height1mm depth2mm width 8pt}$}


   Let $L=L(q,\dot q)$ be a Lagrangian on $\E^3$ which is invariant
with respect to rotations up to derivatives:
                   $$
\delta_i L=d\a_i, \quad \hbox {i.e.}\,,\L_{\hM_i}L=
               {d\over dt}\a_i(q)={\p \a_i(q)\over \p q^k}\dot q^k\,,
                     \eqno (1)
                   $$ 
where
                      $$
\hM_i=\vare_{imk}q^m\p_k,\quad \cases{
                      \hM_1=y\p_z-z\p_y\cr
                      \hM_2=z\p_x-x\p_z\cr
                      \hM_1=x\p_y-y\p_x\cr
                               }, \quad
        [\hM_i,\hM_k]=\vare_{ikm}\hM_m\,.
                          \eqno (2)
                      $$
Show that this generalised symmetry is not essentially generalised,
i.e. one can redefine Lagrangian $\tilde L=L-F$ 
such that for new Lagrangian $\d_i\tilde L=0$, i,e, we can 
  find a function $F=F(q)$ such that
                       $$
        \d_i F=\a_i,\,\,{i.e.}\,\,\,
             \L_{M_i} \left({dF\over dt}\right)=
             \L_{M_i} \left({\p F(q)\over \p q^k}\dot q^k\right)=
               {d\over dt}\a_i(q)={\p \a_i(q)\over \p q^k}\dot q^k\,,
                       \eqno (3)
                      $$  

i.e. for the new Lagrangian $\tilde L=L-F$,
                $$
\delta_i \tilde L=\delta_i\left(L-F\right)=0\,.
           \eqno(4)
                $$

To show the existence of `coboundary' $F$ which transform generalised symmetry
to usual one, recall that it follows from equation (1) that
                   $$
 d\left(\left(\delta\a\right)_{ij}\right)=
\left(\delta\left(d\a\right)\right)_{ij}=(\delta^2 L)_{ij}=0\Rightarrow
     (\delta\a)_{ij}=\w_{ij}\,\,\hbox {are constants}
              \eqno (5)
           $$
i.e.
                 $$
(\d\a)_{ij}=
\d_i\a_j-\d_j\a_i-\vare_{ijm}\a_m=\hM_i(\a_j)-\hM_j(\a_i)-
     \vare_{ijm}\a_m=\w_{ij}
            \eqno (5a) 
                   $$
is a cocycle in constants. On the other hand 
                $$
\w_{ij}=\vare_{ijm}t_m\,,\quad (t_m={1\over 2}\vare_{mpq}\w_{pq}).
               \eqno (6)
                $$
(This simple relation means that $H^2(so(3),\R)=0$)

If we redefine $\a_i\mapsto \a_i+t_i$ then for new $\a_i$ we have
            $$
  (\delta\a)_{ij}=\w_{ij}\mapsto (\delta\a)_{ij}=0\,.
            $$
So from now on we will consider that cocyclw in equation (5) 
vanishes: $\w_{ij}=0$:
              $$
\d_i\a_j-\d_j\a_i-\vare_{ijm}\a_m=
\hM_i(\a_j)-\hM_j(\a_i)-\vare_{ijm}\a_m=0\,.
          \eqno (5c)
            $$
      Now we solve equation (3) using condition (5c).
Applying  operator $\hM_i$ to (5c) we come to
          $$
0=\hM_i\left(\hM_i(\a_j)-\hM_j(\a_i)-\vare_{ijm}\a_m\right)=
\hM^2\a_j-\left[\hM_i\hM_j\right]\a_i-\hM_j\left(\hM_i\a_i\right)
-\vare_{ijm}\hM_i\a_m=
           $$
          $$
\hM^2\a_j-\vare_{ijm}\hM_m\a_i-\hM_j\left(\hM_i\a_i\right)
-\vare_{ijm}\hM_i\a_m=
\hM^2\a_j-\hM_j\left(\hM_i\a_i\right)=0
          $$
Thus we come to the equation
            $$
\hM^2\a_j=\hM_j\left(\hM_i\a_i\right)\,.
      \eqno (7)
            $$
Consider the expansion of $\a_i(q)$
over harmonics:
         $$
\a_i(q)=\sum_l\a_i^{(l)}(q)\,,
        \eqno (8)
         $$
where we denote by $F^{(l)}$ the function which is eigenfunction of
operator $M^2$ with eigenvalue $l(l+1)$\footnote{$^*$}
{In spherical coordinates
     $$
        F^{(l)}(q)=\sum_{m=-l}^lc_m(r)\Y_{lm}(\theta,\varphi)\,,\quad
             \Y_{lm}(\theta,\varphi)=P_{lm}(\theta)e^{im\varphi}
      $$
where $P_{lm}$ are adjoint Legendre polynomials.
In Cartesian coordinates  $\Y_{lm}$ is restriction on the sphere
of harmonic polynomial ($\Delta P(x,y,z)=0$) of the weight $l$  
    }
      
 $$
   \hM^2 F^{(l)}=l(l+1)F^{(l)}\,.
        $$

\smallskip

{\bf Observation}  Condition (5c) implies that zeorth harmonic in
(8) vanishes:
           $$  
\a_i(q)=\sum_{l\geq 1}\a_i^{(l)}(q)
           $$     
Indeed $l=0$ harmonics does not depend on
$\theta,\varphi$, $\a_i^{(0)}(q)=\a_i^{(0)}(r)$, hence
   $\delta_i\a_j^{(0)}=0$. This implies that $\vare_{ijm}\a_m^{(0)}=0$,
 i.e. $\a_m^{(0)}=0$.

\medskip

Now using this Observation, put expansion 
(8) in (7). We come to
          $$
\hM^2\a_i^{(l)}=l(l+1)\a_i^{(l)}=\hM_i(\hM_k\a_k^{(l)})\,,\quad l=1,2,3,\dots
          $$
i.e.
           $$
\a_i=\sum_{l\geq 1}\a_i^{(l)}=\hM_i\left(
       \sum_{l\geq 1} {\hM_k \a_k^{(l)}\over l(l+1)}
             \right)
           $$
We see that
             $$
\a_i=\d_i F\,\, {\rm where}
        F=\sum_{l\geq 1} {\hM_k \a_k^{(l)}\over l(l+1)}
             $$
Thus we solved equation (3).

\bye
