

\magnification=1200
\baselineskip=14pt
\def\vare {\varepsilon}
\def\A {{\bf A}}
\def\t {\tilde}
\def\a {\alpha}
\def\K {{\bf K}}
\def\N {{\bf N}}
\def\V {{\cal V}}
\def\s {{\sigma}}
\def\S {{\Sigma}}
\def\s {{\sigma}}
\def\p{\partial}
\def\vare{{\varepsilon}}
\def\Q {{\bf Q}}
\def\D {{\cal D}}
\def\G {{\Gamma}}
\def\C {{\bf C}}
\def\M {{\cal M}}
\def\Z {{\bf Z}}
\def\U  {{\cal U}}
\def\H {{\cal H}}
\def\R  {{\bf R}}
\def\S  {{\bf S}}
\def\E  {{\bf E}}
\def\l {\lambda}
\def\ll {{\bf l}}
\def\degree {{\bf {\rm degree}\,\,}}
\def \finish {${\,\,\vrule height1mm depth2mm width 8pt}$}
\def \m {\medskip}
\def\p {\partial}
\def\r {{\bf r}}
\def\pt {{\bf p}}
\def\v {{\bf v}}
\def\n {{\bf n}}
\def\t {{\bf t}}
\def\b {{\bf b}}
\def\c {{\bf c }}
\def\e{{\bf e}}
\def\ac {{\bf a}}
\def \X   {{\bf X}}
\def \Y   {{\bf Y}}
\def \x   {{\bf x}}
\def \y   {{\bf y}}
\def \G{{\cal G}}
\def\w {{\omega}}
\def \Tr  {{\rm Tr\,}}
\def\V {{\cal V}}
% I began this file in October  2018

{\tt 28 October 2018}

{\it Here I will try to caclulate the continual integral for free aprticle using  some linear algebra stuff}


 {\tt     To calculate the continual integral for free function we deal with
exponent of the polynomial
       $$
F=
(x_{0}-x_{1})^2
+(x_{1}-x_{2})^2
+(x_{2}-x_{3})^2
+(x_{3}-x_{2})^2+\dots+
(x_{N-1}-x_{N})^2\,,
   \eqno (0.1)
       $$
where initial and final points are fixed
    $$
x_0=a,x_N=b
    $$
One has to calculate the gaussian integral $e^{iCF}$:
             $$
 \int \exp(icF) dx_1dx_2\dots dx_{N-1}
             $$
In the classical  book of Feynman the author caclculates the integral,
 step by step
performing the trick: every next integration over $dx_{i+1}$ 
gives the same answer as the previous 
up to the coefficient depending on $i$.

   We will present here the straightforward  calculation
which is based on the linear algebra.
}


\bigskip


   $F$ in equation (0.1)  is quadratic polynomial over
$N-1$ variables $x_1,\dots,x_{N-1}$. We perform affine
transformation to the new cooridnates such that
in these coordinates $F$ will have only quadratic and zero order terms.

Moreover we will try to consider transfromations with unity Jacobian.


 Consider first affine  transformation
     $$
       \cases
     {
\xi_1=x_1-x_0=x_1-a\cr
\xi_2=x_2-x_1\cr
\xi_3=x_3-x_2\cr
   \dots\cr
\xi_{N-1}=x_{N-1}-x_N\cr
    }
  \Leftrightarrow
   \cases
       {
x_1=\xi_1+a\cr
x_2=\xi_2+\xi_1+a\cr
x_3=\xi_3+\xi_2+\xi_1+a\cr
   \dots\cr
x_{N-1}=\xi_{N-1}+\dots+\xi_1+a\cr
       }\,
     $$
The ``linear''  part iof this transfromation is not 
orthogonal transformation, but it is unimodular 
transformation. In the new coordinates
      $$
F=\sum_{i,k=1}^{N-1}M_{ik}x^ix^k+\sum_{i=1}^{N-1}L_ix^i+N=
    \xi_1^2+\xi_2^2+\xi_3^2+\dots+\xi_{N-1}^2+
   (b-a-\xi_1-\xi_2-\dots-\xi_{N-1})^2\,.
                   \eqno (1.1)
      $$
 Now consider new coordinates
              $$
\xi_i=\eta_i+{b-a\over N}\,, \qquad i=1,\dots,N-1
              $$
One can see that in these coordinates linear terms in (1.1) will be killed:
         $$
F=\sum_{i,k=1}^{N-1}M_{ik}x^ix^k+\sum_{i=1}^{N-1}L_ix^i+N=
    \xi_1^2+\xi_2^2+\xi_3^2+\dots+\xi_{N-1}^2+
   \left(a-b-\xi_1-\xi_2-\dots-\xi_{N-1}\right)^2=
            $$
     $$
     \left(\eta_1^2+\eta_2^2++\dots+\eta_{N-1}^2\right)+
     2\left(\eta_1+\dots+\eta_{N-1}\right){b-a\over N}+
     {(N-1)(b-a)^2\over N^2}+
             $$
             $$
            +
  \left({a-b\over N}-\eta_1-\eta_2-\dots-\eta_{N-1}\right)^2=
     $$
     $$= 
   2\left(\eta_1^2+\eta_2^2+\dots+\eta_{N-1}^2\right)+
   2\sum_{i<j}\eta_i\eta_j+
      {(b-a)^2\over N}\,.
       $$
The linear temrs are cancelled, and we have that
      $$
F=\tilde M_{ik}u^iu^k+
      {(b-a)^2\over N}\,,
     $$
where
       $$
  \tilde M_{ik}=\delta_{ik}+t_it_k\,,\qquad
   (t_i=(1,\dots,1))\,.
       $$

Now we can calculate the integral:    
 $$
I=
\int e^{-cF}dx^1dx^2\dots dx^{N-1}=
\int e^{-c\left(M_{ik}x^ix^k+L_ix^i+N\right)}dx^1dx^2\dots dx^{N-1}=       $$
    $$
\int e^{-c\left(
\sum_{k=1}^{N-1}\xi_k^2+\left(b-a-\sum_{m=1}^{N-1}\xi_m^2 
\right)^2\right)}
    \underbrace
    {
  \left(\p(x_1,\dots,x_{N-1}\over \p\xi_1,\dots,\xi_{N-1})\right)
        }_{\hbox {equals to $1$}}
d\xi_1d\xi_2\dots d\xi_{N-1}=      
    $$
      $$
\int e^{-c\left(
2\sum_{k=1}^{N-1}\xi_k^2+\left(b-a-\sum_{m=1}^{N-1}\xi_m^2 
\right)^2\right)}
d\xi_1d\xi_2\dots d\xi_{N-1}=      
    $$
      $$
\int e^{
      -c\left(
2\sum_{i,k=1}^{N-1}\eta_i\eta_k+{(b-a)^2\over N}
        \right)
            }
    \underbrace
    {
  \left(
 \p(\xi_1,\dots,\xi_{N-1}\over \p\eta_1,\dots,\eta_{N-1})
\right)
        }_{\hbox {equals to $1$}}
d\eta_1d\eta_2\dots d\eta_{N-1}=      
    $$
      $$
\int e^{
      -c\left(
N_{ik}\eta^i\eta^k+{(b-a)^2\over N}
        \right)
            }
d\eta_1d\eta_2\dots d\eta_{N-1}=      
    $$
      $$
\left(\pi\over c\right)^{N-1\over 2}
\sqrt {1\over \det (\tilde M)}
     e^{
      -c\left(
{(b-a)^2\over N}
        \right)
            }\,.
        $$
Notice that  the matrix $\tilde M$ has eigenvector $\bf t$
with eigenvalue $N$, and all other $N-2$ eigenvectors
wchih are orthogonal to the vector $\bf t$
with the eigenvalue $1$. Hence
   $$
\det {\tilde M}=N\,,
  $$
and we have for integral:
      $$
I=\int e^{
      -c\left(
N_{ik}\eta^i\eta^k+{(b-a)^2\over N}
        \right)
            }
d\eta_1d\eta_2\dots d\eta_{N-1}=      
    $$
      $$
\sqrt {\pi\over \det (c\tilde M)}
     e^{
      -c\left(
{(b-a)^2\over N}
        \right)
            }=
 \left(\pi\over c\right)^{N-1\over 2}
   {1\over \sqrt N}
     e^{
      -c\left(
{(b-a)^2\over N}
        \right)
            }\,.
        $$


\bye
