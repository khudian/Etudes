\magnification=1200 \baselineskip=14pt

%\font\magnifiedfiverm=cmr5 at 10 pt

\def\p {\partial}
\def \D {\Delta_{d{\bf v}}}
\def \Ds  {\Delta^{\#}}
\def\t{\tilde}
\def\s {\sigma}
\def\L {\Lambda}
\def\Darboux {$z^A=$  $x^1,\dots,x^n$, $\theta_1,\dots,\theta_n$}
\def\a{\alpha}
\def\O{\Omega}
\def\d{\delta}
\def\dv  {{d{\bf{v}}}}
\def\A {{\cal A}}
\def\R {I\!R}
\def\t {\tilde}

% this was written in December 2018 or later.

{\it  Consider the angle $\varphi$ with the vertex is at the origin.  
Consider the  rod of the length $l=1$.  What is the
equilibrium position of the rod in this angle under the gravitational
force. This problem was sugested by John Parkinson
on tutorials. I was really surprised realising that this problem
is not trivial.}
   
  
  For simplicity consider the case if bisectrix of the angle
is vertical. Thus we have to study the stationary points of the function
          $$
    F=x+y\,,\qquad \hbox{under the constrain}\,\,
     x^2+y^2-2xy\cos\varphi =1\,.
          $$
Solving this constrain we see that there are two triangles if
$x<1$, one triangle if $1<x\leq {1\over \sin\varphi}$,
and no triagnle if $x>{1\over \sin\varphi}$: 
         $$
\cases 
  {
    y_+=x\cos\varphi +\sqrt {1-x^2\sin^2\varphi}\,\,\quad
 \hbox {if $x<{1\over \sin\varphi }$}\cr
    y_-=x\cos\varphi-\sqrt {1-x^2\sin^2\varphi}\,\,\quad
 \hbox {if $x\leq 1$}\cr
       }
         $$
Symmetry arguments: 
      $$
     F(x,y)=F(y,x) 
      $$
give us that one of the stationary points
of the function $F$ is the isocseles triangle: 
      $$
x_0={1\over 2\sin {\varphi\over 2}}\,,  (x_0>11)\,.
     $$
(isocseles triangle).  In this case $x>1$, and this point
is a stationary point of the function $F_+$.

Is this point equilibrium point?  Consider a function
$F_+$ in the vicinity of this point:
         $$
x=
{1\over 2\sin {\varphi\over 2}}+u\,,
         $$

We come to
       $$
\Phi_+(u)=x+y=
x+x\cos\varphi+\sqrt {1-x^2\sin^2\varphi}=
       $$
      $$
(1+\cos\varphi)
         \left(
{1\over 2\sin {\varphi\over 2}}+u
         \right)
            +
  \sqrt{1-\left(
{1\over 2\sin {\varphi\over 2}}+u
         \right)^2\sin^2\varphi
}=
   \left({\cos^2{\varphi\over 2}\over \sin{\varphi\over 2}}
     +
2\cos^2{\varphi\over 2}u\right)+
      $$
      $$
\sqrt {
    1-\cos^2{\varphi\over 2}-
       4\sin{\varphi\over 2}
     \cos^2{\varphi\over 2}u-
       4\sin^2{\varphi\over 2}
     \cos^2{\varphi\over 2}u^2
      }=
   \left({\cos^2{\varphi\over 2}\over \sin{\varphi\over 2}}
     +
2\cos^2{\varphi\over 2}u\right)+
      $$
         $$
\sin {\varphi\over 2}
 \sqrt 
      {
    1- 
      4
     {\cos^2{\varphi\over 2}
          \over\sin{\varphi\over 2}
               }u-
       4
     \cos^2{\varphi\over 2}u^2
     }=
   \left({\cos^2{\varphi\over 2}\over \sin{\varphi\over 2}}
     +
2\cos^2{\varphi\over 2}u\right)+
         $$
       $$
\sin {\varphi\over 2}
        \left
           ( 
    1- 
        2
     {\cos^2{\varphi\over 2}
          \over\sin{\varphi\over 2}
               }u-
       2
     \cos^2{\varphi\over 2}u^2
           -
     {1\over 8}
            \left(
        4\cos^2\varphi
        \left({u\over \sin {\varphi\over 2}}+u^2\right)
       \right)^2  +{1\over 16}\left(4\cos^2{\varphi\over 2}\over 
      \sin{\varphi\over 2}\right)^3+\dots
          \right)=
        $$
       $$
       $$
(Here we use the binom formula:
     $$
\sqrt {1+x}=1+{1\over 2}x+
{1\over 2}\cdot {1\over 2}\left(-{1\over 2}\right)x^2+
{1\over 6}\cdot {1\over 2}\cdot \left(-{1\over 2}\right)
  \cdot \left(-{3\over 2}\right)x^3+\dots=
    1+{x\over 2}-{x^2\over 8}+{x^3\over 16}+
 +\dots=
      $$

\bye
