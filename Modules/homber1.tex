\def\p{\partial}
\def\t {\tilde}
\def \m {\medskip}
\def\degree {{\bf {\rm degree}\,\,}}
\def \finish {${\,\,\vrule height1mm depth2mm width 8pt}$}





\def\a {\alpha}
\def\vare{{\varepsilon}}
\def\l {\lambda}
\def\s {{\sigma}}

\def\G {{\Gamma}}

\def\A {{\bf A}}
\def\C {{\bf C}}
\def\E  {{\bf E}}
\def\K {{\bf K}}
\def\N {{\bf N}}
\def\Q {{\bf Q}}
\def\R  {{\bf R}}
\def\V {{\cal V}}
\def \X   {{\bf X}}
\def \Y   {{\bf Y}}
\def\Z {{\bf Z}}



\def\ac {{\bf a}}
\def\e{{\bf e}}
\def\f {{\bf f}}
\def\n {{\bf n}}
\def\r {{\bf r}}
\def\v {{\bf v}}
\def \x   {{\bf x}}
\def \y   {{\bf y}}


\def\pt {{\bf pt}}
\def \exer {{\sl Exercise$\,\,$}}

 \centerline   
{\bf Determinant and Berezinian. Their homological interpretation.}


\bigskip

I will give here shortly homological interpretation of
(super)determinant.  Reading this text you have to do  exercises.


 First consider determinant.

         {\sl ${\cal x}$1. \bf Determinant}

\medskip

   Let $V$ be linear space. Consider the space $\Pi V\otimes V^*$.

   Consider the space $C$ of polynomials on $\Pi V\otimes V^*$. 
Let $\{\bf e_i\}$ be an arbitrary basis
   in $V$, $\{\bf e^i\}$ be a dual basis in $V^*$,
 and let $\{\bf \Pi \e_i\}$ be corresponding basis in $\Pi V$.
   We denote by $\{\theta^i\}$ coordinates of vectors in $\Pi V$ 
   and by $\{p_i\}$ coordinates of vectors in $V^*$.
   in these bases.

   Note that $\{\theta^i\}$ are odd.

   Polynomials on $C$ are polynomials on $\theta^i, p_i$.

   Consider the polynomial
           $$
           Q=\theta^ip_i=\theta^1p_1+\dots+\theta^np_n
           \eqno (1.1)
           $$
\exer 1.1  Show that $Q$ is well-defined, i.e. it does not depend on the choice of basis $\{\e_i\}$ in $V$.

\exer 1.2 Show that $Q^2=0$.


One can consider $Q$ as a differential on the space $C$.
(This is Koszul-type differential)

    Calculate (co)homology of $Q$:
             $$
          H={Z\backslash B},
             $$
where $Z$ is space of cocycles: $P\in Z$ if $QP=0$ and $B$ is space of coboundaries: $P\in P$ if $P=QR$.

   Consider the following Laplacian on $C$:
          $$
       \Delta P(p,\theta)={\p^2\over \p \theta^i\p p_i} P(p,\theta)
       \eqno (1.2)
          $$

\exer 1.3
  Calculate anticommutator
              $$
            [\Delta, Q]_+=\Delta Q+ Q\Delta
              $$
on the space  $C$.

Consider $C=\oplus_{k,r}C_{kr}$ where $C_{kr}$ is the subspace of polynomials which have weight $k$ on $\theta$
and weight $r$ on $p$, i.e.:
            $$
   P\in C_{k,r} \,\, {\rm if}\,\,     \theta^i{\p\over \p\theta^i}P=kP,\,\, p_i{\p\over \p p_i}P=rP\,.
            $$

\exer 1.4   Calculate $[\Delta, Q]_+$ on subspaces $C_{k,r}$. More exactly show that there exist number $\nu_{k,r}$ such that
for an arbitrary $P\in C_{k,r}$
            $$
[\Delta, Q]_+P=\nu_{k,r}P
            $$



\exer 1.5 Show that all cocycles coboundaries in all subspaces except the subspace $C_{n,0}$, where
$n$-dimension of the vector space $V$.


\exer 1.6 Consider cohomology class $c=[\theta^1\theta^2\dots, \theta^n]$.  Show that it is not trivial.


\exer 1.7  Show that  every cohomology class is proportional to $c$, i.e. $H(Q,C)=\R$.

\medskip

   If you did all these exercises then you are ready to see that for an arbitrary linear operator $A$ on $V$
                $$
                \det A={[\hat Ac]\over [c]}
                \eqno (1.!!!)
                $$
where $[c]$ is an arbitrary non-zero cohomology class, and $\hat Ac$ the induced action of $A$ on $c$.

The last formula is cohomological definition of determinant. The pessimist will tell that it is  just the translation from "Enlgish
(Russian) to French" of the well-known formula
              $$
             \det A={\omega(A\x_1,A\x_2,\dots, A{\x_{n-1}}A\x_n)\over \omega(\x_1,\x_2,\dots, {\x_{n-1}}\x_n)},
             \eqno (1.!!!??)
              $$
where $\omega$ is an arbitrary $n$-form , and ${\x_1,\dots,\x_n)}$
is an arbitrary set of $n$-vectors such that
$\omega(\x_1,\dots,\x_n)\not=0$.

Yes, in some sense pessimist is  right, but on the other had the advantage of the formula  (1.!!!) is that it can be easily
generalised for the
supercase opposite to the formula (1.!!!?).


\bigskip
       {\sl ${\cal x}$1. \bf Berezinian (superdetermiant)}

\medskip

  Now  let $V$ be $p|q$-dimensional  superspace.  Shortly what is it:

  Consider $Z_2$-graded vector space $V_0\oplus V_1$ such that dimension of $V_0$ equals $p$ and dimension of $V_1$ equals $q$.
  
 Let $\{\e_i,\f_\a\}$, $i=1,,\dots,p;\,\a=1,\dots,q$ be a basis in $V$. 
  An arbitrary vector  $\x$ in $V$ can be decomposed.
                    $$
             \x=a^i\e_i+b^\a\f_\a
             \eqno (2.1)
                    $$
Consider
expressions (2.1) where coefficients $a^i$ are even elements and coefficients $b^\a$ are odd elements
of an arbitrary Grassmann algebra $\Lambda$. Thus we come to the set
             $$
          V_\Lambda=\Lambda_0\otimes V_0\oplus  \Lambda_1\otimes V_1
             $$
of $\Lambda$-points of the superspace $V$ \footnote{$^{1)}$}{Superspace is a functor which assigns to an arbitrary
Grassmann algebra  $\Lambda$ the set $V_\Lambda$ of $\Lambda$-points.}




 One can consider even linear operators on the super-space $V$:
They are given by $K$ be an even $p|q\times p|q$  matrices
       $$
  K=\pmatrix
       {
  K_{00} & K_{01} \cr
  K_{10} & K_{11} \cr
  }
       $$
where entries of $p\times p$ matrix $K_{00}$ and $q\times q$ matrix
$K_{11}$ are even numbers (even elements of a Grassmann algebra),
and entries of $p\times q$ matrix $K_{01}$ and $q\times p$ matrix
$K_{10}$ are odd numbers (odd elements of a Grassmann algebra)

   Respectively $\Pi V$ is $q|p$ dimensional superspace.


   Consider the space $\Pi V\otimes V^*$.
   Coordinates on the space $\Pi V$ are
   $\{\theta^i,y^\a\}$ where $\theta^i$ are anticommuting (odd elements of Grassmanm algebra)
   ($i=1,2,\dots,p$) (have parity opposite to the parity of $a^i$) and $y^\a$
   are commuting $\a=1,\dots,q$ (have parity opposite to the parity of $b^\a$)


   Coordinates of the space $\Pi V\otimes V^*$ are
   $\{\theta^i,y^\a, p_i,\varphi_\a\}$, where $\theta^i, \varphi_\a$ are odd and $y^\a,p_i$ are even.



   We have
           $$
           Q=\theta^ip_i+y^\a\varphi_\a.
           \eqno (1.1)
           $$
\exer 1.1  Show that $Q$ is well-defined, i.e. it does not depend on the choice of basis $\{\e_i\}$ in $V$.

\exer 1.2 Show that $Q^2=0$.


In the same way as for pure bosonic case one can consider $Q$ as a differential on the space $C$ and consider
 (co)homology of $Q$:
             $$
          H={Z\backslash B},
             $$
To calculate the cohomology we consider the same Laplacian:
          $$
       \Delta ={\p^2\over \p \theta^i\p p_i}+{\p^2\over \p y^\a\p \varphi_\a}
       \eqno (1.2)
          $$
\exer In the same way like before calculate  $[\Delta, Q]_+=\Delta Q+ Q\Delta$ and show that
cohomology not vanish only in the one-dimensional subspace of polynomials
        $$
      P=a\theta^1\theta^2\dots, \theta^p\varphi_1\dots\varphi_q
        $$

\exer Show that cohomology class  $c=[\theta^1\theta^2\dots, \theta^n \varphi_1\dots\varphi_q]$ is not trivial.



Finally we come to Berezinian. Consider a matrix of an arbitrary even operator:
            $$K=\pmatrix
       {
  K_{00} & K_{01} \cr
  K_{10} & K_{11} \cr
  }
       $$
where entries of $p\times p$ matrix $K_{00}$ and $q\times q$ matrix
$K_{11}$ are even numbers (even elements of a Grassmann algebra),
and entries of $p\times q$ matrix $K_{01}$ and $q\times p$ matrix
$K_{10}$ are odd numbers (odd elements of a Grassmann algebra)


 Calculate the action of this operator on cohomology class $c=[\theta^1\theta^2\dots, \theta^n \varphi_1\dots\varphi_q]$
   we come to the formula:
           $$
       [\hat K c]={\rm Ber K}[c],
           $$
where
           $$
{\rm Ber K}={\det \left(K_{00}-K_{10}K_{11}^{-1}K_{01}\right)\over \det K_{11}}
           $$

Do it!
\bye
