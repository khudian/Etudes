\magnification=1200 \def\I {{\cal I}}






 \centerline {\bf Jets as universal object and Taylor formula for
 jets}
\bigskip
 Reading Krassilshik Verbovetsky I became really  really
 happy understanding the algebraic definition of jets!!!
\def \K {{I\!K}}
\def\D  {\Delta}
   So on:
   Let $P$ and $Q$ be two modules over commutative algebra $A$
 which is algebra over the main field $\!K$. In the space
     linear space $Hom_\K(P,Q)$ one can consider two modules structure.
The first one defining by
                      $$
  \forall \D \in Hom_\K(P,Q), a\in A\quad
            (a\circ\D)(p)=a\D(p)
                  \eqno (1)
                    $$
and the second one defined by
                      $$
  \forall \D \in Hom_\K(P,Q), a\in A\quad
            (a\circ\D)(p)=\D(ap)
                  \eqno (1)
                    $$
It is easy define the gradation. ($Diff=\oplus Diff_n$)
   We denote the first module by $Diff(P,Q)$ and the second one by
  $Diff^+(P,Q)$. The elements of $Hom_\K(P,Q)$ which are in fact
  $A$-linear (i.e. belong to $Hom_A(P,Q)$) are zero order differential
operators.   By induction the operator $\D$ belongs to
 $Diff_{n+1}$ if for every $a\in A$
                             $$
             \delta_a(\D)= \D a-a\D\in Diff_{n}
                   \eqno (2)
                         $$
 Sure the basic example is (see also the example below!)
the following

         {\bf Example 1}
  Let $P$ and $Q$  be the modules of sections (more simple
  the modules of functions) and $A$ be an algebra
 of functions. Then $Diff_n$  is nothing but $n$-order
differential operators.

 But there is another example which is very useful for jets definition!!!

     {\bf Example 2} Let $P$ be a module (simple: the module of functions)
 and  $A$ be arbitrary algebra (simple: algebra of functions)
   Let
               $$
              Q=A_K\otimes P
                                    \eqno (4)
                 $$
(note that tensor product is over $K$ not over $A$!!!) and module
structure on $Q$ is defined by
                  $$
               b\circ (a\otimes p)=ba\otimes p
                    $$
 We consider the map
                     $$
Hom_\K(P,Q)\ni\epsilon\colon\quad \epsilon(p)=1\otimes p
                          \eqno (5)
           $$
 Exercise
 This is infinite-order differential operator!!!

And this leads us to jets (see Krasilshik Verbovetsky)

 A imenno:

 We consider $\mu^k$---the subspace of $Q=A_K\otimes P$
which is generated by terms
 $\delta_{a_0\dots a_k}(\epsilon)(p)$
where $\delta_{a_0\dots a_k}(\epsilon)(p)$
are generated subsequently by commutators by the rule (2).
We define jets space by the formula
               $$
            {\cal I}^k P=(A_K\otimes P)/\mu^k
                    $$
 Magic definition!
     Exercise
We consider the functor $Diff_n(P,\cdot)$:
                    $$
         \forall Q ({\rm over A})\quad
             Q\rightarrow Diff_n(P,Q)
                   $$
i.e. the functor which put in correspondence to the module
 $Q$ $n$-th order differential operators with value in this module.
This fumctor is representing by the space
                   $$
                  Hom_A({\cal I}^n P,Q)=Diff_n(P,Q)
               $$
i.e. differential operators on $P$ with value in $Q$ are representing
by $A$-linear functions on the   space of $n$-th jets
 with value in $Q$!


$$ $$
 Taylor formula in terms of jets looks very funny:
                    $$
     1\otimes f\approx\sum_{n=0}^\infty{f^{(n)}\over n!}
             \left(
             1\otimes x-x\otimes 1
             \right)^n=
                $$
                $$
     f\otimes 1+f'\left(1\otimes x-x\otimes 1\right)+
         {f''\over 2}
           \left(
             1\otimes x^2-2x\otimes x+x^2\otimes 1
             \right)+\dots
                    $$

  {\it Proof}.  Let $\psi_d$ be $A$-linear map corresponding to
  differential operator $d$, i.e.
       $$
    \psi_d(f\otimes g)=f\otimes g'
       $$
This formula defines linear map in all spaces $\I_k=A\otimes
A/\mu_k$. In the space $\I_1$ the differential operator $d$ is
composition of linear map $\psi_d$ and of the universal differential
operator $j_1\colon\quad j_1(f)=1\times f$, which is universal
differential operator of the first order in the space $\I_1$:
             $d\psi_\circ j_1$
   It is easy to see that
                 $$
             \psi_d s_k=ks_{k-1}
                    $$
   where we denote  $s_k=\left(1\otimes x-x\otimes 1\right)^k$.
The action of $\psi_d$ on $s_k$ mimics the action of derivative on
$x^k$.  This leads to Taylor formula  (right hand side mimics
$f(x)\approx \sum_{n=0}^\infty{f^{(n)}\over n!}x^n$)


$I\!\!\! R$  $R$, $I\!\! R$

  \bye
