\magnification=1200 \baselineskip=14pt
\def\vare {\varepsilon}
\def\A {{\bf A}}
\def\t {\tilde}
\def\a {\alpha}
\def\K {{\bf K}}
\def\N {{\bf N}}
\def\V {{\cal V}}
\def\s {{\sigma}}
\def\S {{\Sigma}}
\def\s {{\sigma}}
\def\p{\partial}
\def\vare{{\varepsilon}}
\def\Q {{\bf Q}}
\def\D {{\cal D}}
\def\G {{\Gamma}}
\def\C {{\bf C}}
\def\M {{\cal M}}
\def\Z {{\bf Z}}
\def\U  {{\cal U}}
\def\H {{\cal H}}
\def\R  {{\bf R}}
\def\E  {{\bf E}}
\def\l {\lambda}
\def\degree {{\bf {\rm degree}\,\,}}
\def \finish {${\,\,\vrule height1mm depth2mm width 8pt}$}
\def \m {\medskip}
\def\p {\partial}
\def\r {{\bf r}}
\def\v {{\bf v}}
\def\n {{\bf n}}
\def\t {{\bf t}}
\def\b {{\bf b}}
\def\e{{\bf e}}
\def\ac {{\bf a}}
\def \X   {{\bf X}}
\def \Y   {{\bf Y}}
\def \x   {{\bf x}}
\def \y   {{\bf y}}
\def\f {{\bf f}}
\def\pt {{\bf pt}}
\def\w {\omega}
\centerline  {\bf  Mayer-Vietoris sequence}






Let $M$ be a manifold covered by two open sets
         $$
 M= V_1\cup V_2=\cup V_\a \quad \alpha=1,2,.
     $$
Denote by $N$ intersection of these open sets: $N=V_1\cap V_2$.

  Consider the sequence:
          $$
0{\buildrel \delta\over \longrightarrow}\,H^0(M)\,\,{\buildrel p_1\oplus p_2\over \longrightarrow}\,\, 
H^0(V_1)\oplus H_0(V_2){\buildrel l_1-l_2\over \longrightarrow}\, 
H^0(N) {\buildrel \delta\over \longrightarrow}\,
           $$
           $$
           {\buildrel \delta\over \longrightarrow}\,
            H^1(M)
 {\buildrel p_1\oplus p_2\over \longrightarrow} H^1(V_1)\oplus H^1(V_2)
 {\buildrel l_1-l_2\over \longrightarrow} H^1(N){\buildrel \delta\over \longrightarrow}
          $$
          $$
          {\buildrel \delta\over \longrightarrow}\
 H^2(M)\,\,{\buildrel p_1\oplus p_2\over \longrightarrow}\,\,
H^2(V_1)\oplus H_0(V_2){\buildrel l_1-l_2\over \longrightarrow}\,
H^2(N) {\buildrel \delta\over \longrightarrow}\, 
            $$
            $$
            {\buildrel \delta\over \longrightarrow}\
            H^3(M)
 {\buildrel p_1\oplus p_2\over \longrightarrow} H^3(V_1)\oplus H^3(V_2)
 {\buildrel l_1-l_2\over \longrightarrow} H^3(N){\buildrel \delta\over \longrightarrow}\ldots\to 0
          $$




In this sequence all maps are obvious except the map $\delta\colon H^k(N)\to H^{k+1}(M)$


(In this sequence we have $0$ and $1$ Chech cochaines. E.g. $1$-cochain $\{c_{\a\beta}\}$
such that $c_{12}=-c_{21}=1$ is close and exact cochain. )

I will write this map in such a form that it will be easy to udnerstand how to generalise it on the case
if $M$ is covering by more than 2 open sets.

Let $\{\rho_\a\}=\{\rho_1,\rho_2\}$ be partition of unity which is subordinated to the covering $M=V_1\cup V_2$, i.e.
  $supp \r_\a \subset V_\a$, $\rho_1,\rho_2\geq 0$ and $\rho_1+\rho_2=1$ 
  (For covering by two sets it is very easy to construct)
  
  
  Now consider closed $k$-form $\omega$ such that $[\w]\in H^k(N)$. Then
  $k$-form $\w$ can be consider as Chech $1$-cochain with values in $k$-forms since it is defined
  on the intersection of open sets:
              $$
           c_{12}=\w, c_{21}=-\w   
              $$ 
  Now define
           $$
          \delta [\w]=\left[\sum \rho_\a d\rho_\beta c_{\a\beta}\right] 
           $$
For this special case $d\rho_2=-d\rho_1$ since $\rho-2+\rho_1=1$                     
                     
\bye
