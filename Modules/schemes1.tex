\magnification=1200

\def\R {{\bf R}}
\def\Z {{\bf Z}}
\def\a {\alpha}




   \centerline {\bf Zorn lemma application}

{\it
  We apply Zorn lemma to show that every ideal belongs to maximal ideal.
Can we apply Zorn lemma to show that every family of ideals
has maximal ideal?  Answer is: "No" since positive answer means that ring is 
noetherian. Let us see how conditions of applying Zorn lemma are failed.}


  Let $I$ be an arbitrary ideal in an arbitrary ring $\R$.

   Consider two sets of ideals:

      $$
   M_1=\{\hbox{
set of all finitely generated ideals of ring $\R$ which 
belong to $I$}\}\,
      $$
and
      $$
   M_2=\{\hbox{
set of all ideals of ring $\R$ which possesses the ideal $I$}\}\,,\quad
      $$
 The first set for not-Noetherian rings is not inductive: it may happen
that $M_1$ possesses
well-ordered subset which has not upper bound in $M_1$.
E.g.  consider ring $\R=\R[x_i]$, 
ring of polynomials with real coefficients
which depend on variables 
$\{x_i\}, i=1,2,3,\dots$ 
(every polynomial depend on finite number of variablles). Consider ideals
   $$
  \a_k=\hbox {polynomials which vanish if $x_1=x_2=\dots=x_k=0$}
   $$
and
      $$
   I=\hbox{polynomials which vanish if at least on of variables is 
equal to zero}
      $$

 The ideal $I$ is not finitely generated. Well-ordered 
set of finitely generated
ideals $\a_k$ which belong to $I$ has not upper bound
in the set $M_1$ of finitely generated ideals. We cannot use Zorn lemma
to prove that the family $M_1$ possesses maximal ideal:
the fact that $M_1$ possesses maximal ideal means that $I$ is finitely 
generated. Indeed let $\beta$ be maximal element in $M_1$, then $\beta=I$,
since if $\beta\not=I$, we can expand $\beta$. 
{Recall that the fact that every family of ideals possesses maximal ideal
is nothing but Noether condition.}




  The second set is inductive: every well-ordered subset $N\subseteq M_2$
has upper bound in $M_2$. Indeed consider subset $N$
 of ideals $N=\{\a_\iota\}, N\subseteq M$
such that for arbitrary two ideals $\a_{\iota_1},\a_{\iota_2}\in N$
              $$
        \a_{\iota_1}\subseteq \a_{\iota_2}\,,{\rm or}\,\,
        \a_{\iota_2}\subseteq \a_{\iota_2}\,.
              $$
(This is well-ordered subset in poset\footnote{$^{1}$}
  {`poset' means {\it partially ordered set}} $M$ of ideals.)

  One can see that $N$ has upper bound in $M$
(not obligatory in $N$ itself!). Indeed consider
union of all ideals in $N$:
       $$
      T_N=\sum_\iota \a
       $$
All ideals $\a$ do not possess $1$,hence
$T\not=\R$ since it does not possess $1$ and $T\R\in T$. Hence
$T$ is ideal. We see that every well-ordered subset in $M_2$
has upper bond. Hence by Zorn lemma we come to the existence
of maximal element in $M_2$. In particular we proved very imprortant 

{\it Proposition} Every ideal is contained in maximal ideal.

\bigskip


 {\bf Resum\'e}.  
  Statement about existence of maximal ideals (in a given family) 
sometimes
follows just from  Zorn lemma, sometimes follows from noetherian
property.

Consider another good example.


{\it Proposition } If $x$ is not nilpotent element of ring $\R$,
then there exist prime ideal which does not possess this element.


Now present the proof based on Zorn lemma.

{\sl Proof}. Consider the set
       $$
M=\hbox {set of all ideals which do not contain
any  power of $x$}
       $$
$M$ is not empty, since $\{0\}\in M$.

  For noetherian ring everything is done:
every set of ideals in noetherian ring possesses  maximal ideal
 \footnote{$^{3)}$}{Indeed suppose this is wrong. Then we come
to infintie ascending series of ideals} and maximal ideal is automatically
simple.

 Now forget about the condition of being noetherian ring.

The set $M$ due to Zorn lemma possess maximal element
(this maximal element is not necessary maximal ideal!)
since every well-ordered subset in $M$ has upper bond

  Hence we come to the ideal $\beta$ such that
            $$
\forall N\,,\quad x^N\not\in \beta_x
             $$
Prove that $\beta$ is a prime ideal.

The proof follows from the lemma

{\bf Lemma} For an arbitrary $a\not \beta$ there exist $N$
such that
        $$
      x^N\in (a)+\beta
        $$

Indeed let $ab\in\beta$. Suppose that $a,b\not\in\beta$.
The lemma implies that there exist $M.N$ such that 
$x^M-pa\in\beta, x^N-qb\in \beta$ for some elements $p,q\in \R$.
Hence $x^{N+M}\in\beta$ since $ab\in\beta$,  but $x^{N+M}\not\in \beta$.
   Contradiction.

It remains to prove the lemma. 
Suppose $a\not\in \beta$. Consider subring generated by $\beta$ and $a$.
It is all the ring or the ideal. The condition that $\beta$ is maximal 
ideal which does not possess all the powers of $x$ implies the 
statement of lemma.

\bigskip


{\bf Example} Consider algebra of functions on $\R$, 
how looks ideal $\beta_x$. $\beta_x=(x-a)$, where $a\not=0$
 maixmal ideals are points of $\R$

\bigskip

\centerline {$spec A$}

Let $A$ be a ring. What can we say about $Spec A$?

If ring $A$ is an integer domain and
in aprticular it does not possess nilpotents then $\{0\}$ is  
a prime ideal. In general it is not the case, but we can consider
the set of all ideals containing the ideal $\{0\}$, and the maximal element
of this set wiil be the maximal ideal. The maximal ideal is prime
 (opposite is not true!). We come to point of $Spec A$. Moreover we
see that every point of $Spec A$ is contained in the point of $Specm A$--
every prime ideal is contained in some maximal ideal. Now recall that
factor of $A$ over maximal ideal is a field. 


  We call system uncompatible if
  there exist polynomials $P_i$ such that $\sum P_iF_i\equiv 1$, i.e.
$<F_1,\dots, F_n>=<F_i>$.
span all the ring.


  {\bf Corollalry} (Hilbert's weak Theorem)
Let $F_i(T_\a)=0$ be a system of compatible equations.
Then there exist a field $L$ such that a system have solution in this field.

  Theorem above implies that one can take as such a field, a factor of
  $A$ over some maximal ideal. In this field factors of $T_i$ are roots, 

\bigskip


  {\bf Geometrical points}

Let $A$ be an algebra over ring $K$, a ring 
induced by system of equations
           $$
            F_i(T_\a)=0\, i=1,\dots,n, \a=1,\dots,m
           $$
where $F_i\in K[T-\a]$.
 

Let $L$ be an arbitrary extension of the ring $K$, and
  $X(L)$ set of solutions of this system in $L$. Simple but important
statement
         $$
    X(L)=Hom_K(A,L)
         $$
Indeed let $\tau$ be a homomorphism of $A$ in $L$:
             $$
\tau\colon\quad A=K[T_1,\dots,T_m]\backslash I_{\rm equations}
             $$ 
where we denote by $I_{\rm equations}$ the ideal generated by polynomials
  $F_i$: 
         $$
   I_{\rm equations}=<F_1(T_1,\dots T_m),\dots,F_n(T_1,\dots T_m)>\,.
         $$
Denote by $l_\a$ the value of $\varphi$ on equivalence classes of elements
$[T_i]$:
            $$
  l_\a=\tau([T_a]),\quad {\rm where} [T_a]=
 \{x\in K\colon\quad x-T_a\in I_{\rm equations}\}
            $$
One can see that for arbitrary polynomial $F_i$ in equations
          $$
    F_i(l_1,\dots,l_m)=
            F_i\left(
    \tau\left([T_1]\right),\dots,
   \tau\left([T_1]\right)
           \right)=           
           \tau
            \left(
        F_i\left([T_1]\right),\dots,\left([T_m]\right)
	\right)=
              \tau
            \left(
             \left[
        F_i\left(T_1],\dots,T_m\right)
          \right]
           \right)
           =0,.
          $$
Now prove the converse implication.

Let elements $l_1,\dots,l_m\in L$ of algebra $L$ 
be a solution of equations 
      $$
    F_i(l_1),\dots.l_m)=0\,.
      $$
Consider the homomorphism
$\varphi$ of the ring $K{T_1,\dots,T_m}$ in $L$
  such that $\varphi(T_\a)=l_\a$:
           $$
\varphi\colon\quad \varphi(P(T_1,\dots,T_m))=P(l_1,\dots,l_m)
           $$ 
for an arbitrary polynomial $P\in K[T_1,\dots,T_m]$.

  This homomorphism is well-defined
on factor-algebra $A$ since it vanishes on polynomials 
$F_i$.
  

   $$ $$


   \centerline {$\left|Spec A\right|=1$}
  What can we say about a ring if  it has only one prime ideal.
  If ring is not-empty then
   one can consider maximal ideal which possesses zero ideal.
  Hence there is at least one point-- a maximal ideal $\a$.

  If there is exactlone point, it means that zero ideal is not prime,
i.e. ring possesses at least one nilpotent element.

   Show that maximal ideal possesses all nilpotents.

   Let $\theta\in \a$. The value of $\theta$ on a point $\a$ is equal to zero.
   If  $\theta$ is not nilpotent then there exist a point $x$ such that
  $\theta(x)\not=0$, i.e. there is a prime ideal, which does not possess
   $\theta$, that it is there is another point of $A$. We come to 

  {\bf Proposition} If ring $A$ possesses just one point, then it is
maximum ideal and all its elements are nilpotents.

{\bf Counterexample} Consider for $\Z$ a local ring at the point $(3)$:
              $$
      O_{3}=\{{p\over q}\colon\quad  3\not | q\}
              $$
This ring possesses just one maximum ideal $(3)$, but it has
two points: point $(3)$ and the point $(0)$.


   Let $A$ be a ring with just one point 

\end


