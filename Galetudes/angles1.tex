
       \magnification=1200
   \baselineskip 17 pt

\def\V {{\cal V}}
\def\s {{\sigma}}
\def\Q {{\bf Q}}
\def\D {{\cal D}}
\def\G {{\Gamma}}
\def\C {{\bf C}}
\def\M {{\cal M}}
\def\Z {{\bf Z}}
\def\U  {{\cal U}}
\def\H {{\cal H}}
\def\F {{\cal F}}
\def\grad {{^{\circ}}}
\def\S {{\Sigma}}
\def\s {{\sigma}}


    \centerline {\bf Example of applying Galois theory to elementary problems}

\bigskip

   \centerline  {\bf How to calculate $\sin 6^{\circ}$}

\medskip



 First of all try to find polynomial (with rational coefficients)
 such that $\sin 6^{\circ}$ is its root. Note that $6\cdot 5=30$
 and $\sin 30\grad={1\over 2}$. Hence express $\sin 30\grad$ via $\sin 6\grad$:
                  $$
 \sin 5\varphi=\sin 3\varphi\cos2\varphi+\cos 3\varphi\sin 2\varphi=
          16\sin^5\varphi-20\sin^3\varphi+5\sin\varphi\,.
          \eqno (1)
                  $$
    We come to the polynomial equation
     for $u=\sin 6\grad$:
                   $$
                 16u^5-20u^3+5u=\sin 30\grad={1\over 2}\,.
                   \eqno (2)
                   $$
     We do not hope to solve it in radicals straightforwardly. Try to attack
     it using elementary tools of Galois theory.  It is evident from (1) and (2) that if $u=\sin\varphi$
     is root of (2) then $u^\prime=\sin (\varphi+{2\pi\over 5})$
     is root of this equation too
     ($\sin 5(\varphi+{2\pi\over 5})=$
     $\sin (5\varphi+2\pi)=\sin 5\varphi$). Hence it is easy
     to find all five roots of polynomial (2) using trigonometric
     functions:
                           $$
                           \matrix
                           {
                  u_1=\sin 6\grad\cr\cr
                  u_2=\sin (6\grad+{360\grad\over 5})=\sin 78\grad=\cos 12\grad\,,\cr\cr
             u_3=\sin (6\grad+2\cdot{360\grad\over 5})=\sin 150\grad=-{1\over 2}\,,\cr\cr
             u_4=\sin (6\grad+3\cdot {360\grad\over 5})=\sin 222\grad=-\cos 48\grad\,,\cr\cr
             u_5=\sin (6\grad+4\cdot{360\grad\over 5})=\sin294\grad=-\cos 24\grad\,.\cr
                                 }
                                 \eqno (3)
                                $$


One of the roots of this polynomial is rational number, hence polynomial
$16u^5-20u^3+5u-{1\over 2}$ in (2) is reducible over $\Q$:
it has linear factor $u-{1\over 2}$.

It is more convenient (for calculations) to consider new variable $t=2u$.
We rewrite our polynomial as
                 $$
  {1\over 2}t^5-{5\over 2}t^3+{5\over 2}t-{1\over 2}\,.
                     \eqno (4)
                 $$
This polynomial has root $t=2u=1$. Thus it contains linear factor $t-1$:
             $$
         {1\over 2}t^5-{5\over 2}t^3+{5\over 2}t-{1\over 2}=
         {1\over 2}(t-1)P_4(t)\,.
               $$
It is easy to see that
                $$
      P_4(t)=t^4+t^3-4t^2-4t+1\,.
      \eqno (5)
                $$
We come to four order equation:
                     $$
         P_4(t)=t^4+t^3-4t^2-4t+1=0\,.
                   \eqno (5a)
                     $$
           for $t=2\sin 6\grad$.
 It follows from (3) that its roots
 $t_1,t_2,t_3,t_4$ are
               $$
               <2\sin 6\grad, 2\cos 12\grad,-2\cos 24\grad,-2\cos 48\grad>\,.
               \eqno (6)
               $$
It can be very easy straightforwardly checked that polynomial
$P_4(t)$ is irreducible over $\Q$.

Splitting field of the polynomial $P_4$
(minimal field that contains all the roots of this polynomial)
is $\S(P_4)=\Q(\sin 6\grad, \cos 12\grad,\cos 24\grad,\cos 48\grad)$.

Calculate first degree of extension $[\S(P_4):\Q]$.
Note that $\cos 12\grad=1-2\sin^2 6\grad$,
   $\cos 24\grad=2\cos^2 12\grad-1$, $\cos 48\grad=2\cos^2 24\grad-1$,
   $-\sin 6\grad=2\cos^2 48\grad-1$.

   We see that
 {\bf rational transformation}
                 $$
             t\mapsto 2-t^2
                 \eqno (7)
                  $$
 {\bf transforms roots of $P_4$ to another roots}.

  This transformation defines $\Q$-automorphism $\s$ of the field $\S(P_4)$
  such that:
                   $$
    \s(t_1)=t_2,\quad \s(t_2)=t_3,\quad\s(t_3)=t_4,\quad \s(t_4)=t_1,\quad
                  \eqno (8)
                   $$
  From (7), (8) it is evident that all roots belong to field
  $\Q(t_1)=\Q(\sin 6\grad)$ ($t_3=\s^2(t)=2-(2-t)^2$,
  $t_4=2-(2-(2-t)^2)^2$). Hence
                    $$
              \S(P_4)=\Q(\sin 6\grad)\,.\quad [\Q(\sin 6\grad):\Q]=4\,.
                                  \eqno (9)
                    $$
          We see that splitting field for irreducible polynomial
          $P_4(t)$ is nothing but simple extension $\Q(\sin 6\grad):\Q$.

 We see that Galois group of polynomial (5) that is group of automorphisms
of the field $\Q(\sin 6\grad)=\S(P_4(t))$ contains 4 elements:
                    $$
           G=\G(\S:\Q)=\{1,\s,\s^2,\s^3\}\,.
                            \eqno (10)
                    $$
  It is abelian cyclic group: $\sigma^4=1$.

 This group contains only one proper subgroup $H$ ($H\not=1$, $H\not=G$):
                $$
             H=\{1,\sigma^2\},\quad |H|=2\,.
                 $$

To subgroup $H$ corresponds intermediate field $M=H^\dagger$:
$M=H^\dagger$ is maximal subfield in $Q(\sin 6\grad)$ such that its elements
 are invariant under transformations from $H$, i.e. under transformation $\s^2$:
               $$
   \Q\subset M\subset \Q(\sin 6\grad),\quad
   M=\{a\in \Q(\sin 6\grad)\,\,{\rm such\,\,that}\,\,\s^2(a)=a \}\,,
               $$
               $$
 [\Q(\sin 6\grad):M]=2,\quad [M:\Q]=2\,.
                     \eqno (11)
               $$
   We see that intermediate extensions are quadratic (degree is equal to $2$).
From formula (11) it follows that every element of field $\Q(\sin 6\grad)$
  and in particularly $\sin 6\grad$ belongs to the intermediate field $M$
  or it is a root of {\bf quadratic polynomial with coefficients in} $M$.

   (If $[L:K]=2$ then for arbitrary $a\in L$
   elements $1,a, a^2$ are linear dependent over field $K$,
   hence there exist coefficients $p,q,r,\in K$ such that
   not all are equal to zero and relation $p+qa+ra^2=0$ is obeyed).

   Before studying the content of field $M$ we deliver
    this quadratic polynomial.
       Consider elements
                          $$
                          \matrix
                             {
                         \Q(\sin 6) \ni \alpha=t_1+t_3=t_1+\sigma^2 t_1\cr\cr
                         \Q(\sin 6) \ni \beta=t_1\cdot t_3=t_1\cdot\sigma^2 t_1\cr
                             }\,.
                             \eqno (12)
                           $$
     It is evident that these elements belong to intermediate field
     $M$, because they do not change under the action of automorphism
     $\s^2$.
                From (6) and (12) we see that numbers
                $t_1=2\sin 6\grad$ and $t_3=-2\cos 24\grad$
   are roots of the following quadratic polynomial
                    $$
                P_2(t)=t^2-\alpha t+\beta
                \eqno (13)
                $$
          with coefficients $\alpha$ and $\beta$ in the field $M$.

  Equation $P_2(t)=0$ can be solved elementary in radicals.
  It remains to calcualte $\alpha$ and $\beta$ which belong
   to field $M$ which is quadratic extension of $\Q$ ($[M:\Q]=2$). Thus
   $\alpha$ and $\beta$ are roots of quadratic polynomial
   with rational coefficients. Find these polynomials.

   It follows from (6), (12) and elementary stuff of trigonometric formulae that

   $M\ni \alpha=t_1+t_3=2\sin 6\grad-2\cos 24\grad=$
                       $$
              =     2\sin 6\grad-2\sin 66\grad=
                   4\sin {6-66\over 2}\cos {6+66\over 2}=-2\cos 36=4\sin^2 18\grad-2
                   \eqno (14 a)
                       $$
             and

$M\ni\beta=t_1\cdot t_3=$
                       $$
                       2\sin 6\grad\cdot (-2\cos 24)
              = -4 (\sin 6\grad\cos 24\grad)=
                 -2 (\sin 30\grad-\sin 18\grad)=
                   2\sin 18 \grad-1\,.
                   \eqno (14 b)
                      $$
 We see from these relations that
                    $$
           M=\Q(\sin 18\grad)\,,
                     $$
  and $\sin 18\grad$ is a root of quadratic polynomila with rational coefficients.


 Having this information it is easy to find this polynomial for $\sin 18\grad$.
  One can see that $\sin 18 \grad$ is a root of polynomial $4t^2+2t-1$:
                            $$
                     4\sin^2 18\grad+2\sin 18\grad-1=0\,.
                     \eqno (15)
                             $$
           Hence
                          $$
                       \sin 18\grad={\sqrt 5-1\over 4}\,.
                           $$
{\bf Remark 1} There are many ways to disclose quadratic relation (15).
  Not the most beautiful one but right one is the following:

  $0=\cos 36\grad-\sin 54\grad=(1-2\sin^2 18)-(3\sin 18\grad-4\sin ^3 18\grad)=$
                                $$
             4\sin^3 18\grad-2\sin^2 18\grad-2\sin 18\grad+1=
             (\sin 18\grad-1)(4\sin^2 18\grad+2\sin \grad-1)\,.
                                $$


  {\bf Remark 2}. The number $\tau=2\sin 18\grad={\sqrt 5-1\over 2}$
  is so called "golden ratio". It has many wonderful properties...
  One of the ways to obtain relation (15) straightforwardly
  as relation for golden ratio is to consider triangle
  with angles $(72\grad, 72\grad, 36\grad)$.



We calculated $\sin 18\grad$, now we calculate $\alpha,\beta\in \Q(\sin\grad 18)$
by (14)
and solve quadratic equation (13).

                 $$
                 \alpha=4\sin^2 18\grad-2=-2\sin 18\grad-1=-{1+\sqrt 5\over 2}\,.
                         $$
                             $$
\beta=2\sin 18\grad-1={\sqrt 5-3\over 2}\eqno (16)\,.
                          $$.

So  from (13) and (16) we see that
                $t_1=2\sin 6\grad$ and $t_3=-2\cos 24\grad$,
   are roots of quadratic equation
                    $$
                t^2+ {1+\sqrt 5\over 2}t-{3-\sqrt 5\over 2}=0\,.
                \eqno (17)
                $$
                $$
              t_{1,2}=
             {\pm\sqrt{30-6\sqrt{5}}-\sqrt{5}-1\over 4}
                $$
Positive root of this equation is equal just to $t_1=2\sin 6\grad$:
  Finally we obtain that
                  $$
            \sin 6\grad=
             {\sqrt{30-6\sqrt{5}}-\sqrt{5}-1\over 8}
             \eqno (18)
                   $$
We calcualted $\sin 6\grad$!.

\bigskip

   \centerline {\bf Discussions}

\medskip

We see from (18) that $\sin 6\grad$
(so and $\cos 6\grad$) is expressed through rational numbers
 with arithmetic operations $(+,-,\times,:)$ and operation
 of taking square root. It means that we can construct
 by ruler and compass the angle $6\grad$, i.e.
 we can divide the circle by ruler and compass on 60 equal arcs.


  Now consider the following two questions

  {\bf Question 1} If $\varphi$ is an angle
  can we construct this angle with compass and ruler
 and correspondingly $\sin\varphi$ and $\cos \varphi$
  can be expressed through rational numbers
 with arithmetic operations $(+,-,\times,:)$ and operation
 of taking square root?

 {\bf Question 2} If $N$ is positive integer
 can we divide the circle on $N$ equal arcs with compass and ruler?

In the next sheet I will discuss in details answers on these questions,
 here only two words about it:

In fact the main reason why for $\varphi=6\grad$  the answer on the first question
is positive is the following:
The Galois group for irreducible polynomial $P_4(t)$ in (5)
(Galois group of the splitting field of this polynomial)
corresponding to $\sin 6\grad$ has $4=2^2$ elements.


   One can prove that the answer on the first question is positive
    if the Galois group of corresponding
   irreducible polynomial contains $2^n$ elements.
 This follows from the following lemma:

{\bf Lemma} If $G$ is arbitrary finite group such that $G$
contains $2^n$ elements then there is a subsequence
          of subgroups:
                 $$
       G_0\subset G_1\subset G_2\subset\dots\subset G_{n-1}\subset G_n=G
                      \eqno (2.1)
                 $$
    such that every subgroup $G_k$ in (2.1)
    contains $2^k$ elements and $G_k$ is normal subgroup of
    $G_{k+1}$.


    Indeed according to Lemma and main theorem  of Galois theory if $|G|=2^n$ then
    we can find the sequence of intermediate fields corresponding to
    (2.1)
                      $$
        \Q=M_0\subset M_1\subset \dots\subset M_n
                      $$
          such that $M_k\subset M_{k+1}$ are quadratic normal extension.
     Hence every root of polynomial is a root of quadratic polynomial
          with coefficients in $M_{n-1}$, these coefficients
          are roots of quadratic polynomial with coefficients in $M_{n-2}$
          and so on.

    One can show that if $\varphi=k^\grad$ where $k$ is integer then
       answer on first question is positive iff $k$ is divisible on $3$.

 The second question is strictly related with first one.
  This question  was posed already
   by ancient Greeks:

   It is evident (we know it from school)
   that answer on the first question is positive for $N=2,3,4,6,8$ and
  for $N=2^k$. Little bit more difficult one can obtain that answer is positive for
  $N=5$ (This was known already to ancient Greeks.).

  Gauss proved that for $N=17$ answer is positive too. This was one of the
  most impresssive results of 19 century mathematics.

  Now we describe the general answer.

    We call prime number $p$ Messner prime number if $p=2^k+1$.

    {\bf Exercise} Prove that if $p=2^k+1$ is Messner prime number then
     $k=2^p$.

     For example $3,5,17,257$ are Messner prime numbers.

     {\bf Statement} Circle can be divided on $N$ equal arcs
           by ruler and compass if and only if the number
           $N$ obeys to the following condition:



           {\bf all prime factors that are contained in number $N$
           if $p$ is a prime number such that $N$ is divisible on $p$
           then $p=2$ or $p$ is Messner prime number and $N$ is not divisible
           on $p^2$}


It turns out that in this and only in this case the corresponding
Galois group contains $2^n$ elements.



We see that for $N=7,9,11,13$ answer is negative.


Finally the last question-jock :

{\bf Question 3} Why 50 pence coin has 7 edges?
Answer: Because  $7$ is the smallest number such that
answer on the question 2 is negative.


\bye
