

\magnification=1200 \baselineskip=14pt


\def\p {\partial}
\def \D {\Delta_{d{\bf v}}}
\def \Ds  {\Delta^{\#}}
\def\t{\tilde}
\def\s {\sigma}
\def\vare {\varepsilon}
\def\L {\Lambda}
\def\Darboux {$z^A=$  $x^1,\dots,x^n$, $\theta_1,\dots,\theta_n$}
\def\a{\alpha}
\def\O{\Omega}
\def\d{\delta}
\def\dv  {{d{\bf{v}}}}
\def\A {{\cal A}}
\def\R {I\!R}
\def\t {\tilde}
\def\Q{\bf Q}
\def\C{\bf C}
\def\CP{\bf CP}
\def \deg   {{\rm deg\,}}
             \centerline
             {\bf Nice fractions and Platonic bodies}

\medskip

\centerline {\it {(reflections on the Klein book and Lueroth Theorem...)}}
\medskip

This text is the continuation of the text on Lueroth theorem.

We consider fractions---rational functions on indeterminate $t$ or if you prefer
elements of simple transcendent extensions.


 Consider an arbitrary irreducible fraction $R(x)=f(x)/g(x)$. Suppose its degree
 is equal $N$. Consider the equation
                    $$
                    {f(x)\over g(x)}={f(t)\over g(t)}
                    \eqno (1)
                    $$
     If this equation has exactly $N$ solutions which are fractions on $t$
        (i.e. elements in $\C(t)$ then we call this fraction {\it nice fraction}


        Suppose fractions $\{\psi_0(t),\psi(t_1),\dots,\psi_N(t)\}$
        ($\psi_0(t)=t$)
        are solutions of equation (1) for the nice fractions $F=f/g$, i.e.
     $f(\psi(t))/g(\psi(t))\equiv f(t)/g(t)$.
     Their degrees coincide and degree of $\psi$ must be equal to $1$
     (Is it vraie?) The $N$ substitutions $\{\psi_i(t)\}$
     form invariance group of the fraction $f(x)/g(x)$.

       {\it Every nice fraction $R$ of degree $N$ defines the finite subgroup
       $H_R$ of fractional-linear transformations containing $N$ elements}


       The inverse is true too:

       Let $H$ be the finite subgroup of the group $PGL(2,C)$ of fractional linear
       transformations.  Consider the polynomial
                          $$
                   P_H(x)=\prod_{g_i\in H}\left(x-t^{g_i}\right)=x^N+
                   b_{N-1}(t)x^{N_1}+\dots+b_1(t)x+b_0
                               $$
where $t^{g}$ is the action of fraction-linear transformation on $t$. This polynomial
has $N$ roots $\psi_i(t)=t^{g_i}$
 Here $N$ is number of elements in $H$
 and  $b_k(t)$ are fractions on $t$. At least one of these fractions is not trivial
  (because polynomial has root $x=t$).
 Take one of the fractions
  $b_k=f(t)/g(t)$.  This coefficient is some symmetric polynomial
  on the roots $\{\psi_i(t)\}$ hence {\it this is a nice fraction
  which has just the same roots that polynomial $P_H$}.
  Of course we could consider another coefficient, come to another
  nice fraction with the same roots. It is just related with Lueroth theorem:
  all coefficients have to be expressed via one of non-trivial coefficient.
  Moreover if $R$ is a nice fraction then ${aR+b\over cR+d}$
is a nice fraction which defines the same group.


If  $F=f(z)/g(z)$ is the nice fraction then the fractions
 $F^\prime(z)=aF(z)+b/cF(z)+d$ ($a,b,c,d\in \C$)
and the fraction $\tilde F(z)=F\left(az+b/cz+d\right)$
 are nice fraction too. The nice fractions $F, F^\prime=aF+b/cF+d$
 have the same ramification points. They define the same invariance group.

The nice fractions  $F(z),\tilde F(z)=F\left(az+b/cz+d\right)$
define conjugate invariance subgroups.


  {\bf Example}  Consider the fractions $x^2$ and $x+1\over x$.
  They both are nice fractions
  with different ramification points (see the previous text).
  The transformation $x\to {x+1\over x+1}$ transforms ramification points
  $[0,\infty]$ of $x^2$ to $[-1,1]$. $x^2\to ({x+1/ x+1})^2$.
   One can see that this function is just fractional-linear transformation
  of $x+1/x$:
                      $$
         {\left(x+{1\over x}\right)+2\over \left(x+{1\over x}\right)-2}=
              \left({x+1\over x-1}\right)
                     $$

More tricky to show the relation between nice fractions
    $x^3$ and $x^3-3x-1\over x(x+1)$ considered in the previous text.

But e.g. fraction $x^3/over x^3+(1-x)^3$ is the nice fraction produced from $x^3$ by
changing ramif. points ($[0,\infty]\to [0,1]$)


$$ $$


Now we go to very fundamental result belonging to Klein:

To every Platonic body corresponds
 finite subgroup of rational transformation
 and a class of nice fraction. It is obvious. One can prove that
 it is all!

    This fact is dual to the kindergarten  result:
  there is a finite number of Platonic bodies.
  There are five Platonic bodies: tetrahedra, hexahedra, octahedra, dodecahedral and
  icosahedra. Plus we consider also degenerated bodies: diedres (They have to sides which
   are regular $n$-gones.)
   The point is that not only the statement about the finitness of Platonic bodies
   is analogous to the statement about the finite number of groups
   (up to conjugancy)


  Consider arbitrary Platonic body. Let every vertex is incident to $r$ edges
  and every side is $n$-gon. If $E$  is number of edges then
  $\Gamma$ (number of sides) is equal to $En/2$ and $V$(number of vertices)
  is equal to $Er/2$, According to Euler identity
  $$V-E+\Gamma=2 \eqno (E1)$$. Hence
                   $$
             {1\over r}+{1\over n}={1\over 2}+{1\over E}
             \eqno (E2)
                    $$
 It is very easy to solve this diophantine equation.
 If $r\geq 6$ then $1/r+1/n\leq 1/2$ because $n\geq 3$.
 Hence $r\leq 5$.
 Consider cases $r=2,3,4,5$



 1. $r=5$

    If $n\geq 4$ then $1/r+1/n\leq 1/5+1/4<1/2$. Hence only $n=3$ is allowed.
    If $n=3$ then $E=30$ we come to {\it icosahedra}:
                     $$
          V=12, E=30, \Gamma=10
                 $$
\medskip

 2. $r=4$. If $n\geq 4$ then $1/r+1/n\leq 1/4+1/4\leq 1/2$. Hence again
 only $n=3$ is allowed. If $n=3$ then $E=12$. We come to octahedra:
                             $$
               V=6, E=12, \Gamma=8
                               $$

3.  $r=3$. If $n\geq 6$ then $1/r+1/n\leq 1/3+1/6\leq 1/2$. Hence
 only $n=3,4,5$ are allowed.

 a) If $n=5$ we come to {\it dodecahedral}
(dual to icosahedra): :
                     $$
          V=10, E=30, \Gamma=12
                 $$

b) if $n=4$  we come to {\it hexahedral } (dual to octahedra): :
                     $$
          V=8, E=12, \Gamma=6
                 $$
c)  if $n=3$  we come to {\it tetrahedra}: :
                     $$
          V=4, E=6, \Gamma=4
                 $$

4. $r=2$. It is degenerate case; $E=n$  We come to diedre.

 Solving diophantine equation is key step to prove that there are
 few Platonic bodies.

  $$ $$




Now we will show that the fact that there are few finite groups in group of rational
transformations is reduced to simple diophantine equation too.
(Of course this can be treated pure geometrically: Instead the group of
fractional transformations we can consider the group $S0(3)$
and to prove that this group has finite number of finite
subgroups (so called cristollagr.....))

Let $H$ be finite subgroup. Let $f(x)/g(x)$ be its corresponding fraction and
characteristic equation:
                $$
        {f(x)\over g(x)}={
              f(t)\over g(t)}
              $$


 Here I
repeat up to some modifications Klein considerations:


Let $F(z)$ be a rational function. It maps $\CP$
on $\CP$. We say that the point $z=z_0$ has an index $n=\mu(z_0| F)$ if $z_0$ is the
zero of the order $n$ for the function $F(z)-F(z_0)$. Evidently the index is
 invariant under rational fraction-linear transformations of $z$ and $F(z)$. (In the
case $z_0=\infty$, or $F(z)=\infty$ index can be easily calculated with the help of
transformation $z\to 1/z$). If the point $z_0$ has the index $n$ then the derivative
has zero of the order $n-1$ at the point $z_0$.  On the other hand one can show that
if a degree
\footnote{$^*$}{As always we define $\deg F=\max{f,g}$ if $F$ is a ration of two
polynomials $f,g$ and these polynomials are coprime}
  $\deg F$ of the rational function  $F$ is equal to $N$ then
  the degree of the derivative is equal
to $2N-2$. (It is easy to see for the special gauging when
for rational
function $F(z)$ all points with index $\geq 2$ are regular points of
$\CP$ and its image is regular too. E.g.
one can always by suitable fraction-linear transformation) Hence we come to the following relation:
                        $$
      \hbox {for arbitrary rational function $F(z)$\quad}$$
      $$    \sum_{z\in \CP} \left[\mu(z|F(z))-1\right]=
          \sum_{z\colon\,\, F'(z)=0} \left[\mu(z|F'(z))\right]=2\deg F-2
                        $$
The relation
                      $$
                      \sum_{z\in \CP} \left[\mu(z|F(z))-1\right]=2\deg F-2
                      \eqno (E^*1)
                      $$
can be considered as an analogue of Euler identity (E1)
{\it Proof}  Suppose that for the rational function $F(z)$
 the points $z=0,\infty$ have an order $1$. (You come by this condition
 by suitable fraction-linear transformation which does not affect the relation
 ($E^*1$)). In this case let $F=f(z)/g(z)$ where polynomials $f,g$
 and $g(z)$ has not degenerate zeros. All zeros of
 of the function $F'$ are the same that for the function $fg'-gf'$.
 This function has the degree $N-2$.

 Making suitable fraction-linear transformation of $z$ and of $F$
 we can restrict the function $F$ by the condition



The relation ($E^81$) has the following geometrical
meaning:If $F(z)\colon\,\,\CP\to \CP$
is $N$-sheaves covering  $(N=\deg F)$ with ramification points  ${z_1,\dots,z_k}$
such that ramification index at the point $z_i$ is equal to $\nu_i$ ($\nu_i\geq 2$)
then the relation $\sum_{k}\left(\nu_k-1\right)=2N-2$ holds.
This is the special case of the so called Riemann Hourvitz formula: Let $F$
  be a complex map from complex surface $\bar R$ of the genre $\bar g$
  onto complex surface $R$ of the genre $g$.
  If $F$ is $N$-sheaves covering  with ramification points  ${z_1,\dots,z_k}$
such that ramification index at the point $z_i$ is equal to $\nu_i$ ($\nu_i\geq 2$)
then the relation $\sum_{k}\left(\nu_k-1\right)=2N-2+g+\dots$ holds.





Now {\it revenons a nos moutons}



Let $H$ be finite subgroup of the group of rational transformations of $\CP$.

Consider nice fraction $F_H(z)$ corresponding to this group. Let $z_0$ be a
 point of an index $n\geq 2$, then the point $w=F(z_0)$ is ramification point.
 To the point $z_0$ corresponds stationary subgroup of the index $n$.
 Hence there are $N/n$ points with conjugated stationary subgroups.
 ($N$ is an order of group $H$). Note that the degree of nice fraction $F_H$
 is equal to $N$.
We see that for nice fraction there are numbers $\{n_1,n_2,\dots\}$ such that there are
for every $n_k$ there are $N/n_k$ points of the index $n_k\geq 2$. Then using previous
identity we come to the relation:
                   $$
 \sum_k{N\over n_k}\left(n_k-1\right)=2N-2
                     $$
         (Compare with relation (E2)).  Dividing by $N$ we come to
                       $$
 \sum_k\left(1-{1\over n_k}\right)=2-{2\over N}
 \eqno (E^*2)
                       $$
This is remarkable identity from Klein book which can be considered as analogue (E2)
for Kleinian subgroups.

  Solve this diophantine equation.
  Firs of all note that it follows immediately from
   ($E^*2$) that there could be only two or three ramification points!
   Indeed  right hand side is greater than one and
   smaller than 2. Hence there can be  only two possibilities.

   I. two ramification points:
                   $$
      {1\over n_1}+{1\over n_2}={2\over N}
                   $$
   II. Three ramification points:
             $$
        {1\over n_1}+{1\over n_2}+{1\over n_3}=1+{2\over N}
             $$
One can see that the first equation has only solution $n_1=n_2=N$. For every $N$ nice
fraction has two ramification points of degree $N$. Gauge this points to be
$[0,\infty]$. We come to the function $z^N$ to diedre.




 \bye
