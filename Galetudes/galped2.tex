



\magnification=1200 \baselineskip=14pt


\def\p {\partial}
\def \D {\Delta_{d{\bf v}}}
\def \Ds  {\Delta^{\#}}
\def\t{\tilde}
\def\s {\sigma}
\def\vare {\varepsilon}
\def\L {\Lambda}
\def\Darboux {$z^A=$  $x^1,\dots,x^n$, $\theta_1,\dots,\theta_n$}
\def\a{\alpha}
\def\O{\Omega}
\def\d{\delta}
\def\dv  {{d{\bf{v}}}}
\def\A {{\cal A}}
\def\R {I\!R}
\def\t {\tilde}
\def\l {\lambda}
\def\e {{\bf e}}
\def\x {{\bf x}}
\def\y {{\bf y}}

             \centerline {Cubic and quadric equations; Galois theory for pedestrians.}

{\it This text is written on the base of the book of A. Khovansky
"Galois Theory"}

 We suppose that Galois group exists. This is not trivial.... 

   We suppose that the main field is a field of characteristic
   $0$ which possesses all roots of unity.

In this lecture the expression
``can be expressed'' means.....

  Begin with Vieta Theorem. It tells that 




 Consider polynomial
       $$
P(t)=x^n+\dots
       $$
Then roots of this polynomial obey relations
  
Let $\Sigma(x_1,\dots,x_n)$ be polynomial on variables $x_1,\dots,x_n$.
This polynomial can be expressed via co


 {\bf Proposition 1} Let $A$ be an algebra over field $K$,
and let $H$ be a finite abelian group acting on algebra $A$.
   Denote by $A_H$ elements of algebra $A$ which are invariant under
the action of group $H$.

   Then an arbitrary element of algebra $A$ can be expressed via
elements of invariant subalgebra $A_H$ by taking roots.


   Example.

1) Quadratic equation. 
Algebra $A$ is generated by roots of polynomial
  Abelian group $H=(1,\s)$. $\s(x_1,x_2)=(x_2,x_1)$.

   Consider  $u=x_1+x_2$, $t=(x_1-x_2)^2$. These elements are
 $H$-invariant. $x_1=u+\sqrt t\over 2$ 
         


{\bf Theorem } 
  Let $G$ be a group and $H$ be its invariant subgroup, such that
group $H$ and group $G/H$ are abelian groups.
            $$
    1\triangle H\triangle G
            $$ 

Then every element of algebra $A$ can be expressed via $G$-invariants of
algebra $A$ by taking operations of roots.


   Indeed group $H$ acts on algebra $A$,
and abelian group $G/H$ acts on $H$-invariants of algebra
$G$. Hence  by Proposition 1,
an arbitrary element $x\in A$ can be expressed via
$H$-invariants of algebra $A$. On the other hand since
abelian group $G/H$ acts on $H$-invariants,
every $H$-invariant in its turn can be expressed
via $G/H$ invariants, i.e. $G$-invariants of algebra $A$.


  
 



Denote by $A_G$ the set of elements of the algebra $A$
which are invariant

  Then every element $x$ of the algebra $A$
can be represented as a sum of elements $w_1,\dots,w_k$
such that every $w$

Then there exists 
 basis 





We will use the following lemma:

  {\sl Let $L$ be linear operator on finite-dimensional vector
space $V$ over field which possesses roots of unity. Then
  $L$ is diagonalisable}.
Proof see in Appendix 1.





\centerline {Calculations for cubic equation}

  Consider the cubic equation
    $$
x^3+ax^2+px+q=0
    $$
where coefficients are complex numbers.
We will put $a=0$ almost for ll calculations,
but sometimes it will be useful to consider $a$
as an arbitrary complex number.

We have
       $$
     \matrix
        {
x_1+x_2+x_3=-a\cr
x_1x_2+x_2x_3+x_3x_1=p\cr
 x_1x_2x_3=-q\cr
        }
       $$
Take  root $x_1$.
Consider linear space spanned by 
the orbit of the element $x_1$ under the action of
 the cyclic group  $C_3$: 
    $$
C_3=\{1,s,s^2\},\quad  s=(123), s^2=(132)\,.
    $$
We have 
$sx_1=x_2$, $sx_2=x_3$
The linear space $V_1$ is space
spanned by vectors $\{x_1,sx_1,s^2x_1\}$.
Its dimension is $\leq 3$.
        
 The group $C_3$ is commutatie group. Find a basis
$(w_1,w_2,w_3)$ of eigenvectors [`states'
which are simultaneously measurable]
       $$
  s w_1=\d_1 w_1\, \quad
  s w_2=\d_2 w_2\, \quad
  s w_1=\d_1 w_3,.
       $$
One can see that this is so called Lagrange variables
      $$
     \matrix
         {
w_0=x_1+x_2+x_3\,, \quad &s w_0=w_0\cr
w_1=x_1+\vare^2 x_2+\vare x_3\,,\quad & s w_1=\vare w_1\cr 
w_2=x_1+\vare x_2+\vare ^2x_3\,,\quad & s w_2=\vare^2 w_2\cr
         }
      $$ 
One can see that
       $$
   x={w_0+w_1+w_3\over 3}\,.
       $$
We calculate $w_0,w_1,w_2$. $w_0=a$ is a coefficient.

Later during detailed calculations we put $w_0=a=0$.


Consider numbers
      $$
    z=w_1^3+w_2^3, t=(w_1^3-w_2^3)^2
      $$
 
These numbers are invariants of all permutations, i.e. they
are polybnomials of coefficients. Caclulate these numbers
(assuming $w_0=x_1+x_2+x_3=0$):
        
  We will use the following identity:
       $$
   a^3+b^3+c^3-3abc=(a+b+c)(a^2+b^2+c^2-ab-ac-bc)\,.
         \eqno (8)
       $$
  $$
z=(x_1+\vare^2 x_2+\vare x_3)^3+(x_1+\vare x_2+\vare^2 x_3)^3=
 2(x_1^3+x_2^3+x_3^3)+12x_1x_2x_3+
          $$
          $$
      3x_1x_2\vare (x_1+x_2)(1+\vare)+3x_2x_3\vare (x_2+x_3)(1+\vare)+
      3x_3x_1\vare (x_3+x_1)(1+\vare)=
          $$
(with use of identity (*))
           $$
  6x_1x_2x_3+12x_1x_2x_3-9\vare(1+\vare)x_1x_2x_3=27x_1x_2x_3=-27q\,.
           $$
It is little bit more long to calculate  $t$
        $$
t=\left[(x_1+\vare^2 x_2+\vare x_3)^3-(x_1+\vare x_2+\vare^2 x_3)^3\right]^2=
          $$
          $$
         \left[
      3x_1x_2\vare (x_1-x_2)(\vare-1)+3x_2x_3\vare (x_2-x_3)(\vare-1)+
      3x_3x_1\vare (x_3-x_1)(\vare-1)\right]^2=
        $$
       $$
     \left(3\vare(\vare-1)\right)^2
    \left[x_1x_2(x_1-x_2)+x_2x_3(x_2-x_3)+x_3x_1(x_3-x_1)
     \right]^2=
       $$
     $$
-27\left[x_1^2x_2^2(x_1-x_2)^2+x_2^2x_3^2(x_2-x_3)^2+x^2_3x^2_1
   (x_3-x_1)^2\right]-
      $$
         $$
       -27
       \left[ 
      2x_1^2x_2x_3(x_1-x_2)(x_3-x_1)+
       2x_2^2x_3x_1(x_1-x_2)(x_2-x_3)+
       2x_3^2x_1x_3(x_2-x_3)(x_3-x_1)\right]=
     $$
        $$
-27\left[x^2_1x^2_2(x_1+x_2)^2+x^2_2x^2_3(x_2+x_3)^2+
   x^2_3x^2_1(x_3+x_1)^2-4(x_1^3x_2^3+x_1^3x_3^3+x_2^3x_3^3)\right]-
      $$
         $$
       -54x_1x_2x_3
       \left[ 
      x_1(-x_1^2-x_2x_3+x_1(x_2+x_3))+
      x_2(-x_2^2-x_1x_3+x_2(x_1+x_3))+
      x_3(-x_3^2-x_1x_2+x_3(x_1+x_2)+
         \right]
     $$
Now again using the identity (*) we come to
  $$
-27\left[3 x^2_1x^2_2x_3^2-4\left(
      \underbrace{x_1^3x_2^3+x_1^3x_3^3+x_2^3x_3^3}_{K}\right)\right]
       -54x_1x_2x_3
       \left[ 
      -2(x_1^3+x_2^3+x_3^3)-3x_1x_2x_3
         \right]=
     $$
      $$
-72x^2_1x^2_2x_3^2 
       -54x_1x_2x_3
       \left[ 
      -6x_1x_2x_3-3x_1x_2x_3\right]+108K=
    27\left[27x_1^2x_2^2x_3^3+4(x_1x_2+x_2x_3+x_3x_1)^3\right]=
            $$
             $$      
 27(27q^2+4p^3)\,.
       $$
since due to identity (*)
       $$
K=x_1^3x_2^3+x_1^3x_3^3+x_2^3x_3^3=
   3x^2_1x^2_2x^2_3+
(x_1x_2+x_1x_3+x_2x_3)\left[
   (x_1x_2+x_1x_3+x_2x_3)^2-
     3x_1x_2x_3(x_1+x_2+x_3)
      \right]
       $$
       $$
=   3x^2_1x^2_2x^2_3+
(x_1x_2+x_1x_3+x_2x_3)^3\,.
       $$
Finally we have
        $$
       \cases
        {
z=w_1^3+w_2^3=-27q\cr
t=(w_1^3-w_2^3)^2=27^2\left(q^2+{4p^3\over 27}\right)\cr
      }\,,
  {\rm i.e.}\quad
     w_1^3, w_2^3=27\left(-{q\over 2}+\sqrt{{q^2\over 4}+{p^3\over 27}}\right)
        $$
       $$
x_1={1\over 3}{w_0+w_1+w_2\over 3}=
 {1\over 3}\left(\root 3\of {w_1}+\root 3\of {w_2}\right)=
  \root 3\of {-{q\over 2}+\sqrt{{q^2\over 4}+{p^3\over 27}}}
  +\root 3\of {-{q\over 2}-\sqrt{{q^2\over 4}+{p^3\over 27}}}
        \eqno (\dagger)
       $$
{\bf Remark} A question remains: what bracnh of cubic root 
has be taken?

Answer: Consider $R=w_1w_2$. This magnitude is invariant
with respect to all permutations since
                    $$
    sw_1=\vare w+1. sw_2=\vare^2 w_2\,\quad {\rm and}\,\,
      \sigma_{12}w_1=w_2
                    $$
Hence it is a symmetric polynomial on roots, i.e. polynomial on coefficients.
   Calculate the answer:
          $$
w_1w_2=(x_1+\vare^2 x_2+\vare x_3)(x_1+\vare x_2+\vare^2 x_3)=
  (x_1^2+x_2^2+x_3^2)+(x_1x_2+x_1x_3+x_2x_3)(\vare+\vare^2)=
            $$
        $$
   (x_1^2+x_2^2+x_3^2)-(x_1x_2+x_1x_3+x_2x_3)=
   (x_1+x_2+x_3)^2-3(x_1x_2+x_1x_3+x_2x_3)^2=a^2-3p^2\,.
     \eqno{\dagger}
       $$
Chnaging $w_1\mapsto\vare w_1$, $w_2\mapsto \vare^2 w_2$
is Galois symmetry.
This condition fixes the branches. E.g. in the case $a=0$
we have to choose the branches such that 
        $$
w_1w_2=
         \root 3 \of
{27\left(-{q\over 2}+\sqrt{{q^2\over 4}+{p^3\over 27}}\right)}
         \root 3 \of
{27\left(-{q\over 2}-\sqrt{{q^2\over 4}+{p^3\over 27}}\right)}=-3p^2\,.
        $$
In particular if $\left({q^2\over 4}+{p^3\over 27}\right)>0$,
then one consider the branch that takes real values on real numbers,
then $x_1$ in equation {**} becomes real root, and
           $$
x_2=
  \vare\root 3\of {-{q\over 2}+\sqrt{{q^2\over 4}+{p^3\over 27}}}
  +\vare^2\root 3\of {-{q\over 2}-\sqrt{{q^2\over 4}+{p^3\over 27}}}
\,,\quad
x_3=
  \vare^2\root 3\of {-{q\over 2}+\sqrt{{q^2\over 4}+{p^3\over 27}}}
  +\vare\root 3\of {-{q\over 2}-\sqrt{{q^2\over 4}+{p^3\over 27}}}
        \eqno (\dagger\dagger)
           $$
re two complex roots.  E.g. for equation
   $x^2+3x-18$ we have three roots:
             $$
\cases
        {
x_1=
  \root 3\of {-9+\sqrt{80}}
  +\root 3\of {9-\sqrt{80}}=3\cr
x_2=
  \vare\root 3\of {9+\sqrt{80}}
  +\vare^2\root 3\of {9-\sqrt{80}}\cr
x_3=
  \vare\root 3\of {9+\sqrt{80}}
  +\vare^2\root 3\of {9-\sqrt{80}}\cr
}\,\qquad
  \left(\vare=-{1\over 2}+i{\sqrt 3\over 2}\right)\,.
             $$ 
To see that problem of choosing branch can be non-trivial
consider the following `non-clever solution' of the cubic equation:

 \centerline {Non-clever solution}
Solve again the equation
   $x^3+px+q=0$ with
       $$
     \matrix
        {
x_1+x_2+x_3=0\cr
x_1x_2+x_2x_3+x_3x_1=p\cr
 x_1x_2x_3=-q\cr
        }
       $$

but now we introduce Lagrange variables
in not clever way:
       $$
     \matrix
        {
w_0=x_1+x_2+x_3\cr
w_I=x_1+\vare x_2+\vare^2 x_3\cr
w_{II}=x_2+\vare x_1+\vare^2 x_3\cr
        }\qquad (\vare=e^{2\pi i\over 3})\,.
       $$
(Compare how we have chosen $w_1,w_2$ before)

$w_0=0$. 

Still we calculate in the same way:

  $$
     \matrix
        {
w_0^3=0\cr
u=w_I^3=x_1^3+x_2^3+x_3^3+3\vare(x_1^2x_2+x_2^2x_3+x_3^2x_1)+
              3\vare^2(x_1x_2^2+x_2x_3^2+x_3 x_1^2)+6x_1x_2x_3\cr
v=w_{II}=x_1^3+x_2^3+x_3^3+3\vare^2(x_1^2x_2+x_2^2x_3+x_3^2x_1)+
              3\vare(x_1x_2^2+x_2x_3^2+x_3 x_1^2)+6x_1x_2x_3\cr
       }
       $$
Hence
      $$
u+v=2(x_1^3+x_2^3+x_3^3)+3\vare(1+\vare)(x_1^2x_2+\dots)+12x_1x_2x_3
      $$
$\dots$ means all permutations.Since
   $$
a^3+b^3+c^3-3abc=(a+b+c)(a^2+b^2+c^2-ab-ac-bc)\,,
   \eqno (*)
   $$
$w_0=0$ and $\vare(1+\vare)=-1$, hence 
        $$
u+v=6x_1x_2x_3
-3\left(
  x_1x_2(-x_3)+x_1x_3(-x_2)+x_2x_3(-x_1)
  \right)+12x_1x_2x_3=27x_1x_2x_3=-27q\,,
        $$
       $$
u-v=3(\vare-\vare^2)(x_1^2x_2+x_2^2x_3+x_3^2x_2)+
    3(\vare^2-\vare)(x_1x_2^2+x_2x_3^2+x_3x_2^2)=
       $$
       $$
   3(\vare-\vare)^2
             \left(
      x_1x_2(x_1-x_2)+x_2x_3(x_2-x_3)+x_3x_1(x_3-x_1)\,.
           \right)
       $$
and
     $$
(u-v)^2=9(\vare-\vare^2)^2\left[x_1x_2(x_1-x_2)+x_2x_3(x_2-x_3)+
x_3x_1(x_3-x_1)\right]^2=
     -27\left[K+L\right]\,,
     $$
where
    $$
K=x_1^2x_2^2(x_1-x_2)^2+x_2^2x_3^2(x_2-x_3)^2+x_3^2x_1^2(x_3-x_1)^2\,,
     $$
     $$
  L=2x_1x_2x_3\left[x_1(x_1-x_2)(x_3-x_1)+x_2(x_1-x_2)(x_2-x_3)+
    x_3(x_2-x_3)(x_3-x_1)\right]\,.
    $$
using  equation (*) and  fact that $x_1+x_2+x_3=0$ we see that
    $$
K=x_1^2x_2^2(x_1+x_2)^2+x_2^2x_3^2(x_2+x_3)^2+x_3^2x_1^2(x_3+x_1)^2-
             4(x_1^3x_2^3+x_2^3x_3^3+x_3^3x_1^3)=\,,
    $$
  $$
3x_1^2x_2^2x_3^2-4(x_1x_2+x_2x_3+x_3x_1)
(x_1^2x_2^2+x_2^2x_3^2+x_3^2x_1^2-x_1^2x_2x_3-x_1x_2^2x_3-x_1x_2x_3^2)-
12x_1^2x_2^2x_3^2
  $$
    $$
-9x_1^2x_2^2x_3^2-4(x_1x_2+x_2x_3+x_3x_1)
      \left[(x_1x_2+x_2x_3+x_3x_1)^2-3(x_1^2x_2x_3+x_1x_2^2x_3+x_1x_2x_3^2)
      \right]=
        $$
        $$
   =-9x_1^2x_2^2x_3^2-4(x_1x_2+x_2x_3+x_3x_1)^3=-9q^2-4p^3\,.
    $$
and 
    $$
L=2x_1x_2x_3\left[x_1(x_1-x_2)(x_3-x_1)+x_2(x_1-x_2)(x_2-x_3)+
    x_3(x_2-x_3)(x_3-x_1)\right]=2x_1x_2x_3M\,.
    $$ 
with
    $$
M=x_1(x_1-x_2)(x_3-x_1)+x_2(x_1-x_2)(x_2-x_3)+
    x_3(x_2-x_3)(x_3-x_1)=
    $$
      $$
   \left(-x_1^3+x_1(x_2+x_3)-x_1x_2x_3\right)+
   \left(-x_2^3+x_2(x_1+x_3)-x_1x_2x_3\right)+
   \left(-x_3^3+x_3(x_1+x_2)-x_1x_2x_3\right)=
      $$
    $$
-2(x_1^3+x_2^3+x_3^3)-3x_1x_2x_3=-9x_1x_2x_3
    $$
since $x_1+x_2+x_3=0$. Hence
    $$
L=2x_1x_2x_3M=-18x_1^2x_2^2x_3^2=-18q^2\,.
    $$
We come to
     $$
(u-v)^2=-27(K+L)=-27(-9q^2-4p^3-18q^2)=27(27q^2+4p^3)\,.
     $$
Finally we have
          $$
    \cases
      {u+v=w_I^3+w_{II}^3=-27q\cr
        u-v=w_I^3-w_{II}^3=\pm 27\sqrt {q^2+{4p^3\over 27}}
         }
          $$
  or:
      $$    \cases
        {
w_0=x_1+x_2+x_3\cr
w_I=x_1+\vare x_2+\vare^2 x_3\cr
w_{II}=x_2+\vare x_1+\vare^2 x_3\cr
        }\qquad (\vare=e^{2\pi i\over 3})\,,\qquad
{\rm with}\,\,  
           \cases
      {
       w_I^3+w_{II}^3=-27q\cr
       w_I^3-w_{II}^3=\pm 27\sqrt {q^2+{4p^3\over 27}}
         }
      $$
One can easy to solve the system of linear equations
using the fact that $1+\vare+\vare^2=0$.
E.g. adding first equation mulitplied by $\vare^2$ with second and third
we come to the answer for $x_3$ and so on:
               $$
   \cases
       {
   x_1={w_I+\vare^2 w_{II}\over 3}\cr
   x_2={w_{II}+\vare^2 w_I\over 3}\cr
   x_3={(w_I+w_{II})\vare  \over 3}\cr
        }\,,\quad {\rm with}\quad
         \cases
      {
       w_I^3+w_{II}^3=-27q\cr
       w_I^3-w_{II}^3=\pm 27\sqrt {q^2+{4p^3\over 27}}
         }
         $$
i.e.
             $$
       \cases
            {
          w_I^3=27\left(-{q\over 2}+ 
       \sqrt {{q^2\over 4}+{p^3\over 27}}\right)\cr
          w_{II}^3=27\left(-{q\over 2}- 
       \sqrt {{q^2\over 4}+{p^3\over 27}}\right)\cr
             }\,\,{\rm or}\,\, 
                \cases
            {
          w_I^3=27\left(-{q\over 2}+ 
       \sqrt {{q^2\over 4}+{p^3\over 27}}\right)\cr
          w_{II}^3=27\left(-{q\over 2}- 
       \sqrt {{q^2\over 4}+{p^3\over 27}}\right)\cr
             }
               $$
(We suppose that the branch of function $\sqrt {}$ is chosen)
These equations define $w_I,w_{II}$ not uniquely. There are man solutions:
                  $$
                   \matrix
                  {
    w_I=3\left(-{q\over 2}+
       \sqrt {{q^2\over 4}+{p^3\over 27}}\right)\,,\,
    w_{II}=3\left(-{q\over 2}-
       \sqrt {{q^2\over 4}+{p^3\over 27}}\right)\cr
    w_I=3\vare\left(-{q\over 2}-
       \sqrt {{q^2\over 4}+{p^3\over 27}}\right)\,,\,
    w_{II}=3\left(-{q\over 2}+
       \sqrt {{q^2\over 4}+{p^3\over 27}}\right)\cr
    w_I=3\left(-{q\over 2}+
       \sqrt {{q^2\over 4}+{p^3\over 27}}\right)\,,\,
    w_{II}=3\left(-{q\over 2}-
       \sqrt {{q^2\over 4}+{p^3\over 27}}\right)\cr
    w_I=3\left(-{q\over 2}-
       \sqrt {{q^2\over 4}+{p^3\over 27}}\right)\,,\,
    w_{II}=3\left(-{q\over 2}+
       \sqrt {{q^2\over 4}+{p^3\over 27}}\right)\cr
    w_I=3\left(-{q\over 2}+
       \sqrt {{q^2\over 4}+{p^3\over 27}}\right)\,,\,
    w_{II}=3\left(-{q\over 2}-
       \sqrt {{q^2\over 4}+{p^3\over 27}}\right)\cr
    w_I=3\left(-{q\over 2}-
       \sqrt {{q^2\over 4}+{p^3\over 27}}\right)\,,\,
    w_{II}=3\left(-{q\over 2}+
       \sqrt {{q^2\over 4}+{p^3\over 27}}\right)\cr
                     }
                  $$ 
{\bf Remark} These are solutions of sixth order resolvent
equations. On the other hand we need solutions of
 cubic equation.  How? 

Return again to invariant 
      $$
R=w_Iw_{II}
      $$
This is invariant of all permutations. Calculate it:
         $$
   w_Iw_{II}=(x_1+\vare x_2+\vare^2 x_3)(\vare x_1+ x_2+\vare^2 x_3)=
      (1+\vare^2)(x_1x_3+x_1x_2+x_2x_3)+\vare(x_1^2+x_2^2+x_3^2)=
           $$
           $$
\vare(x_1+x_2+x_3)^2+(\vare^2+1-2\vare)(x_1x_3+x_1x_2+x_2x_3)=
\vare(x_1+x_2+x_3)^2-3\vare)(x_1x_3+x_1x_2+x_2x_3)=a\vare-3p^3\vare
         $$
(Compare with previous answer $(\dagger))$
Here we see that we have to take such branchs tht
the conditionabove holds.
E.g. in the case of one real root we have to multiply on $\vare^2$.



$$ $$

{\bf Appendix}

Let $L^N=1$.

TWe have to prove that operator $L$ does not possess
no-trivial Jorda cells.
Suppose it possesses non-trivial Jorda cell, i.., 
there is a subspace $V$ such that $L$ on this subspace
is an operator $L=\l+I$, where $I$ is nilpotent operator,
and $\l=e^{2\pi i s\over N}$ is a root of unity. 
We have that $I^m=0$, where
$m$ is a size of Jordan cell, we have to prove that
$m=1$, $I=0$. Take an arbitrary vector $\x$. Prove
that $I\x=0$. If this is not the case then there exists 
$k$, ($2\leq k\leq m$) such that $I^k\x=0$, but $I^{k-1}\x\not=0$,
   Take $\y=I^{k-2}\x$.  We have 
       $$
\y=L^N(\y)=(\l+I)^N(\y)=\l^N\y+NI(\y)+0=\y+NI(\y)\not=\y\,.
       $$
Contradiction. Hence $I\x=0$.

  This proof is founded on the classification theorem 
which I will be happy to tell here:



Formulate and prove the general classification Theorem.

 let $L$ be a linear operator on $n$-dimensional vector space $V$.
 Consider its characteristic polynomial
     $$
P(t)=\det{t-L}=\prod_k (t-\l_k)^{r_k}, \sum r_k=n\,.
     $$

To every root $\l_i$ one can assign at least one 
non-zero eigenvector $\e_i$. If $r_i\not=1$, then
situation is not so simple.

We assign to complex number $\l$ the generalised eigenspace $V_\l$, the set
     $$
     V_\l=\{x\colon \,\,  N \,\, {such\,\, that}\,\, (L-\l )^N=0\}
     $$
(One can easy to prove that $V_\l$ is a set). 

{\bf Exercise}
          $$
\l\in \{\l_i\} \Leftrightarrow \, V_\l\not=0\,.
          $$

Now we represent $V_i$ as a direct sum of spaces
         $$
V=\oplus_{k}V_{\l_k}, \quad \dim V_{\l_k}=r_k
         $$
Every subspace is corresponding to eigenvalue $\l_i$.
 Operator $L$ is a direct sum of operators $L_i$,
 such that every $L_i$ acts o $V_i$, and
it is a sum on Jordan. cells on $V_i$.
If Polynomial $P(t)$ is minimal then all $V_i$
are non-reducible Jordan cells.

Before doing this consider the the problem
of decomposition of fraction $1\over P(t)$
on elementary fractions:



  We will do this decomposition.

Denote by
      $$
L_i=\prod_{\l_m\not=\l_i}(L-\l_m)
      $$




  Let $L$ be linear operator on $V$, $\dim V<\infty$
such that $L^n=1$.
We suppose that $V$ is over algebraically closed field.
(In fact it is enough if $V$ is defined over 
cyclotomic subfield.)

P
  Let $\{\l_i\}$ be set of eigenvalues of $L$, set of roots
of polynomial $$
   P(t)=\det (t-L)=\prod_i(t-\l_i)^{r_i}
      $$

For eigenvalues which do not coincide, $r_i=1$, everything is simple.

To every eigenvalue $\l_i$ one corresponds at least
one eigenvector $\e_k$: $L\e_k=\l_k\e_k$, $\l_k={e^{2\pi i s_k\over n}}$,
since $L^n=1$.

  Let eigenvalue $\l$ ($\l=e^{2\pi is_k\over n}$) be 
degenerate: $P(t)=(t-a)^mF(t)$, where $F(\l)\not=0$.
 We want to show that there are exactly $m$ linearly independent
eignevectors with eigenvalue $\l$.


Denote $V_\l$ the set of root vectors of $V$ which correspond to $\l$:
                
      $$3
     V_\l=\{\x\colon \,\, \exists N \,\, {such\,\, that}\,\, (L-\l\x)^N=0\}
       $$
It is evident that $V_\l$ is vector space. We say that $V_\l$ is
geberalised vector space corresponding to value $\l$.

\bye
    
