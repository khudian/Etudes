



\magnification=1200 \baselineskip=14pt


\def\p {\partial}
\def \D {\Delta_{d{\bf v}}}
\def \Ds  {\Delta^{\#}}
\def\t{\tilde}
\def\s {\sigma}
\def\vare {\varepsilon}
\def\L {\Lambda}
\def\Darboux {$z^A=$  $x^1,\dots,x^n$, $\theta_1,\dots,\theta_n$}
\def\a{\alpha}
\def\O{\Omega}
\def\d{\delta}
\def\dv  {{d{\bf{v}}}}
\def\A {{\cal A}}
\def\R {I\!R}
\def\t {\tilde}


             \centerline {Cubic and quadric equations; Galois theory for pedestrians.}

{\it This text is written on the base of the book of A. Khovansky}

 We suppose that Galois group exists. THis is not trivial.... 

   We suppose that the main field is a field of characteristic
   $0$ which possesses all roots of unity.
which possesses all roots of unity.

 {\bf Proposition} Let $A$ be an algebra over field $K$,
and let $G$ be a finite 
commutative group acting on this algebra.

  Then every element $x$ of the algebra $A$
can be represented as a sum of elements $w_1,\dots,w_k$
such that for every $w$

Then there exist 
 basis 











\centerline {Calculations for cubic equation}

It is conveniuent to consider the equation
   $x^3+px+q=0$ with
       $$
     \matrix
        {
x_1+x_2+x_3=0\cr
x_1x_2+x_2x_3+x_3x_1=p\cr
 x_1x_2x_3=-q\cr
        }
       $$
We introduce Lagrange variables
       $$
     \matrix
        {
w_0=x_1+x_2+x_3\cr
w_I=x_1+\vare x_2+\vare^2 x_3\cr
w_{II}=x_2+\vare x_1+\vare^2 x_3\cr
        }\qquad (\vare=e^{2\pi i\over 3})\,.
       $$
$w_0=0$. We have
  $$
     \matrix
        {
w_0^3=0\cr
u=w_I^3=x_1^3+x_2^3+x_3^3+3\vare(x_1^2x_2+x_2^2x_3+x_3^2x_1)+
              3\vare^2(x_1x_2^2+x_2x_3^2+x_3 x_1^2)+6x_1x_2x_3\cr
v=w_{II}=x_1^3+x_2^3+x_3^3+3\vare^2(x_1^2x_2+x_2^2x_3+x_3^2x_1)+
              3\vare(x_1x_2^2+x_2x_3^2+x_3 x_1^2)+6x_1x_2x_3\cr
       }
       $$
Hence
      $$
u+v=2(x_1^3+x_2^3+x_3^3)+3\vare(1+\vare)(x_1^2x_2+\dots)+12x_1x_2x_3
      $$
$\dots$ means all permutations.Since
   $$
a^3+b^3+c^3-3abc=(a+b+c)(a^2+b^2+c^2-ab-ac-bc)\,,
   \eqno (*)
   $$
$w_0=0$ and $\vare(1+\vare)=-1$, hence 
        $$
u+v=6x_1x_2x_3
-3\left(
  x_1x_2(-x_3)+x_1x_3(-x_2)+x_2x_3(-x_1)
  \right)+12x_1x_2x_3=27x_1x_2x_3=-27q\,,
        $$
       $$
u-v=3(\vare-\vare^2)(x_1^2x_2+x_2^2x_3+x_3^2x_2)+
    3(\vare^2-\vare)(x_1x_2^2+x_2x_3^2+x_3x_2^2)=
       $$
       $$
   3(\vare-\vare)^2
             \left(
      x_1x_2(x_1-x_2)+x_2x_3(x_2-x_3)+x_3x_1(x_3-x_1)\,.
           \right)
       $$
and
     $$
(u-v)^2=9(\vare-\vare^2)^2\left[x_1x_2(x_1-x_2)+x_2x_3(x_2-x_3)+
x_3x_1(x_3-x_1)\right]^2=
     -27\left[K+L\right]\,,
     $$
where
    $$
K=x_1^2x_2^2(x_1-x_2)^2+x_2^2x_3^2(x_2-x_3)^2+x_3^2x_1^2(x_3-x_1)^2\,,
     $$
     $$
  L=2x_1x_2x_3\left[x_1(x_1-x_2)(x_3-x_1)+x_2(x_1-x_2)(x_2-x_3)+
    x_3(x_2-x_3)(x_3-x_1)\right]\,.
    $$
using  equation (*) and  fact that $x_1+x_2+x_3=0$ we see that
    $$
K=x_1^2x_2^2(x_1+x_2)^2+x_2^2x_3^2(x_2+x_3)^2+x_3^2x_1^2(x_3+x_1)^2-
             4(x_1^3x_2^3+x_2^3x_3^3+x_3^3x_1^3)=\,,
    $$
  $$
3x_1^2x_2^2x_3^2-4(x_1x_2+x_2x_3+x_3x_1)
(x_1^2x_2^2+x_2^2x_3^2+x_3^2x_1^2-x_1^2x_2x_3-x_1x_2^2x_3-x_1x_2x_3^2)-
12x_1^2x_2^2x_3^2
  $$
    $$
-9x_1^2x_2^2x_3^2-4(x_1x_2+x_2x_3+x_3x_1)
      \left[(x_1x_2+x_2x_3+x_3x_1)^2-3(x_1^2x_2x_3+x_1x_2^2x_3+x_1x_2x_3^2)
      \right]=
        $$
        $$
   =-9x_1^2x_2^2x_3^2-4(x_1x_2+x_2x_3+x_3x_1)^3=-9q^2-4p^3\,.
    $$
and 
    $$
L=2x_1x_2x_3\left[x_1(x_1-x_2)(x_3-x_1)+x_2(x_1-x_2)(x_2-x_3)+
    x_3(x_2-x_3)(x_3-x_1)\right]=2x_1x_2x_3M\,.
    $$ 
with
    $$
M=x_1(x_1-x_2)(x_3-x_1)+x_2(x_1-x_2)(x_2-x_3)+
    x_3(x_2-x_3)(x_3-x_1)=
    $$
      $$
   \left(-x_1^3+x_1(x_2+x_3)-x_1x_2x_3\right)+
   \left(-x_2^3+x_2(x_1+x_3)-x_1x_2x_3\right)+
   \left(-x_3^3+x_3(x_1+x_2)-x_1x_2x_3\right)=
      $$
    $$
-2(x_1^3+x_2^3+x_3^3)-3x_1x_2x_3=-9x_1x_2x_3
    $$
since $x_1+x_2+x_3=0$. Hence
    $$
L=2x_1x_2x_3M=-18x_1^2x_2^2x_3^2=-18q^2\,.
    $$
We come to
     $$
(u-v)^2=-27(K+L)=-27(-9q^2-4p^3-18q^2)=27(27q^2+4p^3)\,.
     $$
Finally we have
          $$
    \cases
      {u+v=w_I^3+w_{II}^3=-27q\cr
        u-v=w_I^3-w_{II}^3=\pm 27\sqrt {q^2+{4p^3\over 27}}
         }
          $$
  or:
      $$    \cases
        {
w_0=x_1+x_2+x_3\cr
w_I=x_1+\vare x_2+\vare^2 x_3\cr
w_{II}=x_2+\vare x_1+\vare^2 x_3\cr
        }\qquad (\vare=e^{2\pi i\over 3})\,,\qquad
{\rm with}\,\,  
           \cases
      {
       w_I^3+w_{II}^3=-27q\cr
       w_I^3-w_{II}^3=\pm 27\sqrt {q^2+{4p^3\over 27}}
         }
      $$
One can easy to solve the system of linear equations. 
E.g. adding first equation mulitplied by $\vare^2$ with second and third
we come to the answer for $x_3$ and so on:
        $$
    \matrix
        {
w_{I}+\vare ^2 w_{II}+w_0=3x_1\cr
w_{II}+\vare^2 w_{I}+w_0=3x_2 
\vare ^2 w_0+w_1+w_2=3x_3\cr
}
        $$
  On the other hand we have first to answer at what extend of uncertainty
the numbers $w_I,w_II$ are defined by the equation above?


Hence
               $$
   \cases
       {
   x_1={w_I+\vare^2 w_{II}\over 3}\cr
   x_2={w_{II}+\vare^2 w_I\over 3}\cr
   x_3={(w_I+w_{II})\vare  \over 3}\cr
        }\,,\quad {\rm with}\quad
         \cases
      {
       w_I^3+w_{II}^3=-27q\cr
       w_I^3-w_{II}^3=\pm 27\sqrt {q^2+{4p^3\over 27}}
         }
         $$
i.e.
             $$
       \cases
            {
          w_I^3=27\left(-{q\over 2}+ 
       \sqrt {{q^2\over 4}+{p^3\over 27}}\right)\cr
          w_{II}^3=27\left(-{q\over 2}- 
       \sqrt {{q^2\over 4}+{p^3\over 27}}\right)\cr
             }\,\,{\rm or}\,\, 
                \cases
            {
          w_I^3=27\left(-{q\over 2}+ 
       \sqrt {{q^2\over 4}+{p^3\over 27}}\right)\cr
          w_{II}^3=27\left(-{q\over 2}- 
       \sqrt {{q^2\over 4}+{p^3\over 27}}\right)\cr
             }
               $$
(We suppose that the branch of function $\sqrt {}$ is chosen)
These equations define $w_I,w_{II}$ not uniquely. There are six solutions:
                  $$
                   \matrix
                  {
    w_I=3\left(-{q\over 2}+
       \sqrt {{q^2\over 4}+{p^3\over 27}}\right)\,,\,
    w_{II}=3\left(-{q\over 2}-
       \sqrt {{q^2\over 4}+{p^3\over 27}}\right)\cr
    w_I=3\vare\left(-{q\over 2}-
       \sqrt {{q^2\over 4}+{p^3\over 27}}\right)\,,\,
    w_{II}=3\left(-{q\over 2}+
       \sqrt {{q^2\over 4}+{p^3\over 27}}\right)\cr
    w_I=3\left(-{q\over 2}+
       \sqrt {{q^2\over 4}+{p^3\over 27}}\right)\,,\,
    w_{II}=3\left(-{q\over 2}-
       \sqrt {{q^2\over 4}+{p^3\over 27}}\right)\cr
    w_I=3\left(-{q\over 2}-
       \sqrt {{q^2\over 4}+{p^3\over 27}}\right)\,,\,
    w_{II}=3\left(-{q\over 2}+
       \sqrt {{q^2\over 4}+{p^3\over 27}}\right)\cr
    w_I=3\left(-{q\over 2}+
       \sqrt {{q^2\over 4}+{p^3\over 27}}\right)\,,\,
    w_{II}=3\left(-{q\over 2}-
       \sqrt {{q^2\over 4}+{p^3\over 27}}\right)\cr
    w_I=3\left(-{q\over 2}-
       \sqrt {{q^2\over 4}+{p^3\over 27}}\right)\,,\,
    w_{II}=3\left(-{q\over 2}+
       \sqrt {{q^2\over 4}+{p^3\over 27}}\right)\cr
                     }
                  $$ 


\bye 
