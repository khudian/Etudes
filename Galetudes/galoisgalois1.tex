\magnification=1200

\baselineskip=14pt

        \centerline {Galois theory for pedestrians}

Short abstract:

  {\it We demonstrate some basic ideas of Galois Theory
   considering quadratic cubic and quartic equations.}



  {\sl Detailed abstract}

  It is well-known that roots $x_1,x_2$ of
quadratic  polynomial $P(x)=x^2+px+q$ obey condition
  $x_1+x_2=-p,x_1 x_2=q$i (Vieta's  formula) . 
  Conisder another functions on roots.  E.g. one can see that
           $$
    x_1^4+x_2^4=p^4-4p^2q+2q^2\,,
           $$ 
but
          $$
    x_1^4-x_2^4=\pm (2pq-p^3\sqrt{p^2-4q}\,.
          $$
   The reason why the first expression is polynomial of roots,
and second expression is proportional to a  square root,
is the following:
  the first expression is invariant with respect to the
permutation $x_1\leftrightarrow x_2$, or more
formally with respect to the action of  group $S_2$ 
of permutation on roots, and the second expression changes
the sign under action of permutation $x_1\to x_2$.
  This statement can be generalised. Let
    $x_1,\dots,x_n$ be roots of $n$-th order polynomial
             $$
    P(x)=x^n+a_1x^{n-1}+a_2x^{n-2}+\dots+a_{n-1}x+a_n\,,
          \eqno (1)
             $$
where $a_1,\dots,a_n$ are parameters.
One can consider the group of permutations of  the roots
which preserve the coefficients of the polynomnial---Galois group.
 In the case of 
 polynomial (1) it is just a group $S_n$
of permutations of roots.

\noindent It turns out that  properties of roots of polynomial
such as

{\it 


    expressability in radicals (problem of solution)

   Does given expression is a polynomial on parameters $a_1,\dots,a_n$ or not?
   
  Does given expression is expressed via
square root operation?

}

\noindent and many others


can be expressed in answered in terms of the Galois group
of the polynomial.



  For example if $x_1,x_2,x_3$ are roots of cubic polynomial
                $$
           x^3+px+q=0\,,
                $$
then the expression
              $$
    A=A(x_1,x_2,x_3)=(x_1^5+x_2^5)(x_2^5+x_3^5)(x_3^5+x_1^5)\,,
              $$
is a polynomial on parameters $p,q$ since this expression is invariant
under all permutations of roots, and the expression
            
              $$
    B=B(x_1,x_2,x_3)(x_1^5-x_2^5)(x_2^5-x_3^5)(x_3^5-x_1^5)
              $$
is expressed via coefficients $p,q$ and square root operation,
since this expression takes two values
under the action of group $S_3$ of all permutations of roots.

  We answer these questions studying  the action of group $S_3$
of permutations of roots on LHS of these expressions.

  In this talk we try to explain these ideas and on the base
of these considerations we will come to the formulae
for solutions of cubic and quartic equations

\bye
