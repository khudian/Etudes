\magnification=1200
\baselineskip=14pt
       Davidik, here is

        \centerline {\bf On a interpretation of Poisson formula}

\medskip

              \centerline  {${\cal x}  1$  Poisson Formula}
\smallskip

   Let $F(x)$  be a good function on real numbers which tends to zero at infinity.
   Let $G(k)$ be the component of its Fourier expansion:
                        $$
        F(x)=\int_{-\infty}^\infty G(k)e^{ikx}dk
                       \eqno (1)
                     $$

 Note that\footnote{$^*$}{We do not need formulae (2),(3) for obtaining Poisson formula.
 We need them for interpretations}
                     $$
   G(k)={1\over 2\pi}\int_{-\infty}^\infty F(x)e^{-ikx}dx
                            \eqno  (2)
           $$
           and in particularly
            $$
  G(0)={1\over 2\pi}\int_{-\infty}^\infty F(x)dx
                            \eqno  (2^\prime)
           $$
One can prove the following very beautiful identity.
                 $$
 \sum_{n\in {\bf Z}}F(na)={2\pi \over a}\sum_{n\in {\bf Z}}
 G\left({2\pi n\over a}\right),
\eqno (3)
                  $$
Here $a$ is an arbitrary parameter. Summation goes over all integers.
This is famous Poisson identity. It says that {\it sum of the values of function over a
lattice  coincides with sum of the values of its Fourier image over reciprocal
lattice.}


{\bf Example 1}. Consider $F(x)=e^{-|x|}$.  One can  see that
                       $$
      F(x)=e^{-|x|}=\int_{-\infty}^{\infty}{e^{ikx}dk\over \pi(1+k^2)},
             \quad G(k)={1\over \pi (1+k^2)}
\eqno (4)
                     $$
and  respectively
                       $$
               G(k)=
{1\over 2\pi}\int_{-\infty}^\infty e^{-|x|}e^{-ikx}dx=
{1\over 2\pi}\left(\int_{0}^\infty e^{-x}e^{-ikx}dx+
\int_{0}^\infty e^{-x}e^{ikx}dx\right)={1\over\pi(1+k^2)}
\eqno (5)
                       $$

Then Poisson formula gives that  for $a>0$
                        $$
\sum_{n\in {\bf Z}}e^{-|n|a}=1+2\sum_{n=1}^\infty
e^{-na}={2\pi \over a}\sum_{n\bf Z}{1\over \pi\left(1+{4\pi^2  n^2\over  a^2}\right)}=
\sum_{n\in {\bf Z}}
    {2a\over a^2+4\pi^2n^2}
\eqno (6)
                         $$

{\bf Example 2}
 Consider $F(x)=e^{-x^2}$.  One can  see that
                       $$
      F(x)=e^{-x^2}=
 {1\over 2\sqrt \pi}
\int_{-\infty}^{\infty}e^{-k^2\over 4}e^{ikx}dk,
             \quad G(k)={e^{-{k^2\over 4}}\over 2\sqrt\pi }
\eqno (4)
                     $$
and  respectively
                       $$
               G(k)=
{1\over 2\pi}\int_{-\infty}^\infty e^{-x^2}e^{-ikx}dx=
{1\over 2\pi}\int_{-\infty}^\infty e^{-(x+{ik\over 2})^2-{k^2\over 4}}dx
={1\over 2\sqrt \pi}e^{-{k^2\over 4}}
\eqno (5)
                       $$

Then Poisson formula gives that  for $a>0$
                        $$
\sum_{n\in {\bf Z}}e^{-n^2a^2}=1+2\sum_{n=1}^\infty
e^{-n^2a^2}=
{\sqrt {\pi}\over a}\sum_{n\in {\bf Z}}
    e^{-\pi^2 n^2\over a^2}
\eqno (6)
                         $$

\medskip

              \centerline  {${\cal x}  2$  Proof of the Poisson Formula}
\smallskip



Before analysing it meaning prove it:

\medskip

{\it Proof  of Poisson formula}

Consider a function
                 $$
 H(x)=\sum_{n\in {\bf Z}}F(x+na)
                 $$
This is periodic function: $H(x)=H(x+a)$. Consider its Fourier expansion in series:
               $$
         H(x)=\sum_{n\in {\bf Z}}c_ne^{2i\pi nx\over a}, \quad {\rm with}\quad
               c_n={1\over a}\int_0^{a}H(x)e^{-2i\pi nx\over a}dx
           $$
(To calculate $c_k$ we multiply the relation above on $e^{-2i\pi kx\over a}$ integrate
it over $x$ using the relation $\int_0^{a}e^{2i\pi (n-k)x\over a}dx$
$=a\delta_{kn}$).

Now considering  the following chain of identities:
                  $$
 \sum_{n\in {\bf Z}}F(na)=H(0)=\sum_{n\in {\bf Z}}c_n=
{1\over a}\sum_{n\in {\bf Z}}\int_0^{a}H(x)e^{-2i\pi nx\over a}dx=
{1\over a}\sum_{n,m\in {\bf Z}}\int_0^{a}F(x+ma)e^{-2i\pi nx\over a}dx=
                  $$
             $$
{1\over a}\sum_{n,m\in {\bf Z}}\int_{ma}^{(m+1)a}F(t)e^{-2i\pi n(t-ma)\over a}dt=
{1\over a}\sum_{n\in {\bf Z}}\int_{-\infty}^{\infty}F(t)e^{-2i\pi nt\over a}dt=
                     $$
                     $$
                 {2\pi \over a} \sum_{n\in\bf Z}G\left({2\pi n\over a}\right)
                     $$

\medskip

              \centerline  {\it ${\cal x}  3$  Poisson Formula  and
              approximation of integral by series}
\smallskip

It is commonplace the relation between Darboux series and integrals:
                   $$
                   \int_\infty^\infty F(x)dx\approx  a\sum_{n\in \bf Z} F(na), \quad
                     \hbox{if $a$ is small, i.e.}\quad
                   \int F(x)dx =\lim_{n\to\infty}\left(a\sum_{n\in \bf Z} F(na)\right)
                    $$
Poisson formula gives exact meaning to the asymptotic of integral by series.
Note that Poisson formula  (3) can be rewritten in the way:
              $$
               a\sum_{n\in {\bf Z}}F(na)=2\pi \sum_{n\in {\bf Z}}
 G\left({2\pi n\over a}\right)=2\pi G(0)+2\pi \sum_{n\not=0}G\left({2\pi n\over a}\right)
              $$
 Now using (2$^\prime$) we come to:
              $$
 a\sum_{n\in {\bf Z}}F(na)=\int_{-\infty}^\infty F(x)dx+
 2\pi \sum_{n\not=0}G\left({2\pi n\over a}\right)
              $$
              or
            $$
           \int_{-\infty}^\infty F(x)dx=
            {1\over a}\sum_{n\in {\bf Z}}F(na)-
 2\pi \sum_{n\not=0}G\left({2\pi n\over a}\right)
 \eqno (\hbox{ approximation of integral})
           $$
This formula gives approximation of integral  by series.

Consider again examples  1 and 2.


{\bf Example 3}

  Function $f=e^{-x}$

Apply the approximation formula to the formulae in the example 1. We come to  ($a>0$):
                       $$
                       a\sum_{n\in {\bf Z}}e^{-|n|a}=2\pi G(0)+
         2\pi \sum_{n\not=0}{1\over \pi\left(1+{4\pi^2 n^2\over  a^2}\right)}=
\int_{-\infty}^\infty e^{-|x|}dx+
\sum_{n=1}^\infty{4\over \left(1+{4\pi^2 n^2\over  a^2}\right)},
                      $$
   It can be rewritten with boundary term:
                $$
     \int_0^\infty e^{-x}dx={a\over 2}+a\sum_{n=1}^\infty e^{-na}-
     \sum_{n=1}^\infty{4\over \left(1+{4\pi^2 n^2\over  a^2}\right)}
                     $$

{\bf Example 4}

  Function $f=e^{-x^2}$

It follows from the Example 2  and the last approximation formula that
                   $$
                   a\sum_{n\in {\bf Z}}e^{-n^2a^2}=
\sqrt {\pi}+2\sqrt\pi\sum_{n=1}^\infty,
    e^{-\pi^2 n^2\over a^2}
                   $$
i.e.
                $$
   \int_{-\infty}^\infty e^{-x^2}dx=a\sum_{n\in {\bf Z}}e^{-n^2a^2}-2\sqrt\pi\sum_{n=1}^\infty,
    e^{-\pi^2 n^2\over a^2}
                $$


{\it N.B.}  The last formula is related with the preancestor  of Seeley formula:
  It is about the following: Let $\lambda_n$ be eigenvalues of the Laplace operator,
  then the asymptotic of the following function $Z(t)=\sum_n e^{-\lambda_nt}$
  can be expressed in terms of basic terms. E.g. if $\lambda_n$ are eigenvalues of operator
  $\partial^2$ on the closed interval $[0.L]$  then
                 $$
    Z(t)=\sum_n e^{-\lambda_nt}=\sum_n e^{-n^2 t\over l^2}\approx {l\over \sqrt t}
                 $$
In the general case:


 One can prove the following: Let ${\lambda_i}$  be frequencies  for
$d$-dimensional drum, i.e. eigenvalues of the Laplacian acting on this drum:
Then
                      $$
                      Z(t)={V\over t^{d/2}}+\dots
                      $$
i.e. one can estimate dimension of the drum and its volume just hearing it!!!!



\bye
