% I began this file on 30 October on the base of the file integral1.tex
      \magnification=1200 %\baselineskip=14pt
\def\vare {\varepsilon}
\def\A {{\bf A}}
\def\I {{\bf I}}
\def\FF {{\bf F}}
\def\a {\alpha}
\def\K {{\bf K}}
\def\s {{\sigma}}
\def\p{\partial}
\def\vare{{\varepsilon}}
\def\L {{\cal L}}
\def\G {{\Gamma}}
\def\C {{\bf C}}
\def\Z {{\bf Z}}
\def\U  {{\cal U}}
\def\R  {{\bf R}}
\def\E  {{\bf E}}
\def\l {\lambda}
\def\degree {{\bf {\rm degree}\,\,}}
\def \finish {${\,\,\vrule height1mm depth2mm width 8pt}$}
\def \m {\medskip}
\def\r {{\bf r}}
\def\v {{\bf v}}
\def\n {{\bf n}}
\def\b {{\bf b}}
\def\ll  {{\bf l }}
\def\ss  {{\bf s }}
\def\e{{\bf e}}
\def\ac {{\bf a}}
\def \X   {{\bf X}}
\def \Y   {{\bf Y}}
\def \x   {{\bf x}}
\def \y   {{\bf y}}
\def\w {{\omega}} 
\def\wv {{\buildrel \rightarrow\over \omega}}
\def\Sv {{\buildrel \rightarrow\over \Sigma}}
\def\grad {\rm grad\,}
\def\div {\rm div\,}
\def\rot {\rm rot\,}

06 November 2013


     \centerline {\bf On caluclation of some surface integrals .}
\m {\it Let $p=p(\r)$ be a scalar function in $\E^3$. 
We consider the following
three surface integrals:
              $$
        \int_M p(\r)d\ss\,,\,\,(1)\,\,,\quad
        \int_M \grad p(\r)d\ss\,,\,\,(2)\,\,,\quad
        \int_M \grad p(\r)\times d\ss\, \,\,(3)\,.
                  $$
Here as always $d\ss$ is vector valued element of surface.

Anybody  who learned basic vector calculus
(or theoretical physics) knows many integrals like these.
I would like to analyze difference between these
three integrals. 


In the case if $M=\p D$ is a boundary of a domain $D$ then
the second integral can be calculated  
by standard application of Gauss-Ostrogradsky formula:
            $$
    \int_{\p D} \grad p(\r)d\ss=
  \int_D \div\grad p(\r)=\int_D\Delta p(\r)dV\,. 
           $$
For the first integral standard calculations (see below) 
show that in the case if $M=\p D$ then
         $$
  \int_{\p D}p(\r)d\ss=\int_D \grad p(\r)dV\,.
           $$
This is nothing but Archimedes's principle.

The third integral vanishes on closed surfaces and moreover
for arbitrary surface $M$ it is reduced to integral over contour $\p M$: 
            $$
 \int_{M} \grad p(\r)\times d\ss=\int_{\p M}p(\r)d\ll,.
            $$
	The third integral looks peculiar: not only volume form and
metric but also vector product is engaged in integrals construction.
It  turns out that this integral  is topological: 
rightly viewed it does not depend on metric structures.


I am grateful to Grigory Vekstein who have focused my attention
on the very beautiful integral (3).
}



\bigskip

\centerline {${\cal x}\,\, 1$ \bf 
Gauss-Ostrogradsky formula 
 and differential forms}

\m

First of all recall Gauss-Ostrogradsky formula for flux of vector field.
 If closed surface $M$ is the boundary of domain $D$, $M=\p D$, then
this  formula expresses the flux of vector field
through the surface via the integral over the domain:
              $$
           \int_{\p D} \K d\ss=\int_D {\rm div\,}\K d^3x\,.
              $$
The most illuminating way to understand this formula, is to use the
language of differential forms.  The integrand in the left hand side is
$2$-form, the integrand in the right hand side is a $3$-form.
More in detail: the flux of vector field $\K$
through the surface $M$ is equal to integral of $2$-form
$\omega_\K=\Omega{\cal c}\K$ over surface $M$:
             $$
   \hbox {Flux of $K$ via $M$}=\int_M \K d\ss=\int\Omega{\cal c}\K\,,
             $$ 
where $\Omega$ is a volume form.
Due to Stokes Theorem:
       $$
   \int_{\p D} \w=\int_D d\w\,,
     $$
 we have that if surface $M$ is a boundary, $M=\p D$ then:
         $$
\hbox {Flux of $K$ via $\p D$}=
\int_{\p D} \K d\ss=\int_{\p D}\Omega{\cal c}\K=
    \int_D d\left(\Omega{\cal c}\K\right)\,.
           $$
Cartan formula gives that
          $
d\left(\Omega{\cal c}\K\right)=\L_\K\Omega=
  \left({\rm div}_\Omega\,\K\right)\Omega\,
        $
hence
        $$
\int_{\p D} \K d\ss=\int_{\p D}\Omega{\cal c}\K=
    \int_D d\left(\Omega{\cal c}\K\right)=\int
 \left({\rm div}_\Omega\,\K\right)\Omega\,.
           $$
In coordinates: if volume form $\Omega=\rho dx\wedge dy\wedge dz$ then
           $$
{\rm div}_\Omega\,\K={d\left(\Omega{\cal c}\K\right)\over \Omega}=
          {
  {d\left(\rho dx\wedge dy\wedge dz {\cal c}
   \left(K_x\p_x+K_y\p_y+K_z\p_z\right)\right)}
      \over \rho dx\wedge dy\wedge dz}=
           $$
        $$
                {
 d\left(\rho\left(K_x dy\wedge dz-K_ydx\wedge dz+
 K_zdx\wedge dy\right)\right)
    \over \rho dx\wedge dy\wedge dz
        }=
       {1\over\rho}
           \left
            (
  {\p (\rho K_x)\over \p x}+
  {\p (\rho K_y)\over \p y}+
  {\p (\rho K_z)\over \p z}
            \right)=
       $$
        $$
  {\p  K_x\over \p x}+
  {\p  K_y\over \p y}+
  {\p \rho K_z\over \p z}+
      K_x{\p\log \rho\over \p x}+
      K_y{\p\log \rho\over \p y}+
      K_z{\p\log \rho\over\p z}\,.
        $$
This is standard stuff. Integral over surface $M=\p D$ 
is reduced to the integral
over domain $D$ due to the fact that
integrand is a differential form, and the Stokes formula works. 
Of course not every surface integral over closed surface $M=\p D$
can be reduced to the integral over domain $D$.  
In general, integral over surface is an 
integral of {\it density}. Reduction happens if the density
is a differential form.

 It often happens when we integrate over surface 
not differential
form, but {\it vector-valued} differential forms. 
This is just the case with integrals (1) and (3).
Note that flux of vector field which was
calculated above is integral of a function over surface, not vector field.
In the case if integrand is vector-valued  and it is differential form,
Stokes-Gauss-Ostrogradsky like formula still works but there are some problems.
 
   Return to calculations of integrals (1) and (3) considered above.

  Both these integrals are integrals of vector valued differential form over
surface---vector field is integrated over surface.
(For the integral (2) scalar-valued form is integrated over surface.)
   The integrals (1) and (3) are defined
in Euclidean space where integration of vector field has a since since
one can add two vectors attached at different points.
In general integral of vector field over surface may be 
defined only in spaces
with absolute parallelism, where you have well defined transport of vectors
 from point to point, independent on paths.

\smallskip

We first recall how to calculate these integrals using 
standard trick\footnote{${^*)}$}{which every physicist 
knows from books like
 'Batygin, Toptygin'}, then we will consider these 
integrals in a more general framework.


\smallskip

{\sl Calculation of integral $\FF=\int_{\p D}p(\r)d\ss$.}

\smallskip

Calculate this integral (1) using the standard trick:
take the scalar product of an arbitrary {\it constant} vector $\ac$   
with  integrand in (1) in surface integral .
  Then we come to the flux of vector field
$p(\r)\ac$ through surface $\p D$: 
       $$
 {\FF}\cdot\ac= \oint_{\p D} \left (p(\r)\ac\right)d \ss\,.
       $$ 
Applying  Gauss-Ostogradsky law to this integral we come to 
       $$
 {\FF}\cdot\ac= \oint_{\p D} \left (p(\r)\cdot \ac\right)d \ss=
     \int {\rm div\,}\left(p(\r)\ac\right)=
    \ac\cdot\int_D \grad p(\r)dV\,.
               $$
Since this relation is obeyed for an arbitrary constant vector $\ac$ then
     $$
  \FF=\oint_{\p D} p(\r)d\ss=\int\grad p(\r)dV\,.
     $$
In particular if pressure $p=\rho gz$ then we come to Archimedes's law:
     $$
  \FF=\oint_{\p D} p(\r)d\ss=\int_D\grad p(\r)dV=\rho g\int_D dV=\rho gV_D\,. 
     $$

\m

{\sl Calculation of integral $\I_M=\int_M \grad p(\r)\times d\ss$.}

\smallskip


 The same trick works for the integral (3): 

$\ac\cdot (\grad p(\r)\times d\ss)=(\ac\times \grad p(\r))\cdot d\ss$.
Hence if $\ac$ is constant vector then
        $$
 \ac \cdot \I_M=\int_{M} \ac\cdot(\grad p(\r)\times d\ss=
 \int_{M}(\ac\times \grad p(\r))d\ss\,.
            $$
If $M=\p D$ is a boundary of domain $D$ then
             $$
 \ac \cdot{\bf I}_{\p D}=
\int_{\p D} \ac\cdot(\grad p(\r)\times d\ss=
 \int_{\p D}(\ac\times \grad p(\r))d\ss=
   \int_D {\rm div\,}(\ac\times\grad p(\r))dV=0\,,
        $$
since ${\rm div\,}(\ac\times\grad p(\r))=
a\cdot{\rm rot\,}\circ{\rm grad\,}p(\r)=0$. We see that 
$\I_{M}=\int_M\grad p(\r)\times d\ss=0$
if $M$ is a boundary. In fact we can say more:
We have that 
 $\ac\times\grad p(\r)=\rot (p(\r)ac))$. Hence
        $$
 \ac \cdot \I_M=\int_{M} \ac\cdot(\grad p(\r)\times d\ss=
 \int_{M}(\ac\times \grad p(\r))d\ss=
 \int_{M}\rot (p(\r)\ac)d\ss=\int_{\p M}p(\r)\ac\cdot d\ll\,.
            $$
Thus we see that for arbitrary surface $M$
             $$
      \I_M=\int_{M} \grad p(\r\times d\ss=\int_{\p M}p(\r)d\ll\,.
             $$

\m






\centerline {${\cal x}\,\, 2$ Vector-valued forms}

\medskip


 Yes, integrals are calculated, but we prefer more illuminating 
way to do it....

One can consider integrals over surfaces in Euclidean spaces
of vector-valued vector form.  
First of all express vector valued area element 
 in terms of vector valued form.

  Let $\r=\r(\xi,\eta)$ be a local parameterisation of surface $M$ in $\E^3$.

Then vector-valued surface element of $M$ is equal to  
                 $$
         d\s=|\r_\xi\times\r_\eta|d\xi\wedge d\eta=
       \sqrt{\r_\xi^2\r_\eta^2-\left(\r_\xi\cdot\r_\eta\right)^2}
                  d\xi\wedge d\eta=
                 $$
                 $$
              \sqrt{
          \left(x_\xi y_\eta-x_\eta y_\xi\right)^2+
          \left(x_\xi z_\eta-x_\eta z_\xi\right)^2+
          \left(z_\xi y_\eta-z_\eta y_\xi\right)^2
                }d\xi\wedge d\eta
               $$
A normal unit vector to the surface is equal to 
 $\n={\r_\xi\times\r_\eta\over |\r_\xi\times\r_\eta|}$ and vector 
surface element is equal to 
             $$
   d\ss=\n d\s=\left(\r_\xi\times \r_\eta\right) d\xi\wedge d\eta
             $$ 
We see that vector surface element is expressed 
trough vector valued $2$-form:
        $$
 \Sv\colon\quad     \wv(\r_\xi,\r_\eta)=\r_\xi\times \r_\eta\,,
 \quad\Sv\big\vert_M d\xi d\eta=d\ss\,.
         $$
In Cartesian coordinates
        $$
        \Sv_{\rm surf}=
       dx\wedge dy{\p \over \p z}-dx\wedge dz{\p \over \p y}+
   dy\wedge dz{\p \over \p x}=
{1\over 2}\epsilon_{ikm}dx^i\wedge dx^k{\p \over \p x^m}\,. 
                  $$

In arbitrary coordinates $u^i=(u^1,u^2,u^3)$
          $$
\Sv_{\rm surf}={1\over 2}\sqrt{\det g}\epsilon_{ikm}du^i\wedge du^k g^{mn}
                 {\p\over \p u^n}\,,
           $$
where $g_{ik}$ is Euclidean metric in coordinates $u^i$.

 This vector-valued form can be considered in arbitrary Riemannian manifold,
but we can consider integral of the form (or any form $p(\r)\Sv$
 over surface only if operation of vectors transport is well-defined
(in spaces with absolute parallelism).

{\it Integral of vector-valued forms over surface cannot be well-defined
in an arbitrary Riemannian space.}

On the other hand for vector valued forms one can consider functionals
     $$
 \Psi_M(\sigma)=\int_M\langle\wv,\s\rangle
     $$
on $1$-forms $\sigma$ which are well-defined since an
integrand $\langle\wv,\sigma\rangle$
is number-valued differential form.
($\langle\,\,.\,\,\rangle$ is contraction of $1$-form 
with vector (covector with vector)).
  In the special case of Euclidean space one can consider 
just constant $1$-forms.  This is a special case which was consider
above when we calculated integrals using ``standard tricks''.

Now we consider  functionals for integrals (1) and (3).
 
\m

\centerline 
{\sl Functional $\Psi_M(\sigma)$ for integral $\int_M p(\r)d\ss$.}

\smallskip


      $$\int_M p(\r)d\ss\longrightarrow\,\,\,
 \Psi_{M}(\s)=\int_{M} \langle p(\r)\Sv,\s\rangle
         $$
LHS is well-defined in Euclidean space. RHS is well defined 
in arbitrary Riemannian manifold.
Using previous calculations  we see that the 
integrand is differential $2$-form such that its value on
arbitrary two vectors $\v_1,\v_2$ tangent to $M=\p d$ is equal to
          $$
 \langle\Sv,\s\rangle(\v_1,\v_2)=\langle\Sv(\v_1,\v_2),\s\rangle=
       \s(\v_1\times\v_2)
$$
Using formulae for vector-valued form $\Sv$ 
we see that $2$-form $\langle\Sv,\s\rangle$ is equal to
              $$
\langle\Sv,\s\rangle=\sigma_z dx\wedge dy+
       \sigma_y dz\wedge dx+\sigma_z dx\wedge dy\,,
              $$
in Euclidean space (in Cartesian coordinates),
and for arbitrary Riemannian manifold
          $$
     \Sv_{\rm surf}=
{1\over 2}\sqrt{\det g}\epsilon_{ikm}du^i\wedge du^k g^{mn}\s_n\,,
           $$
(Here $\sigma-\sigma_i(x)dx^i$.)
Using Stokes Theorem and formulae above we see that
in the case if $M=\p D$ then
   $$
\Psi_{\p D}(\sigma)=\int_{\p D}\langle p(\r)\Sv,\s\rangle=
\int_D\s(\grad p)dV+\int p\div \xi_\s\,,
   $$
where $\xi_\s=g^{ik}\s_k$ is vector 
field corresponding to $1$-form $\s\colon$.

In particular if $\s=d\varphi$ is an exact form then
   $$
\Psi_{\p D}(\sigma=\varphi)=
\int_{\p D}\langle\Sv,d\varphi\rangle=
\int_D\s(\grad p)dV+\int p\Delta\varphi\,,
   $$
$\Delta$ is Laplace-Beltrami operator: 
    $$
 \Delta\varphi=
 {1\over \sqrt{}}{\p\over \p x^m}\left(
  \sqrt{\det g}g^{mn}{\p\varphi\over \p x^n}
     \right)
       $$
This is generalisation of integral (1). In the case 
if $\varphi$ is linear polynomial
in $\E^3$ we come to standard Archimedes's integral. 
It is interesting to look at this formula
in the case if $\varphi$ is an arbitrary harmonic function.

\m

\centerline
{\sl Functional $\Psi_M(\sigma)$for integral 
$\int_M \grad p(\r)\times d\ss$.}


Now consider generalisation of example (3).

We come to the following functional on $1$-forms:
      $$
\int_M \grad p(\r)\times d\ss\longrightarrow\,\,\,
 \Psi_{\p D}(\s)=\int_{M} \langle\grad p(\r)\times\Sv,\s\rangle\,.
         $$
LHS is defined in Euclidean space, RHS is defined in an arbitrary 
Riemannian space. One can see that 
{\it RHS is well-defined 
for arbitrary manifold, it does not depend in fact on Riemannian structure.}
Indeed for
arbitrary two vectors $\v_1,\v_2$ tangent to $M=\p D$  we see that
      $$
\grad d p(\r)\times\Sv \left(\v_1,\v_2\right)=
 \grad p(\r)\times\left(\v_1\times\v_2\right)=
 \v_1 d p(\r)(\v_2)-\v_2d\varphi(\v_1)\,.
      $$
Hence we see that 
      $$ 
 \langle \grad p(\r)\times
 \Sv,\s\rangle(\v_1,\v_2)=\sigma\left(
    \v_1 d p(\r)(\v_2)-\v_2d\varphi\v_1
         \right)=\sigma\wedge d\varphi (\v_1,\v_2)\,.
$$
We come to answer:
      $$
 \Psi_{M}(\s)=\int_{M} \langle\grad p(\r)\times\Sv,\s\rangle
            =
\int_{M}\sigma\wedge dp\,.
      $$
Thus we see that rightly viewed integral (3) does not 
depend on any metric structure on the manifold.

Due to Stokes formula
     $$
\Psi_M(\s)=\int_M\s\wedge d p=\int_M d(\s\wedge p)-\int_M d\s\wedge p=
   \int_{\p M}\s\wedge p-\int_M d\s\wedge p\,. 
     $$
In particular if $1$-form $\s$ is closed the
     $$
\Psi_M(\s)=\int_{\p M}\s\wedge p\,.
     $$

\centerline { Le jeux en vaut la chandelle!}




\bye
