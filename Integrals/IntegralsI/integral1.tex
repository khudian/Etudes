
      \magnification=1200 %\baselineskip=14pt
\def\vare {\varepsilon}
\def\A {{\bf A}}
\def\FF {{\bf F}}
\def\a {\alpha}
\def\K {{\bf K}}
\def\s {{\sigma}}
\def\p{\partial}
\def\vare{{\varepsilon}}
\def\L {{\cal L}}
\def\G {{\Gamma}}
\def\C {{\bf C}}
\def\Z {{\bf Z}}
\def\U  {{\cal U}}
\def\R  {{\bf R}}
\def\E  {{\bf E}}
\def\l {\lambda}
\def\degree {{\bf {\rm degree}\,\,}}
\def \finish {${\,\,\vrule height1mm depth2mm width 8pt}$}
\def \m {\medskip}
\def\r {{\bf r}}
\def\v {{\bf v}}
\def\n {{\bf n}}
\def\b {{\bf b}}
\def\ss  {{\bf s }}
\def\e{{\bf e}}
\def\ac {{\bf a}}
\def \X   {{\bf X}}
\def \Y   {{\bf Y}}
\def \x   {{\bf x}}
\def \y   {{\bf y}}
\def\w {{\omega}} 
\def\wv {{\buildrel \rightarrow\over \omega}}

     \centerline {\bf Integration of vector-valued forms}


\medskip


\centerline {${\cal x} 1$ Archimedes principle and Gauss-
               Ostrogradsky law}


In mathematical physics we often have to use Stokes formula for
integrals which take vector values. Very good example is 
the deriving of Archimedes principle.
   If $p=p(\r)$ is a presure of liquid then the force 
 acting on the body is equal to
       $$
 {\FF}=   \oint_{\p D} p(\r)d \ss=\int_D \nabla p(\r)dV\,.
        \eqno (1a)
       $$ 
In the case if $p(\r)=consta$ then this integral is equal to zero.
In the case if $p(\r)=mgz$, then we come to
       $$
 {\FF}= \oint_{\p D} p(\r)d \ss=\int_D \nabla p(\r)dV
 =mgV=\hbox{weight of the liquid that the body displaces}\,.
        \eqno (1a)
       $$ 
This is standard Archimedes principle.



There is a simple truck to derive the formula (1). It is the following: 
take the scalar product of an arbitrary {\it constant} vector $\ac$   
with  integrand in (1) in surface integral in (1a).
  Then we come to the flux of vector field
$p\cdot\ac$ through surface $\p D$: 
       $$
 {\FF}\cdot\ac= \oint_{\p D} \left (p(\r)\cdot \ac\right)d \ss\,.
        \eqno (2a)
       $$ 
Apply Gauss-Ostogradsky law to this flux we come to 
       $$
 {\FF}\cdot\ac= \oint_{\p D} \left (p(\r)\cdot \ac\right)d \ss=
     \int {\rm div\,}\left(p(\r)\ac\right)=
    \ac\cdot\int_D \nabla p(\r)dV\,.
        \eqno (1a)
               $$
This relation implies (1a) since it
 is true for an arbitrary constant vector 
$\ac$.

Two words about essence of Gauss-Ostogradsky law for students
For student who know differential forms. 

Let $\K$ is an arbitrary vector field and respectively $M$ be an 
arbitrary surface in $\E^3$.  Then the flux of vector field $\K$
throug the surface $M$ is equal to integral of $2$-form
$\omega_\K=\Omega{\cal c}\K$ over surface $M$:
             $$
   \hbox {F`lux of $K$ via $M$}=\int_M \K d\ss=\int\Omega{\cal c}\K\,,
          \eqno (3a)
             $$ 
where $\Omega$ is a volume form.
Due to Stokes Theorem ($\int_{\p D} \w=\int_D d\w$ we have that
in the case if surface $M$ is a boundary, $M=\p D$ then
the integral (3a) is equal to the integral of $3$-form
$\Omega{\cal c}\K$ over body $D$:
         $$
\hbox {F`lux of $K$ via $\p D$}=
\int_{\p D} \K d\ss=\int_{\p D}\Omega{\cal c}\K=
    \int_D d\left(\Omega{\cal c}\K\right)\,.
         \eqno (3b)
           $$
Now the Cartan formula gives that
          $$
d\left(\Omega{\cal c}\K\right)=\L_\K\Omega=
  \left({\rm div}_\Omega\,\K\right)\Omega\,
        $$
and
        $$
\int_{\p D} \K d\ss=\int_{\p D}\Omega{\cal c}\K=
    \int_D d\left(\Omega{\cal c}\K\right)=\int
 \left({\rm div}_\Omega\,\K\right)\Omega\,
         \eqno (3b)
           $$
In coordinates: if volume form $\Omega=\rho dx\wedge dy\wedge dz$ then
           $$
{\rm div}_\Omega\,\K={d\left(\Omega{\cal c}\K\right)\over \Omega}=
          {
  {d\left(\rho dx\wedge dy\wedge dz {\cal c}
   \left(K_x\p_x+K_y\p_y+K_z\p_z\right)\right)}
      \over \rho dx\wedge dy\wedge dz}=
           $$
        $$
                {
 d\left(\rho\left(K_x dy\wedge dz-K_ydx\wedge dz+
 K_zdx\wedge dy\right)\right)
    \over \rho dx\wedge dy\wedge dz
        }=
       {1\over\rho}
           \left
            (
  {\p (\rho K_x)\over \p x}+
  {\p (\rho K_y)\over \p y}+
  {\p (\rho K_z)\over \p z}
            \right)=
       $$
        $$
  {\p  K_x\over \p x}+
  {\p  K_y\over \p y}+
  {\p \rho K_z)\over \p z}+
      K_x{\p\log \rho\p x}+
      K_y{\p\log \rho\p y}+
      K_z{\p\log \rho\p z}\,.
        $$

Now return to the integral (1). The derivation above is alright but it looks
little bit artificial.  We may avoid it introducind 
{\it vector-valued forms}.


\centerline {${\cal x} 2$ Oriented area---vector valued form }

 Let $\r=\r(\xi,\eta)$ be a local parameterisation of surface $M$ in $\E^3$.

Then surface element of $M$ is equal to  
                 $$
         d\s=|\r_\xi\times\r_\eta|d\xi\wedge d\eta=
       \sqrt{\r_\xi^2\r_\eta^2-\left(\r_\xi\cdot\r_\eta\right)^2}
                  d\xi\wedge d\eta=
                 $$
                 $$
              \sqrt{
          \left(x_\xi y_\eta-x_\eta y_\xi\right)^2+
          \left(x_\xi z_\eta-x_\eta z_\xi\right)^2+
          \left(z_\xi y_\eta-z_\eta y_\xi\right)^2
                }d\xi\wedge d\eta
               $$
A normal unit vector to the surface is equal to 
 $\n={\r_\xi\times\r_\eta\over |\r_\xi\times\r_\eta|}$ and vector 
surface element is equal to 
             $$
   d\ss=\n d\s=\left(\r_\xi\times \r_\eta\right) d\xi\wedge d\eta
             $$ 
We see that vector surface element is expressed trough vector valued $2$-form:
        $$
 \wv\colon\quad     \wv(\r_\xi,\r_\eta)=\r_\eta\times \r_\eta\,.
         $$
In Cartesian coordinates
        $$
      {\buildrel \rightarrow\over \omega}=
       dx\wedge dy\p_z-dx\wedge dz\p_y+dy\wedge dz\p_x 
        $$
In arbitrary coordinates $u^i=(u^1,u^2,u^3)$
          $$
\wv=\sqrt{\det g}\epsilon_{ikm}du^i\wedge du^k g^{mn}\p_m\,,
           $$
where $g_{ik}$ is Riemannina metric in coordinates $u^i$.

We have that
          $$
       \int_M d\ss=\int_M \n d\s=\int_M 
      {\buildrel \rightarrow\over \omega}\,.
          $$
In these notations we have immediately that 


01.10.2013

Yesterday Grisha Vekstein showed me the surface integral:
           $$
      \int_M (ds\times \nabla\varphi)
           \eqno (1)
                 $$
 He realises well that this integral is equal to zero. How to show it
properly?    

\bigskip

First of all naive approach.
Use the formulae of ''naive'' vector calculus:

  Take an arbitrary constant vector $\ac$. Then we have
                 $$
         \ac\cdot \int_M (d\ss \times \nabla\varphi)=
     \int_M d\ss \cdot (\nabla\varphi\times \ac)
              \eqno (1a)
             $$
If $M=\p D$ then due to Ostogradsky-Gauss Theorem we have that
     $$
         \ac\cdot \int_M (d\ss \times \nabla\varphi)=
     \int_M d\ss \cdot (\nabla\varphi\times \ac)=
         \int_D {\rm div\,}\left(\nabla\varphi\times \ac\right)=0\,,
                         $$

since $\ac$ is constant vector and
                  $$
{\rm div\,}\left(\nabla\varphi\times \ac\right)=
 {\rm rot\,}\nabla\varphi\times \ac=0
                  $$
Hence the integral (1) is equal to zero too if $M=\p D$.

Alright, the asnwer is equal to zero. Ifg this is the case in
arbitrary coordinates, then it has to be a nice formula,
 On the other 
formulae for gradient of function, vector product, e.t.c. 
in general heavily depend on additional structures such that
Riemannian metric, volume form, e.t.c. 

E.g. if $g$ is Riemannian metric in a given coordinates, then
for gradient of a function
               $
\nabla\varphi=g^{ik}{\p \varphi \over \p x^k}{\p \over \p x^i},\quad
              $
and for vector product of two vectors $\ac,\b$
$(\ac\times \b)_i=\sqrt{\det g}\epsilon_{ikm}a^kb^m$,
and respectively
              $$
(\ac\times \b)^i=\sqrt{\det g}g^{ij}\epsilon_{jkm}a^kb^m,
              $$
Now what object have to be considered instead constant vector
$\ac$, which obviously is meaningless in covariant approach.
Consider an arbitary covector $\omega_i$ (one-form 
${\bf\omega}=\omega_idx^i$)\footnote
{$^*$}{Why covector, not vector? Later we will see that
closed covector palys the role of constant vector.}. 

   We consider now the value of $1$-form $\omega$  on the integrand
 $d\ss\times \nabla\phi$ instead (1a) the integral
                 $$
       \int_M (d\ss \times \nabla\varphi)=
     \int_M d\ss \cdot (\nabla\varphi\times \ac)
              \eqno (1a)
             $$



The fact that 
the integral is equal to zero,
means that the integral in fact {\it does not depend } on metric.

One can see that the integral (1) can be viewed in the following way:

\bye
