

\magnification=1200
\baselineskip=14pt


\def\vare {\varepsilon}
\def\A {{\bf A}}
\def\B {{\bf B}}
\def\t {\tilde}
\def\a {\alpha}
\def\K {{\bf K}}
\def\k {{\bf k}}
\def\x {{\bf x}}
\def\y {{\bf y}}
\def\V {{\cal V}}
\def\L {{\cal L}}
\def\s {{\sigma}}
\def\S {{\Sigma}}
\def\s {{\sigma}}
\def\p{\partial}
\def\vare{{\varepsilon}}
\def\Q {{\bf Q}}
\def\O {{\bf O}}
\def\D {{\cal D}}
\def\G {{\Gamma}}
\def\C {{\bf C}}
\def\N {{\cal N}}
\def\Z {{\bf Z}}
\def\E {{\bf E}}
\def\U  {{\cal U}}
\def\H {{\cal H}}
\def\R  {{\bf R}}
\def\S  {{\bf S}}
\def\E  {{\bf E}}
\def\l {\lambda}
\def\degree {{\bf {\rm degree}\,\,}}
\def \finish {${\,\,\vrule height1mm depth2mm width 8pt}$}
\def \m {\medskip}
\def\p {\partial}
\def\r {{\bf r}}
\def\v {{\bf v}}
\def\n {{\bf n}}
\def\t {{\bf t}}
\def\b {{\bf b}}
\def\c {{\bf c }}
\def\e{{\bf e}}
\def\ac {{\bf a}}
\def \X   {{\bf X}}
\def \Y   {{\bf Y}}
\def \x   {{\bf x}}
\def \y   {{\bf y}}
\def \G{{\cal G}}
\def\w{\omega}
\def\finish {${\,\,\vrule height1mm depth2mm width 8pt}$}


  \centerline  {\bf Huigens principle}
  ({Here I will present my calculations based on memories and textbooks...})
  Consider differential

   Consider in $\E^n$ differential equation
         $$
       \cases
          {
         u_{tt}-\Delta u=0\cr
           u(t,\x)\big\vert_{t=0}=\varphi(\x)\cr
           {\p u(t,\x)\over \p t}\big\vert_{t=0}=\psi(xx)\cr
            }
           $$
    One can see that formal solution in Fourrier series will be
              $$
    u(\x,t)=\int e^{i\k (\x-\y)}\left(\varphi(y)\cos kt+
       \psi(y){\sin kt\over k}\right)d^n\k d^n \y\,,
           \eqno (*)
              $$
 $\k,\x$ are vectors, $k$ is modulus
of vector $\k\,k=|\k|$ (Here and later we often omit coefficients: e.g. in
the formula above we have omitted the coefficient $(2\pi)^{???}$).
(All integrals ar assumed to be generalised functions.)

   We calculate this integral and show that
for odd $n$ it implies Huigens.


Consider Green function
     $$
   G^{(0)}(\x-\y,t)=
    \int e^{i\k (\x-\y)}\cos kt d^n\k \,. 
     $$

(One can rewrite (*) in the following way:
 $u=G *\varphi+\p_t G *\psi$).

Now calculate it.  Ne can perform integration over sphere and radius.
Volume form in $\E^n$ in spherical coordiantes will be
$d^nk= d\Omega_{n}k^{n-1}dk$ and
  $$
   G^{(0)}(\x-\y,t)=
    \int e^{i\k (\x-\y)}\cos kt d^n\k= 
    \int e^{i\k (\x-\y)}\cos kt d\Omega_{n}k^{n-1}dk=
                $$
                $$ 
    \int_0^\infty\cos kt k^{n-1}
                     \left(
                      \left(
            \int_0^\pi e^{ik |\x-\y|\cos\theta}
                \sin^{n-2}\theta d\theta 
                  \right)
                 d\Omega_{n-1}
                  \right)
          dk= 
     $$
            $$ 
    w_{n-1}\int_0^\infty\cos kt k^{n-1}
                     \left(
            \int_{-1}^1 e^{ik |\x-\y|u}
                   (1-u^2)^{n-3\over 2}du 
                  \right)
          dk= 
     $$
Here $w_n$ is volume of $n-1$-dimensional unit sphere
  (unit sphere in $\E^n$),
            $$
 w_0=2,w_1=2\pi\,,w_2=4\pi\,\dots,  
w_n={2\pi^{n+1\over 2}\Gamma\left(n+1\over 2\right)}\,.
                $$
(It is funny to note that volume of $0$-dimensional sphere $\s_0=2$ 
is given by the general formula.)


Now we spcialize calculations for odd $n$. In the case if $n$ is odd, then
   $(1-u^2)^{n-3\over 2}$ is just polynomial on $u$. We have
that for odd $n$
          $$
G^{(0)}(\x-\y,t)=  
              w_{n-1}\int_0^\infty\cos kt k^{n-1}
                     \left(
            \int_{-1}^1 e^{ik |\x-\y|u}
                            \underbrace
                            {
                   (1-u^2)^{n-3\over 2}
                          }_{\hbox{polynomial}\, P_n(u)}
                       du 
                  \right)
          dk=
          $$
              $$
          w_{n-1}\int_0^\infty\cos kt k^{n-1}
                                   \left(
                     \left(
                                   P\left({d\over dz}\right)
            \int_{-1}^1 e^{zu}
                       du 
                  \right)\big\vert_{z=ik|\x-\y|}
                      \right)
          dk=
               $$  
                     $$
          (-1)^{n-1}w_{n-1}\left({d\over dt}\right)^{n-1}
                     \int_0^\infty\cos kt 
                                   \left(
                     \left(
                                   P\left({d\over dz}\right)
            \int_{-1}^1 e^{zu}
                       du 
                  \right)\big\vert_{z=ik|\x-\y|}
                      \right)
          dk\,.
               $$
One can say it in another way:

{\bf Statement} For odd $n$ the Green function belongs is generating by 
differential operator from the function
                $$
f=        \int_0^\infty\cos kt 
           {\sin k|\x-\y|\over k|\x-\y|}
          dk=
 {\pi\over 2}{\rm sgn\,}\left(|\x-\y|-t\right)+
 {\pi\over 2}{\rm sgn\,}\left(|\x-\y|+t\right)
                $$ 



First perform calculations for $n=3$:
        $$
 \hbox{for $n=3$}\,\,  
           G^{(0)}(\x-\y,t)=
    \int e^{i\k (\x-\y)}\cos kt d^3\k \,. 
     $$


   \medskip


  Preliminary calculation:  Calculate preliminary the average
of the function   $e^{i\k(\x-y)}$ over unit $n-1$-dimensional 
 sphere  (in $\k$ space.
 
   Function $\k\x$  is not constant on $n-1$ dimensional sphere $kx=1$,
but it is constant on $n-2$ dimensional spheres $\k\x\cos\theta=c$
($\theta$ is angle between $\k$ and $\x$ and $|c|\leq 1$).
  We have
    $$
F_n(kx)= <e^{i\k \x}>_{k=1}
  ={1\over \s_{n-1}}
  \int_{k=1} e^{i\k\x} d\Omega_{n-1}=
      {\s_{n-2}\over \s_{n-1}}
   \int_0^\pi e^{ix\cos\theta}\sin^{n-2}\theta d\theta
      $$
One can see that answers for even and odd will be different.
   For odd $n$ it is just elementary function, and 
   for even $n$ they are expressed via special function
   $j(x)=\int_0^\pi e^{ix\cos\theta}d\theta$.
   
In moe details.
 First consider special cases
(we ofen omit later all the coefficients....)
         $$
n=2, J(x)=F_2(x)=\s_0\int_0^{\pi} e^{ix\cos\varphi}d\varphi=
        \int_0^{2\pi} e^{ix\cos\varphi}d\varphi\,,
         $$ 
         $$
n=3, F_3(x)=\s_1\int_0^{\pi} e^{ix\cos\varphi}\sin \varphi d\varphi=
        2\pi\int_{-1}^{1} e^{ix u}du=2i{\sin x\over x}\,.
         $$ 
It is easy to see that the answer for $n=0$ produces all the answers
for even $n$ and the answer for $n=3$ produces all the answers for
odd $n$:


   One can see that all fnctions $F_(x)$ can be produced from function
$J(a)$ and $f(a)={\sin a\over a}$ by differentiation, e,g,
                     $$
      F_7(x)=\s_{5}
   \int_0^\pi e^{ix\cos\theta}\sin^{5}\theta d\theta=
   s_5\int_0^\pi e^{ix\cos\theta}\sin^{4}\theta d\cos\theta=
   s_5\int_0^\pi e^{ix u}(1-2u^2+u^4)du=
         $$
            $$
\left(1+2{d^2\over du^2}+4{d^4\over du^4}\right)
        \int_0^\pi e^{ix u}du=
2i\s_5\left(1+2{d^2\over dx^2}+4{d^4\over dx^4}\right)
        {\sin x\over x}
           $$

Now we return to the integral (*). Calculate it for odd $n$.
Using functions $F_n(a)$  which are averaging
of exponent over spere we come to
               $$
  u(t,\x)=C_n\int e^{i\k (\x-\y)}\left(\varphi(y)\cos kt+
       \psi(y){\sin kt\over k}\right)d^n\k d^n\y\,.
           =\int F_n(k|\x-\y|)\left(\varphi(y)\cos kt+
       \psi(y){\sin kt\over k}\right)k^{n-1}dk d^n \y=
              $$
We denote
             $$
     G^{(0)}_n(\x,\y,t)=
   \int F_n(k|\x-\y|)\left(\varphi(y)\cos kt+
       \psi(y){\sin kt\over k}\right)k^{n-1}dk =
             $$
We see that
        $$
   u(\x,t)=
    \int G(\x,\y,t)
    \left(\varphi(y)\cos kt+
       \psi(y){\sin kt\over k}\right)d^n\k d^n\y\
        $$
\bye
