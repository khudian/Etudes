\magnification=1200
\baselineskip=14pt
\def\vare {\varepsilon}
\def\A {{\bf A}}
\def\t {\tilde}
\def\a {\alpha}
\def\K {{\bf K}}
\def\N {{\bf N}}
\def\V {{\cal V}}
\def\s {{\sigma}}
\def\S {{\Sigma}}
\def\s {{\sigma}}
\def\p{\partial}
\def\vare{{\varepsilon}}
\def\Q {{\bf Q}}
\def\D {{\cal D}}
\def\G {{\Gamma}}
\def\C {{\bf C}}
\def\M {{\cal M}}
\def\Z {{\bf Z}}
\def\U  {{\cal U}}
\def\H {{\cal H}}
\def\R  {{\bf R}}
\def\E  {{\bf E}}
\def\l {\lambda}
\def\degree {{\bf {\rm degree}\,\,}}
\def \finish {${\,\,\vrule height1mm depth2mm width 8pt}$}
\def \m {\medskip}
\def\p {\partial}
\def\r {{\bf r}}
\def\v {{\bf v}}
\def\n {{\bf n}}
\def\t {{\bf t}}
\def\b {{\bf b}}
\def\e{{\bf e}}
\def\ac {{\bf a}}
\def \X   {{\bf X}}
\def \Y   {{\bf Y}}
\def \x   {{\bf x}}
\def \y   {{\bf y}}
\def\w {{\omega}}

\centerline  {\bf Funny integrals (Borwein integrals)}

{\it Yesterday my son David showed me the funny integrals 
on the web-blog  of Steven Lundsburg
(professor of Economics, Rochester University), cite:

http://www.thebigquestions.com/2012/03/26/loose-ends/.

I enjoyed them so much! Hope you will enjoy too}


\bigskip
   Consider integrals:
         $$
     A_n=\int_0^\infty \prod_{k=1}^n
 {\sin {x\over 2k-1}\over {x\over 2k-1}}dx,   
         $$
         i.e.
        $$
   A_1=     \int_0^\infty {\sin {x }\over {x }}dx\,,\quad
  A_2=\int_0^\infty {\sin {x }\over {x }}
  {\sin {x\over 3}\over {x\over 3}}dx\,,\quad
  A_3=\int_0^\infty {\sin {x }\over {x }}
  {\sin {x\over 3}\over {x\over 3}}
   {\sin {x\over 5}\over {x\over 5}}dx\,,\dots
\eqno (1)   
        $$
It really looks surprising but the sequence  $\{A_1,A_2,A_3,\dots\}$ looks in the following way:
                  $$
            \left\{{\pi\over 2},{\pi\over 2},{\pi\over 2},{\pi\over 2},
            {\pi\over 2},{\pi\over 2},{\pi\over 2},...\right\}
                  $$         
     Yes, the eighth term is not equal to ${\pi\over 2}$. 
  As author of the blog claims it is equal to
                      $$
         A_8={467807924713440738696537864469\over 935615849440640907310521750000}\pi\quad (???!!!)         
                      $$

This is based on the paper [1].

Shortly the idea of calculating integrals 
(as it was explained in the article [1]) is based on the remark
 that Fourier image
of the function ${\sin ax \over x}$ 
 is characteristic function of the interval $[-a,a]$:
          $$
    {1\over \sqrt 2\pi}\int_{-\infty}^{\infty}e^{ix\xi} {\sin ax \over x}dx=
       \cases{1\,\,\hbox {if $|\xi|<a$}\cr 0 \,\,\hbox {if $|\xi|>a$}\cr}.
          $$
 Now using the fact that
 Fourier image of product is convolution one can reduce the problem of calculating these integrals to calculating integrals of functions related with
 convolutions of characteristic functions of intervals.


\centerline {${\cal x}$ \bf Calculations without Fourier transformation}





At the end of the article [1] authors consider 
 another I much more elementary method which 
is found on the following identity:  
           $$
     \int_0^\infty
   \prod_{k=1}^n{\sin a_k x\over x}
   \prod_{i=1}^m\cos c_i x 
      {\sin x\over x}dx=
    {\pi\over 2}\prod_{k=1}^n a_k\,.
        \eqno (2)
           $$
in the case if
       $$
   \sum |a_k|+\sum |c_i|\leq 1.
            \eqno (2a)
       $$

This identity goes to the work [2] of C. Stormer in 1885 
(see the detailes in [1].)

This identity implies why the first seven terms in (1)
are equal to ${\pi\over 2}$.
Indeed one may read the identity in the following way:
for positive integers ${a_1,a_2,\dots,a_k}$
            $$
     \int_0^\infty
   \prod_{k=1}^n{\sin a_k x\over a_k x}
      {\sin x\over x}dx=
    {\pi\over 2}\,\,
     \hbox{in the case if
       $
   \sum a_k\leq 1
         $}
      \eqno (3)
            $$

 Now we see that integral for (1) 
           $$
     A_n=\int_0^\infty \prod_{k=1}^n
 {\sin {x\over 2k-1}\over {x\over 2k-1}}dx={\pi\over 2}
              \eqno (4)
            $$
            $$ 
{\rm if}\,\, {1\over 3}+{1\over 5}+\dots+{1\over 2n-1}\leq 1
           \eqno (4a)
           $$
This is the case for first seven integrals in (1), but this is not the case
for the integral $A_8$.
Indeed 
    $$
 {1\over 3}+{1\over 5}+{1\over 7}+{1\over 9}+{1\over 11}+{1\over 13}=
     $$
       $$
\left({1\over 3}+{1\over 5}+{1\over 9}\right)+
\left({1\over 7}+{1\over 11}+{1\over 13}\right)=
 {15+9+5\over 45}+{143+77+91\over 1001}=
    {29\over 45}+{311\over 1001}={43024\over 45045}<1
        $$
but
${1\over 3}+{1\over 5}+{1\over 7}+{1\over 9}+{1\over 11}+{1\over 13}+
{1\over 15}=$
       $$
{43024\over 45045}+{1\over 15}={43024+3003\over 45045}=
     {46027\over 45045}>1.
      $$

Nevertheless we still can calculate $A_8$ an come to monstrous answer
above using integrals below.  

\m

  A proof of indentity (2) and calculations below are
 founded on the following 
basic identities\footnote{$^*$}{See [1] for details.The proof below 
  is a variation on...}:
       $$
   {\sin a_k x\over x}\cos c_i x={1\over 2}{\sin(a_k+c_i)x\over x}+
     {1\over 2}{\sin(a_k-c_i)x\over x}\,,
       $$
and
      $$
    {\sin a_k x\over x}=\int_0^{a_k}\cos t x dt.
      $$
It follows from these identities that   
the integrand in relation (4) can be represented by the following
 integral:
           $$
   \prod_{k=1}^n{\sin a_k x\over x}
   \prod_{i=1}^m\cos c_i x 
      {\sin x\over x}=
         $$
         $$
       \int\dots\int_{
       \matrix 
          {
         0\leq t_i\leq a_i, \cr
           (i=1,\dots,k)\cr
           }
          }
       {\sum_{\pm}
  \sin (x\pm t_1x\pm\dots\pm t_nx\pm c_1x\pm\dots+c_mx)\over 2^{n+m}}
         dt_1\dots dt_n\,.
          \eqno (5)
           $$
(Here the summation goes over all combinations of sings $\pm$.)

Hence due to the fact that
        $$
    \int_0^\infty{\sin ax\over x}dx=
        \cases{
 1\,\,{\rm if}\,\, a>0\cr -1\,\,{\rm if}\,\, a<0\cr
              }\,.
        $$
we see that if condition (2a) is obeyed then the integral (2)
equals to the integral of constant $\pi\over 2$ over rectangular
polyhedron with volume $a_1\dots a_m$. This implies the identity (2). 



\m

For example calculate  
$\int_0^\infty {\sin ax\over x}{\sin x\over x}dx$.
    We have that
                $$
    \int_0^\infty \cos ax{\sin x\over x}dx=
    {1\over 2} \int_0^\infty {\sin (1+a)x+\sin (1-a)x\over x}dx={\pi\over 2}
     \,\,{\rm if}\,\, |a|\leq 1.
          $$
 Thus
               $$
\int_0^\infty {\sin ax\over x}{\sin x\over x}dx=
 \int_0^a dt\int \int_0^\infty \cos tx {\sin x\over x}dx=
 \int_0^a dt {\pi\over 2}dx=a{\pi\over 2}\,.
         $$
We see that $\int_0^\infty {\sin ax\over ax}{\sin x\over x}dx={\pi\over 2}$
if $|a|\leq 1$ (It is equal to ${\pi\over 2a}$) for $|a|\geq 1$.

In the similar way we may calculate  
$\int_0^\infty{\sin a_1 x \sin a_2 x\sin x\over x^3}dx$
for positive $a_1,a_2$ such that $a_1+a_2\leq 1$
          $$
\int_0^\infty{\sin a_1x \sin a_2 x\sin x\over x^3}dx=
\int\in\int_{
                    \matrix 
                      { 
          0\leq t_1\leq a_1    \cr 
          0\leq t_2\leq a_2   \cr 
          0\leq x  \leq \infty \cr
                      }
                     }
     \left(
   {\sum_{\pm} sin(x\pm t_1x\pm t_2x)\over 8}
      \right) 
       dt_1 dt_2dx\,.
          $$
The integral of integrand over $x$ equals to ${\pi\over 2}$ since
$1\pm t_1\pm t_1\geq 0$. Hence integral equals to $a_1a_2{\pi\over 2}$.

In the case if condition (2a) is not obeyed we still may use these methods.

\bigskip
{\bf Remark}  When I wrote this text I knew that
in Wikipedia these integrals are called Borwein integrals.....



\centerline       {\bf References}

[1] David Borwein and   Jonathan Borwein.
 {\it Some remarkable properties of  sinc and related integrals.} 
 The Ramanujan Journal 5(2001), no, 1, pp.73---89

See also 
    http://www.thebigquestions.com/borweinintegrals.pdf.
(1991)

\m

[2a]C. Stormer, {\it Sur generalisation de la formulae 
 ${\phi\over 2}={\sin \phi\over 1}-{\sin \phi\over 2}+
{\sin 3\phi\over 3}-\dots$}, Acta Math. {\bf 19} (1885), pp. 341-350



\bigskip


 $\qquad$$\qquad$$\qquad$$\qquad$$\qquad$$\qquad$           (11.04.12)
 \bye       
 
   
