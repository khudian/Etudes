

  % \magnification=1200
  % \baselineskip 17 pt

\def\V {{\cal V}}
\def\s {{\sigma}}
\def\Q {{\bf Q}}
\def\D {{\cal D}}
\def\G {{\Gamma}}
\def\C {{\bf C}}
\def\M {{\cal M}}
\def\Z {{\bf Z}}
\def\U  {{\cal U}}
\def\H {{\cal H}}
\def\R  {{\bf R}}
\def\l {\lambda}
\def\p {\partial}
\def\r {{\bf r}}
\def\v {{\bf v}}
\def\n {{\bf n}}
\def\t {{\bf t}}
\def\b {{\bf b}}
\def\ac {{\bf a}}
\def \X   {{\bf X}}
\def \Y   {{\bf Y}}
\def \E   {{\bf E}}
\def\vare {\varepsilon}
\def\A {{\bf A}}
\def\t {\tilde}
\def\a {\alpha}
\def\K {{\bf K}}
\def\N {{\bf N}}
\def\V {{\cal V}}
\def\s {{\sigma}}
\def\S {{\Sigma}}
\def\s {{\sigma}}
\def\p{\partial}
\def\vare{{\varepsilon}}
\def\Q {{\bf Q}}
\def\D {{\cal D}}
\def\G {{\Gamma}}
\def\C {{\bf C}}
\def\M {{\cal M}}
\def\Z {{\bf Z}}
\def\U  {{\cal U}}
\def\H {{\cal H}}
\def\R  {{\bf R}}
\def\E  {{\bf E}}
\def\l {\lambda}
\def\degree {{\bf {\rm degree}\,\,}}
\def \finish {${\,\,\vrule height1mm depth2mm width 8pt}$}
\def \m {\medskip}
\def\p {\partial}
\def\r {{\bf r}}
\def\v {{\bf v}}
\def\n {{\bf n}}
\def\t {{\bf t}}
\def\b {{\bf b}}
\def\e{{\bf e}}
\def\ac {{\bf a}}
\def \X   {{\bf X}}
\def \Y   {{\bf Y}}
\def \x   {{\bf x}}
\def \y   {{\bf y}}
\def\w {{\omega}}
\def\A{{\bf A}}
\def\B{{\bf B}}
\def\pt{{\bf p}}

{\it
In the coursework in Riemannian geometry appeared an integral.
Its straighforward calculations is interesting...}

\centerline {\bf Appendix}

Straightforward calculations of the lenght of the curve $C'$
lead to the following integral:
           $$
I(z=)\int_0^{2\pi} {d\varphi\over 1-z\cos \varphi}\,,, (|z|<1)
              $$
  Here I present two different ways to calcualte this integral.

{\it First way}

  This integral can be calculated explicitly, the answer is beautiful:
           $$
I(z)=\int_0^{2\pi} {d\varphi\over 1-z\cos \varphi}={2\pi\over \sqrt {1-z^2}}\,.
              $$
Do it.
   One can see that for $|z|<1$,
                    $$
I(z)=\int_0^{2\pi} {d\varphi\over 1-z\cos \varphi}=
   \int_0^{2\pi} \left(1+z\cos\varphi+z^2\cos^2\varphi+\dots\right)d\varphi=
                    $$ 
                    $$
   \int_0^{2\pi} \left(\sum z^n\cos ^n\varphi\right)d\varphi=
  \sum_{n=0}^\infty c_nz^n\,, {\rm where}  \,\, c_n=
\int_0^{2\pi}\cos^n\varphi d\varphi\,.
                    $$ 
Calculate $c_n$:
                     $$
c_n=\int_0^{2\pi}\cos^n\varphi d\varphi=
\int_0^{2\pi}\left(e^{i\varphi}+e^{-i\varphi}\over 2\right)^nd\varphi=
               \cases
                      {
2\pi {C^{k}_{2k}\over 2^k} \hbox {for $n=2k$}\cr
    0  \,\, \hbox { for $n=2k+1$}\cr
                      }\,.
                     $$
since $(a+b)^n=\sum_jC^j_n a^j b^{n-j}$, and 
$\int_0^{2\pi} e^{ik\varphi}d\varphi=0$ if $k\not=0$.
Hence we have that
                    $$
I(z)=\int_0^{2\pi} {d\varphi\over 1-z\cos \varphi}=
2\pi\sum_{n=0}^\infty c_nz^n=
\sum_{n=0}^\infty z^n=\sum {C^n_{2n}z^{2n}\over 2^n}
                ={2\pi\over \sqrt {1-z^2}}\,.
                    $$ 

\m

{\bf Remark} One can see that 
   $$P(z)=\sum C^n_{2n}z^{n}
                ={2\pi\over \sqrt {1-4z}}\$
         $$
Looking at this funtion it is 
difficult to avoid temptation to write something like:
           $$
\sum C^n_{2n}=P(1)=....={2\p\over \sqrt{-3}}\, ???!
           $$

{\it Second way}
                  $$
I(z)=\int_0^{2\pi}{d\varphi\over 1-z\cos\varphi}=
\int_0^{2\pi}{\sin\varphi d\varphi\over \sin\varphi(1-z\cos\varphi)}=
\int_0^{2\pi}{d\cos\varphi\over (\sqrt{1-\cos^2\varphi})(1-z\cos\varphi)}=
                     $$
                     $$
\int_0^1{dw\over (\sqrt{1-w^2})(1-zw)}=
{1\over2}\int_0^1{dw\over (\sqrt{1-w^2})(1-zw)}\,.
                     $$
Now consider an integrand, the  function 
$F(w)={1\over {\sqrt {1-w^2}}(1-zw)}$
in the plane excluding the neigborhood of the interval
$[-1,1]$ which connects the branching points of this function.
   We take the following branch $F'(w)$ of this function
such that it is holomorphic
 function in the plane without neighborhood of the segment
\footnote{$^{1}$}
{If $P=w=u+iv$ is an arbitrary point of complex plane,
  $A=-1$ and $B=1$,
 and $\varphi$ is an angle between $AB$ and $AP$ (anti-clock wise),
 and $\psi$ is an angle between $BA$ and $BP$ (anti-clock wise),
  then  $F'=\sqrt {|AP||BP|}e^{i{\phi+\psi\over 2}}$.
  In particular $F(w)=i{\sqrt{w^2-1}}$ if $w$ is  a real number which is greater
than $1$.}
Now we note that the integral of function over the great circle tends to zero.
   The function $F'(w)$ has a pole at the point $w={1\over z}$.
  Hence if $|z|<1$ $F(z')$ is holomorphic function in plane without
 noiborhood of interval $AB$. We have:
                      $$
  0= \int_{C_1} F'(w)dw+\int_{C_2}F(w)dw=I(z)-
         {1\over z}\int_{C_2}{1\over i\sqrt{w^2-1}\left(w-{1\over z}\right)}=
                     $$
                     $$
           I(z)-{2\pi i\over z}\left({1\over i\sqrt{w^2-1}}\right)
           \big\vert_{w={1\over z}}=I(z)-{2\pi\over \sqrt{1-z^2}}
\Rightarrow   I(z)={2\pi\over \sqrt{1-z^2}}\,.
                      $$ 
where we denote by $C_1$ the closed curve around the interval $AB$,
and $C_2$ the circle of small radius around
      


\bye


