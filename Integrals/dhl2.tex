% this text I wrote in the second half of October 2013.

      \magnification=1200 %\baselineskip=14pt
\def\vare {\varepsilon}
\def\A {{\bf A}}
\def\FF {{\bf F}}
\def\a {\alpha}
\def\K {{\bf K}}
\def\s {{\sigma}}
\def\p{\partial}
\def\vare{{\varepsilon}}
\def\L {{\cal L}}
\def\G {{\Gamma}}
\def\C {{\bf C}}
\def\Z {{\bf Z}}
\def\U  {{\cal U}}
\def\R  {{\bf R}}
\def\E  {{\bf E}}
\def\l {\lambda}
\def\degree {{\bf {\rm degree}\,\,}}
\def \finish {${\,\,\vrule height1mm depth2mm width 8pt}$}
\def \m {\medskip}
\def\r {{\bf r}}
\def\v {{\bf v}}
\def\n {{\bf n}}
\def\b {{\bf b}}
\def\ss  {{\bf s }}
\def\e{{\bf e}}
\def\ac {{\bf a}}
\def \X   {{\bf X}}
\def \Y   {{\bf Y}}
\def \x   {{\bf x}}
\def \y   {{\bf y}}
\def\w {{\omega}} 
\def\wv {{\buildrel \rightarrow\over \omega}}

\def\K{{\bf K}}
\def\locus {\hbox{locus of $\K$}}


\centerline {\bf On Duistermaat-Heckman localisation Theorem II}

  \bigskip

$\qquad$ 28 October 2013

\m

{\it
   Here we will give a formulation (with supermathematics flavor),
the proof and concrete calculations
for DH (Dustermaat-Heckman) localisation formula. This etude is based
on the paper of  Zaboronsky and Schwarz [1] and my etude
[4] (see the previous etude on this topic) which was based on calculations 
of A.Belavin.) It is interesting also to note papers [3] and [4].
}

      \medskip
{\sl
  $\qquad\qquad$$\qquad\qquad$$\qquad\qquad$ If a form, 
  is invariant with respect to odd vector field


$\qquad\qquad$$\qquad\qquad$$\qquad\qquad$  
  $Q=d\circ \iota_\K+\iota_\K\circ d=\sqrt {\L_\K}$ 
   where $\L_\K$ is Lie derivative 

$\qquad\qquad$$\qquad\qquad$$\qquad\qquad$ with respect
 to $U(1)$-vector field $\K$, then integral of this 

$\qquad\qquad$$\qquad\qquad$$\qquad\qquad$ form over manifold
  $M$ is localised at the zero locus 

$\qquad\qquad$$\qquad\qquad$$\qquad\qquad$ of vector fied $K$. This is the
 meaning.

$\qquad\qquad$$\qquad\qquad$$\qquad\qquad$ 
of Dustermaat-Heckman localisation formula.

}
 $$ $$

   During this text it will always be assumed that
$M$ is compact manifold  and $\K$ is compact vector field on it, 
i.e. vector field which generates $U(1)$
action. We denote by $$
   Q_\K=d+\iota_\K\,,\quad \hbox
 {in ``supernotations''}\,\,
Q_\K=\xi^i{\p\over \p x^i}+K^i(x){\p \over \p \xi^i}\,,
              $$
where $x^i,\xi^i=dx^i$ are local coordiantes on $\Pi TM$.

Odd vector field $Q_\K$ is a ``square root'' of a Lie derivative 
$\L_K=\iota_\K\circ d+d\circ \iota_\K$:
               $$
\L_\K=Q_\K^2=\left(\xi^i{\p\over \p x^i}+K^i(x){\p \over \p \xi^i}\right)^2=
       K^i(x){\p \over \p x^i}+\xi^r{\p K^i\over \p \xi^r}{\p\over \p \xi^i}\,,
              \eqno (1) 
              $$
or in classical notations
                   $$
\L_\K=Q_\K^2=(d+\iota_k)^2=d\circ \iota_\K+\iota_\K\circ d\,.
                   $$
We formulate the following version of DH localisation theorem:


{\it 

{\bf Theorem}   Let $H= H(x,dx)$ be a $Q_\K$-invatriant form on $M$,
 i.e.
          $$
dH+\iota_\K H=0\,.
          \eqno (2)
           $$
Then the integral $\int_M H(x,dx)$ is localised at locus of $K$.
This means follows: let $U_K$ be an arbitrary $U(1)$-invariant\footnote{$^*$}
{the condition to be $U(1)$-invariant may be is not necessary. 
We will use it for constructing $U(1)$-ivariant 
partition of unity. This condition is absent in the paper [1].}
 tubular neighborhood of 
locus of $K$ and let $G_U=G_U(x,dx)$ be a $Q_\K$-invariant 
form such that it is equal to $1$
at the locus of vector field $\K$ and it  vanishes out of neighborhood
$U_\K$: 
            $$
Q_\K G_U=0,\,({\rm i.e.\,} dG_U+\iota_\K G_U=0), 
  \quad G_U\big\vert_{\locus}=1,\quad G_U\big\vert_{M\backslash U_K}=0\,. 
\eqno(3)
            $$ 
(Bump-form of zero locus of $\K$.)
(We will prove the existence of such a bump-form)

\medskip

Then 

                $$
   \int_MH=\int_M HG_U\,. 
            \eqno (4)
                $$

}
\medskip

{\bf Example} Let $M$ be a symplectic manifold, i.e.
 non-degenerate closed two-form  $\Omega$ is defined on $M$
($M$ is even-dimensional). Let $h=h(x)$ be a Hamiltonian such that
its Hamiltonian vector field $D_h$ ($D_h\colon\quad 
\iota_{_{D_h}}\Omega=-dh$) is compact, i.e. it 
defines $U(1)$ action. 
  Consider the form
        $$
   H(x,dx)=\exp i\left(\Omega+h\right)\,.
      \eqno (5)
        $$
This form is $Q_\K$-invariant. Indeed since $K$ is hamiltonian vector field
$D_h$ hence
       $$
\iota_\K \Omega+dh=0.{\rm i.e.\,} Q_\K(h+\Omega)=0\Rightarrow
   Q_\K H=0\,.
       $$
Then 
     $$
\int H(x,dx)=\int \exp i\left(\Omega+h\right)={i^n\over n!}
\int \exp i h\underbrace{\Omega\wedge\dots\wedge\Omega}_{\hbox{$n$ times}}
     $$
is localised.


{\bf Remark 1}  Note that this example is a basic example in classical 
background.  Compact vector field $\K$ appears naturally in this example
as hamiltonian vector field of Hamiltonin $h$.  In 
Schwarz-Zaboronsky approach the vector field $\K$ appears
independetly without symplectic structure and 
Hamiltonian. In this approach the localisation formula is stated
for a function $H(x,dx)$ on $\Pi TM$ (sum of differential forms of different
orders). The classical condition that sum of differential forms
is invariant with respect to equivariant differential $d_\K=d+\iota_\K$
becomes the condition  that ``function''
 \footnote{$^2$} 
{$H(x,dx)$ is non-homogeneous differential form on $M$. It is a
function on tangent bundle $\Pi TM$ with reversed parity of fibers.}

$H(x,dx)$ is invariant with respect to odd vector field
$Q_\K$ which is the square root of Lie derivative along the vector field
$\K$ $\colon Q_K^2=\L_\K$.


 

{\bf Remark 2} partiion of unity for form...


{\sl Proof of Theorem}
First we prove the existence of a form $G_U=G_U(x,dx)$ which 
obeys the condition (3), then we will show that an arbitrary 
 $Q_\K$-invariant ``function''
(form) which obeys conditions (3) yields the localisation formula (4).

Using partiion of unity arguments consider a function
$F=F(x)$ such that 
            $$
F(x)\big\vert_{\locus}=0,
 \quad F(x)\big\vert_{M\backslash U_K}=1\,. 
\eqno(6)
            $$ 
(We may consider partiion of unity which is subordinate
 to covering $V_1\cup V_2$,
where $V_1=U_\K$ and $V_2=M\backslash \hbox{locus of $K$}$.


We may assume that $F(x)$ is $\K$-invariant function.
(Here we use the $U(1)$-ivariance of neighborhood of 
locus (see the footnote.)).

   
 It is useful to  consider the differential $1$-form
       $$
    \w_\K\colon \w_\K(\x)\langle \K,,\x\rangle\,,\w_i=g_{im}K^mdx^i\,,
         \eqno (7)
      $$
where $\langle \K,,\x\rangle$ is $U(1)$-invariant Riemannian
metric on $M$. Now we are ready to define form $G_U$ which obeys
the condition (3):
            $$
   G_U(x,dx)=1-Q_\K\left({\w_\K(x,dx)\over Q_\K \w_\K}F(x)\right)
                \eqno(8)
            $$
Straightforward calculations show that this function obeys conditions (3).
Indeed $F(x)=0$ if $x$ belongs to locus of $K$
(and in a vicinity of the locus), hence the right hand side 
of equation (8) is well-defined on the locus of $\K$, where
the form $\w_\K$ is not defined. Using the fact that
$Q_\K\left({\w_\K(x,dx)\over Q_\K \w_\K}\right)=1$ (if $\K(x)\not=0$)
we immediately come to the condition (3).

 Let $\tilde G_U=\tilde G_U(x,dx)$ be an arbitrary $Q_\K$-invariant
 form which obeys the condition
(3). Then consider the difference $L(x,dx)=\tilde G_U-G_U$.
The form  $L(x,dx)$ is $Q_\K$-invariant and it is equal to $0$ at the locus
of $K$, Hence 
              $$
   L(x,dx)=Q_\K\left({\w_\K(x,dx)\over Q_\K \w_\K}L(x,dx)\right)\,.
        \eqno (9)
             $$
Thus we see that $Q_\K$-invariatn  form $G_U(x,dx)$ in (8)
which obeys the condition (3) as well as an arbitrary
$Q_\K$-invariatn  form $\tilde G_U(x,dx)$ 
which obeys the condtion (3) obey the condition
that
      $$
    \matrix
           {
   G_U(x,dx)=1+Q_\K(...)\cr
   \tilde G_U(x,dx)=1+Q_\K(...)\cr
      }
       $$
  This immediately implies the relation (4):
            $$
\int_M H(x,dx)G_U(x,dx)=\int_M H(x,dx)(1+Q_\K(\dots))=
\int_M H(x,dx)
       $$
 since $\int_M Q_\K (\dots)=0$\footnote{$^{**}$}{since $Q_K=d+\iota_K$,
 and $\iota_K \w$ 'does not contain' top form.  
This follows also from the vanishing of divergence of
odd vector field $Q_\K$ with respect to canonical volume form
in $\Pi TM$}\finish


\m

\centerline {\bf Concrete calculations}

   Now based on the Theorem we present concrete calculations.

Let $H=H(x,dx)$ be $Q_\K$ invariant form and 
locus (zero locus)
of $U(1)$-invariant 
vector field $\K$ is a set $\{x_i\}$ of isolated points.

Using bump-form $G_U$, the form which vanishes out vicinites of
points $\{x_i\}$ (see the considerations above) we calculate
$\int_M H(x,dx)$.


 {\bf Lemma}  For an arbitrary $Q_\K$-invariant form $H(x,dx)$ the integral
            $$
   Z(t)=\int H(x,dx)e^{itQ_\K(\w_\K)}\,, 
            $$
where $\w_\K$ is $U(1)$-invariant form 
(7) does not depend on $t$.

  Proof:
              $$
 {dZ(t)\over dt}=
           i\int_M 
  H(x,dx)Q_\K
   \w_\K
  e^{itQ_\K(\w_\K)}=
    i\int_M Q_\K
          \left(
  H(x,dx)e^{itQ_\K(\w_\K)}
          \right)=0\,.
              $$
Now using lemma and bump-form which localises integrand in vicinity of
points $\{x_i\}$ we come to 
           $$
\int_M H(x,dx)=\int_M H(x,dx)G_U(x,dx)=
\left(
\int_M H(x,dx)G_U(x,dx)e^{itQ_\K(\w_\K)}\right)
\big\vert_{t=0}
             $$
             $$
              =
\left(\int_M H(x,dx)G_U(x,dx)e^{itQ_\K(\w_\K)}\right)
\big\vert_{t\to\infty}
           $$ 
Using method of stationary phase and assuming that $d\w$ is non-degenerate
at locus of $\K$  we calculate the last integral
(see [4]) and come to the answer
           $$
\int_M H(x,dx)=
              =
\left(\int_M H(x,dx)G_U(x,dx)e^{itQ_\K(\w_\K)}\right)
\big\vert_{t\to\infty}=
\sum_{x_i}{i^n\over n!}{H(x,dx)\big\vert_{x_i}\over 
  \sqrt  {{\p K\over \p x}\big\vert_{x_i}} }
           $$ 
If $H(x,dx)\big\vert_{x_i}=H_0(x_i)$, where
$H(x,dx)=H_0(x)+H_1(x,dx)+\dots$ is a sum of differential forms. 
 
{\bf Remark} It is crucial for calculation that $d\w$ is non-degenerate
at zero locus of $\K$. Is it an additional condition, or it 
follows from the fact that vector field $\K$ generates $U(1)$-action
(and $M$ is even-dimensional manifold)?
 On one hand I cannot prove this completely, on the 
other hand natural counterexamples deal with non-compact vector field.


$$ $$

\centerline {\bf References}

 [1] Albert Schwarz and Oleg Zaboronsky. 
 {\it Supersymmetry and localisation}. arXiv: hep-th/951112v1

  \m

 [2]  A. Nersessian {\it Antibrackets and non-Abelian 
     equivariant cohomology}
   arXix: hep-th/951081

\m
        
 [3] {\it On the Duistermaat-Heckman localisation 
 formula and Integrable systems} 
arXiv: hep-th/9402041v1

\m


 [4] homepage: maths.manchester.ac.uk/khudian/Etudes/Geometry/Dustermaat-Heckman  localisation formula. 
{\it Etude based on the fragment of the lecture of 
A.Belavin in Bialoveza, summer 2012.} 

\bye
