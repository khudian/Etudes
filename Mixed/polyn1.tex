

   INterpoljatsionnyj mnogochlen Lezhandra,

   Kitajskaja Teorme ob ostatkakh
     i vsjo takoje.

\def\deltapolynomial           {2}
\def \defofderivativespecfial {4}
\def \defofderivativeforpirmes {5}
\def \defofderivativeforprimes {6}

   Let ${\bf x}=\{x_1,x_2,\dots,x_n\}$
   be a sequence of $n$ points on $[0,1]$
   ($x_1<x_2<\dots<x_n$)



     Our first aim is to construct a sequence of polynomials
      which interpolate a  (continuous) function on $[0,1]$.


    Let $f$ be a (continuous) function on $[0,1]$.
   Try to find a polynomial $P$ such that
it coincides with a function $f$ in $n$ points
$\{x_1,\dots,x_n\}$:
             $$
      P(x)\colon\qquad      P(x_i)=f(x_i),
\qquad i=1,\dots,n
              \eqno (1)
              $$
  It is easy to see that if $n=2$ then one can construct
   linear polynomial obeying the condition (1), if $n=3$ then one can construct
   quadratic  polynomial obeying the condition (1).

 One can see that equation (1) has a unique solution in the space
 of polynomials of degree less or equal than $n-1$. Indeed
  if polynomials $P(x)$ and $\tilde P(x)$ obey (1), then polynomial
  $P-\tilde P$ is equal zero in $n$ points. On the other hand its
   degree $\leq n-1$. Hence $P(x)-\tilde P(x)\equiv 0$.
   It remains to construct it.

    For a given sequence of distinct points $\{x_1,\dots,x_n\}$ consider
    a polynomial  $Q(x)=(x-x_1)(x-x_2)\dots(x-x_n)$. Consider polynomial
               $$
         K_{i}(x)={Q(x)\over Q^\prime (x_i)(x-x_i)}
            \eqno(\deltapolynomial)
                  $$
           It is evident that this polynomial has a degree $n-1$
           and it is equal zero in all points ${x_1,\dots,x_n}$
           except the point $x_i$ where it is equal to one.
  Hence a polynomial
                       $$
 P_f(x)=\sum_{k=1}^n  f(x_i){Q(x)\over Q^\prime (x_i)(x-x_i)}
             \eqno (2)
                       $$
   coincides with function $f(x)$ in all points $\{x_1,\dots,x_n\}$.

 One come to remarkable identity: Suppose a function $f$ itself is a polynomial
 of degree $k$. Then if $n>k$ then a polynomial (2) {\bf coincides}
 with $f$ (because it coincides with $f$ in $n>k$ points).
E.g.  let polynomial $P(x)$ has degree less than polynomial $Q(x)$. Considering
 interpolation of $P$ in zeros of polynomial $Q$ we come to the following
 identity:
         $$
  {P(x)\over Q(x)}\equiv\sum_{k=1}^n  {P(x_i)\over Q^\prime (x_i)(x-x_i)}
         $$
   Note that on the other hand:
              $$
   {P(x)\over Q(x)}\equiv\sum_{k=1}^n  {P(x_i)\over Q^\prime (x)(x-x_i)}
              $$
These two formulae produce a nice bunch of examples!!!

These examples look very optimistical. Of course not every function
coincides with interpolating polynomial. Contrary:
situation is very very bad if $f$ is a function
like $|x|$: In this case interpolating polynomials
{\bf do not tend to function}


       $$ $$
Now in full analogy with Lagarange polynomials
we consider China residual Theorem:

Let $N$ be integer. Consider set of primes as set of "points".


\def \Z {\bf Z}

\def \mod {{\rm mod\,\,}}
  A "value" of the "function" $N$ at the "point" $p$
          will be defined as residue of $N$ when dividing by $p$
          (More formally it takes value in the field $\Z/ p\Z$)

For example consider value of "function" $N=10000$
in the "points" $p=2,3,5,7,11,13,17,19$:
                     $$
    N(2)=0 (\mod 2),\,\,
    N(3)=1 (\mod 3),\,\,
    N(5)=0 (\mod 5),\,\,
    N(7)=4 (\mod 7),\,\,
    N(11)=1 (\mod 11),\,\,
         $$
         $$
    N(13)=3(\mod 13),\,\,
    N(17)=4 (\mod 17),\,\,
    N(19)=6 (\mod 19),\,\,
    N(23)=(\mod 23),\,\,
    N(29)=6 (\mod 29)
                     $$

What about "derivative" of "function" in the "point"?

Note that in the case if polynomial $P$ is equal to zero in the point $x_i$, then
its derivative is equal to
the value of the function
\def \defofderivativespecial {4}
\def \defofderivativegeneral {5}
            $$
            {P(x)\over x-x_i}
            \eqno {\defofderivativespecial}
             $$
             in the point $x_i$.

 In general case the  derivative of polynomial in the given point $x_i$
is equal to the value of the function
            $$
            {P(x)-P(x_i)\over x-x_i}
            \eqno (\defofderivativegeneral)
             $$
             in the point $x_i$.

          Now we define the "derivative" of $N$
          in the "point" $p$:

 Unfortunately a number $N$ ("function" N)
and "value of function" $N$ "at the point"
$p$ take value in different rings:

The number $N$ take values in the ring $\Z$.
The "value of function" $N$ "at the point"
$p$ takes value in the field $\Z/p\Z$.

So the operation analogous to operation (\defofderivativegeneral)
is ill-defined in general case. But it is alright if we consider the case
analogous to (\defofderivativespecial):


           Let $p$ divides $N$, i.e. "function" $N$ is equal to zero
           at the "point" $p$. Then we define the "derivative" of $N$
          at the "point" $p$ as the residue of the number
          $N\over p$ when divided on $p$, i.e. the "value" of the
          "function" $N\over p$ at the "point" $p$:
             $$
   N(p_i)=0\Rightarrow N^\prime (p_i)={N\over p_i}(\mod p_i)
                 \eqno (\defofderivativeforpirmes)
              $$
          E.g. for number $N=924=3\cdot 4\cdot 7\cdot 11$
          derivative is defined at the "points" ${2,3,7,11}$:
               $$
     N^\prime(2)=\left(N\over 2\right)(\mod 2)=0 (\mod 2)\,,\quad
     N^\prime(3)=\left(N\over 3\right)(\mod 3)= 4\cdot 7\cdot 11(\mod 3)=
                 2(\mod 3)\,,\quad
               $$
               $$
   N^\prime(7)=\left(N\over 7\right)(\mod 7)= 3\cdot 4\cdot 11(\mod 7)=
                 6(\mod 7)\,\quad
N^\prime(11)=\left(N\over 11\right)(\mod 11)= 3\cdot 4\cdot 7(\mod 11)=
                 5(\mod 11)\,.
               $$

Now we are ready for rewriting Lagrange interpolation formula
for integers:

  Fix a finite set $\{p_1,p_2,\dots,p_n\}$ of prime numbers

 First write the analogue of the polynomial (\deltapolynomial):
 i.e. we have write a "function" $N_i$ such that its
 "value" at all the "points" $\{p_1,\dots,p_n\}$ except the "point"
 $p_i$ is equal to $0$ and its
 "value" at the "point"
 $p_i$ is equal to $1$, i.e.
 we have to find a number $N$ such that
 all the primes $\{p_1,\dots,p_n\}$ except the prime
 $p_i$ divide $N$ and $N$ has remainder $1$ when divided on
 prime $p_i$. In analogy with (\deltapolynomial) using
 (\defofderivativeforprimes) we come to
           $$
   N_i=\left(\right)\left(p_1\cdot p_2\cdot\dots\cdot p_{i-1}\cdot p_{i+1}\cdot\dots\cdot p_n\right)
        $$


%   \footnote{$*$}{Note that Lagrange polynomials do not tend}

   \bye
