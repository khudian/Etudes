  \magnification=1200
  \def \sint {\int_1^\infty {\exp(-at)\over t}dt}
                    $$ $$
            \centerline{\bf One method of Calculating
       Integral   $\int_1^\infty {\exp(-at)\over t}dt$
                 for small $a$}

            \centerline{\bf and Seeley Formulae}
                        \bigskip
            \centerline{\bf O.M. Khudaverdian}
                     \medskip
  We consider calculating of the integral $\sint$
 using ideas inspired by Seeley formulae.
  First the calculation. For estimating the behaviour
of $\sint$ for large $a$ one can consider integration by parts
                      $$
     \sint={e^{-a}\over a}\sum_{n=0}^N{n!(-1)^n \over a^n }
         +{(-1)^N N!\over a^{N+1}}\int_1^\infty e^{-at}t^{N+1}
                                             \eqno (1)
                        $$
{\bf Remark} One can see that this formula does not give
convergent power series. On the other hand for every fixed $N$
 the reminder term is $e^{-a}o(1\over a^{N})$.

 \bigskip


 For small $a$ the Eq.(1) is helpless.

  We consdider the trivial identity
                     $$
         a^{-s}=
              {1\over \Gamma(s)}
       {\int_0^\infty t^{s-1}\exp(-at)dt}
                \quad {\rm for} \quad s>0
                                        \eqno (2)
                    $$

     One can rewrite the l.h.s. of (2) in the way
  to consider its analytical continuation for
neibourhood of $s=0$
                     $$
         a^{-s}=
              {1\over \Gamma(s)}
       {\int_0^\infty t^{s-1}\exp(-at)dt}
                   =
                   $$
                  $$
              {1\over \Gamma(s)}
            \left(
       {\int_0^1 t^{s-1}\left(\exp(-at)-1\right)dt}+
       {\int_0^1 t^{s-1}dt}+
       {\int_1^\infty t^{s-1}\exp(-at)dt}
                     \right)
                    $$
                     $$ =
              {1\over \Gamma(s)}
                         \left(B(s)+
              {1\over s}\right)
                                    \eqno (3)
                    $$
               where we denote
                          $$
                 B(s) =
            \left(
       \int_0^1 t^{s-1}\left(\exp(-at)-1\right)dt+
         \int_1^\infty t^{s-1}\exp(-at)dt
                     \right)
                        $$

       Hence
                         $$
         a^{-s}=
              {1\over \Gamma(s)}
                          \left(
                         B(s)+
              {1\over \Gamma(s+1)}\right)
                                         \eqno (4)
                      $$
   The both parts of (3) are defined and can be differentiate
 in the neibourhood of $s=0$.
           Using that
                     $$
              {1\over \Gamma(s)}\big\vert_{s=0}=0
                \quad {\rm and}\quad
           \left({d\over ds}
              {1\over \Gamma(s)}\right)\Big\vert_{s=0}=
                    \left(
                    {d\over ds}
                    \left(
              {s\over \Gamma(s+1)}\right)\right)\Big\vert_{s=0}=1
                                                \eqno (5)
                       $$

    we come from (4) to
                  $$
         {d\over ds}a^{-s}\big\vert_{s=0}= -\log a=
              B(0)- \Gamma^\prime (1)
                                    \eqno (6)
                      $$
 the derivative  $\Gamma^\prime (1)$  is the famous
  Euler constant
                         $$
            \Gamma^\prime (1)=
                          \lim_{n\rightarrow\infty}
                      \left(
                      1+{1\over 2}
                         +{1\over 3}+
                          {1\over 4}+\dots+
                          {1\over n}-\log n
                      \right)
                                     \eqno (7)
                              $$

             We see from (2---6) that

                    $$
            B(0)=
       \int_0^1 {\exp(-at)-1\over t}dt+
       \int_1^\infty {\exp(-at)\over t}dt=
       \Gamma^\prime(1)-\log a
                                       \eqno (8)
                   $$
        This gives representation for $\sint$
  which is  convenient for small $a$
                      $$
      \int_1^\infty {\exp(-at)\over t}dt=
       \Gamma^\prime(1)-\log a-
       \int_0^1 {\exp(-at)-1\over t}dt=
                  $$
              $$
       \Gamma^\prime(1)-\log a-
          \sum_{n=1}^\infty {(-1)^na^n\over n!n}=
                           \eqno (9)
                     $$
                     $$
     -\log a+\Gamma^\prime(1)+a-{a^2\over 2\cdot 2!}+
       {a^3\over 3\cdot 3!}-{a^4\over 4\cdot 4!}+\dots
                     $$
   (where $\Gamma^\prime(1)$ is nothing but Euler constant $C$:

   \noindent $\Gamma^\prime(1)=C=
   \lim_{n\rightarrow \infty}
   (1+{1\over 2}+{1\over 3}+\dots+{1\over n}-\log n)$)
   Empirically $B(0)$ in (8) can be considered as
  renormalized value of the integral
         $$
         \int_1^\infty t^{s-1}\exp(-at)dt=\dots
                         \eqno (10)
              $$

      And naively (which happens often in Quantum Field Theory)
      one can calculate it using
   Frullany formula:

                       $$
                \int {f(at)-f(bt)\over t}dt=f(0)\cdot\log{b\over a}
                             $$

  The exact result
  differs  from naive expectation on the Euler constant !!!.

   In fact all this stuff is the reduction of the Seeley formulae




   \bigskip

   I think this very trivial exercise is useful

   shtoby vvesti
populjarno ideologiju perenormirovok v formulakh Sili (Seleey)




{\bf Remark 2}

Denote $I(a)=\sint$ and
$K(a)=\sum_{n=1}^\infty {(-1)^na^n\over
n!n}$

 We note that
                $$
            \Gamma^\prime(1)=\int_0^\infty \log t e^{-t}dt=
\int_0^1 \log t e^{-1}dt+\int_1^\infty \log t e^{-1}dt=
         K(1)+I(1)
           $$
 hence we come to
 the identity at $a=1$. Helas our formula does not give
 series for $a=1$...

 {\bf Remark 3} Tut vsjo zatsepleno: Naprimer
                       $$
                       \int_0^1 {e^{-at}-1\over at}dt=
                       \int_0^1 \log t e^{-at}dt
                       $$
            The second integral is related with $\Gamma^\prime(1)$.

            

 \bye
