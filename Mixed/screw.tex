%Zdravstvuj Armen! Shto-to davno o vas nichego ne slyshno. Tut ja
%na dosuge beseduju s odnim armjaninom iz Erevana byvshim mechanikom
%a teperj programmistom. On mne shto-to ochenj tumannoje govoril
%pro tak nazyvajemoje vintovoje ischislenije v mechanike.
%Ja malo shto ponjal no kogda sam stal vyvoditj formuly
%prishjol k sledujushemu etjudu kotoryj mozhet bytj tebe
%i ponravitsja

\magnification=1200

    \centerline {\bf Spirality of particles, Screw in mechanics
      and one invariant of motions in $E^3$}
\bigskip

  Consider affine space $E^3$, fix a point $O$ (nachalo koordinat)
  we come to linear space $I\!R^3$. To arbitrary motion $\cal A$ in $E^3$
  corresponds a pair $(A,b)$ where $A$ is orthogonal transformation
  of $I\!R^3$ and $b$ is a vector:
                    $x^\prime=  Ax+b$
If we consider another nachalo koordinat $O^\prime$
such that vector $OO^\prime$ is equal to $r$ then
to the motion $\cal A$ corresponds a pair $(A,b^\prime)$:
$y^\prime=  Oy+b^\prime$
where $x=y+r$
and
             $$
          b^\prime=b+Ar-r
          \eqno (1)
              $$
       At what extent vector $b$ is defined iniquely
       for a given motion $\cal A$ under a changing of nachalo coordinate?
       Let $\bf n$ be axis corresponding to orthogonal
       operator $A$: linear transformation via $A$ is rotation
       around axis $\bf n$, or in another words $\bf n$
       is eigen vector of $A$: $A\bf n=n$

 It is easy to see from (1) that
   the component of vector $b$ orthogonal to $\bf n$ can be vanished
   by suitable choice of $r$ and
   component of $b$ parallel to axis remains unchanged.
   Formally it follows from the fact that image of operator
   $A-{\bf id}$ is two-dimensional space ($\det(A-1)=0$)
   orthogonal to $b$.

   We come to the invariant of motions: projection of $b$ on $\bf n$.


   {\bf Statement}

   Every motion of $E^3$ can be expressed as rotation over axis
   and translation along this axis, i.e. screw movement
   (vintovoje dvizhenije).

   Shag vinta--projection of $b$ on $n$ is invariant.

   Infinitezimally to this invariant corresponds
   invariant in affine algebra of rotations and translation:
   scalar product $\bf Lp$ of operators of angualr moment and moments.

 This simple statement can be considered as geometrical
 interpretation of spirality...



 \bye

\bye
