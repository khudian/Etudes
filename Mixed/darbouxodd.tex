\def\t{\tilde}
\def\E {{\cal E}}
\def\F {{\cal F}}
\def\L  {\Lambda}
\def\p{\partial}
\def\mod#1 {{\buildrel (#1)\over\approx}}

\bigskip

   \centerline {\bf Appendix 2. A simple proof of Darboux Theorem}

  Using nilpotency of odd variables
  one can directly prove Darboux theorem for an odd symplectic
  supermanifold presenting finite recurrent procedure
  for constructing
  Darboux coordinates from arbitrary coordinates in
  odd symplectic supermanifold.

   Let ${x^i,\dots,x^n\theta_1,\dots,\theta_n}$ be arbitrary
    coordinates in odd symplectic superemanifold $E^{n.n}$
    and $\{\quad,\quad\}$ be odd non-degenerated
    Poisson bracket (Buttin bracket) (\bracket)
     corresponding to the symplectic structure.
    For arbitrary two functions $f$ and $g$
                  $$
                  \matrix
                    {
     \{f,g\}={\p f\over\p x^i}\{x^i,x^j\}{\p g\over\p x^j}+
     {\p f\over\p x^i}\{x^i,\theta_j\}{\p g\over\p \theta_j}+
    (-1)^{p(f)+1}
    {\p f\over\p\theta_i}\{\theta_i,x^j\}{\p g\over\p x^j}\cr
                      +
    (-1)^{p(f)+1}
    {\p f\over\p\theta_i}\{\theta_i,\theta_j\}{\p g\over\p \theta_j}\cr
                   }
               \eqno ({\rm Ap2}.1)
     $$
    Consider
                 $$
     \{x^i,x^j\}=\E^{ij}(x,\theta)\,,
       \{\theta_i,\theta_j\}=\F_{ij}(x,\theta)\,,
\{x^i,\theta_j\}=-\{\theta_j,x^i\}=\Pi^i_j(x,\theta)\,.
                                  \eqno ({\rm Ap}2.2)
                     $$
From definition of symplectic structure it follows that
 $\E^{ij}=\E^{ji}$, $\F_{ij}=-\F_{ji}$ are odd functions
 taking values in Grassmann algebra $\L$
 (as always $\L$ is an arbitrary Grassman algebra)
 and $\Pi^i_j(x,\theta)$ is even non-degenerated
 matrix.

 Without loss of generality suppose
 that $\Pi(x,\theta)\big\vert_{\theta=0}=\delta^i_j$.
 (It is easy to see that one comes to this condition by transformation
 $\theta_i\rightarrow (\Pi)^{-1}\theta_j$ ).

  Consider an algebra $A$ of functions
  on variables $x^1,\dots,x^n,\theta_1,\dots,\theta_n$ taking
  values in Grassmann algebra $\Lambda$ and
  ideal $J_1$ in this algebra generated by odd functions
   $\theta_i, \E^{ij},\F_{ij}$ ($i,j=1,\dots,n$).
   We note that this ideal is nilpotent: the product
   of arbitrary $N$ elements of this ideal is equal to zero, where
   $N=n+{n(n+1)\over 2}+{n(n-1)\over 2}+1$. Hence a natural filtration
    induced by this ideal is finite:
                     $$
    0=A_N\subseteq A_{N-1}\subseteq\dots\subset A_1\subset A_0=A
                           \eqno {{\rm Ap}2.3}
                     $$
           where $A_1=J_1$, $A_{p+1}=J_1\cdot A_{p}$.
   (For example $\theta_i, \F_{ij}\in A_1$,
          $\theta_m{\p \E^{ij}\over x^m}\F_{jk}\in A_3)$.

For two functions $f$ and $g$ in $A$ we say that
$f{\buildrel (p)\over\approx} g$
if $a-b\in A_p$. (For p=N the condition $f=g$ follows from condition
$f{\buildrel (p)\over\approx} g$).

  Relations (Ap1.1, Ap1.2) for initial coordinates
 can be rewritten in the way:
         $$
              \{x^i,x^j\}\mod p
       \{\theta_i,\theta_j\}{\buildrel (p)\over\approx}
       0\,,\,{\rm where}\,\, (\theta_i\in A_1)\,.
          \eqno ({\rm Ap}2.4a)
        $$
        $$
\{x^i,\theta_j\}-\delta^i_j{\buildrel (p)\over\approx}
               0\,\, {\rm and}\, \{x^i,\theta_j\}\big\vert_{\theta=0}=0.
                                  \eqno ({\rm Ap}2.4b)
         $$
 where $p=0$.


Suppose that these relations are satisfied for $p=k$ and find new
coordinates such that they will be satisfied for $p=k+1$.
This gives required recurrent procedure.
At the step $p=N$ we come to Darboux coordinates.

Starting from coordinates
$\{x^1,\dots,x^n,\theta_1,\dots,\theta_n\}$
that obey to conditions (Ap2.4) at $p=k$
expose formulae for new coordinates
$\{\t x^1,\dots,\t x^n,\t\theta_1,\dots,\t\theta_n\}$
that obey to conditions (Ap2.4) at $p=k+1$:
                   $$
                   \tilde x^i=x^i+F^i(x,\theta)\,,\,\,
                   {\rm such\,that}\quad
                   F^i(x,\theta)=
       -\theta_m\int_0^1 t\{x^m,x^i\}\big\vert_{x,t\theta}dt\,,
                    \eqno (Ap2.5)
                    $$
                    $$
             \theta_i=2\theta_i^\prime+L_i(\t x,\theta^\prime)\,,\,\,
                    {\rm such\,that}
                L_i(\t x,\theta^\prime)=
                -\theta_m
                \int_0^1
             t\{\t x^m,\theta^\prime_i\}
             \big\vert_{x,t\theta}dt\,,
                          \eqno ({\rm Ap}2.6)
                    $$
where coordinates $\{\theta^\prime_1,\dots,\theta^\prime_n\}$
are defined by formulae
                    $$
          \theta^\prime_i=\theta_i+K_i(x,\theta)\,,\,\,{\rm such\,that}\quad
            K_i(x,\theta)=
        x^m\int_0^1 t\{\theta_m,\theta_i\}\big\vert_{tx,\theta}dt\,,
                   $$

   Without loss of generality we suppose that
   numerical parts of even coordinates $x^i$ are
  functions on domain $I\!R^n$ and this domain is a disc in
  a vicinity of a point $0$.)

  Conditions (Ap2.5) for $p=k+1$ can be straightforwardly checked
  for coordinates $\{\t x^1,\dots,x^n,\t\theta_1,\dots,\t\theta_n\}$
  using relation (Ap2.2), filtration (Ap2.3) and Jacoby identities
  (\jacoby).

 Using Jacoby identity (2.1) one can show that for coordinates
  ${\t x^1\dots,\t x^n,\theta^\prime_1,\dots,\theta_n^\prime}$
  conditions  (Ap1.4a) are satisfied for $p=k+1$
  and conditions (Ap1.4b) are satisfied for $p=k$.

   Check it in details for Poisson bracket of coordinates
   ${\t x^1,\dots,\t x^n}$
  considering coordinates ${x^1\dots,\theta^\prime_1,\dots,\theta_n^\prime}$
  as functions of coordinates ${x^1\dots,\theta^\prime_1,\dots,\theta_n^\prime}$.
    Note that in  (Ap1.5) $F^i(x)\mod {p+1} 0$, hence
     $\{F^i,F^j\}\mod {2k+2} 0$ and
  $\{\t x^i,\t x^j\}\mod {2k+1} \{x^i,x^j\}+\{x^i,F^j\}+\{F^i,x^i\}$.
  It is easy to see using (Ap1.1) that
  $\{x^i,F^j\}\mod {2k+1} {\p F^j\over\p\theta_i}$, hence
                    $$
  \{\t x^i,\t x^j\}\mod {2k+1} \{x^i,x^j\}-
  2\int t\{x^i,x^j\}\big\vert_{x,t\theta}dt+
  \theta_m\int t^2
  \left(
  {\p\over\p\theta_i}\{x^m,x^j\}+
  {\p\over\p\theta_j}\{x^m,x^i\}\right)\big\vert_{x,t\theta}dt
                   $$
  Using Jacoby identity (2.2) for functions $x^m,x^i,x^j$
  we see that ${\p\over\p\theta_i}\{x^m,x^j\}+
        {\p\over\p\theta_j}\{x^m,x^i\}\mod {2k} {\p\over\p\theta_m}\{x^i,x^j\}$
        thus
                $$
\{\t x^i,\t x^j\}\mod {2k+1} \{x^i,x^j\}-
  \int {d\over dt}\left(t^2\{x^i,x^j\}\big\vert_{x,t\theta}\right)dt\mod {2p+1}
                 \mod {k+1} 0\,\hbox {for all}\,\, k=0,1,\dots\,.
              $$
 In the same way one can check that
 $\{\theta^\prime_i,\theta^\prime_j\}\mod {k+1} 0$.

 Now we consider the following transformation of coordinates
 $\theta^\prime_1,\dots,\theta_n^\prime$.
                $$
    \t\theta_i=\theta_i^\prime-\theta
                 $$


%\bye
