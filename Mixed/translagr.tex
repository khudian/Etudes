





\magnification=1200

\def \a {\alpha}
\def\s {\sigma}

\def \F {{\cal F}}
\def \L {{\cal L}}
\def \D {{\cal D}}
\def \N {{\cal N}}
\def\p{\partial}
\def\d{\delta}
\def \P {{\cal P}}
\def \Pi {{\cal Pi}}
\def \V {{\cal V}}
\def\R {\bf R}
\def \O{\Omega}

  {\it The most exciting point of this etud is that
   the space of planes which are tranvesal
   to the given plane $X$ in the linear space $E$
   is the space of projectors
   on the $X$ and this space is associated to the
   linear space of operators on $E$ which
    vanish {\bf on} $X$ and which take values {\bf in} $X$.
    This space appeared in proving the Theorem of
    irredusible representations for semisimple algebras
    via Whitehead lemmas...}

                 $$ $$
 \centerline{  The description  of lagrangian surfaces transversal to the
      given one.}

   $$ $$

   Let $X$ be given lagrangian surface in linear symplectic space.

   How to desctibe the space of planes transversal to it?

   \smallskip

   First of all we consider the space of all trasversal planes (ignoring the symplectic structure).


   Let $Y_0$ be any transversal plane. Consider the pair $(X,Y_0))$. If $Y$ is any
   other plane transversal to $X$  then one can consider linear operator
    $A\colon\quad Y_0\rightarrow X$ such that
                   $$
                   Y=Y_0\oplus AY_0
                    \eqno (1)
                   $$
   The operator $A$ defines uniquely the operator $W$ on $E$ such that
    $W\Big\vert_{X}=0$. This operator obeys to relation
                  $$
                  {\bf\rm Im}W\subset X\subset {\bf\rm ker}W
                     \eqno (2)
                  $$


     {\bf Proposition 1}
     The space of surfaces transversal to $X$ is affine space. The associated linear
     space is the linear space of operators $W\in {\rm Hom}(E,E)$ obeying to (2).



    In fact the best way to describe the affine space of transcversal surfaces it is to use
     projection operators for which $X$ is kernel, or for which $X$ is identity domain.

     Let $P_0$ be a  projection operator
  corresponding to the plane $Y_0$ transversal to $X$, i.e.
  $P_0\big\vert_{Y_)}={\bf \rm id}$ and ${\rm ker}P_0=X$
     (or $\tilde P_0=1-P_0$ is the projection operator such
      that ${\rm ker}\tilde P_0=Y_0$ and $\tilde P_0\big\vert_X={\bf \rm id}$).
          It is easy to see that projection operator corresponding to the
    the plane $Y$  is equal to the
                      $$
                  P=P_0+W
                       $$

{\bf Proposition 2}
     The affine space of planes transversal to $X$ is
     the affine space of projectors of $E$ on the $X$.
     For arbitrary projector $P_0$ of $E$ on the $X$
     $P$ is    projector $P_0$ of $E$ on the $X$ also,
     if and only if $W=P-P_0$ belongs to the linear space (2)


  {\bf Remark} What is the meaning of the cohomology of (2) ($W^2=0$).

   $$ $$

   Now we return to symplectic and lagrangian structure.
This put additional conditions on $W$ and $P$.

The operator $A$ in (1) is the linear operator on Lagrangian surface $Y_0$
 with values in Lagrangian surface $X$.
 Correspondingly the operator $W$ in (1) is the linear operator the whole symplectic
 space $E$ into itself such that it takes values
 with values in Lagrangian plane $X$ which belongs to its kernel.
 Identifying $E$ with dual space $E^*$ (via symplectic product) we put in
 correspondence to the linear operator $W$ the {\bf bilinear form} $\O$:
                               $$
                               \O(a,b)=<W(a),b>
                                  \eqno (4)
                                $$
    Correspondingly identifying lagrangian plane $X$ with dual to $Y_0$
    we put in correspondence to $A$ in (1) the bilionear form on $Y_0$.
  From (1) it follows  that $<Au,v>+<u,Av>=0$, for arbitrary $u,v,\in Y_0$,
  because $Y$ is lagrangian plane also. Correspondingly for $W$
  $<Wu,v>+<u,Wv>=0$, for arbitrary $u,v,\in E$.
  Hence $\O$ is bilinear symmetric form.


 {\bf Proposition 1L}
     The space of lagrangian planes transversal to
     the given lagrangian plane $X$ in the linear symplectic
     space  is affine space.
     The associated linear
     space is the linear space of symmetric bilinear forms
     on $E$ which vanish on $X$ (forms on $E/X$).

   {\bf Remark}. In component language it is nothing but
    constructing lagrangian surface $Y$ as graph of the linear function
    generating by quadratic function on $Y$.


  One can see easily that condition that $X$ and $Y$ are lagrangian put the following
   constraints on the projector $P$:
                              $$
                   <Pu,v>+<u,Pv>=<u,v>
                                  \eqno (5)
                              $$
Note that from (2) it follows that
 $<Wu,v>+<u,Wv>=0$ $<Wu,v>=<Wv,u>$.
� �ݣ �������, ��� (5) ��� � ����������!!!

{\bf Proposition 2L}
     The affine space of lagrangian planes transversal to
      the lagrangian plane $X$ is
     the affine space of projectors of $E$ on the $X$,
      obeying to (5).
     For arbitrary such projector $P_0$ of $E$ on the $X$
     $P$ is  projector $P_0$ of $E$ on the $X$ which also
       defines transversal lagrangian plane
     if and only if
      $W=P-P_0$ belongs to the linear space (2)
      and $<W...,...>$ is symmetric.

\bye
