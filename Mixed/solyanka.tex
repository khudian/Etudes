                 \magnification=1200

                        \centerline {\bf SOLYANKA}

                          $$ $$

              \centerline {\bf Contents}

              \bigskip

             {\bf 1 Arithmetics}

             1.1 Prosto krassivo!

             1.2 A vot dve zadachki

             1.3 The problem of the coins:DIVIDE ET IMPERA

             1.4 $n$-Points on the Plane (Ochenj staraja zadachka!)

              1.5  Points on Integer Distances

              1.6  Odd and Even

               1.7 $\pi(n)$ for prime numbers

               \medskip

               {\bf 2 Geometry}

                 2.1 Something very amazing about circles

                 2.2 The equation of the curve whose curvature is
               proporional to its length

                  2.3 One approach to higher order invariants

                 2.4 Invariants of Hypersurfaces in Euclidean Space

                  2.5 Spirality of particles, Screw in mechanics
                  and one invariant of motions in $E^3$

                 2.6 Space of planes...

               2.7 Seeley formula and one integral

               2.8 K zadache o kreste

               2.9 Dva slova o jacobian problem

                   \medskip

                    {\bf 3 Algebra}

                 3.1 Representation functor for jets

                3.2 Projective module of Mobius band global sections

                     $$ $$


     \bigskip

          \centerline    {\bf 1.Arithmetics}

                  \bigskip


     \centerline {\bf 1.1 Prosto krassivo:}

      Ramanujan formula for $\pi$:

 Each term of this series produces an additional eight
 correct digits:
                        $$
        {1\over\pi}=
        {2\sqrt 2\over 9801}
        \sum^\infty_{k=0}
                {
          (4k)!(1103+26,390 k)
             \over
             (k!)^4 396^{4k}
                 }        $$

$$ $$

 \centerline {\bf 1.2 A vot dve zadachki}

\medskip

Eto zadacha iz 70 g0dov:


 You have 100 roubles.
 The bottle of bear costs 37 k.
 The empty bottle 12 k.

 You buy on all money bear drink it,
 then return bottles, buy again and so on...

 How much bottles of bear you will drink?

 ***************************
 Two sequences of real numbers $\{x_n\}$ and $\{y_n\}$
  obey to conditions:
                       $$
                       \cases
                         {
                         y_n-x_n=n\cr
                         y_n\cdot x_n=n\cr
                         }
                         $$
   Calculate limit of $y_n-x_n$ when $n$ tends to $\infty$


           $$ $$



     \centerline {\bf 1.3 The problem of the coins}

\bigskip
  \centerline {Along the paper of  Mario Marteli and Gerald Ganon:}


  \centerline {The College Mathematics journal, 28, No5, pp. 365-367,1997}



 It is the very old problem how to find the odd coin from 12 coins
 by 4 measurements...

    \bigskip


   \centerline {\it DIVIDE ET IMPERA}

      $$ $$
  For finding the strategy we first solve three preliminary
  problems:

       $$
       1)\rightarrow 2)\rightarrow 3)\rightarrow 4)
                    $$

  {\bf Proposition 1}
   Number of coins is equal to $3^n$ and {\bf odd coin is heavier}.
    One can find it by $n$ measurements.


    The proof (trivial): The case $n=1$ is evident. Let $n=k+1$. During the
    first measurement we put $3^k$ coins on left and $3^k$ on the
    right. If "$left > right$" by induction we choose odd coin in
    the left set by $k$ measurements. If "left=right" we do it
    with $3^k$ remaining coins.

       \bigskip

     {\bf Proposition 2}
      Number of coins is $3^n$. Coins are marked: they are red or black and
     red coins are heavier or equal by weight to black,
     i.e. odd coin is heavier if it is red and it is  lighter if it
     is black.
     One can find it by $n$ measurements.

     The proof. The case $n=1$ is trivial again.
   Let the statement is right for $n=k$.

      Let $n=k+1$.
     We always can choose $2a$ red and
    $2b$ black coins such that $a+b=3^n$ $(a=0,1,2,\dots)$ and put on the left
    pan $a$ red coins and $b$ black coins, the same on the
     right pan. If "$left> right$" hence the odd
  coin is one of the $a$ red or one of the $b$ black.
  We take $3^n=a+b$ coins ($a$ red from the left and $b$ right
  from the right and by $n$ measurements find the odd one.

    If "$left=right$" the we find odd one in the remaining $3^n$
    coins.

 \bigskip

   {\bf Proposition 3}
       Number of coins is equal to $s_n={3^n-1\over 2}$,
    ($s_1=1,s_2=4,s_3=13,s_4=40,s_5=121,\dots$ ($s_{n+1}=3s_n+1$).

   The coins are not marked and
   we do not know is the odd coin heavier or odd but we have
    {\bf additionally the coin which sure is not odd.}
    One can find it by $n$ measurements!?

    {\bf Proof} The case $n=1$ is evident. We compare the coin
    with test one and see is it odd or not.


   Let we can find the odd one for $n=k$
   and consider the case $n=k+1$.

                   $$
    s_{k+1}=s_k+s_k+s_k+1
                    $$
    We put aside $s_k$ coins. We put $s_k$ coins and
     additional test coin on the left pan and $s_k+1$
      coins on the right pan.
      If "left>right" we throw out the test coin,
      mark the $s_k$ coins on the left pan by red paint,
       mark the $s_k+1$ coins on the right pan by black pain.
       We have $s_k+s_k+1=3^k$ marked coins. Using
        Proposition 2) we find the odd coin by $k$ measurements.

        If "left=right" we know that odd coin
        is in the rest $s_k$ coins and we find it
        with test coins by $k$ measurements.


                $$ $$

   Finally we come to

           {\bf Theorem}
     Number of coins is equal to $p_n s_n-1={3^n-3\over 2}$,
    ($p_1=0,p_2=3,p_3=12,p_4=39,p_5=120,\dots$ ($p_{n+1}=3p_n+3$).

   We do not know is the odd coin heavier or odd!

    One can find it by $n$ measurements!!!


      {\bf Proof}


    Let $n=k+1$.

   One can see that
                  $$
                 p_{k+1}=(p_k+1)+(p_k+1)+(p_k+1)
                    $$

           We put $p_k+1$ coins on the left side and $p_k+1$ on the right.

           a) If "$left>right$" It means
            that odd coin is "red" if it is on the
           left pan and it is right if it is on the
            right pan (See Proposition 2).
            We mark coins on the left pan by red and coins on the
            right pan by black, add one coin from
            the remaining coins. (This additional coin is sure not odd,
            so it can be marked as we want red or black)
               $(p_k+1)+(p_k+1)+1=3^k$. Hence we have $3^k$
               marked coins and we can find the odd coin amongst
               them, according to Proposition 2)


      b) "left=right". In this case we know that odd coin is
      in the remaining $p_k+1$ coins. But $p_k+1=s_k$ (See Proposition 3)

          We take remainig $s_k$ coins and the test coins from
          the coins which we measured also and due to Proposition 3)
          we find it by $k$ measurements.
                 \bigskip
                    It is all!

     {\bf Example} Let we have 12  $\{1,2,3,4,5,6,7,8,9,10,11,12\}$ coins.

    First meauserement:
      $\{1,2,3,4\}$ on the left and $\{5,6,7,8\}$ on the black.

           \medskip

     a)If $\{1,2,3,4\}>\{5,6,7,8\}$ we mark $\{1,2,3,4\}$ by red, $\{5,6,7,8\}$
   by black and consider $\{1_r,2_r,3_r,4_r,5_b,6_b,7_b,8_b,10_b\}$.

     We do second measurement:
       $\{1_r,2_r,5_b\}$ on the left and $\{3_r,4_r,6_b\}$
   on the right.
   If $\{1_r,2_r,5_b\}>\{3_r,4_r,6_b\}$ it means that
   odd is or on the left and red or on the right and black.
   We take the coins
   $\{1_r,2_r,5_b\}$ and find by the third measurement
    the odd one, putting $1_r$ on the left and $_b$ on the right.

    If $\{1_r,2_r,4_b\}=\{3_r,4_r,5_b\}$ it means that
    the odd is in the set $\{7_b,8_b,10_b\}$ and we
    find it by one measurement,
    putting $7_b$ on the left and $8_b$ on the right.

  Now consider the second case:

   \medskip
   a)If $\{1,2,3,4\}=\{5,6,7,8\}$. Then the odd is in the set
    $\{9,10,11,12\}$. We take the coin $1$ as the test coin and add it to
    the set $\{9,10,11,12\}$. Consider the set $\{1,9,10,11,12\}$
    where $1$ is the test coin (sure not odd).

     We do the second measurement according to the proof of Proposition 3):
     putting the coins $\{1,9\}$ on the left pan and
     $\{10,11\}$ on the right pan.
    Now if $\{1,9\}>\{10,11\}$ we mark ( coin by red and 10 and 11 by black
     and during the third measurement we
     put the coin 9 on the left pan and 10 on the right pan
     and find the odd coin in the set $\{9,10,11\}$.
   If $\{1,9\}=\{10,11\}$ then the odd coin is the coin $12$
   and we do not need the third measurement.


        $$ $$
           \centerline {\bf 1.4 $n$-Points on the Plane}

           \medskip
   (Etu zadachu mne zadal Senja Rubanovitch v 1971 godu)


                     $$ $$
   \item{I} There are $n$-points on the plane. For every 2 points there exist
 the third which is on the same line.
 Prove that all the points are on the same line.

 There is the following elegant solution (Daniel informed me about it)

 Esli eti tochki ne lezhat na odnoj prjamoj, rassmotrim
 konfiguratsiju:
 <tochka $A$, prjamaja prokhodjashaja cherez dve tochki $B$ i $C$>,
 tak shto  rasstojanije mezhdu tochkoj i prjamoj naimenjshee vozmozhnoje
  (iz otlichnykh ot nulja). Esli $D$ tretja tochka na prjamoj $BC$
  to pridjom k protivorechiju (estj konfiguratsija s boleje malenjkim rasstojanijem)


  Odnako u menja estj drugoje reshenije:
                   \smallskip
        \centerline{The main idea of solution.}
                  \smallskip

\def\variables#1#2 {#1=1,\dots,#2}
   We can always choose coordinates $(x,y)$ on the plane
   and the numeration of the points in the following way:
    If $(x_i, y_i)$ are coordinates of the $i$-th point
   ($\variables{i}{n}$)   then
             $$
   x_1\leq x_2\leq x_3\leq\dots\leq x_n,\quad
   y_1\leq y_2\leq y_3\leq\dots\leq y_n\,.
             $$
We consider the equivalent problem

  \item{II} From the every point $x_1$ the "point" with constant velocity
     $v_i=y_i$ begins to move.  All the intersections are not binar!
 (it corresponds that for every 2 points there exist
 the third one on the same line).
  Prove that all the lines intersect simultaneously!


 This second formulation admits combinatoril reformulation.
 Every collision is the rearrangement of the points.


 There are $n$  symbols on the line-- $a_1,\dots,a_n$.
 The rules of the game are following:...

 This third reformulation is solved by me.

          $$ $$


      \centerline {1.5\bf Points on Integer Distances}

    \medskip

Yulii Rudyak gave me the following problem:

 {\bf $X$ is the set of the points on the plane, such that
 not all the points are on the same line and all the distances
 between the points are integers. Prove that in this case $X$
 has to be a finite set.}

\smallskip

 {\bf Remark} One can easily to construct for every $N$ a set of
$N$ points which are not on the same line and
 all the distances are
 integer. One can also construct the infinite set $X$
of points not on the same line with rational distances
between them.

E.g. consider the points
 $\{A_0,A_1,\dots.A_p,\dots\}$ which have following
cartesian coordinates:
               $A_0$ has coordinates $(0.2)$ and
the point $A_k$ has coordinates $(k-{1\over k})$ ($k=0,1,2,3,\dots$).
It is easy to see that all the distances between these points are rational.
On the other hand for arbitrary $N$ under
dilatation $(x,y)\rightarrow$ $(N!x,N!Y)$ the subset of first
 $N+1$ points transforms to the set with integer distances.

\smallskip

 Now we go to the proof of the Rudyak,s statement.

 Let $A,B,C$ be three points from the set $X$
 which are not on the same line and
  there lengths are equal to $|AB|=c$, $|BC|=a$

We consider the subset $X_{pq}$ which is defined in the following way:
                            $$
    X_{pq}\colon \quad {\bf r} \in X_{pq}\quad{\rm iff}
                         \,\,{\bf r}\in X\quad {\rm and}\,
                       \cases
                         {
                  |\rho({\bf r},A)-\rho({\bf r},B)|=p\,,\cr
                  |\rho({\bf r},A)-\rho({\bf r},C|= q\,\cr
                         }
                                    \eqno (1)
                        $$
In other words $X_{pq}$ are the points of $X$ whose distances
between $A$ and $B$ differs on $p$ and between $A$ and $C$
differs on $q$.

It is evident that the integers $p$ and $q$ are restricted by
   integers $c$ and $b$ respectively: $0\leq p\leq c$,$0\leq q\leq b$,

Hence it remains to prove that all $X_{pq}$ are finite sets,
because
                        $$
      X=\bigcup_{p\leq c, q\leq b}X_{pq}
                \eqno (2)
                         $$
The fact that every $X_{pq}$ is the finite sets immediately follows from the
  fact that  belongs simultaneously to hyperbole
  $\Gamma_p^1$ {\bf r} {\bf r} with focuses in the points $A$ and $B$,
   defined by the first equation in (1) and
to the hyperbole $\Gamma_q^2$ with focuses in the points $A$ and $C$
 defined by the second equation in (1).
 Hence these hyperboles have not more than $4$ general points,
because their focuses do not belong to the same line.
 (The degenerated hyperboles: $\Gamma^1_0$, $\Gamma^2_0$---lines
 and $\Gamma^1_c$ and $\Gamma^2_b$---rayons are considered also like hyperboles.
 "Hyperboles" $\Gamma^1_c$ and $\Gamma_b^2$ --- rayons
 have infinite set of general points if $A,B,C$ would be on the same line)

 We came to the conclusion that the number of ponts in $X_{pq}$
is not more that $4$. From (3) and (2) it follows that number of points in
 $X$ is not more than $4bc$.
The statement is proved.

 Compare this problem with the problem discussed in
 "$n$-points on the plane"

$$ $$
   \centerline {\bf 1.6 Odd and Even}

\medskip

Corollary 1 and Corollary 2 are the reason why it is not worthless
 to read this text. If you had a trouble to calculate degrees of freedom of
 symmetrical tensors of rank $k$ on $n$-dimensional space
 then may be you will not be angry on this text.

 Problem formulated in Corollary 2 was suggested to me by Senja Veinstein in 1997.


   So...

   \medskip

 In this etude we consider two isomorphisms of
different nature between the following sets.

1) $A$--is the set of not-decreasing finite sequences of natural
numbers. $\{a_1,\dots,a_n\}\in A$ if
   $a_1\leq a_2\leq\dots\leq a_n$

2) $B$ is the set of increasing natural numbers.
$\{a_1,\dots,a_n\}\in B$ if
   $a_1< a_2<\dots< a_n$.

   The first isomorphism $\varphi_1$:


                    $$
  \forall \{a_1,\dots,a_n\}\in A
                   \varphi (\{a_1,\dots,a_n\})=
                    \{a_1,a_2+2,\dots,a_n+n\}\in B
                      \eqno (1)
                            $$

It is evident that $\varphi$ is an isomorphism


To construct the second isomorphism we first identify the space
$A$, the set of not-decreasing finite sequences of natural
numbers, with the set $A^\prime$--the set of not-decreasing
finite sequences of odd natural  numbers by isomorphism
                  $$
     \tau\colon\quad
       \forall \{a_1,\dots,a_n\}\in A_0,
       \tau (\{a_1,\dots,a_n\})=
         \{2a_1-1,2a_2-1,\dots,2a_n-1\}\in A^\prime
            $$
   and consider the following "tricky" isomorphism between
    $A^\prime$ and $B$.
    Let $b=\{b_1,\dots,b_n\}\in B$. For every $b_i\in b$ there
    exist natural numbers $p_i$ and $q_i$ such that
                    $$
            q_i \,\,{\rm is\,\, odd\,\,and}\quad
            b_i=2^{p_i}q_i
            \eqno (4)
                       $$
         We put in correspondence to the sequence
         $\{b_1,\dots,b_n\}$ the sequence
                           $$
         \left\{
     {\underbrace {q_1,q_1,\dots,q_1}_{p_1\, {\rm times}}}\,,\,
     {\underbrace {q_2,q_2,\dots,q_2}_{p_2\, {\rm times}}}\,,
                   \dots\,
     {\underbrace {q_n,q_n,\dots,q_n}_{p_n\, {\rm times}}}
         \right\}
         $$
Then rearrange the odd numbers $\{q_1,\dots,q_n\}$ in
not-decreasing order  $q_{i_1}\leq q_{i_2}\leq\dots\leq q_{i_n}$
we finally come to isomorphism $\varphi_2$:
                             $$
                    \varphi_2 (\{b_1,\dots,b_n\})=
     \left\{
     {\underbrace {q_{i_1},q_{i_1},\dots,q_{i_1}}_{p_{i_1}\, {\rm times}}}\,,\,
     {\underbrace {q_{i_2},q_{i_2},\dots,q_{i_2}}_{p_{i_2}\, {\rm times}}}\,,
                   \dots\,
     {\underbrace {q_{i_n},q_{i_n},\dots,q_{i_n}}_{p_{i_n}\, {\rm times}}}
         \right\}
                         $$
between $B$ and $A^\prime$. The inverse to $\varphi_2$ isomorphism
is following. Consider
                  $$
      \left\{
     {\underbrace {q_1,q_1,\dots,q_1}_{f_1\, {\rm times}}}\,,\,
     {\underbrace {q_2,q_2,\dots,q_2}_{f_2\, {\rm times}}}\,,
                   \dots\,
     {\underbrace {q_n,q_n,\dots,q_n}_{f_n\, {\rm times}}}
         \right\}\in A^\prime
             \eqno (lalala)
                        $$
                        then
                     every number $f_i$ can be uniquely
                     represented as
                    $$
                    f_i=\sum_k 2^k \delta_{ik}\quad {\rm
                    where}\,\, \delta_{ik}=0,1
                        $$


We consider the numbers $2^{k}q_i$ if $\delta_{ik}\not=0$ and
arrange the sequence from $B$.

They are so far from each other isomorphisms $\varphi_1$ and
$\varphi_2$!!!

For every sequence $a$ in $A$, $A^\prime$ or $B$
we denote by $\#(a)$ the number of elements of this sequence and
by $s(a)$ the sum of elements of this sequence.
The following relations are obvious:

1.
                 $$
           \#(\varphi_1(a))=\#(a),
                   $$
2. If $\#(a)=n$, then
                   $$
             s(\varphi_1(a))=
             s(a)+{n(n+1)\over 2}
                $$
                $$
           \quad s(a)=s(\varphi_2(a))
                     \eqno ()
                   $$
These relations leads to interesting corollaries:

{\bf Corollary 1}.

  {\it Dimension of the space of symmetrical tensors of rank $k$ in
   $n$-dimensional
   space is equal to the
 a dimension of the space of antisymmetrical tensors of rank $k$ in
   $\left(n+{k(k+1)\over 2}\right)$-dimensional space.
  space.}





{\bf Corollary 2}.

{\it For every integer number $N$
a number
of choices such that $N$ can be represented as a sum of different
natural numbers is equal
a number
of choices such that $N$ can be represented as a sum of
odd natural numbers.}
 (We do not differ two choices
which differ by rearrangement.)


We can use the corollary 1 for calculating the dimension of symmetrical tensors space:
because
for antisymmetrical tensors it can be calculated  straightforwardly:


{\bf Corollary 3}.

  {\it Dimension of the space of symmetrical tensors of rank $k$ in $n$-dimensional
   space is equal to the}
               $$
               C^k_{n+{k(k+1)\over 2}}=
                     {
                     \left(n+{k(k+1)\over 2}\right)!
                         \over
                         k!
                         \left(n+{k(k-1)\over 2}\right)!
                         }
                       $$
 We denote by $A_n$ the number of choices for odd numbers
and $B_n$ the number of choices for different natural numbers.
It is evident that
              $$
           \sum_{n=1}^\infty B_n x^n=
           \sum_i\left(
           \sum_{n_1\geq 1,\dots n_i\geq 1}
              x^{n_1}x^{n_1+n_2}\dots x^{n_1+\dots+n_i}
           \right)=
            $$
            $$
            \sum_{n=1}^\infty
                 {x^{n(n+1)\over 2}\over
   (1-x)(1-x^2)(1-x^3)\dots (1-x^n)
                 }
              $$
   and
           $$
           \sum_{n=1}^\infty A_n x^n=
           \sum_i\left(
           \sum_{n_1\geq 0,\dots n_i\geq 0}
              x^{2n_1+1}x^{2n_1+2n_2+1}\dots x^{2n_1+\dots+2n_i+1}
           \right)=
            $$
            $$
            \sum_{n=1}^\infty
                 {x^n\over
   (1-x^2)(1-x^4)(1-x^6)\dots (1-x^{2n})
                 }
              $$
       We come to
{\bf Corollary 4}
                         $$
                \sum_{n=1}^\infty
                 {x^n\over
       (1-x^2)(1-x^4)(1-x^6)\dots (1-x^{2n})
                 }=
             \sum_{n=1}^\infty
                 {x^{n(n+1)\over 2}\over
   (1-x)(1-x^2)(1-x^3)\dots (1-x^n)
                 }
                 $$
in a vicinity of $x=0$:


I cannot calculate the number of choices  $A_n$ and $B_n$(see Corollary 2)
but one interesting identity for this.
We denote by $S^p_N$ the number of choices that $N$ can be represented
by $p$ different integers.
It is evident that
                  $$
                  S^p_N=\sum_k S^{p-1}_{N-pk}
                  $$
   Applying this equation many times we come to the

   {\bf Corollary 5}
   The number of choices for $N$ is equal to the number of solutions
    of the equation
                  $$
                2x_1+3x_2+4x_3+........<N
                   $$
                   plus $1$
    where $x_1,\dots,x_n$ are {\bf non-negative} integers.
(Do not confuse with number of values less that $N$!)
In other words we take $N$ (representing as one number)
then calculate the number of $x_1$ such that
  $2x_1< N$,
  ...

  For example
   $S^1_20=1$, $S^2_20=9$,
   $$
   S^3_20=S^2_17+S^2_13+.....$$
      $$
      S^4_20=S^3_16+S^3_12+...$$


>>>>>>>>>>>>>>>>>>>>>>
              $$ $$

                  \centerline {\bf 1.7 $\pi(n)$ for prime numbers}

 {\bf Theorem}

           $\pi(n)\approx {n\over \log n}$



Two years ago I obtained  empirical proof of this Theorem
 considering $N!$ for large $N$ and using the fact that
 for every primary $p$ $N!$ is divisible on $p^r$
 where
   $$
  r\approx \left[N\over p\right]+\left[N\over p^2\right]+\dots
  \approx {Np\over p-1}
         $$
Then considering integrals instead series we come to the "proof".
It is very rough but I guess that the classic proof is founded on
this remark.

\smallskip

Few days ago in the article about Erdos (The Mathematical Intelligence,
{\bf 19}, 2, pp.38---48) I found the marvelous proof that blongs to Erdos.

I will try to write here about his proof.

  Denote by $\Pi (m,n)$ the product of primary numbers
  that belong to the segment $[m,n]$ ($m<n$):
                      $$
                      \Pi(m,n)=\prod_{m\leq p\leq n} p
                      \quad \hbox {where $p$ are prime numbers}
                      $$

  {\bf Lemma 0}.   $C^m_n\leq 2^{n-1}$ (at least for odd n)

\medskip

  {\bf Lemma 1}. If $2m>n-1$ then $\Pi(m,n)\leq C^m_n$.

\medskip

  {\bf Lemma 2}. $\Pi (1,n)<4^n$!!!.

 From this Lemma it obviously follows that
               $$
          \pi(n)<{n\over \log_4 n}
          $$

\medskip

  Lemma 2 belongs to Erdos and
  {\it a brilliant idea, a suprising step
     to prove it is} (from Math. Intell.) is the  content of Lemma 1.

 the statement of Lemma 0 is trivial. If $n$ is odd then
                  $$
   C^m_n={1\over 2}(C^m_n+C^{n-m}_n)<{1\over 2}\sum C^k_n=
         {1\over 2}2^n=2^{n-1}
                   $$
 The statement of lemma 2 is obvious as soon as it is formulated!

 Now prove Lemma 2 by induction. For n=2 it is valid.

 Let $\Pi (1,k)<4^k$ for all $k<n$. Prove for $k=n$.
 If $n$ is even then $\Pi(1,n)=\Pi(1,n-1)$.
 If $n=2q+1$ is odd then using inductive hypothesis and Lemma 1
  we come to
 $$
 \Pi(1,n)=\Pi(1,q+1)\cdot\Pi(q+2,2q-1)<4^{q+1}C^{q+2}_{2q-1}<
     4^{q+1}2^{2q-2}=4^{2q}\,.$$


         $$ $$ $$ $$

          \centerline {\bf 2. Geometry}

             $$ $$

             \centerline {\bf 2.1 Something very amazing about circles}

\medskip

Thinking about Frenet curvature and performing {\bf fault
considerations} I came occasionally to very elementar but amazing
statement.

  {\bf Question}. Consider the point on the plane which moves with
  constant angular velocity $w$ around the point $O$ and with
  constant velocity $v$. What will be its trajectory?

  \smallskip

  The trivial answer is: it will be the circle with centre $O$
  and with radius $R$ such that
                        $$
                                   v=wR\,.
                         $$
 But there is exactly one another case!!!
 The circle with radius ${R\over 2}$
 with centre in the point $O$.
 In fact trivial solution is "ogibajushaja" of these solutions!

 Frome the point of view of differential equations it is
 following:
 The diff.equation
                    $$
                      \cases
                       {
                       \dot x=\sqrt {1-x^2}\cr
                          x(0)=1,\quad o\leq t\leq {\pi\over 2}\cr
                          }
                          $$
               has exactly two solutions:
            the first one $x(t)=1$,
            the second one: $x(t)=\sin t$.


      $$ $$
      \def\p{\varphi}

  \centerline {\bf 2.2 The equation of the curve whose curvature is
  proporional to its length}

\medskip

     Let $x(t), y(t)$ be equation of this curve. Then
 the curvature
          $$
         K(t)={1\over\sqrt{x_t^2+y_t^2}}
                     {d\over dt}
                            \left(
                     {1\over\sqrt{x_t^2+y_t^2}}
                                {\rm arctg}
                                   {y_t\over x_t}
        \right)\,.
                       $$
                   And the equation:
                         $$
                         \cases
                         {
                       {d\over dt}K(t)=\sqrt{x_t^2+y_t^2},\cr
                           K(0)=0\cr
                           }
                           $$

The curve can be rotated and translated so we choose additional
initial conditions:
                        $$
                     x(0)=y(0)=y_t(0)=0
                               $$


   One can see that it is useful to choose as parameter $t$ the
   angle
                     $$
                     \varphi={\rm arctg}{y_t\over x_t}
                      $$
   This immediately leads us to the answer.
   Choosing this parameter we rewrite the equation as

                              $$
                       {d\over d\varphi}\left(
       {1\over\sqrt{x_\p^2+y_\p^2}}
                       \right)=\sqrt{x_\p^2+y_\p^2}
                           $$
so
                    $$
       {1\over\sqrt{x_\p^2+y_\p^2}}=\sqrt{2\p}
                     $$
                and
           $$
           x_\p={\cos\p\over \sqrt{2\p}},\quad y_\phi=x_\p={\sin\p\over \sqrt{2\p}}
           $$
       hence
                   $$
              x(\p)+iy(\p)=\int_0^\p{\exp(i\p)\over\sqrt{2\p}}d\p
                               $$

 In particular the limit point is:
                                      $$
              x(\infty)+iy(\infty)=\int_0^\infty{\exp(i\p)\over\sqrt{2\p}}d\p=
                {1\over \sqrt i}\Gamma{1\over 2}\quad
                ({\rm up\quad to\quad some\quad coefficient})
                                     $$


>>>>>>>>>>>>>>>>>>>>>>>>>>>>>>>>>>>>>>>>>>>>>>>>>

                 $$ $$

                    \centerline {\bf 2.3 One approach to higher order invariants}

\medskip


 We consider curves---  $x^i=x^i(t)$ in $d$-dimensional
eucledian space. Our goal is to construct local integrands
 over these curves which are

1) reparametrization invariant

2) invariant under global translation  and rotation of curves

3) are full divergence!---i.e. their Euler Lagrange equations are trivial

 If we omit condition (3) the answer is invariant densities--
length: $\sqrt {{dx^i\over dt}{dx^i\over dt}}$,...

But how to construct at least one invariant closed density?
In fact in $d$-dimensional case we restrict ourself by densities
whose rank (the order of derivatives) is not higher than $d$


1) $d=1$ . Answer is trivial:

               $$
        A=  x_t
                         \eqno (1)
                $$

2) $d=2$ Less trivial but I think you know it:
                 $$
 A= {x_ty_{tt}-y_tx_{tt}\over x_t^2+y_t^2}
                               \eqno (2)
                   $$
  One can see easily that (2) is total divergence
 (it is evident in polar coordinates).


3) Now I present you the closed density in $3$-dimensional space:
                      $$
                      A=
                      {
                    \det
                   \pmatrix
                      {
           &x_t &y_t & z_t \cr
           &x_{tt} &y_{tt} &z_{tt} \cr
           &x_{ttt} &y_{ttt} &z_{ttt} \cr
                      }
                      \over
            (x_ty_{tt}-y_{t}x_{tt})^2+
            (y_tz_{tt}-z_{t}y_{tt})^2+
            (z_tx_{tt}-x_{t}z_{tt})^2
                      }
                      \cdot
                \sqrt
              {x_t^2+y_t^2+z_t^2}
                                             \eqno (3)
                        $$

  Do you want to check by straightforward calculation
that (3) is total divergence?
Note please that "razmernostj" in (3) and (2) is $[{1\over sec}]$
I came to (3) using (2)...

In fact I think that in $d$-dimensional space all invariant densities
can be expressed in terms of determinant as in (3) and its
first minors.
For example:

  Let we denote by $x^i_{(n)}$ the $n$-th derivative by $t$ of the
 coordinate $x^i$  in $I\!E^d$  $(i=1,...,d)$.
  We consider  the matrix
                        $$
          \pmatrix
                        {
             &x^1_{(1)}, &x^2_{(1)},  &x^3_{(1)}, ...,&x^d_{(1)} \cr
             &x^1_{(2)}, &x^2_{(2)},  &x^3_{(2)}, ...,&x^d_{(2)} \cr
             &x^1_{(3)}, &x^2_{(3)},  &x^3_{(3)}, ...,&x^d_{(3)} \cr
             &..., &...,  &..., ...,&... \cr
             &..., &...,  &..., ...,&... \cr
             &..., &...,  &..., ...,&... \cr
             &..., &...,  &..., ...,&... \cr
           &x^1_{(d)}, &x^2_{(d)},  &x^3_{(d)}, ...,&x^d_{(d)} \cr
                   }
                          \eqno (4)
                      $$
 We denote
                  $$
 \Delta_1= \sqrt {(x^1_{(1)})^2+(x^2_{(1)})^2+(x^3_{(1)})^2+
      (x^d_{(1)})^2}\,,
                    $$
                    $$
    \Delta_2=\sqrt
                        {
              \left(
                 \det
                             \pmatrix
                                    {
                       &x^1_{(1)}, &x^2_{(1)}\cr
                       &x^1_{(2)}, &x^2_{(2)} \cr
                                        }
                                  \right)^2
                               +
                 \dots+
              \left(
                 \det
                             \pmatrix
                                    {
                       &x^{d-1}_{(1)}, &x^d_{(1)}\cr
                       &x^{d-1}_{(2)}, &x^d_{(2)} \cr
                                        }
                                  \right)^2
                     }
                              $$

                    $$
    \Delta_3=\sqrt
                        {
              \left(
                 \det
                             \pmatrix
                                    {
                       &x^1_{(1)}, &x^2_{(1)},&x^3_{(1)}\cr
                       &x^1_{(2)}, &x^2_{(2)},&x^3_{(2)} \cr
                       &x^1_{(3)}, &x^2_{(3)},&x^3_{(3)} \cr
                                         }
                                  \right)^2
                               +
                 \dots+
              \left(
                 \det
                             \pmatrix
                                    {
                    &x^{d-2}_{(1)}, &x^{d-1}_{(1)}, &x^d_{(1)}\cr
                 &x^{d-2}_{(2)}, &x^{d-1}_{(2)}, &x^d_{(2)} \cr
                 &x^{d-2}_{(3)}, &x^{d-1}_{(3)}, &x^d_{(3)} \cr
                                      }
                                  \right)^2
                     }
                              $$
 e.t.c. $\Delta_d$ is nothing that the determinant of the matrix (4).

We consider in $d$-dimensional space the following invariant densities
 the densities
               $$
 A_1=\Delta_1,\quad  A_2={\Delta_2\over\Delta_1^2},  \quad
         A_3={\Delta_3\cdot\Delta_1\over\Delta_2^2},\dots
           \eqno (5)
            $$
It is easy to see that $A_1$ is nothing but the lenghth of the curve,
and it is divergence in the case $d=1$, $A_2$ is the density of
second rank which coincides with (2)
if $d=2$, $A_3$ is the density of third rank which
coincides with (3) if $d=3$.
Sure one can continue this sequence:
In $d$-dimensional case one can consider
ther sequence of the invariant densities
   $\{A_1,...,A_d\}$ such that
the last one is total divergence...

  Do you like it?
Do you know how I come for example to (3)?
                  \bigskip
         \centerline {\bf Section2 continuation}
                     \medskip
  How one can come to the expression (3)


 We recall at first the essence of the formula (2).
If $L$ is the curve in $I\!E^2$ then the tangent unit
 vector to this curve maps the curve $L$ to the curve in $S^1$:
            $$
             \varphi={\rm arctg}{y_t\over x_t}
              \eqno (2.1)
                $$
 and (2) corresponds to the pull back of the volume form $d\varphi$
  on the $S^1$. In the other word the density (2) is constructed via
 the canonical density on the $S^1$ and the density (2.1)
  {\bf which takes values in the unit vectors}---it is
 the composition of the densities.
                  $$
     A_2={d\over dt} {\rm arctg}{y_t\over x_t}
                $$
 Now bearing in mind this interpretation for (2)
we consider for the curve $L$ in the three-dimensional
space the map analogous to (2.1).---Considering
the unit vector to the every point of the curve
 $L$ we transform this curve to the curve
 on the sphere $S^2$:
                         $$
               \theta=
       {\rm arccos}{z_t\over\sqrt{x_t^2+y_t^2+z_t^2}},\quad
                 \varphi={\rm arctg}{y_t\over x_t}
           \eqno (2.2)
                    $$
 We come to the curve on the $2$-dimensional sphere and we apply
 the formula (2) for this curve. Of course the sphere is not
 euclidean but physical intuition say us that in this case
      for the curves on the $S^2$ the analog of (2) will be something like
                $$
      {d\over dt} {\rm arctg}{\sin\theta \cdot\varphi_t\over \theta_t}
                  $$
Combining this density with previous one we come to
 the "answer"
                $$
  A_3 ={d\over dt} {\rm arctg}{\sin\theta \cdot\varphi_t\over \theta_t}=
     {d\over dt}\left (
        {\rm arctg}{\sin
                 \left(
{\rm arccos}{z_t\over\sqrt{x_t^2+y_t^2+z_t^2}}
           \right)
                       \cdot
        \left(
              {\rm arctg}{y_t\over x_t}
                     \right)_t
                 \over
                  \left(
         {\rm arccos}{z_t\over\sqrt{x_t^2+y_t^2+z_t^2}}
             \right)_t}
                 \right)
            \eqno (2.3)
                      $$

 Can you calcule this? I cannot. It is here where I began to think.
 (All previous constructions, you understand,
 do no need paper in spite of
their "artificiel" complicatness...)

 For considering this tremendous expression I consider first the case
 where  in the given point
                 $$
               z_t\big\vert_{t=t_0}=y_t\big\vert_{t=t_0}=0
               $$
 hence in the monstre (2.3)
                    $$
            \theta(t) \big\vert_{t=t_0}={\pi\over 2},\quad
            \varphi\big\vert_{t=t_0}=0
            $$
  Using the multiple time the fact known in school that
               $$
            \left(f(x)x\right)^\prime_{x=0}=f(0)
                    $$
one comes in 15 minutes to the result
                        $$
       A_2= {y_{tt}z_{ttt}-y_{ttt}z_{tt}\over y_{tt}^2+z_{tt}^2}\quad
                {\rm if}\quad
               z_t\big\vert_{t=t_0}=y_t\big\vert_{t=t_0}=0
                     $$
 Now I began to do beautiful tricks  considering
 the global rotations
\def\a{\alpha}
\def\b{\beta}

                $$
         \cases
          {
          x\rightarrow  x\cos\a+z\sin\a\cr
          z\rightarrow -x\sin\a+z\cos\a\cr
                      }\qquad
         \cases
          {
          x\rightarrow  x\cos\b+y\sin\b\cr
          y\rightarrow -x\sin\b+z\cos\b\cr
                      }\qquad
             $$

Where the angles $\a,\b$ are defined by the values
 of $z_t,y_t$ and I came to the (3).
ITo be fair I have to tell you that from the very beginning
 I noticed that determinant of the matrix (4) is the density
 and invariant density and I couldnot see nothing more--
 and everything have to be combined via this determinant?.
but I did not understand how the invariant combinations of minors arise.

 {\bf Conjecture.} Looking on the sequence (5)  I did the following
  conjecture:
     In the $d$-dimensional space one can consider the sequence
       $\{A_1,A_2,\dots,A_d\}$ where
                  $$
   A_1=\Delta_1,  A_2={\Delta_2\over A_1^2},
          A_3={\Delta_3\over A_2^2\cdot A_1^3},
          A_4={\Delta_4\over A_3^2\cdot A_2^3\cdot A_1^4},
                       \dots
         A_d={\Delta_d\over A_{d-1}^2\cdot A_{d-2}^3\cdot
              A_{d-3}^4\dots A_1^d}
                         $$
   In general the recurrent formula maybe is
                                $$
                        A_{n+1}=
      {\Delta_{n+1}\over\prod_{k=1}^{k=n} A_k^{n+2-k}}
              $$
 I have not any proof yet. Of course
   in the way how I constructed the (3)
  I can consider the sequence of the maps
     from the curve $L$ on the $S^{d-1}$ using the unit vectors
 but I affraid now to consider this after my calculations
for $S^2$...

There are the fo following hints that
    this sequence may be is true:

  1) in the first three terms this is right

  2) the dimension is always the same $[{1\over sec}]$

 Sure this has to come from 19 century. Indeed in classical mathematics
 for the curve in $d$-dimensional space there are the sequence of Frenet
  curvatures $\{k_1,...,k_{d-1}\}$. The curvature $k_1$ is proportional
 to $A_2$:
               $$
    k_1={\Delta_2\over\Delta_1^3}={A_2\over\Delta_1}
                $$
  it meausures in $[{1\over cm}]$.
My sequence have to be strictly connected with the sequence
 of curvatures but my goal was to come to divergence-like
 lagrangians. In some sence this means that the last
curvature is proportional to the divergence-like Lagrangian.

  One can consider the $d$-basis in every point of the curve.
The first vector is tangent vector, the first two-dimensional plane
is the plane spanned by the tangent vector and its acceleration
during the motion, e.t.c.--- Hence the movement on the
  $SO(d)$ of the solid particle can be considered.-- I think
it is just what people call rigid particle motion,
 (Armen considered Lagrangians depended on higher curvatures.)
  I think that all these formulaes may be in the other
 interpretation have to be revealed in rigid particle physics.
   In fact I sent the beginning of this text to Armen,
  and still do not know his reaction.
  Do you want to know why and how I spent yesterday
 morning and day on this calculations?
          S Vas butylka!---Togda skazhu!

            Do svidanija O.M.

>>>>>>>>>>>>>>>>>>>>>>>>>>>>>>>>>>>>>>


    $$ $$

\def\p {\partial}
\def \D {\Delta_{d{\bf v}}}
\def \Ds  {\Delta^{\#}}
\def\t{\tilde}
\def\s {\sigma}
\def\L {\Lambda}
\def\Darboux {$z^A=$  $x^1,\dots,x^n$, $\theta_1,\dots,\theta_n$}
\def\a{\alpha}
\def\O{\Omega}
\def\d{\delta}
\def\dv  {{d{\bf{v}}}}
\def\A {{\cal A}}
\def\R {I\!R}
\def\t {\tilde}
\centerline {\bf 2.4 Invariants of Hypersurfaces in Euclidean Space}

   \medskip


   Recently Nikita Sidorov (l`homme Nikita) gave me the following
   school problem: Let $L$ be a closed (convex) curve and a curve $L_{h}$
   is set of points which are on the distance $h$ from $L$
   ($\forall x\in L_h$ $\min_{y\in L}d(x,y)=h$). How to calculate
    the surface of the domain between these curves?
    There is a "school" solutions if we consider curve $L$ as a
    limit of "nogougoljnikov". But I solved in more complicated way
    and my solution can be generalized on $n$-dimensional case.

    Let $(x,y)$ be Cartesian coordinates on $I\!R^2$ and $t$ be any coordinate
    on $L$. Then we consider
   new coordinates $(\rho,t)$ in a tubular neighborhood
    of $L$ such that for every point $A\in I\!R$
                        $$
    \rho(A)=\hbox {distance between $A$ and $L$}=\min_{y\in L}d(A,y)
                                          \eqno (2)
                      $$
 and $t(A)$ be $t$ coordinate of the point $y_0$ such that
      $d(A,y_0)=\rho(A)$.

      One can see that a function $\rho(x,y)$ obeys to equation
                          $$
                       \rho_x^2+\rho_y^2=1,\quad \rho\big\vert_L=0
                                 \eqno (2)
                                 $$
  Moreover if ${\bf r}_0(t)=(x_0(t),y_0(t))$ is an equation of $L$
  then in a vicinity of $L$
                      $$
  {\bf r}(t)={\bf r}_0(t)+\rho{\bf n}(t)\,,
                                 \eqno (3)
                       $$
  where ${\bf n}(t)$ is unit vector normal to $L$ at the point ${\bf r}_0(t)$.
                   $$
             {\bf n}(t)=
             \left\{-{x_t\over\sqrt{x_t^2+y_t^2}},
             {y_t\over\sqrt{x_t^2+y_t^2}} \right\}
             \eqno (4)
            $$
 Calculate Jacobian of coordinate transformaton from Cartesian coordinates
 $(x,y)$ to these new coordinates.
 According to (3,4)
                        $$
                        \cases
                        {
           x(\rho,t)=x_0(t)-\rho{x_t\over\sqrt{x_t^2+y_t^2}}\cr
           y(\rho,t)=y_0(t)+\rho{y_t\over\sqrt{x_t^2+y_t^2}}\cr
                             }
                             \eqno (5)
                             $$
Straightforward calculations (here straightforward still dimension is $2$!)
  give that
                        $$
          \det\left({\p(x,y)\over\p(\rho,t)}\right)=
          \sqrt {x_t^2+y_t^2}+\rho{x_t y_{tt}-y_t x_{tt}\over x_t^2+y_t^2}=
           \sqrt {x_t^2+y_t^2}+\rho {d\over dt}{\rm arctg}{y_t\over x_t}
                   $$
  the first term is an element of the length of $L$.
  The second term is Frenet curvature and it is full derivative.

Now solution follows from the last formula:
The surface of the domain between curves is equal to the
                   $$
                  \int_{L,L_h} dx dy=
                  \int  \det\left({\p(x,y)\over\p(\rho,t)}\right)
                              d\rho dt=
                       $$
                       $$
             \int_{0}^h d\rho
             \int_L dt\left(
\sqrt {x_t^2+y_t^2}+\rho {d\over dt}{\rm arctg}{y_t\over x_t}
             \right)=
              h|L|+\pi h^2
               $$
  Now we try to generalize this result.
  Of course the main problem will be to calculate jacobian
  and we are not so naive to hope that everything will be
 exhausted by total derivatives.

 So go on...

 Let $M^n\hookrightarrow \R^{n+1}$.
Let $x^\mu$ $\mu=0,1,...,n$ be cartesian coordinates on $I\!R^{n+1}$.
  and $f^\mu=f^\mu(t^1,\dots,t^n)$ be local parameterization of hypersurface
   $M^n$. Conisder $\rho({\bf r})=\min d({\bf r}, M)$.
   ${\rm grad} \rho^2=0$ and $\rho_M=0$. It is ray.
 In the same way as before consider transformation
            $$
    x^\mu(\rho,t)=f^\mu(t^1,\dots,t^n)+\rho n^\mu
   $$
  where ${bf n}$ is normal unit vector.
Consider any point $\lambda_0$ on $M^n$ with coordinates $t^1_0,\dots,t^n_0$,
 $x^\mu_0=f^\mu(t^1_0,\dots,t^n_0)$.

Conisder cartesian coordinates $\tilde x^\mu$ and parametrization $\tilde t^i$
 {\rm adjusted to the point $\lambda_0$},
 i.e. such that in a vicinity of $\lambda_0$
           $$
           \cases
            {
            \tilde x^0=A_{ik}t^it^k+o(t^2),\cr
            \tilde x^i=t^i \cr
               }
               $$
  $A_{ik}$ corresponds to bilinear symmetric form on hypersurfaces
  which is defined on tangent vectors and takes values
  in normal vectors.

        It is not awfull to calculate Jacobian
        of transformation from $\rho,tilde t$ to $\tilde x$
        in the point $\lambda_0$. Answer is
                         $$
        \det\left({\p(\t x^0,\t x^i)\over\p(\rho,\t t)}\right)=
                     \det\left(\delta_{ik}+\rho A_{ik}\right)
                         $$
  We note that jacobian of transformation from cartesian coordinates
  to another is equal to one, and jacobian of transformations from
   parameters $t$ to another parameters is proportional to
   the surface density. Hence
                      $$
             \int dx^0\dots dx^n=
             \int \det \left(
               1+\rho A
             \right)dsd\rho
              $$
Expanding determinant by powers of $\rho$ we come to series
of densities on $M$:
               $$
             \int \det \left(
               1+ A
             \right)ds=
             \int ds+\int TrA ds+....+\int\det A ds
               $$

 The first density is surface density , next one is a mean curvature,
 the last density is a total derivative-it is nothing but pull-back of
 volume form on unit sphere $S^{n}$ under the Gauss map of $M^n$
 embedding in $\R^{n+1}$
 in $S^n$ (to every point on $M^n$ corresponds point on $S^n$ via
 unit normal vector ${\bf n}$).

Now we rewrite it in dual notations.
let surface $M$ is given by equation:
            $$
            \Phi=0
             $$
During calculations I received the following answer dual
to previous:
                     $$
                 \int
                 \det
                 \left[
                    \delta_{ik}+
                    \Pi^\mu_i
                 {\p^2 \Phi\over \p x^\mu\p x^\nu}
                     \Pi^\nu_k
                     \right]
                     \cdot
                     |{\rm grad}\Phi|
                      \delta(\Phi)
                    d^{n+1}x\,,
                     $$
where
                      $$
 |{\rm grad}\Phi|=\sqrt
                       {
          {\p\Phi\over \p x^\mu}
            {\p\Phi\over \p x^\mu}
                   }
                    $$
But one can write the following answer too:

 Density of the weight zero:
                $$
                                \det
                 \left[
                    \delta_{ik}+
                      {1\over |{\rm grad}\Phi|^d}
                    \Pi^\mu_i
                 {\p^2 \Phi\over \p x^\mu\p x^\nu}
                     \Pi^\nu_k
                     \right]
                     $$
          I do not understand this answer.
          I forget how I receive all these dual objects
          I forget what is it exactly $\Pi^\mu_i$
       (Immediately after these calculations we loosed lagguage
        of my son).
        But I have to remember it!


     >>>>>>>>>>>>>>>>>>>>>>>>>>>>>>>>>>>>>>>>>>>

     $$ $$
         \centerline {\bf 2.5 Spirality of particles, Screw in mechanics
      and one invariant of motions in $E^3$}
\bigskip
               (Zadacha voznikla pri obsuzhdeniji s Martirosianom Ashotom o mekhanike)

\medskip

  Consider affine space $E^3$, fix a point $O$ (nachalo koordinat)
  we come to linear space $I\!R^3$. To arbitrary motion $\cal A$ in $E^3$
  corresponds a pair $(A,b)$ where $A$ is orthogonal transformation
  of $I\!R^3$ and $b$ is a vector:
                    $x^\prime=  Ax+b$
If we consider another nachalo koordinat $O^\prime$
such that vector $OO^\prime$ is equal to $r$ then
to the motion $\cal A$ corresponds a pair $(A,b^\prime)$:
$y^\prime=  Oy+b^\prime$
where $x=y+r$
and
             $$
          b^\prime=b+Ar-r
          \eqno (1)
              $$
       At what extent vector $b$ is defined iniquely
       for a given motion $\cal A$ under a changing of nachalo coordinate?
       Let $\bf n$ be axis corresponding to orthogonal
       operator $A$: linear transformation via $A$ is rotation
       around axis $\bf n$, or in another words $\bf n$
       is eigen vector of $A$: $A\bf n=n$

 It is easy to see from (1) that
   the component of vector $b$ orthogonal to $\bf n$ can be vanished
   by suitable choice of $r$ and
   component of $b$ parallel to axis remains unchanged.
   Formally it follows from the fact that image of operator
   $A-{\bf id}$ is two-dimensional space ($\det(A-1)=0$)
   orthogonal to $b$.

   We come to the invariant of motions: projection of $b$ on $\bf n$.


   {\bf Statement}

   Every motion of $E^3$ can be expressed as rotation over axis
   and translation along this axis, i.e. screw movement
   (vintovoje dvizhenije).

   Shag vinta--projection of $b$ on $n$ is invariant.

   Infinitezimally to this invariant corresponds
   invariant in affine algebra of rotations and translation:
   scalar product $\bf Lp$ of operators of angualr moment and moments.

 This simple statement can be considered as geometrical
 interpretation of spirality...


       > > >>>>>>>>>>>>>>>>>>>>>>>>

$$ $$
\def \a {\alpha}
\def\s {\sigma}

\def \F {{\cal F}}
\def \L {{\cal L}}
\def \D {{\cal D}}
\def \N {{\cal N}}
\def\p{\partial}
\def\d{\delta}
\def \P {{\cal P}}
\def \Pi {{\cal Pi}}
\def \V {{\cal V}}
\def\R {\bf R}
\def \O{\Omega}

\centerline {\bf 2.6 Space of planes...}

\medskip

  {\it The most exciting point of this etud is that
   the space of planes which are tranvesal
   to the given plane $X$ in the linear space $E$
   is the space of projectors
   on the $X$ and this space is associated to the
   linear space of operators on $E$ which
    vanish {\bf on} $X$ and which take values {\bf in} $X$.
    This space appeared in proving the Theorem of
    irredusible representations for semisimple algebras
    via Whitehead lemmas...}

                 $$ $$
 \centerline{  The description  of lagrangian surfaces transversal to the
      given one.}

   $$ $$

   Let $X$ be given lagrangian surface in linear symplectic space.

   How to describe the space of planes transversal to it?

   \smallskip

   First of all we consider the space of all transversal planes (ignoring the symplectic structure).


   Let $Y_0$ be any transversal plane. Consider the pair $(X,Y_0))$. If $Y$ is any
   other plane transversal to $X$  then one can consider linear operator
    $A\colon\quad Y_0\rightarrow X$ such that
                   $$
                   Y=Y_0\oplus AY_0
                    \eqno (1)
                   $$
   The operator $A$ defines uniquely the operator $W$ on $E$ such that
    $W\Big\vert_{X}=0$. This operator obeys to relation
                  $$
                  {\bf\rm Im}W\subset X\subset {\bf\rm ker}W
                     \eqno (2)
                  $$


     {\bf Proposition 1}
     The space of surfaces transversal to $X$ is affine space. The associated linear
     space is the linear space of operators $W\in {\rm Hom}(E,E)$ obeying to (2).



    In fact the best way to describe the affine space of transcversal surfaces it is to use
     projection operators for which $X$ is kernel, or for which $X$ is identity domain.

     Let $P_0$ be a  projection operator
  corresponding to the plane $Y_0$ transversal to $X$, i.e.
  $P_0\big\vert_{Y_)}={\bf \rm id}$ and ${\rm ker}P_0=X$
     (or $\tilde P_0=1-P_0$ is the projection operator such
      that ${\rm ker}\tilde P_0=Y_0$ and $\tilde P_0\big\vert_X={\bf \rm id}$).
          It is easy to see that projection operator corresponding to the
    the plane $Y$  is equal to the
                      $$
                  P=P_0+W
                       $$

{\bf Proposition 2}
     The affine space of planes transversal to $X$ is
     the affine space of projectors of $E$ on the $X$.
     For arbitrary projector $P_0$ of $E$ on the $X$
     $P$ is    projector $P_0$ of $E$ on the $X$ also,
     if and only if $W=P-P_0$ belongs to the linear space (2)


  {\bf Remark} What is the meaning of the cohomology of (2) ($W^2=0$).

   $$ $$

   Now we return to symplectic and lagrangian structure.
This put additional conditions on $W$ and $P$.

The operator $A$ in (1) is the linear operator on Lagrangian surface $Y_0$
 with values in Lagrangian surface $X$.
 Correspondingly the operator $W$ in (1) is the linear operator the whole symplectic
 space $E$ into itself such that it takes values
 with values in Lagrangian plane $X$ which belongs to its kernel.
 Identifying $E$ with dual space $E^*$ (via symplectic product) we put in
 correspondence to the linear operator $W$ the {\bf bilinear form} $\O$:
                               $$
                               \O(a,b)=<W(a),b>
                                  \eqno (4)
                                $$
    Correspondingly identifying lagrangian plane $X$ with dual to $Y_0$
    we put in correspondence to $A$ in (1) the bilionear form on $Y_0$.
  From (1) it follows  that $<Au,v>+<u,Av>=0$, for arbitrary $u,v,\in Y_0$,
  because $Y$ is lagrangian plane also. Correspondingly for $W$
  $<Wu,v>+<u,Wv>=0$, for arbitrary $u,v,\in E$.
  Hence $\O$ is bilinear symmetric form.


 {\bf Proposition 1L}
     The space of lagrangian planes transversal to
     the given lagrangian plane $X$ in the linear symplectic
     space  is affine space.
     The associated linear
     space is the linear space of symmetric bilinear forms
     on $E$ which vanish on $X$ (forms on $E/X$).

   {\bf Remark}. In component language it is nothing but
    constructing lagrangian surface $Y$ as graph of the linear function
    generating by quadratic function on $Y$.


  One can see easily that condition that $X$ and $Y$ are lagrangian put the following
   constraints on the projector $P$:
                              $$
                   <Pu,v>+<u,Pv>=<u,v>
                                  \eqno (5)
                              $$
Note that from (2) it follows that
 $<Wu,v>+<u,Wv>=0$ $<Wu,v>=<Wv,u>$.
� �ݣ �������, ��� (5) ��� � ����������!!!

{\bf Proposition 2L}
     The affine space of lagrangian planes transversal to
      the lagrangian plane $X$ is
     the affine space of projectors of $E$ on the $X$,
      obeying to (5).
     For arbitrary such projector $P_0$ of $E$ on the $X$
     $P$ is  projector $P_0$ of $E$ on the $X$ which also
       defines transversal lagrangian plane
     if and only if
      $W=P-P_0$ belongs to the linear space (2)
      and $<W...,...>$ is symmetric.

               $$ $$
                 \def \sint {\int_1^\infty {\exp(-at)\over t}dt}
                    $$ $$
            \centerline{\bf 2.7 One method of Calculating
       Integral   $\int_1^\infty {\exp(-at)\over t}dt$
                 for small $a$}

            \centerline{\bf and Seeley Formulae}
                        \bigskip
            \centerline{\bf O.M. Khudaverdian}
                     \medskip
  We consider calculating of the integral $\sint$
 using ideas inspired by Seeley formulae.
  First the calculation. For estimating the behaviour
of $\sint$ for large $a$ one can consider integration by parts
                      $$
     \sint={e^{-a}\over a}\sum_{n=0}^N{n!(-1)^n \over a^n }
         +{(-1)^N N!\over a^{N+1}}\int_1^\infty e^{-at}t^{N+1}
                                             \eqno (1)
                        $$
{\bf Remark} One can see that this formula does not give
convergent power series. On the other hand for every fixed $N$
 the reminder term is $e^{-a}o(1\over a^{N})$.

 \bigskip


 For small $a$ the Eq.(1) is helpless.

  We consdider the trivial identity
                     $$
         a^{-s}=
              {1\over \Gamma(s)}
       {\int_0^\infty t^{s-1}\exp(-at)dt}
                \quad {\rm for} \quad s>0
                                        \eqno (2)
                    $$

     One can rewrite the l.h.s. of (2) in the way
  to consider its analytical continuation for
neibourhood of $s=0$
                     $$
         a^{-s}=
              {1\over \Gamma(s)}
       {\int_0^\infty t^{s-1}\exp(-at)dt}
                   =
                   $$
                  $$
              {1\over \Gamma(s)}
            \left(
       {\int_0^1 t^{s-1}\left(\exp(-at)-1\right)dt}+
       {\int_0^1 t^{s-1}dt}+
       {\int_1^\infty t^{s-1}\exp(-at)dt}
                     \right)
                    $$
                     $$ =
              {1\over \Gamma(s)}
                         \left(B(s)+
              {1\over s}\right)
                                    \eqno (3)
                    $$
               where we denote
                          $$
                 B(s) =
            \left(
       \int_0^1 t^{s-1}\left(\exp(-at)-1\right)dt+
         \int_1^\infty t^{s-1}\exp(-at)dt
                     \right)
                        $$

       Hence
                         $$
         a^{-s}=
              {1\over \Gamma(s)}
                          \left(
                         B(s)+
              {1\over \Gamma(s+1)}\right)
                                         \eqno (4)
                      $$
   The both parts of (3) are defined and can be differentiate
 in the neibourhood of $s=0$.
           Using that
                     $$
              {1\over \Gamma(s)}\big\vert_{s=0}=0
                \quad {\rm and}\quad
           \left({d\over ds}
              {1\over \Gamma(s)}\right)\Big\vert_{s=0}=
                    \left(
                    {d\over ds}
                    \left(
              {s\over \Gamma(s+1)}\right)\right)\Big\vert_{s=0}=1
                                                \eqno (5)
                       $$

    we come from (4) to
                  $$
         {d\over ds}a^{-s}\big\vert_{s=0}= -\log a=
              B(0)- \Gamma^\prime (1)
                                    \eqno (6)
                      $$
 the derivative  $\Gamma^\prime (1)$  is the famous
  Euler constant
                         $$
            \Gamma^\prime (1)=
                          \lim_{n\rightarrow\infty}
                      \left(
                      1+{1\over 2}
                         +{1\over 3}+
                          {1\over 4}+\dots+
                          {1\over n}-\log n
                      \right)
                                     \eqno (7)
                              $$

             We see from (2---6) that

                    $$
            B(0)=
       \int_0^1 {\exp(-at)-1\over t}dt+
       \int_1^\infty {\exp(-at)\over t}dt=
       \Gamma^\prime(1)-\log a
                                       \eqno (8)
                   $$
        This gives representation for $\sint$
  which is  convenient for small $a$
                      $$
      \int_1^\infty {\exp(-at)\over t}dt=
       \Gamma^\prime(1)-\log a-
       \int_0^1 {\exp(-at)-1\over t}dt=
                  $$
              $$
       \Gamma^\prime(1)-\log a-
          \sum_{n=1}^\infty {(-1)^na^n\over n!n}=
                           \eqno (9)
                     $$
                     $$
     -\log a+\Gamma^\prime(1)+a-{a^2\over 2\cdot 2!}+
       {a^3\over 3\cdot 3!}-{a^4\over 4\cdot 4!}+\dots
                     $$
   (where $\Gamma^\prime(1)$ is nothing but Euler constant $C$:

   \noindent $\Gamma^\prime(1)=C=
   \lim_{n\rightarrow \infty}
   (1+{1\over 2}+{1\over 3}+\dots+{1\over n}-\log n)$)
   Empirically $B(0)$ in (8) can be considered as
  renormalized value of the integral
         $$
         \int_1^\infty t^{s-1}\exp(-at)dt=\dots
                         \eqno (10)
              $$

      And naively (which happens often in Quantum Field Theory)
      one can calculate it using
   Frullany formula:

                       $$
                \int {f(at)-f(bt)\over t}dt=f(0)\cdot\log{b\over a}
                             $$

  The exact result
  differs  from naive expectation on the Euler constant !!!.

   In fact all this stuff is the reduction of the Seeley formulae




   \bigskip

   I think this very trivial exercise is useful

   shtoby vvesti
populjarno ideologiju perenormirovok v formulakh Sili (Seleey)




{\bf Remark 2}

Denote $I(a)=\sint$ and
$K(a)=\sum_{n=1}^\infty {(-1)^na^n\over
n!n}$

 We note that
                $$
            \Gamma^\prime(1)=\int_0^\infty \log t e^{-t}dt=
\int_0^1 \log t e^{-1}dt+\int_1^\infty \log t e^{-1}dt=
         K(1)+I(1)
           $$
 hence we come to
 the identity at $a=1$. Helas our formula does not give
 series for $a=1$...

 {\bf Remark 3} Tut vsjo zatsepleno: Naprimer
                       $$
                       \int_0^1 {e^{-at}-1\over at}dt=
                       \int_0^1 \log t e^{-at}dt
                       $$
            The second integral is related with $\Gamma^\prime(1)$.


$$ $$

    \centerline {\bf 2.8 K zadache o kreste}

\medskip

 Related with etud jakob.tex and wanted to remember
 my very old solution of the squares (see square.tex)
I was thinking on the non-linear
map with Jakobian 1.
It is following.

Let $R^n$ be $n$-dimensional space.
We consider the following non-linear map
              $$
   \xi^i=x^i+ a^i F(b^mx^m)
                 \eqno (1)
              $$
where $F(t)$ is a smooth function, $\bf a$
and $\bf b$ are orthogonal vectors.
The differential of this map is matrix
                $$
           \delta^i_m+a^ib^mf^\prime(B^mx^m)
               \eqno (2)
                  $$
It is easy to see that its determinant is equal to $1$.
(The matrix in (2) is equal to $1+S$ where $S$ is nilpotent $S^2=0$,
because vectors $a$ and $b$ are orthogonal, hence
                 $$
\det(1+S)=\det\exp S=\exp{\rm tr}S=1
                   \eqno (3)
                 $$
The transformation (1) is exponential flow of the vector field
\def\p{\partial}
              $$
           {\bf X}=a^iF(b^m x^m){\p\over\p x^i}
                     \eqno (4)
                 $$
and its integral of motion is $b^mx^m$.
In fact I wanted to construct non-linear map
with jakobian 1 which transforms
two families of lines in the lines.---It is impossible!
One can to construct non-linear map with linear
jakobian with these properties:
           $$
           \cases
               {
         x\mapsto x+axy\cr
         y\mapsto y+bxy\cr
                   }
                  \eqno (5)
               $$
This map transforms the square $(0,0), (0,1), (1,0), (1,1)$
to the quatreangle   $(0,0), (0,1), (1,0), (1+a,1+b)$
and all vertical and horizontal lines to the lines also!
Moreover its jakobian is equal
 to
                   $$
             1+2ay+2bx
                            \eqno (6)
                $$
hence  the middle square $(1/3,1/3), (1/3,2/3), (2/3,1/3), (2/3,2/3)$
  whose area is equal to $1/9$ transforms to the
 quatreangle with area wich is $1/9$ part of the area of the
 quatreangle  $(0,0), (0,1), (1,0), (1+a,1+b)$ .

This I did in 1989 in Geneve...
Yesterday I spent 10 hours wanted to reconstruct thsi solution.
My false idea was to find transformation like (5) with Jakobian 1,
but it is impossible, because only
middle quatreangle preserves its property to be $1/9$
part of all! Why? because
                  $$
             \left({2\over 3}\right)^2-
             \left({1\over 3}\right)^2=
               {3\over 9}=
               {1\over 3}
                   \eqno (7)
                      $$
 This is proptery of middle segment...

   $$ $$

                        {\bf 2.9 Dva slova o jacobian problem}

                        \medskip

                      It is a little note about so called Jacobiin problem.

  Let $P(x,y)$ and $Q(x,y)$ be two polynomipals on variables $x,y$
such that
\def\p{\partial}
                     $$
                   \det
                    \pmatrix
                     {
               {\p P\over \p x}
                   {\p P\over \p y}\cr
                    {\p Q\over \p x}
                    {\p Q\over \p y}\cr
                      }
                         =1
                       \eqno (1)
                   $$
Then the inverse functions are always polynomials too.
For example to the polynomial transformation
                  $$
                 \cases
                     {
                    x\mapsto x+(x+y)^3\cr
                    y\mapsto x+y  \cr
                        }
                      $$
 corresponds inverse transformation
                           $$
                \cases
                   {
                    x\mapsto x-y^3\cr
                    y\mapsto y+y^3-x\cr
                      }
                            $$
As is stated in the book of Kirillov: ("Shto takoje chislo"---broshjura,
1993 god) this problem is unsolved till now. (Kak ja ponimaju net
kontrprimera i ne dokazano eto utverzhdenije.)
One can following to this book
consider non-commutative version of this problem.
Let $W$ be associative algebra with unity, generated by two generators
  $p$ and $q$ which obey only to the constraint:
                    $$
                       pq-qp=1
                      $$
  (Weyl algebra)
Let $A$ and $B$ be two polynomials on $p$ and $q$ such that
            $$
         AB-BA=1
                                  \eqno (2)
             $$
Then the endomorphism generated by the map
                 $$
           p\mapsto A, q\mapsto B
                  $$
  has to be an isomorphism!---i.e. $p$ and $q$ can be expressed as
polynomials of $A$ and $B$.
You see that the first problem is quasiclassical limit of the second one
(commutator (2) $\rightarrow$ Poisson bracket $=$ jakobian.)

$$ $$ $$ $$

         \centerline {\bf Algebra}

      $$ $$

       \centerline {\bf 3.1 Representation functor for jets}
\medskip

 Reading Krassilshik Verbovetsky I became really  really
 happy understanding the algebraic definition of jets!!!
\def \K {{I\!K}}
\def\D  {\Delta}
   So on:
   Let $P$ and $Q$ be two modules over commutative algebra $A$
 which is algebra over the main field $\!K$. In the space
     linear space $Hom_\K(P,Q)$ one can consider two modules structure.
The first one defining by
                      $$
  \forall \D \in Hom_\K(P,Q), a\in A\quad
            (a\circ\D)(p)=a\D(p)
                  \eqno (1)
                    $$
and the second one defined by
                      $$
  \forall \D \in Hom_\K(P,Q), a\in A\quad
            (a\circ\D)(p)=\D(ap)
                  \eqno (1)
                    $$
It is easy define the gradation. ($Diff=\oplus Diff_n$)
   We denote the first module by $Diff(P,Q)$ and the second one by
  $Diff^+(P,Q)$. The elements of $Hom_\K(P,Q)$ which are in fact
  $A$-linear (i.e. belong to $Hom_A(P,Q)$) are zero order differential
operators.   By induction the operator $\D$ belongs to
 $Diff_{n+1}$ if for every $a\in A$
                             $$
             \delta_a(\D)= \D a-a\D\in Diff_{n}
                   \eqno (2)
                         $$
 Sure the basic example is (see also the example below!)
the following

         {\bf Example 1}
  Let $P$ and $Q$  be the modules of sections (more simple
  the modules of functions) and $A$ be an algebra
 of functions. Then $Diff_n$  is nothing but $n$-order
differential operators.

 But there is another example which is very useful for jets definition!!!

     {\bf Example 2} Let $P$ be a module (simple: the module of functions)
 and  $A$ be arbitrary algebra (simple: algebra of functions)
   Let
               $$
              Q=A_K\otimes P
                                    \eqno (4)
                 $$
(note that tensor product is over $A$ not over $K$!!!)
and module structure on $Q$ is defined by
                  $$
               b\circ (a\otimes p)=ba\otimes p
                    $$
 We consider the map
                     $$
Hom_\K(P,Q)\ni\epsilon\colon\quad \epsilon(p)=1\otimes p
                          \eqno (5)
           $$
 Exercise
 This is infinite-order differential operator!!!

And this leads us to jets (see Krasilshik Verbovetsky)

 A imenno:

 We consider $\mu^k$---the subspace of $Q=A_K\otimes P$
which is generated by terms
 $\delta_{a_0\dots a_k}(\epsilon)(p)$
where $\delta_{a_0\dots a_k}(\epsilon)(p)$
are generated subsequently by commutators by the rule (2).
We define jets space by the formula
               $$
            {cal I}^k P=(A_K\otimes P)/mu^k
                    $$
 Magic definition!
     Exercise
We consider the functor $Diff_n(P,\cdot)$:
                    $$
         \forall Q ({\rm over A})\quad
             Q\rightarrow Diff_n(P,Q)
                   $$
i.e. the functor which put in correspondence to the module
 $Q$ $n$-th order differential operators with value in this module.
This fumctor is representing by the space
                   $$
                  Hom_A({\cal I}^n P,Q)=Diff_n(P,Q)
               $$
i.e. differential operators on $P$ with value in $Q$ are representing
by $A$-linear functions on the   space of $n$-th jets
 with value in $Q$!





 $$ $$
\centerline {\bf 3.2 Projective module of Mobius band global sections}

   \bigskip

 Let $C$ be an algebra of continuous functions on $[0,2\pi]$.
  Consider the following two subalgebras of $C$
                          $$
             \Lambda=\{f\colon\,\, f\in C, f(0)=f(2\pi)\}\,
                                     \eqno (1)
                           $$
 and
                           $$
              P=\{f\colon\,\, f\in C, f(0)=-f(2\pi)\}\,
                                     \eqno (2)
                           $$
 $P$ can be considered as module over $\Lambda$
 This module is projective and it is not free. More precisely:
                               $$
                           P\oplus P=\Lambda\oplus \Lambda
                                         \eqno (3)
                               $$
We give a proof of (3) below. Equation (3) menas that $P$ is
projective (by definition). From (3) it follows that $P$ is not
free. Indeed suppose that it is free then from embedding (3) it follows
 that it has one generator $f_0$. From (2) it follows that
  $f_0$ vanishes in some point $x_0$. Hence $f_0$ generates
  submodule of $P$. Contradiction.

  Before proving (3) we note that $P$ is nothing but
  module of global continuous sections on Mobius band.
 Condition (3) corresponds to condition that
 Whithney sum of two Mobius bundles is trivial bundle:
                      $$
           M\oplus M=R^2\times S^1
                          \eqno (4)
                      $$
The proof follows from the following geometrical realization
 of (4). Consider in $R^4$ two Mobius bands, bundles
 over circle:
                         $$
                          M_1\colon
                          \cases
                          {
                          x=cos\varphi \cr
                          y=sin\varphi\cr
                          z=tcos {\varphi\over 2}\cr
                          u=tsin {\varphi\over 2}
                           }\,,\qquad
                            M_1\colon
                          \cases
                          {
                          x=cos\varphi \cr
                          y=sin\varphi\cr
                          z=-\tau sin {\varphi\over 2}\cr
                          u=\tau cos {\varphi\over 2}
                           }\,,\quad
                           0\leq\varphi \leq 2\pi, -\infty<t<\infty,
                           -\infty<\tau<\infty
                            \eqno (5)
                          $$
   It is evident that this embedding leads to (4). Now from (5)
   it follows the proof of (3).
   The isomorphism (3) is given by the formula
                     $$
                  \rho \pmatrix {t(\varphi)\cr \tau(\varphi)\cr}=
                  \hat T_{\varphi\over 2}
                  \pmatrix {z(\varphi)\cr u(\varphi)\cr}\,,
                  $$
       where $\hat T$ is operator of rotation:
                      $$
                      \hat T_\varphi
                        =
                        \pmatrix
                          {
                          cos \varphi,sin \varphi\cr
                          -sin \varphi,cos \varphi\cr
                          }
                          $$
        To isomorphism (3) corresponds the following projector
        in $R^2\times S^1$ on M:
                         $$
         \Pi (z,u,\varphi)=\hat T_{\varphi\over 2}\circ
                \pmatrix
                          {
                          1\quad 0\cr
                          0\quad 0\cr
                          }\circ
                          \hat T_{-\varphi\over 2}\,,\quad
                          1-\Pi (z,u,\varphi)=\hat T_{\varphi\over 2}\circ
                \pmatrix
                          {
                          0\quad 0\cr
                          0\quad 1\cr
                          }\circ
                          \hat T_{-\varphi\over 2}\,
                          $$

   Zabavno otmetitj shto etot proektor imejet vneshne bezobidnyj vid:
                            $$
                  \Pi={1\over 2}\left(
                     1+\pmatrix
                          {
                          1\quad 0\cr
                          0\quad -1\cr
                          }\hat T_\varphi
                  \right)
                      $$

Eto v tochnosti Fedosovskoje opisanije rasslojenij!



\bye
