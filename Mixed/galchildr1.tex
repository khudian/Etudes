
       \magnification=1200
   \baselineskip 17 pt

\def\V {{\cal V}}
\def\s {{\sigma}}
\def\Q {{\bf Q}}
\def\D {{\cal D}}
\def\G {{\Gamma}}
\def\C {{\bf C}}
\def\M {{\cal M}}
\def\Z {{\bf Z}}
\def\U  {{\cal U}}
\def\H {{\cal H}}
\def\F {{\cal F}}


   \centerline  {\bf Two words about values of $\sin 18^{\circ}$, $\sin 6^{\circ}$,
   $\sin 20^{\circ}$ and in general $\sin k^{\circ}$}

 \medskip

                   \centerline {${\cal x}$0. {\it Deux ex machina}}

           Calculation of $\sin 6^{\circ}$ and $\sin 18^{\circ}$:

   There are different ways of calculation of $\sin 18^{\circ}$.
   The most beautiful one is
    considering  triangle with angles $72^{\circ}$, $72^{\circ}$,
     $36^{\circ}$. We only represent the result
                       $$
     \sin 18^{\circ}={\sqrt 5-1\over 4}
                                  \eqno (0.1)
                       $$
       and note that $u=2\sin 18^{\circ}$ is the golden ratio:
                        $$
                      u^2+u=1
                  \eqno (0.2)
                        $$

  Using this result it is easy to calculate $\sin 6^{\circ}$.
    Indeed it is evident that
                              $$
                 \sin 6^{\circ}\cdot\cos 24^{\circ}={1\over 2}\sin 18^{\circ}+
                         {1\over 2}\sin 30^{\circ}=
                         {1\over 2}\sin 18^{\circ}+{1\over 4}
                                $$
     On other hand
                                 $$
     \cos 24^{\circ}-\sin 6^{\circ}=\cos (30-6)-\sin 6=\cos (30+6)=1-2\sin^2 18.
                                 $$
          We come to simultaneous equations:
                          $$
                          \cases
                          {
\cos 24^{\circ}-\sin 6^{\circ}=1-2\sin^2 18\cr
                 \sin 6^{\circ}\cdot\cos 24^{\circ}=
                 {1\over 2}\sin 18^{\circ}+{1\over 4}\cr
                        }
                        $$
This system is quadratic equation on $\sin 6$ and coefficients we know from
          (0.1).

  Everything alright except how we guess about this anzatz?

  We try to explain it in next  2 sections
  in the 3 Section we briefly show that $\sin 20$ and $\cos 20$
  is not expressible in square radicals and
  in the fourth section we show that
  angle (and its sinus and cosinus  of angle $k$ are expressible
  in square radicals if and only if this angle is divisble on $3$).

  We formulate by the way the most beautiful result
  of geometry about dividing the circle on equal parts
  using ruler and compass.


\bigskip

  \centerline {${\cal x}$1. {\it Model example}}

\medskip


     We use in fact Galois theory for pedestrians.

           {\bf Statement.1} $\sin 18^{\circ}$ is a root of quadratic polynomial
           with integer coefficients.



       Consider equation
                         $$
                       \sin 5\varphi=\sin\varphi_0
                                      \eqno (1)
                         $$
      and obvious identity
                      $$
             \sin 5\varphi=16\sin^5\varphi-20\sin\varphi+5\sin\varphi
                                     \eqno (2)
                       $$
          It follows from (1) and (2) that equation
                        $$
             16t^5-20 t^3+5t-a=0, |a|\leq 1
                          \eqno (3)
                         $$
      has five real roots (in general not distinct):
                       $$
           t_k=\sin(\varphi_0+72^{\circ}k),
                    \quad k=0,1,2,3,4\,,{\rm where}\,\,
                    \varphi_0\colon  a=\sin 5\varphi_0
                                       \eqno (4)
                         $$
 because $5\times 72^{\circ}=360^{\circ}$.

  Conisider $a=1$. Then $a=1=\sin (5\cdot 18^{\circ})$ and
  from (3) it follows that for this case
  equation (3) has following roots:
                           $$
        <\sin 18^{\circ},1,\sin 18^{\circ},\sin 54^{\circ},\sin 54^{\circ} >
                                     \eqno (5)
              $$

 Polynomial (3) has one rational root (at least) Hence
                       $$
   P_5(t)=16 t^5-20 t^3+5 t-1= (t-1)P_4(t)=(t-1)(16t^4+at^3+bt^2+ct+d)
                 \eqno (6)
                       $$
       where $a,b,c,d$ are {\bf rational numbers}.

       Sure we can calcualte $a,b,c,d$ but at this step
       {\bf we do not care about their values}. We try
       to find information as much as possible
       without bothering about these values!

  Polynomial $P_4(t)$ has four roots but
  they are not distinct. It is easy to see that from this fact follows
  that polynomial $P_4(t)$ in (6) is in fact {\bf square of quadratic polynomial
  with  rational coefficients} Indeed from (5) it follows that
                       $$
  P_4(t)=16t^4+at^3+bt^2+ct+d=P_2(t)^2=(4t^2+pt+q)^2
                           \eqno (7)
                       $$
            where  $\sin 18^{\circ}$ and $\sin 54^{\circ}$ are roots
            of polynomial $P_2(t)$
            ($P_2(t)=4(t-\sin 18^{\circ})(t-\sin 54^{\circ})$)

 On other hand opening brackets in (4) we see that
                        $$
             16t^4+at^3+bt^2+ct+d=
             16t^4+8pt^3+(p^2+8q)t^2+2pqt+q^2
             \eqno (8)
              $$
    Comparing coefficients at $t^3$ and $t^2$ we see that $p$ and $q$
    are {\bf rational numbers}.

    This
     persuades us to find quadratic equation such that $\sin 18^{\circ}$
     and $\sin 54^{\circ}$ are its roots. (We just proved existence of this equation).

     Now denote $u=\sin 18^{\circ}$. It is easy to see that
                               $$
             \sin 54^{\circ}=3t-4t^3=\cos 36^{\circ}=1-2t^2
                                       \eqno (9)
                                $$
   Hence $u$ is a root of equation $3t-4t^3=1-2t^2$, or
   $4t^3-2t^2-3t+1=0$. This equation has obvious
    root $1$. Hence
                            $$
       4t^3-2t^2-3t+1=4t^3-4t^2+2t^2-2t-t+1=(t-1)(4t^2+2t-1)=0
       \eqno (10)
                             $$
         We came to quadratic polynomial $4t^2+4t+1$
         such that $\sin 18$ is its root (compare with (0.2)).
          Thus $u$ is positive root of square equation $4t^2+4t+1=0$
          Hence
                              $$
                           \sin 18^{\circ}={\sqrt 5-1\over 4}
                                      \eqno (11)
                                      $$


       It is natural to ask me: what for we did
            all this analysis. We can from beginning
            to guess right equation (9)
            (Or even (10)). In this case I have not much
            to argue with you but
            in the next section we see how useful becomes
            arguments before (9) to find $\sin 6^{\circ}$.

            \bigskip

                 \centerline {${\cal x}2$ Calculation of $\sin 6^{\circ}$.}

\medskip

            In this case we even are not sure does there exist
            good equation to solve it for obtaining $\sin 6^{\circ}$.

          So analysis becomes important in fact.


   Note that $5\cdot 6=30$. Hence consider equation (3) with parameter
   $a=1/2=\sin 30^{\circ}$. From (1), (2) (4) and elementary trigonometry it follows that
    roots of polynomial (3) in this case are
          $$
           <\sin 6^{\circ},\sin 78^{\circ},\sin 150^{\circ},
           \sin 222^{\circ},\sin 294^{\circ} >=
           $$
           $$
  <\sin 6^{\circ},\cos 12^{\circ},1/2,
           -\cos 48^{\circ},-\cos 24 ^{\circ} >
                          \eqno (2.1)
               $$
 In the same way noting that one root is rational
 we come to the fact that there exist four order polynomial
 $P_4(t)$ with rational coefficients such that
 $P_5=(t-1/2)P_4$. But here story is more funny
 because we cannot so simple come to quadratic equations.

But nevertheless again as in the previous Section we
will try to disclose as much information as possible without
 considering coefficients of Polynomial $P_4(t)$
 (only relying on the fact that they are rational!)

 The roots of polynomial $P_4(t)$ are
                 $$
   t_0=\sin 6^{\circ}, t_1=\cos 12^{\circ},
   t_2=-\cos 24^{\circ}t_0=-\cos 48^{\circ}
   \eqno (2.2)
    $$
Note that $\cos 2\varphi=2\cos^2\varphi-1=1-2\sin^2\varphi$.
Hence  under transformation
        $$
        t\rightarrow 1-2t^2
              \eqno (2.3)
         $$
 {\bf root of this polynomial transforms to another root!!! }:
the
root $t_0$ transforms to $t_1$,
$t_1$ transforms to $t_2$, $t_2$ transforms to $t_3$
and $t_3$ transforms to $t_0$.

Denote transformation (2.3) by the letter $\F$
It is easy to see that $\F^2 t_0=t_2$,$\F^3 t_1=t_0$  e.t.c.

 We see that all roots of polynomial $P_4(t)$ are rationally expressed
 via root $t_0$ by the transformations $\F$, $\F^2$
and $\F^3$.


 Now consider {\bf arbitrary combinations of roots $t_0,t_1,t_2,t_3$}.
 and study at what extent they are invariant under the action
 of transformations $\F, \F^2,\F^3$.
 For example $t_0+t_2$ and $t_0\cdot t_2$ are invariant under transformations
 $\F^2$, $t_0+t_1+t_2+t_3$ are invariant under all transformations.


{\bf Statement (Generalized Vieta Theorem)}.

 Every combination of roots
  obeys to one of the following conditions

  1) is invariant under all transformations $\F$, $\F^2$, $\F^3$

  2) is invariant under transformations $\F^2$
    but is not invariant under transformation $\F$,
    hence the action of transformations $\F$, $\F^2$, $\F^3$
     on every element lead to two distinct elements.

  3) is not invariant under transformation $\F^2$


  In the first case this combination
  is root of linear equation with rational coefficients,
  i.e. are rational numbers!

  In the second case this combination is a root of quadratic polynomial
  (with rational coefficients)!
  (and it is not a rational number)

  In the third case this combination is a root of fourth order equation
  and it is not the root of third or second order equation!

\bigskip

  From this statement it is easy to see how to find combinations
    of roots that are roots of quadratic polynomial!
     It is evident that
                 $$
  t_0+t_2=\sin 6^{\circ}-\cos 24^{\circ}\eqno (2.4)
                          $$
  and
                      $$
             t_0\cdot t_2=-\sin 6^{\circ}\cos 24^{\circ}
                      $$
                      are roots of quadratic polynomial
                      with rational number!!! because
                   $$
            t_0+t_2=t_0+\F^2 t_0,\quad t_0\cdot t_2=t_0\cdot\F^2 t_0
            \eqno (2.4)
                   $$
     do not change under transformation $\F^2$ and action of transformation
     $\F^3$ coincides with action of transformation $\F^2$.


  And just these expressions appeared in the first equation in introduction!


We explain already the anzats of Introduction.

But we continue to calculate $\sin 6$ using polynomial $P_4(t)$.

  Now encouraging by this remark we understand what is the right anzats for solving equation
   $P_4(t)=0$

  Performing all our considerations till now we were very lazy to
   calculate explcitly $P_4(t)$. But now time came to do it because we now that we are on
   the right way:
    Equation for $\sin 6$ (see (3)):
            $$
            P_5(t)=16t^5-20t^3-5t={1\over 2}
            $$
  It is convenient (but not important) to consider variable $t\mapsto 2t$:
                  $$
       P_5(t)=t^5-5t+5t-1=0  \quad (t=2\sin 6)
                   $$
    Dividing this polynomial on $t-1$ (we change $t$ on $2t$)
     we come to polynomial
                  $$
         t^4+t^3-4t^2-4t+1=0
                          \eqno (2.6)
                  $$
  WE ALREADY ARE READY TO ATTACK THIS EQUATION!

   As was established above transformation $t\mapsto \F t=2-t^2$ transforms root
   to the root. So if $t$ is a root then
   $t+\F^2 t$ and $t\cdot\F^2 t$ have to be the root of quadratic polynomial!!!
   Consider new variables
                       $$
       x=t+\F^2 t=t+\left(2-\left(2-t^2\right)^2\right)
                      \eqno (2.7)
                       $$
    and
                     $$
                        y=t\cdot\F^2 t=
                        t\cdot\left(2-\left(2-t^2\right)^2\right)
                      \eqno (2.8)
                        $$

     Using (2.6) we open brackets in (2.7) and see that
                   $$
                   x=t^3-3t-1,\quad y=-(t^3-3t+1)
                   $$
      $x$ and $y$ are roots of quadratic polynomials.

      There are hundred ways to come to this quadratic equation.
      May be the easiest one is again return to trigomnometry:
             Remembering that $t=2\sin 6$
           and $4\cos^3\varphi-3\cos\varphi=\cos 3\varphi$
       it is very easy to see that $x$ and $y$
       are just equal to left hand sides of simultaneous equations of Introduction
      and from ${\cal x}1$ we already now that these numbers are roots of quadratic
      polynomials.
 (Up some integer coefficients $x=1-2\sin 18$
and $y=1+2\sin 18$ (Compare with Introduction)).




 We justified Introduction!!!



                          \bigskip

                          \centerline {${\cal x} 3$}

 Now very very briefly about $\cos 20^{\circ}$

 
                Consider equation
                      $$
                    t^3-3t-1=0
                    $$
          It is remarkable cubic equation.

   According formula $4\cos^3\varphi-3\cos\varphi=\cos 3\varphi$
   and the  first Section it is very easy to see that roots of this equation
   are
                      $$
                      <2\cos 20,-2\cos 40, -2\cos 40>
                      $$
           and transformation $\F\colon\quad t\mapsto 2-t^2$
           transforms root to the root!!!

 In this case there are two not identical transformations:
 $\F$ and $\F^2$. $\cos 20$ is a root of cubic equation
 and in fact it cannot be expressed as a root and their combinattions  of quadratic
 equations.

           Evidently (according to usual Vieta theorem )
           product of its roots is equal to one.


           In the school we proved it using "domino" princip:
                           $$
             8\cos 20\cdot\cos 40\cdot \cos 80=
                            $$
                            $$
             {8\sin 20\cdot\cos 20\cdot\cos 40\cdot \cos 80\over\sin 20}=
              $$
              $$
          {4\sin 40\cdot\cos 40\cdot \cos 80\over\sin 20}=
          $$
          $$
          {2\sin 80\cdot \cos 80\over\sin 20}=
          $$
          $$
          {2\sin 160\over\sin 20}=
          $$
          $$
          1
             $$


 Long years I enjoyed this example and only after learning
  elements of Galois theory understood its real meaning.


\bigskip
\centerline {${\cal x }4$}


We see that $\sin 18$ is root of quadratic equation.

$\sin 6$ is not root of quadratic equation,
but can be expressed via quadratic radicals,
because it can be expressed as root of quadratic
with coefficients which are rational
or roots of quadratic equation.

On the other hand $\cos 20$ cannot be expressed in this way.

In other words we can construct by ruler and compass the
$\sin 6$, $\cos 6$ and hence angle $6^{\circ}$
and divide the circle on the 60 parts but we cannot divide the circle on the
18 paths (360:18=20).


Before formulating general result one exercise

{\bf Exercise}

The  number
                $$
              2^k+1
                             \eqno (4.1)
                  $$
      is prime. Prove that $k$ is power of $2$ ($k=2^n, n=0,1,2,\dots$)

      The prime numbers which can be represented in the way
      (4.1) are called Messner prime numbers
      For example $3=2+1$, $5=2^2+1$ $17=2^4+1$, $257=2^8+1$
      are Messner prime numbers.

 Now I formulate general result without proof.


{\bf Theorem} The circle can be divided on $N$ equal parts with ruler and compass
 or in other words one can construct
 $\sin {360/over n}$, $\cos {360/over n}$ and the angle ${360/over n}$

 \centerline {\bf if and only if}

 in the decomposition of the number $N$ on the prime factors
   all the prime numbers except 2 have power 1 or 0,
   and these prime numbers are Messner prime numbers:
                   $$
          N=2^np_1\cdot\dots\cdot p_n
                       \eqno (4.2)
                    $$
                    where all prime numbers
           $p_1,\dots,p_n$ are different Messner prime numbers.


This theorem has remarkable history:

Ancien Greek studied this problem. For $N=2,3,4,6$
everything is obvious for $N=5, 15$ solution was known to Ancien Greeks.

Problem for $N=7,9,17$ was open...

Gauss did it for $N=17$. Then it was realized general answer...

Galois Theory gives full answer to this problem.

We do not prove here this Theorem.

Only note that for $N$ obeying to above
condition we have $2^n-1$ transformations
transforming roots to the roots
and step by step we can do calcualtion.

           It is easy to see that
           $N=60=4\cdot3\cdot 5$ obeys to Theorem
           and $N=18=2\cdot 3\cdot 3$ does not obey.


From theorem follows answer on the following question:
in what case $\sin k^{\circ}$  for $k$ integer is expressed through quadratic radicals?
For $k=1$ answer {\bf NO} because $N={360\over 1}=360=8\cdot 3\cdot 3\cdot 5$
 does not obey
to Theorem. Analogously for $k=2$ answer {\bf NO} too because
$N={360\over 2}=180=4\cdot 3\cdot 3\cdot 5$.

But for $k=3$ answer {\bf Yes!!!}  because
$N={360\over 2}=180=4\cdot 3\cdot 3\cdot 5$
enjoys the property

Hence for all $k$ that are divisible on $3$ ($k=3n$)
answer is {\bf yes}:we construct angle $3^{\circ}$ and multiply on $k$.

ON other hand for every angle that is not divisible on $3$
answer is {\bf No}: If Yes we come to contradiction:
$3k+p$ (p=1,2) Yes and $3k$ Yes hence subtracting angles
$p$ Yes too. But for $p=1,2$ answer is {\bf No}. Contradiction.






\bye
