\magnification=1200

 In this etude we consider two isomorphisms of
different nature between the following sets.

1) $A$--is the set of not-decreasing finite sequences of natural
numbers. $\{a_1,\dots,a_n\}\in A$ if
   $a_1\leq a_2\leq\dots\leq a_n$

2) $B$ is the set of increasing natural numbers.
$\{a_1,\dots,a_n\}\in B$ if
   $a_1< a_2<\dots< a_n$.

   The first isomorphism $\varphi_1$:


                    $$
  \forall \{a_1,\dots,a_n\}\in A
                   \varphi (\{a_1,\dots,a_n\})=
                    \{a_1,a_2+2,\dots,a_n+n\}\in B
                      \eqno (1)
                            $$

It is evident that $\varphi$ is an isomorphism


To construct the second isomorphism we first identify the space
$A$, the set of not-decreasing finite sequences of natural
numbers, with the set $A^\prime$--the set of not-decreasing
finite sequences of odd natural  numbers by isomorphism
                  $$
     \tau\colon\quad
       \forall \{a_1,\dots,a_n\}\in A_0,
       \tau (\{a_1,\dots,a_n\})=
         \{2a_1-1,2a_2-1,\dots,2a_n-1\}\in A^\prime
            $$
   and consider the following "tricky" isomorphism between
    $A^\prime$ and $B$.
    Let $b=\{b_1,\dots,b_n\}\in B$. For every $b_i\in b$ there
    exist natural numbers $p_i$ and $q_i$ such that
                    $$
            q_i \,\,{\rm is\,\, odd\,\,and}\quad
            b_i=2^{p_i}q_i
            \eqno (4)
                       $$
         We put in correspondence to the sequence
         $\{b_1,\dots,b_n\}$ the sequence
                           $$
         \left\{
     {\underbrace {q_1,q_1,\dots,q_1}_{p_1\, {\rm times}}}\,,\,
     {\underbrace {q_2,q_2,\dots,q_2}_{p_2\, {\rm times}}}\,,
                   \dots\,
     {\underbrace {q_n,q_n,\dots,q_n}_{p_n\, {\rm times}}}
         \right\}
         $$
Then rearrange the odd numbers $\{q_1,\dots,q_n\}$ in
not-decreasing order  $q_{i_1}\leq q_{i_2}\leq\dots\leq q_{i_n}$
we finally come to isomorphism $\varphi_2$:
                             $$
                    \varphi_2 (\{b_1,\dots,b_n\})=
     \left\{
     {\underbrace {q_{i_1},q_{i_1},\dots,q_{i_1}}_{p_{i_1}\, {\rm times}}}\,,\,
     {\underbrace {q_{i_2},q_{i_2},\dots,q_{i_2}}_{p_{i_2}\, {\rm times}}}\,,
                   \dots\,
     {\underbrace {q_{i_n},q_{i_n},\dots,q_{i_n}}_{p_{i_n}\, {\rm times}}}
         \right\}
                         $$
between $B$ and $A^\prime$. The inverse to $\varphi_2$ isomorphism
is following. Consider
                  $$
      \left\{
     {\underbrace {q_1,q_1,\dots,q_1}_{f_1\, {\rm times}}}\,,\,
     {\underbrace {q_2,q_2,\dots,q_2}_{f_2\, {\rm times}}}\,,
                   \dots\,
     {\underbrace {q_n,q_n,\dots,q_n}_{f_n\, {\rm times}}}
         \right\}\in A^\prime
             \eqno (lalala)
                        $$
                        then
                     every number $f_i$ can be uniquely
                     represented as
                    $$
                    f_i=\sum_k 2^k \delta_{ik}\quad {\rm
                    where}\,\, \delta_{ik}=0,1
                        $$


We consider the numbers $2^{k}q_i$ if $\delta_{ik}\not=0$ and
arrange the sequence from $B$.

They are so far from each other isomorphisms $\varphi_1$ and
$\varphi_2$!!!

For every sequence $a$ in $A$, $A^\prime$ or $B$
we denote by $\#(a)$ the number of elements of this sequence and
by $s(a)$ the sum of elements of this sequence.
The following relations are obvious:

1.
                 $$
           \#(\varphi_1(a))=\#(a),
                   $$
2. If $\#(a)=n$, then
                   $$
             s(\varphi_1(a))=
             s(a)+{n(n+1)\over 2}
                $$
                $$
           \quad s(a)=s(\varphi_2(a))
                     \eqno ()
                   $$
These relations leads to interesting corollaries:

{\bf Corollary 1}.

  {\it Dimension of the space of symmetrical tensors of rank $k$ in
   $n$-dimensional
   space is equal to the
 a dimension of the space of antisymmetrical tensors of rank $k$ in
   $\left(n+{k(k+1)\over 2}\right)$-dimensional space.
  space.}





{\bf Corollary 2}.

{\it For every integer number $N$
a number
of choices such that $N$ can be represented as a sum of different
natural numbers is equal
a number
of choices such that $N$ can be represented as a sum of
odd natural numbers.}
 (We do not differ two choices
which differ by rearrangement.)


We can use the corollary 1 for calculating the dimension of symmetrical tensors space:
because
for antisymmetrical tensors it can be calculated  straightforwardly:


{\bf Corollary 3}.

  {\it Dimension of the space of symmetrical tensors of rank $k$ in $n$-dimensional
   space is equal to the}
               $$
               C^k_{n+{k(k+1)\over 2}}=
                     {
                     \left(n+{k(k+1)\over 2}\right)!
                         \over
                         k!
                         \left(n+{k(k-1)\over 2}\right)!
                         }
                       $$
 We denote by $A_n$ the number of choices for odd numbers
and $B_n$ the number of choices for different natural numbers.
It is evident that
              $$
           \sum_{n=1}^\infty B_n x^n=
           \sum_i\left(
           \sum_{n_1\geq 1,\dots n_i\geq 1}
              x^{n_1}x^{n_1+n_2}\dots x^{n_1+\dots+n_i}
           \right)=
            $$
            $$
            \sum_{n=1}^\infty
                 {x^{n(n+1)\over 2}\over
   (1-x)(1-x^2)(1-x^3)\dots (1-x^n)
                 }
              $$
   and
           $$
           \sum_{n=1}^\infty A_n x^n=
           \sum_i\left(
           \sum_{n_1\geq 0,\dots n_i\geq 0}
              x^{2n_1+1}x^{2n_1+2n_2+1}\dots x^{2n_1+\dots+2n_i+1}
           \right)=
            $$
            $$
            \sum_{n=1}^\infty
                 {x^n\over
   (1-x^2)(1-x^4)(1-x^6)\dots (1-x^{2n})
                 }
              $$
       We come to
{\bf Corollary 4}
                         $$
                \sum_{n=1}^\infty
                 {x^n\over
       (1-x^2)(1-x^4)(1-x^6)\dots (1-x^{2n})
                 }=
             \sum_{n=1}^\infty
                 {x^{n(n+1)\over 2}\over
   (1-x)(1-x^2)(1-x^3)\dots (1-x^n)
                 }
                 $$
in a vicinity of $x=0$:


I cannot calculate the number of choices  $A_n$ and $B_n$(see Corollary 2)
but one interesting identity for this.
We denote by $S^p_N$ the number of choices that $N$ can be represented
by $p$ different integers.
It is evident that
                  $$
                  S^p_N=\sum_k S^{p-1}_{N-pk}
                  $$
   Applying this equation many times we come to the

   {\bf Corollary 5}
   The number of choices for $N$ is equal to the number of solutions
    of the equation
                  $$
                2x_1+3x_2+4x_3+........<N
                   $$
                   plus $1$
    where $x_1,\dots,x_n$ are {\bf non-negative} integers.
(Do not confuse with number of values less that $N$!)
In other words we take $N$ (representing as one number)
then calculate the number of $x_1$ such that
  $2x_1< N$,
  ...

  For example
   $S^1_20=1$, $S^2_20=9$,
   $$
   S^3_20=S^2_17+S^2_13+.....$$
      $$
      S^4_20=S^3_16+S^3_12+...$$




 \bye
