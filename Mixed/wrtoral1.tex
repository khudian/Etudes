
\magnification=1200
\baselineskip 17 pt

\centerline {\bf Oral and written assignments.---Drawbacks and Achievement.}


          I am twenty years teaching different subjects in mathematics and physics. I worked in different
          Universities in Russia and Armenia. It is three years that I am lecturer in UK in UMIST.
           The methods of students assignments
           here are drastically different from everything which I knew before.
           (Shortly speaking: here examinations are written, there--- they are oral.)
           The simplest is to blame unknown and to iconise the known past.

           It was just what happened my first year of teaching here.
           Now after three years experience in UMIST I do not think that
           this is black and white problem.  I try in this essay to note
            some good and bad points of these two different systems.
            I do not want to compare them: In some sence they are incomarable.

             The shortest answer is: {\it There are two totally different
              but complimented ways of assignments: written and oral. Both
              have their advantages and disadvantages. The best is to use them both.}


                $$ $$

               \centerline {\bf Definition of problem. Short description of oral and written examinations.}

               \bigskip


                 I will try in this subsection to describe shortly the organization of
                 written and oral examinations. I will speak mainly about oral because
                  the written system I believe is more known to the reader of this essay than to the author.
                  In both systems I will consider an example of examination in mathematics.



                 {\it Written examination}.


                 The lecturer  in the middle of the semester
                 prepares question list based on the lecture course.
                 It contains three type of questions

                 a) basic theoretical questions,

                 b) simple exercises which are variations of exercises considered on lectures or tutorials

                 c) more difficult questions. (sometimes substantially difficult. It depends on the level of students
                  engaged in examination)


          There is very important restriction: students have two hours for all the work. There is not essentially
          much time to think. Most part of the work is performed in the regime: {\it Remember, remember... }


            After examination lecturer anonymously marks examinations works. It is possible to
            make so called marks rescalling. (E.g. nn the case if
           marks pattern drastically contradicts with average pattern examiner usually rescalle marks.)

           Long long after examination (something about one month after) the students receive information
           about examination esult. This information is usually coded by one number. This number is defined
           by the points which student receive for three best problems+ Coursework+
           it could be rescalled. The detailed information about how student solved this or this problem
           is practically unavailable.

     $$ $$

     {\it Oral examination} Here I will try to give more detailed pattern.


    Usually if course contains much practical work (E.g. if it is Lecture Course on "Mathematical Analysis"
      or "Quantum Mechanics" before examination students have to pass oral tests. They show their ability
        to solve problems. E.g. in the course of Mathematical Analysis they have to calculate limits of sequences,
        or integrals. During these tests usually students do not need to give or prove Theorems.)
        Students who pass these oral tests are allowed to participate in oral examination.

         Oral examination begins at morning at 9.00.
         Usually it ends at 3 o` clock Sometimes it continues till late evening.

  If number of students is not more than 20-30 the lecturer does all the work himself. In the case if number of students
  is 30--60 usually  colleagues help to the lecturer.

  One week before examination the lecturer writes so called "tickets". Every ticket contains
   usually tw0 theoretical questions and one problem. Student choose a ticket in a random way.
    Then student takes a sit to prepare for an answer. The time for preparation varies
     from one hour till three hours. (At beginning  about 5 students enter the examination room at morning.
     Other students wait their turn.)

           If student is prepared for answer he approaches lecturer and he began to answer.

       During preparation to answer students usually are not allowed to use their notes. But sometimes
      (on master courses they allowed to use books and notes.) But during conversation with
       lecturer student has to use only the notes which he produces during examination.

     The oral examination it is highly interactive process.  Usually after first two-three minutes
      lecturer interrupt the student and ask him to concentrate more on this or this subject.
      If student does something wrong lecturer tries to help him (of course this has to be noted when the final mark
       will be decided). Very often during oral examination if lecturer see that student does not understand
         something lecturer gives him very little mini-lecture or formulate him the question in
         another way. It makes the examination process to be {\it the learning process}.

         After answering theoretical questions student explains how he solved problem.

         If the problem is not solved or if solution is incomplete then lecturer usually
          gave an advice hint how to solve the problem or give another problem.

          After this lecturer have to decide the mark. (There are four choices:
          "bad", "satisfactory", "good" and "excellent". ) It is a common practic to
          formulate additional problems to decide the final mark.

          After two-three additional questions examiner decides the mark tell it to student
          and put it in the document. Examination is finished. Student knows his mark,
          knows (usually) the reason why it is bad and what is important
           very often learn new stuff furing examinaton.

           In the case if student receives mark "bad" or if by any reason he could not attend
            examination he is allowed to pass this examination at the end of examination session
         \footnote{$^*$}{In the case if student receives the mark higher than "bad" but he is not satisfied by this mark
           he is allowed to pass again the examination but under special permission of Faculty Dean.}.


      In the end of the day examiner sends all documents to Secretariat. Examine is finished.



              {Problems and }




   $$ $$ $$ $$

\bye
