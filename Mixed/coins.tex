\magnification=1200

\centerline {The problem of the coins}

\bigskip
  \centerline {Along the paper of  Mario Marteli and Gerald Ganon:}


  \centerline {The College Mathematics journal, 28, No5, pp. 365-367,1997}



 It is the very old problem how to find the odd coin from 12 coins
 by 4 measurements.

    \bigskip

   \centerline {\it DIVIDE ET IMPERA}

      $$ $$
  For finding the strategy we first solve three preliminary
  problems:

       $$
       1)\rightarrow 2)\rightarrow 3)\rightarrow 4)
                    $$

  {\bf Proposition 1}
   Number of coins is equal to $3^n$ and {\bf odd coin is heavier}.
    One can find it by $n$ measurements.


    The proof (trivial): The case $n=1$ is evident. Let $n=k+1$. During the
    first measurement we put $3^k$ coins on left and $3^k$ on the
    right. If "$left > right$" by induction we choose odd coin in
    the left set by $k$ measurements. If "left=right" we do it
    with $3^k$ remaining coins.

       \bigskip

     {\bf Proposition 2}
      Number of coins is $3^n$. Coins are marked: they are red or black and
     red coins are heavier or equal by weight to black,
     i.e. odd coin is heavier if it is red and it is  lighter if it
     is black.
     One can find it by $n$ measurements.

     The proof. The case $n=1$ is trivial again.
   Let the statement is right for $n=k$.

      Let $n=k+1$.
     We always can choose $2a$ red and
    $2b$ black coins such that $a+b=3^n$ $(a=0,1,2,\dots)$ and put on the left
    pan $a$ red coins and $b$ black coins, the same on the
     right pan. If "$left> right$" hence the odd
  coin is one of the $a$ red or one of the $b$ black.
  We take $3^n=a+b$ coins ($a$ red from the left and $b$ right
  from the right and by $n$ measurements find the odd one.

    If "$left=right$" the we find odd one in the remaining $3^n$
    coins.

 \bigskip

   {\bf Proposition 3}
       Number of coins is equal to $s_n={3^n-1\over 2}$,
    ($s_1=1,s_2=4,s_3=13,s_4=40,s_5=121,\dots$ ($s_{n+1}=3s_n+1$).

   The coins are not marked and
   we do not know is the odd coin heavier or odd but we have
    {\bf additionally the coin which sure is not odd.}
    One can find it by $n$ measurements!?

    {\bf Proof} The case $n=1$ is evident. We compare the coin
    with test one and see is it odd or not.


   Let we can find the odd one for $n=k$
   and consider the case $n=k+1$.

                   $$
    s_{k+1}=s_k+s_k+s_k+1
                    $$
    We put aside $s_k$ coins. We put $s_k$ coins and
     additional test coin on the left pan and $s_k+1$
      coins on the right pan.
      If "left>right" we throw out the test coin,
      mark the $s_k$ coins on the left pan by red paint,
       mark the $s_k+1$ coins on the right pan by black pain.
       We have $s_k+s_k+1=3^k$ marked coins. Using
        Proposition 2) we find the odd coin by $k$ measurements.

        If "left=right" we know that odd coin
        is in the rest $s_k$ coins and we find it
        with test coins by $k$ measurements.


                $$ $$

   Finally we come to

           {\bf Theorem}
     Number of coins is equal to $p_n s_n-1={3^n-3\over 2}$,
    ($p_1=0,p_2=3,p_3=12,p_4=39,p_5=120,\dots$ ($p_{n+1}=3p_n+3$).

   We do not know is the odd coin heavier or odd!

    One can find it by $n$ measurements!!!


      {\bf Proof}


    Let $n=k+1$.

   One can see that
                  $$
                 p_{k+1}=(p_k+1)+(p_k+1)+(p_k+1)
                    $$

           We put $p_k+1$ coins on the left side and $p_k+1$ on the right.

           a) If "$left>right$" It means
            that odd coin is "red" if it is on the
           left pan and it is right if it is on the
            right pan (See Proposition 2).
            We mark coins on the left pan by red and coins on the
            right pan by black, add one coin from
            the remaining coins. (This additional coin is sure not odd,
            so it can be marked as we want red or black)
               $(p_k+1)+(p_k+1)+1=3^k$. Hence we have $3^k$
               marked coins and we can find the odd coin amongst
               them, according to Proposition 2)


      b) "left=right". In this case we know that odd coin is
      in the remaining $p_k+1$ coins. But $p_k+1=s_k$ (See Proposition 3)

          We take remainig $s_k$ coins and the test coins from
          the coins which we measured also and due to Proposition 3)
          we find it by $k$ measurements.
                 \bigskip
                    It is all!

     {\bf Example} Let we have 12  $\{1,2,3,4,5,6,7,8,9,10,11,12\}$ coins.

    First meauserement:
      $\{1,2,3,4\}$ on the left and $\{5,6,7,8\}$ on the black.

           \medskip

     a)If $\{1,2,3,4\}>\{5,6,7,8\}$ we mark $\{1,2,3,4\}$ by red, $\{5,6,7,8\}$
   by black and consider $\{1_r,2_r,3_r,4_r,5_b,6_b,7_b,8_b,10_b\}$.

     We do second measurement:
       $\{1_r,2_r,5_b\}$ on the left and $\{3_r,4_r,6_b\}$
   on the right.
   If $\{1_r,2_r,5_b\}>\{3_r,4_r,6_b\}$ it means that
   odd is or on the left and red or on the right and black.
   We take the coins
   $\{1_r,2_r,5_b\}$ and find by the third measurement
    the odd one, putting $1_r$ on the left and $_b$ on the right.

    If $\{1_r,2_r,4_b\}=\{3_r,4_r,5_b\}$ it means that
    the odd is in the set $\{7_b,8_b,10_b\}$ and we
    find it by one measurement,
    putting $7_b$ on the left and $8_b$ on the right.

  Now consider the second case:

   \medskip
   a)If $\{1,2,3,4\}=\{5,6,7,8\}$. Then the odd is in the set
    $\{9,10,11,12\}$. We take the coin $1$ as the test coin and add it to
    the set $\{9,10,11,12\}$. Consider the set $\{1,9,10,11,12\}$
    where $1$ is the test coin (sure not odd).

     We do the second measurement according to the proof of Proposition 3):
     putting the coins $\{1,9\}$ on the left pan and
     $\{10,11\}$ on the right pan.
    Now if $\{1,9\}>\{10,11\}$ we mark ( coin by red and 10 and 11 by black
     and during the third measurement we
     put the coin 9 on the left pan and 10 on the right pan
     and find the odd coin in the set $\{9,10,11\}$.
   If $\{1,9\}=\{10,11\}$ then the odd coin is the coin $12$
   and we do not need the third measurement.


  \bye
