\magnification=1200
 Related with etud jakob.tex and wanted to remember
 my very old solution of the squares (see square.tex)
I was thinking on the non-linear
map with Jakobian 1.
It is following.

Let $R^n$ be $n$-dimensional space.
We consider the following non-linear map
              $$
   \xi^i=x^i+ a^i F(b^mx^m)
                 \eqno (1)
              $$
where $F(t)$ is a smooth function, $\bf a$
and $\bf b$ are orthogonal vectors.
The differential of this map is matrix
                $$
           \delta^i_m+a^ib^mf^\prime(B^mx^m)
               \eqno (2)
                  $$
It is easy to see that its determinant is equal to $1$.
(The matrix in (2) is equal to $1+S$ where $S$ is nilpotent $S^2=0$,
because vectors $a$ and $b$ are orthogonal, hence
                 $$
\det(1+S)=\det\exp S=\exp{\rm tr}S=1
                   \eqno (3)
                 $$
The transformation (1) is exponential flow of the vector field
\def\p{\partial}
              $$
           {\bf X}=a^iF(b^m x^m){\p\over\p x^i}
                     \eqno (4)
                 $$
and its integral of motion is $b^mx^m$.
In fact I wanted to construct non-linear map
with jakobian 1 which transforms
two families of lines in the lines.---It is impossible!
One can to construct non-linear map with linear
jakobian with these properties:
           $$
           \cases
               {
         x\mapsto x+axy\cr
         y\mapsto y+bxy\cr
                   }
                  \eqno (5)
               $$
This map transforms the square $(0,0), (0,1), (1,0), (1,1)$
to the quatreangle   $(0,0), (0,1), (1,0), (1+a,1+b)$
and all vertical and horizontal lines to the lines also!
Moreover its jakobian is equal
 to
                   $$
             1+2ay+2bx
                            \eqno (6)
                $$
hence  the middle square $(1/3,1/3), (1/3,2/3), (2/3,1/3), (2/3,2/3)$
  whose area is equal to $1/9$ transforms to the
 quatreangle with area wich is $1/9$ part of the area of the
 quatreangle  $(0,0), (0,1), (1,0), (1+a,1+b)$ .

This I did in 1989 in Geneve...
Yesterday I spent 10 hours wanted to reconstruct thsi solution.
My false idea was to find transformation like (5) with Jakobian 1,
but it is impossible, because only
middle quatreangle preserves its property to be $1/9$
part of all! Why? because
                  $$
             \left({2\over 3}\right)^2-
             \left({1\over 3}\right)^2=
               {3\over 9}=
               {1\over 3}
                   \eqno (7)
                      $$
 This is proptery of middle segment...



\bye
