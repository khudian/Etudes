\magnification=1200

\def\vare {\varepsilon}
\def\G {\Gamma}
                    $$ $$
26 July 2014


            \centerline{\bf Sum $1+2+3+\dots=-{1\over 12}$}
\medskip

{\it Yesterday David showed me the file in 'Youtube'
 where it is given very elegant `proof' of
this relation (see http://goo.gl/bYh4DL)
  I decided to write down here the standard calculations which lead to this
result. Here they are} 

\medskip
{\sl We present here calculation of analytical continuation 
of Riemann $\zeta$-function. In particular we come to Euler formula
of calculation of $\zeta$-function 
at negative integers. The title of this \'etude $=$ 
value of $\zeta$-function at $s=-1$.}

Recall that for $\G$-function
             $$
\G(s)=\int_0^\infty t^{s-1}e^{-t}dt\,.
   \eqno (1a)
             $$
It obeys the relation 
          $$
 \G(s+1)=s\G(s)\,.
    \eqno (1b)
          $$
These relations define $\G$-function  for all complex plane. 
Since $\G(1)=1$ hence due to (1b) $\G(n+1)=n!$. One can see that
that $\G$-function has poles at all non-positive integers. Indeed
             $$
\G(s)={s(s+1)(s+2)...(s+n)\G(s)\over s(s+1)...(s+n)}=
 {\G(s+n+1)\over s(s+1)\dots (s+n)}\,.
             $$
Hence in a vicinity of an arbitrary  point $s$, for  $s=-n+\vare$
($n=0,-1,-2,\dots$) 
                $$
\G(s)=\G(-n+\vare)=
 {\G(1+\vare)\over (-n+\vare)(-n+1+\vare)\dots (-1+\vare)\vare}=
{(-1)^n\over n!}\left({1\over \vare}+\dots\right)\,.
            \eqno (1c)
                $$


Express $\zeta$-function, $\zeta(s)=\sum_{n=1}^\infty {1\over n^s}$  
in terms of $\G$-function.

It follows from (1a) that
             $$
\zeta(s)=\sum_{n=1}^\infty {1\over n^s}=
  \sum_{n=1}^\infty\left({1\over \G(s)}\int_0^\infty t^{s-1}e^{-nt}dt\right) 
  ={1\over \G(s)}\int_0^\infty t^{s-1}{e^{-t}dt\over 1-e^{-t}} 
             \eqno (2)
                     $$
Consider an expansion of function $e^{-t}\over 1-e^{-t}$ 
in a vicinity of point $t=0$:
           $$
   {e^{-t}\over 1-e^{-t}}=
   \sum_{k=-1}^{\infty}\Psi_k t^k={1\over t}-{1\over 2}+
{t\over 12}-{t^3\over 720}+\dots\,.
        \eqno (3) 
           $$              
{\bf Remark} Notice that all coefficients $\Psi_{2k}$
in this expansion vanish for $k\geq 1$ since
                 $$
{e^{-t}\over 1-e^{-t}}=-{1\over 2}+
\underbrace{{1\over 2}{\rm cotan\,}{t\over 2}}
 _{\hbox{even function}}\,.
              \eqno (3b)
                 $$ 
Now using expansion (3) we see that
right hand side in  (2) is convergent for $\Re s>2$:
           $$
\zeta(s)=
  {1\over \G(s)}\int_0^\infty t^{s-1}{e^{-t}dt\over 1-e^{-t}} 
  ={1\over \G(s)}\int_0^1 t^{s-1}{e^{-t}dt\over 1-e^{-t}} 
  +{1\over \G(s)}\int_1^\infty t^{s-1}{e^{-t}dt\over 1-e^{-t}}= 
         $$
         $$
  ={1\over \G(s)}\sum_{k\geq -1}\int_0^1 
      \Psi_k t^{s+k-1}dt 
  +{1\over \G(s)}\int_1^\infty t^{s-1}{e^{-t}dt\over 1-e^{-t}}\,. 
         $$      
For $\Re s>2$\footnote{$^*$}{  \noindent
To calculate analytical continuation we
  represented the integral  $\int_0^\infty$
as a sum of integrals $\int_0^1$ and $\int_1^\infty$. 
One can show that 
the function ${e^{-t}\over 1-e^{-t}}$
can be replaced under integral $\int_0^1$
by series $\sum \Psi_k t^k$ in spite of the fact
that there is no convergence at the point  $t=1$.
To avoid this additional work one may
 take an arbitrary $a\colon 0<a<1$ and 
  represent the integral
as a sum of integrals $\int_0^a$ and $\int_a^\infty$.
In this case we have no any problem with series convergence: 
$\zeta(s)={1\over \G(s)}\int_0^\infty t^{s-1}{e^{-t}dt\over 1-e^{-t}} =$
           $$
={1\over \G(s)}\int_0^a t^{s-1}{e^{-t}dt\over 1-e^{-t}} 
  +{1\over \G(s)}\int_a^\infty t^{s-1}{e^{-t}dt\over 1-e^{-t}}= 
        \sum_{k=-1}^\infty {\Psi_k a^{k+s}\over (k+s)\G(s)} 
     +{1\over \G(s)}\int_a^\infty {t^{s-1}e^{-t}dt\over 1-e^{-t}} 
           $$
One can see that we again come to (5) since 
only the term which possesses $a^{0}=1$
gives contribution to the integral.
(Intersting remark for curious reader that the answer
does not depend on a choice of $a$).

                  
}
         $$
\zeta(s)=\sum{1\over n^s}=
   \sum_{k=-1}^\infty {\Psi_k\over (k+s)\G(s)} 
     +{1\over \G(s)}\int_1^\infty t^{s-1}{e^{-t}dt\over 1-e^{-t}}\,. 
           \eqno (4)
           $$
Now note that the integral in this expression is analytical function
for all $s$. Hence equation (4) defines analytical continuation,
meromorphic $\zeta$-function for all $s$. 

 We focus at the values of $\zeta (s)$ at non-positive integers.
Non-positive integers are poles of $\G$-function (see 1(c)). 
Hence the 
second integral vanishes at these points.
  Due to  relation (1c) we have that all terms which are proportional
to $1\over \G(s)(k+s)$ for $k\not=m$ vanish too at the point $s=-m$. 
We have from (1c) that
              $$
\zeta(-m)=
 \sum_{k=1}^{-\infty} {\Psi_k\over \G(s)(s+k)}\big\vert_{s=-m}=
   {\Psi_{m}\over \G(s)(s+m)}\big\vert_{s=-m}\,.
              $$
Now recall the behaviour of $\G$-function in a vicinity of the pole
$s=-m$. Due to (1c) $\G(s)=\G(-m+\vare)=
 {(-1)^m\over m!}\left({1\over \vare}+O(1)\right)$ and
           $$
   \zeta(-m)=
   {\Psi_{m}\over \G(s)(s+m)}\big\vert_{s=-m}=
    \lim_{\vare\to 0}{\Psi_m\over \G(-m+\vare)\vare}=(-1)^mm!\Psi_m\,.
       \eqno (5)
           $$
 This is the answer. 
In particular due to (3b) 
       $$
\zeta(-2m)=0\,,\quad \hbox{for $m\geq 1$}
    \eqno (5b)
      $$
We have
         $$
      \zeta(0)=\Psi_{0}=-{1\over 2}\,,
   \quad \zeta(-1)=-\Psi_{1}=-{1\over 12}\,,\quad
      \zeta(-2)=0\,,\quad
    \zeta(-3)=-6\Psi(3)={1\over 120}\,,\zeta(-4)=0,
         $$
and so on,



 One can say that in Pickwickian sense 
        $$
1^m+2^m+3^m+\dots+k^m+\dots`='m!(-1)^m\Psi_{-m}\,,
        $$
in particular we come to very famous in string theory paradox:
        $$
1+2+3+\dots+k+\dots`='-\Psi_{-1}=-{1\over 12}\,.
        $$


Note that relation (5b) describes all so called {\it trivial zeros of 
$\zeta$} function.



\bye
