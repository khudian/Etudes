\magnification=1200

 $$ $$
\centerline {\bf Projective module of Mobius band global sections}

   \bigskip

 Let $C$ be an algebra of continuous functions on $[0,2\pi]$.
  Consider the following two subalgebras of $C$
                          $$
             \Lambda=\{f\colon\,\, f\in C, f(0)=f(2\pi)\}\,
                                     \eqno (1)
                           $$
 and
                           $$
              P=\{f\colon\,\, f\in C, f(0)=-f(2\pi)\}\,
                                     \eqno (2)
                           $$
 $P$ can be considered as module over $\Lambda$
 This module is projective and it is not free. More precisely:
                               $$
                           P\oplus P=\Lambda\oplus \Lambda
                                         \eqno (3)
                               $$
We give a proof of (3) below. Equation (3) menas that $P$ is
projective (by definition). From (3) it follows that $P$ is not
free. Indeed suppose that it is free then from embedding (3) it follows
 that it has one generator $f_0$. From (2) it follows that
  $f_0$ vanishes in some point $x_0$. Hence $f_0$ generates
  submodule of $P$. Contradiction.

  Before proving (3) we note that $P$ is nothing but
  module of global continuous sections on Mobius band.
 Condition (3) corresponds to condition that
 Whithney sum of two Mobius bundles is trivial bundle:
                      $$
           M\oplus M=R^2\times S^1
                          \eqno (4)
                      $$
The proof follows from the following geometrical realization
 of (4). Consider in $R^4$ two Mobius bands, bundles
 over circle:
                         $$
                          M_1\colon
                          \cases
                          {
                          x=cos\varphi \cr
                          y=sin\varphi\cr
                          z=tcos {\varphi\over 2}\cr
                          u=tsin {\varphi\over 2}
                           }\,,\qquad
                            M_1\colon
                          \cases
                          {
                          x=cos\varphi \cr
                          y=sin\varphi\cr
                          z=-\tau sin {\varphi\over 2}\cr
                          u=\tau cos {\varphi\over 2}
                           }\,,\quad
                           0\leq\varphi \leq 2\pi, -\infty<t<\infty,
                           -\infty<\tau<\infty
                            \eqno (5)
                          $$
   It is evident that this embedding leads to (4). Now from (5)
   it follows the proof of (3).
   The isomorphism (3) is given by the formula
                     $$
                  \rho \pmatrix {t(\varphi)\cr \tau(\varphi)\cr}=
                  \hat T_{\varphi\over 2}
                  \pmatrix {z(\varphi)\cr u(\varphi)\cr}\,,
                  $$
       where $\hat T$ is operator of rotation:
                      $$
                      \hat T_\varphi
                        =
                        \pmatrix
                          {
                          cos \varphi,sin \varphi\cr
                          -sin \varphi,cos \varphi\cr
                          }
                          $$
        To isomorphism (3) corresponds the following projector
        in $R^2\times S^1$ on M:
                         $$
         \Pi (z,u,\varphi)=\hat T_{\varphi\over 2}\circ
                \pmatrix
                          {
                          1\quad 0\cr
                          0\quad 0\cr
                          }\circ
                          \hat T_{-\varphi\over 2}\,,\quad
                          1-\Pi (z,u,\varphi)=\hat T_{\varphi\over 2}\circ
                \pmatrix
                          {
                          0\quad 0\cr
                          0\quad 1\cr
                          }\circ
                          \hat T_{-\varphi\over 2}\,
                          $$

   Zabavno otmetitj shto etot proektor imejet vneshne bezobidnyj vid:
                            $$
                  \Pi={1\over 2}\left(
                     1+\pmatrix
                          {
                          1\quad 0\cr
                          0\quad -1\cr
                          }\hat T_\varphi
                  \right)
                      $$

 \bye
