\magnification=1200
\def\variables#1#2 {#1=1,\dots,#2}

   \centerline {n-points on the plane}

   (Etu zadachu mne zadal Senja Rubanovitch v 1971 godu)


                     $$ $$
   \item{I} There are $n$-points on the plane. For every 2 points there exist
 the third which is on the same line.
 Prove that all the points are on the same line.

                   \smallskip
        \centerline{The main idea of solution.}
                  \smallskip
   We can always choose coordinates $(x,y)$ on the plane
   and the numeration of the points in the following way:
    If $(x_i, y_i)$ are coordinates of the $i$-th point
   ($\variables{i}{n}$)   then
             $$
   x_1\leq x_2\leq x_3\leq\dots\leq x_n,\quad
   y_1\leq y_2\leq y_3\leq\dots\leq y_n\,.
             $$
We consider the equivalent problem

  \item{II} From the every point $x_1$ the "point" with constant velocity
     $v_i=y_i$ begins to move.  All the intersections are not binar!
 (it corresponds that for every 2 points there exist
 the third one on the same line).
  Prove that all the lines intersect simultaneously!


 This second formulation admits combinatoril reformulation.
 Every collision is the rearrangement of the points.


 There are $n$  symbols on the line-- $a_1,\dots,a_n$.
 The rules of the game are following:...

 This third reformulation is solved by me.


   Daniel informed me about very elegant solution (belongonging to Silvester):
   Suppose that points are not on the one line.  Then choose the
   line $l$ formed by points $A,B$ and point $C\not\in L$ such that the distance
   between the point $C$ and the line $l$ is the smallest possible (not-zero).

   Then considering lines $AC,BC$ we come to the smaller distance. Contradiction.

   \medskip  Now my solution:  We have $N$ points $\{1,\dots,N\}$
   moving with different velocities along the line.

\def\s {\sigma}
   Consider the set of states: In every state the $k$-th point
   occupies the place $\s(k)$, $a_{\s(k)}=k$.
    We say that the state belong to the set $\cal B$
   if there exist numero $i$ such that
                    $$
          i>\sigma (N)\,,\qquad {\rm and}\quad a_i<a_{i+1}\,,
                 $$
i.e. the last point $N$ is not on the last place and there exist a
pair $(a_i,a_{i+1})$ such that this pair is standing on the right????

E.g.  $(1,2,7,5,3,4,6)\in \cal B$,
$(1,7,6,5,4)\in \cal B$,
$(1,4,5,3,2,6,7)\in \cal B$,
$(1,4,5,3,2,7,6)\in \cal B$
{\it Lemma} Any transformation cannot take the state out the $\cal B$.
Theorema follows from this lemma.



\bye
