\bigskip
\centerline {\bf Appendix 3}

\medskip
\def\t{\tilde}
\def\p{\partial}
 In this Appendix we prove that for any given
 special canonical transformation  $\{x^i,\theta_j\}\rightarrow$
 $\{\t x^i,\t\theta_j\}$ of the first kind
 (\cantransform a) there exists Hamiltonian that generates this
 transformation via differential equations (\hamiltoniandeformation)
 and this Hamiltonian is defined uniquely by the condition
                        $$
Q=O(\theta^2)\quad (Q(x,\theta)=Q^{ik}\theta_i\theta_k+\dots
                                       \eqno ({\rm Ap}3.1)
                      $$

\def\Ocal {{A}}

For every Hamiltonian (odd function)
$Q(x,\theta)$ obeying to condition (Ap3.1)
consider one-parametric family of functions (coordinates)
 $\{y^i(t),\eta_j(t)\}$ ($i,j=1,\dots,n$)
 that are solution to differential equation (\hamiltoniandeformation):
                   $$
                    \cases
                     {
 {d y^i(t)\over dt}= \{Q(y,\eta),y^i\}=
 -{\p Q(y,\eta)\over \p\eta_i}\,,\cr
{d \eta_j(t)\over dt}= \{Q(y,\eta),\eta_j\}=
{\p Q(y,\eta)\over\p \eta_i}\,,\cr
              }
              \quad
              (0\leq t\leq 1)\,,
                        \eqno ({\rm Ap}3.2)
                    $$
with initial conditions
                    $$
y^i(t)\big\vert_{t=0}=x^i,\,
  \eta_i(t)\big\vert_{t=0}=\theta_i\,.
                   $$

From condition (Ap3.2) it follows that at every $t$
coordinates $\{y^i(t),\eta_j(t)\}$
are related with initial coordinates by
special canonical transformation of the first kind:
$y^i(t)=x^i+O(\theta)$, $\eta_i(t)\vert_{t=0}=0$.
We note that for every special canonical transformation
of first kind $\{x^i,\theta_j\}\rightarrow$
 $\{\t x^i,\t\theta_j\}$
  functions $\{\theta_i(x,\theta)\}$ are uniquely defined
  by functions $\{\t x^i(x,\theta)\}$. Functions
  $\t x^i(x,\theta)=x^i+f^i(x,\theta)$
  obey to conditions:
                     $$
     \{x^i+f^i(x,\theta),x^j+f^j(x,\theta)\}=0\,,{\rm with}\quad
               f^i(x,\theta)\in O(\theta)\,.
                          \eqno ({\rm Ap}3.3)
                     $$
Thus we need to prove only the following Lemma:

{\bf Lemma 3}
   {\it For every set of functions $\{f^i(x,\theta)\}$ ($i=1,\dots,n$)
   obeying conditions} (Ap3.3) {\it there exists unique Hamiltonian
   $Q$ obeying condition} (Ap3.1)
    {\it such that functions $\{y^i\}$ solutions to differential
   equation (Ap3.2) obey conditions $y^i(t)\vert_{t=1}=x^i+f^i(x,\theta)$
   ($i=1,\dots,n$).}

   Prove this Lemma.

 Consider a ring $A$ of functions on $(x^1,\dots,x^n$,
  $\theta_1,\dots,\theta_n)$.(As always functions take values in an arbitrary
  Grassmann algebra $\L$).
  Consider in $A$ the following gradation:
  $A_{(p)}$ is a space of functions that are linear combinations
   of $p$-th order monoms on variables $\{\theta_1,\dots,\theta_n\}$:
   $f\in A_{(p)}$ iff $\sum_k\theta_k{\p f\over\p\theta_k}=pf$.
   $A_{(p)}=0$ for $p\geq n+1$.
 For every function $f\in A$ we denote by $f_{(p)}$ its component in $A_{(p)}$.
  $f=f_{(0)}+f_{(1)}+\dots+f_{(n)}$.
   It is evident that for canonical Poisson bracket
   (\darbouxtheorem)
                     $$
                     \{f,g\}_{(p)}=\sum_{i=0}^n \{f_{(i),g{(p+1-i)}}\}\,.
                       \eqno (Ap3.4)
                       $$
  Consider also a corresponding filtration:
              $$
               0=\Ocal^{(n+1)}\subset \Ocal^{(n)}\subset
 \dots\subset \Ocal^{(1)}\subset \Ocal^{(0)}=\Ocal\,,
              $$
 where $A^{(p)}=\oplus_{k\geq n} A_{(k)}$.
(Do not confuse this filtration with
filtration (Ap2.2) considered in the Appendix 2).

  We denote by $\Ocal^+$ ($\Ocal^-$) a subspace of even-valued
  (odd valued) functions in $\Ocal$.
   Respectively we denote by $\Ocal^{\pm}_{(k)}=\Ocal_{(k)}\cap\Ocal^{\pm}$
  and $\Ocal^{\pm(k)}=\Ocal^{(k)}\cap\Ocal^{\pm}$.


  We note first that condition (Ap3.1) () implies that
  solutions to equations (Ap3.2) are well defined.
  Indeed consider arbitrary function $\varphi(x,\theta)$,
  odd Hamiltonian $Q\in A^{-(2)}$ and differential equation
  $\dot\varphi=\{Q,\varphi\}$. Projecting this differential
  equation on the subspace $A_{(p)}$ we come  using (Ap3.4) to
  $\varphi_{(p)}=\{Q_{(p+1)},\varphi_{(0)}\}+\dots+
  \{Q_{(2)},\varphi_{(p-1)}\}$. This equations can be solved
  recurrently because
  Recurrently  we come to the solution:
  $\varphi_{(0)}\big\vert_{t=a}=\varphi_{(0)}\big\vert_{t=0}$,
  $\varphi_{(1)}\big\vert_{t=a}=\varphi_{(1)}\big\vert_{t=0}+$
  $t\{Q_{(2)},\varphi_{(0)}\}$ and for arbitrary $p$:

                     $$
  \varphi_{(p)}\big\vert_{t=a}=
 \sum_{\sigma\in I_p}
                 {a^{|\sigma|}\over |\sigma|!}
              \underbrace
 {\{Q_{i_1},\{Q_{(i_2)},\dots,\,\{Q_{(i_k)},}_{|\sigma|\,\, {\rm times}}
 \varphi_{(q)}\vert_{t=0}\}\dots\}\}=
  \left[
                    \sum_{k=0}^n
                      {a^k\over k!}
    \underbrace
 {\{Q,\{Q,\dots,\,\{Q,}_{k\,\, {\rm times}}
 \varphi_{(q)}\vert_{t=0}\}\dots\}\}
 \right]_{(p)}
                     \,,
                    $$
     where $\sigma$ is multiindex $\sigma=[i_1,\dots,i_k]$,
        $|\sigma|=k$,
         $I_p$ is a finite set of multiindexes $\sigma$
         such that $i_1+\dots i_k+q-k=p$.
  It is important to note that in this expression
  for every
                    $$
  \hbox{for every p}\,
  \varphi_{(p)}\big\vert_{t=a}=\varphi_{(p)}\big\vert_{t=a}+
  a\{Q_{(p)},\varphi_{(0)}\}+\hbox{terms depending on $Q_{(2)},\dots,Q_{p}$}\,.
                \eqno ({Ap3}.5)
                $$

 \def\N{{\cal N}}
   \def\H {{\cal H}}
 Denote by $\N$ a space  of even-valued functions $\{f^i(x,\theta)\}$
  ($i=1,\dots,n$) such that these functions obey to condition (Ap3.3).
  Consider a map that assigns to every Hamiltonian $Q\in A^{-(2)}$
  the solutions $\{y^i(t)\vert_{t=1}\}$ to differential equations
  (Ap3.2). Thus we define map $\H\colon$ $A^{-(2)}\rightarrow\N$.
   Relations (Ap3.5) for $\varphi=x^i$ imply that
                $$
      f^i_{(p)}=-{\p Q_{p+1}\over\p\theta_i}+
      \hbox{terms depending on
            $Q_{(2)},\dots,Q_{p}$}\eqno ({Ap3}.6)
              $$
   where we denote by dots {terms depending on
            $Q_{(2)},\dots,Q_{p}$}.
   Consider a map $\delta$
    a map $\delta\colon$ $\N\rightarrow A^{-(2)}$
   such that $f^i(x)\mapsto -\sum \theta_if^i(x,\theta)\theta_i$.

    If $\t Q=\delta (f^i)$ then from (Ap3.3) and (\darbouxtheorem)
    it follows that
   $f^i=-\{x^i,\t Q\}-\sum_m\theta_m\{x^i,f^m\}+\sum_m\theta_m\{f^i,f^m\}$
   Projection of this equation on subspace $A_{(p)}$ implies
                    $$
    f^i_{(p)}=-{\p \t Q_{p+1}\over\p\theta_i}+
    \hbox{terms depending on
    $f^i_{(1)},\dots f^i_{(p-1)}$}.
       $$
We see that $\delta$ is injection and
comparing this relation with
relation (Ap3.6) we see that
 for every $p=1,\dots,n$ $\t Q_{(p)}=Q_{(p)}+$  terms depending on
            $Q_{(2)},\dots,Q_{p}$. Thus the map
            $\delta\circ\H$ is surjective.
 For every $\{f^i\}\in\N$ the odd function
 $Q=(\H\circ\delta)^{-1} (-\theta_mf^m)$
 is the required unique Hamiltonian in $A^{(2)}$.
\finish.



%\bye
