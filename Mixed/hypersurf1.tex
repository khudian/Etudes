

\magnification=1200 \baselineskip=14pt


\def\p {\partial}
\def \D {\Delta_{d{\bf v}}}
\def \Ds  {\Delta^{\#}}
\def\t{\tilde}
\def\s {\sigma}
\def\L {\Lambda}
\def\Darboux {$z^A=$  $x^1,\dots,x^n$, $\theta_1,\dots,\theta_n$}
\def\a{\alpha}
\def\O{\Omega}
\def\d{\delta}
\def\dv  {{d{\bf{v}}}}
\def\A {{\cal A}}
\def\R {I\!R}
\def\t {\tilde}
\centerline {\bf Invariants of Hypersurfaces in Euclidean Space}


   Recently Nikita Sidorov (l`homme Nikita) gave me the following
   school problem: Let $L$ be a closed (convex) curve and a curve $L_{h}$
   is set of points which are on the distance $h$ from $L$
   ($\forall x\in L_h$ $\min_{y\in L}d(x,y)=h$). How to calculate
    the surface of the domain between these curves?
    There is a "school" solutions if we consider curve $L$ as a
    limit of "nogougoljnikov". But I solved in more complicated way
    and my solution can be generalized on $n$-dimensional case.

    Let $(x,y)$ be Cartesian coordinates on $I\!R^2$ and $t$ be any coordinate
    on $L$. Then we consider
   new coordinates $(\rho,t)$ in a tubular neighborhood
    of $L$ such that for every point $A\in I\!R$
                        $$
    \rho(A)=\hbox {distance between $A$ and $L$}=\min_{y\in L}d(A,y)
                                          \eqno (2)
                      $$
 and $t(A)$ be $t$ coordinate of the point $y_0$ such that
      $d(A,y_0)=\rho(A)$.

      One can see that a function $\rho(x,y)$ obeys to equation
                          $$
                       \rho_x^2+\rho_y^2=1,\quad \rho\big\vert_L=0
                                 \eqno (2)
                                 $$
  Moreover if ${\bf r}_0(t)=(x_0(t),y_0(t))$ is an equation of $L$
  then in a vicinity of $L$
                      $$
  {\bf r}(t)={\bf r}_0(t)+\rho{\bf n}(t)\,,
                                 \eqno (3)
                       $$
  where ${\bf n}(t)$ is unit vector normal to $L$ at the point ${\bf r}_0(t)$.
                   $$
             {\bf n}(t)=
             \left\{-{x_t\over\sqrt{x_t^2+y_t^2}},
             {y_t\over\sqrt{x_t^2+y_t^2}} \right\}
             \eqno (4)
            $$
 Calculate Jacobian of coordinate transformaton from Cartesian coordinates
 $(x,y)$ to these new coordinates.
 According to (3,4)
                        $$
                        \cases
                        {
           x(\rho,t)=x_0(t)-\rho{x_t\over\sqrt{x_t^2+y_t^2}}\cr
           y(\rho,t)=y_0(t)+\rho{y_t\over\sqrt{x_t^2+y_t^2}}\cr
                             }
                             \eqno (5)
                             $$
Straightforward calculations (here straightforward still dimension is $2$!)
  give that
                        $$
          \det\left({\p(x,y)\over\p(\rho,t)}\right)=
          \sqrt {x_t^2+y_t^2}+\rho{x_t y_{tt}-y_t x_{tt}\over x_t^2+y_t^2}=
           \sqrt {x_t^2+y_t^2}+\rho {d\over dt}{\rm arctg}{y_t\over x_t}
                   $$
  the first term is an element of the length of $L$.
  The second term is Frenet curvature and it is full derivative.

Now solution follows from the last formula:
The surface of the domain between curves is equal to the
                   $$
                  \int_{L,L_h} dx dy=
                  \int  \det\left({\p(x,y)\over\p(\rho,t)}\right)
                              d\rho dt=
                       $$
                       $$
             \int_{0}^h d\rho
             \int_L dt\left(
\sqrt {x_t^2+y_t^2}+\rho {d\over dt}{\rm arctg}{y_t\over x_t}
             \right)=
              h|L|+\pi h^2
               $$
  Now we try to generalize this result.
  Of course the main problem will be to calculate jacobian
  and we are not so naive to hope that everything will be
 exhausted by total derivatives.

 So go on...

 Let $M^n\hookrightarrow \R^{n+1}$.
Let $x^\mu$ $\mu=0,1,...,n$ be cartesian coordinates on $I\!R^{n+1}$.
  and $f^\mu=f^\mu(t^1,\dots,t^n)$ be local parameterization of hypersurface
   $M^n$. Conisder $\rho({\bf r})=\min d({\bf r}, M)$.
   ${\rm grad} \rho^2=0$ and $\rho_M=0$. It is ray.
 In the same way as before consider transformation
            $$
    x^\mu(\rho,t)=f^\mu(t^1,\dots,t^n)+\rho n^\mu
   $$
  where ${bf n}$ is normal unit vector.
Consider any point $\lambda_0$ on $M^n$ with coordinates $t^1_0,\dots,t^n_0$,
 $x^\mu_0=f^\mu(t^1_0,\dots,t^n_0)$.

Conisder cartesian coordinates $\tilde x^\mu$ and parametrization $\tilde t^i$
 {\rm adjusted to the point $\lambda_0$},
 i.e. such that in a vicinity of $\lambda_0$
           $$
           \cases
            {
            \tilde x^0=A_{ik}t^it^k+o(t^2),\cr
            \tilde x^i=t^i \cr
               }
               $$
  $A_{ik}$ corresponds to bilinear symmetric form on hypersurfaces
  which is defined on tangent vectors and takes values
  in normal vectors.

        It is not awfull to calculate Jacobian
        of transformation from $\rho,tilde t$ to $\tilde x$
        in the point $\lambda_0$. Answer is
                         $$
        \det\left({\p(\t x^0,\t x^i)\over\p(\rho,\t t)}\right)=
                     \det\left(\delta_{ik}+\rho A_{ik}\right)
                         $$
  We note that jacobian of transformation from cartesian coordinates
  to another is equal to one, and jacobian of transformations from
   parameters $t$ to another parameters is proportional to
   the surface density. Hence
                      $$
             \int dx^0\dots dx^n=
             \int \det \left(
               1+\rho A
             \right)dsd\rho
              $$
Expanding determinant by powers of $\rho$ we come to series
of densities on $M$:
               $$
             \int \det \left(
               1+ A
             \right)ds=
             \int ds+\int TrA ds+....+\int\det A ds
               $$

 The first density is surface density , next one is a mean curvature,
 the last density is a total derivative-it is nothing but pull-back of
 volume form on unit sphere $S^{n}$ under the Gauss map of $M^n$
 embedding in $\R^{n+1}$
 in $S^n$ (to every point on $M^n$ corresponds point on $S^n$ via
 unit normal vector ${\bf n}$).

Now we rewrite it in dual notations.
let surface $M$ is given by equation:
            $$
            \Phi=0
             $$
During calculations I received the following answer dual
to previous:
                     $$
                 \int
                 \det
                 \left[
                    \delta_{ik}+
                    \Pi^\mu_i
                 {\p^2 \Phi\over \p x^\mu\p x^\nu}
                     \Pi^\nu_k
                     \right]
                     \cdot
                     |{\rm grad}\Phi|
                      \delta(\Phi)
                    d^{n+1}x\,,
                     $$
where
                      $$
 |{\rm grad}\Phi|=\sqrt
                       {
          {\p\Phi\over \p x^\mu}
            {\p\Phi\over \p x^\mu}
                   }
                    $$
But one can write the following answer too:

 Density of the weight zero:
                $$
                                \det
                 \left[
                    \delta_{ik}+
                      {1\over |{\rm grad}\Phi|^d}
                    \Pi^\mu_i
                 {\p^2 \Phi\over \p x^\mu\p x^\nu}
                     \Pi^\nu_k
                     \right]
                     $$
          I do not understand this answer.
          I forget how I receive all these dual objects
          I forget what is it exactly $\Pi^\mu_i$
       (Immediately after these calculations we loosed lagguage
        of my son).
        But I have to remember it!
        


            \bye
