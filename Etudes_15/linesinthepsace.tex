\magnification=1200


\def\a{\alpha}
\def\x{{\bf x}}
\def\y{{\bf y}}
\def\R {{\bf R}}
    Questions.


1. Can we represent $\R^3$ as a union of skew lines.


2. Can we represent $\R P^3$ as a union of lines


3. How look space of all lines in $\R^3$?

4.  How look space of all lines in $\R P^3$

5. How look submanifold of lines which knite $P^3$
in the manifold of all lines.


   Discuss these quesions. Some of them look
 like `olimpiad-like'


  The second question has a much more clever answer, that the first one, 
at least I do not know not articficial answer on the first question which does 
not use it. Here the construction which belongs to Plucker 
(Julius Plucker, teacher of Felix Klein.)


  Represent $\R P^3$ as a set of non-intersecting 
lines $=$  Represent $\R ^4$ as a set of $2$-dimensional subspaces
 which intersect only  at origin.

\smallskip

Let $I$ be a linear operator in $\R^4$ such that $I^2=-1$, e.g. 
$I\colon \quad(x^0,x^1, x^2,x^3)<mapsto (x^2,-x^1, x^3,-x^2)$\footnote{$^*$}{In fact $I$ 
defines complex structure in $\R^4$.}


   Now assign to any point $\x\in \R^4$ the plane
                           $$
                  \a(\x)=\{u\x+v I(\x)\}, \quad {\rm where}\,\, u,v \hbox {are arbitrary numbers}
                           $$
This is easy to see that for two arbitrary point $\x,\y$,
 planes $\a(\x)$ and $\a(\y)$ coincide
or interesect only at origin.

   Indeed suppose a point $\y\not=0$ belongs to the plane $\a(\x)$, 
($\x\not=0$). Show that in this case planes $\a(\x)$ and $\a(\y)$ 
coincide.
   If $\y\in \a(\x)$ then
  $\y=u_0\x+v_0 I(\x)$ for some coefficients $u_0,v_0$. Then
  a point $I(\y)$ belongs also to this plane, since
                                  $$
  I(\y)=I(u_0\x+v_0 I(\x))=-v_0\x+u_0 I(\x).
                                  $$
These two points $\y$ and $I(\y)$ are distinct points in $\R^4$ 
since $I^2=-1$.  We see that planes $\a(\x), \a(\y)$ 
coincide in two non-zero distinct points $\x,\y$ and at the origin. Hence
these tow planes coincide. 


 Thus we presented $\R^4$ as a union of $2$-dimensional subspaces such that
any two subspaces or coincide, or intersect only at origin,
i,e. we  presented $\R P^3$ as a union of projective lines  such that
any two two lines or coincide, or intersect only at origin,

It is a complex structure $I$ which defines this ``knitting''.
  Any complex strucutre in $\R P^3$ defines knitting of $\R^4$ on lines.


\bigskip \centerline {${\cal x} 2$}. Set of all lines in $\R P^3$.

   Now what about the set of all lines in $\R P^3$. The set of all lines in $\R P^3$
it is a Grassmanian $G^2_4$---a set of all $2$-dimensional subspaces in $\R^4$.

Let $l$ be an arbitrary line in $\R P^3$, i.e., a $2$-dimensional 
subspace $\a_l$ in $\R^4$.

  Take an arbitrary two points $\x,\y\in\a_l$ and consider its so called plucker coordinates
                  $$
         p^{ik}(\x,\y)=x^iy^k-x^ky^i
                  $$ 
 It is obvious that changing a pair of  points $\x,\y$ on another pair points on the plane
changes all $p^{ik}$ on a number (which is equal to determinant of  
corresponding $2\times 2$ matrix).


Hence one can assign to $2$-dimensional subspace in $\R^4$ (line in $\R P^3$)
  $6$ numbers  $\{p_{01},p_{02},p_{03},p_{12},p_{13},p_{23},\}$ defined up to a coeffciient,
i.e.  we define a map  from $G^2_4$ to $G^1_6$---set of $1$-dimensional subspaces in
$R^6$, i.e. a point in $\R P^5$. This map is evidently injection, but is it bijection? No!!!!
  Plucker coordinates $\{p^{ik}\}$   
of projective line in $\R P^3$ cannot be entries  
of an arbitrary anisymmetrical matrix.  The matrix  $p^{ik}(\x,\y)=x^iy^k-x^ky^i$ has rank $2$
(or less). Antisymmetric matrix may have rank $4$ or $2$ or $0$. Hence  one can see that
 $\{p^{ik}\}$ are Plucker coordiantes of projective line if and only if
                        $$
 det ||p^{ik}||=(p^{01}p_{23}-p^{02}p^{13}-p^{03}p^{12})^2=0\,
                        $$
{\bf Remark}  This relation between Plucker coordiantes
can be viewed  a la Ptolemey-theorem:
Consider rectangle $ABCD$ inscribed in circle with vertices at the points 
$A=\{0\},B=\{1\}, C=\{2\}, D=\{3\}$.   Assign to edges and diagonales
plucker coordinates:  $AB\mapsto p^{01}$, etc.  Then
               $$
   {AB\cdot CD+BC\cdot DA=AC\cdot BD }_{\hbox {Ptolemey Theorem}}
        \Rightarrow              p^{01}p^{23}+p^{12}p^{30}=p^{02}p^{13}
               $$
Alek Weinstein explained me that this is a sourse of cluster algebras
\footnote{$^*$}
{Consider Plucker coordinates of Grassmanian $G^2_n$. We have
 $n(n-1)\over 2$ coordinates $\{p_{ik}\}$, $1<i<k<leq n$.
To choose between the full set of coordinates
}
           





Thus we see that the set of all lines in $<R P^3$ it is the quadric hypersurface
                                $$
p_{01}p_{23}-p_{02}p_{13}-p_{03}p_{12}=0\,.
                                $$
in $<R P^5$. This quadric in suitable (homogenous ) coordinates has a form
                   $$
   x^2+y^2+z^2-t^2-u^2-v^2=0
                   $$
this is like one-sheeted hyperboloid: it contains planes.


Two-dimensional...



   What about lagrangian surfaces  on thsi hyperboloid.
   They correspond to surfaces in $<R P^3$.....


\bye
