


\magnification=1200
 \centerline {\bf On Teichmuller representative}

\medskip

 Consider for every integer $x$ and arbitrary prime $p$ a sequence:
              $$
           \{u_{n}\}\colon\quad u_1=x, u_{n+1}=u_n^p,
           \quad   u_n=x^{p^n}
           \eqno (1)
              $$
It is easy to see using generelased small Fermat Theorem that
            $$
            p^n| \left(u_{n+1}-u_n\right)
            \eqno (2)
            $$
because:
          $$
u_{n+1}-u_n=x^{p^n}(x^{p^{n+1}-p^n}-1)
          $$
Then if $p|x$ it is evident. Let $x$ be coprime with $x$. Then
$p^{n+1}-p^n=\varphi(p^{n+1})$ is just equal to the quantity of
numbers coprime with $p^n$. Hence  $x^{\varphi(p^n)}$ is divisible
on $p^n$. Hence we proved (2).

\def \Z {{\bf Z}}
The relation (2) has the following interpretation in p-adic ring
$\Z_p$: The sequence (1) tends to p-adic number which is a
solution of the equation
              $$
              x^p=x
               $$



 \bye
