
\magnification=1200 \baselineskip=14pt
\def\vare {\varepsilon}
\def\A {{\bf A}}
\def\t {\tilde}
\def\a {\alpha}
\def\K {{\bf K}}
\def\N {{\bf N}}
\def\V {{\cal V}}
\def\s {{\sigma}}
\def\S {{\Sigma}}
\def\s {{\sigma}}
\def\p{\partial}
\def\vare{{\varepsilon}}
\def\Q {{\bf Q}}
\def\D {{\cal D}}
\def\G {{\Gamma}}
\def\C {{\bf C}}
\def\M {{\cal M}}
\def\Z {{\bf Z}}
\def\U  {{\cal U}}
\def\H {{\cal H}}
\def\R  {{\bf R}}
\def\E  {{\bf E}}
\def\l {\lambda}
\def\degree {{\bf {\rm degree}\,\,}}
\def \finish {${\,\,\vrule height1mm depth2mm width 8pt}$}
\def \m {\medskip}
\def\p {\partial}
\def\r {{\bf r}}
\def\v {{\bf v}}
\def\n {{\bf n}}
\def\t {{\bf t}}
\def\b {{\bf b}}
\def\e{{\bf e}}
\def\ac {{\bf a}}
\def \X   {{\bf X}}
\def \Y   {{\bf Y}}
\def \x   {{\bf x}}
\def \y   {{\bf y}}


   If $(p_1,p_2\dots,p_n)$ are prime numbers then
for every integers $(a_1,a_2,\dots,a_n)$ there exists $N$ such that
   $N\equiv a_i (mod\, p_i)$, ($i=1,2,\dots n$).  
This is famous Chinese residual Theorem.

  If $(x_1,x_2,\dots x_n)$ are $n$ points on the line $\R$ then
for every real numbers $(y_1,y_2,\dots,y_n)$ there exists polynomial
   $P(x)$ such that $P(x_i)=y_i$, ($i=1,\dots,n$).
In fact one can choose unique polynomial of order $\leq n$ obeying these
conditions.  This polynomial is:
                     $$
P(x)=\sum_{i=1}^n 
  {\prod_{m\not=i}(x-x_m)\over\prod_{m\not=i}(x_i-x_m) }y_i
                     $$
(We suppose that the points $x_1,\dots,x_n$ are distinct)

E.g. for sets $(x_1,x_2,x_3)$ and $(y_1,y_2,y_3)$
                     $$
P(x)=
 {(x-x_2)(x-x_3)\over (x_1-x_2)(x_1-x_3)}y_1+
 {(x-x_1)(x-x_3)\over (x_2-x_1)(x_2-x_3)}y_2
 {(x-x_1)(x-x_2)\over (x_3-x_1)(x_3-x_2)}y_3
                     $$
(we suppose that $x_1,x_2,x_3$ are distinct points

{\it what happens if they are not dinstinct?})


  One can see the striking resemblance between these two formulae
  considering the following idea which revolutionased the mathematics
of XX century. 
  A commutative algebra (ring) can be considered as 
algebra (ring)of functions on some set. An arbitrary integer $N$
can be considered as a function on prime numbers with values in reminders.
  Using this idea one can see the striking resemblance
between the Chinese formula and Lagrange polynomial approximation formula

  In the same way we can write:

Let $N$ be ionteger. We consider a function
 $N(p_i)=\hbox{remainder when divided on $p_i$}$
 For example the number $N=124$ defines a function
                     $$
N(2)=0\,(mod\, 2)\,,
\quad N(3)=1\,(mod\, 3)\,,
N(5)=4\,(mod\, 5)\,,
N(7)=5\,(mod\, 7)\,,
N(11)=3\,(mod\, 7)\,,
                     $$             
  \bye
