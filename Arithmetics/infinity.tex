

       \magnification=1200

     \centerline {\bf Points on Integer Distances}

    \medskip

Yulii Rudyak gave me the following problem:

 {\bf $X$ is the set of the points on the plane, such that
 not all the points are on the same line and all the distances
 between the points are integers. Prove that in this case $X$
 has to be a finite set.}

\smallskip

 {\bf Remark} One can easily to construct for every $N$ a set of
$N$ points which are not on the same line and
 all the distances are
 integer. One can also construct the infinite set $X$
of points not on the same line with rational distances
between them.

E.g. consider the points
 $\{A_0,A_1,\dots.A_p,\dots\}$ which have following
cartesian coordinates:
               $A_0$ has coordinates $(0.2)$ and
the point $A_k$ has coordinates $(k-{1\over k})$ ($k=0,1,2,3,\dots$).
It is easy to see that all the distances between these points are rational.
On the other hand for arbitrary $N$ under
dilatation $(x,y)\rightarrow$ $(N!x,N!Y)$ the subset of first
 $N+1$ points transforms to the set with integer distances.

\smallskip

 Now we go to the proof of the Rudyak,s statement.

 Let $A,B,C$ be three points from the set $X$
 which are not on the same line and
  there lengths are equal to $|AB|=c$, $|BC|=a$

We consider the subset $X_{pq}$ which is defined in the following way:
                            $$
    X_{pq}\colon \quad {\bf r} \in X_{pq}\quad{\rm iff}
                         \,\,{\bf r}\in X\quad {\rm and}\,
                       \cases
                         {
                  |\rho({\bf r},A)-\rho({\bf r},B)|=p\,,\cr
                  |\rho({\bf r},A)-\rho({\bf r},C|= q\,\cr
                         }
                                    \eqno (1)
                        $$
In other words $X_{pq}$ are the points of $X$ whose distances
between $A$ and $B$ differs on $p$ and between $A$ and $C$
differs on $q$.

It is evident that the integers $p$ and $q$ are restricted by
   integers $c$ and $b$ respectively: $0\leq p\leq c$,$0\leq q\leq b$,

Hence it remains to prove that all $X_{pq}$ are finite sets,
because
                        $$
      X=\bigcup_{p\leq c, q\leq b}X_{pq}
                \eqno (2)
                         $$
The fact that every $X_{pq}$ is the finite sets immediately follows from the
  fact that  belongs simultaneously to hyperbole
  $\Gamma_p^1$ {\bf r} {\bf r} with focuses in the points $A$ and $B$,
   defined by the first equation in (1) and
to the hyperbole $\Gamma_q^2$ with focuses in the points $A$ and $C$
 defined by the second equation in (1).
 Hence these hyperboles have not more than $4$ general points,
because their focuses do not belong to the same line.
 (The degenerated hyperboles: $\Gamma^1_0$, $\Gamma^2_0$---lines
 and $\Gamma^1_c$ and $\Gamma^2_b$---rayons are considered also like hyperboles.
 "Hyperboles" $\Gamma^1_c$ and $\Gamma_b^2$ --- rayons
 have infinite set of general points if $A,B,C$ would be on the same line)

 We came to the conclusion that the number of ponts in $X_{pq}$
is not more that $4$. From (3) and (2) it follows that number of points in
 $X$ is not more than $4bc$.
The statement is proved.




\bye
