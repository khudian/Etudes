From khudian@manchester.ac.uk Mon Mar 24 17:25:00 2014
Date: Mon, 24 Mar 2014 17:25:00 +0000 (GMT)
From: khudian@manchester.ac.uk
To: khudian@sekhmet.umist.ac.uk
Subject: Re: Derangements and....



   Mark Wildon <Mark.Wildon@rhul.ac.uk>



   Dear Mark
Thank you very much for your letter

        Fi rst of all sorry  for so late answer...
  Immediately when I received your  letter,first I was very 
happy that you answered, second I began immediately (yes immediately!) 
to write the answering letter where I 
wanted to describe you my idea of the proof. After writing about 50--60
 lines  of the  text I realised that I cannot immediately to remember 
the proof (Last time I was thinking about it around 1998).
I stopped and decided to answer 
later. During the next week
 on one hand I tried to remember the proof, on
the other hand I tried to find in my archive the 
papers where it was written.  The next week I did not nothing
since I was very busy. Both goals were failed.... 
In fact this semester I have double teaching
and too other things... not to make the list of complains very long...
  Now I have decided to stop temporarly these  attemtps.
I will send you the letter as it was written on 10-th March when
I received your letter. I hope I could send you continuation later,,
but sure that only when Easter holidays will begin I will be able to 
 return to this problem.

 Mark, sorry, but may be you will find it interesting.



last remark: in my so called `proof' 
I consider permuation of not identical points.
it is more close may be to braids group....
it means that I consider not all bad permutations but only
permitted bad permutations....

So below you will see the beginning of the letter which I began to write 
you two weeks ago....

  %%%%%%%%%%%%%%%%%%%%%%%%%%%%%%%%%%%%%%%%%%%%%%%%%%%%%%%%%%%%%
  %%%%%%%%%%%%%%%%%%%%%%%%%%%%%%%%%%%%%%%%%%%%%%%%%%%%%%%%%%%
10 March 2014.

  The idea of the proof is the following:
Consider dual problem:
There are $N$ lines such that any two of them are not parallel
and for any two lines $a,b$ there is a third line $c$ such that
$c$ intersects lines $a$ and $b$ at the point where lines
$a$ and $b$ intersect. Prove that all these lines
intersect at one point.
   Every line can be viewed as  a graph of the point
  which moves with constant velocity along the standard line $\R$.
     The  problem above can be reformulated in the following
'kinematical way':
Consider $N$ points $A_1,\dots, A_N$ which move
on the line such that velocities $v_1,v_2,\dots,v_N$ of these particles
are all different.
then at every moment when two points collide the third collides them.
Prove that all the bodies collide simutlaneously.
It is this reformulation which goes to group of permutations:
Take the moment of time $t\to -\infty$ when all the points
  are ordered $A_N<A_{N-1}<\dots < A_2<A_1$ by coordinate.
and relation between velocities (before all collisions) are
     $v_N>v_{N-1}>\dots >v_2>v_1$. The $N$-th particle
is on the left and it has the maximum velocity,,,,
  After all collisions at the time $t\to +\infty$ all the points
  are ordered $A_1<A_2<\dots < A_{N-1}<A_N$ by coordinate.

For example collision of three points is when
points $A_i<A_k< A_n$ collide and for velocities
        $v_i>v_k>v_n$.


We come to the following reformulation:

We have $N$ points $\{A_i\}$, $i=1,\dots,n$. 
Any point has the differnet weight
('velocity'). The initial position is $N,N-1,N-2,\dots, 3,2,1$
the higher is number the higher is weight.
At any state the position is $a_1,a_2,\dots. a_N$
$w(a_i)$ is the weight of the point ('velocity')
which takes values ${1,2,3,\dots, N}$  For example at initial
position $N,N-1,N-2,\dots,3,2,1$,
  $w(a_1)=N, w(a_2)=N-1$, $w(a_3)=N-3,$, i.e. $w(a_k)=N-k$.
Weights are decreasing.
At the final position $w(a_k)=k$
We say that $f$ is permited permutation if
$r$ consecutive points
  have decreasing  weights
            $$
w(a_k)>w(a_{k+1})>\dots >w(a_{k+r-1})
            $$
and they permute to the reverse order:
        $$
a_k, a_{k+1},\dots, a_{k+r-1} ------>
a_{k+r-1}, a_{k+r-2},\dots,a_{k+1} a_k$
        $$

E.g. (743)---->(347) is permitted permutation.

We say that permutation is bad if it is permitted permutation of
more than 2 elements.

E.g. (98)--->(89) is permitted but not bad, but
  permutations   (531)--->(135) and (7321)--->(1237) are bad

We say that  permutation 
             $F\colon
 (N,N-1,N-2,\dots, 4,3,2,1)----> (1,2,3,4,\dots, N)$
is exceptional bad permutation.

Proposition 

  If exceptional bad permutation $F$ is a product of
$k$ bad permutations, $F=f_1\circle f_2\circle \dots \circle f_k$ is
then $k=1$.

  It is this proposition that I remember that I proved in 1997,
and I cannot recall it.   
**********************************************************************
*******************************************************

    Sure I will be happy to asnwer all the questions.....
  


     Best

                      Hovik
                                     Dr Hovhannes Khudaverdian
                                    The University of Manchester
                                       School of Mathematics
                                   e-mail: khudian@manchester.ac.uk
                                        tel. 0161-2008975



On Mon, 10 Mar 2014, Mark Wildon wrote:

> Dear Hovik,
>
> Yes, your problem is due to Sylvester: there is a reference
> to the original statement in 'Proofs from the Book' by
> Aigner and Zeigler. The 'book proof' they give uses distances
> in R^2. They mention that the 7 point projective plane
> (7 points, 7 lines, any three points on a unique line) shows
> that the result doesn't follow just from incidence properties.
>
> Your proof sounds very interesting, and I would be interested
> to see it.
>
> The permutation you mention sending (1,2,...,n) to (n,n-1...,1)
> is the unique longest element in the symmetric group S_n, where
> the length of a permutation g is defined to the number of pairs
> {i,j} such that i < j and g(i) > g(j). (Such pairs are called
> inversions.) An equivalent definition is that the length of g
> is the minimum length of an expression of g using only the
> transpositions (1,2), (2,3), ..., (n-1,n). Of course this may
> well not be of any relevance to your proof, but just in case
> it is interesting to you, I thought I'd mention it.
>
> Best wishes, Mark
>
> Dr Mark Wildon                          mark.wildon@rhul.ac.uk
> Department of Mathematics           www.ma.rhul.ac.uk/~uvah099
> Royal Holloway                               tel: 01784 414021
> Egham TW20 0EX                              office: 240 McCrea
>
> On 5 Mar 2014, at 16:50, <khudian@manchester.ac.uk> wrote:
>
>>
>>        Dear Mark!
>>  Thank you very much for your letter
>> I came on your seminar just occasionally, enjoyed it very much!
>> and by the way rcalled  the problem
>> which I call `$n$ drunk people problem'.
>>
>>
>>  In fact I am from mathematical physics, combinatorics for me it is just
>> hobby.
>> Sorry, I have one question:
>>  There is the following famous problem (I think it is Silvester problem)
>> in elementary mathematics:
>>  There are $N$ points on the line such that every
>> three points are on the same line. Prove that all these points are
>> on one line. This statement has elementary proof.
>> Many many years ago trying to solve this problem myself
>> (fortunately or uinfortunately I could not find the well-known proof
>> elementary proof) I reduced this problem to the combinatorial
>> problem of permutations in $S_n$:
>> we call the element $g\in S_n$ `bad' if it is permutation
>> of more than 2 elements.   Consider  permutation
>> $1,2,3,4,...,n-3,n-2,n-1,n\to n,n-1,n-2,n-3,dots,3,2,1$
>> This permutaion sure is bad permutation (suppose $n\geq 3$).
>> One can prove that
>> it cannot be represented as a product of more than one bad permutations.
>>
>> This combinatorial fact proves that all the points are on one line.....
>>
>>  Do you have any comment on this?
>>
>>   Sure I will be happy to speak with you about this problem if
>> you woud like..
>>
>>           Hovik
>>
>> PS You may call me just Hovik (this is short version of my name)
>>
>>
>>
>>                                    Dr Hovhannes Khudaverdian
>>                                   The University of Manchester
>>                                      School of Mathematics
>>                                  e-mail: khudian@manchester.ac.uk
>>                                       tel. 0161-2008975
>>
>>
>>
>> On Wed, 5 Mar 2014, Mark Wildon wrote:
>>
>>> Dear Hovhannes,
>>>
>>> I found your very nice derangements proof on the web.
>>>
>>> In my course I give a similar (but not identical proof):
>>> dividing your equation (1) by n!, it states that the
>>> convolution of the sequences
>>> 	(1, 1/2!, 1/3!, 1/4!, ... )
>>> with the sequence
>>> 	(S_1/1, S_2/2!, S_3/3!, ... )
>>> is equal to the constant sequence (1,1,...). This is
>>> equivalent to the generating function identity
>>> 	exp(x) F(x) = 1/(1-x)
>>> where F(x) = \sum S_k/k! x^k, as in your note.
>>>
>>> Both our proofs are related to binomial inversion: I think
>>> this technique is not as widely known as it deserves.
>>>
>>> Best wishes, Mark
>>>
>>> Dr Mark Wildon                          mark.wildon@rhul.ac.uk
>>> Department of Mathematics           www.ma.rhul.ac.uk/~uvah099
>>> Royal Holloway                               tel: 01784 414021
>>> Egham TW20 0EX                              office: 240 McCrea
>>>
>>>
>>>
>>
>
>
>
