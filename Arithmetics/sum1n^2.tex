 \centerline {\bf On one very beautiful formula for $\sum {1\over n^2}$}

\magnification=1200
\medskip
\def\a{\alpha}

  {\it Written under impression of the article
  "How to compute $\sum 1/n^2$ by solving triangles"
    Author: Mikael Passare, math.GM/0701039})

\medskip

  Let $AB$ be a segment of unit length. For any point $C$ on the
  plane one can consider coordinates $(u,v)$ or $(\a,\beta)$ where
  $|CA|=e^{-u}, |CA|=e^{-v}$
 and respectively $\a=\angle CAB,\quad \beta=\angle CBA$
It turns out that transformation
      $$
  u=u(\a,\beta), v=v(\a,\beta)
      $$
has jacobian equal to unity!, i.e. it is area preserving transformation!.

\medskip

Set of coordinates $\a,\beta$ is:$\a,\beta\colon \a+\beta< \pi, \a,\beta\geq 0$.
Evidently it is triangle of the area $\pi^2/2$.
One third of this triangle transforms to the domain:$D=\{u,v,\colon u,v>0, e^{-u}+e^{-v}\leq 1\}$.
It is easy to calculate area of this domain:
         $$
     \hbox{Area of $D$}=\int_0^\infty \log{1\over 1-e^{-x}}=\int_0^\infty\sum {e^{-nx}\over n}=\sum {1\over n^2}
         $$

Because map $u=u(\a,\beta), v=v(\a,\beta)$ is area preserving, then
  we see that
          $$
   \sum {1\over n^2}={\pi^2\over 2}:3={\pi^2\over 6}
          $$

\bye

The set $(u,v)$ spans domain which is naturally related
 with sum $\sum {1\over n^2}$.
