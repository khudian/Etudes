\magnification=1200 %\baselineskip=14pt
\def\vare {\varepsilon}
\def\A {{\bf A}}
\def\t {\tilde}
\def\a {\alpha}
\def\K {{\bf K}}
\def\N {{\bf N}}
\def\V {{\cal V}}
\def\s {{\sigma}}
\def\S {{\Sigma}}
\def\s {{\sigma}}
\def\p{\partial}
\def\vare{{\varepsilon}}
\def\Q {{\bf Q}}
\def\D {{\cal D}}
\def\G {{\Gamma}}
\def\C {{\bf C}}
\def\M {{\cal M}}
\def\Z {{\bf Z}}
\def\U  {{\cal U}}
\def\H {{\cal H}}
\def\R  {{\bf R}}
\def\S  {{\bf S}}
\def\E  {{\bf E}}
\def\l {\lambda}
\def\degree {{\bf {\rm degree}\,\,}}
\def \finish {${\,\,\vrule height1mm depth2mm width 8pt}$}
\def \m {\medskip}
\def\p {\partial}
\def\r {{\bf r}}
\def\v {{\bf v}}
\def\n {{\bf n}}
\def\t {{\bf t}}
\def\b {{\bf b}}
\def\c {{\bf c }}
\def\e{{\bf e}}
\def\ac {{\bf a}}
\def \X   {{\bf X}}
\def \Y   {{\bf Y}}
\def \x   {{\bf x}}
\def \y   {{\bf y}}
\def \G{{\cal G}}


\centerline{\bf Harmonic series diverge. Concequences.}


We mention here two problems for pupils, which are related
strictly with the fact that harmonic series $s_n=1+{1\over
2}+{1\over 3}+\dots+{1\over n}$ diverge, (more precisely
$1+{1\over 2}+{1\over 3}+\dots+{1\over n}\approx \log n $).




{\bf Problem 1} Put without a glue $N$ equal bricks on each other
 such that the construction have maximum length  $L_N$.  Estimate
 $L_N$.

  Answer is little bit counter-intuitive:
  $L_N$ tends to infinity if $N\to \infty$ (More precisely $L_N\approx \log N$). Show it.


  \medskip

 Suppose all the bricks have length $2$.
 Consider series $\{x_1,x_2,x_3,\dots,x_N\}$, where
where $x_k$ is $x$ coordinate of the $k$-th brick counting from
the top.  The brick on the top does not fall, hence:
                  $$
              x_2-1\leq x_1\leq x_2+1
                   $$
The brick on the top and the next brick does not fall, hence:
             $$
             x_3-1\leq {x_1+x_2\over 2}\leq x_3+1
             $$
and so on:
             $$
       x_4-1\leq {x_1+x_2+x_3\over 3}\leq x_4+1,\,\,\,x_5-1\leq {x_1+x_2+x_3+x_4\over 4}\leq
       x_5+1,\dots
             $$
            $$
x_{k+1}-1\leq {x_1+x_2+x_3+\dots+x_k\over k}\leq
x_{k+1}+1,\,\,\hbox{for every $1\leq k\leq N-1$}
            $$
We see that the construction has maximum length $L_N=x_N+1$ if:
                $$
x_2=1+x_1, x_3=1+{x_1+x_2\over 2},x_4=1+{x_1+x_2+x_3\over
3},\dots, x_{N}=1+{x_1+x_2+x_3+\dots+x_N\over N}
                $$
Suppose $x_1=0$, then we one can see that
           $$
  x_2=1,x_3=1+{1\over 2},x_4=1+{1\over 2}+{1\over 3},\dots,
  x_N=1+{1\over 2}+{1\over 3}+\dots+{1\over
              N-1}
           $$
We see that the length of the construction is defined by harmonic series and it tends to $\infty$!

{\bf Remark} By the way we come to identity which is not
absolutely trivial:
             $$
 1+{1+\left(1+{1\over 2}\right)+\left(1+{1\over 2}+{1\over 3}\right)+
    \dots+\left(1+{1\over 2}+{1\over 3}+\dots+{1\over k}\right)\over
    k}=1+{1\over 2}+{1\over 3}+\dots+{1\over k+1}
             $$


{\bf Problem 2}.  The ant is at the  the point $A$ on ribbon $AB$. Length of the ribbon is equal to
$1km$.  Every second ant travels $1cm$ over the ribbon, then at the end of the second the length of the ribbon
is increased on $1km$.  Show that ant will reach the point $B$. Estimate the time\footnote{$^*$}{On dit that
this problem was suggested by A.D. Sakharov}

 Here harmonic series interfere too!

  Denote by $h_n$ the length of the ribbon which ant already passed.
  (The length of the ribbon is increasing homogenously hence $h_n>n$).
   After $k$ seconds the length of the ribbon becomes $(k+1)\Delta$ (where $\Delta=10^5$).
   Denote by $\delta_n={h_n\over (n+1)\Delta}$.
   The ant reaches the point $B$ on the ribbon when $\delta_N=1$.
       One can see that
                $$
   \delta_{n+1}=\left(h_n+1\right)\cdot {(n+2)\over \Delta(n+1)}={h_n(n+2)\over \Delta(n+1)}=
   \delta_n+{1\over\Delta( n+2)}
                $$
We see that
                $$
  1\approx \delta_N={1\over \Delta}\left(1+{1\over 2}+{1\over 3}+\dots+{1\over N+1}\right)\Rightarrow
  N\approx e^{\Delta}
                $$
 So ant reaches the point $B$ after $e^\Delta$ seconds (it is much more than Universe time $10^{43}$)

 \bye
