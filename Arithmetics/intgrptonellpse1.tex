\magnification=1200
\baselineskip=14pt
\def\vare {\varepsilon}
\def\A {{\bf A}}
\def\t {\tilde}
\def\a {\alpha}
\def\K {{\bf K}}
\def\N {{\bf N}}
\def\V {{\cal V}}
\def\s {{\sigma}}
\def\S {{\Sigma}}
\def\s {{\sigma}}
\def\p{\partial}
\def\vare{{\varepsilon}}
\def\Q {{\bf Q}}
\def\D {{\cal D}}
\def\G {{\Gamma}}
\def\C {{\bf C}}
\def\M {{\cal M}}
\def\Z {{\bf Z}}
\def\U  {{\cal U}}
\def\H {{\cal H}}
\def\R  {{\bf R}}
\def\E  {{\bf E}}
\def\l {\lambda}
\def\degree {{\bf {\rm degree}\,\,}}
\def \finish {${\,\,\vrule height1mm depth2mm width 8pt}$}
\def \m {\medskip}
\def\p {\partial}
\def\r {{\bf r}}
\def\v {{\bf v}}
\def\n {{\bf n}}
\def\t {{\bf t}}
\def\b {{\bf b}}
\def\e{{\bf e}}
\def\ac {{\bf a}}
\def \X   {{\bf X}}
\def \Y   {{\bf Y}}
\def \x   {{\bf x}}
\def \y   {{\bf y}}
\def\w {{\omega}}

\centerline  {\bf Integer points on the ellipse}

\bigskip

Consider on $\R^2$ an ellipse
             $$
   ax^2+2bxy+cy^2=1,\quad (ac-b^2>0, a>0)
   \eqno (1)
             $$
such that
                $$
       ac-b^2=1,\,\,{\rm and}\,\,\,a,c,b\in \Z,
                $$
   i.e. quadratic form  $ax^2+2bxy+cy^2$ is defined by symmetric
   matrix in $SL(2,\Z)$\footnote{$^*$}{A group $SL(2,\Z)$ is a group of $2\times 2$ matrices with integer entries.}.
\medskip

What about integer points (points with integer coordinates) on this ellipse, in the interior of this  ellipse?

{\it {\bf Fact 1} The interior of the ellipse (2) possesses $4$ points with integer coordinates
(except the origin $(0,0)$ ). All these points are on the ellipse (1).}


{\bf Remark } It is well-known that any domain $M$ of the area $1$
possesses at least two points $\r_1,\r_2$ such that vector $\r_2-\r_1$ has integer coordinates (Minkovsky lemma).
  This implies the following

{\it {\bf Fact 2} Any central-symmetric convex domain $M$ of the area $S(M)=4$ possesses at least one
point with integer coordinates except the point $(0,0)$}.

It is evident that for an arbitrary $\varepsilon>0$, there exists
central-symmetric convex domain $M_\varepsilon$ of the area $S(M)=4-\varepsilon$ which does not possess
 any point with integer coordinates except the point $(0,0)$.
On the other hand it follows from the {Fact 1} that the ellipse
(1) is a central-symmetric convex domain of the area
$S(\Delta)=\pi<4$ which possesses 4 integer points.

\medskip

   {\sl Proof of the Fact 1}.

  This is evident in the case if $a=b=1$ and $b=0$. Ellipse becomes circle which possesses exactly four
   integer points $(1,0)$,  $(1,1)$, $(-1,0)$ and $(1,1)$ (on the boundary).

  The Fact 1 follows from the following Proposition

\m

  {\bf Proposition}  {\it A matrix equation $X^+X=B$
          has a solution $X\in SL(2,\Z)$  if $B$ is symmetric matrix in
          $SL(2,\Z)$.}



\m

Indeed let $X=\pmatrix {\a &\beta\cr \gamma &\delta}$, ($\a,
\beta\,\gamma, \delta\in \Z$) be a solution of the equation (3)
where $B$ is a matrix of quadratic form $ax^2+2bxy+cy^2$, which
defines an ellipse (1): $B=\pmatrix {a &b\cr b&c}$. Then linear
transformation
             $
         \pmatrix {x\cr y\cr}\rightarrow  \pmatrix {\a &\beta\cr \gamma &\delta}\pmatrix {x\cr y\cr}=
         \pmatrix {\alpha x+\beta y\cr \gamma x+\delta y\cr}
             $
transforms circle $x^2+y^2=1$ onto the ellipse (1). This linear
transformation  establishes one-one map of lattice of points with
integer coordinates onto itself, since $\det X=1$. Points with
integer coordinates on the ellipse (1) are images of the points
  $(1,0)$,  $(1,1)$, $(-1,0)$ and $(1,1)$. They are 4 points
  $(\alpha,\gamma)$, $(\beta,\delta)$,  $(-\alpha,-\gamma)$ and $(\beta,\delta)$.

  It remains to prove the Proposition.

  {\sl Proof}

  Consider matrices $
              S=\pmatrix {0 & 1\cr -1 &0}$ and $T=\pmatrix {1 &1\cr -1 &0}$
 in $SL(2,\Z)$, $S^2=1$ and $T^3=1$\footnote{$^*$}{Matrices $T,S$ are generators of the group $SL(2,\Z)$.
 The proof of Proposition is in the spirit of the proof of this statement.}.
        Analyze the action of these matrices $T$ and $S$ on the quadratic form $ax^2+2bxy+cy^2$:
                             $$
B=\pmatrix {a &b\cr b &c}\to S^+BS=\pmatrix {0 & -1\cr 1 &0}\pmatrix {a &b\cr b &c}\pmatrix {0 & 1\cr -1 &0}=
                  \pmatrix {c &-b\cr -b &a}
                             $$
and
                           $$
B=\pmatrix {a &b\cr b &c}\to T^+BT=\pmatrix {1 & -1\cr 1 &0}\pmatrix {a &b\cr b &c}\pmatrix {1 & 1\cr -1 &0}=
                  \pmatrix {a+c-2b &a-b\cr a -b &a}\,.
                           $$
It suffices to show that subsequent actions of these transformations a
matrix $B=\pmatrix {a &b\cr b &c}$ can be transformed
to unity matrix $E=\pmatrix {1 &0\cr 0 &1}$, i.e.
                $$
          M^+BM=E,\quad {\rm where}\,\,\,
M=S^{m_1}T^{n_1}S^{m_2}T^{n_2}\dots S^{m_{k-1}}T^{n_{k-1}}S^{m_k}T^{n_k}\,.
                        \eqno (2)
              $$
In this case matrix  $B=X^+X$ and the matrix
$X=T^{2n_k}S^{m_k}T^{2n_{k-1}}S^{m_{k-1}}\dots T^{2n_2}S^{m_2}T^{2n_1}S^{m_1}$
is a solution of equation in Proposition.

  To prove the relation (2) note that if $b=0$ (in matrix $B$) then $B=E$.
    If $B\to S^+BS$ then $b\to -b$, if
     $B\to T^+BT$ then $b\to a-b$ and if
     $B\to T^+T^+BTT$ then $b\to c-b$.  On the other hand
     if $b>0$ then $|a-b|<b$ or $|c-b|<b$. Therefore acting on
     $B$ by one of the matrices $T$, or $T^2$, $TST$, $ST$, or $ST^2$ or $STST$
     we decrease absolute value of $b$ at lest on one. Repeating this procedure we come to $b=0$
     \footnote{$^{**}$}
      {More puristic way to say it is following:  For a given matrix $B$ consider a set $\cal M$
     of all matrices  $K^+BK$ where $K$ is a matrix generated by matrices $S$ and $T$
       ($K=S^{m_1}T^{n_1}S^{m_2}T^{n_2}\dots
                  S^{m_{k-1}}T^{n_{k-1}}S^{m_k}T^{n_k}$). Consider in this set the matrix
                  $B_0=\pmatrix{a &b\cr b &a}$ such that
                 entry $b=B_{12}$ is minimal. Show that $b=0$, thus $B_0=E$.
                 Suppose that $b\not=0$, then acting on $B$ by matrices $S$ and $T$ we can decrease the value of $|b|$.
                 Contradiction.}\finish




\bye
