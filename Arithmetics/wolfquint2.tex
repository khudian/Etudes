



\magnification=1200

\def\p {\partial}
\def\a {\alpha}

  \centerline{\bf   Волчья квинта----Wolf inteval and
  how many notes there are in octave}

\bigskip
 
  {\it We know that quinte (fifth) $={3\over 2}$ and on the piano
 it is equal to   $2^{7\over 12}\approx {3\over 2}$,
If we have 12 fifths it will take 7 octaves:
              $$
 \left(2^{7\over 12}\right)^{12}=2^7
  \approx  \left(3\over 2\right)^{12}\Leftrightarrow
   {3^{12}\over 2^{12+7}}=
   {3^{12}\over 2^{19}}={531441\over 524288}\approx 1\,.
              $$
 These are numbers\footnote{$^*$}
{The numbers $3^{12}$ and $2^{19}$...Oh, sweet memory:
 I used very much these numbers 
about 25 years ago playing different games
 with David and Tigran.
Now they appear in another manifestation:
  we are looking here for integers $p,q$
such that $3^p\approx 2^q$.}
 which  are related with   
so called ``wolf quinte'' and 
to openednss of circles in music       $$
\hbox{C-dur}\,{\buildrel quinte \over \rightarrow}\,
\hbox{G-dur}\,{\buildrel quinte \over \rightarrow}\,
\hbox{D-dur}\,{\buildrel quinte \over \rightarrow}\,
\hbox{A-dur}\,{\buildrel quinte \over \rightarrow}\,
\hbox{M-dur}\,{\buildrel quinte \over \rightarrow}\,
\hbox{H-dur}\,{\buildrel quinte \over \rightarrow}\,
      $$
      $$
\hbox{Fis-dur}\,{\buildrel quinte \over \rightarrow}\,
\hbox{Cis-dur}\,{\buildrel quinte \over \rightarrow}\,
\hbox{Ges-dur}\,{\buildrel quinte \over \rightarrow}\,
\hbox{Des-dur}\,{\buildrel quinte \over \rightarrow}\,
\hbox{As-dur}\,{\buildrel quinte \over \rightarrow}\,
\hbox{F-dur}\,{\buildrel quinte \over \rightarrow}\,
       $$
You return to C-dur but in the new manifestation!


In fact this is related with continuous fraction for `quinta':

We will study this fraction and will see by the way why there
are 12 notes.}

\bigskip

Let $\alpha$ be a number such that
      $$
  \alpha\colon\quad  3=2^\a     
      $$
Then continuos fraction  of $\a$ gives a good approximation
to $\a$ by rational numbers:
    $$
\a=[M_1,M_2,M_3,M_4,\dots]
    $$
We have $\a=M_1+\dots$, i.e.
        $$
    2^\a=2^{M_1+\dots}=3\,,
        $$
i.e.
        $$
     2^{M_1}<3\,,\quad 2^{M_1+1}>3\Rightarrow M_1=1\,.
        $$
Then $\a=M_1+{1\over M_2+\dots}=1+{1\over M_2+\dots}$, i.e.
        $$
    2^\a=2^{1+{1\over M_2+\dots}}=3\,,
        $$
i.e.
          $$
    2^{{1\over M_2+\dots}}={3\over 2}\Rightarrow
      \left({3\over 2}\right)^{M_2+\dots}=2\,,
          $$
i.e.
        $$
      \left(3\over 2\right)^{M_2}<2\,,\quad {\rm but}\,\,
\left(3\over 2\right)^{M_2+1}>2
        $$
i.e.
        $$
    3^{M_2}<2^{M_2+1}\,,\,{\rm but}\,\,
      3^{M_2+1}>2^{M_2+2}\Rightarrow M_2=1\,.
         $$
Then $\a=M_1+{1\over M_2+{1\over M_3+\dots}}=
         1+{1\over 1+{1\over M_3+\dots}}$, i.e.
        $$
    2^\a=2^{1+{1\over 1+{1\over M_3+\dots}}}=3\,,
        $$
i.e.
          $$
    2^{{1\over 1+{1\over M_3+\dots}}}={3\over 2}
            \Rightarrow
      \left(3\over 2\right)^{1+{1\over M_3+\dots}}=2\,,
            \Rightarrow
      \left(3\over 2\right)^{{1\over M_3+\dots}}={4\over 3}\,,
            \Rightarrow
      {3\over 2}=\left(4\over 3\right)^{M_3+\dots}\,,
          $$
i.e.
        $$
      \left({4\over 3}\right)^{M_3}<{3\over 2}\,,
     \quad {\rm but}\,\,
      \left({4\over 3}\right)^{M_3+1}>{3\over 2}\,,
        $$
i.e.
        $$
    2^{2M_3+1}<3^{M_3+1}\,,
   \quad {\rm but}\,\,
    2^{2M_3+3}>3^{M_3+2}\Rightarrow M_3=1\,,
         $$
Then
 $\a=M_1+{1\over M_2+{1\over M_3+{1\over M_4+\dots}}}=
       1+{1\over 1+{1\over 1+{1\over M_4+\dots}}}  $, i.e.
        $$
    2^\a=2^{1+{1\over 1+{1\over 1+{1\over M_4+\dots}}}}=3\,,
        $$
i.e.
          $$
    2^{{1\over 1+{1\over 1+{1\over M_4+\dots}}}}={3\over 2}
            \Rightarrow
      \left(3\over 2\right)^{1+{1\over 1+{1\over M_4+\dots}}}=2\,,
            \Rightarrow
      \left(3\over 2\right)^{{1\over 1+{1\over M_4+\dots}}}={4\over 3}\,,
            \Rightarrow
      {3\over 2}=\left(4\over 3\right)^{1+{1\over M_4+\dots}}\,,
          $$
i.e.
           $$
        \left(4\over 3\right)^{1\over M_4+\dots}=
         {3\over 2}\cdot {3\over 4}
          \Rightarrow
         \left(9\over 8\right)^{M_4+\dots}={4\over 3}\,,
           $$
i.e.
        $$
      \left({9\over 8}\right)^{M_4}<{4\over 3}\,,
     \quad {\rm but}\,\,
      \left({9\over 8}\right)^{M_4+1}>{4\over 3}\,,
        $$
i.e.
        $$
    3^{2M_4+1}<2^{3M_4+2}\,,
   \quad {\rm but}\,\,
    3^{2M_4+3}>2^{3M_4+5}\Rightarrow M_4=2\,,
         $$
Indeed if $M_4=2$ then $3^5=243<2^8=256$, but
if  $M_4=3$ then $3^7=729\times 3=2187>2^11=1024\times 2=2048$. 

  Continue:
 $\a=M_1+{1\over M_2+{1\over M_3+{1\over M_4+{1\over M_5+\dots}}}}=
       1+{1\over 1+{1\over 1+{1\over 2+{1\over M_5+\dots}}}} $, 
i.e.
        $$
    2^\a=2^{1+{1\over 1+{1\over 1+{1\over 2+{1\over M_5+\dots}}}}}=3\,,
        $$
i.e.
          $$
    2^{{1\over 1+{1\over 1+{1\over 2+{1\over M_5+\dots}}}}}={3\over 2}
            \Rightarrow
      \left(3\over 2\right)^{1+{1\over 1+{1\over 2+{1\over M_5+\dots}}}}=2\,,
            \Rightarrow
      \left(3\over 2\right)^{{1\over 1+{1\over 2+{1\over M_5+\dots}}}}
           ={4\over 3}\,,
            \Rightarrow
      {3\over 2}=\left(4\over 3\right)^
        {1+{1\over 2+{1\over M_5+\dots}}}\,,
          $$
i.e.
           $$
    \left(4\over 3\right)^{1\over 2+{1\over M_5+\dots}}=
         {3\over 2}\cdot {3\over 4}
          \Rightarrow
         \left(9\over 8\right)^{2+{1\over M_5+\dots}}={4\over 3}\,,
          \Rightarrow
         \left(9\over 8\right)^{{1\over M_5+\dots}}={4\over 3}\cdot 
           {64\over 81}
          \Rightarrow
         \left(256\over 243\right)^{M_5+\dots}={9\over 8}
           $$
i.e.
        $$
      \left({256\over 243}\right)^{M_5}<{9\over 8}\,,
     \quad {\rm but}\,\,
      \left({256\over 243}\right)^{M_5+1}>{9\over 8}\,,
        $$
i.e.
        $$
    2^{8M_5+3}<3^{5M_5+2}\,,
   \quad {\rm but}\,\,
    2^{8M_5+11}>3^{5M_5+7}\Rightarrow M_5=2\,,
         $$
Indeed if $M_5=2$ then $2^{19}=524288<3^{12}=531441$
(these are famous phone numbers!!!!)
, but
if  $M_5=3$ then 
            $$
   2^{27}=134217728>3^{17}=129140163
            $$

Calculate next?  Let us try:

Repeat recurrently:

We already have:

 $\a=M_1+{1\over M_2+{1\over M_3+{1\over M_4+{1\over M_5+
   {1\over M_6+\dots}}}}}=
       1+{1\over 1+{1\over 1+{1\over 2+
{1\over 2+{1\over M_6+\dots}}}}} $, `
i.e.
        $$
    2^\a=2^{1+{1\over 1+{1\over 1+
{1\over 2+{1\over 2+{1\over M_6+\dots}}}}}}=3\,,
        $$
i.e.
          $$
    2^{{1\over 1+{1\over 1+{1\over 2+{1\over 2+{1\over M_6+\dots}}}}}}
          ={3\over 2}
            \Rightarrow
      \left(3\over 2\right)^{1+{1\over 1+{1\over 2+
{1\over 2+{1\over M_6+\dots}}}}}=2\,,
            \Rightarrow
\left(3\over 2\right)^{{1\over 1+{1\over 2+{1\over 2+
{1\over M_6+\dots}}}}}
           ={4\over 3}\,,
            \Rightarrow
            $$
            $$
      {3\over 2}=\left(4\over 3\right)^
        {1+{1\over 2+{1\over 2+{1\over M_6+\dots}}}}\,,
          $$
i.e.
           $$
    \left(4\over 3\right)^{1\over 2+{1\over 2+{1\over M_6+\dots}}}=
         {3\over 2}\cdot {3\over 4}
          \Rightarrow
         \left(9\over 8\right)^{2+{1\over 2+{1\over M_6+\dots}}}=
         {4\over 3}\,,
          \Rightarrow
           $$
           $$
         \left(9\over 8\right)^{{1\over 2+{1\over M_6+\dots}}}
       ={4\over 3}\cdot 
           {64\over 81}
          \Rightarrow
\left(256\over 243\right)^{2+{1\over M_6+\dots}}={9\over 8}\,,
          \Rightarrow
\left(256\over 243\right)^{{1\over M_6+\dots}}={9\over 8}\cdot 
   \left({243\over 256}\right)^2
          \Rightarrow
             $$
             $$
{256\over 243}=
         \left(
  9\cdot 243^2\over 8\cdot 256^2
    \right)^{M_6+\dots}=
         \left(
  3^{12}\over 2^{19}
    \right)^{M_6+\dots}
                   $$
i.e.
        $$
   \left(3^{12}\over 2^{19}\right)^{M_6}<{256\over 243}
     \quad {\rm but}\,\,
   \left(3^{12}\over 2^{19}\right)^{M_6+1}>{256\over 243}
        $$
i.e.
        $$
   3^{12M_6+5}<2^{19M_6+8}
   \quad {\rm but}\,\,
   3^{12M_6+17}>2^{19M_6+27}
\Rightarrow M_6=3\,,
         $$
Indeed if  $M_6=3$ then 
            $$
 {3^{12M_6+5}\over 2^{19M_6+8}}=
 {3^{41}\over 2^{65}}\approx 0.9886\,
    \quad {\rm and}\quad
 {3^{12(M_6+1)+5}\over 2^{19(M_6+1)+8}}=
 {3^{53}\over 2^{84}}\approx 1.00201\,
            $$
This we can continue.....


   The rules for the fraction:

          $$
\matrix        {
            3^{M_2}<2^{M_2}\cr
             2^{2M_3+1}<3{M_3+1}\cr
            3^{2M_4+1}<2^{3M_4+2}\cr
             2^{8M_5+3}<3{5M_5+2}\cr
            3^{12M_6+5}<2^{19M_6+8}\cr
                \dots\cr     
                     }
            $$

      Aproximations:

      $$
\log_2 3=[1,1,1,2,2,3\dots]=1+{1\over 1+{1\over 1+{1\over 2+
{1\over 2+{1\over 3+\dots}}}}}
      $$
 
  1) $\a=\log_2 3\approx 1$
  

  2) $\a=\log_2 3\approx 1+{1\over 1}=2$
  
  3) $\a=\log_2 3\approx 1+{1\over 1+{1\over 1}}={3\over 2}$
  
There are two notes: C and G.

\medskip

  4) $\a=\log_2 3\approx 1+{1\over 1+{1\over 1+{1\over 2}}}=
{8\over 5}$
  
There are 5 notes in octave,  the third is quinte.
          $$
    2^{3\over 5}\approx 1.5157\approx {3\over 2}\,.
          $$
\medskip

 5) $\a=\log_2 3\approx 1+{1\over 1+{1\over 1+{1\over 2+{1\over 2}}}}=
{19\over 12}$

there are 12 notes in octave, the seventh is quinte:
              $$
   2^{7\over 12}\approx 1.498307...\approx {3\over 2}\,,
          $$

\medskip

 6) $\a=\log_2 3\approx 
1+{1\over 1+{1\over 1+{1\over 2+{1\over 2+{1\over 3}}}}}=
{65\over 41}$

there are 41 notes in octave, the twenty-forth  is quinte?
               $$
    2^{24\over 41}\approx 1.5004194...\approx {3\over 2}\,.
          $$


         


\bye 
