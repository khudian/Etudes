
\magnification=1200

\def\G {{\Gamma}}
\def\m{\medskip}

\m
 \centerline {\bf Mean arithmetic and mean geometric}
\m
   \centerline {Hovik Khudaverdian}

\m

  Let $a_1,\dots,a_n$ be non-negative real numbers.


  Then one can consider the following two means:

  Mean arithmetic:
                 $$
            A=A(a_1,\dots,a_n)={a_1+\dots+a_n\over n}
                 $$
and  mean geometric
              $$
        \Gamma=\root n\of {a_1\cdot a_2\cdot\dots\cdot a_n}
              $$

   One of the famous inequalities is
   that
         $$
          A\geq \Gamma,\quad A=\G \iff  a_1=\dots=a_n
          \eqno (0.1)
         $$


         \bigskip

          \centerline   {\bf ${\cal x} 1$  The case of two numbers}

\medskip
It is most elementary case.

  {\bf Problem 1.1.1} Prove the identity
                 $$
             (x-y)^2=(x+y)^2-4xy
             \eqno (1.1a)
                 $$
\m

   {\bf Problem 1.1.2} Show that it follows from this identity:
            $$
           (x+y)^2\geq 4xy
           \eqno (1.1b)
            $$

\m

   {\bf Problem 1.1.3} Prove the inequality (0.1) for $n=2$, i.e. for $a_1,a_2\geq 0$
                $$
             {a_1+a_2\over 2}\geq \sqrt a_1a_2,\quad  {a_1+a_2\over 2}= \sqrt a_1a_2  \iff a_1=a_2
             \eqno (1.1c)
                $$



  Another proof:


   Let $a_1,a_2$ be arbitrary non-negative numbers, $A={a_1+a_2\over 2}$ its mean arithmetic
   and $\G=\sqrt {a_1a_2}$ its mean geometric.

       Consider system of equations  (simultaneous equations)
            $$
            \cases
              {
     x+y=A,\cr xy=\G
          }
          \eqno (1.2a)
            $$

\m

   {\bf Problem 1.2.1}
   Suppose that system (1.2a) has a solution.
    Write down quadratic polynomial $x^2+px+q$ such that
    roots of this polynomial are numbers $x,y$ which are
    the solution of the system (1.2a) (if they exist)

\m

   {\bf Problem 1.2.2}  Consider discriminant of this quadratic polynomial ($D=p^2-4q$)
      Find relation between discriminant of this quadratic polynomial and
        inequality $A\geq \G$

       ({\it Discriminant is just equal to $p^2-4q=4A^2-4\G$ It is greater or equal to zero iff two roots are real,i.e.
       if the system (1.2a)has a solution in real numbers. We see that $D\geq 0$ is just inequality (0.1) for two numbers.)}


   \m

     {\bf Problem 1.2.3}
  Let numbers $x,y$ satisfy the equations.
            $$
            \cases
              {
     x+y=2,\cr xy=2500
          }
          \eqno (1.2a)
            $$
Hence their mean arithmetic is equal to $A={x+y\over 2}=1$ and  their mean geometric is equal
to $\sqrt {xy}=\sqrt 2500=50 $. We see that mean geometric is {\it bigger} than mean arithmetic.
Find a mistake.


\bigskip

  \centerline {\it  Geometric meaning}
\m


Consider a segment $AB$. Let $D$ be a point in this segment, such that $|AD|=a, |DB|=b$. Let a point $O$
be a centre of the segment $AB$. Consider a circle with a centre at the point $O$ and with radius $R={a+b\over 2}$.

The segment $AB$ is a diameter of this circle. Take a perpendicular to the seqment $AB$ at the point $D$.
 Denote by $C$ a point of the intersection of the perpendicular with circle.

   Consider a triangle
$ACB$. $\angle ACB=90^\circ$. $CD$ is the height of this triangle.

  {\bf Problem 1.3.1}

    Consider $\triangle ADC$ and $\triangle BDC$. Show that they are similar.

\m

   {\bf Problem 1.3.2}  Deduce that height is mean geometric of the segments $AD,DB$,
   $DC=\sqrt {AD \cdot DC}=\sqrt ab$

\m
   {\bf Problem 1.3.2}  Show that height is smaller or equal than the radius  $R={a+b\over 2}$.


    Thus we come to  geometrical proof and interpretation of the inequality (0.1):
                           $$
             \hbox{Radius of the circle $\geq $ Height on the hypothenuse}
                           $$


 \bigskip

          \centerline   {\bf ${\cal x} 2$  Attempts to generalise, $n=3,4$}

\m

Try to generalise.

Case $n=3$

{\bf Problem 2.1.1}  Prove the identity
                 $$
               a^3+b^3+c^3-3abc=(a+b+c)(a^2+b^2+c^2-ab-ac-bc)
                 $$
\m

{\bf Problem 2.1.2}
 Using (0.1) for $n=2$ or otherwise prove that  $a^2+b^2+c^2-ab-ac-bc\geq 0$.
  {\it Consider $a^2+b^2\geq 2ab$, $a^2+c^2\geq 2ac$, $c^2+b^2\geq 2cb$ and add these relations}


\m

 {\bf Problem  2.1.3}  Prove the inequality (0.1) for $n=3$


 Alright but it "rabbit of the hat"


Consider case $n=4$. Surprisingly it is easier

Let $a_1,a_2,a_3,a_4$ are four non-negative integers.

  Consider numbers $A_1,A_2$ such that $A_1$ is mean arithmetic of $a_1,a_2$ and
  $A_2$ is mean arithmetic of $a_3,a_4$:
               $$
              A_1={a_1+a_2\over 2},\quad   A_2={a_3+a_4\over 2}
              \eqno (2.1.1)
               $$
Recall inequality (0.1) for two numbers:
               $$
               {x+y\over 2}\geq \sqrt {x y}
               \eqno (2.1.2)
               $$
Using (2.1.1) and few time this inequality we come to the following chain of inequalities:
                $$
      A= {a_1+a_2+a_3+a_4\over 4}=
      {A_1+A_2\over 2}\geq \sqrt {A_1 A_2}=
     \sqrt {\left({a_1+a_2\over 2}\right)}\sqrt {\left({a_3+a_4\over 2}\right)}\geq
                $$
          $$
      \geq\sqrt {\sqrt {a_1a_2}}\sqrt {\sqrt {a_3a_4}}=\root 4\of {a_1a_2a_3a_4}=\G
          $$
Thus we prove (0.1) for $n=4$

  {\bf Problem 2.2.1} Perform the considerations above.

  We see that case $n=4$ is essentially easier than $n=3$



 \bigskip

          \centerline   {\bf ${\cal x} 3$  Inequality for  $n=2^k$}

\m

Using results of the  ${\cal x} 2$ for $n=4$ one can step by step (i.e. by induction) prove it
for $n=2^k$

 {\bf Problem 3.1.1}  Do it.


 \bigskip

          \centerline   {\bf ${\cal x} 4$  General case}


\m

 Let $a_1,\dots,a_n$ be a set of non-negative numbers.

We call the set of these numbers 'normal' set if the inequality (0.1) holds  ($A\geq \G$), i.e.
we call the set of positive numbers $\{a_1,\dots,a_n\}$  'normal set' if
                            $$
                   {a_1+a_2+\dots+a_n\over n} \geq \root n\of {a_1a_2\dots a_n}
                            $$
Do there exist normal sets?

   {\bf Problem 4.1.1}  Show that the set of equal numbers is normal set.

\m

{\bf Problem 4.1.1} Show that the set which contains at least one naught is  'normal' set

\m

 Now more serious problem:


\m

{\bf Problem 4.2.1}   Let $\{a_1,\dots,a_n\}$ be any set of positive integers and $c$ any positive constant.

    Prove that in this case the set  of numbers
             $$
             \left\{{a_1\over c},\dots,{a_n\over c}\right\}
             $$
is normal if and only if the initial set is
normal too.

\m


Why we are interesting in these exercises?  Because we want to reduce our problem to the special
case  {\it when} mean geometric is equal to $1$

 We come to very important exercise:


 \m
{\bf Problem 4.2.2}
 Let $\{a_1,\dots,a_n\}$ be any set of positive integers and $$c=\G=\sqrt {a_1 a_2\dots a_n}$$ be
 mean geometric of these numbers
Consider the set of numbers
             $$
\left \{{a_1\over c},\dots,{a_n\over c}\right\}= \left\{{a_1\over \G},\dots,{a_n\over \G}\right\}
             $$
  Prove that the mean geometric of this new set is equal to 1.

\m


{\bf Problem 4.2.3} It follows from these two exercises that to prove inequality $A\geq \G$ in general case
it is enough to prove it for the case when mean geometric is equal to $1$.  Show it!




 \bigskip

          \centerline   {\bf ${\cal x} 5$  General case, final attack}

\m

Everything now is ready for the final attack. In the paragraphs 2 and 3   we proved the inequality
(0.1) for $n=2^k$. On the other hand in the previous paragraph we reduced this problem to the case
when mean geometric is equal to $1$.

   Now let $x_1,\dots,x_n$ be a set of $n$ non-negative real numbers such that their mean geometric is equal to 1.
   (By the way this means that all these numbers have to  be positive if they are non-negative)

   We call this set 'nice' set

 We have to prove that their mean arithemtic is greater or equal to 1:
                          $$
                   x_1+\dots+x_n\geq n, x_1+\dots+x_n=n \iff x_1=\dots=x_n=1
                   \eqno (5.1)
                          $$

In other words we have to prove that nice set is normal set too.

 If $n=2^k$ then we already proved it in the paragraph 3.


 Suppose $n\not=2^k$. Consider any $k$ such that $n\leq 2^k$. (E.g. if $n=1000$ we can take any $k \geq 10$ )

   Consider a set of numbers which contain numbers $\{x_1,\dots,x_n\}$ and which contain number $1$, $2^k-n$ times:
               $$
 A= \{x_1,\dots,x_n,\underbrace{1,1,\dots,1}_{\hbox{ $2^k-n$ times}}\}
               $$

     {\bf Problem 5.1.1} Show that the set $A$ is nice set.


      \m

      But this nice set contains already $2^k$ elements. Hence this nice set is normal, i.e. its mean arithemtic
      is equal to 1.

      \m
{\bf Problem 5.1.2} Using results of  ${\cal x} 3$ show that  for the nice normal set $A$
             $$
       x_1+\dots+x_n+\underbrace{1+\dots+1}_{\hbox{ $2^k-n$ times}}=2^k
             $$
Deduce from this fact (5.1

Thus we proved the inequality (0.1)

\bigskip


{\bf Problem 5.1.2}  In the proof of (5.1) we fix the value of mean geometric.

  Try to make analogous proof keeping mean arithmetic to be 1.


\m

How we proved (0.1). First we proved for $n=2$, then for $n=4,8,16,32,\dots$
then we reduced this problem then for arbitrary $n$ we reduced problem
for the case if $\G=1$ and proved for this case.  We climb up and climb down by the
ladder of induction.


 \bigskip

          \centerline   {\bf ${\cal x} 6$  Another proof which seems to be very simple}

\m

How we proved (0.1). First we proved for $n=2$, then for $n=4,8,16,32,\dots$
then we reduced this problem then for arbitrary $n$ we reduced problem
for the case if $\G=1$ and proved for this case.

 Consider another "proof" of the statement (0.1)

 Let $a_1,\dots,a_n$ is a set of $n$ arbitrary non-negative numbers. Fix their mean arithemtic $A$,
 i.e. consider all the sets of positive numbers $\{a_1,\dots,a_n\}$ such that
                    $$
                   {a_1+\dots+a_n\over n}=A
                   \eqno (6.1)
                    $$
Prove that for an arbitrary set which satisfies
 (6.1) mean geometric is less or equal to $A$. Suppose it is not true. Hence
 there exist a point where mean geometric is greater than $A$.

Note that at the point $a_1=a_2=\dots=a_n=A$ mean geometric is equal to $A$.

Consider arbitrary set $\{a_1,\dots,a_n\}$ obeying (6.1). Suppose mean geometric
is greater than $A$. Then at least two numbers $a_i,a_j$ are not equal, because
if all the numbers are equal then mean geometric is equal to $A$.
Change numbers $a_i,a_j$ on the numbers $a_i'=a_j'={a_i+a_j\over 2}$. We do not destroy
the condition (6.1) but mean geometric is increased! because $a_ia_j<\left({a_i+a_j\over 2}\right)^2$.

 Now choose a set where mean geometric is greatest. We come to contradiction because
  by equalising non-equal numbers we make mean geometric greater.

   This proof looks deceptively simple. Indeed here we assume by default that
   the maximum attaints. (Sure it is true because (6.1) defines compact domain and
   mean geometric is continuous function, but this need hard stuff from mathematical analysis)












\bye
