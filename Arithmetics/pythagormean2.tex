%From j.montaldi@manchester.ac.uk Fri Jun 30 14:13:11 2006
%Date: Fri, 30 Jun 2006 14:13:23 +0100
%From: James Montaldi <j.montaldi@manchester.ac.uk>
%To: Oganes Khudaverdian <khudian@ma.umist.ac.uk>
%Subject: pythagoras

%Voila!

%James



\documentclass[12pt]{article}
\usepackage{a4}
\usepackage{pslatex}

\parindent=0pt
\parskip=6pt
\def\G {G}%{\Gamma}}
\def\m{\medskip}
\def\b{\bigskip}
\def\RMS{\textsc{rms}}
\def\note#1{\noindent\framebox{\parbox{0.9\textwidth}{#1}}}
\def\problem#1{\medskip\noindent\textbf{Problem #1 }}


%%%%%%%%%%%%%%%%%%5

\begin{document}

\hfill\fbox{\large\sf MM@M, 2006}

\vskip1cm

 \centerline {\Large\bf Pythagorean Means}
\m
   \centerline {Hovik Khudaverdian}

\b

\centerline{\textbf{Introduction}}

\centerline{\textbf{Means of two numbers}}

\m

Let $a$ and $b$ be two non-negative (real) numbers.

There are many different \emph{means} that can be associated to
them:


\begin{description}

\item[\textit{Arithmetic Mean}:] $A = \displaystyle \frac{a+b}2$.

\item[\textit{Geometric Mean}:] $\G = \sqrt{ab}$.

\item[\textit{Harmonic Mean}:] $H$ is defined by $\displaystyle \frac1H =
\frac12\left(\frac1a + \frac1b\right).$

\end{description}

These three are called the \emph{Pythagorean means}.

Then there are some others that are used in various applications,
the most famous of which is the root-mean-square (or \textsc{rms})
(used a lot in the physics of power), which is

\begin{description}
\item[\textit{Root-mean-square}:] $\displaystyle \RMS = \sqrt{\frac{a^2 + b^2}2}$.
\end{description}


And more generally, the \emph{power-mean} of power $p$,

$$M_p = \left(\frac{a^p + b^p}2\right)^{1/p}.$$

Notice that $A=M_1$. What are $M_{-1}$ and $M_2$ ?

This expression is not defined if $p=0$, but interesting things
happen as $p\to 0$.

\bigskip

\textbf{Explore:}\quad  For given numbers $a,b$, how do these
different means compare?  Especially the Pythagorean ones.


\newpage


  Let $a_1,\dots,a_n$ be non-negative real numbers.
One can consider the following two means:

\noindent\textit{Arithmetic Mean}:
                 $$
            A=A(a_1,\dots,a_n)={a_1+\dots+a_n\over n}
                 $$
\noindent\textit{Geometric Mean}:
              $$
        \Gamma=  \sqrt[n]{a_1\cdot a_2 \cdots a_n}
              $$

   One of the famous inequalities is
   that
         $$
          A\geq \Gamma,\quad\mbox{and moreover}\quad A=\G \iff  a_1=\dots=a_n
          \eqno (*)
         $$


         \bigskip

          \centerline   {\S 1 \textbf{The case of two numbers}}

\medskip
This is most elementary case.

  \problem{1.1.1} Prove the identity
                 $$
             (x-y)^2=(x+y)^2-4xy
             \eqno (1.1a)
                 $$


   \problem{1.1.2} Deduce that:
            $$
           (x+y)^2\geq 4xy
           \eqno (1.1b)
            $$

\m

   \problem{1.1.3} Prove the inequality $(*)$ for $n=2$, i.e. for $a_1,a_2\geq 0$
                $$
             {a_1+a_2\over 2}\geq \sqrt{a_1a_2},\quad\mbox{and}\quad  {a_1+a_2\over 2}= \sqrt {a_1a_2}  \iff a_1=a_2
             \eqno (1.1c)
                $$



  Another proof:


   Let $a_1,a_2$ be arbitrary non-negative numbers, $A={a_1+a_2\over 2}$ its arithmetic mean
   and $\G=\sqrt {a_1a_2}$ its geometric mean.

       Consider system of equations  (simultaneous equations)
            $$
            \cases
              {
     x+y=A,\cr xy=\G
          }
          \eqno (1.2a)
            $$

\m

   \problem{1.2.1}
   Suppose that system (1.2a) has a solution.
    Write down the quadratic polynomial $x^2+px+q$ such that
    roots of this polynomial are numbers $x,y$ which are
    the solution of the system (1.2a) (if they exist)

\m

   \problem{1.2.2}  Consider the discriminant of this quadratic polynomial ($``D=b^2-4ac$''). %p2-4q$)
      What is the relation between the discriminant of this quadratic polynomial and
        the inequality $A\geq \G$~?

       \note{({\it Discriminant is just equal to $p^2-4q=4A^2-4\G$. It is greater or equal to zero iff two roots are
       real,i.e.\
       if the system (1.2a) has a solution in real numbers. We see that $D\geq 0$ is just inequality $(*)$ for two
       numbers.)}}


   \m

     \problem{1.2.3}
  Let the numbers $x,y$ satisfy the equations.
            $$
            \cases
              {
     x+y=2,\cr xy=2500
          }
          \eqno (1.2a)
            $$
Hence their arithmetic mean is equal to $A={x+y\over 2}=1$ and their
geometric mean is equal to $\sqrt {xy}=\sqrt 2500=50 $. We see that
the geometric mean is {\it bigger} than the arithmetic mean. Find a
mistake.


\bigskip

  \centerline {\it  Geometric meaning}
\m


Consider a segment $AB$. Let $D$ be a point in this segment, such
that $|AD|=a, |DB|=b$. Let $O$ be the centre of the segment $AB$.
Consider the circle with centre at the point $O$ and radius
$R={a+b\over 2}$.

The segment $AB$ is a diameter of this circle. Take a perpendicular
to the segment $AB$ at the point $D$.
 Denote by $C$ a point of the intersection of the perpendicular with circle.

   Consider the triangle
$ACB$. $\angle ACB=90^\circ$. $CD$ is the height of this triangle.

  \problem{1.3.1}

    Consider $\triangle ADC$ and $\triangle BDC$. Show that they are similar.

\m

   \problem{1.3.2}  Deduce that the height is the geometric mean of the segments $AD$ and
   $DB$: that is,  $DC=\sqrt {AD \cdot DC} = \sqrt ab$

\m
   \problem{1.3.2}  Show that height is smaller than or equal to the radius  $R={a+b\over
   2}$. When can they be equal?


    Thus we come to  geometrical proof and interpretation of the inequality $(*)$:
                           $$
             \hbox{Radius of the circle $\geq $ Height on the hypothenuse}
                           $$


 \newpage

          \centerline   {\S 2 \textbf{  Attempts to generalise: $n=3,4$}}

\m

Try to generalise.

Case $n=3$

\problem{2.1.1}  Prove the identity
                 $$
               a^3+b^3+c^3-3abc=(a+b+c)(a^2+b^2+c^2-ab-ac-bc)
                 $$

\problem{2.1.2}
 Using $(*)$ for $n=2$ or otherwise prove that  $a^2+b^2+c^2-ab-ac-bc\geq 0$.
  ({\it Hint: consider $a^2+b^2\geq 2ab$, and similar expressions involving $a,c$ and $b,c$ })
  % and add these relations})


\m

 \problem{ 2.1.3}  Prove the inequality $(*)$ for $n=3$


 Alright but it's like pulling a rabbit out of a hat. It doesn't help
 us understand how to prove the general case.

\m

Consider the case $n=4$. Surprisingly it is easier.

Let $a_1,a_2,a_3,a_4$ be four non-negative integers.

  Consider numbers $A_1,A_2$ such that $A_1$ is arithmetic mean of $a_1,a_2$ and
  $A_2$ is arithmetic mean of $a_3,a_4$:
               $$
              A_1={a_1+a_2\over 2},\quad   A_2={a_3+a_4\over 2}
              \eqno (2.1.1)
               $$
Recall inequality $(*)$ for two numbers:
               $$
               {x+y\over 2}\geq \sqrt {x y}
               \eqno (2.1.2)
               $$
Using (2.1.1) and few time this inequality we come to the following
chain of inequalities:
                $$
      A= {a_1+a_2+a_3+a_4\over 4}=
      {A_1+A_2\over 2}\geq \sqrt {A_1 A_2}=
     \sqrt {\left({a_1+a_2\over 2}\right)}\sqrt {\left({a_3+a_4\over 2}\right)}\geq
                $$
          $$
      \geq\sqrt {\sqrt {a_1a_2}}\sqrt {\sqrt {a_3a_4}}=\root 4\of {a_1a_2a_3a_4}=\G
          $$
Thus we prove $(*)$ for $n=4$

  \problem{2.2.1} Perform the considerations above.

  We see that case $n=4$ is essentially easier than $n=3$



 \newpage

          \centerline   {\S 3 \textbf{Inequality for  $n=2^k$}}

\m

Using results of \S2 for $n=4$ one can step by step (i.e. by
induction) prove it for $n=2^k$

 \problem{3.1.1}  Do it.


 \bigskip

          \centerline   {\bf ${\cal x} 4$  General case}


\m

 Let $a_1,\dots,a_n$ be a set of non-negative numbers.

We call the set of these numbers a `normal set' if the inequality
$(*)$ holds  ($A\geq \G$), i.e.\ the set of positive numbers
$\{a_1,\dots,a_n\}$  is a `normal set' if
                            $$
                   {a_1+a_2+\dots+a_n\over n} \  \geq \ \sqrt[n]{a_1a_2\dots a_n}
                            $$
Do there exist normal sets?

   \problem{4.1.1}  Show that any set of equal numbers is a normal set.

\m

\problem{4.1.2} Show that the set which contains at least one naught
is a normal set.

\m

 Now to a more serious problem:


\m

\problem{4.2.1}   Let $\{a_1,\dots,a_n\}$ be any set of positive
integers and $c$ any positive constant.

    Prove that in this case the set  of numbers
             $$
             \left\{{a_1\over c},\dots,{a_n\over c}\right\}
             $$
is normal if and only if the initial set is normal too.

\m


Why we are interested in these exercises?  Because we want to reduce
our problem to the special case  when the geometric mean is equal to
$1$

 We come to a very important exercise:


 \m
\problem{4.2.2} Let $\{a_1,\dots,a_n\}$ be any set of positive
integers and let
    $$c=\G=\sqrt {a_1 a_2\dots a_n}$$
be the geometric mean of these numbers Consider the set of numbers
             $$
%\left \{{a_1\over c},\dots,{a_n\over c}\right\}=
 \left\{{a_1\over \G},\dots,{a_n\over \G}\right\}
             $$
  Prove that the geometric mean of this new set is equal to 1.

\m


\problem{4.2.3} It follows from these two exercises that to prove
inequality $A\geq \G$ in general case it is enough to prove it for
the case when geometric mean is equal to $1$.  Show this!




\newpage

          \centerline   {\bf ${\cal x} 5$  General case, final attack}

\m

Everything now is ready for the final attack. In paragraphs 2 and 3
we proved the inequality $(*)$ for $n=2^k$. On the other hand in the
previous paragraph we reduced this problem to the case when
geometric mean is equal to $1$.

   Now let $x_1,\dots,x_n$ be a set of $n$ non-negative real numbers such that their geometric mean is equal to 1.
   (By the way this means that all these numbers have to  be positive if they are non-negative)

   We call this set a `nice set'.

 We have to prove that their arithmetic mean  is greater or equal to 1:
                          $$
                   x_1+\dots+x_n\geq n, \quad\mbox{and}\quad x_1+\dots+x_n=n \iff x_1=\dots=x_n=1
                   \eqno (5.1)
                          $$

In other words we have to prove that a nice set is normal too.

 If $n=2^k$ then we already proved it in paragraph 3.


 Suppose $n\not=2^k$. Consider any $k$ such that $n\leq 2^k$. (E.g. if $n=1000$ we can take any $k \geq 10$ )

   Consider the set of numbers which contains numbers $\{x_1,\dots,x_n\}$ and which contain number $1$, $2^k-n$ times:
               $$
 A= \{x_1,\dots,x_n,\underbrace{1,1,\dots,1}_{\hbox{ $2^k-n$ times}}\}
               $$

     \problem{5.1.1} Show that the set $A$ is a nice set.


      \m

      But this nice set contains already $2^k$ elements. Hence this nice set is normal, i.e. its arithmetic mean
      is greater than or equal to 1. \marginpar{equal to ?}

      \m
\problem{5.1.2} Using results of  ${\cal x} 3$ show that  for the
nice normal set $A$
             $$
       x_1+\dots+x_n+\underbrace{1+\dots+1}_{\hbox{ $2^k-n$ times}}\geq 2^k
             $$\marginpar{\vskip-18mm$= 2^k$??}
Deduce from this fact (5.1)

Thus we proved the inequality $(*)$

\bigskip


\problem{5.1.2}  In the proof of (5.1) we fixed the value of
geometric mean. Now try to make an analogous proof keeping the
arithmetic mean to be 1, instead of the geometric mean.



 \bigskip
 \bigskip
\newpage

          \centerline   {\bf ${\cal x} 6$  Another proof which seems to be very simple}

\m

How did we prove $(*)$? First we proved it for $n=2$, then for
$n=4,8,16,32,\dots$ and then for arbitrary $n$ we reduced the
problem to the case where $\G=1$ and proved it for this case.

 Here is another ``proof'' of the statement $(*)$

 Let $a_1,\dots,a_n$ be a set of $n$ arbitrary non-negative numbers. Fix their arithmetic mean $A$,
 i.e. consider all the sets of positive numbers $\{a_1,\dots,a_n\}$ such that
                    $$
                   {a_1+\dots+a_n\over n}=A
                   \eqno (6.1)
                    $$
To prove that for an arbitrary set which satisfies
 (6.1) its geometric mean is less than or equal to $A$, begin by supposing it is not true. Hence
 there exist a point where the geometric mean is greater than $A$.

Note that at the point $a_1=a_2=\dots=a_n=A$ the geometric mean is
equal to $A$.

Consider arbitrary set $\{a_1,\dots,a_n\}$ obeying (6.1). Suppose
the geometric mean is greater than $A$. Then at least two numbers
$a_i,a_j$ are not equal, because if all the numbers are equal then
geometric mean is equal to $A$.  Change the numbers $a_i,a_j$ to the
numbers $a_i'=a_j'= {a_i+a_j\over 2}$. We do not destroy the
condition (6.1) but the geometric mean is increased! because
$a_ia_j<\left({a_i+a_j\over 2}\right)^2$.

 Now choose a set where the geometric mean is greatest. We arrive at a contradiction because
  by equalising non-equal numbers we can make the geometric mean greater.

This proof looks deceptively simple. However, here we assume by
default that there is a point where the maximum is attained. (Sure
it is true because (6.1) defines a compact domain and the geometric
mean is a continuous function, but this needs hard stuff from
mathematical analysis)

\newpage
\centerline{\S 7 \textbf{More inequalities}}

Given positive numbers $a_1,\dots, a_n$ in addition to the geometric
and arithmetic mean, there is the \emph{harmonic mean} $H$ defined
by
$$ \frac1H = \frac1n \left(\frac1{a_1} + \frac1{a_2}+\cdots +
\frac1{a_n}.\right).$$

Furthermore, let $M=\max\{a_1\dots,a_n\}$ and
$m=\min\{a_1,\dots,a_n\}$.

\problem{7.1} Find inequalities relating $m,M,\Gamma,A$ and $H$.

\end{document}
