%n this version on 20 january 2007
\magnification=1200 \baselineskip=14pt
\def\vare {\varepsilon}
\def\A {{\bf A}}
\def\t {\tilde}
\def\a {\alpha}
\def\K {{\bf K}}
\def\N {{\bf N}}
\def\V {{\cal V}}
\def\s {{\sigma}}
\def\S {{\Sigma}}
\def\s {{\sigma}}
\def\p{\partial}
\def\vare{{\varepsilon}}
\def\Q {{\bf Q}}
\def\D {{\cal D}}
\def\G {{\Gamma}}
\def\C {{\bf C}}
\def\M {{\cal M}}
\def\Z {{\bf Z}}
\def\U  {{\cal U}}
\def\H {{\cal H}}
\def\R  {{\bf R}}
\def\E  {{\bf E}}
\def\l {\lambda}
\def\degree {{\bf {\rm degree}\,\,}}
\def \finish {${\,\,\vrule height1mm depth2mm width 8pt}$}
\def \m {\medskip}
\def\p {\partial}
\def\r {{\bf r}}
\def\v {{\bf v}}
\def\n {{\bf n}}
\def\t {{\bf t}}
\def\b {{\bf b}}
\def\e{{\bf e}}
\def\ac {{\bf a}}
\def \X   {{\bf X}}
\def \Y   {{\bf Y}}
\def \x   {{\bf x}}
\def \y   {{\bf y}}

\centerline  {\bf Arithmetico-geometrical mean, periods and potential of the circle}

 let $M(a,b)$ be arithmetico-geometrical mean of two positive
 numbers. The proof that it is equal to elliptic integral is very good demonstaration of so called
  Zagier-Kontsevitch periods\footnote{$^*$}{ Kontsevitch and Zagier conjectured that  two periods (integrals) are equal if
it can be checked by standard operations.}. Namely consider the following integral:
           $$
    S(a,b)=\int_0^{\pi\over 2}{1\over \sqrt {(a^2\cos^2 t+b^2\sin^2 t)}}dt
        \eqno (1)
           $$
We show that this integral does not change if $(a,b)\to (a_1,b_1)$ where $a_1$ is arithmetic mean of
$a,b$ and $b_1$ is geometric mean of
$a,b$. This implies that integral is proportional to aarithm.-geom.mean.


 The point is that we consider changing of integrand such that only $a,b$ change in {1}
 \footnote{$^{\dagger}$}{There is one beautiful anzats:
 $\varphi\colon \tan \varphi={b\over a}\tan t$. Then one can easy check that
 ${dt\over \sqrt {(a^2\cos^2 t+b^2\sin^2 t)}}={d\varphi\over \sqrt {(b^2\cos^2 \varphi+a^2\sin^2 \varphi)}}$,
 i.e. integrand is invariant under changing $t\mapsto \varphi$, $a\to b$, $b\to a$. But this is "zamechanije v bok" }.


First note that if we put $t\mapsto \sin t$ then we see that
         $$
     \int_0^{\pi\over 2}{ab\over \sqrt {(a^2\cos^2 t+b^2\sin^2 t)}}dt=
     {1\over a}\int_0^1{dt\over \sqrt{1-t^2}\sqrt {1-{a^2-b^2\over a^2}t^2}}\,.
     \eqno (2)
         $$
Here we see so called elliptic integral of the first order:
          $$
      K(k)=\int_0^1{dt\over \sqrt{1-t^2}\sqrt {1-k^2t^2}}
      \eqno (3)
           $$
and  we have that
        $$
        S(a,b)={1\over a}K\left({\sqrt{a^2-b^2}\over a}\right)={1\over b}K\left( {\sqrt {b^2-b^2}\over b^2}\right)
        \eqno (4)
        $$
Now we study integral (2). We consider anzats such that it changes only parameter $k$ in integral  (3).

$$ $$
  \bigskip

\def\k{{\kappa}}
  Consider anzats:
               $$
               u={(1+\k)t\over 1+\kappa t^2}
               \eqno (2.1)
               $$

Note that $u(0)=0$, $u(1)=1$ and if $0\leq k\leq 1$ and $0\leq \k \leq 1$ then $u(t)$ is regular reparameterisation
of the interval $[0,1]$.
  We see that
         $$
      du={(\kappa+1)(1-\k t^2)\over (1+\k t^2)^2}dt
      \eqno (2.2)
         $$
and   $$
   1-u^2=1-{(1+\k)^2t^2\over (1+\kappa t^2)^2}={(1+\kappa t^2)^2-(1+\k)^2t^2\over (1+\kappa t^2)^2}=
       {(1-t^2)(1-\k^2 t^2)\over (1+\kappa t^2)^2}
       \eqno (2.3)
      $$
 Kontsevitch and Zagier conjectured that  two periods (integrals) are equal if
it can be checked by standard operations.
This makes us to believe that anzats is not bad...

Now put $\k=k$.

Then for  integrand in  the (3) using (2.1), (2.2) and (2.3) we have
         $$
         {dt\over \sqrt{1-t^2}\sqrt {1-k^2t^2}}=
         \left({dt\over du}du\right)
         \left({1\over (1+k^2t^2)\sqrt{1-u^2}}\right)=
         $$
         $$
         \left({(1+\k t^2)^2\over (\kappa+1)(1-\k t^2)}du\right)
         \left({1\over (1+k^2t^2)\sqrt{1-u^2}}\right)=
      {du\over u\sqrt {1-u^2}}{t\over 1-kt^2}
         $$
Comparing with (2.1) one can see that
       $$
{t\over 1-kt^2}={u\over \sqrt {(1+k)^2-4ku^2}}
       $$
Hence we have finally that
  $$
  {dt\over \sqrt{1-t^2}\sqrt {1-k^2t^2}}={du\over \sqrt {1-u^2}}{1\over \sqrt {(1+k)^2-4ku^2}}=
  {du\over (1+k)\sqrt{1-u^2}\sqrt {1-{k'}^2u^2}}
$$
where
         $$
       k'={2\sqrt k\over 1+k}
         $$
Returning to elliptic function we see that


      {\bf Proposition} Under transformation $t\to {(k+1)t\over (1+kt^2)}$ the integrand (3)
      remains invariant (up to a multiplier) and
                  $$
              K(k)={1\over 1+k}K\left({2\sqrt k\over 1+k}\right)
                  $$
In particular returning to the integral $S(a,b)$ we see that
             $$
           S(a,b)={1\over a}K\left({\sqrt{a^2-b^2}\over a}\right)
             $$
and
           $$
          S\left({a+b\over 2}, \sqrt {ab}\right)=
          {2\over a+b}K\left({\sqrt {\left({a+b\over 2}\right)^2-ab}\over{\left({a+b\over 2}\right)}}\right)=
      {2\over a+b}  K\left({a-b\over a+b}\right)=
           $$
  (due to Proposition)
           $$
   {2\over a+b}\cdot{1\over 1+{a-b\over a+b}}
   K\left(2{\sqrt{{a-b\over a+b}\over }\over 1+{a-b\over a+b}}\right)=
   {1\over a}K\left({\sqrt{a^2-b^2}\over a}\right)=S(a,b)
           $$
We come to

{\bf Lemma}{{\bf Gauss}} Let numbers $a_1,b_1$ are respectively arithemtic and geometric means of numbers $a,b$.
Then
             $$
    S(a,b)=    S(a,b)=\int_0^{\pi\over 2}{dt\over \sqrt {(a^2\cos^2 t+b^2\sin^2 t)}}=S(a_1,b_1)
             $$
It is not difficult to prove that sequences $\{a_n\}$, $\{b_n\}$  has the same limit. It is called
Gauss arithmetic-geometric mean $M(a,b)$
      $$
   S(a,b)=S(M,M)=\int_0^{\pi\over 2}{dt\over \sqrt {(M^2\cos^2 t+M^2\sin^2 t)}}={\pi\over 2M}
      $$
Hence we come to the answer:
                   $$
           M(a,b)={2\over\pi}\int_0^{\pi\over 2}{dt\over \sqrt {(a^2\cos^2 t+b^2\sin^2 t)}}={2S(a,b)\over \pi}
                   $$

$$ $$

  {\it Potential of the circle.}

It is easy to see that the integral $S(a,b)$ describes the potential of the circle.

Consider circle of the radius $R$ with charge $Q$. Then potential on the distance $\rho$ from the centre equals
 to
           $$
 U(\rho)=   {Q\over 2\pi }
      \int_0^{2\pi}{dt\over \sqrt {\rho^2+R^2-2R\rho \cos t}}
           $$
 One can see that
         $$
         \rho^2+R^2-2R\rho \cos t=\rho^2+R^2-2R\rho \left(2\cos^2 {t\over 2}-1\right)=
         (\rho+R)^2\sin^2 t+(\rho-R)^2\cos^2 t
         $$
and
         $$
      U(\rho)=   {Q\over 2\pi }
      \int_0^{2\pi}{dt\over \sqrt {\rho^2+R^2-2R\rho \cos t}}=
                 $$
           $$
            {Q\over \pi }\int_0^{\pi}{dt\over \sqrt {((\rho+R)^2\cos^2 t+(\rho-R)^2\sin^2 t)}}=
       {2Q\over \pi }S(\rho+R,\rho-R)\,.
         $$
       and finally we have:
         $$
       U(\rho)={2Q\over \pi }S(\rho+R,\rho-R)={2Q\over \pi}{2\pi M(\rho+r,\rho-R)}={Q\over M(R+\rho,R-\rho)}
         $$
{\bf Theorem}({\bf Gauss}) Let the charge $Q$ is homogeneously distributed on the circle of the radius $R$.
Then the potential on the distance $\rho$ is equal to
           $$
         {Q\over M(\rho+R,\rho-R)}
           $$


  \bye
