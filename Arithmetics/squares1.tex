From khudian@manchester.ac.uk Mon Dec 29 14:36:50 2014
Date: Mon, 29 Dec 2014 14:36:50 +0000 (GMT)
From: khudian@manchester.ac.uk
To: Yuri Bazlov <yuri.bazlov@manchester.ac.uk>
Subject: Задачка для детишек и ещё durackij wopros po kompjuteru...


            Здравствуйте Юра!
   Поздравляю Вас и всю вашу большую семью с Новым Годом!
        Успехов и здоровья!
****************************************************************
   Вот задачка, которую я `придумал' (конечно, наверное она известна)
   для своего внука:
     Дети, много много детей стоят в ряд---1-ый, 2-ой, 3-ий, и.т.д.
  На всех надеты свитера надеты правильно, а не шиворот-навыворот
   По команде: ``Раз!''--все дети снимают и надевают снова свитеры,
  но проблема в том, что каждый раз когда дети снимают
   и надевают снова свитеры, они надевают их шиворот-навыворот.
   По команде: ``Два!''--каждый второй ребёнок снимает
                         и надевает снова свой свитер,
   По команде: ``Три!''--каждый третий  ребёнок снимает
                         и надевает снова свой свитер,
   Конечно же, каждый раз "шиворот-навыворот" повторяется:
           например, на счёт  `Два' все дети стоящие на чётных
   местах в итоге наденут свитеры правильно, потому что они два раза
снимали и надевали их шиворот-навывоворот, (на счёт "Раз" и на счёт "Два"),
  на счёт `Три' , дети, стоящие на 3-ьем, 9-ом и 15-ом местах
   тоже в итоге наденут свитеры правильно, потому что они два раза
снимали и надевали их шиворот-навывоворот (на счёт "Раз" и на счёт "Три"),
    а вот дети стояшие на 6-ом, 12-ом, 18-ом местах
    местах наденут свитеры неправильно, потому что они три раза
    снимали и надевали их шиворот-навывоворот (на счёт "Раз", на счёт "Два"
      и  на счёт "Три"),

    Всё это продолжается.....

   По команде: ``Четыре!''--каждый четвёртый  ребёнок снимает
                         и надевает снова свой свитер,
   По команде: ``Пять!''--каждый пятый  ребёнок снимает
                         и надевает снова свой свитер,
   По команде: ``Шесть!''--каждый шестой  ребёнок снимает
                         и надевает снова свой свитер,

   и.т.д., и.т.д..........

   Конечно же после команды, например, "Шесть" больше первых
           шесть детей никто уже не беспокоит,
после команды, например, "Десять" больше первых
десять  детей тоже никто уже  не беспокоит,


    Вопрос:  В конце концов на каких детях
    свитеры будут надеты правильно и на каких
     шиворот-навыворот?

%%%%%%%%%%%%%%%%%%%%%%%%%%%%%%%%%%%%%%%%%%%%%%%%%%%%%%%%%%%%%%%%%
%%%%%%%%%%%%%%%%%%%%%%%%%%%%%%%%%%%%%%%%%%%%%%%%%%%%%%%%%%%%%%%%%
%%%%%%%%%%%%%%%%%%%%%%%%%%%%%%%%%%%%%%%%%%%%%%%%%%%%%%%%%%%%%%%%%


  Ответ: на детях с номерами 1,4,9,16,25,...$n^2$,...
   свитеры будут надеты  шивпрпт-навыворот, а на остальных
    детях правильно.....

*****************************************************************



   U menia k Vam vopros, no absloljutno ne  srochnyj!
   Ja bezvulazno sizhu doma, dazhe zanimajusj.
Moj drug posylajet mne interesnyje statju, knigi==
pdf faily, no oni nemnogo boljshije (neskoljko megabait)
Ja volnujusj za resursy pamiati ne stoljko mojego kompjutera
skoljko mojego accounta na sekhmete ()kuda prikhodit moaj pochta)
i na kompjutere v Institute>
   Konechno ja posmotrel v google kommandy, no situatsija
   ne ochenj jasnaja: kakaya po Vashemu prostaja komanda,
kotoraja opredeljajet estj li mesto i skoljko?




                                     Dr Hovhannes Khudaverdian
                                    The University of Manchester
                                       School of Mathematics
                                   e-mail: khudian@manchester.ac.uk
                                        tel. 0161-2008975

