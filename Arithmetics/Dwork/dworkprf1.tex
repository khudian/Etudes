\def\A {\bf A}
\def\t {\tilde}
\def\a {\alpha}
\def\K {{\bf K}}
\def\N {{\bf N}}
\def\V {{\cal V}}
\def\s {{\sigma}}
\def\S {{\Sigma}}
\def\s {{\sigma}}
\def\vare{{\varepsilon}}
\def\Q {{\bf Q}}
\def\F {{\bf F}}
\def\D {{\cal D}}
\def\G {{\Gamma}}
\def\C {{\bf C}}
\def\M {{\cal M}}
\def\Z {{\bf Z}}
\def\U  {{\cal U}}
\def\H {{\cal H}}
\def\R  {{\bf R}}
\def\l {\lambda}
% This file I begin on 17 September 2007. I would like to understand Dwork Theorem
\def\degree {{\bf {\rm degree}\,\,}}
\def \finish {${\,\,\vrule height1mm depth2mm width 8pt}$}
\def \m {\medskip}
\def\p {\partial}
\def \Tr {{\rm Tr}}
\def\ord {{\rm ord\,}}
\documentclass[12pt]{article}
\usepackage{amsmath,amsthm}

\theoremstyle{theorem}
%\newtheorem{thm}{Khimera}


\numberwithin{equation}{section}



\begin{document}


  \centerline {\bf Dwork theorem}

  \centerline {attempts to understand its proof based on works of Dwork, paper of Candelas and my ideas }



\tableofcontents

\section*{Preliminaries}

Let $P(x_1,\dots,x_n)$ be a polynomial with coefficients in prime
finite field $\F_p$. This polynomial defines variety $M_P$ in
$n$-dimensional affine space over the field $\F_p$ or over an algebraic
extensions $\F_{p^r}$ of this field. Denote by $\nu_r$ the number of distinct
solutions of equation $P(x_1,\dots,x_n)=0$ considered over the field $\F_{p^r}$
($r=1,2,3,\dots $), i.e. the number of points of the variety  $M_P$ considered
in affine space over the field $\F_{p^r}$ and consider the following generating
function
          \begin{equation}\label{zetafunctionasgenfunctionofsolut}
        Z_P(t)=\exp\left(\sum_{r=1}^\infty {\nu_rt^r\over r}\right)
          \end{equation}

The famous result of B. Dwork is that zeta function $Z_P(t)$ is rational function.

I try in this text to present a proof of Dwork theorem.
I wrote this text after studying original work of B. Dwork, the text of Candelas
and the book of Koblitz.




\medskip

{\bf Example 0.1} Let $P(x,y)=x-y$. Then evidently $\nu_r=p^r$ and
\begin{equation}\label{trivialxampleofzetafunction}
         Z_P(t)=\exp\left(\sum_{r=1}^\infty {p^rt^r\over r}\right)=
          \exp\left(\log{1\over 1-pt}\right)={1\over 1-pt}
          \end{equation}

{\bf Example 0.2} Let $P(x,y)=x^2+y^2-1$ defines circle.
   To calculate $\nu_r$ consider the  following map
            $$
         \tau[u:v]=[1-u^2: 2u: 1+u^2]
            $$
which establishes one-one correspondence between
points of  projective plan and the conic $x^2+y^2=z^2$ in two-dimensional projective space.

This conic has two points more than circle $x^2+y^2=1$ if and only if the equation
$x^2+y^2=0$ has not-trivial solution in $\F_{p^r}$, i.e. if $\sqrt {-1}\in \F_{p^r}$.
(Points $[\sqrt{-1}:1:0]$ and $[-\sqrt{-1}:1:0]$ in $\R P^3$)




The number of points on projective line (over $\F_{p^r}$) is equal to $p^r+1$.
Hence we see that  for the variety $x^2+y^2=1$
         \begin{equation}\label{numberofpointoncircle}
    \nu_r=
     \begin{cases}
        p^r-1\,\, \hbox {if $p=4k+1$}\\
         \begin{cases}
        p^r-1\,\, \hbox {if $r$ is even}\\
         p^r+1\,\, \hbox {if $r$ is odd}\\
     \end{cases}\hbox {if $p=4k+3$}
     \end{cases}
\end{equation}
  Hence  for the circle $x^2+y^2=1$
\begin{equation}\label{zetafunctionforcircleif$p=4k+1$}
Z(t)=\exp\left(\sum_{r=1}^\infty{(p^r-1)t^r\over r}\right)=\exp\left(\log{1\over 1-pt}-\log{1\over 1-t}\right)=
     {1-t\over 1-pt}
\end{equation}
if $p=4k+1$
and
              $$
Z(t)=\exp\left(\sum_{r\,\,{\rm even}}{(p^r-1)t^r\over r}+\sum_{r\,\,{\rm odd}}{(p^r+1)t^r\over r}\right)=
\exp\left(\sum_{r=1}^\infty{p^rt^r\over r}-\sum_{r=1}^\infty{(-1)^rt^r\over r}\right)=
       $$
\begin{equation}\label{zetafunctionforcircleif$p=4k+3$}
=\exp\left(\log{1\over 1-pt}-\log (1+t)\right)=
     {1+t\over 1-pt}
\end{equation}

\medskip

{\it Another more wise definition of zeta function}
\def\p {{\bf p}}
 Let $\p=(x_1,x_2,\dots,x_n)$ be an arbitrary solution of the equation $P(x)=0$ considered over algebraically closed
 field $\bar \F_p$. Apply Frobenius homomorphism $\sigma\colon \,\,a\mapsto a^p$ . The
  point $\sigma \p=(\sigma x_1,\dots,\sigma x_n)$ is a solution of thi equation too.
  Consider the orbit of this solution under Frobenius homomorphism:
                 \begin{equation}\label{orbitofsolution}
  \{\p, \sigma\p,\sigma^2\p,\dots,\sigma^{k-1}\p\}
\end{equation}
We assume that $\sigma^i \p\not=\p$ for $i\leq k-1$ and $\sigma^k\p=\p$, i.e. all coordinates $(x_1,\dots,x_n)$
of the point $\p$ belong to the field $\F_{p^k}$ and at least one coordinate does not belong to the smaller subfield.
We say that the point $\p$ has the orbit of the length $k$.


  Denote by $d_n$ number of different orbits on the variety $P(x)=0$ of the length $n$. It is evident now that
   \begin{equation}\label{relationbetweenorbitandnumberofpoints}
    \nu_r=\sum_{n\colon n|r} nd_n
\end{equation}
 E.g. $\nu_1=d_1$, $\nu_2=d_1+2d_2$, $d_4=d_1+2d_2+4d_4$

Now applying \eqref{relationbetweenorbitandnumberofpoints} to \eqref{zetafunctionasgenfunctionofsolut} we come to
                 $$
                Z_P(t)=\exp\left(\sum_{r=1}^\infty {\nu_rt^r\over r}\right)=
        \exp\left(\sum_{r=1}^\infty {\left(\sum_{n\colon n|r} nd_n\right)\over r}t^r\right)=
                 $$
\begin{equation}\label{zetafunctionmorenatural}
  \exp\left(\sum_{n,s=1}^\infty{nd_nt^{ns}\over ns}\right)=
  \exp\left(\sum_{n=1}^\infty d_n\left(\sum_{s=1}^\infty{t^{ns}\over s}\right)\right)=
  \prod_{n=1}^\infty{1\over (1-t^n)^{d_n}}
\end{equation}


This formula serves much more natural definition of Zeta function, it is much more motivated:
Compare with usual zeta-function.
This formula explains the special way of choosing of generatig function \eqref{zetafunctionasgenfunctionofsolut}.
Moreover this formula explains the following fact:


  {\bf Proposition 01} Let $Z_M(t)=\sum c_nz^n$ is an expansion of the function in vicinity of the point
  $0$. (This expansion exists because zeta-function of any surface can me majorated by zeta
  function in the case $\nu_r=p^rn$)).
   Then coefficients $c_n$ are integers.


\section {Reduced zeta-function}

  Techically much easier to calculate reduces zeta function:
  \begin{equation}\label{zetafunctionreduced}
        Z^*_P(t)=\exp\left(\sum_{r=1}^\infty {\nu^*_rt^r\over
        r}\right),
          \end{equation}
where $\nu_r^*$---is number of solutions of equation $P(x_1,x_2,\dots,x_n)=0$ over $\F_{p^r}$ such that all
coordinates are not equal to zero, $x_1\not=0,x_2\not=0,\dots,x_n\not=0$

Using inclusion exclusion principe it is easy to show that usual zeta function can be rationally expressed
via reduced zeta-function and vice versa.
The following example explains everything:

 let $P=P(x,y,z)$ defines variety in $\A^3$. Consider polynomials
              $$
        P_1(x,y)=P(0,x,y), P_2(x,y)=P(x,0,y), P_3(x,y)=P(x,y,0),
              $$
and
             $$
          P_{12}(x)=P(0,0,x),P_{13}(x)=P(0,x,0),P_{23}(x)=P(x,0,0), \,\, {\rm and}\,\, P_{123}=P(0,0,0)
             $$
Then using ''exclusion-inclusion'' we see that
  W e see that
 \begin{equation}\label{expressionofusualzetaintermsofreduced}
    Z_P(t)=Z^*_P(t)Z^*_{P_1}Z^*_{P_2}Z^*_{P_3}Z^*_{P_{12}}Z^*_{P_{13}}Z^*_{P_{23}}Z^*_{P_{123}}
\end{equation}
and on the other hand using ''exclusion-inclusion'' we see that
 \begin{equation}\label{expressionofreducedintermsofusual}
    Z^*_P(t)={Z_P(t)\over Z_{P_1}Z_{P_2}Z_{P_3}}{Z_{P_{12}}Z_{P_{13}}Z_{P_{23}}\over Z_{P_{123}}}
\end{equation}

Hence it is enough to prove that reduced zeta-function is rational


\section{How to calculate number of zeros.}

  For any given $r$ one can write down the formula which counts the number of
   points  $\nu_r$ or $\nu_r^*$. E.g. if $r=1$ then
         \begin{equation}\label{numberofpoints}
    \nu_1={1\over p}\sum_{x_\mu\in \F_p}\exp\left[{2\pi \sqrt{-1}\over p}\left(x_0P(x_1,\dots x_n)\right)\right]
\end{equation}
  because
  \begin{equation}\label{sums}
 \sum_{x\in \F_p}\exp\left[{2\pi \sqrt{-1}\over p}\left(xy\right)\right]=
 \begin{cases}0 \,\,{\rm if}\,\,y\not=0\\ p \,\,{\rm if}\,\,y=0\end {cases}
\end{equation}
In \eqref{numberofpoints} that $x_\mu=(x_0,x_1,\dots,x_n)$. We assume in \eqref{numberofpoints}
natural mapping of $\F_p=\{\bar 0, \bar 1,\dots, \bar {p-1}\}$ into integers $\{ 0,  1,\dots,  {p-1}\}$.

In the case if we calculate $\nu_r$ for $r\geq 2$ we have to be more tricky.
Instead \eqref{sums} we have now
\begin{equation}\label{sumsextended}
 \sum_{x\in \F_{p^r}}\exp\left[{2\pi \sqrt{-1}\over p}\Tr \left(xy\right)\right]=
 \begin{cases}0 \,\,{\rm if}\,\,y\not=0\\ p^r \,\,{\rm if}\,\,y=0\end {cases}
\end{equation}
Hence
         \begin{equation}\label{numberofpointsforbigger r}
    \nu_r={1\over p^r}\sum_{x_\mu\in \F_{p^r}}\exp\left[{2\pi \sqrt{-1}\over p}{\rm Tr}
         \left(x_0P(x_1,\dots x_n)\right)\right],
                \end{equation}
 where
                        $$
                 {\rm Tr}\,a=a+a^p+a^{p^2}+\dots+a^{p^{r-1}}
                        $$

is a trace of linear operator $L_a\colon \,L_a b=a\cdot b$ in the space $\F_{p^r}$ considered as linear space
over $\F_p$.  It is useful to note the following: if $a$ is a root of monic irreducible polynomial
  $p(x)=t^n+a_{n-1}t^{n-1}+\dots$ over $\F_p$   then $\Tr a=-a_{n-1}$.


 In the case if we would like to calculate $\nu_r^*$ then one can see that
 in previous formulae we have make summation not over all the field $\F_{p^r}$
 but over $\F_{p^r}^*=\F_{p^r}\backslash \{0\}$. One can see that
        $$
    \sum_{x_0\in \F_p^r,x_1,\dots,x_n\in \F^*_{p^r}}\exp\left[{2\pi \sqrt{-1}\over p}{\rm Tr}
         \left(x_0P(x_1,\dots x_n)\right)\right]=p^r\nu_r^*=
         $$
                      $$
\sum_{x_0,x_1,\dots,x_n\in \F^*_{p^r}}\exp\left[{2\pi \sqrt{-1}\over p}{\rm Tr}
         \left(x_0P(x_1,\dots x_n)\right)\right]+(p^r-1)^n\,.
                       $$
Hence
        \begin{equation}\label{numberofpointsforbiggerreduced}
    \nu^*_r={1\over p^r}\sum_{x_\mu\in \F^*_{p^r}}\exp\left[{2\pi \sqrt{-1}\over p}{\rm Tr}
         \left(x_0P(x_1,\dots x_n)\right)\right]+{(p^r-1)^n\over p^r}
                \end{equation}
On this step calculation of $\nu_r$ seems to be at least not more difficult than calculations of $\nu_r^*$.
On the other hand formulae above are not suitable for calculations ensemble for all the fields.

And here comes first of Dwork`s great ideas: {\it to write these formulae in one field}

\subsection {Representations of formulae \eqref{numberofpointoncircle} in $p$-adic numbers}
 For this formulae consider with B. Dwork the field $Q_p$ of $p$-adic numbers  and its algebraic extensions.

 Begin with simple case, the field $\F_p$.

  $Z_p/pZ_p=\F_p$ where $Z_p$ is  ring of $p$-adic integers. It is evident.
  What is important that there is a map (Teichmuller map) $T\colon \F_p\to Z_p$:
  $T(a)$ is a $p$-adic integer such that
  \begin{equation}\label{Teich1}
    T(a)^p=T(a)
\end{equation}
and $T(a)$ maps to $a$ under canonical projection $Z_p\to Z_p/pZ_p=\F_p$, i.e. the first digit of $T(a)$ in standard
 $p$-adic expansion $a$. It is easy to see that
              $$
           T(a)=\lim_{n\to \infty}a^{p^n}
              $$
   (we identifies elements $\{\bar 0,\bar 1,\dots,\bar{p-1}\}$) with integers $\{0,1,2,3,\dots,p-1\}$).

We call integer Teichmuller integer if it obeys the condition \eqref{Teich1}.

Non-zero Teicmuller integers forms regular $p-1$-polygon of $p-1$ roots of unity
($\vare^{p-1}=1$) in $p$-adic integers.

  One can see that
  \begin{equation}\label{propertiesofteichmulmap1}
   T(ab)=T(a)T(b)\,\,{\rm and}\,\, T(a+b)=T(a)+T(b)+\dots
\end{equation}

Consider Teichmuller image,polynomial $\hat P(x)$ with coefficients in $\Z_p$ of polynomial $P(x_1,\dots,x_n)$:
 If
    $$
  P(x_1,\dots,x_n)=\sum c_{m_1m_2\dots m_n}x_1^{m_1}x_2^{m_2}\dots x_n^{m_n},\,\,c_{m_1m_2\dots m_n}\in \F_p
    $$
    then
\begin{equation}\label{teichmullerimageofpolynomial}
    \hat P(x_1,\dots,x_n)=\sum T(c_{m_1m_2\dots m_n})x_1^{m_1}x_2^{m_2}\dots x_n^{m_n}
\end{equation}

Hence due to \eqref{propertiesofteichmulmap1} the formula \eqref{numberofpoints} can be rewritten as
         \begin{equation}\label{numberofpointsinteichmul}
    \nu_1={1\over p}\sum_{x_\mu\colon x_\mu^p=x_\mu}\exp
    \left[{2\pi \sqrt{-1}\over p}\left(x_0\hat P(x_1,\dots x_n)\right)\right]
\end{equation}
The summation goes over all sets  ${x_0,x_1,\dots,x_n}$ such that all elements are Teichmuller integers.

The formula for $\nu_1^*$ according to \eqref{numberofpointsforbiggerreduced} looks
      \begin{equation}\label{numberofpointsinteichmul}
    \nu_1={1\over p}\sum_{x_\mu\colon x_\mu^{p-1}=1}\exp
    \left[{2\pi \sqrt{-1}\over p}\left(x_0\hat P(x_1,\dots x_n)\right)\right]+{p-1\over p}
\end{equation}
The summation goes over all sets  ${x_0,x_1,\dots,x_n}$ such that all elements are non-zero-Teichmuller integers,
i.e. belong to $p-1$-regular polygon in $Z_p$.

Teichmuller maps can be generalised on the whole algebraic closure of the field $\F_p$.

We need some technical stuff for algebraic extensions of $p$-adic fields.

 \section {Teichmuller embedding of $F_{p^r}$ in unramified
 extension}

 Let $K:Q_p$ be finite algebraic extension of $Q_p$. For any $\a$
  let $p_\a(t)=t^{k}+a_{k-1}t^{k-1} t^{k-1}+\dots+a_1 t+a_0$ be minimum polynomial of $\a$, i.e.
  $p_\a(t)\in Q_p[t]$, it is irreducible and $\a$ is its root. Then
  coefficient $a_k$ is equal to the product of $\a$ and its conjugates, it is determinant of
  linear operator $L_\a\colon L_\a x=\a\cdot x$. Then
    \begin{equation}\label{norm}
    |\a|_p=\root k \of {|\det \a|_p}=\root k \of {|a_k|_p}
\end{equation}
The norm of element $a_k$ is defined as usual, this is an element
belongs to $Q_p$.

We define an order of $\a$ $\ord \a$ such that
\begin{equation}\label{defnitionoforder}
    {1\over p^{\ord \a}}=|\a|_p
\end{equation}

E.g. all elements of $\Q_p$ have integer order.

 Consider following two examples:

{\bf Example 1} Consider the extension of $\Q_p$ by polynomial $t^p-1$.
  Splitting field of this polynomial is extension of degree $p-1$. Indeed it is standard
  that
           $$
t^p-1=(t-1)(t^{p-1}+t^{p-1}+\dots+t+1)
           $$
  and polynomial $t^{p-1}+t^{p-1}+\dots+t+1$ is irreducible because
\begin{equation}\label{irreducibility}
t^{p-1}+t^{p-1}+\dots+t+1={t^p-1\over t-1}=(t-1)^{p-1}+p(t-1)^{p-2}+{p(p-1)\over2}(t-1)^{p-3}+\dots+\dots+p
\end{equation}
where $\tau=t-1$
We see that $p$ divides all coefficients of this polynomial (except highest) and $p^2$ does not divide the last coefficient,
i.e. it is irreducible polynomial; the number $\vare=\exp{2\pi\sqrt{-1}\over p}$ generates the field
extension of degree $p-1$. To see that this is ramified extension note that element
\begin{equation}\label{numberlambda}
    \lambda=\exp{2\pi\sqrt{-1}\over p}-1
\end{equation}
is a root of the polynomial  $t^{p-1}+pt^{p-2}+{p(p-1)\over2}t^{p-3}+\dots+\dots+p$. Its order
according to the formula \eqref{defnitionoforder} is equal to $p^{1\over p-1}$.

\medskip

{\bf Example 2} Consider the extension of $\Q_p$ by polynomial $t^{p-1}-p$
 This is irreducible polynomial
 (easy to check by Eisenstein Test).
Following Dwork (or Candelas???)
denote this number
 by $\pi$.

  One can see that the order of the number $\pi$ as well as the order of the number $\lambda$ is equal to
  $p^{1\over p-1}$. In the case if $p=3$ it is easy to see that one can take $\pi=\sqrt {-3}$, then
  $\gamma=\exp{2\pi\sqrt{-1}\over 3}-1=-{1\over 2}+{\pi\over 2}$ i.e. $\Q_3(\pi)=\Q_3(\gamma)$.

\medskip
I think that
{\bf Proposition}
\begin{equation}\label{proposition1}
\Q_p(\pi)=\Q_p(\pi)
\end{equation}

I tried to prove this but my calculations are very annuying, that is why sometimes it seems to me that they are wrong.
Commencing is very simple: Idea is to do the proof Hensel lemma-like:
Since $\pi^{p-1}=p$ any element in $\Q_p(\pi)$ can be presented as
               $$
               \a_n\pi^n,\,\, \hbox{where $\a_n\in \{o,1,2,3,\dots, p-1\}$}
               $$
We try to solve an equation $x^p=1$ using this expansion, but perturbation arguments are not very simple...


\medskip

Later we will see that numbers $\lambda$ and $\pi$ play very important role in constructing splitting function.

Now it is a time to return to unramified extensions.



We say that extension is unramified if all elements have an integer order.

Unramified extension produces by lifting of extension of $\F_p$ to $p$-adic numbers.


Consider the polynomial $t^{p^r}-t$. For field $\F_p$ all the roots of this polynomial form the
field $\F_{p^r}$. So it is not surprising that this polynomial has to be important now.
Consider the splitting field $\Sigma(t^{p^r}-t)$ of this polynomial over $\Q_p$.

If



\end{document}
