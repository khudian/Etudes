
\magnification=1200

\def\Z {{\bf Z}}
\def\A {{\bf A}}
\def \mod {{\rm mod\,}}
\def\a{{\alpha}}
\def\t{\tilde}
\def\F{\bf F}

\centerline {\bf $\t\Z_{10}=\Z_2\oplus \Z_5$}


\medskip

\def\m{\medskip}

   It is very natural to consider $10$-adic numbers in spite of the fact
   that $10$ is not prime.

 The simplest way to define the ring $\tilde\Z_{10}$ is the following:
            The set  $\t\Z_{10}$ is the set of formal series $\sum_{n=0}^\infty a_n 10^n$
             where $a_n$ are numbers $\{0,1,2,3,4,5,6,7,8,9\}$.
One can naturally introduce ring structure
mimicking addition and multiplication rules, so called
''slozhenije i umnozhenije stolbikom''. E.g.
If $x=\sum_{n=0}^\infty a_n10^n$, $y=\sum_{n=0}^\infty b_n10^n$ then
  $x+y=\sum_{n=0}^\infty c_n10^n$, where $c_n$ are defined by
              $$
  c_n=
                  \cases
                  {
  a_n+b_n+r_n \,\,\hbox {if $a_n+b_n+r_n<10$}\cr
a_n+b_n+r_n-10 \,\,\hbox {if $a_n+b_n+r_n \geq 10$}\cr
             }
             \,\,\,
  n=0,1,2,\dots
              $$
   and
                 $$
   r_0=0,\,\,             r_{k+1}=
  \cases
                  {
  0\,\,\hbox {if $a_n+b_n+r_k<10$}\cr
  1 \,\,\hbox {if $a_n+b_n+r_n \geq 10$}\cr
             }
             \,\,{\rm for}\,\, k\geq 0
                 $$
To be more precise consider presentation of $p$-adic numbers by infinite sequence:
$x=(x_0,x_1,x_2,\dots, x_n)$ where
                     $\,x_0=a_0, x_1=a_0+10a_1,\dots, x_k=\sum_{n=0}^{k} a_n 10^n$, $\dots$
if $x=\sum_{n=0}^\infty a_n 10^n$. The natural projection
$P_k\colon \sum_{n=0}^\infty a_n 10^n\to \sum_{n=0}^{k} a_n 10^n$
projects $x_{k+1}$ on $x_k$.
One can see that if $x=\sum_{n=0}^\infty a_n 10^n=(x_0,x_1,x_2,x_3,\dots)$,
$y=\sum_{n=0}^\infty b_n 10^n=(y_0,y_1,y_2,y_3,\dots)$ then
               $$
  x+y=z=\left(z_0,z_1,z_2,z_3,\dots\right)
               $$
where $z_k=P_k(x_k+y_k)$ and
            $$
  xy=w=\left(w_0,w_1,w_2,w_3,\dots\right)
               $$
where $w_k=P_k(x_ky_k)$
       This ring is not integral domain:
  (see examples below).

\m

It is well=known that If integer $N$ is product of different primes
then $\Z\slash N\Z$ is direct sum of fields. In particular
$\Z/10\Z=\F_2+\F_5$, (here as always $\F_p=\Z\slash pz$ prime field
of characteristic $p$ for prime $p$)--- The China`s algorithm
establish the isomorphism between $\F_{p_1}\oplus \F_{p_2}$ and
$\Z\slash p_1p_2\Z$ if $p_1\not=p_2$.

This can be be prolonged:



\smallskip

  {\bf Proposition}  {\it The ring $\t\Z_{10}$ is isomorphic to the direct sum of the rings $\Z_2$ and $\Z_5$.}

\medskip

  Present explicitly the maps $\phi\colon \Z_{10}\to \Z_2\oplus \Z_5$ and inverse map $\psi\colon \Z_2\oplus \Z_5\to \Z_{10}$
  which establish this isomorphism.
  If $x=\sum_{n=0}^\infty a_n 10^n\in \Z_{10}$ then
                   $$
    \psi(x)=\left(\sum_{n=0}^\infty \left(5^n a_n\right) 2^n,\sum_{n=0}^\infty \left(2^n a_n\right) 5^n\right)
    \in \Z_2\oplus \Z_{5}
                   $$
Note that this map sends rational integrals on the diagonal: Image of $\phi$ on $\Z$ is diagonal in $\Z\oplus \Z$.

  The inverse map is little bit not so obvious:

 Let $(x,y)\in \Z_2\oplus \Z_{5}$ where $x=(x_0,x_1,x_2,\dots,)$ with $x_k=\sum_{n=0}^k a_n 2^n$
 and $y=(y_0,y_1,y_2,\dots,)$ with $y_k=\sum_{n=0}^k b_n 5^n$. Show that there exists
  $z=(x_0,x_1,x_2,\dots,)$ with $z_k=\sum_{n=0}^k c_n 10^n$ which obeys the conditions:
                                    $$
              z_k=x_k (\mod 2^{k+1}),\,\,\,z_k=y_k (\mod 5^{k+1}), k=0,1,2,3,\dots
                                    $$
 and $z_k$ are uniquely defined by these conditions. It follows from this statement that map $\psi=\phi^{-1}$
 is defined by equation $\psi(x,y)=z$

For $k=0$ this is obvious. Suppose we proved it for $k\leq l$.
 For $k=l+1$ we have equations
                $$
       z_{l+1}=z_l+c_{l+1}10^{l+1}=x_{l+1}=x_l+a_{l+1}2^{l+1}(\mod 2^{l+2})
                 $$
and
                $$
             z_{l+1}=z_l+c_{l+1}10^{l+1}=y_{l+1}=y_l+b_{l+1}5^{l+1}(\mod 5^{l+2})
                $$
on unknowns $a_{l+1}, b_{l+1}\in \{0,1,2,3,4,5,6,7,8,9\}$.
These equations have solutions and these solution is unique because due to inductive hypothesis
for $k=l$ $z_l=x_l+\delta^{(2)}2^{l+1}$ and $z_l=x_l+\delta^{(5)}5^{l+1}$.

{\bf Remark} The maps $\phi,\psi$ are ''lifting'' of the maps establishing isomorphism between
ring $\Z\slash 10\Z$ and direct sum of the fields $\F_2$ and $\F_5$.

\medskip

 {\bf Example}. Consider an elements $(1,0)\in \Z_2\oplus \Z_5$.
   Find an element $z=\sum_{n=0}^\infty c_n10^n$ such that $\phi (z)=(1,0)$.
   If $z=(z_0,z_1,z_2,z_3,\dots)$ then one can see that
                   $$
         z_0=5, z_1=25, z_2=625,...
                   $$
we come to.... Yes! you are right we come to the sequence which know very well being innocent pupil in the school:
 sequence ${5,25,625,\dots}$---sequence of numbers such that finialdigits of their squares coicide with these numbers:
   $5^2=2{\bf 5}$,  $25^2=6{\bf 25}$,..  In the language of $10$-adic numbers $z=\psi(1.0)$ is $10$-adic number
   $(5,25,625,\dots)$ such that $x^2=x$. It is because
                   $$
       x^2=(1,0)^2=(1,0)=x
                   $$
Now we see that the second non-trivial solution of the equation $x^2=x$ in the ring $\Z_2\oplus \Z_{5}$ is
$w=(0,1)$. One can see that
                   $$
       \psi(0,1)=(6,76,376,\dots)
                   $$

\m

{\bf Remark} Note that it follows immediately from proposition that $\Z_{10}$ is not an integral domain;
e.g. $(a,0)(0,b)=(0,0)=0$.
 $(1,0)(0,1)=0$.


\bigskip

\centerline {\it PseudoTeichmullers for $\Z_{10}$\it }
\m
\def\F{\bf F}
 Recall standard facts about Teichmuller map.
If $p$ is prime then  the
ring $\Z_p$ possesses all roots of degree $p-1$ of unity, i.e. there exist a map
$T\colon \F_p\to \Z_p$ (Teihmuller map) such that
                $$
                T^p(\bar a)=T(a),
                    $$
One can see that $T(\bar a\bar b)=T(\bar a)T(\bar b)$ and $T(\bar +a\bar b)=T(\bar a)+T(\bar b)=p\Z_p$
Here $\F_p=\{\bar 0,\bar 1,\bar 2,\dots\}$ is a prime field of characteristic $p$.

 Roughly speaking for any $a\in {1,2,\dots,p-1}$ there is $p$-adic number, i.e.  a sequence $\{x_0,x_1,x_2\}$
 such that $x_{k+1}=x_k (mod p^k)$ and $x_k^p=x_k+\dots$.
 One can see that
                 $$
         T(\bar a)=\lim_{n\to\infty}a^{p^n}=(a,a^p,a^{p^2},\dots,a^{p^n},\dots)
                 $$
because $a^{p^{n+1}}=a^{p^{n+1}-p^n}a^{p^n}$ and $a^{p^{n+1}-p^n}=a^{\varphi(p^n)}=1+...p^{n+1}$


What happens in  $\tilde\Z_{10}$ ?  We already know that for $\bar a=5,6\in \Z\slash 10\Z$
$\t T(\bar 5)=(5,25,\dots)$ and $\t T(6)=(6,76,\dots)$
the order of these elements is equal to $2$.


Now look for all elements

  $\bar a=1$  $\t T(\bar 1)=1$. Order is equal to $2$ ($1^2=1$)

\m


  $\bar a=\bar 2$. We have that   $\bar 2=(\bar 0,\bar 2)$ in $\F_2\oplus F_{5}$.
     We see that $\t T(\bar 2)=(0,T_5(\bar 2)=\left(0,2^{5^\infty}\right)$.
        $2^{5^\infty}=(2,32,)$
  Its image in $\t\Z_{10}$ is equal to $2^{5^\infty}$

In the pedestrians language we come to the sequence:

  $(2,32,432,\dots)$ such that  $2^5=3\bf 2$, $32^5=...4\bf 32$, $432^5=...4\bf 432$

\m



  $\bar a=\bar 3$. We have that   $\bar 3=(\bar 1,\bar 3)$ in $\F_2\oplus F_{5}$.
     We see that $\t T(\bar 3)=(1,T_5(\bar 2)=\left(0,3^{5^\infty}\right)$. ($T_2(1)=1$).
        $3^{5^\infty}=(3,\dots 43, \dots 443,\dots)$, $3^5=...43$, $43^5=...443$


  Now we have to calculate the image of $(1,3^{5^\infty})$  in $\t\Z_{10}$. Is it equal to $3^{5^\infty}$? Yes



In the pedestrians languange we come to the sequence:

  $(3,43,443,\dots)$ such that  $3^5=4\bf 3$, $43^5=...4\bf 43$, $443^5=...\bf 443$


\m

$\bar a=\bar 4$. We have that   $\bar 4=(\bar 0,\bar 4)$ in $\F_2\oplus F_{5}$.
Note that order of $\bar 4$ in $\F_5$ is equal to $2$.
     We see that $\t T(\bar 4)=(0,T_5(\bar 4)=\left(0,4^{5^\infty}\right)$.
        $4^{5^\infty}=(4,\dots 24, \dots624 ,\dots)$, $4^5=...2\bf 4$, $24^5=...6\bf 24$, $624^5=\dots \bf 624$...
In fact cubes not only fifth orders have the same end:
  $4^3=...\bf 4$, $24^5=...\bf 24$, $624^5=\dots \bf 624$...

The image of $(0,4^{5^\infty})$  in $\t\Z_{10}$. Is it equal to $4^{5^\infty}$? Yes

In the pedestrians languange we come to the sequence:

  $(4,24,624,\dots)$ such that
   $4^3=...4$, $24^3=...24$,... and

  $4^5=4\bf 24$, $24^5=...6\bf 243$,


  \m


    $\bar a=\bar 5$. We have that   $\bar 5=(\bar 1,\bar 0)$ in $\F_2\oplus F_{5}$
    We know already  that $\t T(5)=(5,25,625,\dots)$.
   $10$-adic number $x=5^{2^\infty}=(5,25,625,\dots)$
    obeys the equation $x^2=x$

\m
  $\bar a=\bar 6$. We have that   $\bar 6=(\bar 0,\bar 1)$ in $\F_2\oplus F_{5}$
We know already  that $\t T(6)=(6,76,376,\dots)$.
   $10$-adic number $y=6^{5^\infty}=(6,76,376,\dots)$
    obeys the equation $x^2=x$

\m

$\bar a=\bar 7$. We have that   $\bar 6=(\bar 1,\bar 2)$ in $\F_2\oplus F_{5}$
    $10$-adic number $x=7^{5^\infty}$ obeys the equation $x^5=x$.


\m

$\bar a=\bar 8$. We have that   $\bar 6=(\bar 0,\bar 3)$ in $\F_2\oplus F_{5}$
    $10$-adic number $x=8^{5^\infty}$ obeys the equation $x^5=x$.


\m

$\bar a=\bar 9$. We have that   $\bar 9=(\bar 1,\bar 4)$ in $\F_2\oplus F_{5}$
    $10$-adic number $x=9^{5^\infty}$ obeys the equation $x^3=x$ (and $x^5=x$).





\m


  \bye
