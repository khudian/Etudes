
\magnification=1200 \baselineskip 17 pt
\def\Q{{\bf Q}}
\def\Z{{\bf Z}}
   \centerline {\bf  Integers and "Integers"}

\medskip
 \copyright Homepage of Math Department Sheffield University
 \smallskip


   Consider field $\Q({\sqrt {-n}})$
   (where $n$ is not divisible on any square except $1$)
   and consider the ring $K_n$
   of integers in this field, where the set
   $K_n$ is defined as set of elements of $\Q({\sqrt {-n}})$
   such that these elements are roots of a quadratic polynomial
   with integer coefficients and the numbers $2p$ and $p^2+nq^2$ are
   integers:
               $$
      K_n=\{a=p+q{\sqrt {-n}}\colon\,\,p,q\in\Q,\,\, 2p,p^2+nq^2\in\Z\},.
                $$

  One can easily to see that
         $K_n$ is the ring such that
      $$
      K_n=\{a=p+qp_n\}
      $$

      where $p$ and $q$ are integers  and
                      $$
                         p_n=
                      \cases
                        {
                       \sqrt {-n}\qquad {\rm if\,\, n=1,2 (mod 4)}\cr
                       {1+\sqrt {-n}\over 2}\qquad
                       {\rm if\,\, n=3(mod 4)}\cr
                       }\,.
                            $$

  {\bf Theorem 1.} The ring $K_n$ has unique factorization only
  for

   $n=1,2,3,7,11,19,43,67,163$.

{\bf Theorem 2.} If $K_n$ has unique factorization and $n=3(mod 4)$
 then the value of modular function $j$ on $p_n$ is integer!!!
  $j(p_n)\in\Z$
 Modular function $j(z)$ is the function of zero weight?
              $$
              j(z)={1\over q}+744+196884q=21493760q^2+864299970q^3+\dots,
                $$
       where $q=\exp{2\pi\sqrt{-1}z}$.


  Now use these Theorems and expansion of modular function for $n=43,67,167$:

                $$
 \Z\ni j(p_n)=\exp({-2\pi\sqrt{-1}p_n})+744+\dots=
 \exp\left({-2\pi\sqrt{-1}\left({1+\sqrt {-n}\over 2}\right)}\right)+744+\dots=
             $$
                $$
-\exp(\pi\sqrt n)+744+\dots
                $$
where we denote by dots something very small.


We see that $\exp(\pi\sqrt n)$ is very close to integer, e.g.
            $$
            \exp(\pi\sqrt 167)=262537412640768743.9999999999992501...\approx
262537412640768744
           $$
 up $10^{-12}$!!!



                            \bye
