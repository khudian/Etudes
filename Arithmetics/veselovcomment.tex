






\magnification=1200

\def\p {\partial}
\def\a {\alpha}


 \centerline {\bf A number such that its square 
 finishes on it and Veselov's comment on it}


 Find  a number $x$ such that its square , a number $x^2$
finishes with it. More precisely this means the following:
We say that a number $x$ {\it  has $m$-digits and it is finished with it}
 if
                  $$
            10^m\leq x <10^{m+1} \,, and\quad
            x^2-x \,\hbox {divides $10^m$}\,.      
            \eqno(1)
                   $$
     {\bf Remark}           
                   E.g. a number $x=
                   0625$ has 4 digits and it is finished with it:                          $$                     
          625^2-625=390000\,,  390000         
            \,\hbox {divides $10.000$}\,.      
                   $$
  \medskip


                    

This is very old problem for me \footnote{$^1)$}{It is one of the first
problems which I solved when I was a kid }
About five years ago Sasha Veselov made very beatuiful comment on this.
It follows from the {\bf Theorem } below that $10$ is not a prime number!

Indeed, suppose that $10$ is prime. Then $10$-adic is a field
and in the field {\sl  there is no} solutions of equation
                            $$
                            x^2=x
                            $$
This contradicts the statement of the theorem.

One can say that Veselov's statement is very sophisticated way to prove
that a number $10$ is not a prime.


\medskip

{\bf Theorem}  {\it There are two exactly two sequences
               $$
              a_1,a_2,\dots, a_n,\dots=5,25,625,0625, 
              90625,890625,2890625,\dots
               $$
and               
               $$
              b_1,b_2,\dots, b_n,\dots=6,76,376,\dots
               $$
such that


1) all the numbers in these sequences obey 
the condition, that their squares finishes  by them.

2) Any number 
$a_n$ in the first sequence possesses  
not more than $n$ digits:
 $a_n<10^n$, respectively any number   
$b_n$ in the second sequence possesses  
not more than $n$ digits:
 $a_n<10^n$.   

3) Any number such that its square finish by it belongs to the first or 
to the second sequence}

\medskip

 This follows from the following 
inductive statement

 {\bf  Lemma}
 
 Suppose by induction that a number $a_n$ contains $n$ digits
 (zeroes are permitted (see the Remark)), {\it and it is finished by it.}
 Then there is   exists a number $a_{n+1}$ which contains $n+1$ digits
 (zeroes are permitted (see the Remark)) {\it and it is finished by it.}
                   $$
           a_n<10^n,  a_n^2-a_n=0(mod 10^n) \Rightarrow
           \hbox {there exists a number $a_{n+1}$ such that}\,  
                        $$
                        $$
            a_{n+1}\colon a_{n+1}<10^n, 
           a_{n+1}^2-a_{n+1}=0(mod 10^{n+1})\,,
           \eqno (4)
                    $$
   where    $a_{n+1}=10^nx_n+a_n$ ($x=0,1,\dots,9$)
          (if $x =0$ then 
                        a number $a_{n+1}$ may ''slip''   a digit.
                        (see the remark after equation (1).))                        First note that  
                       $$
                       a_n^2-a_n=10^{n}s_n\,,
                        $$
 and                       
                       $$
                       b_m^2-b_m=10^{m}t_m\,,
                        $$                  
             with $s_n$ and $t_m$ which are integers. Hence           
                          $$
                    a^2_{n+1}-a_{n+1}=(10^nx_n+a_n)^2-(10^n x_n+a_n)=
     10^n\left(10^nx^2+2x_na_n+s_n-x\right)\,.
                          $$
         and respectively                 
                          $$
                    b^2_{m+1}-b_{m+1}=(10^mx+b_m)^2-(10^m x+b_m)=
     10^m\left(10^mx^2+2x_mb_m+t_m-x\right)\,.
                           $$
    We see that                       
expressions 
     $10^n\left(10^nx^2+2xa_n+s_n-x\right)$
and
     $\left(10^m(10^mx^2+2x_mb_m+t_m-x\right)$
has to be divisible on $10^{n+1}$. 
Hence one has to choose $x_n$ such that
                              $$
x_n(2a_n-1)+s_n=0(mod 10)\,,
                              \eqno (5)
                               $$
 where
                $$
          a_{n+1}=10^nx_n+a_n
          \eqno (6)
                $$
                or respectively
                               $$
y_m(2b_m-1)+t_m =0(mod 10)\,,
                              \eqno (5')
                               $$
 where
                $$
          b_{n+1}=10^nx_n+b_n
          \eqno (6')
                $$

    Solve equation (5) or equation (5')

$$
\matrix
{
a=1, &x+s=0, & x=9s(mod 10)\cr
a=2, &3x+s=0, & x=3s(mod 10)\cr
a=3, &5x+s=0, & \hbox {no solutions}\cr
a=4, &7x+s=0, & x=7s(mod 10)\cr
a=5, &9x+s=0, & x=s(mod 10)\cr
a=6, &x+s=0, & x=9s(mod 10)\cr
a=7, &3x+s=0, & x=3s(mod 10)\cr
a=8, &5x+s=0, & \hbox {no solutions}\cr
a=9, &7x+s=0, & x=7s(mod 10)\cr
}
\eqno (7)
$$
 We see that  system (5) {\it has no solutions} 
 if $a=3$  or if $a=8$.
 On the other hand in the equation (5)
 when we change $a_n$ on $a_{n+1}$ 
 then according equation (6)
 the multiplier $2a_n-1$ is changed on the
 $2a_{n+1}-1=2\left(10^nx+a_n\right)-1$.
 We see that the multiplier$(2a-1)$ remains the same modulo 10
 This means that the multiplier $(2a-1)$ never will be equal to $5$.

 Lemma is proved.

Consider examples.

\medskip


            {\bf 1}
 Take $N=1$ and $a_1=5$, ($10^{n_1}=10$),

 $a_1^2-a_1=20$, $s_1={a_1^2-a_1\over 10}=2$. Choose $x_2$ in (5).  We have
                                $$
                    (2a_1-1)x_2+2=0(mod 10)\,,   
                    9x_2+2=0 \Rightarrow x_2=2( mod 10)\,,
                      a_2=10\cdot 2+5=25\,.         
                                $$

    Respectively   take $M=1$ and $b_1=6$, ($10^{m_1}=10$),

 $b_1^2-b_1=30$, $t_1={b_1^2-b_1\over 10}=3$. Choose $y_2$ in (5').
 
 We have
                                $$
                    (2b_1-1) y_2+3=0 \Rightarrow y_2=7( mod 10)\,,
                      b_2=10\cdot 7+6=76\,.         
                                $$
                       
 

                                Thus we have two sequences
                               $$
                               \{5,25,\dots\};
                               \{6,76,\dots\}
                                $$
     $$ $$                           

            {\bf 2}
Take $N=2$ and $a_2=25$,($10^{n_2}=100$),

 $a_2^2-a_2=600$, $s_2={a_2^2-a_2\over 100}=6$. 
 Choose $x$ in (5).  We have
                                   $$
        (2a_2-1)x_2+s_2=0(mod10)\,, 49x_2+6 = (mod 10)\,,x_2=6 (mod10) 
                           \,,a_3= 100\cdot 6+25=625\,.         
                                   $$
Respectively take $M=2$ and $b_2=76$,($10^{n_2}=100$),

 $b_2^2-b_2=5700$, $t_2={b_2^2-b_2\over 100}=57$. Choose $x$ in (5).  We have
                                   $$
151x+57=0(mod10)\,,x+7=0  (mod 10)\,,x=3 (mod10) 
                           \,,b_3= 100\cdot 3+76=376\,.         
                                   $$


Thus we have two sequences



                               $$
                               \{5,25,625\dots\};
                               \{6,76,376\dots\}
                                $$

\medskip

{\bf 3}  Take $N=3$ and $a_3=625$,($10^{n_3}=1000$),

 $a_3^2-a_3=390.000$, $s_3={a_3^2-a_3\over 1000}=390$. 
 Choose $x$ in (5).  We have
                                   $$
 (2a_3-1)x_3+s_3=0(mod 10)\,, 1249x+390=0(mod10)\,,
 9x=0  (mod 10)\,,x=0 (mod10) \,,
                               $$
                          $$     
a_4=100\cdot 0+25=0625\,.         
                                   $$
Respectively take $M=3$ and $b_3=376$,($10^{n_3}=1000$),

 $b_3^2-b_3=141000$, $t_3={a_3^2-a_3\over 1000}=141$. 
 Choose $x$ in (5).  We have
                              $$
 (2b_3-1)y_3+t_3=0(mod10)\,, 751y_3+141=0(mod 10)\,,y_3+1=0(mod 10)\,,
 y_3=9\,,
                 $$
                 $$
                 b_4= 1000y_3+b_3=9376\,.
                                   $$
Thus we have two sequences



                               $$
                               \{5,25,625, 0625\dots\};
                               \{6,76,376, 9376\dots\}
                                $$

\medskip

 {\bf  4} 
 Take $N=4$ and $a_4=0625$, ($10^{n_4}=10.000$),

 $a_4^2-a_4=390.000$, $s_4={a_4^2-a_4\over 10.000}=39$. 
 Choose $x$ in (5).  We have
                                   $$
                (2a_4-1)x_4+39=0(mod10)\,,
                   1249x_4+39=0(mod 10)
                9x_4+9=0(mod10),x_4=9,\,,
                $$
                $$
a_5=10.000\cdot 9+0625=90625
           $$
 

Respectively  take $M=4$ and         $b_4=9376$, ($10^{n_4}=10.000$),

 $b_4^2-b_4=8.790.000$, $t_4={b_4^2-b_4\over 10.000}=8790$. 
 Choose $y_4$ in (5' ).  We have
                                   $$
                (2b_4-1)y_4+8790=0(mod10)\,,
                18751y_4+0=0 (mod10)\,, y_4=0
                $$
                $$
                b_5=09376
                $$
Thus we have two sequences



                               $$
                               \{5,25,625, 0625,90625\dots\};
                               \{6,76,376, 9376,09376\dots\}
                                $$


\medskip
  
        
{\bf 5}
  Take $N=5$ and $a_5=90625$, ($10^{n_5}=100 .000$),

 $a_5^2-a_5=8212800000$, $s_5={a_5^2-a_5\over 100.000}=82128$. Choose $x$ in (5).  We have
                                   $$
181249x+82128=0(mod10)\,,9x+8=0(mod10)\,,x=8.\,,
a_6=100.000\cdot 8+90625=890.625
                                           $$

 Respectively take $M=5$ and $b_5=09376$, ($10^{n_5}=100 .000$),

 $b_5^2-b_5=87900000$, $s_5={b_5^2-b_5\over 100.000}=879$. 
 Choose $y_5$ in (5).  We have
                                   $$
18751y_5+879=0(mod10)\,,y_5+9=0(mod10)\,,y_5=1.\,,
b_6=100.000\cdot 1+09376=109.376
                     $$
Thus we have two sequences



                               $$
                               \{5,25,625, 0625,90625,890625,\dots\};
                               \{6,76,376, 9376,09376\dots\}
                                $$


\medskip

{\bf 6}
Take $N=6$ and $a_5=890625$, ($10^{n_5}=100 .000$),

 $a_5^2-a_5=793212000000$, $s_6={a_6^2-a_6\over 100.000}=793212$. 
 Choose $x$ in (5).  We have
                                   $$
1781249x_6+793212=0(mod10)\,,9x_6+2=0(mod10)\,,x_6=2.\,,
                     $$
                      $$
a_7=100.000\cdot 2+890625=2.890.625
                         $$

 Respectively take $M=6$ and $b_6=109.376$, ($10^{m_6}=100 .000$),

 $b_6^2-b_6=11.96.300.000$, $s_5={b_5^2-b_5\over 100.000}=11.963$. 
 Choose $y_6$ in (5).  We have
                                   $$
218,751y_6+11.963=0(mod10)\,,y_6+3=0(mod10)\,,y_6=7.\,,
b_7=100.000\cdot 1+109376=7  . 109.376
$$
Thus we have two sequences



                               $$
                          \{5,25,625, 0625,90625,890625,2890625,\dots\};
                                     $$
                                     $$
                          \{6,76,376, 9376,09376,109376,7109376\dots\}
                                $$
    which obey the Theorem{$^{2)}$}{unfortunately this is
    the edge of capacities of my calculatrice}                     


 \bye
