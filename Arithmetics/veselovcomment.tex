






\magnification=1200

\def\p {\partial}
\def\a {\alpha}


 \centerline {\bf A number such that its square 
 finishes on it and Veselov's comment on it}


 Find  a number $x$ such that its square , a number $x^2$
finishes with it. More precisely this means the following:
We say that a number $x$ {\it  has $m$-digits and it is finished with it}
 if
                  $$
            10^m\leq x <10^{m+1} \,, and\quad
            x^2-x \hbox {divides $10^m$}\,.      
            \eqno(1)
                   $$
     {\bf Remark}           
                   E.g. a number $x=
                   0625$ has 4 digits and it is finished with it:                          $$                     
          625^2-625=390000\,.         
                   $$
  \medskip


                    

This is very old problem for me {It is one of the first
problems which I solved. } 

{\bf Theorem}  There are two exactly two sequences
               $$
              a_1,a_2,\dots, a_n,\dots=5,25,625,
               $$
and               
               $$
              b_1,a_2,\dots, a_n,\dots=6,76,376,
               $$
such that


1) all the numbers in these sequences obey 
the cohdition, that their squares finishes  by them.

2) Any number  $a_n$ in the first sequence possesses  
not more than $n$ digits:
 $a_n<10^n$.   

2) Any number such that its square finish by it belongs to the first or 
to the second sequence

 This follows from the following 
induction statement

 {\bf  Lemma}
 
 Suppose by induction that a number $a_n$ contains $n$ digits
 (zeroes are permitted (see the Remark)). and it is finished by it.
 Then there is   exists a number $a_{n+1}$ which contains $n+1$ digits
 (zeroes are permitted (see the Remark)) and it is finished by it.
                   $$
           a_n<10^n,  a_n^2-a_n=0(mod 10^n) \Rightarrow
           \hbox {there exists a number $a_{n+1}$ such that}\,  
                        $$
                        $$
           a_{n+1}\colon a_{n+1}<10^n, 
           a_{n+1}^2-a_{n+1}=0(mod 10^{n+1})\,,
           \eqno (4)
                    $$
   where    $a_{n+1}=10^n x+a_n$ ($x=0,1,\dots,9$)
          (if $x =0$ then 
                        a number $a_{n+1}$ may ''slip''   a digit.
                        (see the remark after equation (1).))                        First note that  
                       $$
                       a_n^2-a_n=10^{n}s_n
                        $$
             with $s_n$ integer. Hence           
                          $$
                    a^2_{n+1}-a_{n+1}=(10^nx+a_n)^2-(10^n x+a_n)=
     10^n\left(10^nx^2+2xa_n+s_n-x\right)\,.
                          $$
This expression has to be divisible on $10^{n+1}$. 
Hence one has to choose $x$ such that
                              $$
x(2a_n-1)+s_n=0(mod 10)
                              \eqno (5)
                               $$

Consider examples.



 Take $N=1$ and $a_1=5$, ($10^{n_1}=10$),

 $a_1^2-a_1=20$, $s_1={a_1^2-a_1\over 10}=2$. Choose $x$ in (5).  We have
                                $$
                      9x+2=0 \Rightarrow x=2( mod 10)\,,
                      a_2=10\cdot 2+5=25\,.         
                                $$
 Take $N=2$ and $a_2=25$,($10^{n_2}=100$),

 $a_2^2-a_2=600$, $s_2={a_2^2-a_2\over 100}=6$. Choose $x$ in (5).  We have
                                   $$
49x+6=0(mod10)\,, 9x+6=0  (mod 10)\,,x=6 (mod10) 
                           \,,a_3= 100\cdot 6+25=625\,.         
                                   $$
  Take $N=3$ and $a_3=625$,($10^{n_3}=1000$),

 $a_3^2-a_3=390.000$, $s_3={a_3^2-a_3\over 1000}=390$. Choose $x$ in (5).  We have
                                   $$
1249x+390=0(mod10)\,, 9x=0  (mod 10)\,,x=0 (mod10) \,,
a_4=100\cdot 0+25=0625\,.         
                                   $$
  Take $N=4$ and $a_4=0625$, ($10^{n_4}=10.000$),

 $a_4^2-a_4=390.000$, $s_4={a_4^2-a_4\over 10.000}=39$. Choose $x$ in (5).  We have
                                   $$
1249x+39=0(mod10)\,,9x+9=0(mod10),x=9.\,,
a_5=10.000\cdot 9+0625=90625
                                   $$
  Take $N=5$ and $a_5=90625$, ($10^{n_5}=100 .000$),

 $a_5^2-a_5=8212800000$, $s_5={a_5^2-a_5\over 100.000}=82128$. Choose $x$ in (5).  We have
                                   $$
181249x+82128=0(mod10)\,,9x+8=0(mod10)\,,x=8.\,,
a_6=100.000\cdot 8+90625=890.625
                                   $$
  Take $N=6$ and $a_6=890625$, ($10^{n_6}=1.000.000$),
.
 $a_6^2-a_6=793212.000.000$ 
 $s_5={a_5^2-a_5\over 100.000}=793212$. Choose $x$ in (5).  We have
                                   $$
1781249x+793212=0(mod10)\,,9x+2=0(mod10)\,,x=2\,,
a_7=1.000 .000\cdot 2+890625=2.890.m625
                                   $$
We come to the answer:

there is the infinite sequence  $\{5,25,625,0625,90625,890625,2890625,\dots \}$
such that
 $$
 5^2=25, 25^2=625, 625^2=390.625, 90625^2=8212
 $$
 \bye
