






\magnification=1200

\def\p {\partial}
\def\a {\alpha}


 \centerline {\bf A number such that its square 
 finishes on it and Veselov's comment on it}


 Find  a number $x$ such that its square , a number $x^2$
finishes with it. More precisely this means the following:
We say that a number $x$ {\it  has $m$-digits and it is finished with it}
 if
                  $$
            10^m\leq x <10^{m+1} \,, and\quad
            x^2-x \hbox {divides $10^m$}\,.      
            \eqno(1)
                   $$
     {\bf Remark}           
                   E.g. a number $x=
                   0625$ has 4 digits and it is finished with it:                          $$                     
          625^2-625=390000\,.         
                   $$
  \medskip


                    

This is very old problem for me {It is one of the first
problems which I solved. } 

{\bf Theorem}  There are two exactly two sequences
               $$
              a_1,a_2,\dots, a_n,\dots=5,25,625,
               $$
and               
               $$
              b_1,a_2,\dots, a_n,\dots=6,76,376,
               $$
such that


1) all the numbers in these sequences obey 
the cohdition, that their squares finishes  by them.

2) Any number  $a_n$ in the first sequence possesses  
not more than $n$ digits:
 $a_n<10^n$.   

2) Any number such that its square finish by it belongs to the first or 
to the second sequence

 This follows from the following 
induction statement

 {\bf  Lemma}
 
 Suppose by induction that a number $a_n$ contains $n$ digits
 (zeroes are permitted (see the Remark))
 and {\it it is finished by it}:

                  $$
            a_n <10^{n} \,, and\quad
            a_n^2-a_n \quad \hbox {divides $10^{n}$}\,.    
                   \eqno(3)
                   $$
    Then there exists a number $x=0,1,2,3,4,5,6,7,8,9$
 such that a number $a_{n+1}=10^n x+a_n$
 contains $n+1$ digits and {\it it is finished by it}:
                       $$
                a_{n+1}<10^{n+1}\,,  
            a^2_{n+1}-a_{n+1} \quad \hbox {divides $10^{n+1}$}\,.
            \eqno(4)
                       $$
          (if $x =0$ then 
                        a number $a_{n+1}$ may ''slip''   a digit.
                        (see the remark after equation (1).))                     

  Prove the lemma, i.e. prove that (3) implies (4). 
   First note that by inductive hypothesis (3) 
                       $$
                       a_n^2-a_n=10^{n}s_n
                        $$
             with $s_n$ integer. Hence           
                          $$
                    a^2_{n+1}-a_{n+1}=(10^nx+a_n)^2-(10^n x+a_n)=
     10^n\left(10^nx^2+2xa_n+s_n-x\right)\,.
                          $$
This expression has to be divisible on $10^{n+1}$. Hence one has to choose $x$ such that
                              $$
                              2xa_n+s_n-x \quad\hbox {has to be divisible on 10}
                              \eqno (5)
                               $$

Consider examples.



 Take $N=1$ and $a_1=5$.
 $a_1^2-a_1=20$, $s_1={a_1^2-a_1\over 10}=2$. Choose x in (5).  We have
                                $$
                      2xa_n+s_n-x=2\cdot x\cdot  5+2-x 
                      \quad\hbox{is divisible on 10}\Rightarrow x=2,
                      a_2=2,a_2=25\,.         
                                $$
 Take $N=2$ and $a_2=25$.
 $a_2^2-a_2=600$, $s_2={a_2^2-a_2\over 100}=6$. Choose x in (5).  We have
                                $$
                      2xa_n+s_n-x=2\cdot x\cdot  25+6-x 
                      \quad\hbox{is divisible on 10}\Rightarrow x=6,
                      a_2=2,a_2=25\,.         
                                $$


 
take
 $N=2$ and $a_2=25$. $a_2^2-a_2=600$, $x=s_2={a_2^2-a_2\over 100}=6$. $a_3=10^nx+a_2=625$
 
 
 take
 $N=3$ and $a_3=625$. $a_2^2-a_2=390.000$, $x=s_2={a_2^2-a_2\over 1000}=390$. $a_4=a_3=625$
 
 take
 $N=4$ and $a_4=0625$. , $x=s_4={a_2^2-a_2\over 10000}=39$. $a_5=90625$
 
 $N=5$ and $a_5=90625$. , $x=s_5={a_5^2-a_5\over 100000}=39$. $a_5=90625$
 
 

 \bye
