%n this version on 20 january 2007
% I write this on april 2012 on the base of the former text

\magnification=1200 \baselineskip=14pt
\def\vare {\varepsilon}
\def\A {{\bf A}}
\def\t {\tilde}
\def\a {\alpha}
\def\K {{\bf K}}
\def\N {{\bf N}}
\def\V {{\cal V}}
\def\s {{\sigma}}
\def\S {{\Sigma}}
\def\s {{\sigma}}
\def\p{\partial}
\def\vare{{\varepsilon}}
\def\Q {{\bf Q}}
\def\D {{\cal D}}
\def\G {{\Gamma}}
\def\C {{\bf C}}
\def\M {{\cal M}}
\def\Z {{\bf Z}}
\def\U  {{\cal U}}
\def\H {{\cal H}}
\def\R  {{\bf R}}
\def\E  {{\bf E}}
\def \sn {{\rm sn\,}}
\def\l {\lambda}
\def\degree {{\bf {\rm degree}\,\,}}
\def \finish {${\,\,\vrule height1mm depth2mm width 8pt}$}
\def \m {\medskip}
\def\p {\partial}
\def\r {{\bf r}}
\def\v {{\bf v}}
\def\n {{\bf n}}
\def\t {{\bf t}}
\def\b {{\bf b}}
\def\e{{\bf e}}
\def\ac {{\bf a}}
\def \X   {{\bf X}}
\def \Y   {{\bf Y}}
\def \x   {{\bf x}}
\def \y   {{\bf y}}

\centerline  {\bf The arithmetico-geometrical mean, }

\centerline {\bf the potential of circle and elliptic functions.}

\centerline {\it $\cal x $0 The arithmetico-geometrical mean. What is it.}


 Let $M(a,b)$ be the arithmetico-geometrical mean of two positive
 numbers $a,b$, i.e. for
                $$
   M(a,b)=\lim_{n\to \infty} a_n=\lim_{n\to \infty}b_n,
                $$
 where $a_0=a,b_0=b$ and  $a_{n+1}={a_n+b_n\over 2}$ (arithmetic mean) and
 $b_{n+1}=\sqrt {a_nb_n}$ (geometric mean.   Since geometric mean is less or equal to arithemtic mean
 this limit is well defined.


{\bf Examples}
     $$
    \matrix
       {
        &a       &b\cr
	&10	&1\cr
	&5,5	&3,16227766\cr
	&4,33113883	&4,170434885\cr
	&4,250786858	&4,250027349\cr
	&4,250407103	&4,250407086\cr
	&4,250407095	&4,250407095\cr
	&4,250407095	&4,250407095\cr
        &\dots &\dots
          }\quad
                 \matrix
           {
	&a	&b\cr
	&100	&10\cr
	&55	&31,6227766\cr
	&43,3113883	&41,70434885\cr
	&42,50786858	&42,50027349\cr
	&42,50407103	&42,50407086\cr
	&42,50407095	&42,50407095\cr
	&42,50407095	&42,50407095\cr
	&42,50407095	&42,50407095\cr
       &\dots &\dots\cr
            }
          \qquad
                 \matrix
         {
         &a      &b\cr
	&18	&8\cr
	&13	&12\cr
	&12,5	&12,489996\cr
	&12,494998	&12,494997\cr
	&12,4949975	&12,4949975\cr
	&12,4949975	&12,4949975\cr
           &\dots &\dots\cr
          }
          \eqno (0)
         $$
We see that after four-five iterations we come to an answer up to $10^{-8}$.

\centerline {\it $\cal x $1 Integral representation of the arithmetico-geometrical mean}

For the arithmetico-geometric mean there is a beautiful representation through the integral:
Consider the following integral
                      $$
                    S(a,b)=\int_0^{\pi\over 2}{1\over \sqrt {(a^2\cos^2 t+b^2\sin^2 t)}}dt\,.  
                       \eqno (1)
                       $$
 {\bf Lemma}(Gau\ss $\,$)  The integral  $S(a,b)$
does not change if $(a,b)\to (a_1,b_1)$ where $a_1$ is arithmetic mean of
$a,b$ and $b_1$ is geometric mean of $a,b$:
                        $$
                    S(a,b)=S\left({a+b\over 2}, \sqrt {ab}\right) \,,\quad (a,b\geq 0)
                       \eqno (1a)
                        $$
Thus this lemma implies that
             $$
      S(a,b)=\lim S(a_n,b_n)=S(M(a,b). M(a,b))={\pi\over 2 M(a,b)}\,.
             $$
 We come to the following formula
                  $$
   {\pi\over 2 M(a,b)}=S(a,b)=  \int_0^{\pi\over 2}{1\over \sqrt {(a^2\cos^2 t+b^2\sin^2 t)}}dt
                  $$
 or
                 $$
 M(a,b)={2\over \pi \int_0^{\pi\over 2}{1\over \sqrt {(a^2\cos^2 t+b^2\sin^2 t)}}dt }\,.
 \eqno (1c)
                 $$
    We see that the formula (1c) gives the nice and short way to calculate
numerically the integral (1) since calculation of $M(a,b)$  is very convergent  (see example (0) above).
     This is why Gau\ss $\,$ came to this notion.

 \m
 
 {\bf Example} Calculate   {\it potential of the circle.}\footnote{$^*$}
{Calculation of this integral was the reason why Gau\ss $\,$
introduced this notion (?)}.
Consider circle of the radius $R$ with charge $Q$ in the plane.
Then potential on the distance $\rho$ (in the plane) from the centre equals
 to
           $$
 U(\rho)=   {Q\over 2\pi }
      \int_0^{2\pi}{dt\over \sqrt {\rho^2+R^2-2R\rho \cos t}}
           $$
 One can see that
 \footnote{$^{**}$}{canonical way of calculating potential it is an expansion on Legendre polynomials.}
         $$
         \rho^2+R^2-2R\rho \cos t=\rho^2+R^2-2R\rho \left(2\cos^2 {t\over 2}-1\right)=
         (\rho+R)^2\sin^2 {t\over 2}+(\rho-R)^2\cos^2 {t\over 2}
         $$
and
         $$
      U(\rho)=   {Q\over 2\pi }
      \int_0^{2\pi}{dt\over \sqrt {\rho^2+R^2-2R\rho \cos t}}=
                 $$
           $$
            {Q\over \pi }\int_0^{\pi}{dt\over \sqrt {((\rho+R)^2\cos^2 t+(\rho-R)^2\sin^2 t)}}=
       {2Q\over \pi }S(\rho+R,\rho-R)={Q\over M(R+\rho,R-\rho)}\,.
         $$
We come to very beautiful

{\bf Fact}({\bf Gau\ss}) $\,$ Let the charge $Q$ is homogeneously distributed on the circle of the radius $R$ on the plane.
Then the potential at the point of the plane on the distance $\rho$ is equal to
           $$
       U(\rho)={Q\over M(\rho+R,\rho-R)}
           $$


\centerline {\it $\cal x $2 Proof of Gauss lemma}.



Now prove the Gau\ss $\,$ lemma (1a).




First note that under substitution $x=\sin t$ we come to
         $$
         S(a,b)=
     \int_0^{\pi\over 2}{ab\over \sqrt {(a^2\cos^2 t+b^2\sin^2 t)}}dt=
     {1\over a}\int_0^1{dx\over \sqrt{1-x^2}\sqrt {1-{a^2-b^2\over a^2}x^2}}\,.
     \eqno (2)
         $$
Here we see so called complete elliptic integral of the first kind:
          $$
      K(k)=\int_0^1{dx\over \sqrt{1-x^2}\sqrt {1-k^2x^2}}
      \eqno (2a)
           $$
and we have that
        $$
        S(a,b)={1\over a}K\left({\sqrt{a^2-b^2}\over a}\right)\,.
        \eqno (2b)
        $$

{\bf Remark} Sure we need only complete elliptic integral for calculating
(1) but to understand deeper the symmetries which imply these properties
we need also also the function, so called Jacobi first kind elliptic integral:
           $$
       u=\int_0^X{dt\over \sqrt{1-x^2}\sqrt {1-k^2x^2}}
         \,,\quad {\rm where}\,\,  X={\rm sn\,} (u;k)\,.
     \eqno (2c)
                $$
 Thus we come to elliptic function, ``elliptic sinus''

\m

   Rewrite the statement of lemma $S(a,b)=S({a+b\over 2},\sqrt {ab})$ in terms of function $K(k)$:
            $$
            {1\over a}K\left({\sqrt{a^2-b^2}\over a}\right)
 =S(a,b)=S\left({a+b\over 2}\right)={2\over a+b}K\left(
 \sqrt {\left({a+b\over 2}\right)^2-\left(\sqrt {ab}\right)^2}\over {a+b\over 2}\right)=
            {2\over a+b}K\left({a-b\over a+b}\right)
            $$
If we put $k={a-b\over a+b}$ then $b={1-k\over 1+k}a$ and ${\sqrt {a^2-b^2}\over a}={2\sqrt k\over 1+k}$.
Thus  the relation $S(a,b)=S\left({a+b\over 2},\sqrt {ab}\right)$ is equivalent to the relation
                     $$
  K(k)=K\left({a-b\over a+b}\right)= {a+b\over 2a}K\left({\sqrt{a^2-b^2}\over a}\right)=
  {1\over 1+k}K\left({2\sqrt k\over 1+k}\right)
                     $$
Now we will proof the relation
             $$
             K(k)={1\over 1+k}K\left({2\sqrt k\over 1+k}\right)\
             \eqno (3)
             $$
which is equivalent to the relation  $S(a,b)=S\left({a+b\over 2},\sqrt {ab}\right)$.

Left and right hand sides are integrals:  We have to prove that
          $$
          \int_0^1{dx\over \sqrt {(1-x^2)(1-k^2x^2)}}={1\over 1+k}
          \int_0^1{dy\over \sqrt {(1-y^2)(1-k'^2y^2)}}
          \eqno (3a)
          $$
We will prove it if we will find the anzats  $y=F(x)$ such that
              $$
                 {dx\over \sqrt {(1-x^2)(1-k^2x^2)}}={1\over 1+k}{dy\over \sqrt {(1-y^2)(1-k'^2y^2)}},
                 \quad\left(k'={2\sqrt k\over 1+k}\right)
        \eqno (3b)
              $$
and $F(0)=0, F(1)=1$\footnote{$^*$}
{ The proof that it is equal to elliptic integral is very good demonstration of so called
  Zagier-Kontsevitch periods
( Kontsevitch and Zagier conjectured that  two periods (integrals) are equal if
it can be checked by standard operations.).}
In this case we will immediately will come to the relation (3a).


{\bf Observation}   The following fraction
          $$
        y={(1+k)x\over 1+kx^2}
        \eqno (3c)
          $$
is the suitable anzats.
(In other words the ansatz (3a) is the integral of differential equation (3b) with initial condition $y(0)=0$).

{\sl Proof}. $\,$ Direct calculations:
We have that
$dy={(k+1)(1-k t^2)\over (1+k x^2)^2}dx$
and   $$
   1-y^2=1-{(1+k)^2x^2\over (1+k x^2)^2}={(1+k x^2)^2-(1+k)^2x^2\over (1+k x^2)^2}=
       {(1-x^2)(1-k^2 x^2)\over (1+k x^2)^2}
      $$
Then
         $$
         {dx\over \sqrt{1-x^2}\sqrt {1-k^2x^2}}=
         \left({dx\over dy}dy\right)
         \left({1\over (1+k^2x^2)\sqrt{1-y^2}}\right)=
         $$
         $$
         \left({(1+k x^2)^2\over (k+1)(1-k x^2)}dy\right)
         \left({1\over (1+k^2x^2)\sqrt{1-y^2}}\right)=
      {dy\over y\sqrt {1-y^2}}{x\over 1-kx^2}
         $$
We have that ${x\over 1-kx^2}={y\over \sqrt {(1+k)^2-4ky^2}}$
(To see it note that $z=\left({x\over 1-kx^2}\right)^2$
is rational function on $y$:
$z=\left({t\over 1-kx^2}\right)^2={x^2\over (1+kx^2)^2-4kx^2}=
{1\over (1+k)^2y-4ky})$).  This leads to the relation  (3b).

\m

Sure this is ``rabbit from the hat'' proof.
The mistery , the ``raison d`etre'' of the anzats (3a)
can be explained by the theory of elliptic functions
\footnote{$^\dagger$}{There is one beautiful anzats:
 $\varphi\colon \tan \varphi={b\over a}\tan t$. Then one can easy check that
 ${dt\over \sqrt {(a^2\cos^2 t+b^2\sin^2 t)}}={d\varphi\over \sqrt {(b^2\cos^2 \varphi+a^2\sin^2 \varphi)}}$,
 i.e. integrand is invariant under changing $t\mapsto \varphi$, $a\to b$, $b\to a$. But this is "zamechanije v bok"}.

  We try to explain the phenomenon on the following toy example:

 \centerline  {\it $\cal x$4  Toy example: trigonometric functions instead elliptic.}
  
  Consider the following toy -analogue of the Observation (3):

   {\bf Toy-observation} Consider differential equation:
                 $$
             {dx\over \sqrt{1-x^2}}=M{dy\over \sqrt{1-y^2}}
                 $$
                 where $M$ is a parameter.
           Try to find integrals for this equation.

     E.g. for $M=2$ $x=2y\sqrt{1-y^2}$. For $M=3$ $x=3y-4y^3$,...

     We cannot find non-trivial integral of motion 
     for $M=\sqrt 2$. Here the reason is obvious:
       The differential equation ${dx\over \sqrt{1-x^2}}=M{dy\over \sqrt{1-y^2}}$ means
         that
                $$
           \arcsin x=M\arccos y+C
                $$
       Function $\arcsin x$ is multi-valued function which is defined up to $\pm 2\pi k$
       and function $M\arccos y$ is defined up to  $\pm 2\pi Mk$

             We see that the differential equation
             ${dx\over \sqrt{1-x^2}}=M{dy\over \sqrt{1-y^2}}$ is integrable if $M$
             is rational.


             In the case of elliptif functions (3b)-like differential equation
                 $$
                 {dx\over \sqrt {(1-x^2)(1-k^2x^2)}}=...{dy\over \sqrt {(1-y^2)(1-k'^2y^2)}},
                 $$
                if parallelograms of periods for elliptic are related with rational transformations.

                In particular Gau\ss $\,$ transformation   corresponds to transformatioon
                $(\omega_1,\omega_2)\to (\omega_1,2\omega_2)$. The correspondi
    In the case of elliptic functions the Gau\ss $\,$ anzats leads  to the follwoing relation
    between elliptic functions:
                      $$
   u=\int_0^X{dx\over \sqrt {(1-x^2)(1-k^2x^2)}}=
   {1\over 1+k}\int_0^{Y(X)}{dy\over \sqrt {(1-y^2)(1-k'^2y^2)}}\,,
                      $$
                      with
                      $$
   Y(X)={(1+k)X\over 1+kX^2}, {\rm and}\,\,\, k'={2\sqrt k\over 1+k}
                       $$
 (If $X=1$ then $Y=1$ and we come to the relation (3)).

 Now we have that for elliptic function ${\rm sn\,}(x;k)$ (see 2(c)) the following identity:
                  $$
  {sn\left((k+1)u;k'\right)\over k+1}=
  {sn\left(u;k\right)\over 1+k\sn^2 (u;k)}
                  $$
 This is the relation between elliptic functions  with periods $(K,iK')$ and $(K,2iK')$.




        $\qquad$                      Hovhannes Khudaverdian   (2006---April 2012)

\bye 