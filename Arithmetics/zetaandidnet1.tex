
  \centerline {\bf From simple identity to zeta-function}

\bigskip

It is always nice to find a meaning of identities.


Consider the following identity:
        $$
     (1-t)^p(1-t^2)^{p^2-p\over 2}(1-t^3)^{p^3-p\over 3}(1-t^4)^{p^4-p^2\over 4}
     (1-t^5)^{p^5-p\over 5}(1-t^6)^{p^6-p^3-p^2+p\over 6}(1-t^7)^{p^7-p\over 7}....=1-pt,
        $$
where $p$ is prime number.
You guess now how looks the rule general which governs this infinite
product?  Yes, you are right. It is:


It is following:
          $$
       \prod_{n=1}^\infty (1-t^n)^{\sum_{k\colon\, k|n} \mu(k)p^{n\over k}},
          $$
where  $$
   \mu(n)=\cases {1 \,\,{\rm if}\, n=1 \cr (-1)^k \hbox {if $n=p_1p_2\dots p_k$
is a product of different primes $p_i\not=p_j$} \cr
=0 \hbox{}\cr}
      $$ is Mobius function.

Now let me explain where from comes this identity:

Denote by $\nu_k=\#{\bf F}_{p^r}$---cardinality (number of elements) of the field ${\bf F}_{p^r}$----extension
of degree $r$ of the field ${\bf F}_{p}$.
Consider Frobenius automorphism  $\sigma\colon \quad s\mapsto s^p$ of the field ${\bf F}_{p^r}$.

For any element $s$ of this field consider the sequence
              $$
\{s,\sigma s,\sigma^2 s,\sigma^3 s,\dots, \sigma^{k-1} s\}
              $$
such that $s\not=\sigma^i s$  for all $1\leq i\leq k-1$ but $\sigma^{k}s=s$. Thus we assign to any element
of the field non-negative integer $k(s)$---the length of the sequence.
E.g. $k(s)=0$ if $s\in {\bf F}_{p}$, $k(s)=2$ if $s\in {\bf F}_{p^2}$
and $s\not\in k(s)={\bf F}_{p}$.

We see that the set of all elements in $s\in {\bf F}_{p^r}$ can be considered as union of cycles:
               $$
             p^r=\sum_{k\colon\,\, k|r} kd_k
               $$
E.g. $p=d_1$, $p^2=d_1+2d_2$, $p^3=d_1+3d_3$, $p^4=d_1+2d)2+d_4$,\dots


\bye
