\magnification=1200


\def \t{\tilde}


  Denote by $S^k_n$ a dimension of the space of symmetric tensors of the rank $k$
  in $n$-dimensional space $V^n$
It is well-known that
                          $$
             S^k_n=C^k_{n+k-1}
                          $$
    This
   formula has many different proofs. I like the following.
 Let $\bf K$ be a symmetric tensor of rank $k$ in $n$-dimensional space $V$  and
    $K_{i_1i_2\dots i_k}$ be set of components of this tensor
    ($i_1\leq i_2\leq \dots\leq i_{k-1} \leq i_k$).

Consider {\it antisymmetric} tensor $\bf K'$ in $n+k-1$-dimensional
space $V'$ with components $K_{i_1i_2\dots i_k}$
    ($i'_1< i'_2< \dots<i'_{k-1}<i'_k\leq n+k-1$
such that
      $$
i'_1=i_1\,,\quad  i'_2=i_2+1, ,\quad i'_3=i_2+2,\dots,,i'_k=i_2+k-1,
\eqno (1)
      $$

Thus we establish isomorphism betweens space of symmetric tensors of rank $k$ with the space of antisymmetric tensors
 of rank $k$ in $n+k-1$ dimensional space which has dimension $C^k_{n+k-1}$...



 Now another formula:
       $$
      1+ S^1_n+S^2_n+\dots+S^k_n=C^k_{n+k}
      \eqno (3)
       $$
Again this formula has many proofs. One of the proofs is based on consideration of polynomials.
Another very elementary proof:

 Consider $n+k$-dimensional vector space $\tilde V$.

 Let $\bf \tilde K$ with components
 $K_{i_1i_2\dots i_k}$
    ($ i'_1< i'_2< \dots<i'_{k-1} < i'_k\leq n+k$ be antisymmetric tensor in $V$.

Now we assign to the tensor  $\tilde  K$ the following tensor:

  Note that $i'_1\leq n+1$, $i'_2\leq 2$, $i'_3\leq n+3,\dots, i'_k\leq n+k $ because
  $ i'_1< i'_2< \dots<i'_{k-1} < i'_k\leq n+k$.

  If $i'_1=n+1$, then we assign to  $\bf \tilde K$  the tensor with no indices.


   If $i'_1\leq n$ but $i'_2=n+2$, then we assign to  $\bf \tilde K$ the tensor of the rank $1$
             $$
            K_{i_1}=\t K_{i_1;\,n+2;\,n+3;\,\dots;\, n+k}
             $$

  If $i'_1\leq n$ and $i'_2\leq n+1$ $i'_3=n+3$, then
  we assign to  $\bf \tilde K$ the symmetric tensor of the rank $2$
  in $n$-dimensional space such that
             $$
            K_{i_1 i_2}=\t K_{i_1\,i_2-1\,n+2\,n+3\,\dots\, n+k}
             $$
If   $i'_1\leq n$, $i'_2\leq n+1$, $i'_3\leq n+2$  and
   $i'_4=n+3$, then we assign to  $\bf \tilde K$ the symmetric tensor of the rank $3$
in the $n$-dimensional space such that
             $$
            K_{i_1\, i_2\, i_3}=\t K_{i_1\,i_2-1\,i_3-2\,n+3\,n+4\,\dots\, n+k}
             $$
 for $i_1\leq i_2\leq i_3$
and so on. Hence we establish isomorphism which leads to the relation (3).

  Of course all these isomorphisms are highly non-canonical, mais they lead to identities (2,3).





 \bye
