


\magnification=1200 \baselineskip=14pt
\def\vare {\varepsilon}
\def\A {{\bf A}}
\def\t {\tilde}
\def\a {\alpha}
\def\K {{\bf K}}
\def\N {{\bf N}}
\def\V {{\cal V}}
\def\s {{\sigma}}
\def\S {{\Sigma}}
\def\s {{\sigma}}
\def\p{\partial}
\def\vare{{\varepsilon}}
\def\Q {{\bf Q}}
\def\D {{\cal D}}
\def\G {{\Gamma}}
\def\C {{\bf C}}
\def\M {{\cal M}}
\def\Z {{\bf Z}}
\def\U  {{\cal U}}
\def\H {{\cal H}}
\def\R  {{\bf R}}
\def\E  {{\bf E}}
\def\l {\lambda}
\def\degree {{\bf {\rm degree}\,\,}}
\def \finish {${\,\,\vrule height1mm depth2mm width 8pt}$}
\def \m {\medskip}
\def\p {\partial}
\def\r {{\bf r}}
\def\v {{\bf v}}
\def\n {{\bf n}}
\def\t {{\bf t}}
\def\b {{\bf b}}
\def\e{{\bf e}}
\def\ac {{\bf a}}
\def \X   {{\bf X}}
\def \Y   {{\bf Y}}
\def \x   {{\bf x}}
\def \y   {{\bf y}}




{\tt 14 November 2018.}



{\tt The file below was written more that six years ago. It is about the elegant
 (at least it seems to me such) way to calculate the number
                   $$
S_k=\#\{\hbox {number of permutations of $k$ elements which
replace every element}\}\,.$$ 
E.g. $S_1=0$,$S_2=1$, $S_3=2$.

Few days ago I realise that the $Z_2$-verion of the file 
  (an information about is $S_2$ odd or even) is interesting 
for calculating determinants (see the etude linearalgebr.tex).
I present it here as the addendum to the main text.

}

\centerline{\bf One combinatorial problem}

{\it Autumn, 2012}

{\it It was long long long ago when I solved the following exercise:
Denote by $S_k$ a number of sequences of $n$ natural numbers 
$\{1,2,3,\dots,k\}$ such that all the numbers are on the wrong places,
i.e. the first number is not $1$, the second number is not $2$, e.t.c.

   I forget the calculations. I just rememeber that they were not nice, but
the answer  was beautiful, something like $S_k\approx k!/e$. Two months ago
   I found a following beautiful solution. Here it is}:

   
          $$
 \hbox{One can see that}\quad   \sum_{k=1}^nC^k_nS_k=n!\,,
\eqno (1)
          $$ 
where $C^k_n=\pmatrix{n\cr k\cr}={n!\over k! (n-k)!}$.
(The right hand sight of equation (1) is the number of
all permutations of the set with $n$ elements. The summand $C^k_n S_k$
  in the left hand side is the number of permutations such that 
  exactly $n-k$ elements are fixed\footnote{$^*$}{One can see that
it is reasonable to assume that $S_0=1$}.)

Recall that the  $n$-th derivative of the product $FG$ of two functions
$F$ and $G$ is given by the formula
        $$
     \left( d^n\over dx^n\right)\left(F(x)G(x)\right)=
     \sum_{k=1}^n C^k_n\left( d\over dx\right)^k\left( F(x)\right)
        \left( d\over dx\right)^{n-k}\left(G(x)\right)\,.
       $$ 
 Comparing this formula with relation (1) we see that if we choose 
       $$
F=\sum {S_k\over k!} x^k\,,\qquad {\rm and }\,\,   G=e^x\,,
       $$
  then
         $$
     \left( d^n\over dx^n\right)\left(F(x)G(x)\right)_{x=0}=
      \left( d^n\over dx^n\right)\left(F(x)e^x\right)_{x=0}=
   \sum_{k=1}^n C^k_n S_k=n!\,.
         $$
Hence
        $$
F(x)e^x={1\over 1-x}=1+x+x^2+x^3+\dots
        $$
in a vicinity of $x=0$. We come to the answer:
       the sequence $S_k$ is such that
       $$
F=\sum_{k=0}^\infty {S_k\over k!} ={e^{-x}\over 1-x}\,.
        $$
Using this formula we write down the explicit formula for $S_k$.
Denote by $s_k={S_k\over k!}$. We have that
          $$
\sum_{k=0}^\infty {S_k\over k!}x^k=\sum_{k=0}^\infty s_kx^k={e^{-x}\over 1-x}=
\left(1-x+{x^2\over 2}-{x^3\over 3!}+
{x^4\over 4!}-{x^5\over 5!}+\dots\right)
\left(1+x+x^2+x^3+\dots\right)
          $$
        $$
=1+\left(1-1\right)x+\left(1-1+{1\over 2!}\right)x^2+
   \left(1-1+{1\over 2!}-{1\over 3!}\right)x^3+
\left(1-1+{1\over 2!}-{1\over 3!}+{1\over 4!}\right)x^4+\dots
        $$
i.e.
         $$
    s_k={S_k\over k!}=\sum_{p=0}^k {(-1)^p\over p!},\,\,
        {\rm and}\,\, S_k=k!\sum_{p=0}^k {(-1)^p\over p!}\,.
              \eqno (2)
         $$
    
In particular
          $$
         s_\infty=\lim_{k\to\infty}=
\sum_{p=0}^\infty {(-1)^p\over p!}={1\over e}\,,
      \eqno (3)
           $$
i.e. the probability that all terms of the sequence 
  $\{1,2,3,\dots,N\}$
are on the wrong places equals to $\approx {1\over e}$ when $N\to \infty$.

\medskip

\centerline {Is $S_k$ even or odd?}.

This question is in particular important  e.g. in linear algebra
 
We can straightforwardly  answer this
question, analyzing the formula (2): 
            $$
          \hbox
                   {
      $S_k$ is even if $k$ is odd,
and $S_k$ is odd if $k$ is even
               }
                   \eqno (4) 
            $$
 This is much better
to come to the answer just analyzing the formula
(1) over the field $Z_2$.
Notice that
                     $$
              \sum_{{\rm even}\,\, k} C^k_n=
              \sum_{{\rm odd}\,\, k} C^k_n=2^{n-1}\,.
                     \eqno (5)
                     $$ 
Now it has just to check that the sequence (4) obeys the equations
                      $$
                      $$ 
It is evident that for every $N$, the sysem of $N+1$ equations
                     $$
   \sum_{i=0}^n C^i_n S_i=n!\,, {\rm i.e.}\,\,
         \cases 
                {
             C_0^0 S_0=S_0=1\cr
             C_1^0 S_0+C_1^1 S_1=S_0+S_1=1\cr
             C_2^0 S_0+C_2^1 S_1+C_2^2 S_2=S_0+3S_1+S_2=2\cr
             C_3^0 S_0+C_3^1 S_1+C_3^2 S_2+C_3^3 S_3=S_0+3S_1+3S_2+S_3=6\cr
             \dots\cr
                }    
                     \eqno (5)             
                     $$
on $N+1$ variables $S_0,S_1,S_2,\dots S_N$   has unique solution
   $S_0=1$, $S_1=0$, $S_2=1$,$S_3=3$,$\dots$.
    Hence it has unique solution in the field $Z_2$. On the
other hand the sequence identity (5) implies that
the sequence (4) is the solution of equations (5).
    This proves that all $S_k$  have parity $k+1$.

\bye

