\magnification=1200

\def\p{\partial}
\def\t {\tilde}
\def \m {\medskip}
\def\degree {{\bf {\rm degree}\,\,}}
\def \finish {${\,\,\vrule height1mm depth2mm width 8pt}$}

\def\a {\alpha}
\def\vare{{\varepsilon}}
\def\l {\lambda}
\def\s {{\sigma}}

\def\G {{\Gamma}}

\def\A {{\bf A}}
\def\C {{\bf C}}
\def\E  {{\bf E}}
\def\K {{\bf K}}
\def\N {{\bf N}}
\def\Q {{\bf Q}}
\def\R  {{\bf R}}
\def\V {{\cal V}}
\def \X   {{\bf X}}
\def \Y   {{\bf Y}}
\def\Z {{\bf Z}}



\def\ac {{\bf a}}
\def\e{{\bf e}}
\def\f {{\bf f}}
\def\n {{\bf n}}
\def\r {{\bf r}}
\def\v {{\bf v}}
\def \x   {{\bf x}}
\def \y   {{\bf y}}


\def\pt {{\bf pt}}



10 November 2013


\centerline {\bf Chebyshev approximation and Helly's Theorem}

\m

  {\it  Helly's Theorem states that $m\geq n+2$ convex bodies in $\R^{n}$
 have non-empty intersection if any $n+1$ of them have non-empty intersection.
This Theorem stated by German mathematician Helly in 1913 has
many different proofs\footnote{$^{1)}$}.
It can be proved using just  elementary mathematics
(excellent topic for pupils in the school). On the other hand 
one of its proofs uses notion of Chech cohomology.

  In this etude I try to show application of Helly's Theorem
to theory of approximation of functions.
  I am writing this etude inspired and 
based on the wonderful article
of V.G.Boltiansky and N.M.Yaglom "Convex bodies"
(Encyclopaedia of Elementary Mathematics. Volume 5. Geometry. Moscow 1966
(in Russian)) 



}

Helly's theorem on convex bodies have the 
following very interesting application to theory of approximation
of continuous functions by polynomials of order $n$. Here we 
consider in detail the case when we approximate a function by lines
(polynomials of order $n=1$) and briefly formulate the general case.
(The idea of the proof is not very different for general case).

We consider continuous functions on the interval $[a,b]$. 
We define the distance $d_\infty$ between continuous functions as
            $$
    d(f,g)=||f-g||_\infty=\max_{x\in [a,b]}|f(x)-g(x)|\,.
            $$
  We say that the line $L_f=kx+b$ is a line which is the closest 
to the function $f$ if for an arbitrary line $l$,
  $d(f,l)\geq d(f.L_f)=\vare$:
           $$
\max_{x\in [a,b]}|f(x)-kx-b|=\vare\,\,
\hbox{and for arbitrary line $y=k'x+b'$}\,\,    
\max_{x\in [a,b]}|f(x)-k'x-b'|\geq\vare\,.
           $$
The following Theorem is obeyed:

\smallskip

{\bf Theorem 1} Let the  line $L_f$ be a closest line 
to the continuous
function $f=f(x)$, $x\in [a,b]$.
If $\vare$ is the 
distance between this line and the function $f$ then there exist
three points $x_1,x_2,x_3$, $a\leq x_1<x_2<x_3\leq b$
such that the distance between function $f$ and the line $L_f$
at these points 
are $\pm\vare$, and signs are alternating:
             $$
   \cases
    {
  f(x_1)-L_f(x_1)=\vare\cr
  f(x_2)-L_f(x_2)=-\vare\cr
  f(x_3)-L_f(x_3)=\vare\cr
      }\,,
  \quad {\rm or}\quad
   \cases
    {
  f(x_1)-L_f(x_1)=-\vare\cr
  f(x_2)-L_f(x_2)=\vare\cr
  f(x_3)-L_f(x_3)=-\vare\cr
      }\,,
  \eqno (1)
          $$

Respectively for approximation by polynomials of order $n$ we have:

{\bf Theorem 1$^*$} Let the $n-th$ order polynomial
$P^{(n)}_f$
is the closest $n$-th order polynomial  to the continuous
function $f=f(x)$, ($x\in[a,b]$
in the $n+1$-dimensional linear space of all 
$n$-th order polynomials
$P^{(n)}(x)=a_0x^{n}+a_1x^{n-1}+a_1x+a_0$,
 ($a_0,\dots.a_n$
are arbitrary real numbers.). 
If $\vare$ is the 
distance between the parabola $P^{(n)}_f$ and 
the function $f$ then in the interval $[a,b]$ 
there exist
$n+1$  points $x_1,\dots x_{n+1}$, $a\leq x_1<\dots<x_{n+1}\leq b$
such that the distance between function $f$ and the parabola $P_f$
at these points is $\pm\vare$ and signs  are alternating, i.e.
             $$
   \cases
    {
  f(x_1)-L_f(x_1)=\vare\cr
  f(x_2)-L_f(x_2)=-\vare\cr
  f(x_3)-L_f(x_3)=\vare\cr
     \dots  \cr
  f(x_{n+1})-L_f(x_{n+1})=(-1)^n\vare\cr
      }\,,
  \quad {\rm or}\quad
   \cases
    {
  f(x_1)-L_f(x_1)=-\vare\cr
  f(x_2)-L_f(x_2)=\vare\cr
  f(x_3)-L_f(x_3)=-\vare\cr
     \dots  \cr
  f(x_{n+1})-L_f(x_{n+1})=(-1)^{n+1}\vare\cr
      }\,.
             $$

{{\bf Example} {\it Chebyshev approximation and Chebyshev polynomials.}

   Consider Chebyshev polynomails $\{T_k\}$,
                      $$
T_k(x)={1\over 2^k}\cos k \arccos x\,,\quad -1\leq x\leq 1\,,
                      $$
                   $$
T_1(x)=x,
T_2(x)={2x^2-1\over 2}\,, T_3(x)={4x^3-3x\over 4}\,
  T_4(x)={8x^4-8x^2+1\over 8}\dots\,, 
                    $$
(${1\over 2}T_{k-1}(x)+2T_{k+1}(x)=xT_k(x)$.).

The basic property of Chebyshev polynomials is that for every natural
$n$, the polynomial
$T_n(x)$ is the polynomial which is 
closest to zero in the $n$-dimensional
 affine space of all polynomials
 of order $n$ with leading term $x^n$:
            $$
d(T_n)=\max_{x\in [-1,1]}|T_n(x)|={1\over 2^{n-1}}\leq
 \min_{a_1,a_2,\dots,a_n}\max_{x\in [-1,1]}
  |x^n+a_1x^{n-1}+\dots+a_{n-1}x+a_n|\,.
            $$
It implies that the 
polynomial $P^{(n)}=x^{n+1}-T_{n+1}$ is the closest
to the function $f=x^{n+1}$ in the linear space
of $n$-th order polynomials\footnote{$^{1)}$}{In formulation of Theorem
$1^*$ we deal with linear space of $n$-th order polynomials, and
regarding the basic property of Chebyshev polynomials we deal with
{\it linear} space of $n$-th order  polynomials with leading term $x^n$.
E.g. polynomial $T_4(x)-x^4$ belongs to the linear space of $3$-rd order
polynomials, in spite of the fact that its leading term is proportional to
$x^2$. }
The distance between this parabola and function $y=x^{n+1}$
is equal to $\vare={1\over 2^n}$.
At the points $x_i=\arccos {2\pi i\over n}$
the distance is $\pm \vare$ and signes are alternating:
         $$
  x_i^{n+1}-P^{(n)}(x)=-T_{n+1}(x)=-{(-1)^i\over 2^n}\,.
         $$

In this special case the nodes of Chebyshev polynomials
are equidistant. The Theorem tells that
the property of changing the signs is kept in the general case.
This statement is very important for practical 
calculations of Chebyshev polynomials.

               
We sketch the proof of the Theorem for approximation by lines.


First of all formulate the following corollary form Helly's Theorem:

  {\bf Corollary }
 Let $\cal M$ be the set of parallel segments such that
 this set  belongs to bounded domain in $\R^2$. 
Suppose that for an arbitrary three
segments there exists a line which intersects these segments.
Then there exists a line which intersects all the segments.

Respectively if for arbitrary $k+2$ segments there 
exists $k$-th order polynomial   which intersects these segments.
Then there exists $k$-th order polynomial  which 
intersects all the segments.

 We first sketch the proof of Theorem based on this corollary
then prove the Corollary.

  We will prove the Theorem for lines,
i.e. for approximation by polynomials of the order $n=1$. 
The idea of proof is the same for
an arbitrary $n$.

{\sl Proof}




 Let $L_f\colon y=kx+b$ be a closest line to the function $f$.
Let a distance be equal to $\vare$:
                       $$
   \max_{x\in [a,b]}|f(x)-kx-b|=\vare,\quad
\forall k',b'\,,\quad
\max_{x\in [a,b]}|f(x)-k'x-b'|\geq \vare. 
                    $$





Pick arbitrary $\vare'<\vare$. Consider the set
$\{d_x\}$ ($x\in [a,b]$)
of segments $d_x=[a_x,b_x]$ such that points $a_x,b_x$
have coordinates
                $$
   a_x=(x,f(x)-\vare'), b_x=(x,f(x)+\vare')\,.
                $$ 
  The set of these segments is the set of vertical segments
centered at the points of graph of the function $f$
with length $2\vare'$.

It follows from corollary 1 that for every $\vare'<\vare$
there exist three points $x_1,x_2,x_3$ such that 
{\it there is no a line}
which intersects corresponding segments $d_{x_1},d_{x_2}, d_{x_3}$.
Indeed if for arbitrary three points
$x_1,x_2,x_3$ there exists  a line
which intersects corresponding segments $d_{x_1},d_{x_2}, d_{x_3}$
then due the corollary there exists a line $L'$ 
which intersects  all the segments
$\{d_x\}$, i.e. the distance between line
$L'$ and a function $f$ is less or equal to $\vare'$. This contradicts
to the fact the line $L_f$ is the closest line.

We come to the following observation:

{\bf Observation 1} For every $\vare'\colon\,\,0<\vare'<\vare$
there exist three points $x_1,x_2,x_3\in [a,b]$ such that 
the distance between arbitrary line and the function $f$
at one of these points is greater than $\vare'$.
             

This observation plus  continuity arguments implies the following 
observation:



{\bf Observation 2} There exist three points 
$a\leq x_1\leq x_2\leq x_3\leq b$ such that  
the distance between arbitrary line and the function $f$
at one of these points is greater or equal than $\vare$.


Indeed consider the sequence 
$\{\vare_n\}$ such that $0<\vare_n<\vare$ and 
$\vare_n\to vare$, e.g. $\vare_n=vare-{1\over n+M}$ (for enough big $M$).
Choose for every $\vare_n$ points $\{x^{(n)}_1,x^{(n)}_2,x^{(n)}_3\}$
such that at these points the distance between every line
including the line $L_f$ is greater than $\vare_n$. 
Due to compactness of the segment $[a,b]$  we
can pick  from this sequence the
subsequence 
$\{x^{(n_k)}_1, x^{(n_k)}_2, x^{(n_k)}_3\}$ 
such that $\lim_{k\to \infty}x^{(n_k)}_1=x_1$,
$\lim_{k\to \infty}x^{(n_k)}_2=x_2$,
$\lim_{k\to \infty}x^{(n_k)}_3=x_3$.
One can see that points $\{x_1,x_2,x_3\}$ are the points
where for an arbitrary line $L$, the 
distance between function $f$ and the line $L$ is bigger that
every $\vare_n$. This we come to the the statement of Observation 2.

Now prove the Theorem using the Observation 2.
 
Using Observation 2 choose the points $\{x_1,x_2,x_3\}$
and show that the relations (1) are obeyed.




Consider 
$\Delta_i=f(x_i)-L_f(x_i)$, $i=1,2,3$. 
We have to show that
all $\Delta_i$ have the modulus $\vare$ and signes are alternating:
               $$
|\Delta_1|=|\Delta_2|=|\Delta_3|=\vare,\quad
\Delta_1\Delta_3>0,\quad
 \Delta_1\Delta2<0\,,
      \eqno (1a) 
     $$
i.e. conditions (1) are obeyed. 
If these conditions are not obeyed
then it is easy to show that one can always 
find a line $L$ such that
its distance to the function $f$ at all points $x_1,x_2$ and
$x_3$ is less than $vare$. This contradicts to Observation 2. 

Suppose for example that $\Delta_1>0$ and 
$\Delta_2>0$.
If $\Delta_3>0$ then one can choose $\delta>0$ such that 
all the distances between function $f$
and the line $L=L_f+\vare$ at points $x_1,x_2,x_3$ 
are less than $\vare$.
 If $\Delta_3<0$ then rotating the line $L_f$ around the point
$(x_2,L_f(x_2))$ on a smal angle we again come to the line
$L'$ such that 
all the distances between function $f$
and the line $L'$ at points $x_1,x_2,x_3$ 
are less than $\vare$.
Hence if $\Delta_1>0$ then $\Delta_2<0$.

By analogous considerations one can easy show that in all the cases
when conditions (1),(1a) are not obeyed then
one can choose another line $L'$
such that the distance between the line $L'$
and a function $f$ at all points $x_1,x_2,x_3$ is less than $\vare$.   
This implies the statement of Theorem.

\smallskip



Finally we prove the Corollary1.

Let $\cal M$ be  the set of parallel segments. WLOG we may suppose that
all the segments are vertical. 
Consider an arbitrary vertical segment $d$,
 Denote by $\Pi_d$ the set of lines which
intersects with the segment $d$. Every line which intersect
this segment is non-vertical line, $y=kx+b$.
We parameterise all not-vertical lines by pairs
$(k,b)$. One can see that the set of the pairs 
which correspond to the set of line $\Pi_d$ is the convex set:
If segment connects the points $(x_0,y_0), (x_0,y_0+d)$ ($d>0$) then  
the condition that the line $kx+b$ intersect this segment is:
           $$
   y_0\leq  kx_0+b\leq y_0+d 
             $$
These conditions define the strip, the convex set in the plane $(k,b)$.
Now the Corollary follows from Helly's Theorem.\finish

\m

{\bf Example} Consider approximation of the function $f=\sin x$
on the interval $[0,\pi/2]$.
 One can see that in this case the closest line
is uniquely defined by the condition (1) of the Theorem.
If $L_f\colon \,\,y=kx+b$ is the closest line, 
and the distance is equal to $\vare$
then
         $$
 \vare=(kx+b-\sin x)\big\vert_{x=0}=
  -\left(\sin x-\left(kx+b\right)\right)\big\vert_{x=x_0}=
   (kx+b-\sin x)\big\vert_{x=1}\,.
    $$
The points $x_1,x_2,x_3$ are:  $x_1=0$, $x_3=1$ and the middle point $x_2$
is defined by the stationary point condition.
 We come to simultaneous equations
         $$
      \cases
      {
       b=\vare\cr
       \sin x_0-kx_0-b=\vare\cr
         \cos x_0=k\cr       
 k{\pi\over 2}+b-1=\vare\cr
      }
         $$
Solving this system we come to 
  $k={2\over\pi}$, $x_0=\arccos{2\over \pi}$ and
               $$
  \vare=b={1\over 2}\left(
 \sin\arccos {2\over \pi}-{2\over\pi}\arccos{2\over \pi}
     \right)=
  {\sqrt{\pi^2-4}-2\arccos {2\over \pi}\over 2\pi}\approx
   0.10526
      $$ 
The line $y={2\over \pi}x+b$ with $b\approx 0.10526$ is the closest line
to the function $f=\sin x$ on the interval $[0,\pi/2]$.  The distance
is equal to $\vare\approx 0.10526$.
\bye
