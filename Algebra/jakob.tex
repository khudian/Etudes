 %\magnification=1200



%\font\magnifiedfiverm=cmr5 

  It is a little note about so called Jacobian problem.

  Let $P(x,y)$ and $Q(x,y)$ be two polynomipals on variables $x,y$
such that
\def\p{\partial}
                     $$
                   \det
                    \pmatrix
                     {
               {\p P\over \p x}
                   {\p P\over \p y}\cr
                    {\p Q\over \p x}
                    {\p Q\over \p y}\cr
                      }
                         =1
                       \eqno (1)
                   $$
Then the inverse functions are always polynomials too.
For example to the polynomial transformation
                  $$
                 \cases
                     {
                    x\mapsto x+(x+y)^3\cr
                    y\mapsto x+y  \cr
                        }
                      $$
 corresponds inverse transformation
                           $$
                \cases
                   {
                    x\mapsto x-y^3\cr
                    y\mapsto y+y^3-x\cr
                      }
                            $$
As is stated in the book of Kirillov: ("Shto takoje chislo"---broshjura,
1993 god) this problem is unsolved till now. (Kak ja ponimaju net
kontrprimera i ne dokazano eto utverzhdenije.)
One can following to this book
consider non-commutative version of this problem.
Let $W$ be associative algebra with unity, generated by two generators
  $p$ and $q$ which obey only to the constraint:
                    $$
                       pq-qp=1
                      $$
  (Weyl algebra)
Let $A$ and $B$ be two polynomials on $p$ and $q$ such that
            $$
         AB-BA=1
                                  \eqno (2)
             $$
Then the endomorphism generated by the map
                 $$
           p\mapsto A, q\mapsto B
                  $$
  has to be an isomorphism!---i.e. $p$ and $q$ can be expressed as
polynomials of $A$ and $B$.
You see that the first problem is quwsiclassical limit of the second one
(commutator (2) $\rightarrow$ Poisson bracket $=$ jakobian.)
      \bye
