


%%% eto tex-file (ne LateX) a obychnyj TeX

\magnification=1200

  Dorogoj Volodya

 Eto cohenj interesnaja kontsruktsija
kotoruju mohzno ispoljzovatj dlja vychilsnije realjnykh kogomologij.
ja popytajusj objasnitj ejo po krajnej mere na algoritmicheskom urovne.



Zeta funktsija algebrajicheskogo mnogoobrazia nad konechnym polem
eto projizvodjashaja funtsija kotoraja opissyvajet
posledovateljnostj kolichestva tochek mnogoobrazija
ot stepeni rashirenija polja.
 Vychislenije Zeta-funtksiji to odin iz spossobov vychislenija
kogomologij.


 Podrobneje:
Rassmotrim $n$-mernoje vectornoje prostransvo $A^n$
nad polem $F_p$  ($F_p=Z/pz$).
Nabor mnoochelnov (s koeffitsientami iz $F_p$) zadajot
mnogoobrazije kotoroje my budem oboznachatj bukvoj $X$.

Pustj Eto mnogoobrazije soderzhit $v_1$ tochek
(kolichetsvo tochek mnogoobrazija shcitajestsja pereborom vsekh $p^n$
tochek $A^n$ i proverkoj oni lezhat v $X$ ili net.)

Teperj my budem rashirjatj pole koeffitsientov:
S  $F_p$ perejdjom k $F_{p^2}$ zatem k k $F_{p^3}$ i.t.d.
jasno shto kolichestvo tochek mnogoobrazija budet rasti.
Oboznachim cherez $n_k$ kolichestvo tochek (to estj kolichestvo)
nulej mnogochelnov (s koeffitsinetami iz $F_p$!!!!)
v pole $F_{p^k}$
A teperj mozhno opredelitj zeta-funktsiju kak:
                $$
Z(t)=\exp P(t)
                  $$
gde
             $$
           P(t)=\exp \sum {n_kt^k\over k}
             $$
Otkuda takoje vitievatoje opredelnije?
 Primer: rassmotrim pustoje uravnenije, to estj
 v kachestve $X$ rassmotrim vse  tochki! Togda:
      $n_k=p^k$, $P(t)=\log (1-pt)$ i
$Z(t)={1\over 1-pt}$. Zamenoj $t=p^{-s}$ mozhnopidti k obychnoj
zeta-funtsiji

 Teperj samoje udiviteljnoje:

  Okazyvajetjsa $Z(t)$ ratsionaljnaja funtsija, po-drugomu:
posledovateljnostj $n_k$ rekurrentnaja posledovateljnostj
 Poetomu ejo mozhno schitatj na kompjutere.

 Dostatochno vychislitj pervyje neskoljko chlenov posledovateljosti,
daljshe vsjo shiotajejsja.


 Primer Rassmotrim mnogochlen $x^2-y^2=0$ nad $A^2$

Kolichestvo reshenij  uravnenija $x^2-y^2=0$ v $F_{p^k}$
ravno $2p^k-1$ tak kak $x$ proizvoljnyj element i  $y=\pm x$.
Esli $x\not =0$ to  kazdomu $x$ sootvetstvujut dva $y$.

Itak $n_k=2p^k-1$. Legko poshitatj $P$ i $Z$:
                    $$
                 P(t)=\sum_{k=1}^0 {2p_k-1\over k}t^k=
              =\log {1-t\over (1-pt)^2}
                      $$

                  $$
  Z(t)= {1-t\over (1-pt)^2}
                   $$

Boleje interesnyj primer: rassmotrim "okruzhnostj"
   $x^2+y^2=1$ v $F_p$. Ja eto segodnja schital:
vrode by otvet: $n_k=p^k+1$ i
        $$
        Z(t)={1\over (1-t)(1-pt)}
             $$

Vopros: zachem vsjo eto nuzhno?

 Napirmer mozhno na mashine shciotatj kogomologiji
nastojashejikh mnogoobrazij.  Tedutsirovatj
obychnyje mnogoobrazija po modulju $p$
shcitajt zeta-funtskiju. Ona soderzhit informatsiju
o kogomologijakh po modulju $p$. Daljhse ty sam ponimajehsj
kak eto vazhno.

Govorjat shto takim sposobom byli podschitany realjnyje otvety.



               Hovik

\bye

                        Dr Hovhannes Khudaverdyan
                                   Department of Mathematics
                                      tel. 0161-2008975


On Tue, 25 Jan 2005, Vladimir Kornyak wrote:

> Dorogoy Ovik
>
> >   Esli tebe netrudno, poka popriderzhi denjgi pri sebe.
>
> Konechno netrudno - oni mnogo mesta ne zanimayut,
> deystvitel'no, vdrug Mitya ob'yavitsa.
>
> > Ja tut razgovarival s rebjtatami pro $\zeta$-funktsiju konechnogo
> > polya i v razgovore vdrug vspomnilisj tvoji kogomologiji
> > nad konechnymi poljami: eto dejsviteljno khoroshij vychisliteljnyj spossob
> >  schitatj chisla Betti pri pomoshi tvojikh neravenstv.
>
> A chto takoe $\zeta$-funktsija konechnogo polya?
> Yesli konstruktsii etoy funktsii ne vyhodyat iz polya,
> ona dolzhna byt' polinomom?
>
> Volodya
>
>
