\magnification=1200

\def\p{\partial}
\def\t {\tilde}
\def \m {\medskip}
\def\degree {{\bf {\rm degree}\,\,}}
\def \finish {${\,\,\vrule height1mm depth2mm width 8pt}$}

\def\a {\alpha}
\def\vare{{\varepsilon}}
\def\l {\lambda}
\def\s {{\sigma}}

\def\G {{\Gamma}}

\def\A {{\bf A}}
\def\C {{\bf C}}
\def\E  {{\bf E}}
\def\K {{\bf K}}
\def\N {{\bf N}}
\def\Q {{\bf Q}}
\def\R  {{\bf R}}
\def\V {{\cal V}}
\def \X   {{\bf X}}
\def \Y   {{\bf Y}}
\def\Z {{\bf Z}}



\def\ac {{\bf a}}
\def\e{{\bf e}}
\def\f {{\bf f}}
\def\n {{\bf n}}
\def\r {{\bf r}}
\def\v {{\bf v}}
\def \x   {{\bf x}}
\def \y   {{\bf y}}


\def\pt {{\bf pt}}


{\bf Gamma function}

\def\G {\Gamma}

  We know very well the integral representation of Gamma function
        $$
   \G(x)=(x-1)!=\int_0^\infty t^{x}e^{-t}dt
           \eqno (1)
        $$
We also know very well that
         $$
   B(x,y)={\G(x)\G(y)\over \G(x+y)}\,,
              \eqno (2)
         $$
where
          $$
    B(x,y)=\int_0^1 t^{x-1}(1-t)^{y-1}dt\,.
          $$

In fact Eiuler worked with 
a different definition of Gamma-function,
which was natural continuation on all $x$ of factorial.
Using this definition he came to formulae (1) and (2).

  I cannot avoid temptation to 
 recall here this topic.


  In fact Euler  observed gamma-function in the following way:
He noted that
            $$
k!={(N+k)!\over (k+1)_N}={N!(N+1)_k\over (k+1)_N }=
       \lim_{N\to \infty}{N! N^k\over (k+1)_N}\,,
            $$
where $(A)_r=A(A+1)\dots (A+r-1)$. Hence one can
define
            $$
x!=\lim_{N\to \infty}{N! N^x\over (x+1)_N}
            $$
and respectively
         $$
\G(x)=(x-1)!=\lim_{N\to \infty}{N! N^{x-1}\over (x)_N}\,.
       \eqno (3)
         $$

This definition looks not very beautiful, but one can easily imply
equations (1), (2) and anbtoher identites.

  E.g. one can easy to see that equation (3) implies
 that
           $$
{1\over \G(x)}=xe^{\gamma x}\prod_{k=1}^\infty 
  \left(\left(1+{x\over k}\right)e^{-{k\over n}}\right)\,,
            $$
where $\gamma=\lim_{N\to \infty}(1+{1\over 2}+\dots+{1\over N}-\log N)$
is Euler constant.

Indeed    
             $$ 
        {1\over\G(x)}=
         $$
\bye
