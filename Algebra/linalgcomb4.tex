

\magnification=1200 \baselineskip=14pt
\def\vare {\varepsilon}
\def\A {{\bf A}}
\def\t {\tilde}
\def\a {\alpha}
\def\K {{\bf K}}
\def\N {{\bf N}}
\def\V {{\cal V}}
\def\s {{\sigma}}
\def\S {{\Sigma}}
\def\s {{\sigma}}
\def\p{\partial}
\def\vare{{\varepsilon}}
\def\Q {{\bf Q}}
\def\D {{\cal D}}
\def\G {{\Gamma}}
\def\C {{\bf C}}
\def\M {{\cal M}}
\def\Z {{\bf Z}}
\def\U  {{\cal U}}
\def\H {{\cal H}}
\def\R  {{\bf R}}
\def\E  {{\bf E}}
\def\l {\lambda}
\def\degree {{\bf {\rm degree}\,\,}}
\def \finish {${\,\,\vrule height1mm depth2mm width 8pt}$}
\def \m {\medskip}
\def\p {\partial}
\def\r {{\bf r}}
\def\v {{\bf v}}
\def\n {{\bf n}}
\def\t {{\bf t}}
\def\b {{\bf b}}
\def\e{{\bf e}}
\def\ac {{\bf a}}
\def \X   {{\bf X}}
\def \Y   {{\bf Y}}
\def \x   {{\bf x}}
\def \y   {{\bf y}}




{\tt 16 November 2018.}


\centerline {On one linear algebra problem}

 {\it Two weeks ago, David (my son) suggested me the following problem:
  There are 101 coins. For every coin you can divide the
rest of the coins on two sets on  50 coins such that these two sets have
the same weight. Prove that all the coins have the same
weight.  Occasionally I came to the solution, it was highly unexpected
(I used one combinatorial result)
and  very beautiful. I totally changed
my opinion about this problem  (at first I consider it as a boring
exercise in linear algebra).
I would like to add also, that
   David was informed about this problem  from mathematician,
Vladimir Dotsenko, and I never forget the  beautiful proof of
Nullstelenzats which was suggested by Vladimir (see e.g. the etude
"On a simple proof of Nullstelensatz" in my homepage
 "www.maths.manchester.ac.uk/khudian": (the subsection
   /Etudes/Algebra/)).

I what follows, there is my solution of this problem.
In Appendix 1, I put the solution of this problem suggested by 
James Montaldi, and in Appendix 2 some statement related with my proof.
}


\medskip

Here is my solution:

\smallskip

    Denote the weights of the coins
    by $\{x_1,x_2,\dots,x_{101}\}$.  The problem evidently is reduced
to the following: Consider the matrix of the size $101\times 101$,
$M=||M_{ik}||$, $i,k=1,\dots,101$),
such that all diagonal elements of this matrix are equal to
zero and in every row some  50 non-diagonal entries are equal to $+1$
and another fifty non-diagonal entries are equal to $-1$.

   Then it is evident that the vector
     $$
\e=\{1,1,\dots,1\}
      \eqno (1)
     $$
is the eigenvector of the matrix $M$ with eigenvalue $0$. 
We have to prove that this is all: the space of zero eigenvectors
is $1$-dimensional, or in other words,
the vector $\e$ says that rank of the matirx $M$ is less than $101$.
We have to prove that it is just equal to $100$.


Consider the characterisitic polynomial
     $$
  P(\lambda)=\det (\lambda-M)=\sum_{k=0}^{101} 
a_k\lambda^{101-k}\,.
     $$
We have that $a_0=1$, the existence of 
eigenvector with zero eigenvalue
meams that $a_{101}=\det M=0$. To prove that
the zero eigenvalue subspace is one-dimensional, i.e.
all the coins have the same weihgt, we have to prove
that $$
   a_{100}\not=0\,.
     $$ 
Now write the formula 
 for  $a_{100}$:
    $$
a_{100}=\sum_{} \sigma_{k_1 k_2\dots k_{101}} M_{1k_1}M_{2k_2}\dots 
 M_{100 k_{100}}
 M_{101 k_{101}}\,,
      \eqno (3)
    $$
where $\sigma_{k_1k_2\dots k_{101}}$ is the sign of permutation, and
  in every monom for all $\{i\}$
except just one $i_0$,  $k_i\not=i$, and
  $M_{ik_i}=\pm 1$.
 The number of these monoms is equal to $101\times S_{100}$,
where $S_{100}$ is the number of permutations
of $100$ numbers such that all the elements change the place,
and the point is that number $S_{100}$ is ODD!!!!
Hence the number of monoms
in the expression  (3) is odd, and all the monoms are equal to
$\pm 1$, i.e. $a_{100}\not=0$.  In other words
 if we will write the expression
for $a_{100}$ over the field $Z_2=Z\backslash 2Z$ we come to
                    $$
a_{100}=\sum_{i=1}^{101}\left(\sum_{r\not=i, k_r\not=r} 
M_{rk_r}\right)=101\cdot S_{100}=1\cdot 1=1\not=0 (mod 2)\,.
                      $$
The central point of the proof is the
fact that $S_{100}$ is an odd number.
In fact one can see that the number $S_k$ of permutations of $k$ elements
which displace all the elements is odd if $k$ is even, and vice versa,
  $S_k$ is even if $k$ is odd; see also
the etude 
 "On number of permutations wich displace all the elements"
in my homepage
 "www.maths.manchester.ac.uk/khudian": (the subsection
   /Etudes/Arithmetics))

\m

         {\bf Appendix 1}

The solution above indicates that the field $Z_2=Z\backslash 2Z$
  is the crucial. 
James Montaldi presented the solution which is just based on the concept
of this field.  His solution is the following:




for the matrix $M$ which has to be considered (
it vanishes on diagonal and
 it is equal to $+1$ or $-1$ at all non-diagonal entries)
one can just immediately calculate all eigenvectors and eigenvalues,
but over the filed $Z_2$!
Indeed consider the matrix $A=M+I$, where $I$ is identity matrix.
 The  $A$ is the
matrix such that all  the entries
of this matrix are equal to $+1$ or $-1$. The vector
  $e={1,1,1,\dots,1}$ which corresponds to the coins of the same weights
is  the eigenvector of the matrix $A$
with eigenvalue $n$, and ANY VECTOR WHICH IS ORTHOGONAL TO THE VECTOR
$e$ is the eigenvector of $A$ with eigenvalue $0$, OVER FIELD $Z_2$.
Thus we see that the matrix $M=A-I$ has one eigenvector with eigenvalue
$n-1$ and $n-1$ vectors with eigenvalue $0$ over the field
$Z_2$. Hence this matrix   $M$
  has rank $n-1$ over the field $Z_2$.
Thus for matrix  $M$
  if $Mx=0$ then $x$ is proportional to $e$, i.e. rank
of $M$ is equal to $n-1$ if $n$ is odd.

I think that the main step is to change the field $R\to Z_2$.
James works just over field $Z_2$ and his solution
is of course more natural.
    I could not overcome the psichological barriers, and my solution
is in terms of even and odd numbers.
    In fact we have be surprised that
the problem about real numbers, the field $Z_2$ plays the major role in
\footnote{$^*$}{comment of Yuri Bazlov}!
 


\m

         {\bf Appendix 2}

Of course one can prove using this fact more general statement:
  Let $M$ be $n\times n$ matrix, such that alldiagonal entries
vanish, and all non-diagonal entries are equal to $\pm 1$.
Then for the characteristic polynomial
             $
     \det (\lambda-M) =a_k\lambda^{n-k}
             $,
for every coefficient $a_k$,
                $$
a_k=C^k_n S_k={n!\over k!(n-k)!}(k+1) \,,(mod 2)
                   $$
and in particular
              $$
a_rkK\not=0\,, \hbox {if $C^k_n$ and $k+1$ are odd numbers}.
            $$

           
 
\bye
