\magnification=1200
\baselineskip=14pt
\def\vare {\varepsilon}
\def\A {{\bf A}}
\def\t {\tilde}
\def\a {\alpha}
\def\K {{\bf K}}
\def\N {{\bf N}}
\def\V {{\cal V}}
\def\s {{\sigma}}
\def\S {{\Sigma}}
\def\s {{\sigma}}
\def\p{\partial}
\def\vare{{\varepsilon}}
\def\Q {{\bf Q}}
\def\D {{\cal D}}
\def\G {{\Gamma}}
\def\C {{\bf C}}
\def\M {{\cal M}}
\def\Z {{\bf Z}}
\def\U  {{\cal U}}
\def\H {{\cal H}}
\def\R  {{\bf R}}
\def\E  {{\bf E}}
\def\l {\lambda}
\def\degree {{\bf {\rm degree}\,\,}}
\def \finish {${\,\,\vrule height1mm depth2mm width 8pt}$}
\def \m {\medskip}
\def\p {\partial}
\def\r {{\bf r}}
\def\v {{\bf v}}
\def\n {{\bf n}}
\def\t {{\bf t}}
\def\b {{\bf b}}
\def\e{{\bf e}}
\def\ac {{\bf a}}
\def \X   {{\bf X}}
\def \Y   {{\bf Y}}
\def \x   {{\bf x}}
\def \y   {{\bf y}}
\def\f {{\bf f}}
\def\pt {{\bf p}}


   \centerline  {\bf Transcendency of the number $"e"$}
 
 \bigskip
 
    
    {\it This fact was proved by Hermit in 1873. The essential simplification of the proof was performed by Hilbert
    in 1983 , and Gordan and Hurwitz. I  expose here this proof  based on the Easter lectures of  F. Klein in 1894.}
    
\bigskip    
 \centerline {${\cal x}1$  Preliminary construction }

\m   
  
  We consider the following algebraical transformation on polynomials:
             $$
P(x)\to \hat P(x)=\sum_k p_k(x)k! \quad {\rm if}\quad P(x+h)=\sum p_k(x)h^k\,.
\eqno (1.1a)
             $$
  In other words is the finite sum
           $$
           \hat P(x)=P(x)+P'(x)+P''(x)+\dots=\sum_k {P^{(k)}(x)},
           \eqno (1.1b)
           $$
  where ${P^{(k)}(x)}$ is $k$-th derivative of the polynomial $P(x)$.
  

   One can see that this transformation can be rewritten using the integration in the following way:
             $$
  P(x)\rightarrow \hat P(x)=\int_0^\infty e^{-t}P(x+t)dt\,.
  \eqno (1.1c)          
             $$
  
  
  {\bf Proposition}  {\it For an arbitrary polynomial $P(x)$ and an arbitrary integer $k$ the following identity holds:}
                 $$
             e^k\hat P(0)=P(k)+q_{k,P}, \,\,{\rm where}\,\, q_{k,P}=e^k\int_0^ke^{-t}P(t)dt\,.
                \eqno (1.2)
                 $$
  Proof follows immediately from the formula (1c) for the transformation $P\to \hat P$:
                   $$
     e^k\hat P(0)=e^k\int_0^\infty e^{-t}P(t)dt=e^k\int_0^k e^{-t}P(t)dt+\int_k^\infty e^{k-t}P(t)dt=
       $$
       $$
  =\int_0^\infty e^{-t}P(k+t)dt+q_{k,P}=\hat P(k)+q_{k,P}\,.
                   $$
  \m




 \centerline {${\cal x}2$  Proof of the transcendency}


    Let $e$ be algebraic number, i.e. there exist rational numbers $C_i$ such that
                  $$
           C_0+C_1e+C_2e^2+\dots+C_Ne^N=0\,.
           \eqno (2.1)
                  $$
  We can suppose that all $C_i$ are  integers.

   \bigskip


We will show that this is absurdity.

\bigskip
  Consider the following polynomial
             $$
 \Phi_p(x)={x^{p-1}\over (p-1)!}\left[(1-x)(2-x)\dots (N-x)\right]^p,
 \eqno (2.2)
             $$
  depending on integers $N$ and $p$. We suppose that $N$ is the degree of the polynomial (1)
  and $p$ is a prime number.


\m

   The scheme of the proof is the following:  We will choose a big integer $M$ and multiply  the equation $C_0+C_1e+C_2e^2+\dots+C_Ne^N=0$ on the integer $M$. Thus we come to the number
                 $$
  \Psi=M(C_0+C_1e+C_2e^2+\dots+C_Ne^N)\,.
  \eqno (2.3)
                 $$
   We will show that the integer part of the number $\Psi$ is not equal to zero, thus coming to absurdity.

\m

 Take                       $$
       M=\hat \Phi(0)=\Phi(0)+\Phi'(0)+\dots
                    \eqno (2.4)
                    $$
       according to the relation (1.1b).

One can easy to see that $M$ is indeed an integer number. Indeed it follows from the 
construction (2.2) that in the expansion $\Phi(x)=\sum c_kx^k$ all the coefficients $c_k$ 
vanish for $k=0,1,2,\dots,p-1$ and
the polynomial $\Phi(x)$ equals to
                      $$
\Phi(x)={r_{p-1}x^{p-1}\over (p-1)!}+{r_{p}x^{p}\over (p-1)!}+{r_{p+1}x^{p+1}\over (p-1)!}+\dots+
{r_{2p-1}x^{2p-1}\over (p-1)!}
                              \eqno (2.5)
                      $$
where  all $r_i$ are integers and
                  $$
        r_{p-1}=(n!)^p,
        \eqno (2.6)
                  $$
Hence
              $$
 M=M=\hat \Phi(0)=\sum_k k!c_k=r_{p-1}+ r_pp+r_{p+1}p(p+1)+\dots+
r_{2p-1}p\cdot(p+1)\cdot\dots \cdot(2p-1)
\eqno (2.7)
              $$
is an integer.

  Now calculate the number $\Psi$ in (2.3) for $M=\hat\Phi(0)$.  Using the Proposition (1.2) we have
           $$
        \Psi=M(C_0+C_1e+\dots+C_Ne^N)=\sum_{k=0}^Ne^k\hat\Phi(0)C_k=
                 $$
                 $$
        \underbrace{\hat\Phi(0)C_0+\hat\Phi(1)C_1+\hat\Phi(2)C_2+\dots+
        \hat\Phi(N)C_N}_{I}+\underbrace{\sum_{k=0}^Ne^kq_{k,\Phi}C_k}_{II}
        \eqno (2.8)
           $$


Now we show that if $p$ is enough big prime number then  $I$ in this equation is an integer which is not equal to zero
and II is a small number (its module is less than 1).
           We have
             $$
 I=\hat\Phi(0)C_0+\hat\Phi(1)C_1+\hat\Phi(2)C_2+\dots+\hat\Phi(N)C_N
             $$
All $C_i$ are integers. We already proved than $\hat\Phi(0)=M$ is an integer (see the equation (2.7)). 
Analogously one can see that
all $\hat\Phi(k)$ for $k=1,2,3,\dots,N$ are integers. Indeed according to (1.1) we have that
for   $k=1,2,3,\dots,N$
     $$
  \Phi(k+t)={(k+t)^{p-1}\over (p-1)!}\left[(1-k-t)(2-k-t)\dots (N-k-t)\right]^p=
         {t^p\over (p-1)!}L_k(t)
     $$
where $L_k(t)$ is a polynomial on $t$ with integer coefficients: $L_k(t)=\l_{k,r}t^r$  Hence according to (1.1a)
         $$
     \hat\Phi(k)=\sum_r \l_{k,r}{(p+r)!\over (p-1)!}=l_{1,0}p+l_{1,1}p(p+1)+\dots
     \eqno (2.9)
         $$
We see that all $\hat \Phi(k)$ are integers. Moreover from the equations (2.6) and (2.7)
we see that the integer $M=\hat\Phi(0)$ is the integer {\it which is not divisible on the prime number $p$}.
On the other hand from the equation (2.9)we see that all the integers
 $\hat\Phi(k)$ are the integers {\it which are divisible on the prime number $p$}.
 Hence we see that the part $I$ of the number $\Psi$ in the equation  (2.8) all the terms $\hat \Phi(k)C_k$
 are divisible on $p$ for  $k=1,2,3,\dots, N$ and the term  $C_0\hat \Phi(0)$ {is not divisible on $p$} in the case
 if $p$ is bigger than $C_0$. This we see that the first term I is an integer which is not equal to zero.

This is the central part of the proof.



It remains to prove that the the module of the second term II in (2.8)  is smaller than 1if
the prime number $p$ is big enough. One can very easy to estimate II.
Indeed one can see that 
         $$
     |\Phi_p(x)|<{N^{p-1}C^p\over (p-1)!} \,\, \hbox{for $0\leq x\leq N$}    
         $$
where $C$ is the upper bound of the polynomial $(1-x)(2-x)\dots (N-x)$.  Hence
$\Phi_p(x)$ uniformly tends to $0$ if $p\to \infty$ and
                   $$
       \left|\sum_{k=0}^Ne^kq_{k,\Phi}C_k\right|\leq\sum_{k=0}^N |C_k|e^k\left|\int_0^ke^{-t}\Phi_p(t)dt\right|\to 0\,\,
       {\rm if}\,\,p\to \infty\,.
       $$
Hence we prove that the second term in (2.8) tends to zero. 

We prove that $\Psi\not=0$ for enough big $p$. This contradicts  to  (2.1).\finish 
    \bye