%From khudian@manchester.ac.uk Mon Nov 28 13:17:13 2016
%Date: Mon, 28 Nov 2016 13:17:13 +0000 (GMT)
%From: khudian@manchester.ac.uk
%To: Alexander Veselov <A.P.Veselov@lboro.ac.uk>
%Subject: O nedifferentsirujemoj funtksii v teoriji verojatnostej


             Privet Sasha,
     Vo pervykh boljshoe spasibo za vsjo!
    Vo vtorykh boljshoe spasibo za
potrisasjushij vecher s butylkoj italianskogo
likjora nedopitogo Ferapontovym , i s tvojimi 
matematicheskimi rasskazami. A pro
orthogonaljnyje polynomy $P_n(x)=P(n,x)$
          eto super!
     V tretjikh naschot  spetsfunktsii kotoryje
ne analitichny.
    Ja vernulsia domoj i posmotrel moji staryje zapisi:
V 1996 godu ja khotel  poniatj sviazj teoriji verojatnosti
s teorijej mery
Poniatno shto to, shto ja napishu trivilajno, no
   vsjo -taki napishu:
  Rasmotrim dlia proizvoljnogo $q\colon 0\leq q\leq 1$
    na otrezke $[0,1]$ funktsiju:

          $$
    F=F_q(x)=F\left(\sum {a_n(2^n)}\right)=
      \sum a_nq^n
    \left(p\over q\right)^{a_1+a_2+\dots a_{n-1}},
          $$
gde $a_i$ eto nuli libo edinichki, $p=1-q$, $0\leq q\leq 1$
      $x=\sum a_n/2^n$
argument funtskii  $F=F_q$
eto chislo v dvojichnoj sisteme
i funtskija $F_q$ otobrazhajet $[0,1]$ v $[0,1]$.
    Esli $q=1/2$ to $F$ tozhdestvennaja funtksija

Naprimer
         $$
     F(0)=0, F(1/2)=q,  F(1/4)=q^2,   F(1/8)=q^3
     F(1/2+1/4)=q+qp     F(1/2+1/8)=q+q^2p
       F(1/2+1/4+1/8)=q+pq+p^2q\,,\dots
         $$
$F$ monotnonno vozrastajushaja ne differentsirujemaja funtskija
esli $q\not=1/2$. (Esli $q=1/2$ to $F$ is identity)
V chom smysl etoj funkciintsii:
eto konechno elementrano, no v 1996 godu ja
izuchal teoremu Kolmogorova o rekonstruktsiji
verojatnostnogo prostranstva s meroj
po zadannym sluchajnym velichinam.
    V chastnosti dlia brosanija monetki
mozhno postrojitj verojatsnoje
prostranstvo i v slucaje ravnykh verojatnostej orla
i  reshki poluchaetsja
  v kachestve verojtatnostnogo
prostranstva s meroj, prostranstvo
$\Omega_{1/2}=$ interval $[0,1]$ 
s meroj Lebega.
(Eto byla zadacha v knige Lamperti
   Teorija verojatnosti)
   Kazhdoje chislo v intervale
$x=0,a_1a_2a_3....$ v dvojichnoj sisteme,
   $a_i=0,1$
   oboznahcajet sobytije: shto pri $i$-om brosaniji
vypala $a_i$  ($a_i=0,1$).
  Naprimer verojatnostj sobytija shto v trjokh brosanijakh
v pervom i v tretjem brosanij vypadet  $1$ a vo vtorom
vypadet $0$ eto estj mera mnozhestva $A_{101}$
chisel zapisj kotorykh  imejet vid
$0,101.......$
    i mera etogo mnozhestva konechno ravna $1/8$:
                 $$
   \int_{A_{101}}dF=
     F(0.1011111111...)-F(0.101)=
F(0.11)-F(0.101)=(1/2+1/4)-(1/2+1/8)=1/8
                  $$
   Ja togda reshil postrojit verojatnsonoje
prostranstvo dlia sluchaja,
kogda verojatnostj vypadenija chisla $0$
ravna $q\not=1/2$ 
i verojatnostj vypadenija
chisla $1$ ravna $p=1-q$. 
Tak ja i prishjol k funktsii $F(x)=F_q(x)$ opisannoj vyshe.
   Naprimer verojatnostj sobytija shto v trjokh brosanijakh
v pervom i v tretjem brosanij vypadet  $1$ a vo
vtorom vypadet $0$
eto estj mera mnozhestva $A_{101}$ chisel zapisj
kotorykh imejet vid
$0,101.......$
    i mera etogo mnozhestva konechno ravna:
                 $$
   \int_{A_{101}}dF_q=
     F_q(0.1011111111...)-F_q(0.101)=
    F_q(0.11)-F_q(0.101)=
    F_q(1/2+1/4)-F_q(1/2+1/8)=
    (q+qp)-(q+q^2p)=qp^2\,.
                  $$
Konechno eto prostoje uprazhnenije po teoriji verojatnostej,
no mozhet bytj eto interesno.


          Hovik
\bye

                                     Dr Hovhannes Khudaverdian
                                    The University of Manchester
                                       School of Mathematics
                                   e-mail: khudian@manchester.ac.uk
                                        tel. 0161-2008975



