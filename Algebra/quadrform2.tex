\def \h {{\cal q}}
\def \e {{\bf e}}
\def \R {{\bf R}}
\def \Q {{\bf Q}}
\def \K {{\bf K}}
\def \C {{\bf C}}
\def\a {{\alpha}}

                        \bigskip
                  \centerline {\bf On relation between quadratic and bilinear form}

             \bigskip

             \centerline {\bf ${{\cal x} 1}$  {\bf Standard stuff}}
\bigskip

\def\l {\lambda}
  Bilinear symmetric form $B(x,y)$ on vector space $V$ defines quadratic form
                   $$
         A(x)=B(x,x)
                     \eqno (1.1)
                    $$
which obeys the homogeneity relations:
                   $$
                  A(\l x)=\l^2 \Q(x)\quad
                  \hbox {for arbitrary coefficient $\l$ from the field of scalars}
                  \eqno (1.2)
                   $$
and parallelogram identity:
                        $$
         A(x+y)+A(x-y)\equiv 2A(x)+2A(y)
         \eqno (1.3)
                        $$
Conditions (1.1), (1.2) immediately follow from the relation (1.1)

At what extent  conditions (1.2), (1.3) define bilinear form?

The answer seems to be easy: It follows from (1.2) that
                   $$
      B(x,y)={1\over 2}\left(A(x+y)-A(x)-A(y)\right)
      \eqno (1.4)
                   $$
Question: Is it right that (1.4) defines symmetric bilinear form if conditions (1.2),
(1.3) are obeyed?

\bigskip


     \centerline {\bf ${{\cal x} 2}$  An easy proof of linearity over $\bf Q$}
\medskip

We suppose that field of scalars is the field of characteristic zero,
i,e, field containing $\Q$. We denote the field by scalars by $\K$.
(It can be that $\K=\R$ ois a field of real numbers, or $\K=\C$ is a field
of complex numbers. In some examples $K$ is just arbitrary extension of $\Q$.

\smallskip

One can easy prove the following

{\bf Proposition}

\smallskip

 {\it Let function $A(x)$  obeys homogeneity and parallelogram conditions.
  Then the function $B(x,y)$ defined by (1.4) is symmetric bilinear form over $\bf Q$
 (not over field $\R$ of scalars).}

\smallskip

Consider function $B(x,y)$ defined by (1.4). It is evident that it is symmetric:
                   $$
               B(x,y)=B(y,x)
               \eqno (2.1)
                   $$
{\it Claim}: $B(x,y)$ is biadditive, i.e.
                 $$
                 B(x+x',y)=B(x,y)+B(x',y),\quad B(y,x+x')=B(y,x)+B(y,x')
                 \eqno (2.2)
                 $$
It suffices to prove that $B(x+x',y)=B(x,y)+B(x',y)$.

Proof:  First note that according to (1.2) $A(2x+y)+A(y)=
{1\over2}A(2x+2y)+{1\over 2}A(y)$ $=2A(x+y)+2A(y)$. Hence
$A(2x+y)-A(2x)-Q(y)=2A(x+y)-2A(x)-2A(y)$,i.e.
                      $$
           B(2x,y)=2B(x,y)
           \eqno (2.3)
                      $$
   Now using this relation and condition (1.2) we see that
                    $$
  2B\left({u+v\over 2},y\right)+2B\left({u-v\over 2},y\right)=
  A\left({u+v\over 2}+y\right)+A\left({u-v\over 2}+y\right)-
A\left({u+v\over 2}\right)-A\left({u-v\over 2}\right)-2A\left(y\right)
                    $$
                    $$
=2A\left({u\over 2}+y\right)-2 A\left({u\over 2}\right)-
2 A\left(y\right)=4B({u\over 2},y)=2B(u,y)
\eqno (2.4)
                    $$

\medskip

The statement  of Proposition follows easy from this claim:
For every rational $\l={p\over q}$
                     $$
             B\left(\l x,y\right)=\l B(x,y)\quad {\bf if}\quad \lambda={p\over q}
                  \quad \hbox {is rational number}
                  \eqno (2.5)
                     $$

\smallskip {\bf Remark}  Note that the linearity over $\Q$, i.e. additivity can be
proved under weaker condition  if we put instead homogeneity condition (1.2)
just the condition   $A(0)=0$. T

\medskip

Is it possible to prove the linearity for  arbitrary real $\lambda$?




\medskip

     \centerline {\bf ${{\cal x} 3}$  Continuity interferes}
\medskip

    Suppose that field of scalars is $\R$ (or $\C$).

   Suppose that  function $A(x)$ is continuous (with respect to some norm)
on the vector space $V$.  Then  $B(x,y)$ is continuous also with respect of this norm.
  $B(x,y)$ is linear over $\Q$.
  Hence $B(x,y)$ is linear over $\bf R$ because it is linear over $\Q$.


Note that $A(x)$ obeying condition (1.3) itself defines a norm
in the case if the following additional conditions hold:
                   $$
        A(x)\geq 0, \quad A(x)=0 \,\,{\bf iff}\,\,  x=0
                   \eqno (3.1)
                   $$
and  so called triangle inequality
                     $$
\sqrt {A(x+y)}\leq  \sqrt {A(x)}+\sqrt {A(y)}
                     \eqno (3.2)
                     $$
We see that in this case $A$ is continuous with respect to the norm which is defined
by $A$. (This means by the way that if $V$ is finite-dimensional then it is continuous
with respect to any norm, because all norms on finite dimensional space are equivalent.)


We come to fully algebraic reformulation of continuity condition:

\smallskip

{\bf Theorem} {\it The function $A(x)$ which obeys the homogeneity condition (1.2),
parallelogram condition (1.3) positivity condition (3.1) and triangle condition (3.2)
defines bilinear symmetric form $B(x,y)$}

\smallskip

\medskip

Note that one can prove this Theorem in the manner  "Dieux ex machina"
without using explicitly concept of continuity\footnote {$^*$}{Nevertheless we have to use
the definition of real numbers. here becomes evident the vicious cirvle with continuity conception:
We define real numbers just in the way continuity be acceptable!}.

\medskip {\bf Question}  At what extent it is important the triangle condition?
Triangle inequality and positivity conditions seem to be not adequate at linear stuff.
At what extent they are important?
Does there exist an example of form $A$ which obeys conditions (1.2) and (1.3)
such that corresponding form $B(x,y)$ is not linear?



\medskip

     \centerline {\bf ${{\cal x} 4}$  Attempts to construct counterexample}
\medskip


   Consider first finite-dimensional case.

Let $A(x)$ be form which satisfies conditions (1.2),(1.3).

Let $\{{\bf e}_i\}$  be basis in vector space $V$
Without loss of generality suppose that
               $$
               B({\bf e}_i,{\bf e}_k)=0
               \eqno (4.1)
               $$
If condition (4.1) is not obeyed one can consider
$A'=A-x^B({\bf e}_i,{\bf e}_k)x^k$ which satisfies to conditions (1.2), (1.3), (1.4)

It follows from (1.2), (1.3) that
               $$
   B(\l x,\l y)=\l^2B(x,y)
   \eqno (4.2)
               $$
Put $\l\to \l+\mu$ in (4.2).
We see that for any basis vector ${\bf e}_i$
                       $$
                       B(\lambda {\bf e}_i,\mu {\bf e}_i)=0
                       \eqno (4.3)
                       $$
                       for an arbitrary coefficients $\l,\mu$
and
                         $$
  B(\lambda {\bf e}_i,\mu {\bf e}_k)=-  B(\mu {\bf e}_i,\l {\bf e}_i)
                \eqno (4.4)
                         $$
Let $h_\a$ be a basis in $\bf R$ over $\Q$\footnote{$^*$}
{to construct this basis we need Choice Axiom}


\medskip

     \centerline {\bf ${{\cal x} 4'}$  Attempts to construct counterexample
     in two dimensions.}
\medskip


Consider the case of two-dimensional vector space.
(If finite-dimensional counterexample exists then it exists in two dimesions)

Let ${\bf e,f}$ be basis in $V$. Let $K$ be an extension of the field $\bf Q$
with the basis $\{r_\a\}$. We consider now an extension of $\bf Q$
which is a subfield in $\bf R$.
If $K=\bf R$ then we come to Hammel basis.


Suppose that $B(e,e)=B(f,f)=0=B(e,f)$. Then form is totally characterized by
                $$
                S_{\a\beta}=B(r_\a {\bf e}, r_\beta {\bf f})
                \eqno (4'.1)
                $$
$S$ is antisymmetric according to  (4.4).

Express quadratic form in terms of $S$:

Every vector $x\in V$ is nothing but the pair of vectors $a,b$
where $\{a^\a,b^\beta \}$ are expansion components:
                   $$
   x=a^\a r_\a {\bf e}+b^\beta r_\beta {\bf f}
   \eqno (4'.2)
     $$

For every vector $x$
                $$
       A(x)=a^\a S_{\a\beta}b^\beta=S(a,b)
       \eqno (4'.3)
                $$

The condition of homogeneity restricted  for $\l\in K$ will give that:
            $$
A(\l x)=a^\a A_\a^\gamma(\l)S_{\gamma\delta}A_\beta^\delta (\l) b^\beta=
       \l^2 A(x)=\l^2  a^\a S_{\a\beta}b^\beta
       \eqno (4'.4)
            $$
To find counterexample we have to find non-trivial bilinear (over $\Q$) antisymmetric $S$
form
which obeys the condition:
               $$
      S(\l a, \l b)=\l^2 S(a,b),\quad \hbox{for all $\l\in \K$}
      \eqno (4'.5)
               $$
To be more exact we have to rewrite $\l a$ as $A_\l a$,
where $A(\l)$ is linear operator in $K$ (over $\bf Q$) corresponding to the multiplication
on $\lambda$. It is easy to make the following conclusion:

{\it There is no counterexample in finite extension of $\Q$, i.e. bilinear form
 $B$ is linear over any finite field extension}.
 This follows from the fact that to multiplication on $\l$ corresponds
 transformation $S\to A^+ SA$ of matrix $S$

 It is very instructive to consider the case of extension of degree $3$.


\medskip

     \centerline {\bf ${{\cal x} 5}$  Counterexample?}
\medskip

In this section I try to construct counterexample for the case when $\K$
is simple transcendent extension.

I still can say nothing about extension by $\R$.

\def\Q {{\bf Q}}

Consider simple transcendental extension $K=\Q(t)=\Q(\pi)$

Then $V$ is two-dimensional vector space over the field $K=\Q(t)$.
Elements of $\Q(t)$ can be viewed as fractions of polynomials.
Elements of $V$ as columns with two entries.
  Consider
                 $$
         S(f(t),g(t))=f(t)g'(t)-f'(t)g(t),
          \eqno (5.1)
                 $$
where $f(t),g(t)\in \Q(t)$ are fractions, $h'(t)$ is derivative.

It is evident that  $S(f,g)$ is the antisymmetric bilinear (over $\Q$).
One can easy to see that form $S(f,g)$
(obeys the condition   (4'5):
          $$
   S(hf,hg)=hf(hg)'-(hf)'hg=h^2(fg'-f'g)=h^2S(f,g)\quad \hbox{for every $h(t)\in \Q(t)$}
          $$
Hence it defines quadratic form on the space $V$:
               $$
    \hbox {for every} \quad {\bf x}=\pmatrix {f(t)\cr g(t)\cr}\in V\quad
                A({\bf x})=f(t)g'(t)-f'(t)g(t)
               $$
Form $A(\bf x)$ is defined over tw0-dimensional vectors space $V$ over $K=\Q(t)$.
It takes values in the field $K=\Q(t)$. It obeys homogeneity condition (1.2):
              $$
              A(h{\bf x})=h^2(f(t)g'(t)-f'(t)g(t))=h^2(t)A({\bf x})
              $$
and it obeys the parallelogram condition (1.3)


The quadratic form $A$ is not identical zero in spite of the fact that
it is equal to zero on the vectors
${\bf e}=\pmatrix {1\cr 0\cr}$, ${\bf v}=\pmatrix {0\cr 1\cr}$
and ${\bf e}+{\bf f}=\pmatrix {1\cr 1\cr}$. Its corresponding
biadditive form $B(x,y)$ defined by (1.4) is equal to
                   $$
       B(x,y)=B\left(\pmatrix {f(t)\cr g(t)\cr},\pmatrix {h(t)\cr w(t)\cr}\right)=
       {1\over 2}\left(A(x+y)-A(x)-A(y)\right)=
                        $$
                        $$
       {1\over 2}\left((f+h)(g+w)'-(f+h)'(g+w)-(fg'-f'g)-(hw'-h'w)\right)=
                   $$
                    $$
{1\over 2}(fg'-f'g)+{1\over 2}(hw'-h'w)
                    $$
This is symmetric bilinear over $\Q$ form which produces $A$ by (1.1) but $B$
is not linear over $\K=\Q(t)$: $B({\bf e},{\bf e})=B({\bf f},{\bf f})=B({\bf e},{\bf f})=0$
but it is not identical zero.



\medskip

     \centerline {\bf ${{\cal x} 6}$  Are there non-trivial generalisations?}
\medskip

The constructions above has trivial generalisation:

E.g. Let $\K$ be any finite transcendent extension  $\K=\Q(t_1,\dots,t_n)$
where all $t_1,dots,t_n$ are undeterminates.   Let
                $$
                D=\sum A^i(t){\partial\over \partial t_i}
                $$
be an arbitrary vector field, (coefficients $A_i(t)$) belong to $\Q(t_1,\dots,t_n)$.
Then
                   $$
    A=fDg-Dfg
                   $$
obviously is quadratic form which has  properties similar to form considered in previous
paragraph.

But how to reach field $\R$. Weather there exists a field $L$ such that
$\R=L(t)?$ In this case one can construct counterexample in the manner of previous paragraph.
And this will be form on $\R$

 \bye
