
\magnification=1200 


\baselineskip=14pt
\def\vare {\varepsilon}
\def\A {{\bf A}}
\def\t {\tilde}
\def\a {\alpha}
\def\K {{\bf K}}
\def\N {{\bf N}}
\def\w {\omega}
\def\s {{\sigma}}
\def\S {{\Sigma}}
\def\s {{\sigma}}
\def\p{\partial}
\def\vare{{\varepsilon}}
\def\Q {{\bf Q}}
\def\D {{\cal D}}
\def\G {{\Gamma}}
\def\C {{\bf C}}
\def\L {{\cal L}}
\def\Z {{\bf Z}}
\def\U  {{\cal U}}
\def\H {{\cal H}}
\def\R  {{\bf R}}
\def\S  {{\bf S}}
\def\E  {{\bf E}}
\def\l {\lambda}
\def\M {{\cal M}}
\def\degree {{\bf {\rm degree}\,\,}}
\def \finish {${\,\,\vrule height1mm depth2mm width 8pt}$}
\def \m {\medskip}
\def\p {\partial}
\def\r {{\bf r}}
\def\pt {{\bf pt}}
\def\v {{\bf v}}
\def\n {{\bf n}}
\def\t {{\bf t}}
\def\b {{\bf b}}
\def\c {{\bf c }}
\def\e{{\bf e}}
\def\k{{\bf k}}
\def\l{{\bf l}}
\def\ac {{\bf a}}
\def \X   {{\bf X}}
\def \Y   {{\bf Y}}
\def \x   {{\bf x}}
\def \y   {{\bf y}}
\def \G{{\cal G}}
\def\ss  {\sigma_{\rm sph}}
\def\grad {{\rm grad\,}}


{\it 18 January 2018}


\centerline {Two formulae for Lie groups}

I will recall here calculations of
      $$
   \Sigma_1=e^{-A}Xe^{A}=Ad_A(X)
          \eqno (1)
      $$
and calculation of
      $$
   \Sigma_2=e^{-A}e^{A+\vare X}\,,\qquad {\rm for}\,\, \vare^2=0
          \eqno (2)
      $$
Both formulae are very useful.
(E.g. the second one is inevitable if we would like to perform
explicit calculations with left invariant vetor fields.)  
This is standard  to calculate
these sums using differential equations. Here I will  
calculate just by brute force
combinatorial calculations.

\bigskip

 The first formula comes naturally from 
differential equation:
to calculate $\Sigma_1$ in (1) consider a function 
  $$
  S_X(t)=e^{-tA}Xe^{tA}\,.
   $$ 
This function is a solution
of differential equation
      $$
     \cases
     {{dS\over dt}=-[A,S(t)]=-ad_A S(t)\cr
    S(t)_{t=0}=X\cr}
      $$
Solving this equation we have that
         $$
  S(t)=X-\int_0^t [A,S(\tau)]d\tau\,.
         $$
Writing this integral recurcsively we come
to perturbation expansion or to
     $$
    S(t)=e^{-t ad_A}X\,.
     $$

Now combinatorial ({\tt brute force}) solution:
        $$
  \Sigma_1=e^{-A}Xe^{A}=\sum_{n,m}{(-1)^n A^n X A^m\over n!m!}=
  \sum_n\left(\sum_{k=0}^n
  {(-1)^k A^k X A^{n-k}\over k!(n-k)!}\right)=
       $$
       $$
  \sum_n{1\over n!}
     \left(\sum_{k=0}^n
               (-1)^kC^k_n
        A^k X A^{n-k}\right)\,.
              $$
One can see that polynomials 
      $$
 {\cal D}_n(A,X)=\sum_{k=0}^n(-1)^kC^k_nA^nXA^{n-k}
       $$
belong to Lee algebra ${\cal G}(A,X)$,
(they are up to a sign so called Dynkin polynomials):
         $$
      D_n(A,X)=XA^n-nAXA^{n-1}+\dots+(-1)^nA^nX=
         $$
          $$
      (-1)^n[A,\dots,[A,X
       \underbrace{]\dots]}_{n \hbox{times}}=
       ad_A^n X\,.
    \eqno (Dynkin)
          $$ 
        $$
{\cal D}_1=-AX+XA=-[A,X]\,,
{\cal D}_2=XA^2-2AXA+A^2X=[A,[A,X]]
      $$
     $$
{\cal D}_3=XA^3-3AXA^2+3A^2XA-A^3X=-[A,[A,[A,X]]]\,\quad\,,
      $$
and so on.

We come to
       $$
\Sigma_1=  \sum_n{1\over n!}
     \left(\sum_{k=0}^n
               (-1)^kC^k_n
        A^k X A^{n-k}\right)=
        \sum_n{ (-1)^nad_A^n X\over n!}=
         e^{-ad_A}X=
         $$
    $$
1-[A,X]+{[A,[A,X]]\over 2}-{[A,[A,[A,X]]]\over 6}+\dots
     $$
    \bigskip

\medskip

Much more funny  formula (2):


Differential equation. Consider function
     $$
   S(t)=e^{-tA}e^{t(A+\vare X)}
     $$
it obeys equation:
       $$  \cases
     {{dS\over dt}=e^{-tA}\vare X e^{tA} S(t)\cr
    S(t)_{t=0}=X\cr}
       $$
  Brute force calculations:
         $$
    \Sigma_2= e^{-A}e^{A+\vare X}=
        $$
        $$
       \sum {(-1)^nA^n\over n!}
   \left(\sum {A^m\over m!}+
 \sum
    \overbrace
          {
 {A^{m-1}X+A^{m-2}XA+A^{m-3}XA^2+\dots+XA^{m-1}
\over m!}
    }^{\hbox {$m$ terms}}
\right)=
         $$
         $$
  1+\sum_r\left(\sum_{p+q=r}
   {(-1)^pA^p\over p!}
    \overbrace
       {
  {A^{r-q-1}X+A^{r-q-2}XA+A^{r-q-3}XA^2+
   \dots+XA^{r-q-1}\over (r-q)!}
      }^{\hbox {$(r-q)$ terms}}
\right)=
         $$
     $$
1+\sum_{m,n}
\left(
\sum_{p}{(-1)^pA^p\over p!}
\right)
{A^mXA^n\over (m+n+1)!}
        $$
Here changing $p+m\to m$ we come to
         $$
\Sigma_2=1+\sum_{m,n,p=0,\dots,m}
{(-1)^pA^mXA^n\over p!((m-p)+n+1)!}=
\sum_{m,n}
t_{mn}{A^mXA^n\over (m+n+1)!}\,,
        $$
where
   $$
   t_{mn}=
\sum_{p=0}^m{(-1)^p\over p!(m+n+1-p)!}\,.
   $$

\smallskip {\it Observation}
       $$
   t_{mn}=(-1)^mC^m_{m+n}\,.
       $$
It follows from this observation that
from formula ($Dynkin$) for Dynkin polynomials, that
     $$
\Sigma_2=1+
\sum_{m,n}
t_{mn}{A^mXA^n\over (m+n+1)!}=
1+\sum_{N}{1\over N+1}
     \underbrace
           {
   \left(
       \sum_{k=0}^N
     (-1)^kC^k_N A^kXA^{N-k}
    \right)
        }_{\hbox 
 {Dynkin's polynomial ${\cal D}_N$}}=
     $$
      $$
   1+\sum_{N=0}^\infty{(-1)^N ad_A^N X\over (N+1)!}=
       $$    $$
1+X-{1\over 2}[A,X]+
       {1\over 6}\left[
    A,\left[A,X\right]
          \right]-
       {1\over 24}
        \left[A,\left[
    A,\left[A,X\right]
          \right]\right]- 
       {1\over 120}
      \left[A,
        \left[A,\left[
    A,\left[A,X\right]
          \right]\right]
\right]+\dots=\,.
    $$

It remains to prove {\tt Observation}.

  {\tt Proof} 
   We will use Pascal's tree identity:
            $$
     C^k_n+C^{k+1}_n=C^{k+1}_{n+1}\,.
            $$   
Then    
$$
   t_{mn}=
\sum_{p=0}^m{(-1)^p\over p!(m+n+1-p)!}=
        $$
        $$
  C^0_{m+n+1}-C^1_{m+n+1}+C^2_{m+n+1}-C^3_{m+n+1}+\dots+
      (-1)^mC^m_{m+n+1}=
   $$
    $$
[C^0_{m+n}]-[C^0_{m+n}+C^1_{m+n}]+[C^1_{m+n}+C^2_{m+n}]
-[C^2_{m+n}+C^3_{m+n}]+\dots+
   (-1)^m[C^{m-1}_{m+n}+C^m_{m+n}]=
           $$
           $$=(-1)^mC^m_{m+n}
    \hbox{\finish}$$

  
\bye
