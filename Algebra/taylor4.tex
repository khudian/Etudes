

\magnification=1200
\baselineskip=14pt
\def\vare {\varepsilon}
\def\A {{\bf A}}
\def\t {\tilde}
\def\a {\alpha}
\def\K {{\bf K}}
\def\N {{\bf N}}
\def\V {{\cal V}}
\def\s {{\sigma}}
\def\S {{\Sigma}}
\def\s {{\sigma}}
\def\p{\partial}
\def\vare{{\varepsilon}}
\def\Q {{\bf Q}}
\def\D {{\cal D}}
\def\G {{\Gamma}}
\def\C {{\bf C}}
\def\M {{\cal M}}
\def\Z {{\bf Z}}
\def\U  {{\cal U}}
\def\H {{\cal H}}
\def\R  {{\bf R}}
\def\S  {{\bf S}}
\def\E  {{\bf E}}
\def\l {\lambda}
\def\ll {{\bf l}}
\def\degree {{\bf {\rm degree}\,\,}}
\def \finish {${\,\,\vrule height1mm depth2mm width 8pt}$}
\def \m {\medskip}
\def\p {\partial}
\def\r {{\bf r}}
\def\pt {{\bf p}}
\def\v {{\bf v}}
\def\n {{\bf n}}
\def\t {{\bf t}}
\def\b {{\bf b}}
\def\c {{\bf c }}
\def\e{{\bf e}}
\def\ac {{\bf a}}
\def \X   {{\bf X}}
\def \Y   {{\bf Y}}
\def \x   {{\bf x}}
\def \y   {{\bf y}}
\def \G{{\cal G}}
\def\w {{\omega}}
\def \Tr  {{\rm Tr\,}}
\def\V {{\cal V}}
\def\s {{d\over dt}}
\def \finish {${\,\,\vrule height1mm depth2mm width 8pt}$}

% I began this file in 22 December  2018

{\tt 22 December 2018}

\centerline {\bf On Taylor identity}
A  standard proof of Taylor Theorem 
for smooth (infinitely differentiable) function,
     $$
f(x)=\sum_{k=0}^n f^{(k)}(x_0){(x-x_0)^k\over k!)}+
     O(x^{n+1})
    \eqno (1)
       $$
contains  a `nasty' part related with 
estimation of residul term
$O(x^{n+1})$.   There is an elegant  proof  of Taylor 
Theorem  which is based on the identity  
              $$
f(x)=\sum_{k=0}^n f^{(k)}(x_0){((x-x_0)^k\over k!)}+
{1\over n!}\int_{x_0}^x {d^{n+1}\over dt^{n+1}}f(t)(x-t)^ndt
        \eqno (2)
            $$
This identity immediately leads to (1).


   Few weeks  ago my friend
Sasha Karabegov  acquinted  me with the problems
suggested on the PUTNAM competition 
in USA universities
\footnote{$^{1)}$}
{see https://kskedlaya.org/putnam-archive/2018.pdf}.
  Sasha was local organiser of this
competition  in his University.  He has suggested 
the beautiful solutions of some questions,
and  in particular
of the question A5 on this competition
\footnote{$^{2)}$}
{see the fourth solution of this question
in  https://kskedlaya.org/putnam-archive/2018s.pdf
or the Appendix  to this text}.
   His elegant proof is based on the identity (2).
In fact in this proof he deduces 
in elementary way
the following Theorem:

\medskip

{\it Let $f(x)$ be a smooth 
function on $\bf R$ such that
this function and  
 its all derivatives at all the points 
take non-negative values.
Then the condition that function $f$ vanishes 
at arbitrary pointmplies that it vanishes at all the points:
              $$
\forall x, \forall n   f^{(n)}(x)\geq 0\, {\rm and}\,\,
    f(x_0)=0\Rightarrow f(x)\equiv 0\,.
          \eqno (3)
              $$
}

This statement is related with the Bernstein's 
Theorem on monotone functions\footnote{$^{3)}$}
 {see the second solution
of the question in 
 https://kskedlaya.org/putnam-archive/2018s.pdf}.
 


\bigskip
In what follows I expalin  the identity (1)
and reproduce Karabegov's proof.

\medskip

\centerline {\bf Taylor identity}

{\tt I have known  this identity  ` for hundred  years'. 
Karabegov's proof makes me to realise that
this is really very effective.}

\medskip

   Let $f=f(x)$ be a smooth function. Then
          integrating by parts we come to
                $$
f(x)=
f(0)+\int_0^x {d f(t))\over dt}dt=
f(0)+\underbrace{\int_0^x {d f(t))\over dt}\cdot 1 dt}
        _{\hbox{I}}=
         $$
          $$
f(0)+\int_0^x \s f(t)\s\left(t-x\right)dt=
f(0)+\s f(t)(t-x)\big\vert^x_0-
   \int_0^x {d^2\over dt^2} f(t)\left(t-x\right)dt=
            $$
           $$
f(0)+f'(0)x+
   \underbrace
            {
 \int_0^x {d^2\over dt^2} f(t)\left(x-t\right)dt
           }_{\hbox{II}}=
            $$
            $$
f(0)+f'(0)x-    {1\over2}\int_0^x {d^2\over dt^2} f(t)
    \s\left(\left(t-x\right)^2\right)dt
           =
           $$
        $$
f(0)+f'(0)x
-   {1\over 2} {d^2\over dt^2} f(t)
    \left(t-x\right)^2\big\vert^x_0
   +
   {1\over 2}\int_0^x {d^3\over dt^3} f(t)
    \left(t-x\right)^2 dt=
          $$
            $$
f(0)+f'(0)x
+f^{''}(0) {x^2\over 2}  
   +
             {1\over 2}
      \underbrace
         {
\int_0^x {d^3\over dt^3} f(t)
    \left(x-t\right)^2 dt
        }_{\hbox{III}}=
          $$
          $$
f(0)+f'(0)x
+f^{''}(0){x^2\over 2}
-    {1\over 6 }\int_0^x {d^3\over dt^3} f(t)
    \s\left(\left(x-t\right)^3\right)dt
           =
          $$
            $$
f(0)+f'(0)x
+f^{''}(0){x^2\over 2}
+   {1\over 6} {d^3\over dt^3} f(t)
    \left(t-x\right)^3\big\vert^x_0
   -
   {1\over 6}\int_0^x {d^4\over dt^4} f(t)
    \left(x-t\right)^3 dt=
          $$
            $$
f(0)+f'(0)x
+f^{''}(0) {x^2\over 2}  
+f^{'''}(0) {x^3\over 6}  
   +
             {1\over 6}
      \underbrace
         {
\int_0^x {d^4\over dt^4} f(t)
    \left(x-t\right)^3 dt
        }_{\hbox{IV}}=
          $$
           $$
f(0)+f'(0)x
+f^{''}(0){x^2\over 2}
+f^{'''}(0){x^3\over 6}
-    {1\over 24 }\int_0^x {d^4\over dt^4} f(t)
    \s\left(\left(t-x\right)^4\right)dt
           =
          $$
            $$
f(0)+f'(0)x
+f^{''}(0){x^2\over 2}
+f^{'''}(0){x^3\over 6}
-   {1\over 24} {d^4\over dt^4} f(t)
    \left(t-x\right)^4\big\vert^x_0
   +
   {1\over 24}\int_0^x {d^5\over dt^5} f(t)
    \left(t-x\right)^4 dt=
          $$
            $$
f(0)+f'(0)x
+f^{''}(0) {x^2\over 2}  
+f^{'''}(0) {x^3\over 6}  
+f^{''''}(0) {x^4\over 24}  
   +
             {1\over 24}
      \underbrace
         {
\int_0^x {d^5\over dt^5} f(t)
    \left(x-t\right)^4 dt
        }_{\hbox{IV}}=
          $$
and so on:
          $$
\ldots=\sum_{k=1}^n f^{(k)}(0){x^k\over k!}+
         {1\over n!}\int_0^x 
{d^{n+1}\over dt^{n+1}}f(t)(x-t)^ndt
          $$

\bigskip


\centerline {\bf Appendix }
{\tt Here I reproduce the Karabegov's proof
of the Theorem (3).}.

Shortly speaking his proof is the following:
if $f(x_0)=0$ then for all $x\leq x_0$,
 $f(x)=0$ also 
since $f'(x)\geq 0$ for $x\leq x_0$.  Thus all derivatives
of the smooth function $f$ vanish at the point $x_0$. Hence
  it follows from the identity (1) that  for all $x$
   and for all $n$,
            $$
f(x)=
{1\over n!}\int_{x_0}^x 
{d^{n+1}\over dt^{n+1}}f(t)(x-t)^ndt\,.
   \eqno (A1)
            $$
for an arbitary $n$.
Hence integrating this identity we see that
for every $x_1$ and for every $n$
         $$
\int_{x_0}^{x_1} f(x)dx=
\int_{x_0}^{x_1} dx\left({1\over n!}\int_{x_0}^x 
f^{(n+1)}(t)(x-t)^ndt\right)=
         $$
         $$
{1\over n!}\int_{x_0}^{x_1} dt
\left(
   \int_{t}^x 
f^{(n+1)}(t)(x-t)^ndx\right)=
{1\over (n+1)!}\int_{x_0}^{x_1}
f^{(n+1)}(t)(x_1-t)^{n+1}\,.
    \eqno (A2)
         $$
Choose an arbitrary $x_1>x_0$. Then  
it follows from equations 
(A2) and
(A1) that
       $$
\int_{x_0}^{x_1} f(x)dx=
{1\over (n+1)!}\int_{x_0}^{x_1}
f^{(n+1)}(t)(x_1-t)^{n+1}\leq
         $$
        $$
{x_1-x_0\over n+1}
\left(
 {1\over n!}\int_{x_0}^{x_1}
f^{(n+1)}(t)(x_1-t)^{n}
\right)=
{x_1-x_0\over n+1}f(x_1)\Rightarrow f(x_1)=0{\hbox{\finish}}
          $$
since this inequality holds or arbitrary $n$.

\bye


