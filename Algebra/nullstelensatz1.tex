\def\vare {\varepsilon}
\def\A {{\bf A}}
\def\t {\tilde}
\def\a {\alpha}
\def\K {{\bf K}}
\def\N {{\bf N}}
\def\V {{\cal V}}
\def\s {{\sigma}}
\def\S {{\Sigma}}
\def\s {{\sigma}}
\def\p{\partial}
\def\vare{{\varepsilon}}
\def\Q {{\bf Q}}
\def\D {{\cal D}}
\def\G {{\Gamma}}
\def\C {{\bf C}}
\def\M {{\cal M}}
\def\Z {{\bf Z}}
\def\U  {{\cal U}}
\def\H {{\cal H}}
\def\R  {{\bf K}}
\def\E  {{\bf E}}
\def\l {\lambda}
\def\degree {{\bf {\rm degree}\,\,}}
\def \finish {${\,\,\vrule height1mm depth2mm width 8pt}$}
\def \m {\medskip}
\def\p {\partial}
\def\r {{\bf r}}
\def\v {{\bf v}}
\def\n {{\bf n}}
\def\t {{\bf t}}
\def\b {{\bf b}}
\def\c {{\bf c }}
\def\e{{\bf e}}
\def\ac {{\bf a}}
\def \X   {{\bf X}}
\def \Y   {{\bf Y}}
\def \x   {{\bf x}}
\def \y   {{\bf y}}
\def \G{{\cal G}}

\centerline {\bf On a simple proof of Nullstelensatz}

\bigskip


  Vladimir Dotzenko wrote the article [1] where he described a simple proof of
  Nullstellensatz for the field $\C$ of complex numbers.  As it is claimed in this article,
  this proof is "a part of mathematical folklore".
  (The standard proof of this Theorem possesses a "difficult part". (See e.g.
  the excellent book "Algebraic geometry for pedestrians" of Miles Read [2].))



  I would like to retell this proof, paying little bit more attention on
  its crucial non-standard part.


  \m

   {\bf Theorem 1} Let $M=\{f_1,f_2,\dots,f_k\}$ be a set of polynomials in the ring
   of polynomials of $n$ complex variables. 
   


 Then   or these polynomials have common root or there exist polynomials
    $g_1,\dots,g_n$ (over complex numbers) such that
    $f_1g_1+\dots+f_kg_k\equiv 1$.

In other words an ideal $I$ generated by polynomials $\{f_1,f_2,\dots,f_k\}$
in the ring $\R=\C[x_1,\dots,x_n]$ of polynomials on $\C^n$
equals to $\R$ if  these polynomials have not common root.

\m

It is famous Hilbert's {\it  Nullstelensatz.}


One can consider another formulation of this theorem:

\m

{\bf Theorem $\bf 1'$}  Let $M=\{f_1,f_2,\dots,f_k\}$ be a set of polynomials over complex numbers.
  Then if for an arbitrary polynomial $F\in \R$ set of common  roots of polynomials
   $\{f_1,f_2,\dots,f_k\}$ belongs to the set of roots of polynomial $F$:
                    $$
         f_1(x_0)=f_2(x_0)=\dots=f_n(x_0)\Rightarrow  F(x_0)=0\,,
                    $$
   then there exists natural $m$ such that $F^m$ belongs to the ideal $I=(f_1,\dots,f_n)$.

\m

This Theorem is equivalent to previous one (see any standard textbook)

The proof of the Theorem 1 follows from the following

\m

{\bf Lemma 1}  Let $\R:\C$ be a field extension of the field $\C$. Let $\{a_i\}$ ($i=1,2,3,\dots$) be a set of
elements of $\R$ such that the span of these  elements over $\C$ is $\R$, i.e. for an arbitrary $x\in \R$
there exists a finite set $\{a_{i_1},\dots,a_{i_p}\}$ such that
$x=\lambda_1a_{i_1}+\lambda_2a_{i_2}+\dots+\lambda_pa_{i_p}$.
Then $\K=\C$.

  In other words
 an arbitrary field $\bf K$ which is an extension of the field $\C$ of complex numbers coincide
 with $\C$ {\it or} degree of the extension is uncountable\footnote{$^*$}{In the paper [2] author gives this second formulation
equivalent formulation of the lemma:

\m

We  prefer the  first formulation above , since the case
when algebraic dimension is more than finite  could be little bit confusing for a reader.}.

\m

Theorem follows from the lemma by means of the following standard textbook considerations:



{\it Proof (of Theorem 1')}.
Let $I=(f_1,\dots,f_n)$ be an ideal generated by the polynomials $\{f_1,\dots,f_n\}$.

Suppose $I\not=\C[x_1,\dots,x_n]$.
 Consider the maximal ideal $J$ ($J\not=\R$) in $\C[x_1,\dots,x_n]$ which contains $I$
and a field ${\bf L}=\C[x_1,\dots,x_n]\backslash J$.

Consider the countable set of polynomials  (e.g. polynomials $\{x_1^{m_1}x_2^{m_2}\dots x_k^{m_k}\}$
which span  the ring $\C[x_1,\dots,x_n]$. Hence equivalence classes of these polynomials
span the field  ${\bf L}=\C[x_1,\dots,x_n]\backslash J$. It follows from the lemma that
  the field $\bf L$ is isomorphic to the field $\C$ of complex numbers. Let $a_i\in \C$ be the image
   of equivalence class $[x_i]$ of monomial $x_i$. Since $f_i\in J$ image of an equivalence class of
    polynomial $f_i$ is equal to zero. Hence the point $x_i=a_i$ is a common root of polynomials $\{f_i\}$.
  $x_i$. Contradiction.

  \m

  Now we go to the central part of this topic, we prove the lemma.

  \m

  {\it Proof of the lemma}

   Let field extension $\R:\C$ be spanned by the countable set of vectors $\{a_i\}$ ($i=1,2,3,\dots$).

   Prove that for arbitrary $\theta\in \R$,   $\theta\in \C$.

 Consider the following uncountable set of elements in $\R$:
                $$
             {\cal M}=\left\{{1\over \theta-z}\right\}\,,
                $$
      where the set $z$ runs over all complex numbers except a number $\theta$ (if $\theta\in \C$).
       (If $\theta\in \C$ we have nothing to prove but we consider this case too.)

\m

    {\it Claim}: There exists a finite subset of elements in $\M$
    which are linear dependent elements (over $\C$.)

\m

  This claim implies the lemma. Indeed let $\left\{{1\over \theta-z_i}\right\}$ be a  finite subset 
  of linear dependant vectors, i.e.
                       $$
           \sum_i {c_i\over \theta-z_i}=0\,,
                             $$
            where all coefficients $\{c_i\}$ are complex numbers 
            and at least one of the complex numbers $c_i$ is not equal to zero.
  This is an algebraic equation on $\theta$ over algebraically closed field $\C$.
 Hence $\theta\in \C$.

\smallskip

                It remains to prove the claim.

                \m

   Denote by $\R_r$ the span of the first $r$ vectors $\{a_1,a_2,\dots, a_r\}$.
   We have a sequence of $\{\R_k\}$ of finite-dimensional space $\R_1\subseteq \R_2\subseteq\dots \R_i\subseteq \K_{i+1}\subseteq ...$ and
$\cup_{r=1}^\infty \R_r=\R$.

         Consider the subsets ${\cal M}_k={\cal M} \cap \R_r$.
        At least one of these subsets, say $\M_k$ possesses infinite number of elements 
        (in fact incountable number of elements)
        since the set $\M=\cup_k \M_k$ is uncountable. The infinite subset $\M_k$ belongs to 
        finite-dimensional space $\R_k$.
         We see that there exists
    $N+1$ linear dependent elements $\left\{{1\over \theta-z_i}\right\}$ ($i=1,z\dots,N+1$)
    in $\R_k$ ($N$ is dimension of the space $\R_k$). Claim is proved.





\centerline{\bf References}


[1] V. Dotzenko  "On a proof of Hilbert Nullstelensatz Theorem",
 {\it Matematicheskoje prosveshenije, 3, v6, pp.116---118, (2002)} (in Russian)

[2] 

\bye
