

\magnification=1200 \baselineskip=14pt
\def\vare {\varepsilon}
\def\A {{\bf A}}
\def\t {\tilde}
\def\a {\alpha}
\def\K {{\bf K}}
\def\N {{\bf N}}
\def\V {{\cal V}}
\def\s {{\sigma}}
\def\S {{\Sigma}}
\def\s {{\sigma}}
\def\p{\partial}
\def\vare{{\varepsilon}}
\def\Q {{\bf Q}}
\def\D {{\cal D}}
\def\G {{\Gamma}}
\def\C {{\bf C}}
\def\M {{\cal M}}
\def\Z {{\bf Z}}
\def\U  {{\cal U}}
\def\H {{\cal H}}
\def\R  {{\bf R}}
\def\E  {{\bf E}}
\def\l {\lambda}
\def\degree {{\bf {\rm degree}\,\,}}
\def \finish {${\,\,\vrule height1mm depth2mm width 8pt}$}
\def \m {\medskip}
\def\p {\partial}
\def\r {{\bf r}}
\def\v {{\bf v}}
\def\n {{\bf n}}
\def\t {{\bf t}}
\def\b {{\bf b}}
\def\e{{\bf e}}
\def\ac {{\bf a}}
\def \X   {{\bf X}}
\def \Y   {{\bf Y}}
\def \x   {{\bf x}}
\def \y   {{\bf y}}


{\tt Letter of Sasha Borovik, 24 November 2018}

{\tt on problem of 101 coins}

\m


  So everything boils down to a statement about
natural numbers:
 
{\it




We are given 101 natural numbers, and after removal of any one of them
from the list the remaining numbers can be redistributed in two groups of 50
numbers each in such a way that the sum of numbers in each group is the
same. Prove that  all  numbers are equal.

}

{\sl Here is a re-write of the solution suitable for junior
school kids.}



After removal of any of numbers the sum of the remaining
numbers is
even. Hence all numbers have the same parity: either they
are all
even, or they are all odd. In the first case, divide all
numbers by 2.
In the second case, subtract 1 from all numbers. In both
cases we get
the same problem, but with smaller numbers, and this
continues until
one of the numbers is 0. If numbers are all zero, we are
done. Assume
that there is a non-zero number left. Notice that we can
no longer
make further subtractions of 1, but only divisions by 2,
and we
continue them until some other number becomes odd. At this
point,
either removal of 0, or removal of that odd number creates
100 numbers
with the odd sum, a contradiction.


{\sl Ideologically this is a hidden calculation in base 2
arithmetic (this
is why 2-adic numbers appeared  in  Karabegov's
arithmetic (this
is why 2-adic numbers appeared in  Karabegov's
 beautiful  solution),
alike the Russian Peasants' Multiplication or  Russian
Peasants'
Greatest Common Divisor.
 
 Pedagogically, this is a nice example of the use of
 abstraction: in
 the original formulation with coins, the problem is next
 to imposible
 for a schoolchild -- you cannot subtract 1 from the value
 of a coin,
 it is called debasement of the coinage and, I am afraid,
 is likely to
 remain a criminal offence in this country. It helps to
 show  to a
 child that he/she can forget about coins 
 and get freedom of action --
 the child will be likely to look for this freedom in other
 problems as well.
 
  Historically, in my previous incarnation I had a chance to
see, and
participate in, creation of mathematical olympiad problems
-- this was
the same kind of process as the one demonstrated to us by
Hovik. Thanks to Hovik, I had a chance to revive sweet memories
of my younger
years.
Alas, nostalgia is not what it used to be ...
 

}


Best wishes -- Sasha
\bye
