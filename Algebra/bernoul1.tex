\def\p{\partial}
\def\t {\tilde}
\def \m {\medskip}
\def\degree {{\bf {\rm degree}\,\,}}
\def \finish {${\,\,\vrule height1mm depth2mm width 8pt}$}





\def\a {\alpha}
\def\vare{{\varepsilon}}
\def\l {\lambda}
\def\s {{\sigma}}

\def\G {{\Gamma}}

\def\A {{\bf A}}
\def\C {{\bf C}}
\def\E  {{\bf E}}
\def\K {{\bf K}}
\def\N {{\bf N}}
\def\Q {{\bf Q}}
\def\R  {{\bf R}}
\def\V {{\cal V}}
\def \X   {{\bf X}}
\def \Y   {{\bf Y}}
\def\Z {{\bf Z}}



\def\ac {{\bf a}}
\def\e{{\bf e}}
\def\f {{\bf f}}
\def\n {{\bf n}}
\def\r {{\bf r}}
\def\v {{\bf v}}
\def \x   {{\bf x}}
\def \y   {{\bf y}}


\def\pt {{\bf pt}}



\centerline {\bf Bernoulli numbers, Bernoulli polynomials...}

\m

{\it We know that Bernoulli numbers appear everywhere in mathematics.  I will consider here
 two topics: basic formulae for integrals leading to Euler-Maclaurin  type formulae
  and Fourier transformations formulae for calculations
 of $\zeta$ function at even points. In both cases naturally appear the same series of polynomials...

 Shortly:  Bernoulli Polynomials are polynomials $\{B_n\}$ of the degree $n$ which are convenient for
  integration by part as well as polynomials $\{x^n\}$. They also are orthogonal to constant function. 
  Bernoulli numbers are values of Bernoulli polynomials in boundary points.
}


\bigskip

   \centerline {\bf ${\cal  x} 1$ Integral and area of trapezium }
   \m
   Everybody who heard about integral knows that $\int_a^bf(t)dt$ equals approximately to the area of trapezoid
   with altitude $(b-a)$ and parallel sides equal to the values of the function $f$ at the points
   $a,b$:
                         $$
                         \int_a^bf(t)dt\,\approx\,(b-a)\cdot {f(a)+f(b)\over 2}
                         \eqno (1)
                         $$
   Very simple question: How this formula follows from the formula of integration by parts
     ($\int f(x)dx=xf(x)-\dots$)?
     (I was surprised realising that I never asked myself this simple question before.)
     \m

     {\sl Answer}:  Instead $\int f(x)dx=xf(x)-\int xf'(x)dx$ take
     $\int f(x)dx=(x+c)f(x)-\int (x+c)f'(x)dx$ putting $x+c$ instead $x$, where $c$ is an arbitrary constant.
     Thus we come to
                        $$
                       \int_a^b f(t)dt=(x+c)f(x)\big\vert^b_a-\int^b_a (t+c)f'(t)dt
                        \eqno (2)
                        $$
    Now if we choose $c=-{a+b\over 2}$ we come to (1).
                      $$
                \int_a^b f(t)dt=\left(x-{a+b\over 2}\right)f(x)\big\vert^b_a-\int^b_a \left(x-{a+b\over 2}\right)f'(t)dt={b-a\over 2}(f(a)+f(b))+\dots
                \eqno (2a)
                      $$
    One can go further performing integration by part.  Keeping in mind formula (2a)
    instead  an expansion
                 $$
                 \int f(x)dx=xf(x)-{x^2\over 2\,\,}f'(x)+{x^3\over 3!}f''(x)-{x^4\over 4!}f'''(x)+\dots
                 \eqno (3)
                 $$
     we  consider  an expansion
                   $$
             \int f(x)dx=B_1(x)f(x)-{B_2(x)\over 2\,\,}f'(x)+{B_3(x)\over 3!}f''(x)-{B_4(x)\over 4!}f'''(x)
             +\dots\,,       \eqno (3a)
                   $$
where polynomials $\{B_1(x), B_2(x),B_3(x),\dots\}$ are defined by the relations ${dB_{k+1}(x)\over dx}=kB_k(x)$:
                            $$
    B_1(x)=x+c_1,\, B_2(x)=2\left({x^2\over 2}+c_1x+c_2\right),\,
   B_3(x)=6\left({x^3\over 6}+c_1{x^2\over 2}+c_2x+c_3\right),
                            $$
                               $$
           B_4(x)=24\left({x^4\over 24}+c_1{x^3\over 6}+c_2{x^2\over 2}+c_3x+c_4\right), \,\,
           \hbox {and so on}\,,
           \eqno (3b)
                               $$
  where $c_1,c_2,c_3,\dots$ are an arbitrary constants.
  We have for an interval $(a,b)$ that
                       $$
     \int_a^b f(t)dt=\sum_{n=1}^N(-1)^{n-1}{B_n(x)\over n!}f^{(n-1)}(x)\vert^b_a+
     {(-1)^N\over N!}\int_a^bB_N(t)f^{(N)}(t)dt=
                           $$
                           $$
                   B_1(b)f(b)-B_1(a)f(a)-{B_2(b)f'(b)-B_2(a)f'(a)\over 2}+
                   {B_3(b)f''(b)-B_3(a)f''(a)\over 6}+\dots
                                                 \eqno (4)
                       $$
Now encouraged by the trapezoid formula
 choose  $c_1=-{a+b\over 2}$.
 Then
                   $$
                   B_2(a)=B_2(b)\,.
                   \eqno (5)
                   $$
                   since $B_1(a)=B_1(b)$ if $c_1=-{a+b\over 2}$.
 We want to keep the relation (5) for all $B_k(x)$ for $k\geq 2$:
              $$
                                B_k(a)=B_k(b)\quad  \hbox{for all $k\geq 2$}
                   \eqno (5a)
              $$
 In this case the formula
 (4) becomes:
                $$
                \int_a^b f(t)dt=(b-a)\cdot {f(a)+f(b)\over 2}+
            \sum_{n\geq 2}{(-1)^{n-1}\over n!}B_n(a)\left(f^{(n-1)}(b)-f^{(n-1)}(a)\right)
            \eqno (6)
                    $$


   \centerline {\bf ${\cal  x} 2$  Bernoulli polynomials and numbers}

 The condition (5a) fixes uniquely all constants $c_2,c_3,\dots,c_4,\dots$ in (3).
 We come to recurrent formula for polynomials $B_n(x)$:
                      $$
                      B_0(x)\equiv 1,\quad
                      \cases
                           {
       B_{k}(x)\colon \quad {dB_{k}(x)\over dx}=kB_{k-1}(x)\cr
                         \int_a^bB_k(x)dx=0,\,\,{\rm i.e.} \,  B_{k+1}(a)=B_{k+1}(b)
                        }
                        \quad (k=1,2,3,\dots)
                        \eqno (2.1)
                      $$
 {\it One can say roughly that polynomials $B_n(x)=x^n+\dots$ are "deformations" of polynomials $x^n$
  suitable for integration by part.}
 \m


 These polynomials are:
                        $$
                        \matrix
                           {
                       &B_0(x)=1\cr
                       &B_1(x)=x-{a+b\over 2}\cr
                       &B_2(x)=(x-a)(x-b)+{1\over 6}(b-a)^2\cr
                       &B_3(x)=(x-a)^3-{3\over 2}(x-a)^2(b-a)+{1\over 2}(x-a)(b-a)^2\cr
                       &B_4(x)=(x-a)^4-2(x-a)^3(b-a)+(x-a)^2(b-a)^2-{1\over 30}(b-a)^4\cr
                                  &....
                               }
                               \eqno (2.1a)
                           $$
    Consider {\it normalised } polynomials choosing  $a=0,b=1$:
                 $$
         B_0(x)=1,\,\,  B_n'(x)=nB_{n-1},
         \int_0^1 B_n(x)dx=0,\, {\rm i.e.}\, B_{n+1}(0)=B_{n+1}(0),
         n=1,2,\dots \colon
                      $$
                      $$
                      \matrix
                           {
                       &B_0(x)=1\cr
                       &B_1(x)=x-{1\over 2}\cr
                       &B_2(x)=x^2-x+{1\over 6}\cr
                       &B_3(x)=x^3-{3\over 2}x^2+{1\over 2}x\cr
                       &B_4(x)=x^4-2x^3+x^2-{1\over 30}\cr
                       &B_5(x)=x^5-{5\over 2}x^4+{5\over 3}x^3-{x\over 6}\cr
                       &B_6(x)=x^6-3x^5+{5\over 2}x^4-{1\over 3}x^2+{1\over 42}\cr
                       &B_7(x)=x^7-{7\over 2}x^6+{7\over 2}x^5-{7\over 6}x^3+{1\over 6}x\cr
                           &\dots \cr
                              }
                              \eqno (2.2)
                        $$
          {\bf Exercise 1} Show that relation  between normalised polynomials $B^{[0,1]}_n$ in (7) and
           polynomials $B^{[a,b]}_n$ is
                             $$
                             B^{[a,b]}_n(x)=(a-b)^nB^{[0,1]}_n\left(x-a\over b-a\right)
                             \eqno (2.3)
                             $$
  This formula controls the behaviour of Bernoulli polynomials under changing of $a,b$.


          We define {\it Bernoulli number} $b_n$ as a value of polynomial (2.2) at the points $0$ or $1$
          (or polynomial (2.1a) divided by a coefficient $(b-a)^n$)
                            $$
                        b_n=B_n(0)=B_n(1).
                            $$
        We have
                               $$
   b_0=1, b_1=-{1\over 2}, b_2={1\over 6}, b_3=0, b_4=-{1\over 30}, b_5=0, b_6={1\over 42},b_7=0,\dots
                               $$
           Interesting observation:

          \m

          {\bf Proposition 1}  {\it Bernoulli numbers $b_n$ are equal to zero  if $n$
          is an odd number bigger than 1.}

          \m


          This proposition follows from the following very beautiful property of Bernoulli polynomials:



          \m

          {\bf Proposition 2}  {\it Let $\{B_n(x)\}$ be a set of Bernoulli polynomials
          corresponding to the interval $(a,b)$  (see eq. (7)). Let $P$ be a reflection with respect
          to the middle point  ${a+b\over 2}$ of the interval $(a,b)$:
                               $$
                           P\colon\,\, x\mapsto  a+b-x
                           \eqno (2.4)
                               $$
            Then all Bernoulli polynomials (except $B_1$) are eigenvectors of this transformation:
                             $$
                        B_n(Px)=B_n(x) \,\,\hbox{for all even $n$, $n=0,2,4,\dots$}
                        \eqno (2.5a)
                                 $$
and}                           $$
B_n(Px)=-B_n(x) \,\,\hbox{for all odd $n\geq 3$, $n=3,5,7,\dots$}
\eqno (2.5b)
                               $$
                   \m

Indeed it follows from (2.5b)
                   that $b_n=B_n(a)=-B_n(b)=-b_n$ for odd $n\geq 3$. Thus $b_n=0$ for $n=3,5,7,\dots$.

                   \m

           The statement of this Proposition 2 is irrelevant to the choice of $a,b$.
          To prove the Proposition  it is suffice to consider the special case $a=-b$.
             In this case the transformation $P$ in (2.4) is just $x\mapsto-x$.
           Thus in this case  the statement of Proposition is that
           Bernoulli polynomials  $B_n(x)$ are even polynomials ($B_n(x)=B_n(-x)$) if $n$ is even,
            and they are odd polynomials
           if $n$ is an odd number greater than 1 ($B_n(x)=-B_n(-x)$).


           Prove it by induction. Suppose that for $n\leq 2N$ this is true.
           Then consider polynomial $B_{2N}(x)$.
           We have that $\int_{-a}^a B_{2N}(x)dx=0$, hence $\int_0^a B_{2N}(x)dx=0$ since
           by induction hypothesis this is an even polynomial.  Hence
            $$
            B_{2N+1}(x)=(2N+1)\int_0^x  B_{2N}(t)dt\,.
            $$
     Indeed this polynomial obeys the differential equation $B'_{2N+1}(x)=(2N+1)B_{2N}(x)$
     This polynomial is also an odd polynomial. Hence it obeys the boundary condition
            $\int_{-a}^aB_{2N+1}(x)dx=0$.  It remains to prove that $B_{2N+2}$ is an even polynomial.
            We have that  $B_{2N+2}(x)=\int_0^x    B_{2N+1}(t)dt+c_{2N+2}$, where $c_{2N+2}$
            is a constant chosen by the boundary condition  $\int_a^aB_{2N+2}(x)dx=0$.
            We see that  $B_{2N+2}$ is even since $B_{2N+1}$ is an odd polynomial and constant is an even polynomial.
\m


   \centerline {\bf ${\cal  x} 3$ Integral and area of trapezium (revisited)---Euler-Maclaurin formula }

\m

     Now equipped by the knowledge of formulae
     return to the last formula from the first paragraph:
                     $$
        \int_a^b f(t)dt=(b-a)\cdot {f(a)+f(b)\over 2}+
            \sum_{n\geq 2}{(-1)^{n-1}\over n!}B_n(a)\left(f^{(n-1)}(b)-f^{(n-1)}(a)\right)=
            \eqno (3.1)
                                     $$
                       $$
                       (b-a)\cdot {f(a)+f(b)\over 2}+
            \sum_{n\geq 2}{(-1)^{n-1}\over n!}b_n(b-a)^n\left(f^{(n-1)}(b)-f^{(n-1)}(a)\right)=
            \eqno (3.1a)
                     $$
                     $$
                       (b-a)\cdot {f(a)+f(b)\over 2}+
            \sum_{k\geq 1}{(-1)^{2k-1}\over (2k)!}b_{2k}(b-a)^{2k}\left(f^{(2k-1)}(b)-f^{(2k-1)}(a)\right)=
            \eqno (3.1a)
                                $$
                                $$
  (b-a)\cdot {f(a)+f(b)\over 2}+{1\over 6}(b-a)^{2}\left(f'(b)-f'(a)\right)
  -{1\over 30}(b-a)^{4}\left(f'''(b)-f'''(a)\right)+\dots
                                                              $$

     We are ready to write down asymptotic formula for series:
      Dividing the interval $[0,1]$ on $N+1$ parts consider the formula above for any interval
      $\left[{k\over N}, {k+1\over N}\right]$ , then making summation we come to:
                         $$
     \int_0^1f(x)dx=
     {1\over 2N}f(0)+
     \left(
            f
            \left({1\over N}\right)+f\left({2\over N}\right)+\dots+
            f\left({N-1\over N}\right)
            \right)+
                         {1\over 2N}f(0)+
                           $$
                           $$
                           +
\sum_{k\geq 1}{(-1)^{2k-1}\over (2k)!}{b_{2k}\over N^{2k}}\left(f^{(2k-1)}(1)-f^{(2k-1)}(0)\right)
                         $$
      it is well-known Euler-Macklourin asymptotic formula.
                         
{\bf Remark} Our notation for bernoulli numbers is not standard. Bernoulli numbers are
$B_n=b_{2n}$. 

              {\bf Exercise} Use this formula for the functions $f=x^r$ to express sums
              $\sum_{i=1}^N i^r$ via Bernoulli numbers.

                      \m

    \centerline {\bf ${\cal  x} 4$ Fourier image of Bernoulli polynomials and $\zeta$-function }

                   \m

               Bernoulli polynomials are deformations of $x^n$ which are convenient for integration by part.
               Function $e^x$ is eigenvalue of derivation operator. This means that Bernoulli polynomials
               have "good" Fourier transform. Do calculations. Consider Fourier polynomials for the interval $[0,1]$
               (see 2.2) and  an orthonormal basis $c_k\{e^{2\pi ikx}\}$ where ....   Indeed
                          $$
                \left<B_n(x),e^{2\pi ik}\right>=\int_0^1B_n(x),e^{2\pi ikx}\sim {1\over k^{n}}
                        $$
    Hence

             $$
             \left<B_n(x),e^{2\pi ik}\right>\sim \sum {1\over k^{2n}}=\zeta(2n)
             $$
    Notice that square of the norms of Bernoulli polynomials can be expressed via Bernoulli numbers
    due to their properties...


                      \bye   