From khudian@manchester.ac.uk Wed Feb 10 20:51:41 2016
Date: Wed, 10 Feb 2016 20:51:41 +0000 (GMT)
From: khudian@manchester.ac.uk
To: David Khudaverdyan <khudaverdian@gmail.com>
Subject: Gamma-funktsija i vsjo takoje.


\def\G {\Gamma}
   My khorosho znajem shto
        $$
   \G(x)=(x-1)!=\int_0^\infty t^{x}e^{-t}dt
           \eqno (1)
        $$
My k tomu zhe znajem shto
         $$
   B(x,y)={\G(x)\G(y)\over \G(x+y)}\,,
              \eqno (2)
         $$
gde funtsijsa
          $$
    B(x,y)=\int_0^1 t^{x-1}(1-t)^{y-1}dt\,.
          $$
   Okazyvajetjsa Eiler ottalkivalsia ot sovsem drugogo
predstavlenija Gamma-funtksii
  i integraljnoje predstvalenije (1)
poluchil kak pobochnyj fact vyvodia formulu (2).
     Euler pytajasj obosbhitj factorial zametil shto
            $$
k!={(N+k)!\over (k+1)_N}={N!(N+1)_k\over (k+1)_N }=
       \lim_{N\to \infty}{N! N^k\over (k+1)_N}
            $$
gde $(A)_r=A(A+1)\dots (A+r-1)$
   poslednjeje vyrazhenije dopuskajet prodolzhenija
na netselyje $x$:
     Euler pytajasj obosbhitj factorial zametil shto
            $$
x!=\lim_{N\to \infty}{N! N^x\over (x+1)_N}
            $$
i sootvetstvenno:
         $$
\G(x)=(x-1)!=\lim_{N\to \infty}{N! N^{x-1}\over (x)_N}
         $$

   Opredelenie s kotorogo startoval Euler  vygljadit neukljuzhe,
no nachicnaja s nim rabotatj vidishj i ego prelestj.
Esli khocheshj ja poshlju tebe pdf file knigi gde vsjo eto estj.
i

\bye


