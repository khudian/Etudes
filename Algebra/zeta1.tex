
\def\A {\bf A}
\def\t {\tilde}
\def\a {\alpha}
\def\K {{\bf K}}
\def\N {{\bf N}}
\def\V {{\cal V}}
\def\s {{\sigma}}
\def\S {{\Sigma}}
\def\s {{\sigma}}
\def\vare{{\varepsilon}}
\def\Q {{\bf Q}}
\def\D {{\cal D}}
\def\G {{\Gamma}}
\def\C {{\bf C}}
\def\M {{\cal M}}
\def\Z {{\bf Z}}
\def\U  {{\cal U}}
\def\H {{\cal H}}
\def\R  {{\bf R}}
\def\l {\lambda}
\def\degree {{\bf {\rm degree}\,\,}}
\def \finish {${\,\,\vrule height1mm depth2mm width 8pt}$}
\def \m {\medskip}
\def\p {\partial}


%\documentclass[12pt]{article}
%\usepackage{amsmath,amsthm}



%\begin{document}

%\maketitle

  \centerline {it is a draft of Lecture Notes of H.M. Khudaverdian.}

  \centerline { Manchester, Autumn 2004}

%\tableofcontents

$\Q$


%\section*{Preliminaries}


\def\ansatz {substitution}



   Let $F_p$ be a primary field of characteristic  $p$, i.e. $F_p=Z/pZ$

Consider the field $\Omega_p$--algebraically closure of the field
$F_p$, i.e. the field which possesses all the roots of all the
polynomials with coefficients over $F_p$. The simplest way to
construct this field is following.

First of all consider finite algebraic extensions  of $F_p$.


  One can show that every finite algebraic extension of the field
  $F_p$ is the field $F_{p^k}$.
  It is the field which possesses all the roots of the polynomial
             $$
             x^{p^k}-x=0
             $$
How to describe this field?  It is very convenient for these
purposes to consider so called Frobenius homomorphism:Take
arbitrary root of the polynomial above which does not belong to
$F_p$. Denote it by $\theta$.
  Consider for arbitrary element $\theta\in \Sigma(f)$ the sequence of elements:
                  $$
             \{\theta,\varphi\theta,\dots,\}
                  $$

\bye
