\def\m {\smallskip}
\def \G {{\cal G}}
\def \p {\partial}
\def \t {\tilde}
\def \a  {\alpha}
\def \d  {\delta}
\def \L {\Lambda}
\def \drightarrow {\buildrel{\der}\over\longrightarrow}
\def \der {d_{_{E.L}}}
\def \C   {{\cal C}}
\def \B   {\overline}
\def \V {{\cal V}}
\def \vsubs {\cap\phantom{\vrule height2mm depth0mm width0.4pt}
                  \vrule height2mm depth0mm width0.4pt}
\def\interv{\phantom{\vrule height6mm depth0mm width0.4pt}}
\def\intervs{\phantom{\vrule height4mm depth0mm width0.4pt}}

\def \U {{\cal U}}
\def \P {{\cal P}}
\def \Ga {\Gamma}
\def \min {{\cal n}}
\def \dR  {{\bf R}}
\def \R  {{\bf R}}
\def \vare {\varepsilon}
\def \Hs {H^1(\G,\L^0(M))}


\def\Postn{16}

     \centerline {\bf
  Double complex and its spectral sequences 1}
                  \medskip
 Now we give a brief sketch on the topic how
 to apply spectral sequences technique
for calculations of cohomology of double complexes.
 (See for the details for example [\Postn].)

Let $E^{**}=\{E^{p.q}\}$ $(p,q=0,1,2,...)$ be a
 family of abelian groups
(modules, vector spaces) on which are defined two differentials
$\p_1$ and $\p_2$ which define complexes in rows and in columns of
 $E^{*.*}$ and which commute with each other:
                            $$
          \p_1\colon E^{p.q}\rightarrow E^{p.q+1}\,,\p_1^2=0,\,
          \p_2\colon E^{p.q}\rightarrow E^{p+1.q}\,,\p_2^2=0,\,
                    \p_1\p_2=\p_2\p_1\,.
                                       \eqno (A2.1)
                            $$
   $\{E^{**},\p_1,\p_2\}$ is called double complex.

  ( It is convenient to consider $E^{p.q}$ for all
integers $p$ and $q$ fixing that $E^{p.q}=0$ if $p<0$ or $q<0$.)

 One can consider  "antidiagonals":
   ${\cal D}^m=\{E^{p.m-p}\}$ ($p=0,1,...,m$)
  which form complex with differential
                             $$
                        Q=(-1)^q\p_2+\p_1
                                                  \eqno (A2.2)
                             $$
 which evidently obeys to condition $Q^2=0$.
                               $$
       0\rightarrow{\cal D}^0{\buildrel Q\over\rightarrow}
          {\cal D}^1{\buildrel Q\over\rightarrow}{\cal D}^2
                   \rightarrow\dots\,.
                                                 \eqno (A2.3)
                               $$
            The cohomologies $H^m(Q)$ of this complex
are called  the cohomologies of double complex $(E^{**},\p_1,\p_2)$.

 The rows and the columns complexes define the cohomologies
  $H(\p_1)$ and  $H(\p_2)$ of $E^{**}$.

One can consider the filtration corresponding to the double complex
$\{E^{*.*},\p_1,\p_2\}$
                          $$
    \dots\subseteq X^m\subseteq
  X^{m+1}\subseteq\dots\subseteq X^1\subseteq X^0
                                              \eqno (A2.4)
                           $$

                     $$
  {\rm where}\qquad\quad
     X^k= \bigoplus_{q\geq 0,p\geq k} E^{p.q}
                                      \eqno (A2.5)
                    $$

 and sequence of the spaces  $\{E^{p.q}_r\}$
 $(r=0,1,2,\dots$ corresponding to this filtration
                           $$
             E^{p.q}_r =Z^{p.q}_r\big/B^{p.q}_r\quad
                  (E^{p.q}_0=E^{p.q})\,.
                                                \eqno(A2.6)
                             $$
In (A2.6) $Z^{p.q}_r$ ("$r$-th order cocycles")
 is the space of the elements in $E^{p.q}$
 which are leader terms of
 cocycles of the differential $Q$ up to $r$--th order
w.r.t. the filtration (A2.4), i.e.
                          $$
       \{Z^{p.q}_r\}=\{E^{p.q}_r\ni c\colon\quad
       \exists {\t c}= c
           (mod\, X_{p+1}) \,{\rm such\,that}\,
             Q{\t c}=0(mod\,X_{p+r})\}\,.
                                                \eqno(A2.7)
                          $$
   It means that there exists ${\t c}=(c,c_1,c_2,\dots,c_{r-1})$
where $c_i\in E^{p+i.q-i}$  such that

\noindent$Q (c,c_1,c_2,\dots,c_{r-1})\subseteq X_{p+r}\,$:
                          $$
     \p_1 c=0,\p_2 c=\p_1 c_1,\p_2 c_1=\p_1 c_2,\dots,
    \p_2 c_{r-2}=\p_1 c_{r-1},\,{\rm so}\,
     Q {\t c}=\p_2 c_{r-1}\in X_{p+r}\,.
                           $$
 Correspondingly $B^{p.q}_r$ is the space
of up to $r$--th order borders:
                          $$
       \{B^{p.q}_r\}=\{E^{p.q}_r\ni c\colon\quad
       {\cal 9} {\t b}\in X_{p-r+1}\,{\rm such}\quad{\rm that}\,
             Q{\t b}=c\,.
                                                \eqno(A2.8)
                          $$
   It means that there exist ${\t c}=(b_0,b_1,b_2,\dots,b_{r-1})$
where $b_i\in E^{p-i.q+i}$  and

\noindent $ Q (b_0,b_1,b_2,\dots,b_{r-1})=c$:
                          $$
     \p_1 b_0+\p_2 b_1=c,\p_1 b_1+\p_2 b_2=0,
       \p_1 b_2+\p_2 b_3=0,\dots,
                \p_1 b_{r-1}=0\,.
                                 \eqno (A2.9)
                           $$
  For example $E^{p.q}_1=H(\p_1,E^{p.q})$.

 We denote by $[c]_r$ the equivalence class of the
 element $c$ in the $E^{p.q}_r$ if $c\in Z^{p.q}_r$.

 It is easy to see that the sequence $\{E^{p.q}_r\}$
   $r=0,1,2,\dots$ is stabilized after finite number of the steps:
($E^{p.q}_{r_0}=E^{p.q}_{r_0+1}=\dots=E^{p.q}_\infty$,
 where $r_0=max\{p+1,q+1\}$.

  Let $H^m(Q, X_p)$
 be cohomologies groups of double complex truncated by
 filtration (A2.4) (we come to $H^m(Q, X_p)$   considering
  $\{{\cal D}\cap X^p,Q\}$
 as subcomplex of (A2.3), $H^m(Q)=H^m(Q,X^0)$.
 We denote by  $_{(p)}H^m(Q)$ the image of
$H^m(Q, X_p)$ in $H(Q)$   under
 the  homomorphism
 induced by the embedding ${\cal D}\cup X_p\rightarrow {\cal D}$.
  The spaces $_{(p)}H^m(Q)$    are embedded in each other
                      $$
   0\subseteq \,_{(m)}H^m(Q)\subseteq \,_{(m-1)}H^m(Q)\subseteq
        \dots\,_{(1)}H^m(Q)\subseteq
                 \,_{(0)}H^m(Q)=H^m(Q)\,.
                                                \eqno(A2.10)
                     $$

 The spaces $E^{p.q.}_\infty$ considered above
 are related with (A2.10) by
 the following relations:
                              $$
         E^{p.m-p}_\infty=_{(p)}H^m(Q)
                         \big/
                    _{(p+1)}H^m(Q)   \,.
                                             \eqno (A2.11)
                               $$
 In particular  $E^{0.m}_\infty$ is canonically embedded
 in $H^m(Q)$.

  The formula (A2.11) is the basic formula which expresses
 the cohomology
$H(Q)$ of the double complex $\{E^{p.q},\p_1,\p_2\}$ in terms of
$\{E^{p.q}_\infty\}$.
 From (A2.10, A2.11) it follows that
                             $$
           H^m(Q)\simeq\bigoplus_{i=0}^{m} E^{p-i.i}\,.
                                                \eqno (A2.12)
                            $$
The essential difference of (A2.12) from  (A2.11) is that
 in (A2.12) the isomorphism of l.h.s. and of r.h.s.
{\it is not canonical}.

 The importance of the sequence $\{E^{*.*}_r\}$ ($r=0,1,2,\dots$)
is explained by the fact that its terms (and so
$\{E^{*.*}_\infty\}$) can be calculated in a recurrent way. Namely
one can consider differentials (See for details [\Postn.])
  $d_r\colon\, E^{p.q}_r\rightarrow E^{p+r.q+1-r}_r$
 such that  $\{E^{*.*}_r, d_r\}$ form spectral sequence, i.e.
                            $$
              E^{*.*}_{r+1}=H(d_r,E^{*.*}_r).
                                           \eqno (A2.13)
                            $$
The differentials $d_r$ are constructed in the following way:
   $d_0=\p_1\colon\, E^{p.q}=E^{p.q}_0
\rightarrow E^{p.q+1}=E^{p.q+1}_0$.

   \noindent
If $c\in E^{p.q}$ and $\p_1 c=0\leftrightarrow
 [c]_1\in E^{p.q}_1$ then
 $d_1[c]=[\p_2c],\,d_1\colon\,E^{p.q}_1
      \rightarrow E^{p+1.q}_1$.

\noindent  In general case for $[c]_r\in E^{p.q}_r$
  $d_r[c]_r=[Q{\t c}]_r\,\,d_r\colon\,
 E^{p.q}_r\rightarrow E^{p+r.q+1-r}_1$,

 where $\t c\colon\, c-\t c\in X^{p+r}$ (see the definition (A2.7)
 of $Z^{p.q}_r$).

 One can show that definition of $d_r$ is correct, $d_r^2=0$ and
 (A2.13) is obeyed [\Postn].

Using (A2.13) one come after finite number of steps to
    $E^{p.q}_\infty$ calculating each $E^{p.q}_r$  as
the cohomology group of the $E^{p.q}_{r-1}$:
 $E^{p.q}_1=H(d_0,E^{p.q})$,
    $E^{p.q}_2=H(d_1,E^{p.q}_1$ and so on.

    The spaces $E^{p.q}_r$  can be considered intuitively as
$r$--th order (with respect to differential $\p_2$)
 cohomologies of differential $Q$ .
The operator $\p_1$ is zeroth order approximation for differential
$Q$.
  The calculations of
 $E^{p.q}_\infty$ via (A2.13) can be considered as
perturbational calculations.
\smallskip
One can develop this scheme considering in perturbative calculations
not the operator $\p_1$, but $\p_2$ as zeroth order approximation.


Instead filtration (A2.4) one has consider the "transposed"
filtration
                          $$
    \dots\subseteq\, ^t X^m\subseteq\,
  ^t X^{m+1}\subseteq\dots\subseteq\, ^t X^1\subseteq X^0
                                              \eqno (A2.14)
                           $$
                     $$
   {\rm where}\qquad\quad
 \,^t X^k= \bigoplus_{p\geq 0,q\geq k} E^{p.q}
                    $$
   and corresponding transposed
  spaces $\{\,^t E^{p.q}_r\}$. For example
                      $$
                  E^{p.q}_1=H(\p_1, E^{p.q}),\quad
                   ^t E^{p.q}_r=H(\p_2, E^{p.q})\,.
                       $$
Instead spectral sequence $\{E^{*.*}_r,d_r\}$ one has to consider
 transposed spectral sequence $\{^tE^{*.*}_r,\,^td_r\}$:
                             $$
               d_0=\p_1,\rightarrow \,^td_0=\p_2\,;
               d_1[c]_1=[\p_2 c]_1,\rightarrow \,
              ^td_1 [c]_1= [\p_1 c]_1\,,
                         $$
and so on.

  The relations between spaces
 $\{E^{p.q}_\infty\}$ and $\{^t E^{p.q}_\infty\}$
 which express in different ways
the cohomology $H(Q)$ is one of the applications of the method
described here.

 {\bf Example.} Let ${\bf c}=(c_0,c_1.c_2)$ where
 $c_0\in E^{0.2}, c_1\in E^{1.1}, c_2\in E^{2.0}$ be cocycle
 of the differential $Q$:
$Q(c_0,c_1.c_2)=0$ i.e. $\p_1 c_0=0,\p_2 c_0=-\p_1 c_1,
                        \p_2 c_1=\p_1 c_2$.
 To the leading term $c_0$ of this cocycle w.r.t. the
filtration (A2.4) corresponds the element
 $[c_0]_\infty$ in $E^{0.2}_\infty$
which  represents the cohomology class
 of the cocycle ${\bf c}$ in
   $E^{0.2}_\infty$.

In the case if the equation
$(c_0,c_1.c_2)+Q(b_0,b_1)=(0,c_1^\prime,c_2^\prime)$
 has a solution, i.e. the leading term $c_0$
 of the cocycle ${\bf c}$  can be cancelled
  by changing of this cocycle on a coboundary, then
the element   $[c_1^\prime]_\infty\in E^{1.1}_\infty$
 represents the cohomology class of the cocycle ${\bf c}$ in
   $E^{1.1}_\infty$.

 In the case if the equation
$(c_0,c_1.c_2)+Q(b_0,b_1)=(0,0,\t c_2)$ have a solution, i.e. the
leading term and next one both can be cancelled, by redefinition on
a coboundary, then
  $[{\t c_2}]_\infty\in E^{2.0}_\infty$
 represents the cohomology class of the cocycle ${\bf c}$ in
   $E^{2.0}_\infty$.

  To put correspondences between
  the cohomology class of the cocycle ${\bf c}$
 and corresponding elements from
 transposed spaces
$^tE^{0.2}_\infty,\,^t E^{1.1}_\infty\,^tE^{1.1}_\infty$ we have to
do the same, changing only the definition of leading terms, which we
have to consider now w.r.t. the filtration (A2.14).

 To the leading term $c_2$ of this cocycle w.r.t. the
filtration (A2.14) corresponds the element
 $[c_2]_\infty$ in $\,^tE^{2.0}_\infty$
which  represents the cohomology class
 of the cocycle ${\bf c}$ in
   $\,^tE^{2.0}_\infty$.
In the case if the equation
$(c_0,c_1.c_2)+Q(b_0,b_1)=(c_0^\prime,c_1^\prime,0)$
 has a solution, i.e. the leading term $c_0$
 of the cocycle ${\bf c}$  can be cancelled
  by changing of on a coboundary, then
the element $[c_1^\prime]_\infty$
 represents the cohomology class of the cocycle ${\bf c}$ in
   $\,^tE^{1.1}_\infty$.
 In the case if the equation
 $(c_0,c_1.c_2)+Q(b_0,b_1)=(\t c_0,0,0)$
 has a solution, then $[{\t c_0}]$
 represents the cohomology class of the cocycle ${\bf c}$ in
   $\,^tE^{0.2}_\infty$.



 [\Postn]  Postnikov,M.M.: Lectures on Geometry,
  {\it Semestre III, Lecture \#19,
 Semestre V, Lecture \#23}
  Moscow, Nauka, (1987).


\bye
