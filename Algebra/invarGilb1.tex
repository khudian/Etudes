\def\p{\partial}
\def\t {\tilde}
\def \m {\medskip}
\def\degree {{\bf {\rm degree}\,\,}}
\def \finish {${\,\,\vrule height1mm depth2mm width 8pt}$}





\def\a {\alpha}
\def\vare{{\varepsilon}}
\def\l {\lambda}
\def\s {{\sigma}}

\def\G {{\Gamma}}

\def\A {{\bf A}}
\def\C {{\bf C}}
\def\E  {{\bf E}}
\def\K {{\bf K}}
\def\N {{\bf N}}
\def\Q {{\bf Q}}
\def\R  {{\bf R}}
\def\V {{\cal V}}
\def \X   {{\bf X}}
\def \Y   {{\bf Y}}
\def\Z {{\bf Z}}



\def\ac {{\bf a}}
\def\e{{\bf e}}
\def\f {{\bf f}}
\def\n {{\bf n}}
\def\r {{\bf r}}
\def\v {{\bf v}}
\def \x   {{\bf x}}
\def \y   {{\bf y}}


\def\pt {{\bf pt}}
\def \wt {\widetilde}


\centerline {\bf Theory of invatriants}

\m

 {\it I am reading the book of D. Hilbert: Lecture course "Theory of algebraic invariants"
   Summer semester 1897.  I try to expose here the main ideas of this course
   as I understand it.}

   \m

          Consider binary form of the order $n$
                    $$
         f=a_0x^n+na_1x^{n-1}+{n(n-1)\over 2}a_2x^{n-2}y^2+\dots+a_ny^n=\sum_{k=0}^n C^k_nx^{n-k}y^k
                    $$
    The action of group $GL(2,\R)$ and of its subgroup $SL(2,\R)$ on $(x,y)$ induces the action of this group
    and its subgroup on $n+1$-dimensional space of coefficients.
   We say that polynomial $\cal I (a_0,\dots,a_n)$ is invariant if it is $SL(2,R)$-invariant.

   There are natural weights and degrees. We say that degree of any coefficient $a_i$ is equal to $1$.
   One can consider invariants of given degree $g$. Degree of
 $(a_0^{m_0} a_1^{m_1}...)$ equals $m_0+m_1+\dots$.

   We say that invariant ${\cal I}(a_i)$ has {\it weight} $p$ if
                $$
                {\cal I}(a^g_i)=(\det g)^p{\cal I}(a_i)\,.
                $$
   Consider fundamental vector fields of the action of the group $GL(2.\R)$ on $(x,y)$
                $$
D_x=x{\p\over \p x},\,\,D_y=y{\p\over \p y}, \,\,T_x=y{\p\over \p x},\,\,T_y=x{\p\over \p y}\,.
                $$

These vector fields form a basis of  representation of the Lie algebra $gl (2,\R)$ with commutator relations:
                 $$
   [D_x,D_y]=0, [D_x,T_x]=-T_x, [D_x,T_y]=T_y\,, [D_y,T_x]=T_x, [D_y, T_y]=-T_y, [T_x,T_y]=D_y-D_x\,.
                 $$
   The generators  $\{D_x-D_y, T_x, T_y\}$ are generators of representation of the Lie algebra $sl (2,\R)$

Consider the induced presentation of these vector fields on the space $\{a_0,a_1,a_2,\dots,a_n\}$. We have:
                    $$
     \widetilde D_x=na_{_0}{\p\over \p a_{_0}}+(n-1)a_{_1}{\p\over \p a_{_1}}+(n-2)na_{_2}{\p\over \p a_{_2}}+\dots+
        a_{_{n-1}}{\p\over \p a_{_{n-1}}}=
              \sum_{k=0}^{k=n}(n-k)na_{_k}{\p\over \p a_{_k}}
                    $$
                    $$
      \widetilde D_y=a_{_1}{\p\over \p a_{_1}}+2a_{_2}{\p\over \p a_{_2}}+\dots+na_{_n}{\p\over \p a_{_n}}=
              \sum_{k=0}^{k=n}ka_{_k}{\p\over \p a_{_k}}
                    $$
                $$
   \widetilde T_x=a_0{\p \over \p a_1}+2a_1{\p \over \p a_2}+\dots+(n-1)a_{n-1}{\p \over \p a_n}
                $$
and
               $$
\widetilde T_y=a_n{\p \over \p a_{n-1}}+2a_{n-1}{\p \over \p a_{n-2}}+\dots+a_1{\p \over \p a_0}
               $$
(We come to these formulae considering infinitesimal transformations of $x,y$ for binary form

 The action of vector fields $\widetilde D_x, \widetilde D_y, \widetilde T_x, \widetilde T_y$
 on the space of coefficients are defined by the  condition that binary form (1) remains unchanged).
 (Il faut expliquer plus...)

The commutator relations will be:
                    $$
[\widetilde D_x,\widetilde D_y]=0, [\widetilde D_x,\widetilde T_x]=+\widetilde T_x,
[\widetilde D_x,\widetilde T_y]=-\widetilde T_y\,,
[\widetilde D_y,\widetilde T_x]=-\widetilde T_x,
[\widetilde D_y, \widetilde T_y]=+\widetilde T_y,
[\widetilde T_x,\widetilde T_y]=-\widetilde D_y+\widetilde D_x\
                    $$
{\bf Attention: the "mysterious" changing of signs because of changing of representation!}

\m

  Later we often omit the sign $\,\,\widetilde {}\,\,$ if this will not lead to confusion.

  \m

   Polynomial ${\cal I}(a_0,a_1,a_2,\dots,a_n)$ is $SL(2,\R)$-invariant
   if the following conditions hold:
                        $$
          (\widetilde D_x-\widetilde  D_y)  {\cal I}(a_0,a_1,a_2,\dots,a_n)=0\,,
               \eqno (I{\rm dilatation \,\,condition \,\,})
                   $$
   and
                   $$
                      \widetilde T_x {\cal I}(a_0,a_1,a_2,\dots,a_n)=0\,\,
                      \eqno (II)
                   $$
                   $$
                      \widetilde T_y {\cal I}(a_0,a_1,a_2,\dots,a_n)=0\,.
                      (III)
                   $$
i.e. if polynomial is invariant under the action of $sl(2,\R)$ algebra of vector fields.


  One can easy se that every $SL(2,R)$ polynomial is the sum of invariant polynomials
    with different weights.
since the filtration by degree is compatible
   with invariance.  To put it in a more formal way consider
   in the universal envelopping algebra the element
                $$
   D= {D_x+D_y\over n}
                $$
One can see that

1)  $D$ defines the degree of polynomial:
               $$
      DP_g=gP_g\,\,,\,\,\,  D(a_0^{m_0} a_1^{m_1}...)=m_0+m_1+\dots
               $$
 and  all operators $D_x,D_y, T_x,T_y$ commute with $D$. This means that to find
invariants we can consider only homogeneous polynomial of the degree $g$.

\m

Consider polynomial $P_g$ of the  degree $g$.

{\bf Definition} We say that polynomial $P$ has weight $p$ if $\tilde D_y P=pP$.
         $$
    p(a_0^{m_0} a_1^{m_1}...a_n^{m_n})=a_1+2a_2+3a_3+\dots+na_n
         $$


         $$
   (\widetilde D_x+\widetilde D_y)P_g=n DP_g=ngP_g\,.
          $$
On the other hand if this polynomial is $SL(2,R)$-invariant then
$(\widetilde D_x-\widetilde  D_y)P_g=0$.  Hence we have that
      for $SL(2,R)$-invariant polynomial of the order $g$
       $$
    D_yP=D_xP={ng\over 2}
       $$
i.e. the weight of the invariant polynomial of degree $g$ equals to $p={ng\over 2}$.


{\sl Exercise }   Show that invariant polynomial is invariant with respect to
the transformation $a_k\leftrightarrow a_{n-k}$.

\m

\m





We come to

    {\bf Proposition}  Let  ${\cal I}(a_0,a_1,a_2,\dots,a_n)$ be a polynomial of  degree $g$.

    Then it is invariant polynomial of the order $p$ if and only if

    1) its weight equals to  $p={2g\over n}$ and
                   $$
                      \widetilde T_x {\cal I}(a_0,a_1,a_2,\dots,a_n)=
                      \widetilde T_y {\cal I}(a_0,a_1,a_2,\dots,a_n)=0\,.
                   $$

\m
\m
Technically it is much easier to calculate the action of generators $D_x, D_y$
on polynomial, than the action of $T_x,T_y$.

Later analyzing invariant polynomials we consider polynomials of fixed degree $g$
(homogeneous polynomials.)

To be invariant they have to be isobaric (have the fixed weight $p$) and $p$ equals to $ng/2$.

In fact one can prove the following useful statement

{\bf Lemma}  Let $P$ be homogeneous isobaric polynomial of the degree $g$ and the weight $p=ng/2$.
Then the condition $T_x P=0$ implies the condition $T_y P=0$.  

This is very useful lemma. To find invariant polynomials we just have to 
consider isobaric polynomials of the
definite weight $p=ng/2$ which obey differential equation $T_xP=0$.


{\bf Example} Find invariants of the degree $g=1,2$ for the form $f=a_0x^2+2a_1xy+a_2y^2$.

For $g=1$, $p=ng/2=1$. $P=Ma_1$. This is not invariant: $T_xa_1=a_0\not=0$.


For $g=2$ $p=2$.   $p(a_0^{m_0} a_1^{m_1}a_2^{m_2})=m_1+2m_2=2$. Hence
  $P=Ma_0a_2+Na_1^2$, The equation  $T_x(Ma_0a_2+Na_1^2)=0$ means
               $$
  0=T_x(Ma_0a_2+Na_1^2)=(a_0\p_1+a_1\p_2)(Ma_0a_2+Na_1^2)=2Na_1a_0+Ma_1a_0\Rightarrow M=N
               $$
We see that for $g=2$ there is one invariant $P=a_0a_2-a_1^2$ (discriminant)


 {\bf Example} the form $f=f=a_0x^3+3a_1x^2y+3a_2xy^2+a_3y^3$.
 


 \m

 {\sl Proof of the lemma}.  Let $P$ be polynomial of the degree $g$ and of the weight $p$ such that
 $2p=ng$ and $T_xP=0$. One can see that
 the operator $\wt T_x$ decreases the weight on $1$ and the operator $\wt T_y$ increases the weight on $1$
 since $[\wt D_y,\wt T_x]=-\wt T_x$ and $[\wt D_y,\wt T_y]=+\wt T_y$. Both operators do not change the degree $g$.
  Consider the sequence of polynomials
   $P_r=\wt T_x^r$. We have that the weight of the polynomial $P_r$ equals to $p+r$. On the other hand
   its degree is always $g$. But evidently for arbitrary polynomial $p\leq ng$. Hence there exists $k\geq 1$
   such that
            $$
     T_x^{k}=0, \,\, T_x^{k-1}       
            $$
We show that $k=1$.




 \m



\bye 





http://www.maths.manchester.ac.uk/undergraduate/ugstudies/timetables/index.htm