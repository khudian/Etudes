\def\vare {\varepsilon}
\def\A {\bf A}
\def\t {\tilde}
\def\a {\alpha}
\def\K {{\bf K}}
\def\N {{\bf N}}
\def\V {{\cal V}}
\def\s {{\sigma}}
\def\S {{\Sigma}}
\def\s {{\sigma}}
\def\p{\partial}
\def\vare{{\varepsilon}}
\def\Q {{\bf Q}}
\def\D {{\cal D}}
\def\G {{\Gamma}}
\def\C {{\bf C}}
\def\M {{\cal M}}
\def\Z {{\bf Z}}
\def\U  {{\cal U}}
\def\H {{\cal H}}
\def\R  {{\bf R}}
\def\E  {{\bf E}}
\def\l {\lambda}
\def\degree {{\bf {\rm degree}\,\,}}
\def \finish {${\,\,\vrule height1mm depth2mm width 8pt}$}
\def \m {\medskip}
\def\p {\partial}
\def\r {{\bf r}}
\def\v {{\bf v}}
\def\n {{\bf n}}
\def\t {{\bf t}}
\def\b {{\bf b}}
\def\ac {{\bf a}}
\def \X   {{\bf X}}
\def \Y   {{\bf Y}}
\def\Tr {{\rm Tr\,}}
\def\Ber {{\rm Ber\,}}
   \centerline{\bf On one construction related with Berezinian}


Let $A$ an operator on the linear (super)space $V$. Let $X$ be an
invariant subspace of $V$ (for all $h\in X, Mh\in X$)
Then the action of operator $A$ on the factor-space  $V/X$ is well-defined.

One can also prolongate the action of the operator
 on the superspace
                  $$
                  V_X=V\oplus\Pi X,
                  $$
where $\Pi $ is $V$ is a parity  reversing functor.

Then it is easy to see that characteristic functions(polynomials) of the operators
$A\vert_{V/X}$ and $A\vert_{V\oplus\Pi X}$ coincide:
                     $$
     \Ber \left(1+zA\vert_{V/X}\right)=\Ber \left(1+zA\vert_{V\oplus\Pi X}\right)
                     $$
Symbolically one can express this phenomenon via the following relation:
                  $$
                  V-X=V+\Pi X
                  $$


\bye
