

\magnification=1200
\baselineskip=14pt

\def\A {{\bf A}} 
\def\B {{\cal B}}
\def\C {{\bf C}}
\def\CC {{\cal C}}
\def\Cl {{\tt \hbox{Cliff}}}
\def\E {{\bf E}}
\def\EE {{\cal E}}
\def\F {{\cal F}}
\def\FF {{\cal F}}
\def\G {\Gamma}
\def\GG {{\cal G}}
\def\H {{\bf H}}
\def\K {{\bf K}}
\def\L {{\cal L}}
\def\M {{\cal M}}
\def\N {{\bf N}}
\def\R {{\bf R}}
\def\Sb {{\bf S}}
\def\SS {{\cal S}}
\def\Tr {{\rm Tr\,}}
\def\V {{\cal V}}
\def\X {{\bf X}}
\def\XX {{\cal X}}
\def\Y {{\bf Y}}
\def\Z {{\bf Z}}

\def\a {\alpha}
\def\ac {{\bf a}}
\def\b {{\bf b}}
\def\bs {{\bf s}}
\def\c {{\bf c}}
\def\d {\delta}
\def\dist {{\tt \hbox{distance}}}
\def\e {{\bf e}}
\def\f {{\bf f}}
\def\finish {\blacksquare}
\def\g {{\bf g}}
\def\grad {{\rm grad\,}}
\def\h {\hbar}
\def\k {{\bf k}}
\def\l {{\bf l}}
\def\m {{\bf m}}
\def\n {{\bf n}}
\def\p {\partial}
\def\pb {{\bf p}}
\def\pt {{\bf pt}}
\def\q {{\bf q}}
\def\r {{\bf r}}
\def\s {\sigma}
\def\t {{\bf t}}
\def\tS {{\tilde \Sigma}}
\def\td {\tilde}
\def\v {{\bf v}}
\def\vare {\varepsilon}
\def\x {{\bf x}}
\def\y {{\bf y}}
\def\w {\omega}


\centerline{\bf Stirling+Fedoruk}

{\it Famous Stirling formula is a good problem which explains 
staionary phase method including its non-trivial part.
I deduce standard Stirling formula using Ted
Voronov's comments on Fedoruk paper}    

{\tt I returned to this text again in June 2020...(see another version
in math Blog on October 2019)}


Method of stationary phase...    Almost everybody knows that
it is about how to calculate the leading term in
in the integral
       $$
      \int\exp {i S(x)\over \hbar}u(x)dx  
       $$
if $\hbar \to 0$.   If $x_0$ is  unique
stationary point of function $S$,
         $$
	S(x)=S(x_0)+A_{ab}(x^a-x_0^a)(x^b-x^b_0)+S_+(x) 
	 $$
and Hessian at this point is not degenerate,	 then
              $$
      \int\exp {i S(x)\over \hbar}u(x)dx = 
	   C\exp {i S(x_0)\over \hbar}
	   {(\pi \hbar)^{n\over 2}\over \sqrt {\det A}}
	   \left(1+O(\hbar)\right)\,.
	   \eqno (01)
	       $$
People say that if $\hbar\to 0$ then
the main contibution to the integral is
given by vicinity of the stationary point, and this is just 
gaussian integral. 
	       
    What about other terms of expansion?

    There is deep and powerfull statement:
    All other terms are polynomial in $\hbar$!
    Namely
      $$
      \int \exp {iS(x)\over \hbar}u(x)dx=
	   C\exp {i S(x_0)\over \hbar}
	   {(\pi \hbar)^{n\over 2}\over \sqrt {\det A}}
	     $$
	     $$
	     \underbrace
	      {
   \left(
 \exp \left(-{\hbar\over 2i}
A^{ab}
{\p \over \p x^a}
{\p \over \p x^b}
\right)
\right)
\left(\exp {iS_+(x)\over \hbar}
\right)
      }_{I}
 \eqno (02)
      $$
   {\it whereas the expression $I$ is an expansion
   over  positive powers of $\hbar$.}, i.e. 
               $$
	       \hbox
{Only positive powers of $\hbar$ have imput in $O(\hbar)$}  
        \eqno (03)
             $$
 

On one hand everybody knows the statement (01),
on the other way the statement (02) is not simple, and it is
highly non-trivial.

    The best way to see this 
    statement it is to study famous Stirling formula.

    Recall that Striling formula says that
         $$
	N!=\left(N\over e\right)^N\sqrt {2\pi N}
	\left(1+{1\over 12 N}+{1\over 288 N^2}+\dots\right)
	\eqno (\hbox{Stirling})
	   $$
The "easy part" of Stirling formula, i.e. the statement
         $$
	N!\approx \left(N\over e\right)^N\sqrt {2\pi N}\,,\quad
	N!= \left(N\over e\right)^N\sqrt {2\pi N}
	\left(1+O\left(1\over N\right)\right)\,,
	   $$
is related with (01). Equation ({\rm Stirling})
is the 'difficult part'.
of Stirling formula. In particular studying this equation
it is impossible to avoid the question: why in formal series
appear only negative powers of $N$ (compare with (02,03)).


    \bigskip

   \centerline {Calculations of Stirling formula} 
    
    This is standard that  
        $$
N!=\int_0^\infty t^Ne^{-t}dt=
\int_0^\infty e^{-t+N\log t}dt\,.
       $$
Calculate it.

To use stationary phase method consider substitution
  $t=N(1+x)$ (The point $t=-N$ is stationary point, 
  we consider $t=N(1+x)$ not just $t=1+x$ to have 
  'big coefficient' in a vicinity of  stationary point)

      Thus we have:
           $$
	  N!=\int_0^\infty t^Ne^{-t}dt=
\int_0^\infty e^{-t+N\log t}dt=
\int_{-1}^\infty e^{-N(1+x)+
N\log\left[N\left(1+x\right)\right] }Ndx=
	   $$
       $$
\int_{-1}^\infty e^{-N(1+x)+N\log (N(1+x))}Ndx=
e^{-N}N^{N+1}\int_{-1}^\infty e^{N\left(
\log(1+x)-x\right)}dx=
         $$
$$
N\left(N\over e\right)^N\int_{-1}^\infty e^{-N
   \left(
  {x^2\over 2 }
- {x^3\over 3 }
 +{x^4\over 4 }+\dots
   \right)}dx=
N\left(N\over e\right)^N
\int e^{-N{x^2\over 2}}
     e^{
      N
     \left(
 {x^3\over 3 }
 -{x^4\over 4 }+\dots
     \right)
 }dx\,.
 \eqno (2)
      $$

To go further use identity:
      $$
  \sqrt {2\pi N}e^{-Nx^2\over 2}=
\int e^{-k^2\over 2N}e^{ikx}dk
\eqno (***)
       $$ 
 The meaning  of his identity see later
 (see equation {(\tt Fedoruk identity)})

This identity implies that
     $$
\int e^{-n{x^2\over 2}}
      \varphi(x)dx=
{1\over \sqrt {2\pi n}}
\int e^{-k^2\over 2n}e^{ikx}dkdx
\int \varphi (p)e^{ipx}dp=
   $$
   $$
{1\over \sqrt {2\pi n}}
\int  e^{-k^2\over 2n}e^{ikx}\varphi (p)e^{ipx}dpdkdx=
     $$
        $$  
{2\pi \over \sqrt {2\pi n}}
\int e^{-k^2\over 2n}\delta (k+p)\varphi (p)dpdk
=\int e^{-k^2\over 2n}\varphi (k)dk=
{2\pi \over \sqrt {2\pi n}}
 e^{-{1\over 2n}\left({d\over
dx}\right)^2}\varphi(x)\big\vert_{x=0}
        $$
Identity (***) allows us to rewrite (2)
in terms of differential operator.
[In fact doing this trick we formally we used due to Fedoruk
the following identity
      $$
\int f(x_0-x)u(x)dx=
\int \tilde f(k)
e^{ik(x_0-x)}
\tilde u(p)e^{ipx}dx dk dp=
  $$
  $$
\int \tilde f(k)
e^{ikx_0}
\tilde u(p)e^{i(p-k)x}dx dk dp
\int \tilde f(k)
e^{ikx_0}
\tilde u(p)\delta(p-k) dk dp=
  $$
  $$
\int \tilde f(k)
e^{ikx_0}
\tilde u(k) dk=
 f\left({1\over i}{d\over dx}\right)
      u(x)\big\vert_{x=0}\,.
      \eqno (\hbox {Fedoruc identity})
     $$
]

Namely apply identity (***)  to equation (2):
 $$
N!=
\left(N\over e\right)^N\sqrt {2\pi N}
\int_{-1}^\infty e^{-N
   \left(
  {x^2\over 2 }
- {x^3\over 3 }
 +{x^4\over 4 }+\dots
   \right)}dx=
   $$
   $$
\left(N\over e\right)^N\sqrt {2\pi N}
\int e^{-N{x^2\over 2}}
     e^{
      N
     \left(
 {x^3\over 3 }
 -{x^4\over 4 }+\dots
     \right)
 }dx=
     $$
     $$
\left(N\over e\right)^N\sqrt {2\pi N}
\int e^{-N{x^2\over 2}}
\underbrace
  {
     e^{
      N
     \left(
 {x^3\over 3 }
 -{x^4\over 4 }+\dots
     \right)
 }
 }_{\hbox {terms of order $\geq 3$ in $x$}}
 dx=
  $$
  $$
\left(N\over e\right)^N\sqrt {2\pi N}
\left\{
 e^{-{1\over 2N}\left({1\over i}
 {d\over dx}\right)^2}
    \left[
     e^{
      N
     \left(
 {x^3\over 3 }
 -{x^4\over 4 }+\dots
     \right)
 } 
\right]
\right\}_{x=0}\,.
     $$

... and here the mystery began: since the expression
$    \left[
     e^{
      N
     \left(
 {x^3\over 3 }
 -{x^4\over 4 }+\dots
     \right)
 } 
\right]$ possesses only the terms of the order $\geq 3$
in exponent the last integral possess only terms
which are proportional to $1\over N $:
         $$
	 {1\over N}\approx \hbar
	 $$
    $$
\int e^{-{1\over N}\left({d\over dx}\right)^2}
    \left[
     e^{
      N
     \left(
 {x^3\over 3 }
 -{x^4\over 4 }+\dots
     \right)
 } 
\right]dx=
1+{1\over 12 N}+{1\over 288 N^2}+\dots
    $$
Try to show it. We have 
  $$
  N!=
\left(N\over e\right)^N\sqrt {2\pi N}
\left(1+O\left({1\over N}\right)\right)\,,
   $$
   where

 $$
1+O\left({1\over N}\right)=
\left\{
 e^{-{1\over 2N}
  \left(
   {1\over i}
{d\over dx}
    \right)^2
      }
    \left[
     e^{
      N
     \left(
- {x^3\over 6 }
 +{x^4\over 24 }+\dots
     \right)
 } 
\right]
   \right\}_{x=0}=
     $$
    $$
\left[1
       -
      {1\over 2N}
  \left(
    {1\over i}
{d\over dx}
     \right)^2
     +
  {1\over 8N^2}
  \left(
    {1\over i}
{d\over dx}
     \right)^4
   -
  {1\over 48N^3}
  \left(
    {1\over i}
{d\over dx}
     \right)^6
         +\dots
\right]
    $$
      $$
  \left[
   1+
 N\left(
    {x^3\over 3}
     -
   {x^4\over 4}
    +\dots
    \right)
    +
     {1\over 2}N^2
 \left(
    {x^3\over 3}
     -
   {x^4\over 4}
      +\dots
    \right)^2
      +\dots
   \right]=
       $$
        $$
\left[1
       +
      {1\over 2n}
    {d^2\over dx^2}
     +
  {1\over 8n^2}
    {d^4\over dx^4}
        +
  {1\over 48n^3}
   {d^6\over dx^6}
         +\dots
\right]\,\,
     $$
       $$
  \left[
   1+
 n\left(
    {x^3\over 3}
     -
   {x^4\over 4}
     +\dots
    \right)
    +
     {1\over 2}n^2
 \left(
    {x^3\over 3}
     -
   {x^4\over 4}
      +\dots
    \right)^2
      +\dots
   \right]
      =
      \eqno (\hbox {})
       $$
       Calculate in detail 
contribution given by this expression.

First show that all terms which give contibution 
to this expression have negative degree by $N$.
  
  Consider  the action of the terms
   $\left[
  {1\over 2n}\left({1\over i}{d\over dx}\right)^2
    \right]^\lambda$
which act on the monoms
     $$
     F_{p_1}F_{p_2}\dots F_{p_r},
     $$
 where every $F_{p_k}$ is a monom of the order $p_k$,
  ($p_i\geq 3$)  which belong to  the term
     $$
   n\left(
   {x^3\over 3}
    -
   {x^4\over 4}
     +
  {x^5\over 5}
    -
   {x^6\over 6}
    +\dots
   \right)\,.
    $$
The condition that the action
of the operator 
   $\left[
  {1\over 2n}\left({1\over i}{d\over dx}\right)^2
    \right]^\lambda$
on the monom 
$F_{p_1}F_{p_2}\dots F_{p_r}$,
is not zero is the following:
    $$
   2\lambda=p_1+p_2+\dots+p_r\,,
     $$
and the power of $n$ corresponding to this monom is
equal to 
      $$
\left({1\over n}\right)^\lambda \cdot 
   n^r=n^{r-\lambda}\,.
       $$
Use this formula.

First of all note that all $p_i\geq 3$, hence
 $2\lambda\geq 3r$, hence
 $$
r-\lambda\leq r-{3r\over 2}<0\,.
 $$
Thus in the case of Stirling formula we proved hard part of
stationary phase method, showing that
 only negative powers of $N$
 (positive poweers of $\hbar\approx {1\over N}$)
 contirbute to the integral.


\bigskip


Now calculate the contribution of power ${1\over n^k}$
for $k=1,2,3\dots$.

We have that 
      $$
   \cases
{
2\lambda=p_1+\dots+p_r\cr
r-\lambda=-k\cr 
}\,, \quad  (p_i\geq 3)\,.
      $$
Hence
      $$
   \cases
      {
2\lambda=p_1+\dots+p_r\geq 3r\cr
\lambda=r+k\cr
     }\Rightarrow 2r+2k\geq 3r, \,\,{\rm i.e.}\,\,
   r\leq 2k
      $$
Thus we see that
 {\tt contribution to terms of order ${1\over n^k}$
is given by  action of
  $\exp{-{1\over n}\left({1\over i}{d\over dx}\right)^2}$
on terms which possess not more than $2k$ monoms
 

\medskip

Now calculate contribution to the term $1\over N$. 


I  Calculate contribution to ${1\over n}$, $k=1$:
      $$
   \cases
      {
2\lambda=p_1+\dots+p_r\geq 3r\cr
\lambda=r+1\cr
     }\Rightarrow r=1,2\,.
      $$
   $$
a) r=1\,,
   \cases
      {
2\lambda=p_1\cr
\lambda=r+1=2\cr
     }\,,p_1=4\,,\quad 
b) r=2\,,  \cases
      {
2\lambda=p_1+p_2\geq 3r\cr
\lambda=r+1=3\cr
     }\,,p_1=p_2=3,  
   $$
According to these calculations and to
equation (contribution),
the contibution is equal to:
       $$
     {1\over 8N^3}
     \left({1\over i}{d\over dx}\right)^4
     \left[n\left(-{x^4\over 4}\right)\right]
     +
     \left(-{1\over 48 N^3}\right)
     \left({1\over i}{d\over dx}\right)^6
     \left[
 {1\over 2}N^2\left(x^2\over 3\right)^2
   \right]={1\over 12 N}
       $$
Thus
      $$
    N!=\left(N\over e\right)^N\sqrt {2\pi N}
    \left(1+{1\over 12 N}
        +O\left(1\over N^2\right)\right)\,.
      $$
\bye

I  Calculate contribution to ${1\over n^2}$, $k=2$:
      $$
   \cases
      {
2\lambda=p_1+\dots+p_r\geq 3r\cr
\lambda=r+2\cr
     }\Rightarrow r=1,2,3,4\,.
      $$
  $$
a) r=1\,,
   \cases
      {
2\lambda=p_1\cr
\lambda=r+2=3\cr
     }\,,p_1=6\,,
      $$
       $$ 
b) r=2\,,  \cases
      {
2\lambda=p_1+p_2\geq 3r\cr
\lambda=r+2=4\cr
     }\,,2\lambda=8,\, p_1=3, p_2=5 \,\, {\rm or}\,\,
p_1=p_2=4\,, 
     $$
 $$ 
c) r=3\,,  \cases
      {
2\lambda=p_1+p_2+p_3\geq 3r\cr
\lambda=r+2=5\cr
     }\,,2\lambda=10\,, p_1=p_2=3, p_3=4\,,
   $$
 $$ 
d) r=4\,,  \cases
      {
2\lambda=p_1+p_2+p_3+p_4\cr
\lambda=r+2=6\cr
     }\,,2\lambda=12\,, p_1=p_2=p_3=p_4=3\,,
   $$

On the base of these considerations calculate $n!$ up to
the terms $1\over n$.
   We have  according previous considerations that

     $$
n!=\exp \left(-{1\over 2n}\left(
            {1\over i}{d\over dx}\right)^2\right)
       \exp\left(n\left({x^3\over 3}-{x^4\over 4}
 +{x^5\over 5}+\dots\right)\right)_{x=0}=
         $$
         $$
\left(n\over e\right)^n\sqrt {2\pi n}
   \left[
     1+
  {1\over 8n^2}
    {d^4\over dx^4}
     \left(
     -n{x^4\over 4}
      \right)+
  {1\over 48n^3}
    {d^6\over dx^6}
     \left(
    +{1\over 2}n^2{x^6\over 9}
      \right)
        +
       \dots
    \right]=
      $$
        $$
        $$

\bye
