
%\magnification 1200

\def\V {{\cal V}}
\def\s {{\sigma}}
\def\Q {{\bf Q}}
\def\D {{\cal D}}
\def\G {{\Gamma}}
\def\C {{\bf C}}
\def\M {{\cal M}}
\def\Z {{\bf Z}}
\def\U  {{\cal U}}
\def\H {{\cal H}}
\def\F {{\cal F}}
\def\grad {{^{\circ}}}
\def\S {{\Sigma}}
\def\s {{\sigma}}


   \centerline{\bf On one properties of discriminants}

  Let $$P(x)=x^n+a_{n-1} x^{n-1}+a_{n-2} x^{n-2}+\dots+a_1x+a_0 \eqno (1)$$ be a polynomial where
  $a_i$ are indeterminants over $\C$.


  Consider its derivative a polynomial
     $$Q(x)=nx^{n-1}+(n-1)a_{n-1} x^{n-2}+\dots+a_1 \eqno (2)$$.
     It is convenient to consider  fields
                       $$
     \L=\C(a_1,a_2,\dots,a_{n-1})\quad{\rm and}\qquad
       K=\L(a_0)=\C(a_1,a_2,\dots,a_{n-1},a_0)\eqno (2)$$

 Consider also {\it discriminant} of the polynomial $P\in K[x]$ (1)
 i.e. resultant of the polynomials $P(x)$ and $Q(x)$ over the
 field $K$ (polynomial $Q\in L[x]\subseteq K[x]$). If
 $\mu_i, (i=1,\dots,n)$---roots of the polynomial $P(x)$ then
               $$
     D=\prod_{i\not =j} (\mu_i-\mu_j)
               $$
 Discriminant is a resultant of the polynomial and its derivative
 $Q$:
  \def\a{\alpha}
 if $\delta_\a$ ($\a=1,\dots,n-1$) are roots of derivative $Q$ then
 up to a coefficient $$
 D(P)=R(P,Q)=\prod_{i, \a} (\mu_i-\delta_\a)=\prod_{\a}P(\delta_\a)=\prod_{\a}Q(\mu_i)
          $$


   One can see that discriminant is a polynomial of degree $n-1$
   with respect to indeterminant $a_0$.
   It is convenient later to denote the indeterminant $a_0$ by the
   letter $y$ and consider {\it discriminant} as a polynomial with respect to $y=a_0$ over the
   field $L$:
               $$
      D=d(y)=\prod_{\a}P(\delta_\a)=
      \left(y+a_1\delta_1+\dots+a_{n-1}\delta_1^{n-1}+\delta^n_1\right)\cdot\dots\cdot
      \left(y+a_1\delta_1+\dots+a_{n-1}\delta_1^{n-1}+\delta^n_1\right)=
                       $$
                       $$
      y^{n-1}+\dots+(-1)^{?}a_1^{n}
                   $$
Polynomial $d(y)$ is an irreducible polynomial over field $L$ as
well as derivative polynomial $Q(x)$

    Consider polynomial
           $$
           G(x)=-\int_0^x Q(u)du=y-P(x)
               $$
  If $$y=y_i=F(\delta_i)$$ then $\delta_i$ is a root of the
  polynomial $P(x)$. Hence polynomial $P$ and its derivative has joint root,
  i.e.  $$d(y_i)=0 \quad{\rm if}\quad y_i=F(\delta_i)$$

  We see that $y_i$ belongs to the field $L(\delta_i)$. $D(y)$ is irreducible,
  hence there exist a polynomial $F(y)$ such that
              $$
          \delta_i=F(y_i)
              $$


   {\bf Proposition}.

     Fields $\L_D=\L[x]/(Q(x))$ and $\L_Q=\L_D=\L[y]/(Q(y))$ are
     isomorphic,
     i.e. adding to the field $\L$ one of the roots $\delta_i$ of derivative
     polynomial $\Q(x)$ leads to the same field as adding to the
     $\L$ a number $y_i=-P_0(\delta_i)$.





     It follows from the Proposition that
                  $$
                  P(x,y)=(x-F(y))^2 P_{n-2}\quad \hbox{projected on the field $\L_D$},i.e.
                  \eqno (main)
                  $$
or in the other words

 Let $I=<P(x,y),d(y)>$ be an ideal generated by polynomials $P,d$.


     Polynomial $F(y))$ considered above is defined modulo $D(y)$, respectively
     polynomial $G(x)$ is defined modulo $Q(x)$.
     We can take their degree $\leq n-1$.





\bye
