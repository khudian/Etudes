\def\p{\partial}
\def\t {\tilde}
\def \m {\medskip}
\def\degree {{\bf {\rm degree}\,\,}}
\def \finish {${\,\,\vrule height1mm depth2mm width 8pt}$}





\def\a {\alpha}
\def\vare{{\varepsilon}}
\def\l {\lambda}
\def\s {{\sigma}}

\def\G {{\Gamma}}

\def\A {{\bf A}}
\def\C {{\bf C}}
\def\E  {{\bf E}}
\def\K {{\bf K}}
\def\N {{\bf N}}
\def\Q {{\bf Q}}
\def\R  {{\bf R}}
\def\V {{\cal V}}
\def \X   {{\bf X}}
\def \Y   {{\bf Y}}
\def\Z {{\bf Z}}



\def\ac {{\bf a}}
\def\e{{\bf e}}
\def\f {{\bf f}}
\def\n {{\bf n}}
\def\r {{\bf r}}
\def\v {{\bf v}}
\def \x   {{\bf x}}
\def \y   {{\bf y}}


\def\pt {{\bf pt}}


\centerline {\bf On Cayley formula}


\medskip


 Let $G$ be a group  of matrices which preserve the matrix $M$:
             $$
         G=\{Y\colon \,\,\, YMY^+=M\}\,,
             $$
and let $\cal G$ be  a Lie algebra of these group
       $$
    {\cal G}=\{X\colon \,\,\, XM+MX^+=0\}\,,
       $$

Let $Y(X)=G(X)$ be a formal polynomial on $X$, $G=c_kX^k$ such that
               $G(X)$ belongs to group $G$ if $X\in \cal G$



{\bf Observation}  All polynomials
      $$
  G(X)=G(X)={F(X)\over F(-X)}
      $$
belong to the group if $X$ belongs to the algebra.


 If $M$ is non degenerate matrix then all formal
polynomials $G(X)=c_0+\dots$ with $c_0\not=0$ are given by this
formula.

\medskip
 The proof follows from the fact that
  $G(X)G(-X)=1$.


{\bf Remark} If $F(x)$ is rational fraction we come to Cayley
formula, if $F(x)=\exp X$ we come to exponent.

 \bye
