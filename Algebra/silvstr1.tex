

   \centerline {\bf On  number of real roots. (Silvester Theorem)}


   \bigskip

    {\bf Theorem}   The number of real roots of the polynomial
    $f=x^n+a_{n-1}+\dots+a_0$ with real coefficients $a_0,\dots,a_{n-1}$
    is equal to the signature
    of the quadratic form given by the $n\times n$ matrix
    $a_{ij}=s_{i+j-2}$ where $i,j=1,\dots,n$ and
    $s_k=x_1^k+\dots+x^k$ are Newton polynomials
    polynomials on coefficeints ($s_1=-a_{n-1}$, $s_2=a_{n-1}^2-2a_{n-2},\dots$)
        (  $\{x_1,\dots,x_n\}$ is the  set of complex roots of this polynomials.)

  I know the very beautiful proof of this Theorem.
  (It comes from Prasolov + ......)


   Assume that all roots are distinct.
   Consider the set of polynomials
   $\{h_i(x)\}$ of degree $\leq n-1$ such that polynomial
   $h_{i}$ is equal to $1$ at the root  $x_i$ and it is equal to zero
   at all other roots,which are equal to $1$:
                          $$
              h_i(x)={f(x)\over (x-x_i)f'(x_i)}
                        $$
                        (In general roots are complex and these polynomials are complex)
Note that polynomials $\{h_i(x)\}$ are the base of Lagrange interpolation formula:

For every polynomial $p$ of degree $\leq n$
                             $$
            p(x)\equiv p(x_i)h_i(x)
                             $$
   These formulae play the role of China reiminders isomorphism
   on the ring of polynomials)


Now consider complex $n$-dimensional vector space $V$ of all complex polynomials
factorised by $f$.  The set of polynomials ${h_i(x)}$ is the basis in this space. The
components $(p_1,\dots,p_n)$ of every polynomial with respect of this basis is just the
values of these polynomials at roots: $p_i=p(x_i)$ (according Lagrange interpolaion
formula)

Every element of this space defines linear operator $L_g: L_gp=gp$ (modulo $f$)

Consider the symmetric bilinear form   $A$
 such that its value on every pair $g,r$ is equal to the trace of the
 of the operator $L_{gr}$. It is evident that basis $\{h_i\}$
 is orthonormal basis with respect of this form because $h_ih_j\equiv 0$ if $i\not=j$:
                        $$
                        A(h_i,h_j)=\delta_{ij}
                 $$
 Hence  we come to the formula
                   $$
       A(g,r)=g_1r_1+\dots+g_nr_n=\sum_{i=1}^ng(x_i)r(x_i)
                            $$
It follows from this formula that
                  $$
              A(x^p,x^q)=\sum_{i=1}x_i^{p+q}=s_{p+q}
                     $$
We see that matrix  $a_{ij}=s_{i+j-2}$ is just the matrix of symmetri bilinear form $A$
in the real basis $\{1,x,x^2,\dots\}$.

Now suppose that $2q$ roots of this polynomial are complex and the rest $r-2q$ are
real. Thus $2q$ polynomials $h_1,\dots h_{2q}$ are complex  and the rest are real.



Consider the real basis $\{a_1,b_1,\dots,a_q,b_q,h_{q+1,\dots,h_n}\}$, where
$h_1=a_1+ib_1$, $h_2=a_1-ib_1$, $h_3=a_2+ib_2$, $h_4=a_2-ib_2,\dots$. Since
$A(h_i,h_j)=\delta_{ij}$ hence
                     $$
   A(a_i,a_j)={1\over 2}\delta_{ij},\quad A(a_i,b_j)=0,\quad
   A(b_i,b_j)={-1\over 2}\delta_{ij}
                      $$
                      Hence the signature of this form
                      is equal to $n-2q$. It is just
                      equal to the number of real roots.

 \bye
