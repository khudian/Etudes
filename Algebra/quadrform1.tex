

   Privet Davidik!

   %Eto obychnyj Tex-file,. Mozheshj chitatj  s ekrana
   % a mozheshj i kompilirovatj


     Pustj   $B(x,y)$ bilinejnaja symmetricheskaja forma na linejnom prostranstve $V$
     (razmerrnostj konechnaja).

     Ej sootvetstvujet
           kvadratichnaja forma
                        $$
                 Q_B(x)=B(x,x)
                   \eqno (1)
                        $$
  kotoraja kak legko ponjatj udovletvorjajet pravilu
  parallelogramma:
                $$
                Q(x+y)+Q(x-y)=2Q(x)+2Q(y)
                        \eqno (2)
                 $$
i odnorodna:
              $$
              Q(kx)=k^2Q(x)
              \eqno (3)
              $$

  Esli $B(x,y)$ polozhiteljno opredelena to legko proveritj (legko?)
  forma $Q(x)$ opredeljajet normu:
                   $$
                   ||x||=\sqrt {Q(x)}
                   \eqno (4)
                   $$
Eta norma kak i sledujet ozhidatj udovletvorjajet vsem uslovijam
normy: $||x||\geq 0$ i $||x||=0$ iff $x=0$  v silu polozhiteljnoj
opredeljonnosti $B$. V silu neravenstva Cauchy-bunyakovskogo
vypooljanejtsa i neravenstvo treugoljnika:
               $$
||x+y||\leq ||x||+||y|| \eqno (5)
               $$
To estj ponjatjije rasstojanija ne protivorechit nashej
intujitsiji.


A teperj vopros:  Kakim uslovijam dolhzna udovletvorjajt $Q(x)$
shtoby ej sootvetstvovala bilinejnaja forma.

 Legko ponjatj shto uslovija  (3) i (2) neobkhodimy:
 Esli $B$ sootvetstvujushjee $Q$ (to estj (1) imejet mesto bytj)
 sushestvujet to nikuda ne deneshsja,  $B(x,y)$ dolzhno
 opredeljajtjs po sledujushej formule:
              $$
              B(x,y)={1\over 2}\left(Q(x+y)-Q(x)-Q(y)\right)
              \eqno (6)
               $$
ili mozhno takoj formuloj:
             $$
              B(x,y)={1\over 4}\left(Q(x+y)-Q(x-y)\right)
              \eqno (7)
             $$

  A dostatochny li oni?  Vot v chom vopross... Ili po drugomu:
  a pravda li shto formula (6) ili (7)
  vydajut na gora bilinejnuju simmetricheskuju formulu.

  To shto (6) i (7) vydajut simmetricheskij objekt--ochevidno.
  To shto oni vydajut odin i tot zhe objekt: eto tozhe ochevidno
  (is pravila parallelogramma)


Nemnogo pomuchivshisj (ja vsegda zabyvaju kak, eto golovnaja bolj
no mozhno ) mozhno pokazatj (po-mojemu ispoljzuja lishj praivlo
parallelogramma bez uslovija odnorodnosti (3)) shto $B(x,y)$
opredeljonnoje cherez udovletvojrjajet tak nazyvajemu usloviju
additivnosti:
               $$
B(x+x',y)=B(x,y)+B(x',y) \eqno (8)
               $$
Kazalosj by eto pochti uzhe vsjo: Vedj iz  (8) i izhe  vytekajet
shto dlja ljubogo tselogo $m$
           $$
   B(mx,y)=mB(x,y)
           $$
Zametim tut tozhe odnorodnostj ne ispoljzujetjsa.

A teperj legko ponjatj shto dljaljubogo ratsionaljnogo $a$
       $$
       B(ax,y)=aB(x,y)
       \eqno (9)
       $$
My prishli k vyvodu: shto ljubaja forma $Q(x)$ udovletvorjajusja lishj
pravilu paralellogramma  sootvetstvujet objektu $B(x,y)$
kotoryj linejej nad ratsionaljnymi chislami, to estj:
                   $$
 B(ax+bx',y)=aB(x,y)+bB(x',y)
 \eqno (10)
                   $$
                   \medskip
  (Kontroljnyuj vopros v storonu: Mnohestvo dejstviteljnykh chissel javljajetjsa
  beskonechno-mernym prostranstvom nad mnozhestvom ratsionaljnykh chissel.)
\medskip

  A teperj: kakoje usovije nuzhno nalozhitj na formu $Q(x)$
  shtob $B(x,y)$ v (6,7) bylo by linejnym nad ljubymi dejstviteljnymi.

  Otvet? Praivlo treugoljnika: esli funtksija $Q(x)$ indutsirujet normu,
  (udovletvojrjaet praivlam odnorodnosti polozhiteljnoj opredeljonnosti
  i pravilu treugoljnika)
  i udovletvorjajet praivlu prallelogramma to $B$ v (9) linenjno.

  Dokazateljstvo: Esli $Q$ dejstviteljno opredeljajet normu, to functsija
         $f(x)=B(x,y)$ gde $B$ (oppredeljajetjs (9))
         nepreryvna po otnosheniju k etoj norme, a znachit ravenstvo
         (9) vypolnjajetsja dlja vsekh dejstviteljnykh chissle
         kolj skoro ono vypolnjajetjsa dlja ratsionaljnykh. Vsjo!!!!

         Strannoje oshushenije ostajotsja posle etogo dokazateljstva?

         A prichom tut nepreryvnostj. MY zhe dokazyvajem linejnostj.

          Na samom dele slovo nepreryvnostj ochenj sootvetstvujet nashej intuiitsjii
          i zashifrovyvajet soboju kuchu ponjatij.

  Mozhno formaljno ne govoritj slovo nepreryvnostj
  i pritvorjajsja shto ego ne znajeshj chestno i tupo otsenivatj
dlja ljubogo dejstviteljnogo $a$ rasstojanije mezhdu $B(ax,y)$
i $B(a_nx,y)$, gde $\{a_n\}$ posledovateljnostjh ratsionaljnykh chissel
kotorajaj stremitsja (opredeljajet???) dejstviteljnoje chislo $a$.
Da eto tak..

Kazalosj by vysheprivedjonnoje rassuzhdenije velikolepnoje podtverzhdenije tezisa:
         ""Mavr sdelal svojo delo, mavr mozhet umeretj"
My vospoljzovalisj lsovom nepreryvnostj a potom vybrossili.



No tut zhe voznikajet drugoj vopros? A naskoljko vsjotaki nuzhno uslovje normy?
Mogli by my bez nego obojtisj?

Ja nikak ne mogu pridumatj kontrpirmera, to estj primera posledovateljnosti
kotorjaja udovletvorjajet uslovijam parallelogramma i odnorodnosti,
no ne privodit k norme potomu shto ne udovletvorjajet usloviju treugoljnika
(Mozhet bytj dlaj kontrpirmera nuzhno rassmotretj objekt kotoryj udovletvojrjaet
lishj pravilu parallelgoramma, ne znaju....)

I esho vopros: Nepreryvnostj kotooj my poljzovalisj
vkljcuhajet v sebja kuchu trebovanij v tom chisle i trebovanije
na polozhiteljnuju opredeljnoostj formy.
Kazalosj by dlja svjazi (1) eto ne nuzhno.


V zakljuchenijii dva dopolnenija:

  $$ $$

  Istorija voprosa: Vopros voznik tri goda tomu nazad u mojego druga.
  Togda my byli pomolozhe i ponaivnejee. Ochenj bystro my dokazali
  svojstvo  (10) i vrode by togda ja lego dokazal shto dlja konechnomernykh
  prostranstv vsjo prosto i vsja eta bodjaga s neraventsvom treugoljnika ne nuzhna.
  Problemy voznikajut v beskonechnomeriji. No ja nikak ne mogu eto vosstanovitj.
  Mozhet ja ne prav byl, a mozhet bytj sejchas psotarel...



  No sdrugoj storony my togda zastrjali na (10) i daljehse ne poshli.
  Nam uslovije nepreryvnosti kazalosj vneshnim doponiteljnym trebovanije.

  Potom na menja svet prolilsja (eto bylo dva goda tomu nazad)
  kogda ja ponjal shto nepreryvnostj opredeljajetsja v terminakh
  samoj formy!!!!
  No ja nikak ne mogu privesti kontrpirmer.

  Shob ponjatj shto ne vsjo gladko v datskom korolevstve.

  Predlagaju sledujushee kazalosj by bezobidnoje uprazheninie:
  uprazhnenije


   $$ $$

   Zadacha: Privesti primer funcktsiji $F(x)$ gde $x$ proizovljnoje dejstviteljnoje chislo
   tak shto $F(x)$ udovlet\-vo\-rjajet usloviju additivnosti:
                     $$
                     F(x+x')=F(x)+F(x')
                   $$
     no linejnoj ne javljajetjsa !!!!

     Okazyvajejtsja eto ne ochenj legko. V kakom-to smysle reshenije
     nevozmozhno bez aksiomy vybora:

      Reshenije: Rassmotrim $R$ kak linejnoje porstranostvo
      nad $Q$ i vvedjom v njom basis $\{e_i\}$ tak shto pervyj element $e_1=1$
       (zametim shto indeks $i$
      probegajet neschotnoje mnozhestvo znachenij.) (Takoj bazis nazyvajetjsa
      bazis Gamelja????)

   Teperj jasno shto ljuboje chislo odnoznachno mozheno predstavitj vvide
   summy ratsionaljnogo i irratsionaljnogo:
                    $$
                    x=x^1+x'
                    \eqno (11)
                    $$
                    Imenno dlja etogo nuzhen bazis Gamelja!

Zametim shto bez basia Gamelja razlozhenije (11) ne poluchitsja,
a dlja basisa Gamelja nuzhna transfinitnaja induktsija, to estj aksioma vybora!!!!

A teperj s uchotmo (11) stroim: $F(x)=x^1$

Jasno stho eta funtsija udovletvorjajet usloviju additivnosti
  Na ratsionaljnykh ona ravna $x$
i na mnogikh irratsionaljnykh chislkah (no ne na vsekh!)
ona ravna nulju.


Vidno shto eto uzhasnyj objekt....

Nu ladno...

  \bye
