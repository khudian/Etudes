
% this text I began in October 2012

\def\p{\partial}
\def\t {\tilde}
\def \m {\medskip}
\def\degree {{\bf {\rm degree}\,\,}}
\def \finish {${\,\,\vrule height1mm depth2mm width 8pt}$}





\def\a {\alpha}
\def\vare{{\varepsilon}}
\def\l {\lambda}
\def\s {{\sigma}}

\def\G {{\Gamma}}

\def\A {{\bf A}}
\def\C {{\bf C}}
\def\E  {{\bf E}}
\def\K {{\bf K}}
\def\N {{\bf N}}
\def\Q {{\bf Q}}
\def\R  {{\bf R}}
\def\V {{\cal V}}
\def \X   {{\bf X}}
\def \Y   {{\bf Y}}
\def\Z {{\bf Z}}



\def\ac {{\bf a}}
\def\e{{\bf e}}
\def\f {{\bf f}}
\def\n {{\bf n}}
\def\r {{\bf r}}
\def\v {{\bf v}}
\def \x   {{\bf x}}
\def \y   {{\bf y}}


\def\pt {{\bf pt}}



\centerline {\bf On Jordan normal form}

  {\it you may find in any good textbook of linear algebra 
what is it Jordan normal form.
  I discussed with my students appearance of Jordan normal form 
in many other places, in
  particular in decomposition of finite Abelian groups. 
Here is the exposition of standard linear algebra subject from 
this point of view. }


\medskip

Consider a pair $(A,V)$ where $A$ is a linear operator on vector space
 $V$. If $V$ is vector field over algebraically closed field
(e.g. $\C$, we consider only this case) 
then $A$ has an eigenvector, i.e. $V$ possesses 
one-dimensional subspace $L_A$ invariant with respect to operator $A$.


{\bf Definition} Let $A$ be a non-zero linear operator acting on 
finite-dimensional vector space $V$.
We say that the pair $(A,V)$ is Jordan cell  if  the vector 
space $V$ possesses unique 
one-dimensional invariant subspace.

\m

\m

{\bf Example} Let $J^{\lambda,n}=(A,\R^n)$ be a pair of 
$n$-dimensional arithmetic vector space
$\R^n$ and an operator $A$ such that in the canonical basis 
$\{\e_1,\dots,\e_n\}$
                          $$
   A\e_1=\l\e_1,\quad {\rm and}\,\, A_{\e_i}=\lambda\e_i+\e_{i-1}\quad
   \hbox {for $i=2,\dots,n$}
                       $$
   It is easy to see that this is Jordan cell. Indeed let 
$\f=x^i\e_i$ be a vector which defines
   invariant subspace: $\f\not=0$ and $A\f=\mu \f$, i.e.
                   $$
   A(x^i\e_i)=x^1\l\e_1+x^2(\l \e_2+\e_1)+\dots+x^n(\l\e_n+\e_{n-1})=\mu x^i\e_i
      $$
 This implies that $x_i=0$ if $i=2,\dots,n$, i.e. $\f$ 
is proportional to $\e_1$. We see that there is a unique invariant subspace.



   Note that if $(A,V)$ is a Jordan cell and $L$ is invariant subspace
then  the sequence
        $$
    0\rightarrow L\rightarrow V\rightarrow [V]\rightarrow 0
        $$
is not splitted with respect to operator $A$, i.e. every $n-1$-dimensional
subspace in $V$ is not invariant with respect to operator $A$.

\m

 {\bf Statement A} If $V$ is Jordan cell with respect to 
linear operator $A$ acting on it, then
 $(V,A)$ is isomorphic to $J^{\l,n}$  for some number $\l$ and natural $n$.


 \m
 
 This statement can be formulated in the following equivalent way:

  {\bf Statement A$^\prime$}  Let n-dimensional vector space $V$ 
be a Jordan cell with respect to operator $A$. 
  Then there exists a number $\l$  
  and a flag of invariant subspaces $\{V_k\}$, $k=1,\dots,n$  such that  
  every $V_k$ provided with action   of $A$ on it is isomorphic to $J^{\l,k}$.
  
  This flag is unique.
  
   It is easy to see that this statement implies 

{\bf Proposition A$^{\prime\prime}$}  Let $A$ be $N$-dimensional 
space where acts an operator $A$. Then
       $V$ is a direct sum of Jordan cells.

   \m
   
   
   Prove the statement by induction on dimension.
   
   For $n=1$ it is alright. Suppose it is true if ${\rm dim\,}V=k$. 
     Prove it for
   ${\rm dim\,}V=k+1$.   

  It is illuminating first to consider  the case $k=1$.
   
   Let $V$ be two-dimensional Jordan cell with respect to operator $A$.
  Let $\e_1\not=0$ be an eigenvector of operator $A\colon A\e=\l\e$.
 By definition of Jordan cell it is unique (up to  a coefficient).


  Consider an arbitrary non-zero vector $\f$ which is not 
  colinear to the vector $\e$.
  Let $Af=\mu \e+a\f$. Vector $\f$ is not eigenvector since 
  $V$ is Jordan cell. Hence $a\not=0$.
  By changing $\f\to {1\over a}\f$ we come to $A\f=\mu\f+\e$. 
  Show that $\mu=\l$. If this is not the case then
 the vector $\f'=\f+{\e\over \mu-\l}$ 
is second eigen-vector of operator $A$:
           $$
  A\f'= A\left(\f+{\e\over \mu-\l}\right)=\mu\f+{\l\e\over\mu-\l}=
      \mu\left(\f+{\e\over \mu-\l}\right)=\mu\f'\,.
      $$
 Contradiction. Hence $\mu=\l$ and $A\f=\l\e+\f$.
Thus pair $(V,A)$ is Jordanian cell $J^{\l,2}$.


Now consider general case. Let $V$ ne $n+1$-dimensional Jordanian cell, 
$n\geq 1$.
Let $\e_1\not=0$ be an eigenvector of operator $A\colon A\e=\l\e$.
 By definition of Jordan cell this is unique (up to  a coefficient)
eigenvector.
 
 Consider $n-1$ vector space $[V]=V\backslash L_{\e}$, 
  where $L_{\e}$ is one-dimensional invariant
 subspace spanned by the vector $\e$. 
$[V]$ is Jordan cell too with respcet to the action of operator $[A]$.
 and dimension of this Jordanian cell is equal to $n$. 
Hence by inductive hypothesis, it is isomorphic to $J^{\mu,n}$ 
for some eigenvalue $\mu$, i.e. there exists a basis 
 $\{[\f_1],[\f_2],\dots [\f_n]\}$ in the factor-space $[V]$ such that
         $A[\f_1]=\mu[\f_1]$ and $A[\f_{i+1}]=\mu[\f_{i+1}]+[\f_i]$.
  This means that for vectors $\{\e,\f_1,\dots,\f_n\}$
                   $$
                 \cases
               {
        A\e=\l\e\cr
          A\f_1=\mu\f_1+a_1\e\cr
         A\f_2=\mu\f_2+\f_1+a_2\e\cr
         A\f_3=\mu\f_3+\f_2+a_3\e\cr
         A\f_4=\mu\f_4+\f_3+a_4\e\cr
        \dots\dots\dots\dots\cr
         A\f_n=\mu\f_n+\f_{n-1}+a_n\e\cr
               }
                   $$
Consider the first two equations in the system above.
We already showed that $a_1\not=0$ 
and $\mu=\l$ since if $a_1=0$ or $\l\not=\mu$ then the operator
$A$ possesses second eigen vector in the span of vectors $\e$ and $\f$
(see the case $k=2$ above). 

Hence by multiplying  all the vectors $\f_i$ by $1\over a_1$  we come to
the equations
                $$
                 \cases
               {
        A\e=\l\e\cr
       A\f_1=\l\f_1+\e\cr
         A\f_2=\l\f_2+\f_1+a_2\e\cr
         A\f_3=\l\f_3+\f_2+a_3\e\cr
         A\f_4=\l\f_4+\f_3+a_4\e\cr
        \dots\dots\dots\dots\cr
         A\f_n=\l\f_n+\f_{n-1}+a_n\e\cr
               }
                   $$
  Now one can see (again) by induction that 
that we have the flag of spaces $\{V_k\}$ such of these spaces is $J^{\l,k}$.

Indeed consider spaces $V_k$ which are spans of vectors 
$\e,\f_1,\dots,\f_{k-1}$. All these spaces are Jordanian cells with same 
eigenvalue $\l$.  By induction hypothesis  there exists a basis
$\e,\f'_1,\f'_2,\dots,\f'_{n-1}$ in $V_{n-1}$ such that
the 
          $$
      \cases
               {
        A\e=\l\e\cr
       A\f'_1=\l\f'_1+\e\cr
         A\f'_2=\l\f'_2+\f'_1\cr
         A\f'_3=\l\f'_3+\f'_2\cr
         A\f'_4=\l\f'_4+\f'_3\cr
        \dots\dots\dots\dots\cr
         A\f'_{n-1}=\l\f'_{n-1}+\f_{n-2}\cr
               }\,.
            $$
Then for vector $\f_n$
        $$
      A\f_n=\l \f_n+b_0\e+b_1\f'_1+
  b_2\f'_2+b_3\f'_3+\dots+b_{n-1}\f'_{n-1}
       $$

If we choose 
        $$
     \f'_n=\f_n-b_0\f_1+-b_1\f_2-\dots
        $$
we come to the answer\finish         


      \centerline {Jordanclells in Abelian group classification}

Consider the abelian finite group 
         $$
       C_k=\Z\backslash k\Z
         $$

Then if $k=p^2$ ($p$ is prime number) then $C_k$ is Jordanian cell.
The group $C_{p^n}$ possesses subgroup $C_p$ and the sequence
       $$
  0\rightarrow C_p\rightarrow C_p^2\rightarrow C_p\rightarrow 0
       $$ 
is not splitted.


  Standard classfication theorem says  
        that $C_N$ is direct sum of the groups
   $C_{p_k^{n_k}}$ where $N=\prod_i p_i^{n_i}$ is expansion over primes
This is expansion on Jordanian cells.

   $C_{p^k}$ is ``analog'' of $k$-dimensional  Jordanian cell. 
\bye
