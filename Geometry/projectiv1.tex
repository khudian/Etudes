% I commenced  this version on 7-th  March 2009
\magnification=1200 %\baselineskip=14pt
\def\vare {\varepsilon}
\def\A {{\bf A}}
\def\t {\tilde}
\def\a {\alpha}
\def\K {{\bf K}}
\def\N {{\bf N}}
\def\V {{\cal V}}
\def\s {{\sigma}}
\def\S {{\Sigma}}
\def\s {{\sigma}}
\def\p{\partial}
\def\vare{{\varepsilon}}
\def\Q {{\bf Q}}
\def\D {{\cal D}}
\def\G {{\Gamma}}
\def\C {{\bf C}}
\def\M {{\cal M}}
\def\Z {{\bf Z}}
\def\U  {{\cal U}}
\def\H {{\cal H}}
\def\R  {{\bf R}}
\def\E  {{\bf E}}
\def\l {\lambda}
\def\degree {{\bf {\rm degree}\,\,}}
\def \finish {${\,\,\vrule height1mm depth2mm width 8pt}$}
\def \m {\medskip}
\def\p {\partial}
\def\r {{\bf r}}
\def\v {{\bf v}}
\def\n {{\bf n}}
\def\t {{\bf t}}
\def\b {{\bf b}}
\def\c {{\bf c }}
\def\e{{\bf e}}
\def\ac {{\bf a}}
\def \X   {{\bf X}}
\def \Y   {{\bf Y}}
\def \x   {{\bf x}}
\def \y   {{\bf y}}
\def \G{{\cal G}}
\def\w{{\omega}}
\centerline  {\bf  Projective transformations:
pedestrian's point of view}

\bigskip

\centerline  {\bf $\cal x$0. Projective transformations}

\bigskip

   The group of projective transformations of $n$-dimensional projective space $\R P^n$ is  $PGL(n,\R)$.
   It is
   is a factor of $GL(n+1,R)$ by matrices which do not move any line, i.e.
   it is the factor of $SL(n+1,R)$ by the centre of this group.
  This is immediate consequence of definition of projective
  space as set of  lines in $\R^{n+1}$ passing through the origin---
   the linear transformations in $GL(n+1,\R)$ induces projective
   transformations.  If $[x^0:x^1:\dots:x^n]$---homogeneous
   coordinates in $\R P^n$  and $u^1,u^2,\dots,u^n$ not-homogeneous coordinates
   in $\R P^n$  such that
              $$
           u^i={x^n\over x^0},\qquad i=1,\dots,n,\,\,x^0\not=0
              $$
   (in a chart of $\R P^n$) then projective
   transformation in homogeneous coordinates is
                  $$
              x^\mu\mapsto A^\mu_\nu x^\nu\quad (\mu,\nu=0,1,2,\dots,n)
              \eqno (0.2)
                  $$
      and in nothomogeneous coordinates it will be
                    $$
           u^i\mapsto {A_k^i u^k+t^i\over 1+\w_ru^r}
                    \eqno (0.3)
                    $$

 The definitions of projective group follows from (1).


    Let us stand on  naive point of view,.  Projective transformations
    has to be understood  in the following way:
                $$
   Projective\,\, transformations\,\, are\,\, transformations\,\, which\,\, transform\,\, lines\,\, to\,\, lines
                $$


   The transformation (3) transforms an arbitrary line to the line\footnote{$^*$}
    {I do not want to make this statement exact, involving
     "infinities"}. This is because line in $\R P^n$ it is a plane in
     $\R^{n+1}$, and the transformation (0.2) transforms planes to planes.

     Hence it is projective transformation in the naive sense.


{\bf Question:}  {\it Is the converse true? Is it true that
any transformations transforming lines to lines are all described by (3)?}


   {\bf Yes, it is }



\m
      This statement and its proof sure was
     known centuries ago , but not so easy to find it in modern
     textbooks.
  I show that this is true, on infinitesimal
  level. All calculations are elementary.

\m

\centerline  {\bf $\cal x$1. Infinitesimal transformations of lines to lines}

\m

     Let $F$ be
    a transformation of $\R P^n$ which transforms straight lines to straight lines.
     (In other words a vector field
            ${\bf F}=F^m(u){\p\over \p u^m}$ is considered).


              Take an arbitrary line: $s_r u^r=1$.  Write down the
              condition that this line transforms to a
              line:
                           $$
                         s'_mu^m-1=0 \,\,\hbox {if and only if}\,\, s_m u'^m-1=0
                         \eqno (1.1)
                           $$
         where $s'_m=s'_m(s_1,\dots,s_n)$ are coefficients of
         the transformed line, $u'^m$ are coordinates of
         transformed points:
                          $$
    s'_m(s)=s_m+\vare L_m(s),\,\,\, u'^m=u^m+\vare F^m(u),\quad  (\vare^2=0)
    \eqno (1.2)
                          $$
  (The condition $\vare^2=0$ encodes the fact that we consider infinitesimal small magnitudes of the first order only)
 Taking care about proportionality coefficient in (1.2) we come
  to the equation:
               $$
   (s_m+\vare L_m(s))u^m-1=(1+\vare\Lambda(u,s))\left(s_m(u^m+\vare F^m(u)-1\right)),\qquad (\vare^2=0)
               $$
  Due to $\vare^2=0$ this equation is equivalent to the equation:
              $$
         L_m(s)u^m-s_mF^m(u)=\Lambda(s,u)(s_mu^m-1)
           \eqno (1.3)
              $$
Solve this equation in polynomials (Why in polynomials?).

\m

\centerline  {\bf $\cal x$2.  Description of vector fields transforming lines to lines}

\m

The equation (1.3) means that $(s_mu^m-1)$ divides
$L_m(s)u^m-s_mF^m(u)$.

 We have to find polynomials $F^m(u)$ which obey this equation,
 tio.e. $F$ such that there exist polynomials (Why polynomials?)
 $L_m,\Lambda$ such that the equation (2.1) is obeyed.

It is linear equation. Solve it in steps


I case. $0$-degree polynomials: $F^m(u)=F^m$:
         $$
         L_m(s)u^m-s_mF^m=\Lambda(s,u)(s_mu^m-1)
         $$

%            $$
%F^m(u)=F^m+F^m_pu^p+F^m_{pq}u^pu^q+\dots
 %           $$

\m

We see that arbitrary $F^m$  obeys the equation. (We can put
$L_m(s)=s_m s_rF^r$, $\Lambda=s_rF^r$)

\m

II case. First degree polynomials:  $F^m(u)=F^m_pu^p$. Then
         $$
        L_m(s)u^m-s_mF^m_pu^p=\Lambda(s,u)(s_mu^m-1)
         $$
We see that arbitrary $F^m_p$  obeys the equation. (We can put
$L_m(s)=F^p_ms_p$, $\Lambda=s_rF^r$)

\m III case. Second degree polynomials:  $F^m(u)=F^m_{pq}u^pu^q$.
Then
         $$
        L_m(s)u^m-s_mF^m_{pq}u^pu^q=\Lambda(s,u)(s_mu^m-1)
         $$
Solve this equation. Let $\Lambda(s,u)=\L(s)+\L_k(s)s^k+\dots$.
Comparing zeroth order terms with respect to $u$ we come to
condition $\Lambda(s)=0$. Comparing terms of the first order we
come to the equation:
               $$
         L_m(s)u^m=-\Lambda_k(s)s_k, \,\, {\rm
         i.e.}\,\,\Lambda_m(s)=-L_m(s)
               $$
Now looking on second order terms we come to
          $$
        -s_mF^m_{pq}u^pu^q=\L_k(s)s^k s_mu^m=-L_k(s)u^k s_mu^m, i.e.
          $$
           $$
         2F^m_{pq}=L_p\delta^m_q+L_q\delta^m_p
           $$
We come to solution:
             $$
   F^m_{pq}={1\over 2}\left(t_p\delta^m_q+t_q\delta^m_p\right), \,\,\hbox {where $t_i$ are constants}
             $$
where we can put $L_r(s)=t_r, \L(s,r)=-t_r �:�$


IV and the  last case. Show that in the higher degrees there are not solutions.
Indeed let $F$ is of the order $k$ over $u$ with $k\geq 3$. Then left hand side posseses
 temrs of first degree and third degree over $u$. Hence $\Lambda$ cannot be a polynomial.
 If degree of $\L$ is less or equal to $n$ then the right hand side possesses the terms pf the order
 $n$ and $n+1$.


Collecting all the cases we come to the answer


\m
  The algebra of transformations (vector fields) which transform line to the line is the following:
                  $$
    F^m(u)=r^m+a^m_pu^p+{1\over 2}\left(t_p\delta^m_q+t_q\delta^m_p\right)u^pu^q\,.
                  $$
One can easy see that this algebra coincides with algebra of infinitesimal transformations from (0.3) \finish

\bye
