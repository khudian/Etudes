
   \magnification=1200
   \baselineskip 17 pt
% 18 November 2012.
%I began this file two years ago, Now I add the pencil...

\def\V {{\cal V}}
\def\s {{\sigma}}
\def\Q {{\bf Q}}
\def\D {{\cal D}}
\def\G {{\Gamma}}
\def\C {{\bf C}}
\def\M {{\cal M}}
\def\Z {{\bf Z}}
\def\U  {{\cal U}}
\def\H {{\cal H}}
\def\R  {{\bf R}}
\def\l {\lambda}
\def\p {\partial}
\def\r {{\bf r}}
\def\v {{\bf v}}
\def\n {{\bf n}}
\def\t  {\tilde}
\def\b {{\bf b}}
\def\ac {{\bf a}}
\def \X   {{\bf X}}
\def \Y   {{\bf Y}}
\def \E   {{\bf E}}
\def \N   {{\bf N}}


    \centerline {\bf On canonical isomorphisms $TT^*M=T^*TM=T^*T^*M$ }

{\sl I wrote this file two-three years ago. Now I just added the 
pencil of isomorphisms. }

\medskip

  Let $M$ be manifold. Establish and study canonical isomorphisms $TT^*M=T^*TM=T^*T^*M$

                        Calculations in coordinates


It may sounds surprising but calculations in coordinates are transparent and illuminating.


 First consider local coordinates on $TM$ and $T^*M$ corresponding to
local coordinates $(x^i)$ on $M$.

  Local coordinates for $TM$ are $(x^i,t^j)$: every vector $\r\in TM$ is a vector
  $t^i{\p\over \p x^i}$, $t^i(\r)=dx^i(\r)$.
         If   $\tilde x^\mu=\tilde x^\mu(x^i)$  are new local coordinates  on $M$
then
                  $$
            d\t x^\mu\left(t^i{\p\over \p x^i}\right)={\p \tilde x^\mu(x^i)\over \p x^i}dx^i
            \left(t^i{\p\over \p x^i}\right)={\p \tilde x^\mu(x^i)\over \p x^i}t^i\,.
                  $$
Hence changing of local coordinates in $TM$ is
                    $$
                    (x^i,t^j)\mapsto (\t x^\mu, \t t^\nu),\quad \t x^\mu=\t x^\mu(x^i),\,\,
                 \tilde t^\mu=\pmatrix{\mu\cr i\cr}t^i,
                 \eqno (1)
                    $$
where we denote ${\p \tilde x^\mu(x^i)\over \p x^i}$ by  $\pmatrix{\mu\cr i\cr}$.



  Respectively local coordinates for $T^*M$ are $(x^i,p_j)$. For every $1$-form $w\in T^*M$
  $p_i=w\left({\p\over \p x^i}\right)$. Under changing of local coordinates on $M$
             $\tilde x^\mu=\tilde x^\mu(x^i)$, coordinates $(p_i)$ change to new coordinates $(p_\mu)$:
                           $$
p_\mu=w\left({\p\over \p \t x^\mu}\right)=
w\left({\p x^i(\tilde x^\mu)\over \p \tilde x^\mu }
{\p\over \p  x^i}\right)={\p x^i(\tilde x^\mu)\over \p \tilde x^\mu }p_i
                           $$
Hence changing of local coordinates in $T^*M$ is
                        $$
                        (x^i,p_k)\mapsto (\tilde x^\mu,\tilde p_\nu),
                        \quad \t x^\mu=\t x^\mu(x^i),\,\,
                           p_\mu=\pmatrix{i\cr \mu\cr}p_i,
                           \eqno (2)
                        $$
where we denote ${\p x^i(\tilde x^\mu)\over \p \tilde x^\mu }$ by  $\pmatrix{i\cr \mu\cr}$


  Now using (1),(2) we define coordinates on the spaces
$TT^*M$, $T^*TM$ and $T^*T^*M$.


The space  $TT^*M$ is tangent space to the space $T^*M$. The local coordinates on $TT^*M$
corresponding to local coordinates $(x^i,p_j)$
on $T^*M$ are coordinates $(x^i,p_j;\xi^k,\rho_m)$; $\xi^k=dx^i(\r),\rho_m=dp_m(\r)$.
Under changing of local coordinates  $(x^i)$ to coordinates $\t x^\mu=\tilde x^\mu(x^i)$
coordinates $(\xi^i)$ and $(\rho_m)$ transform to new coordinates $(\t \xi^\mu)$ and $(\t \rho_\nu)$ respectively.
It follows from (1) that
                       $$
                    \t\xi^\mu={\p \t x^\mu\over \p x^i}\xi^i+{\p \t x^\mu\over \p p_i}\rho_i=\pmatrix{\mu\cr i\cr}\xi^i
                    \eqno (3)
                       $$
  because  ${\p \t x^\mu\over \p p_i}=0$.
  For transformation of coordinates $(\rho_m)$ calculations are longer:
            $$
  \t\rho_\mu={\p \t p_\mu\over \p x^i}\xi^i+{\p \t p_\mu\over \p p_i}\rho_i
            $$
We see that   ${\p \t p_\mu\over \p p_i}={\p\over \p p_i}
\left(\t p_\mu=\pmatrix {k\cr \mu\cr}p_k\right)=\pmatrix {i\cr \mu\cr}$ and
                           $$
{\p \t p_\mu\over \p x^i}={\p\over \p x^i}\left(\t p_\mu=\pmatrix {k\cr \mu\cr}p_k\right)=
 \pmatrix {\nu\cr i\cr} \pmatrix {k\cr\nu \mu\cr}p_k,
                           $$
where we denote as always by  $\pmatrix {\nu\cr i\cr}$ the partial derivative ${\p \t x^\mu\over \p x^i}$
and by $\pmatrix {k\cr\nu \mu\cr}$ the partial derivative ${\p^2 x^k\over \p \t x^\nu \p \t x^\mu}$.
The summation over repeated indices is assumed.  Finally we come to
                  $$
  \t\rho_\mu={\p \t p_\mu\over \p x^i}\xi^i+{\p \t p_\mu\over \p p_i}\rho_i=
  \xi^i\pmatrix {\nu\cr i\cr} \pmatrix {k\cr\nu \mu\cr}p_k+\pmatrix {i\cr \mu\cr}\rho_i
  \eqno (4)
                  $$


     Summarising:

    {\bf Proposition 1} {\it To local coordinates $(x^i)$ on $M$
    one can naturally assign local coordinates on $TT^*M$  $(x^i,p_j;\xi^k,\rho_m)$ such that
    under changing of coordinates $(x^i)\mapsto (\t x^\mu)$ on $M$ these coordinates transform in the following way}
                                $$
            \t p_\mu=\pmatrix{j\cr\mu\cr}p_j, \quad
            \t\xi^\mu=\pmatrix{\mu\cr i\cr}\xi^i,\quad
            \t\rho_\mu=
            \xi^i\pmatrix {\nu\cr i\cr} \pmatrix {k\cr\nu \mu\cr}p_k+\pmatrix {i\cr \mu\cr}\rho_i
            \eqno (*)
                                $$
\bigskip
Now consider coordinates  on $T^*TM$ and their transformation rules.
 If $(x,t)$ coordinates on $TM$ (see (1)) and $(x,t,\pi,\tau)$ corresponding coordinates on $T^*TM$
 ($\pi_k=w\left({\p\over \p x^k}\right)$,
 $\tau_m=w\left({\p\over \p t^m}\right)$) then according to (2)
 under changing of coordinates $(x^i)\mapsto (\t x^\mu)$, the coordinates $(\pi_m)$ transform to coordinates
 $(\t \pi_\mu)$, the coordinates $(\tau_k)$ transform to coordinates
 $(\t \tau_\nu)$ such that
                  $$
\t\pi_\mu={\p x^i\over \p \t x^\mu}\pi_i+{\p t^k\over \p \t x^\mu}\tau_k,\,\,\,\,
\t\tau_\nu={\p x^i\over \p \t t^\nu}\pi_i+{\p t^k\over \p \t t^\nu}\tau_k
                  $$
Since ${\p x^i\over \p \t t^\nu}=0$
and  ${\p t^k\over \p \t t^\nu}={\p x^k\over \p \t x^\mu}$
then $$\t\tau_\nu=\pmatrix {k\cr\nu}\tau_k$$.  For $\t\pi_\mu$ we have
                  $$
\t\pi_\mu=\pmatrix {i\cr\mu}\pi_i+{\p\over \p \t x^\mu}\left({\p x^k\over \p\t x^\nu}t^\nu\right)\tau_k=
\pmatrix {i\cr\mu}\pi_i+\pmatrix {k\cr\mu\nu}\pmatrix {\nu\cr i}t^i\tau_k\,.
                  $$

     Summarising:

    {\bf Proposition 2} {\it To local coordinates $(x^i)$ on $M$
    one can naturally assign local coordinates on $T^*TM$  $(x^i,t^j;\pi_k,\tau_j)$ such that
    under changing of coordinates $(x^i)\mapsto (\t x^\mu)$ on $M$ these coordinates transform in the following way}
                                $$
            \t \tau_\mu=\pmatrix{j\cr\mu\cr}\tau_j, \quad
            \t t^\mu=\pmatrix{\mu\cr i\cr}t^i,\quad
            \t\pi_\mu=
            t^i\pmatrix {\nu\cr i\cr} \pmatrix {k\cr\nu \mu\cr}\tau_k+\pmatrix {i\cr \mu\cr}\pi_i
            \eqno (**)
                                $$


Comparing  Propositions 1 and 2 we see that
  the map
                         $$
           t^i=\xi^i,\,\,\, \tau_j=p_j,\,\,\, \pi_k=\rho_k
                      $$
             establishes isomorphism between the spaces $T^*TM$ and $TT^*M$ which does not depend on
             the choice of local coordinates.
In fact one can consider the {\it pencil}  of maps 
          $$
 t^i={\ac}\xi^i,\,\,\, \tau_j={\bf b}p_j,\,\,\, \pi_k={\bf ab}\rho_k
             $$
where ${\bf a,b}\not=0$.
\bye
