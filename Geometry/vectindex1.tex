
   \magnification=1200
   \baselineskip 17 pt

\def\V {{\cal V}}
\def\s {{\sigma}}
\def\Q {{\bf Q}}
\def\D {{\cal D}}
\def\G {{\Gamma}}
\def\C {{\bf C}}
\def\M {{\cal M}}
\def\Z {{\bf Z}}
\def\U  {{\cal U}}
\def\H {{\cal H}}
\def\R  {{\bf R}}
\def\l {\lambda}
\def\p {\partial}
\def\r {{\bf r}}
\def\v {{\bf v}}
\def\n {{\bf n}}
\def\t {{\bf t}}
\def\b {{\bf b}}
\def\ac {{\bf a}}
\def \X   {{\bf X}}
\def \Y   {{\bf Y}}
\def \E   {{\bf E}}
\def \N   {{\bf N}}
\def\b {{\beta}}
\def\CP {{\bf CP}}
\def\vare{\varepsilon}
    {\bf On the index of vector field.}

First of all  two-three exercises on calculation of index:

Ex. 1 Show the vector field of index $n$ at zero. ($n\geq 1$)

  Answer:  $z^n\p_z$. If $n=1,2$ then this field has zero at infinity and
   an index $2-n=1,0$ at infinity:


Ex.2   What happens if $n=0$. The field $\p_z$ has not zero at zero and has index $2$
at infinity. How to construct the field of index zero?

Non-analytically: $(e^{-{1\over r^2}})\p_z$??

But there is another answer:

Ex.3 Calculate an index of the field
$X^m=A^m_{ik}x^ix^k+\dots$.


Let a field $X$ is equal to zero at the point zero. Then one can consider
the linear operator:
                  $$
          \hat X u=[u,X]
                   $$
If this operator is non-degenerate then t
 the sign of determinant of this operator is an index.
If it is degenerate then to calculate an index is more difficult.
Consider the case where the linear operator $\hat X\equiv 0$. Then one can consider bilinear
form with values in tangent vector
                   $$
       \hat X(u,v)=\left[\left[X,u\right],v\right]
                   $$
In components it means that we consider $A^m_{ik}$ if $X=X^m=A^m_{ik}x^ix^k+\dots$
It is degenerate point. To calculate the index we remove the degeneracy.
Then this instead this point will appear $2,4,\dots,...$ points....

Instead field $X=X^m=A^m_{ik}x^ix^k+\dots$  consider the field
$X=X^m-\vare^m$ where $\vare^m=A^m_{ik}x_0^ix_0^k$ Then instead degenerate
point $0$ will appear the points which are the solution of equation
                   $$
                  A^m_{ik}x^ix^k=\vare^m=A^m_{ik}x^i_0x^k_0
                     $$
   E.g. in two-dimensional case this equation has 2 or 4 solutions.


   The index at every solution will be the sign of correspoindg determinant. We come to the answer
                        $$
        {\rm index} X=\sum {\rm sgn}\det\left(A^m_{ik}x^k\right)
                  $$
where summation goes over all points $x^k$ which obey the equation
$A^m_{ik}x^ix^k=\vare^m=A^m_{ik}x^i_0x^k_0$.
(If all operators are non-degenerate)

It is evident from the last formula that in odd dimensional case we come to the index
zero, because if $x^k$ is solution then $-x^k$ is solution too.
In even dimensional case the answer depends in particularly on the number of
real solutions.


E.g. consider two cases

1.   $X=(x^2-y^2)\p_x+2xy\p_y$ (it is just $z^2\p_z$) Then equations
     $x^2-y^2=x_0^2-y_0^2,2xy=2x_0y_0$ has two real solutions $(x,y)$
     ,$(-x,-y)$. Both give contribution $+1$ to index.

 2. $X=x^2\p_x+y_2\p_y$. In this case the equations $x^2=x^2_0,y^2=y^2_0$
 has four real solutions  $(x=x_0,y=y_0),(x=-x_0,y=-y_0)(x=x_0,y=-y_0)(x=-x_0,y=y_0)$
The solutions $(x=x_0,y=y_0),(x=-x_0,y=-y_0)$ give contribution $+1$,
the  solutions $(x=-x_0,y=y_0),(x=x_0,y=-y_0)$ give contribution $-1$


                $$ $$




    Thinking about Riemann-Horwitz formula I decided
 to do the following simple
    exercise:


   Consider the vector field
                $$
                A=(y-\b y^3)\p_x-x\p_y
                $$
and its image under the map   $z\to z^2\colon\,\,\, \CP\to \CP$.
The push-forward of the field $A$ is well defined under this map because
$A$ is invariant under transformation $z\to -z$. If $\b=0$ then the special points of the field
are just ramification points and sum of the indices preserves.
If $\beta\not=0$ then there are degenerate points $\pm 1/\b$. What is their contribution?
 I think thee calculations are useful, because they contain the part when one
 have to calculate an index in so called degenerate case (If vector field is equal zero and
 linear part is nothing but the sign of determinant of this linear operator)

  Second: Calculation of index will reveal the relation with Riemann-Hourwitz formula...


   So begin. The map  $z\to z^2$ is $u=x^2-y^2,v=2xy$. Its jacobian:
        $$
        \pmatrix
        {2x  &-2y\cr
         2y   &2x\cr
        }
    $$
        We see that
              $$
               \pmatrix
               {2x  &-2y\cr
         2y   &2x\cr
               }
         \pmatrix
        {y-\b y^3\cr
         -x\cr
        }
        =
         \pmatrix
        {4xy-2\b xy^3\cr
         2y^2-2x^2-2\b y^4 &2x
        }
              $$

The vector
                 $$
(y-\b y^3)\p_x-x\p_y\,\,\hbox{ at the point $(x,y)$}
                 $$
goes to the vector
               $$
               (4xy-2\b xy^3)\p_u-\left(2y^2-2x^2-2\b y^4\right)\p_y\,\,
               \hbox{ at the point $(u=x^2-y^2,v=2xy)$}
               $$
Pay attention on the fact that it is well defined: the transformation
$(x,y)\to (-x,-y)$ which does not change the point    $(u=x^2-y^2,v=2xy$
it does not change the vector too.


\bigskip

  {The index of initial vector field}

  The initial vector field is equal to zero in three points
  $(0,0)$, $(0,{1\over \sqrt \b})$ and $(0,-{1\over \sqrt \b})$
The indices of the initial vector in these points is equal
to $=1,-1,-1$ (It is just calculation of sign of determinant)
Now calculate the index at infinity. Perform coordinate transformation
corresponding to $z\to 1/z$:
                        $$
     x'={x\over x^2+y^2},y'=-{y\over x^2+y^2}\,,
     $$
     $$
       \pmatrix
        {x'_x  &x'_y\cr
         y'_x   &y'_y\cr
        }=
\pmatrix
        { {y^2-x^2\over (x^2+y^2)^2} &{-2xy\over (x^2+y^2)^2}\cr
          {2xy\over (x^2+y^2)^2}  &{y^2-x^2\over (x^2+y^2)^2}\cr
        },\quad
        \det
        \pmatrix
        {x'_x  &x'_y\cr
         y'_x   &y'_y\cr
        }={1\over (x^2+y^2)^2}
     $$
  We see that
          $$
                  \pmatrix
        {x'_x  &x'_y\cr
         y'_x   &y'_y\cr
        }
        \pmatrix
        {y-\b y^3\cr
         -x\cr
        }=
        {1\over \left({x^2+y^2}\right)^2}
\pmatrix
        { y^2-x^2 &-2xy\cr
          2xy  &y^2-x^2\cr
        }
        \pmatrix
        {y-\b y^3\cr
         -x\cr
        }=
          $$
          $$
        {1\over \left({x^2+y^2}\right)^2}
          \pmatrix
        { y(x^2+y^2)-\b y^3(y^2-x^2)\cr
        x(x^2+y^2)-2\b y^3\cdot xy  \cr
        }
             $$
          Thus
the vector
                 $$
(y-\b y^3)\p_x-x\p_y\,\,\hbox{ at the point $(x,y)$}
                 $$
goes to the vector
               $$
               \left[{y\over x^2+y^2}-\b y^3 {y^2-x^2\over (x^2+y^2)^2}\right]\p_{x'}+
               \left[{x\over x^2+y^2}-\b y^3 {2xy\over (x^2+y^2)^2}\right]\p_{y'}\,\,
               $$

           $$\hbox{ at the point $\,\,x'={x\over x^2+y^2},y'={y\over x^2+y^2})$}$$

It means that in coordinates $x',y'$  the vector field has appearance
                     $$
                     d
                     $$
        Here I stop. In the case $\b=0$ we see that sum of the indices is equal
        t0 $1+1=2$. I would like to show that index at infinity is equal to $3$.
        (because sum at the other points was equal to $1+(-1)+(-1)=-1$)
        but our field is not equal to zero at infinity.
\bye
