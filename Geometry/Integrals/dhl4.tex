% this text I began to write on 1-st November
%on the base of the text dhl2.tex

      \magnification=1200 %\baselineskip=14pt
\def\vare {\varepsilon}
\def\A {{\bf A}}
\def\FF {{\bf F}}
\def\a {\alpha}
\def\K {{\bf K}}
\def\s {{\sigma}}
\def\p{\partial}
\def\vare{{\varepsilon}}
\def\L {{\cal L}}
\def\G {{\Gamma}}
\def\C {{\bf C}}
\def\Z {{\bf Z}}
\def\U  {{\cal U}}
\def\R  {{\bf R}}
\def\E  {{\bf E}}
\def\l {\lambda}
\def\degree {{\bf {\rm degree}\,\,}}
\def \finish {${\,\,\vrule height1mm depth2mm width 8pt}$}
\def \m {\medskip}
\def\r {{\bf r}}
\def\v {{\bf v}}
\def\n {{\bf n}}
\def\b {{\bf b}}
\def\ss  {{\bf s }}
\def\e{{\bf e}}
\def\ac {{\bf a}}
\def \X   {{\bf X}}
\def \Y   {{\bf Y}}
\def \x   {{\bf x}}
\def \y   {{\bf y}}
\def\w {{\omega}} 
\def\wv {{\buildrel \rightarrow\over \omega}}

\def\K{{\bf K}}
\def\locus {\hbox{locus of $\K$}}


$\qquad$ {\sl 28 October 2013}

\bigskip

\centerline {\bf On Duistermaat-Heckman localisation Theorem II}

\m

{\it
   Here we will give a formulation (with supermathematics flavor),
the proof and concrete calculations
for DH (Dustermaat-Heckman) localisation formula. This etude is essentially
based on the papers of Armen Nersessian [1], and
of  Oleg Zaboronsky and Albert Schwarz [2], my etude
[4] (see the previous etude on this topic) which was based on calculations 
of A.Belavin.) It is interesting also to note the paper [3].
This etude is a  developed exposition of my talk
on the Geometry seminars in Manchester
(17 October and 23 October, 2013).}


\m
      \medskip
{\sl
  $\qquad\qquad$$\qquad\qquad$$\qquad\qquad$ If a form, 
  is invariant with respect to odd vector field


$\qquad\qquad$$\qquad\qquad$$\qquad\qquad$  
  $Q=d\circ \iota_\K+\iota_\K\circ d=\sqrt {\L_\K}$ 
   where $\L_\K$ is Lie derivative 

$\qquad\qquad$$\qquad\qquad$$\qquad\qquad$ with respect
 to $U(1)$-vector field $\K$, then integral of this 

$\qquad\qquad$$\qquad\qquad$$\qquad\qquad$ form over manifold
  $M$ is localised at the zero locus 

$\qquad\qquad$$\qquad\qquad$$\qquad\qquad$ of vector fied $K$. This is the
 meaning.

$\qquad\qquad$$\qquad\qquad$$\qquad\qquad$ 
of Dustermaat-Heckman localisation formula.

}
 $$ $$


\centerline {${\cal x} 0\quad$\bf Recallings}


  Recall briefly the DH (Duistermaat-Heckman) localisation formula
and perform some calculations based on calculations in [4].

Let $(M,\Omega)$ be compact symplectic supermanifold
($\Omega$ is non-degenerate closed two form, ${\rm dim\,}M=2n$).
Let $H$ be a Hamiltonian and
$\K=D_H\colon\, dH=-\iota_\K\Omega$, its Hamiltonian vector field.
Let vector field $K$ obeys the following conditions:
         $$
\hbox{
$\K=D_H$ is compact vector field, i.e. it
defines $U(1)$-action on $M$\footnote{$^{1)}$}
{Vector field is compact if its Lie group is compact subgorup in group of
diffeomorphisms (see [2]).}
}
\eqno (0.1)
         $$
      $$
\hbox {Zero locus of vector field $\K$, $\K(x_i)=0$,
 is a set $\{x_i\}$ of 
      isolated points}
   \eqno (0.2)
     $$

DH-localisation formula states that
if conditions (0.1) and (0.2) are obeyed then
            $$
    \int e^{i H}dV_\Omega=
      \int e^{i(H+\Omega)}=
            \sum_{x_i} 
                {
 e^{i H}\sqrt {\det \Omega_{ik}}
              \over 
   \sqrt {\det {\rm Hess\,}H}
             }\big\vert_{x_i}
              =
     \sum_{x_i} 
              {
             e^{i H(x_i)}
             \over 
             \sqrt 
              {
             \det 
              \left(
               {\p K(x)\over \p x}
              \big\vert_{x=x_i}
               \right)
               }
              }\,.
      \eqno (0.3)
            $$
{\sl Comments to this formula:}

1. Here and later we often omit all the coefficients proportional to 
$\pi^a, n!, i^n$, 

2.  $x_i\colon \, \K(x_i)=0$, is a locus (zero locus) of Hamiltonian
vector field $\K$, i.e. stationary points of Hamiltonian $H$,
  
3. $dV_\Omega$ is invariant volume form:
      $$
dV_\Omega=\Omega^n=
\underbrace{\Omega\wedge \dots\wedge \Omega}_{\hbox{$n$-times}}
\,\hbox {is Lioville volume form}\,,
       $$
 in local coordinates
$dV_\Omega={\rm Pfaf\,}\Omega d^{2n}x=
 \sqrt {\det\Omega}d^{2n}x$,
${\rm Hess\, }H={\p^2 H\over \p x^i\p x^k}$ is bilinear form
at stationary points; as well as ${\p K\over \p x}$ is linear
operator at zero locus of vector field $\K$.

\m


   Shortly show how to calculate (0.3) using ideas of [4].


Let $\w$ be  an arbitrary $\K$-invariant $1$-form:
          $$
   \L_\K\w=d\circ\iota_\K\w+\iota_\K\circ d\w=0\,.
            \eqno (0.4)
          $$
Consider `partition function
               $$
 Z(t)=\int_M e^{i\left(\left(H+\Omega\right)+td_\K\w\right)}\,,
       \eqno (0.5)
               $$
where $d_\K=d+\iota_\K$.
One can see that condition (0.4) and condition
$d_\K(H+\Omega)=0$ imply that
this partition function does not depend on $t$:
          $$
{dZ(t)\over dt}=
 i\int_M d_K
 \left(\w e^{i\left(\left(H+\Omega\right)+td_\K\w\right)}\right)
        =0\,,
     \eqno (0.6)
          $$
  because for an arbitrary differential form $F$,
$\int_Md F=0$ (Stokes Theorem) and $\int_M\iota_\K F=0$ also,
since form $\iota_\K F$ has order less than equal $2n$
($2n$ is the dimension of $M$ is an order of  top form.)

Partition function $Z(t)$ at $t=0$ is the left hand side of equation (0.3),
the initial integral;  this function at $t\to \infty$ 
can be calculated using
stationary phase method. So using (0.6) we reduce calculations of the
integral to quasiclassical calculations for $t\to \infty$:
            $$
Z(0)=\lim_{t\to\infty}Z(t)=\sum_{k,r}
         {t^r\over k!r!}
     \int_M e^{i\left(H+th\right)}\tilde\Omega^r\Omega^m\,,
        \eqno (0.7)
            $$
where $\tilde\Omega=d\w$, $h=\iota_\K \w$.
 Now calculate partition function at $t\to \infty$.
$dh=d(\iota_\K\w)=-\iota_\K{\tilde\Omega}$. Hence at zero locus of 
$\K$, i.e. $dh=0$ we have 
$$
{\rm Hess\,}H\big\vert_{x_i}={\p^2 H\over \p x^m\p x^n}\big\vert_{x_i}=
   \tilde\Omega_{mn}\big\vert_{x_i}\,.
       \eqno (0.8)
 $$
Hence using the fact that for symmetric bilinear form $A(\x,\x)$
in $k$-dimensional Euclidean space  $\R^k$ 
                $$
           \int_{\R^k}e^{it A(\x,\x)}d^k x= 
           \int_{\R^k} 
e^{itA_{ij}x^i x^j}d^k x=
 {e^{i\pi k\over 4}\sqrt{\pi^k}\over t^{k\over 2}\sqrt {\det A}}\,,
             $$
we obtain that at the 
quasiclassical limit for partition function $Z(t)$ in (0.7)
is equal to 
       $$
\lim_{t\to\infty}Z(t)=\sum_{r=0}^n
         {t^r\over (n-r)!r!}
     \int_M e^{i\left(H+th\right)}\tilde\Omega^r\Omega^{n-r}=
          $$
          $$
       \lim_{t\to\infty} \sum_{r=0}^n\sum_{x_i} 
                {t^r\over (n-r)!r!}
                 {
 e^{i H}\sqrt {\det \tilde\Omega_{ik}}
              \over 
  t^{n} \sqrt {\det {\rm Hess\,}H}             
             }\big\vert_{x_i}
  =\sum_{x_i} 
                 {
 e^{i H}\sqrt {\det \tilde\Omega_{ik}}
              \over 
   \sqrt {\det {\rm Hess\,}H}             
             }\big\vert_{x_i}
       $$
Now choose $\w$ such that $\tilde\Omega=d\w$ is non-degenerate
at locus of $K$.
We have $dh=\iota_\K\tilde\Omega$. Hence  at locus of $\K$  
                  $$
{\rm Hess\,}H={\p^2 H(x)\over \p x^m x^n}=
   \tilde\Omega_{mr}{\p K^r\over \p x^n}\,,
            $$
and we have finally that
          $$
\lim_{t\to \infty}Z(t)=\sum_{x_i} 
                 {
 e^{i H}\sqrt {\det \tilde\Omega_{ik}}
              \over 
   \sqrt {\det {\rm Hess\,}H}             
             }\big\vert_{x_i}
\sum_{x_i} 
 {e^{i H}
              \over 
   \sqrt {\det {\p K\over \p x}}             
             }\big\vert_{x_i}
          $$
Thus due to relation (0.6) leads to (0.3).

{\bf Remark 1} The form $\tilde\Omega=d\w$ and new Hamiltonian $h=\iota_\K\w$
define the same Hamiltonian vector field $\K$ as a pair $(\Omega,H)$.
  On the other hand the pair $(\tilde\Omega,\w)$ is more suitable for
calculation of quasiclassical approximation. The $U(1)$-vector field
$\K$ is fundamental object of DH-localisation formula, not the
pair which produces this field (see in detail ${\cal x} 2$).

\m

{\bf Remark 2}


One of the way to produce $\K$-invariant form $\w$ is the following:
One can 
take  $\w$-covector $\K$ with respect to 
$U(1)$-invariant metric: $\w=\w_idx^i$,
 $w_i=g_{ik}K^k$ and $g_{ik}$ is $U(1)$-invariant Riemannian metric
(average over group $U(1)$).
It is crucial for calculation that $\tilde\Omega=d\w$ 
is non-degenerate
at zero locus of $\K$. Is it an additional condition, or it 
follows from the fact that vector field $\K$ generates $U(1)$-action
(and $M$ is even-dimensional manifold)?
 On one hand I cannot prove this completely, on the 
other hand natural counterexamples deal with non-compact vector field.


\bigskip



\centerline {${\cal x} 1\quad$
   \bf DH-formula and supersymmetric mechanics. Nersessian's approach.}

{\it The considerations of this paragraph are based on the work [2]}

\m

 The calculations above can be put in supersymmetric framework.
Differential form on $M$ can be considered as a function
on $\Pi TM$---tangent bundle to $M$ with reversed parity of fibers
         $w_i(x)dx^i\to w_i(x)\xi^i,\dots$.
Integral of form over $M$ is the integral of a function over supermanifold
$\Pi TM$ with invariant volume form $dx^1\dots dx^{2n}d\xi^1\dots
d\xi^{2n}$.


 In the very nice paper [1] Armen Nersessian suggested the supersymmetric
framework of the calculations above. I will try to explain it here.
Recall that for an arbitrary Poisson manifold $M$ 
(manifold with Poisson bracket $\{\,\,,\,\,\}$) one can consider
odd Koszul bracket $[\,\,,\,\,]$ on $\Pi TM$ such that
for arbitrary functions $f,g$ on $M$ we have that
         $$
   [f,g]=0\,,[f,dg]=\{f,g\}\,\,[df,dg]=d\{f,g\}\,.
  \eqno (1.1)
         $$
In local coordinates $
  [x^i,x^k]=0$, $[x^i,\xi^k]=\Omega^{ik}$,
$[\xi^i,\xi^k]=\xi^r\p_r\Omega^{ik}$.

If Poisson structure is symplectic one then 
          $$
  [\Omega, F]=dF\,,\qquad (\Omega=\Omega_{ik}\xi^i\xi^k)
        \eqno (1.2)
        $$
If $H$ is an arbitrary Hamiltonian on $M$ and $\K=D_H$ Hamiltonian
vector field then
          $$
     [H,F]=\iota_{\K}F
\eqno (1.3)
          $$
We see that
              $$
   (d+\iota_k)F=[\Omega+H,F]
              $$
and
           $$
\L_\K F=(d+\iota_\K)^2=
 [H+\Omega, [H+\Omega,F]]=[[H,\Omega],F]\,.
           $$
Thus we come to core of Dustermaat-Heckman formalism:

Form $F$ is invariant with 
respect to odd vector field $d_\K=d+\iota_\K$
if it is integral of motion of 'Hamiltonian' $H+\Omega$,
form $F$ is invariant with respect to Hamiltonian vector
field $\K=D_H$ if it is integral of motion of 'Hamiltonian'
$G=[H,F]$.

The partition function (0.5) can be rewritten as
      $$
Z(t)=\int e^{i(H-\Omega-t[H+\Omega,\tilde G])}\,.
      $$
{\bf Remark 3} Hamiltonians $\{H+\Omega,H-\Omega,\Omega\}$ form superalgebra.
 


  \bigskip

\centerline {${\cal x} 2$\bf Schwarz-Zaboronsky supersymmetric formalism}

\m

In this paragraph we will speak about approach developed in the paper [2],
where supergeometry is powerfully used for 
formulating localisation formula in a more general case. 


\m
   It will always be assumed that
$M$ is compact manifold  and $\K$ is compact vector field on it, 
i.e. vector field which generates $U(1)$
action. We denote by $$
   Q_\K=d+\iota_\K\,,\quad \hbox
 {in ``supernotations''}\,\,
Q_\K=\xi^i{\p\over \p x^i}+K^i(x){\p \over \p \xi^i}\,,
              $$
where $x^i,\xi^i=dx^i$ are local coordinates on $\Pi TM$.

Odd vector field $Q_\K$ is a ``square root'' of a Lie derivative 
$\L_K=\iota_\K\circ d+d\circ \iota_\K$:
               $$
\L_\K=Q_\K^2=\left(\xi^i{\p\over \p x^i}+K^i(x){\p \over \p \xi^i}\right)^2=
       K^i(x){\p \over \p x^i}+\xi^r{\p K^i\over \p \xi^r}{\p\over \p \xi^i}\,,
              \eqno (1) 
              $$
or in classical notations
                   $$
\L_\K=Q_\K^2=(d+\iota_k)^2=d\circ \iota_\K+\iota_\K\circ d\,.
                   $$
We formulate the following version of DH localisation theorem:


{\it 

{\bf Theorem}   Let $H= H(x,dx)$ be a $Q_\K$-invariant form on $M$,
 i.e.
          $$
dH+\iota_\K H=0\,.
          \eqno (2)
           $$
Then the integral $\int_M H(x,dx)$ is localised at locus of $K$.
This means follows: let $U_K$ be an arbitrary $U(1)$-invariant\footnote{$^*$}
{the condition to be $U(1)$-invariant may be is not necessary. 
We will use it for constructing $U(1)$-invariant 
partition of unity. This condition is absent in the paper [1].}
 tubular neighborhood of 
locus of $K$ and let $G_U=G_U(x,dx)$ be a $Q_\K$-invariant 
form such that it is equal to $1$
at the locus of vector field $\K$ and it  vanishes out of neighborhood
$U_\K$: 
            $$
Q_\K G_U=0,\,({\rm i.e.\,} dG_U+\iota_\K G_U=0), 
  \quad G_U\big\vert_{\locus}=1,\quad G_U\big\vert_{M\backslash U_K}=0\,. 
\eqno(3)
            $$ 
(Bump-form of zero locus of $\K$.)
(We will prove the existence of such a bump-form)

\medskip

Then 

                $$
   \int_MH=\int_M HG_U\,. 
            \eqno (4)
                $$

}
\medskip

{\bf Example} Let $M$ be a symplectic manifold, i.e.
 non-degenerate closed two-form  $\Omega$ is defined on $M$
($M$ is even-dimensional). Let $h=h(x)$ be a Hamiltonian such that
its Hamiltonian vector field $D_h$ ($D_h\colon\quad 
\iota_{_{D_h}}\Omega=-dh$) is compact, i.e. it 
defines $U(1)$ action. 
  Consider the form
        $$
   H(x,dx)=\exp i\left(\Omega+h\right)\,.
      \eqno (5)
        $$
This form is $Q_\K$-invariant. Indeed since $K$ is Hamiltonian vector field
$D_h$ hence
       $$
\iota_\K \Omega+dh=0.{\rm i.e.\,} Q_\K(h+\Omega)=0\Rightarrow
   Q_\K H=0\,.
       $$
Then 
     $$
\int H(x,dx)=\int \exp i\left(\Omega+h\right)={i^n\over n!}
\int \exp i h\underbrace{\Omega\wedge\dots\wedge\Omega}_{\hbox{$n$ times}}
     $$
is localised.

\smallskip

{\bf Remark 4}  Note that this example is a basic example in classical 
background.  Compact vector field $\K$ appears naturally in this example
as Hamiltonian vector field of Hamiltonin $h$.  In 
Schwarz-Zaboronsky approach the vector field $\K$ appears
independently without symplectic structure and 
Hamiltonian. In this approach the localisation formula is stated
for a function $H(x,dx)$ on $\Pi TM$ (sum of differential forms of different
orders). The classical condition that sum of differential forms
is invariant with respect to equivariant differential $d_\K=d+\iota_\K$
becomes the condition  that ``function''
 \footnote{$^2$} 
{$H(x,dx)$ is non-homogeneous differential form on $M$. It is a
function on tangent bundle $\Pi TM$ with reversed parity of fibers.}

$H(x,dx)$ is invariant with respect to odd vector field
$Q_\K$ which is the square root of Lie derivative along the vector field
$\K$ $\colon Q_K^2=\L_\K$.




{\bf Remark 5} 'Super-language' becomes essentially important
for constructing of partition of unity for forms.

\smallskip

{\sl Proof of Theorem}
First we prove the existence of a form $G_U=G_U(x,dx)$ which 
obeys the condition (3), then we will show that an arbitrary 
 $Q_\K$-invariant ``function''
(form) which obeys conditions (3) yields the localisation formula (4).

Using partition of unity arguments consider a function
$F=F(x)$ such that 
            $$
F(x)\big\vert_{\locus}=0,
 \quad F(x)\big\vert_{M\backslash U_K}=1\,. 
\eqno(6)
            $$ 
(We may consider partition of unity which is subordinate
 to covering $V_1\cup V_2$,
where $V_1=U_\K$ and $V_2=M\backslash \hbox{locus of $K$}$.


We may assume that $F(x)$ is $\K$-invariant function.
(Here we use the $U(1)$-invariance of neighbourhood of 
locus (see the footnote.)).

   
 It is useful to  consider the differential $1$-form
       $$
    \w_\K\colon \w_\K(\x)\langle \K,,\x\rangle\,,\w_i=g_{im}K^mdx^i\,,
         \eqno (7)
      $$
where $\langle \K,,\x\rangle$ is $U(1)$-invariant Riemannian
metric on $M$. Now we are ready to define form $G_U$ which obeys
the condition (3):
            $$
   G_U(x,dx)=1-Q_\K\left({\w_\K(x,dx)\over Q_\K \w_\K}F(x)\right)
                \eqno(8)
            $$
Straightforward calculations show that this function obeys conditions (3).
Indeed $F(x)=0$ if $x$ belongs to locus of $K$
(and in a vicinity of the locus), hence the right hand side 
of equation (8) is well-defined on the locus of $\K$, where
the form $\w_\K$ is not defined. Using the fact that
$Q_\K\left({\w_\K(x,dx)\over Q_\K \w_\K}\right)=1$ (if $\K(x)\not=0$)
we immediately come to the condition (3).

 Let $\tilde G_U=\tilde G_U(x,dx)$ be an arbitrary $Q_\K$-invariant
 form which obeys the condition
(3). Then consider the difference $L(x,dx)=\tilde G_U-G_U$.
The form  $L(x,dx)$ is $Q_\K$-invariant and it is equal to $0$ at the locus
of $K$, Hence 
              $$
   L(x,dx)=Q_\K\left({\w_\K(x,dx)\over Q_\K \w_\K}L(x,dx)\right)\,.
        \eqno (9)
             $$
Thus we see that $Q_\K$-invariant  form $G_U(x,dx)$ in (8)
which obeys the condition (3) as well as an arbitrary
$Q_\K$-invariatn  form $\tilde G_U(x,dx)$ 
which obeys the condtion (3) obey the condition
that
      $$
    \matrix
           {
   G_U(x,dx)=1+Q_\K(...)\cr
   \tilde G_U(x,dx)=1+Q_\K(...)\cr
      }
       $$
  This immediately implies the relation (4):
            $$
\int_M H(x,dx)G_U(x,dx)=\int_M H(x,dx)(1+Q_\K(\dots))=
\int_M H(x,dx)
       $$
 since $\int_M Q_\K (\dots)=0$\footnote{$^{**}$}{since $Q_K=d+\iota_K$,
 and $\iota_K \w$ 'does not contain' top form.  
This follows also from the vanishing of divergence of
odd vector field $Q_\K$ with respect to canonical volume form
in $\Pi TM$}\finish


\m

{\it Concrete calculations}

   Now based on the Theorem we present concrete calculations.
which are very similar to calculations in paragraph $0$.

Let $H=H(x,dx)$ be $Q_\K$ invariant form and 
locus (zero locus)
of $U(1)$-invariant 
vector field $\K$ is a set $\{x_i\}$ of isolated points.

Using bump-form $G_U$, the form which vanishes out vicinities of
points $\{x_i\}$ (see the considerations above) we calculate
$\int_M H(x,dx)$.


 {\bf Lemma}  For an arbitrary $Q_\K$-invariant form $H(x,dx)$ the integral
            $$
   Z(t)=\int H(x,dx)e^{itQ_\K(\w_\K)}\,, 
            $$
where $\w_\K$ is $U(1)$-invariant form 
(7) does not depend on $t$.

  Proof:
              $$
 {dZ(t)\over dt}=
           i\int_M 
  H(x,dx)Q_\K
   \w_\K
  e^{itQ_\K(\w_\K)}=
    i\int_M Q_\K
          \left(
  H(x,dx)e^{itQ_\K(\w_\K)}
          \right)=0\,.
              $$
Now using lemma and bump-form which localises integrand in vicinity of
points $\{x_i\}$ we come to 
           $$
\int_M H(x,dx)=\int_M H(x,dx)G_U(x,dx)=
\left(
\int_M H(x,dx)G_U(x,dx)e^{itQ_\K(\w_\K)}\right)
\big\vert_{t=0}
             $$
             $$
              =
\left(\int_M H(x,dx)G_U(x,dx)e^{itQ_\K(\w_\K)}\right)
\big\vert_{t\to\infty}
           $$ 
Using method of stationary phase and assuming that $d\w$ is non-degenerate
at locus of $\K$\footnote{$^*$}{See the remark 2}  we 
calculate the last integral
(see [4]) and come to the answer
           $$
\int_M H(x,dx)=
              =
\left(\int_M H(x,dx)G_U(x,dx)e^{itQ_\K(\w_\K)}\right)
\big\vert_{t\to\infty}=
\sum_{x_i}{i^n\over n!}{H(x,dx)\big\vert_{x_i}\over 
  \sqrt  {{\p K\over \p x}\big\vert_{x_i}} }
           $$ 
If $H(x,dx)\big\vert_{x_i}=H_0(x_i)$, where
$H(x,dx)=H_0(x)+H_1(x,dx)+\dots$ is a sum of differential forms. 
 

$$ $$

\centerline {\bf References}

 [1]  A. Nersessian {\it Antibrackets and localisation of (path) integrals
                    }
   arXix: hep-th/9305181, (published in JETP)
\m

 [2] Albert Schwarz and Oleg Zaboronsky. 
 {\it Supersymmetry and localisation}. arXiv: hep-th/951112v1,
  (published in CMP)


 \m


        
 [3] {\it On the Duistermaat-Heckman localisation 
 formula and Integrable systems} 
arXiv: hep-th/9402041v1

\m


 [4] homepage: maths.manchester.ac.uk/khudian/Etudes/Geometry/Dustermaat-Heckman  localisation formula. 
{\it Etude based on the fragment of the lecture of 
A.Belavin in Bialoveza, summer 2012.} 

\bye
