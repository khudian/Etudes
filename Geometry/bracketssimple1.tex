
   \magnification=1200
   \baselineskip 17 pt
% 18 November 2012.
%I began this file two years ago, Now I add the pencil...

\def\V {{\cal V}}
\def\s {{\sigma}}
\def\Q {{\bf Q}}
\def\D {{\cal D}}
\def\G {{\Gamma}}
\def\C {{\bf C}}
\def\M {{\cal M}}
\def\Z {{\bf Z}}
\def\U  {{\cal U}}
\def\H {{\cal H}}
\def\R  {{\bf R}}
\def\l {\lambda}
\def\p {\partial}
\def\r {{\bf r}}
\def\v {{\bf v}}
\def\n {{\bf n}}
\def\t  {\tilde}
\def\b {{\bf b}}
\def\ac {{\bf a}}
\def \X   {{\bf X}}
\def \Y   {{\bf Y}}
\def \E   {{\bf E}}
\def \N   {{\bf N}}
\def\vare{\varepsilon}

    \centerline {\bf Brackets on $T^*M$ and $\Pi T^*M$ for pedestricans. }
  I would like to deduce formulae for Poisson brackets odd and even using
  mainly their invariance.
  
  Let $M$ be a manifold with local coordinates $x^i$.
  It is illuminating from very beginning suppose that coordinates have an arbitrary parity
  $p(x^i)=p(i)=0,1$, i.e. $M$ is a supermanifold. Let $x^*_i$ be coordinates
  in the fibre. Cooridnates $x^*_i$ have the same parity as $x^i$ for cotangent bundle $T^*M$
  and they have opposite parity for $\Pi T^*M$,
                      $$
                      p(x_i^*)=p(x^i)+\vare\,,
                      $$ 
 where $\vare=0$ for $T^*M$ and $\vare=1$ for $\Pi T^*M$.
                                            Under changing of coordinates $x^i=x^i(x^{i'})$
                 $$
      x^*_i={\p x^{i'}(x)\over \p x^i}x^*_{i}=x^{i'}_{i}x^*_{i'}           
                 $$
  
  Write down bracket $[F,G]$ of two functions, bilinear combination of first derivatives.
    This is invariant at least for linear transformations at least in the case when all coordinates are even. Hence:
                         $$
                     \left[F(x,x^*), G(x,x^*)\right]=
                     (-1)^{A(F,G,i)}{\p F(x,x^*)\over \p x^i} 
                     {\p G(x,x^*)\over \p x^*_i}
                             +   
                             (-1)^{B(F,G,i)}{\p F(x,x^*)\over \p x^*_i}
                     {\p G(x,x^*)\over \p x^i}\,,
                     \eqno (1.2)
                         $$
                         where $A=A(F,G,i)$, $B=B(F,G,i)$ are functions on parity of
                         $F,G$ and $i$.
We will find conditions on sign factors $(-1)^A, (-1)^b$
which are put by invariance of this expression under coordinate transformation (1).

Under arbitrary coordinate transformation (1.1) we have that
               $$
    {\p \over \p x^{i'}}={\p x^{i}(x)\over \p x^{i'}}{\p \over \p x^{i}}+
    {\p x_i^*\over \p x^{i'}}{\p\over \p x^*_i}=
            x^{i}_{i'}{\p \over \p x^{i}}+{\p\over \p x^{i'}}
            \left(x^{k'}_i x^*_{k'}\right){\p\over \p x^*_i}=
                    $$
                    $$
            x^{i}_{i'}{\p \over \p x^{i}}+
            x^m_{i'}x^{k'}_{mi} x^d_{k'}x^*_d{\p\over \p x^*_i} 
               $$
and
                $$
{\p \over \p x^*_{i'}}=
{\p x^*_{i}(x,x^*)\over \p x^*_{i'}}{\p \over \p x^*_{i}}=
{\p \over \p x^*_{i'}}\left(x^{m'}_{i}x^*_{m'}\right){\p \over \p x^*_{i}}
=(-1)^{(p(i)+p(i'))p(i')}x^{i'}_i{\p \over \p x^*_{i}}\,.
                $$

  We apply these formulae to equation (2). What is a condition that (2) does not change 
  under an arbitrary transformation.
  
  First note that an arbitrary transformation is a combination of linear transformation and a transformation
  which has identical differential (i a given point). 
  We first study restrictions which follows from the fact that equation (2) does not change under 
  linear transformation then we study special transformation.
  
Calculations imply the following answer:
  
  
  1.  Invariance under linear transformation implies that sign functions $A,B$ obey the following conditions:
                 $$
                 \cases
                 {
                 A(F,G,i)=A_0(F,G)+p(i)\left(p(F)+\varepsilon \right)\cr
                 B(F,G,i)=B_0(F,G)+p(i)\left(p(F)+1 \right)\cr
                 }
                 $$
and invariance under special  transformations implies that
               $$
               B_0(F,G)=
               \cases
               {A_0(F,G)+1\,,\hbox {if $\varepsilon=0$, i.e. in the case of $T^*M$}
                \cr
                A_0(F,G)+F \,,\hbox {if $\varepsilon=1$, i.e. in the case of $\Pi T^*M$}
               }
                 $$
We see that even Poisson bracket has the appearance:
               $$
                     \left[F(x,x^*), G(x,x^*)\right]_0=
                      (-1)^{A_0(F,G)+iF}
                     \left(
                     {\p F(x,x^*)\over \p x^i}
                     {\p G(x,x^*)\over \p x^*_i}
                     -{\p F(x,x^*)\over \p x^*_i}
                     {\p G(x,x^*)\over \p x^i}
                           \right)\,,
                     \eqno (1.4)
               $$               
and odd Poisson bracket (Schouten-Butin-antibracket,,,,) hast eh appearance:
        $$
            \left[F(x,x^*), G(x,x^*)\right]_1=
                      (-1)^{A_0(F,G)+iF+i}
                     \left(
                     {\p F(x,x^*)\over \p x^i}
                     {\p G(x,x^*)\over \p x^*_i}
                     +(-1)^F{\p F(x,x^*)\over \p x^*_i}
                     {\p G(x,x^*)\over \p x^i}
                           \right)\,,
                     \eqno (1.5)
        $$
Expression (1.4) is well-defined on $T^*M$ and expression $\Pi T^*M$ is well-defined
on $\Pi T^*M$.
  
Further restriction on sign function $A_0(F,G)$  follow from derivations....antisymm. 
  \bye