\magnification=1200
\baselineskip=17pt
%\def\tan {\rm tan\,}

  Dear Pierre
you  remember about 15 years ago
you  told me about
your teacher in Mathematics, (his name was something
like  Gudar...) who told  you about relations
between   functions    $\tan \theta$ and $sinh theta$:
             $$
1+\tan^2 \theta=\cos \theta 
     \qquad\qquad\qquad             
1+sinh^2\theta=cosh^2\theta
\eqno (1) 
             $$
Why I recall this?

A week ago I considered
not very common a realisation of
  hyperbolic plane:
  $R^2$ with induced Riemannian metric
           $$
G=(dx^2+dy^2-dz^2)\big\vert_{z=sqrt {1+x^2+y^2}}=
  {(1+y^2)dx^2-2xydxdy+(1+x^2)dy^2\over 1+x^2+y^2}
           $$
(this is upper sheet of hyperboloid
  $z^2-x^2-y^2=1$ with metric 
induced by metric $dx^2+dy^2-dz^2$ in ${\bf E}^{2.1}$)

One can see that
           $$
{\rm lines}\,\, y=kx\,,\quad
{\rm hyperbolas}\,\,  {y^2\over p^2}-x^2=1
           $$
are geodesics of this metric.

One can see that these geodesics are 
rotations of geodesic $\pmatrix {x=t\cr y=0\cr 
z=\sqrt {1+t^2}}$
on hyperboloid

Consider two geodesics
             $$
   y=x\tan\theta\,,\quad {\rm and}\quad
    {y^2\over {\rm sh}^2 \theta}-x^2=1
        \eqno (1a)
             $$
These geodesics coincide on absolute---they
are asimptotically the same.

The geodesic
    $y=x\tan\theta$ ($k=\tan\theta$)
is rotation of geodesic 
   $\pmatrix {x=t\cr y=0\cr 
z=\sqrt {1+t^2}}$ on the angle $\theta$:
            $$
    \pmatrix
      {
    \cos\theta &\sin\theta &0\cr
    -\sin\theta &\cos\theta &0\cr
          0     &0          & 0\cr
            }
  \pmatrix {t \cr 0 \cr\sqrt {1+t^2}\cr}
              $$
  
In the same way
the geodesic
    ${y^2\over {\rm sh}^2 \theta}-x^2=1$
is rotation of geodesic 
   $\pmatrix {x=t\cr y=0\cr 
z=\sqrt {1+t^2}}$ on the hyperbolic angle $\theta$:
            $$
    \pmatrix
      {
    1 & 0&0\cr
    0&{\rm \cosh}\theta &{\rm \sinh}\theta\cr
    0&{\rm \sinh}\theta &{\rm \cosh}\theta\cr
            }
  \pmatrix {t \cr 0 \cr\sqrt {1+t^2}\cr}
              $$
  
We see the meaning of relation (1).

May be it is commonplace, 
 I just wanted to mention it.



\bye
