
  %\magnification=1200 


\baselineskip=14pt
\def\vare {\varepsilon}
\def\A {{\bf A}}
\def\t {\tilde}
\def\bs {{\bf s}}
\def\a {\alpha}
\def\d {\delta}
\def\K {{\bf K}}
\def\N {{\bf N}}
\def\w {\omega}
\def\s {{\sigma}}
\def\S {{\Sigma}}
\def\s {{\sigma}}
\def\p{\partial}
\def\vare{{\varepsilon}}
\def\Q {{\bf Q}}
\def\D {{\cal D}}
\def\B {{\cal B}}
\def\G {{\Gamma}}
\def\C {{\bf C}}
\def\L {{\cal L}}
\def\F {{\cal F}}
\def\Z {{\bf Z}}
\def\U  {{\cal U}}
\def\H {{\bf H}}
\def\S  {{\bf S}}
\def\E  {{\bf E}}
\def\l {\lambda}
\def\degree {{\bf {\rm degree}\,\,}}
\def \finish {${\,\,\vrule height1mm depth2mm width 8pt}$}
\def \m {\medskip}
\def\p {\partial}
\def\r {{\bf r}}
\def\pt {{\bf pt}}
\def\v {{\bf v}}
\def\n {{\bf n}}
\def\t {{\bf t}}
\def\b {{\bf b}}
\def\c {{\bf c }}
\def\e{{\bf e}}
\def\ac {{\bf a}}
\def \X   {{\bf X}}
\def \Y   {{\bf Y}}
\def \x   {{\bf x}}
\def \y   {{\bf y}}
%\def \G{{\cal G}}
\def\ss  {\sigma_{\rm sph}}
\def \grad {{\rm grad\,}}
\def\e {{\bf e}}
\def\f {{\bf f}}
\def\g {{\bf g}}
% I began this file on  13 September  2017

%\newcommand{\bs}{{\boldsymbol{s}}}

\newcommand{\rh}{{\boldsymbol{\rho}}}
\newcommand{\Ps}{{\boldsymbol{\Psi}}}

\newcommand{\der}[2]{{\frac{\partial {#1}}{\partial {#2}}}}
\newcommand{\lder}[2]{{\partial {#1}/\partial {#2}}}
\newcommand{\dder}[3]{{\frac{\partial^2 {#1}}{\partial {#2}\partial {#3}}}}
%\newcommand{\R}[1]{{\mathbb R}^{#1}}
\newcommand{\RR}{\mathbb R}
%\newcommand{\Z}{{\mathbb Z_{2}}}
\newcommand{\ZZ}{{\mathbb Z}}

\newcommand{\CC}{\mathbb C}
\newcommand{\NN}{{\mathbb N}}



\documentclass[12pt]{article}
\usepackage{amsmath,amsthm}


\usepackage{amsmath,amssymb,amsfonts,amsthm}


\theoremstyle{theorem}
\newtheorem{thm}{Theorem}

\theoremstyle{lemma}
\newtheorem{lm}{Lemma}


%\theoremstyle{remark}
%\newtheorem{rm}{Remark}

%\theoremstyle{Lemma}
%\newtheorem{lm}{Lemma}

\numberwithin{equation}{section}


\title{Geometry of Differential operators}
\date{}



\begin{document}
\maketitle

 \centerline {H.M. Khudaverdian}



 {\it Following the book of Nikulin-Schafarevitch
 ``Geometries and groups''  we will present here
 the classification of 
  locally Euclidean $2$-dimensional Riemannian manifold.   
The answer is:

   0)
 1)  Cylindre,

  2)   Twisted cylindre, 

 2) Lobachevsky plane of tori

 3)  Klein bottle  



  }

\section {Locally Eucldean Geometry}
\label{localeuclidgeometry}

 Let $(M,G)$ be a Riemannian manifold.

  We say that it is locally Euclidean if 
in a vicinity of every point there are 
local coordinates  $u_i$ such that
      \begin{equation*}
     G(u,v)=du_1^2+\dots+du_n^2\,,
        \end{equation*}
where $n$  is dimension of manifold. We mostly consider here
  the case $n=2$.

The condition above means that
for every point $\D\in M$ there exists small
neighboorhoud $O_\vare(\D)$  such that $O_\vare(D)$
is isometric to the interior of the disc $O_\vare(D)$,
where $D$ is an arbitrary point of Euclidean plane $\E^2$. 

  The radius $\vare$ of the neighborhood depends on a point.
Suppose that the following additional condition is fixed:
there exists poisitive constant $r\geq 0$ such that
radius of all neighborhoods is bigger or equal to $r$.
(This condition automatically holds for compact manifolds.)
We see that  for every point  $\D$ there exist local
coordinates
  $(u_{_\D},v_{_\D})$\footnote{now and later we will consider only 
$2$-dimensional case.  Sure these and many considerations can
be easily generalised for an arbitrary $n$.} such that
      \begin{equation}\label{loceuclid}
     G(u,v)=du_{_\D}^2+dv_{_\D}^2\,,\quad
    u_\D(\D)=v_\D(\D)=0\,,\quad
         \end{equation}
and the following additional condition is obeyed:
         \begin{equation}\label{stronglyloceuclid}
\hbox {local coordinates $u_{_\D},v_{_\D}$ run 
in the disc $0\leq u_{_\D}^2+v_{_\D}^2<r $ }\,,   
        \end{equation}
where radius $r$ does not depend on
a choice of a point $D$.

In our consideration we will call the Riemannian manifold
locally Euclidean if not only condition \eqref{loceuclid}
but condition \eqref{stronglyloceuclid} is also obeyed. 

Sometimes if we want to differ between these cases we will call
Riemannian manifold strongly locally Euclidean if both conditions
are obeyed.


In the case if $M$ is compact manifold, then  condition
\eqref{loceuclid} implies condition \eqref{stronglyloceuclid}
(under the suitable choice of $r$). In general this is not true.
E.g. the sheet
   $-1<x<1$ of $\E^2$ is evidently locally Euclidean in the sense
of definition \eqref{loceuclid} but it is not locally Euclidean
in the sense of \eqref{stronglyloceuclid} (the radius $r$
becomes smaller and smaller when $x\to 1$).

We will call coordinates $(u_\D,v_\D)$
Euclidean coordinates on $M$ adjusted to the point $\D$.





\section {Equivalence on $E^2$ and discontinuous action of group}




   Let $r$ be an equivalence realtion of points on $\E^2$,
and  there exists $\delta>0$ such that
for every two $r$-equivalen distinct  points $A,B$ 
 the distance between these points
is greater than $\delta$: $d(A,B)>\delta$ if $ArB$
but $A\not=B$.

   We say that  an isometry $g$ is an $r$- isometry,
if it preserves $r$-equivalence:
              $$
    ARB\Leftrightarrow  A^gRB^g\,.
              $$
\begin{lm}
Let  $\G=\G_r$ be a set of isometries of $\E^2$
such that
       \begin{equation*}
      ArB \Leftrightarrow \hbox{there exist $g\in \G$
such that $B=A^g$}\,,
\end{equation*}

\end{lm}


  WLOG  choose an arbitrary point $A\in \E^2$,
since $ArA$ then there exists $F\in \G$ such that
$F(A)=A$. Hence $F$ is identity.  This follows from the lemma:


  \begin{lm}
If $F$ is an isometry with a fixed point
such that
sends any point to the equivalent point,
then  $F={\rm identity}$.
 \end{lm}

 Let $F$ be an arbitrary isometry such that
       \begin{equation*}
\hbox{for an arbitrary point $D$}\,\,
      DrF(D)\,.
\end{equation*}



One can prove that $F\in\G_r$. Indeed 
choose an arbitrary point $D$.
Since $DrF(D)$, thus there exists an isometry
$g\in \G$ such that $F(D)=g(D)$. 
 Due to the lemma $g^{-1}\circ Fi=identity$, i.e. $F=g$.

Thus we see that the set $\G$ is a group.

 It remains to  prove the lemma.
Let $A$ be a fixed point of the isometry $F$,
and $BrF(B)$ for every point $B$. Consider
the domain $O_{\delta\over 2}(A)$. 
For an arbitrary
point $X$ in this  domain $F(X)$ belongs to the domain too,
since $F$ is isometry, and $F(A)=A$. Let $Y=F(X)$.
Then $YrX$, but the distance between these points is less than 
$\delta$. Hence $F(X)=X$.  We proved that $F$ is identity in
the interior of the disc.  Show that this is true for an arbotrary
point $X$. Conisder on the ray $AX$ an arbitrary point $B$
which is insider the disc $O_{\delta\over 2}(A)$.
Isometry sends line to the line, and rays to the rays, and
points $A$ and $B$ remain intact under the action of isometry
  $F$.  Hence $F(X)=X$.

\subsection{Properly discontinuous and uniformly 
discontinuous action}
   We say that group $\G$ acts on manifold $M$ {\it properly discontinuous}
if for arbitrary compact $K\subset M$, if the equation
           \begin{equation*}
     g\colon   K\cap g(K)\not=\empty                
           \end{equation*}
has only finite number of solutions.

\smallskip

  In the case if $M$ is provided with metric,
    we say that group $\G$ acts on manifold $M$ {\it uniformly 
discontinuous} if there exists $\delta$ such that $\delta>0$
and for every point $A\in M$,
           \begin{equation*}
        d(A, g(A))\leq \delta \Rightarrow g={\rm identity}
            \end{equation*}


\begin{thm}
For manifold with metric these two definitions are equivalent.
  Metric is iniformly discontinuos if and only if it is properly
discontinous
\end{thm}

\begin{proof}
 Let group $\G$ acts on manifold $M$ with metric properly discontinous.
  Show that its action is iniformly discontinuous.
  Pick an arbitrary point $A\in M$
\end{proof}


\begin{lm} Let $\G$ be uniformly discontinuous
subgroup of isometries, then
an  isometry $F\in \G$ is identity if it
has at least one fixed point    
\end{lm}


 \begin{lm}(Chasles' lemma)
Let $F$ be an isometry of $\E^2$.
  Then  the following dichotomy is obeyed:

 $F$ preserves orientation, and
 $F$ is rotation, i.e. there exists a point $O$ such that
   for an arbitrary point $B$
         $$
     F(B)=A+{\rm Rot}_\varphi(AB)
         $$             
or

  $F$ changes the orientation, and there exist a line $\bf l$,
and a vector $\N$ directed along this line 
         such that for arbitrary point $B$
           $$
F(B)={\rm Reflect}_{\bf l}(B)+\N
           $$

\end{lm}

These two lemmas imply the Theorem:

  \begin{thm}
   \end{thm}
We say that the subgroup $\G$ of isometries
is {\it properly discontinuous} if for an arbitrary
compact set $K$ 
the equation
       $$
F\colon\,\, F\in\G, K\cup F(K)\not=\empty
       $$

there exist only finite number of
isometries in $\G$ such that 
\begin{lm} If $F$ belongs to 
\end{lm}


These definitions are equivalent.
Indeed let $\G$ be properly discontinuous.


 lee d $g\in \G$
such that $g(F(D))=D$, i.e. $F=g^{-1}$.
Due to the lemma $g\circ F={\rm identity}$, i.e. 

it follows from this lemma that the set $\G$ contains all 
isometries which send 
isometries which preserve relation $r$ Indeed let $H$ be an arbitrary $r$-isometry,
 


 
One can see that $\G_r$ is subroup of isometries.





  We define the action of group $\G=\G_R$ on $\E^2$
in the following way.



 Let $\G$ be a set of isometries of $\E^2$ such that
for arbitrary two points $A,B\in \E^2$
      \begin{equation*}
      ARB \Leftrightarrow 
      \hbox{there exist $g\in \G$ such that $B=A^g$}\,,
\end{equation*}
i.e. 

\subsection {Uniformly discontinuous groups on plane}

  Let $\G$ be properly discontinbuous group acting on  $\E^2$.

First suppose that
 $\G$ does not possess transformations changing orientation.

If $\G$ possesses at least one non-identity element, hence
this element is a translation $T=T_\ac$ since rotations
(presence of fixed points) and Chasles' transformation
 (it changes  orinetation)
are forbidden.   Choose a translation such that vector $\ac$
has minimal length. It can be maximum two elements of minimal
length\footnote{if there are three vectors in general position,
then equivalent points will be very close to each other}. 
Thus we come to groups

  \begin{itemize}
     \item
   $\Z$ group of translation $\{T_{m\ac}\}$:
           \begin{equation*}
           \G\ni A\colon A(\x)=\x+m\ac\,, m=0,1,2,3,\dots
           \end{equation*}
     \item
   $\Z\times \Z$ group of translation $\{T_{m\ac+n\b}\}$:
           \begin{equation*}
           \G\ni A\colon A(\x)=\x+m\ac+n\b\,,\,\, m,n=0,1,2,3,
\dots
           \end{equation*}
\end{itemize}

Now suppose that the group $\G$ possesses at least one
element which does not preserve orientation, i.e.  Chales' e
lement
           \begin{equation}\label{chasles1}
           \G\ni S\colon S(\x)=\x_{||}+\x_\perp+\c
           \end{equation}
where vector $\c=\c_{||}$. We have $S^2=T_{2\c}$,
i.e. our grop possess subgroup $\{T_{n\ac\}}$
or subgroup $\{T_{m\ac+m\b}\}$, where $\ac,\b$
are linearly independent.

  I-case. Group $\G$ is generated by trasnformations
$S$ in \eqref{chasles1}  and translation $T_{\ac}$,
i.e.  
       $$
  S^2=T_{2\c}=T_{m\ac}, m=0,1,2,3,4,\dots
       $$    
If $m$ is even, $m=2k$, then $ST_{-k\ac}$ is reflection,
and it possess fixed points,
Hence $m$ is odd, and we see that $\c={2k+1\over 2}\ac$.
One can consider $\c={\ac\over 2}$ changing $S\mapsto ST_{k\ac}$.

If we choose the basis ${\e,\f}$ such that
  $\e$ is parallel to the axis of Chasles' reflection, and
$\f$ is orthogonal to the axis of Chasles' reflection, then
we come to

Thus we see that
       group $G$ is equal to
       \begin{equation*}
 \G \ni A\colon\,\, A(x\e+y\f)=\left(x+{m\over 2}\right)\e+
           (-1)^my\f
      \end{equation*}


  Now we consider the second case when
  group $\G$ possesses subgroup $\{T_{m\ac+n\b}\}$.

  Show that in this case $\ac$ and $\b$ have to be orthogonal
to each other. In the same way like above we conclude that
Chasles' element is reflection with respecto to axis directed
along $\ac$ and translation on the vector $\ac\over 2$.

Now notice that
        \begin{equation*}
         S\circ T_{m\b}=T_{m\b'}\circ S\,,\quad
   {\rm where}\,\, \b'=\b_{||}-\b_\perp
         \end{equation*}
and
        \begin{equation*}
         S\circ T_{m\b}\circ S=T_{m\b'}\circ T_{\ac}
         \end{equation*}
This implies that vector $\b$ has to be parallel to $\ac$
or orthogonal (not to priduce too dense lattice of 
equivalent points.)

  

\section{Main statement}

\begin{thm}
Let  $(M,G)$ be locally Euclidean $2$-dimensional manifold
in stronger sense, i.e. both conditions
\eqref{loceuclid} and \eqref{stronglyloceuclid} are obeyed.


Then  
\end{thm}

  We prove this Theorem in three steps.

  I-st step. We will construct surjection  $\pi\colon \E^2\to M$
which is isometry for small distances (less than $r$)


 \medskip

   II-step On the base of this surjection we will 
consider the subroup
of isometries which preserve $\pi$
              $$
\G_\pi=\{F\colon \,\, F\,\, 
 \hbox{is isometry and }\,\,\pi\circ F=\pi \}
              $$ 
and will prove that this group is uniformly discontinuous.

\begin{proof}

Then we will prove that the Riemannian manifold $M$ is isometric
to the factor of $\E^2$ with respect to the gropu $\G_\pi$,
i.e. $M$ is  plane $\E^2$ or cylindre, or  twisted cylinder,
or torus, or Klein bottle.


Choose two arbitrary points $D$ on $\E^2$
and $\D$ on $M$ and conisder the covering 
$\pi\colon\E^2\to M$
such that $\pi(D)=\D$ and for every point $X\in \E^2$
 $\pi(X)$ is defined in the following way
   
Consider the vector $\X=OX$.

Choose points $D=X_0,X_1,X_2,dots X_n=X$
on the segment  $DX$ such that 
the distance between these points is less than $r$,
recurrently  define $F(X_i)$ for $i=0,1,2,\dots,n$.

First define $\pi(X_1)$ as a point on manifold $M$
with coordinates $u_\D=(X_1)_1, v_\D=(X_1)_2$
where  $(u_\D,v_\D)$ are Euclidean coordinates
adjusted to the point $\pi(D)=\pi(X_0)=\D$
(see section\label{localeuclidgeometry}) , and 
$\left((X_1)_1, (X_1)_2\right)$ are components
of the vector $DX_1$, then recurrently:
if we already have  defined $\pi(X_i)$ for point $X_i$
($i=1,2,\dots,n-1$)
we define $\pi(X_{i+1})$ as a point on manifold $M$
with coordinates 
          $$
 u_{X_i}=(X_{i+1})_1-(X_{i})_1, 
v_{X_i}=(X_{i+1})_2-(X_{i})_2\,,
              $$
where  $(u_{X_i},v_{X_i})$ are Euclidean coordinates
adjusted to the point $X_i$ and 
$\left((X_{i+1})_1-(X_{i})_1, (X_{i+1})_2-(X_{i})_2\right)$ 
are components
of the vector $X_i X_{i+1}$.

One can see that function $\pi$ is well defined
on all the plane, its value does not depend on
on a choice of partition 
$O=X_0,X_1,X_2,dots X_n=X$.
\end{proof}


One can show that the map $\pi\colon \E^2\to M$
preserves small distances:
                \begin{equation*}
        d(\varphi(A),\varphi(B))=d(A,B)\,,\quad
    {\rm if}\,\, d(A,B)<r
              \end{equation*}
(here as always $r$ is parameter which defines locality
of geometry on $M$ (see \eqref{stronglyloceuclid}))

This is evident for points which are at the distance
$<r$ from initial pointi $D$, and one can prove this recurrently
for arbitrary close point $A,B$ considering partitions
  $D=A_0,A_1,\dots,A_n=A$ and
  $D=B_0,B_1,\dots,B_n=B$  and `small' trapecies
 $A_iB_iB_{i+1}A_{i+1}$. 

(see details in {Shaf})
One can see that the map $\pi$ is surjection.               

Prove it. Let $\B$ be an arbitrary point on $M$.
Consider the set of points  $\B_0\B_1\B_2\dots\B_n$
such that $\B_n=\B$ and the distance between points
is less tham $r$. Thus we will see recurrently that all 
these points  are covered by $\pi$.


  Choose an arbitrary point $\B\in M$.

Now we will conisder a group $\G$ and will show
that 
        $$
   M=\E^2\backslash \G
        $$
We define this group as a group of isometries
which preserve the covering map $\pi$, i.e.
Consider an arbitrary isometry $F$ of $\E^2$.
We say that isometry $F$ belongs to group $\G$
for arbotrary point $D\in \E^2$

     \begin{equation*}
	\pi\left(F\left(D\right)\right)=     
\pi\left(D\right)\,,
     \end{equation*}
i.e.     
\begin{equation*}
\G=\{\hbox {isometries $F$ of $\E^2$  such that}\,\, 
\pi\circ F\equiv\pi\}
     \end{equation*}
One can see that this is a group.
We still now nothing about the group $\G$,
except that it possesses identity element.
Hovever we can easy to prove that
this group is
uniformly discontinuous.

Let  $F\in \G$ be an arbitrry non-identical element in $\G$
\footnote{notice that we still know nothing
does this element exist or no.},
i.e  there exists $B\in \E^2$ such that $B'=F(B)\not=B$.

      Suppose that  $d(B,B')<r$. Consider on manifold
 $M$ points $\B=\pi(B)$ and $\B'=\pi(B')$.
 Then due to the properties of surjection
  $\pi$, $d(\B,\B')<r$ also, i.e. the point
  $\B'$ belongs to the chart $(u_\B,v_\B)$. 
$\pi$ is local bijection, hence $\B\not=\B'$.
On the other hand $\B=\B'$ since $F\in \G$.
  Contradiction. 

\begin{thm}
The surjection $\pi$ is a covering. The group
$\G$ acts freely on the preimages $\pi^{-1}(\D)$.
\end{thm}

 

 Consider preimages of the  point $\B\in M$.




  Let $\{B_i\}$ be a set of the points such that
            $$
  \hbox{for every $B_i$}\,\,   \pi(B_i)=\B
            $$
This set possesses at least one point. 

{\bf Exercise}

Consider 

\end{document}






\begin{thm}  $\G$ is subgroup of transformations.
\end{thm}

\begin{lm}ioo
\end{lm}

\begin{lm}ioo
\end{lm}


  {\bf Exercise}  Let $H$ be an isometry such that
      it sends equivalent points to the

          
     \centerline {Covering  $M\to \E^2$ }

Pick an arbitrary point $D$ of $\E^2$ and an arbitrary point
  $\cal D$ on $M$ and define a map  
$\Phi_0\colon \,,D_\vare\leftrightarrow {\cal D}_\vare$.

  Now for an arbitrary point $K$ on $\E^2$ we  define $\Phi(K)$
considering the continuaqy
tion of the map  


%\end{document}





\bye
