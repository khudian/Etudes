
%  \magnification=1200 
% this is the file of the Galois lecture
% which I did 22 November 2017

\documentclass[12pt]{article}
\usepackage{amsmath,amsthm}


\usepackage{amsmath,amssymb,amsfonts,amsthm}

\newtheorem{theorem}{Theorem}
\newtheorem{proposition}{Proposition}
\newtheorem{lemma}{Lemma}
\newtheorem{corollary}{Corollary}
\theoremstyle{definition}
\newtheorem{definition}{Definition}%[section]
%\theoremstyle{remark}
\newtheorem{example}{Example}%[section]
\newtheorem{remark}{Remark}%[section]
%

%\theoremstyle{theorem}
%\newtheorem{thm}{Theorem}

%\theoremstyle{lemma}
%\newtheorem{lm}{Lemma}


%\theoremstyle{remark}
%\newtheorem{rm}{Remark}

%\theoremstyle{Lemma}
%\newtheorem{lm}{Lemma}

\numberwithin{equation}{section}


\title{Locally Euclidean Geometries and hyperbolic geometry}
\date{}



\begin{document}
\maketitle




%\baselineskip=14pt
\def\vare {\varepsilon}
\def\AA {{\bf A}}
\def\A {{\cal A}}
\def\t {\tilde}
\def\bs {{\bf s}}
\def\a {\alpha}
\def\d {\delta}
\def\K {{\bf K}}
\def\N {{\bf N}}
\def\w {\omega}
\def\s {{\sigma}}
\def\S {{\Sigma}}
\def\s {{\sigma}}
\def\p{\partial}
\def\vare{{\varepsilon}}
\def\Q {{\bf Q}}
\def\D {{\cal D}}
\def\B {{\cal B}}
\def\G {{\Gamma}}
\def\C {{\bf C}}
\def\R {{\bf R}}
\def\L {{\cal L}}
\def\F {{\cal F}}
\def\Z {{\bf Z}}
\def\U  {{\cal U}}
\def\H {{\bf H}}
\def\S  {{\bf S}}
\def\E  {{\bf E}}
\def\l {\lambda}
\def\degree {{\bf {\rm degree}\,\,}}
\def \finish {${\,\,\vrule height1mm depth2mm width 8pt}$}
\def \m {\medskip}
\def\p {\partial}
\def\r {{\bf r}}
\def\pt {{\bf pt}}
\def\v {{\bf v}}
\def\n {{\bf n}}
\def\t {{\bf t}}
\def\b {{\bf b}}
\def\c {{\bf c }}
\def\e{{\bf e}}
\def\ac {{\bf a}}
\def \X   {{\bf X}}
\def \Y   {{\bf Y}}
\def \x   {{\bf x}}
\def \y   {{\bf y}}
%\def \G{{\cal G}}
\def\ss  {\sigma_{\rm sph}}
\def \grad {{\rm grad\,}}
\def\e {{\bf e}}
\def\f {{\bf f}}
\def\g {{\bf g}}




  \centerline {H.M.Khudaverdian}

\m

\centerline {{\tt GALOIS GROUP} lecture for UG students, 
             University of Manchester}


\m

     \centerline {22 November 2017}

\m

{\tt The talks is based on the book 
``Geometry and groups'' of 
S.Nikuin, R.Shafarevich}

\medskip

\begin{abstract}


Geometry on the surface of the cylinder is 
locally Euclidean. An "ant-mathematician" who lives 
on the cylinder will not  distinguish the geometry of  
the surface at small distances from the Euclidean  
geometry; the Pythagorean Theorem will be almost the same, 
and for "not too large" triangles the sum of the angles will
 be $\pi$.
      In the first part of talk, we will study locally 
Euclidean two-dimensional geometries. We will study  
these geometries by using discrete subgroups of the 
isometry group of the Euclidean 
plane $E^2$. The list of locally Euclidean geometries is 
exhausted by the geometries  on the surface of the cylinder, 
on the surface of the torus, on the surface of the 
"twisted cylinder" (the Moebius band), 
and on the so-called Klein bottle.
         In the second part of the talk, we will 
consider the set of locally Euclidean  geometries, 
and will show that this set can be naturally 
parametrized by the points of the Lobachevsky 
(hyperbolic) plane.


\end{abstract}


\tableofcontents

\section {Locally Euclidean surfaces}

\subsection 
{ (Uniformly) locally Eucldean surfaces}

  We consider locally Euclidean $2$-dimensional 
geometries.  An arbitrary $2$-dimensional  geometry 
can be considered as
$2$-dimensional  Riemannian surface, $(M,G)$.
$M$ is a surface, and $G$ defines scalar product of
tangent vectors, i.e. length of an arbitrary curve.
For arbitrary curve
  $\x=\x(t), t_1\leq t\leq t_2$
             $$
 \hbox {length of the curve}=
   \int_{t_1}^{t_2} \sqrt{\left(\v(t),\v(t)\right)}dt\,,\,
\hbox{scalar product}\,\,  
\left(\v(t),\v(t)\right)=G(\v(t),\v(t))\,, 
             $$
where $\v(t)$ is velocity vector. 
 In local coordinates $x^i$,
the curve has appearance  $\x(t)=x^i(t)$,
    $G=g_{ik}(x)dx^idx^k$,
  $\v(t)=v^i(x(t))\p_i$,
and 
the scalar product of velocity vector on itself 
is equal to
      $$
\left(\v,\v\right)=
 v^i(x(t))g_{ik}(x(t))v^k(x(t))=
 {d x^i(x(t))\over dt}
   g_{ik}(x(t))
 {d x^k(x(t))\over dt}\,,
      $$
i.e.
      $$
 \hbox {length of the curve}=
   \int_{t_1}^{t_2} \sqrt{\left(\v(t),\v(t)\right)}dt
      =
    \int_{t_1}^{t_2} \sqrt{
  {d x^i(x(t))\over dt}
   g_{ik}(x(t))
 {d x^k(x(t))\over dt} \,.   
             }dt
      $$
We consider surfaces which locally look as Euclidean plane.

We say that $M$ is uniformly locally Eucldean
if

\begin {itemize}

\item 
in a vicinity of arbitrary points there exist
 Eucldean coordinates, i.e. the coordinates  $u,v$
such that  $G=du^2+dv^2$ in these coordinates.


\item this  neighborhood is enough large:
there exists $r>0$ such that 
   in a vicinity of arbitrary point there exist
Eucldiean coordinates  $u,v$ which are  defined at least in
the circle of radius $\geq r$.

\end{itemize}



\medskip

{\bf Exercise}  Show that the surface of 
cylindre
is locally Eucldean.

\smallskip

{\bf Exercise}  Show that the surface of sphere
is not locally Eucldean.

\smallskip

{\bf Exercise}  Show that domain $a<x<b$ of $\E^2$
is locally Eucldean but it is not uniformly 
locally Eucldean, and compare it with surface
of cylindre, (which is uniformly locally 
Euclidean.)

(We suppose that metric on the surface
$M$ in $\E^3$
is  the metric induced from $\E^3$.)

In this talk we will consider only uniformly locally 
Euclidean surface, and we will call 
them sometimes just
locally Eucldean.



  \subsection
{Examples of locally Euclidean surfaces and 
subgroups of $\E(2)$.}

We considered surface of cylindre. It is
uniformly locally Euclidean surface.
  How to come to another examples.
Here we suggest the regular way to come
to all such examples. It is in the spirit of
Klein Erlangen programme.



\subsubsection
{Subgroups of group $\E(2)$ and surfaces}

 
  Let $\G$ be an arbitrary subgroup
of group of isometries of $\E^2$.


Assign to the group $\G$ a
space $M_\G$ of orbits of $\G$-group action
on $\E^2$,
$M_\G=\E^2\backslash \G$.

 We denote the points of the space $M_\G$, 
by handwriting letters $\A,\B,C,\D,\dots$. These
points are orbits of group $\G$ action.
Every  point $A\in \E^2$ produces the point
$\A\in \E^2\backslash \G$, the equivalence class
of a point $A$ with respect to the group $\G$:
         $
\A=[A]_\G\,,\qquad  g\in \G\,,
A'=g(A)\in [A] 
        $.

  To establish the geometry on $M$
we define the distance between points $\A,\B$
as the minimal distance between the orbit
   $\{A^g\}$ and $\{B^g\}$:
            $$
\hbox{if $\A=[A]$ and $\B=[B]$ then}\,\,
d(\A,\B)=\min_{g,g'\in\G}d(A^g,B^{g'})
            $$ 
{\bf Exercise 1}  Let $\ac\not=0$ be an arbitrary vector
in $\E^2$ and $\G=\{T_{n\ac}\}$ be a group of translations
generated by the translation on vector $\ac$:
                 $$
T_{n\ac}\colon \r\to \r+n\ac\,.
                 $$

  Describe the geometry $M_\G$
and show that this is geometry of cylinder.



\m

{\bf Exercise 2}  Let $\G=C_2$ be group of reflection
with respect to the line $l$.

  Describe the geometry $M_\G$
and show that this is not uniformly locally Euclidean manifold.

  



\subsubsection 
{Uniformly discontinuous subgroups of $\E(2)$}

We say that the subgroup $\G$ acts properly discontinuous
on $\E^2$ if there exists $\delta$ such that
for an arbitrary point $A\in \E^2$, and for an arbitrary
non-identity element $g\in \G$ 
        $$
      d(A,g(A))\geq \delta\,.
        $$
 In other words it means that the distance between 
distinct points of an arbitrary orbit exceeds the  $\delta$.

Why these groups are interesting?  
Because every such a group
defines locally Eucldean manifold\footnote
{In fact the inverse implication is true also,
 and 
the set of all
 (uniformly) locally Euclidean surfaces
is in one-one correspondence with the
set of all uniformly discontinuous subgroups
of group $\E^2$.(see further)}.  



\begin{proposition}\label{1}

If action of group $\G$ has a fixed point
 (there exists $\r_0\in \E^2$ such that 
for all $g\in \G$,
$g(\r_0)=\r_0$) ({\it il faut dire mieux}),
then $M_\G$ is not uniformly locally 
Euclidean.
\end {proposition}
\begin{proof}
 If $A=\r_0$ is a fixed point, then
for arbitrary $g\not=1$  $d(A,A^g)\geq \delta$, and
on the other hand $d(A,A^g)=0$, if $\G$ 
acts uniformly discontinuou. Contradiction.
\end{proof}


\begin{proposition}\label{2}
If $\G\in \E(2)$  is uniformly discontinuous group, then
the surface  $M_\G=\E^2\backslash \G$ is uniformly local.  

\end{proposition}

\begin{proof}.

Let $\G$ be uniformly discontinuous, i.e.
there exists $\delta >0$ such that 
for an arbitrary $g\in \G$
and for an arbitrary point $A\in \E^2$
         $d(A^g,A)<\delta \Rightarrow
   g=1$.
  Let $B$ be an arbitrary point  
which belongs to the disc $D_{\delta\over 2}(A)$
Consider orbits $\A$ and $\B$ of these points.
It is easy to see from triangle inequality that
for arbitrary points $A'\in\A$ and $B'\in \B$
the distance between these points is bigger or equal to
  $r={\delta\over 2}$. Indeed let $A"=A^g$, and 
$B'=B^h$. Denote $\tilde B=B^{hg^{-1}}$. Then
$d(A',B')=
d\left(A, \left(B^h\right)^{g^{-1}}\right)=
d\left(A, \tilde B\right)$, and by triangle inequality
     $$
d(A',B')=d(A,\tilde B)\geq \left|d(B,\tilde B)-d(A,B)\right|
>\delta
     $$
if $B\not=\tilde B$.
Thus we see that in the case if two points $A$
and $B$ are closer than $\delta\over 2$,
then the distance between orbits $\A$ and $\B$
is equal to the distance $d(A,B)$.
This implies  that $M_\G$ is uniformly locally Euclidean
if $\G$ is uniformly discontinous.




\end{proof}

\subsubsection {Classiication of all
uniformly discontinuous subgroups of $\E(2)$}


  First of all recall the classic Theorem:
 
\begin{theorem} (Chazles+?)
  Any isometry of $\E^2$ is rotation, or translation or
  glided reflection.
\end{theorem}

  This Theorem possesses two statements. First that an arbitrary
(even non-linear map ) which is isometry has appearance
            $$
   F(\r)=A\r+\b\,,
            $$ 
where $A$ is linear operator, and the second statement
that this linear map is rotation (with respect to some centre)
or translation or glided reflection.

    We can prove the first statement under the assumption that
  $F(\r)$ is smooth map of $\E^2$ in $\E^2$.


     Suppose now  that $F(\r)=A(\r)+\b$.
   
\smallskip

  I-st case) orientation is preserved, i.e. $\det A=1$.
If $A=1$ then it is translation, if $A\not=1$ then operator
$A-1$ is invertible, and
              $$
\r=A(\r)+\b=\ac+\left(A\left(\r-\ac\right)\right)\,,
{\rm where}\,\,,  \ac= (1-A)^{-1}(\b)\,,
              $$
i.e. this affine transformation is a rotation around
the point $O-\ac$.

\smallskip

II-nd case) orientation is not preserved, i.e. $\det A=-1$.
  This operator has eigenvector  $\n$ 

One can see that we come to reflection with respect to the line
along the vector $\n$ and translation along $\n$, i.e. glided
reflection.
 


  Now using this Theorem classify 
   uniformly discontinuous subgroups of isometry group.



   Let $\G$ be a subgroup of $\E(2)$ which acts on
$\E^2$ uniformly discontinuous, i.e. there exists
$\delta>0$ such that for an arbitrary point $A\in \E^2$
and an arbitrary $g\in \G$
                  $$
                 d(A,A^g)<\delta 
       \Rightarrow g=1\,.
                   $$
 It follows from proposition \ref{1} that the subgroup $\G$
contains only translations and non-trivial 
glide reflections.

  If group $\G=e$ this is trivial: $M_\G=\E^2$.

Denote by $\G^{(0)}$ the subgroup 
of $\G$
which preserve orientation, i.e. 
subgroup of translations.


\begin{proposition}\label{3}
The subgroup $\G^{(0)}$ of uniformly discontinuous
grroup $\G$
of orientation preserving transformations
is 

\begin{itemize}
\item the group of translations generated
  by arbitrary non-zero vector $\ac$
        $$
 \G^{(0)}=\G^{(0)}0_{\ac}=
\{T_{n\ac}\colon T_{n\ac}(\r)=\r+n\ac,\,,
{\rm where}\,\, n=0,\pm 1,\pm 2,\dots\}
        $$
\item the group of translations generated
  by arbitrary two non-zero linearly independent 
vectors $\ac,\b$
        $$
 \G^{0)}=\G^{(0)}_{\ac,\b}=
\{T_{m\ac+n\b}\colon T_{m\ac+n\b}(\r)=
\r+m\ac+n\b,\,,
{\rm where}\,\, m,n=0,\pm 1,\pm 2,\dots\}
        $$
\end{itemize}


\end{proposition}

On the base of this proposition study 
the general case.

\m

Let $\G$ be an arbitrary uniformly 
discontinuous subgroup of $\E(2)$.
Then $\G^{(0)}=\G^{(0)}_\ac$
or $\G^{(0)}=\G^{(0)}_{\ac,\b}$.

{\tt First case} $\G^{(0)}=\G^{(0)}_\ac$.

if $\G=\G_0=\{T_{n\ac}\}$,
and $M=\E^2\backslash \G$ is cylindre.

Now suppose  $\G\not=\G_0$,
i.e. $\G$ possesses glide reflections.
Let $S=S_{l,\b}\in \G$
(reflection with respect to the line 
$l$ directed along the vector
$\b$ and translation
on the vector $\b$ ). Then
$S^2=T_{2\b}$. Hence $\b={k\ac\over 2}$
for some integer $k$. This integer $k$ has to be odd,
since if $p=2p$ then the transformation $T_{-p\ac}S$
is the reflection, and it possesses fixed point
(see the porposition \ref{1}). 
We see that in this case $G$ is generated by translation
$T_\ac$ and glided reflection  $S_{l,{\ac\over 2}}$

In this case
$M=\E^2\backslash \G$ is twisted cylindre, 
(Mobius strip
with infinite sides)
\m


{\tt Second case} $\G^{(0)}=\G^{(0)}_{\ac,\b}$.

Let   $\G_0=\{T_{n\ac}\}$. Study again 
 two possibilites.

First:
$\G=\G_0=\{T_{n\ac}\}$. In this case

$M=\E^2\backslash \G$ is a torus.


Suppose now that $\G\not=\G_0$ and it possesses glided
reflections. One can see that this can happen
only if vectors $\ac$ and $\b$ are orthogonal to each
other. We will come to
$M=\E^2\backslash \G$ is a Klein bottle.

We come to the theorem classifying
all the locally Euclidean surfaces corresponding
to uniformly discontinuous isometry subgroups.

 \begin{theorem}  Let $\G$ be 
an arbitrary uniformly discontinuous
subgroup of isometries group of $\E^2$. 
Then the following possibilities
may occur ({\it ca il faut dire mieux!!!})

\begin{itemize}

\item
I-st case (trivial)

$\G={e}$ has only identity element. Then $M=M\backslash \G=\E^2$ 

\item
II-nd case 


Group $\G=\{T_{m\ac}\}$
is generated by translation on vector $\ac$,
where $\ac$ is an arbitrary non-zero vector.
Then $M=M\backslash \G$ is cylindre 

\item
III-rd  case 


Group $\G\colon \G^{(0)}=\{T_{n\ac}\}$,
however, $\G\not=\G^{(0)}$. This group 
is generated by translation on vector $\ac$,
and glide reflection $S_{l,{\ac\over 2}}$,
where the line
$l$ goes along the vector $\ac$
Then $M=M\backslash \G$ is twisted cylindre 
(Mobius).

\item
IV-th case 


Group $\G=\{T_{m\ac+\n\b}\}$
is generated by translation on vectors $\ac$
and $\b$,
where $\ac,\b$ are  arbitrary 
linearly independent vectors.
Then $M=M\backslash \G$ is a torus

\item
V-th case 


Group
is generated by translation on 
vectors $\ac$
and $\b$, and glided 
reflection $S_{l,{\ac\over 2}}$,
where the line
$l$ goes along the vector $\ac$
where $\ac,\b$ are  arbitrary non-zero vectors which are
{\it orthogonal!} to each other.
Then $M=M\backslash \G$ is a Klein bottle.

\end{itemize}
\end{theorem}

\subsection  {The final classification}

We classified in the Theorem above all 
locally Euclidean surfaces which are generated
by uniformly discontinuous groups.

Is an arbitrary uniformly locally Euclidean surface
$M$ generated by uniformluy discontinuous group?
  Yes!


To prove this statement we consider the 
following construction.

Take an arbitrary point $A\in \E^2$ 
and $\A\in M$. Choose  arbitrary Cartesian
coordinates $x,y$ in $\E^2$ adjusted to the point
$A$ ($x_A=y_A=0$ and choose 
arbitrary local Euclidean 
coordinates  $\{u_0,v_0\}$ on $M$
in vicinity of point $\A$ ($u_M=v_M=0$).
Thus we define a map $\E^2\to M$
  for an arbitrary point $\r\in \E^2$
which is in the disc $D_r(A)$.
If point is on the distance bigger than $r$
but less than $2r$ we can consider two discs, e.t.c....


This we will construct a map
   $$
  \begin{matrix}
        \E^2\cr
         p\downarrow\cr
          M\cr
     \end{matrix}
      $$
such that $p(A)=\A$.
Taking the preimage of arbitrary point
$\B\in M$ one come to the set of points
 which define the uniformly discontinuous group.


We come to 

\begin{theorem}  Let $M$ be an arbitrary 
uniformly locally Euclidean surface. Then there
exists uniformly discontinuous group
  $\G$ such that 
       $$
    M=M_\G
       $$
\end{theorem}

It follows from this Theorem and Proposition \ref{3}
the following corollary

\begin{corollary} Let  $M$
be an arbitrary uniformly locally Eucldean
surface. Then $M=\E^2$ or $M$ is cylindre,
or $M$ is twisted cylidnre,
or $M$ is torus, or $M$ is Klein bottle.
\end{corollary}


\section {Space of locally Eucldean geometries}

  Geometries on cylindres are similar to each other,
  the same about geometries of twisted cylindres and Klein 
bottles.

  Consider the space of geometries on tori.

Every lattice $T_{\ac,\b}$ defines
geometry $M_{\ac,b}=M\backslash 
{\{T_{m\ac+n\b}\}}$.

Questions arises. 

  At what extend lattices depend on vectors
$\ac,b$. At what extend geometries (tori)
 $M_{\ac,\b}$ depend on lattices?

  What is a geometry in the space
of geometries on tori?


\subsection {Lattices---
Geometries on tori---unimodular group }

\begin{definition}
Consider pair of arbitrary 
lattices $T_{\ac,b}$
and $T_{\ac',\b'}$ and corresponding tori
  $M_{\ac,\b}$ and $M_{\ac',\b'}$.

\smallskip
Two geometries 
  $M_{\ac,\b}$ and $M_{\ac',\b'}$
are the same if 
    \begin{equation}\label{sim1}
\hbox{lattices $T_{\ac,b}$ and $T_{\ac',\b'}$
coincide}\,.
   \end{equation}

\smallskip

Two geometries are the same if
      \begin{equation}\label{sim2}
\hbox{there exists isometry $F$ such that
$\ac'=F(\ac)$ and $\b'=F(\b)$\,.}
   \end{equation}


Two geometries  $M_{\ac,b}$
and $M_{\ac',b'}$ are similar if 
       \begin{equation}\label{sim3}
\ac'=\lambda\ac\,, \b'=\lambda \b\,.
   \end{equation}


\end{definition}

The first and second  conditions are obvious,
the condition is natural. 
They lead
to very amazing consequences.

\m

\begin{proposition}
Two lattices $T_{\ac,\b}$ and $T_{\ac',\b'}$
coincide if and only if
            $$
  \begin{pmatrix}
      \ac'\cr
       \b'\cr
  \end{pmatrix}=  
     \begin{pmatrix}
      p & q\cr
        r & s\cr
  \end{pmatrix}
     \begin{pmatrix}
      \ac\cr
       \b\cr
  \end{pmatrix}
            $$
such that
      $$
    ps-qr=\pm 1
      $$
and
       $$
p,q,r,s \quad \hbox {are integers}
       $$
in other words if these lattices are related by
unimodular transformation in integers.


\end{proposition}


The proof is evident, and it is very illuminating
tpo do the following exercises.

\m

{\bf Exercise}
Check that 
 lattices 
     $$
  T_{\ac,\b} \quad {\rm and}\quad
   T_{41 \ac+5\b, 8\ac+\b}
     $$
are the same in spite of the fact that 
parallelograms 
$\Pi_{\ac,\b}$ and
$\Pi_{41 \ac+5\b, 8\ac+\b}$
look very different!

\smallskip

{\bf Exercise}
Prove that lattices $T_{\ac,\b}$
and $T_{\ac',\b}$
coincide if and only if
parallelograms  $\Pi_{\ac,\b}$
and $\Pi_{\ac',\b'}$
have the same area.

\m

Now identify all similar tori and tori 
which are related
with each other by unimodular transfromation, 
and find distance between geometries.

Let $T_{\ac,\b}$ be a lattice.

Assign to this lattice the  complex number
         $$
z={a_x+ia_y\over b_x+ib_y}
         $$
in the case if $\ac\times \b$ is poisitve, and
the inverse complex number in the case if
  $\ac\times \b$ is negative.

Thus using equations \eqref{sim2}
and \eqref{sim3} we come to 

\begin{proposition}  Set of all lattices
is in one-one correspondence with
points of upper half-plane $\bf H$.
\end{proposition}

It follows from \eqref{sim1}
the following fundamental statement

\begin{theorem}
Set of all lattices
is the set of all points of upper half-plane,
and two points define the same geometry
if they are related with transformation 
  \begin{equation}\label{sim4}
   z'={pz+q\over rz+s}\,,{\rm with}\,
      ps-qr=1\,,
   \quad {\rm or}\,\,
   z'={p\bar z+q\over r\bar z+s}\,,{\rm with}\,
      ps-qr=-1\,,
        \end{equation}
where
      \begin{equation}\label{integers}
\hbox{p,q,r,s, are integers}
    \end{equation}
i.e. they are related with transformation
of group $PSL_2(\Z)$ and its conjugate. 
\end{theorem}




Notice that we come to Lobachevsky (hyperbolic)
plane \footnote
{If we omit the condition \eqref{integers}
we come to the group \eqref{sim4} 
of all 
holomorphic bijections
${\bf H}\leftrightarrow \bf H$.}.

 One can see using equation \eqref{sim1}
that the following statement holds:


\begin{theorem}
The set of all geometries on tori 
can be parameterised by the points 
of Lobachevsky (hyperbolic) plane.
 Two geometries are similar if and only if
they are related by transformation
\eqref{sim4}, i.e. transformation of
 group $PSL(2,R)+{\rm conjugate}$ . 
\end{theorem}


\subsection {Geodesics and Metric of hyperbolic geometry}


  Group of isometries defines geodesics and metrics
(up to proportionality) of hyperbolic geometry

 We know that ransformations 
  \begin{equation}\label{hypisometry}
 z'={pz+q\over rz+s}\,,{\rm with}\,
      ps-qr>0\,,
   \quad {\rm or}\,\,
   z'={p\bar z+q\over r\bar z+s}\,,{\rm with}\,
      ps-qr<1\,,
 \end{equation}
i.e. transformations \eqref{sim4} without
condition \eqref{integers}\footnote{One can see that
this is equivalent \eqref{sim4} in spite of the fact that
we did not put the condition of unimodularity} 
are isometries of this geometry. 
\m

\subsubsection{Hyperbolic isometries}\label{exercises}


{\bf Exercise} Show that translations
are hyperbolic isometries


\smallskip

\noindent {\bf Exercise} Show that dilations 
are hyperbolic isometries


\smallskip


\noindent {\bf Exercise}  Show that inversion with 
centre at $y=0$ is hyperbolic isometry. 


\smallskip

\noindent {\bf Exercise}  Show that arbitrary
hyperbolic isometry is composition of these transformations. 



\subsubsection {Geodesics as locus of fixed points}



  \begin{lemma}
Let a curve $C$ be a locus of fixed points of an isometry.
Then this curve is geodesic.
\end{lemma}

\m

{\bf Exercise} Prove this lemma

\m

Using this lemma and exercises in \ref{exercises} 
one can see that
an arbitrary upper-half circle with the centre
at $y=0$ and
vertical lines are geodesics.

\subsubsection
  {Formula for distance}



A distance on $\H$,
$d(z_1,z_2)$  has 

\begin{itemize}

\item
  to obey metric axioms: positive, triangle,.....

\item
 has to be  $PSL(2,\R)+{\rm conjug.}$-nvariant,
i.e. invariant  with respect to transformations
\eqref{hypisometry}


\item
be additive on geodesics


\end{itemize}

\begin{theorem}
Let $G$ be a Riemannian metric which is 
invariant with resepct to the
action of the group \eqref{hypisometry} of $\bf H$.
 Then  up to multiplier is defined
by equation
         $$
G={dz d\bar z\over (z-\bar z)(\bar z-z)}
       $$

\end{theorem}

This theorem follows from the lemma 

  \begin{lemma}
Let $d=d(ia,ib)$ be metric on vertical line $y=0$,
($a,b>0$).
  Then
         \begin{equation*}
      d(ia,ib)=\left|\log\left(a\over b\right)\right|\,,
       \end{equation*}
\end{lemma}
since using group of isometry we can define the distance
between two arbitrary points.

Prove the lemma

\begin{proof}Transformation $z\to\lambda z$, and 
$z\to -{\lambda\over z}$ are symmetry transfromations,
hence for arbitrary points $z=ia,ib$
and for arbitrary real $\l$
   $$
d\left(ia,ib\right)=
d\left(i\lambda,i\lambda b\right)=
d\left(i{\lambda\over a},i{\lambda\over b}\right)=
   $$
Moreover since vertical line is geodesic hence
for arbitrary three points $ia,ib,ic$
      $$
   d(ia,ic)=d(ia,ib)+d(ib,ic)\,,\qquad
{\rm if}\,\, a<b<c
      $$
It follows from these equations that 
$d(ia,ib)=\log {a\over b}$



\end{proof}
\end{document}

\bye
