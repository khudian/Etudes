
%  \magnification=1200 

\documentclass[12pt]{article}
\usepackage{amsmath,amsthm}


\usepackage{amsmath,amssymb,amsfonts,amsthm}

\newtheorem{theorem}{Theorem}
\newtheorem{proposition}{Proposition}
\newtheorem{lemma}{Lemma}
\newtheorem{corollary}{Corollary}
\theoremstyle{definition}
\newtheorem{definition}{Definition}%[section]
%\theoremstyle{remark}
\newtheorem{example}{Example}%[section]
\newtheorem{remark}{Remark}%[section]
%

%\theoremstyle{theorem}
%\newtheorem{thm}{Theorem}

%\theoremstyle{lemma}
%\newtheorem{lm}{Lemma}


%\theoremstyle{remark}
%\newtheorem{rm}{Remark}

%\theoremstyle{Lemma}
%\newtheorem{lm}{Lemma}

\numberwithin{equation}{section}


\title{Locally Euclidean Geometries}
\date{}



\begin{document}
\maketitle

 \centerline {H.M. Khudaverdian}



%\baselineskip=14pt
\def\vare {\varepsilon}
\def\AA {{\bf A}}
\def\A {{\cal A}}
\def\t {\tilde}
\def\bs {{\bf s}}
\def\a {\alpha}
\def\d {\delta}
\def\K {{\bf K}}
\def\N {{\bf N}}
\def\w {\omega}
\def\s {{\sigma}}
\def\S {{\Sigma}}
\def\s {{\sigma}}
\def\p{\partial}
\def\vare{{\varepsilon}}
\def\Q {{\bf Q}}
\def\D {{\cal D}}
\def\B {{\cal B}}
\def\G {{\Gamma}}
\def\C {{\bf C}}
\def\R {{\bf R}}
\def\L {{\cal L}}
\def\F {{\cal F}}
\def\Z {{\bf Z}}
\def\U  {{\cal U}}
\def\H {{\bf H}}
\def\S  {{\bf S}}
\def\E  {{\bf E}}
\def\l {\lambda}
\def\degree {{\bf {\rm degree}\,\,}}
\def \finish {${\,\,\vrule height1mm depth2mm width 8pt}$}
\def \m {\medskip}
\def\p {\partial}
\def\r {{\bf r}}
\def\pt {{\bf pt}}
\def\v {{\bf v}}
\def\n {{\bf n}}
\def\t {{\bf t}}
\def\b {{\bf b}}
\def\c {{\bf c }}
\def\e{{\bf e}}
\def\ac {{\bf a}}
\def \X   {{\bf X}}
\def \Y   {{\bf Y}}
\def \x   {{\bf x}}
\def \y   {{\bf y}}
%\def \G{{\cal G}}
\def\ss  {\sigma_{\rm sph}}
\def \grad {{\rm grad\,}}
\def\e {{\bf e}}
\def\f {{\bf f}}
\def\g {{\bf g}}



  \centerline    {\bf Locally Euclidean geometries and 
       hyperbolic geometry}

  \centerline {H.M.Khudaverdian}

\centerline {Galois lecture for UG students, University of Manchester}



{\tt This talks is based on the book of S.Nikuin, R.Shafarevich}

\medskip

\begin{abstract}


Geometry on the surface of the cylinder is 
locally Euclidean. An "ant-mathematician" who lives 
on the cylinder will not  distinguish the geometry of  
the surface at small distances from the Euclidean  
geometry; the Pythagorean Theorem will be almost the same, 
and for "not too large" triangles the sum of the angles will
 be $\pi$.
      In the first part of talk, we will study locally 
Euclidean two-dimensional geometries. We will study  
these geometries by using discrete subgroups of the 
isometry group of the Euclidean 
plane $E^2$. The list of locally Euclidean geometries is 
exhausted by the geometries  on the surface of the cylinder, 
on the surface of the torus, on the surface of the 
"twisted cylinder" (the Moebius band), 
and on the so-called Klein bottle.
         In the second part of the talk, we will 
consider the set of locally Euclidean  geometries, 
and will show that this set can be naturally 
parametrized by the points of the Lobachevsky 
(hyperbolic) plane.


\end{abstract}


\tableofcontents

\section {(Uniformly) locally Eucldean surfaces}

  We consider locally Euclidean $2$-dimensional 
geometries.  An arbitrary $2$-dimensional  geometry 
can be considered as
$2$-dimensional  Riemannian surface---$(M,G)$.
$M$ is a surface, and $G$ defines scalar product of
tangent vectors, i.e. length of an arbitrary curve.
For arbitrary curve
  $\x=\x(t), t_1\leq t\leq t_2$
             $$
 \hbox {length of the curve}=
   \int_{t_1}^{t_2} \sqrt{\left(\v(t),\v(t)\right)}dt\,,\,
\hbox{scalar product}\,\,  
\left(\v(t),\v(t)\right)=G(\v(t),\v(t))\,, 
             $$
where $\v(t)$ is velocity vector. 
 In local coordinates $x^i$,
the curve has appearance  $\x(t)=x^i(t)$,
    $G=g_{ik}(x)dx^idx^k$,
  $\v(t)=v^i(x(t))\p_i={dx^i(t)\over dt}{\p\over \p x^i}$,
and 
the scalar product of velocity vector on itself 
is equal to
      $$
\left(\v,\v\right)=
 v^i(x(t))g_{ik}(x(t))v^k(x(t))=
 {d x^i(x(t))\over dt}
   g_{ik}(x(t))
 {d x^k(x(t))\over dt}\,,
      $$
i.e.
      $$
 \hbox {length of the curve}=
   \int_{t_1}^{t_2} \sqrt{\left(\v(t),\v(t)\right)}dt
      =
    \int_{t_1}^{t_2} \sqrt{
  {d x^i(x(t))\over dt}
   g_{ik}(x(t))
 {d x^k(x(t))\over dt} \,.   
             }dt
      $$
We say that $M$ is (uniformly) locally Eucldean
if

1) in a vicinity of arbitrary points there exist
 Eucldean coordinates, i.e. the coordinates  $u,v$
such that  $G=du^2+dv^2$ in these coordinates.


2) This  neighborhood is enough large:
there exists $r>0$ such that 
   all local cEucldiean coordinates  $u,v$ are defined in
the circle of radius $\geq r$.


{\bf Remark} 
  Surface is called locally Euclidean if the first condition
is obeyed. In the case if the second condition is
obeyed also, 
the surface is called {\it uniformly locally  Euclidean.}


\medskip

{\bf Exercise}  Show that the surface of sphere
is not locally Eucldean.


{\bf Exercise}  Show that domain $a<x<b$ of $\E^2$
is locally Eucldean but it is not uniformly 
locally Eucldean.

(We suppose that metric on the surface
$M$ in $\E^3$
is  the metric induced from $\E^3$.)

In this talk we will consider only uniformly locally 
Euclidean surface, and we will call them sometimes just
locally Eucldean.

   \section {Examples of locally Euclidean surfaces and 
subgroups of $\E(2)$.}

{\bf Exercise}  Consider surface of cylinder.  



\subsection{Subgroups of group $\E(2)$ and surfaces}

 
  Let $\G$ be an arbitrary subgroup
of group of isometries of $\E^2$.


Assign to the group $\G$ a
space $M_\G$ of orbits of $G$-group action
on $\E^2$,
$M_\G=\E^2\backslash \G$, i.e. 
           

 We denote the points of the space $M_\G$, 
by handwriting letters $\A,\B,C,\D,\dots$. These
points are orbits of group $\G$ action.
Every  point $A\in \E^2$ produces the point
$\A\in \E^2\backslash \G$, the equivalence class
of a point $A$ with respect to the gropu $\G$:
         $$
\A=[A]_\G\,,\qquad  g\in \G\,,
A'=g(A)\in [A]\,. 
        $$
To establish the geometry on $M$
we define the distance between points $\A,\B$
as the minimal distance between the orbit
   $\{A^g\}$ and $\{B^g\}$:
            $$
\hbox{if $\A=[A]$ and $\B=[B]$ then}\,\,
d(\A,\B)=\min_{g,g'\in\G}d(A^g,B^{g'})
            $$ 
{\bf Exercise 1}  Let $\ac\not=0$ be an arbitrary vector
in $\E^2$ and $\G={T_{n\ac}}$ be group of translations
generated by the translation on vector $\ac$:
                 $$
T_{n\ac}\colon \r\to \r+n\ac\,.
                 $$

  Describe the geometry $M_\G$
and show that this is geometry of cylinder.



\m

{\bf Exercise 2}  Let $\G=C_2$ be group of reflection
with respect to the line $l$.

  Describe the geometry $M_\G$
and show that this is not uniformly locally Euclidean manifold.

  



\subsection {Uniformly discontinuous subgroups of $\E(2)$}

We say that the subgroup $\G$ acts properly discontinuous
on $\E^2$ if there exists $\delta$ such that
for an arbitrary point $A\in \E^2$, and for an arbitrary
non-identity element $g\in \G$ 
        $$
      d(A,g(A))\geq \delta\,.
        $$
 In other words it means that the distance between 
distinct points of an arbitrary orbit exceeds the  $\delta$.

Why these groups are interesting?  Because every such group
defines locally Eucldean manifold. 



\begin{proposition}\label{1}

If action of group $\G$ has a fixed point
 (there exists $\r_0\in \E^2$ such that 
for all $g\in \G$,
$g(\r_0)=\r_0$) ({\it il faut dire mieux}),
then $M_\G$ is not uniformly locally 
Euclidean.
\end {proposition}
\begin{proof}
 If $A=\r_0$ is a fixed point, then
for arbitrary $g\not=1$  $d(A,A^g)\geq \delta$, and
on the other hand $d(A,A^g)=0$, if $\G$ 
acts uniformly discontinuou. Contradiction.
\end{proof}


\begin{proposition}\label{2}
If $\G\in \E(2)$  is uniformly discontinuous group, then
the group $\G$ is uniformly discontinuous.  

\end{proposition}

 {\tt Sketch of the proof}.

Let $\G$ be uniformly discontinuous:
         $$
\exists\delta\geq 0 \,,,
\hbox{such that for an arbitrary}\,\,
 A\in \E^2\,, g\in \G\,, d(A^g,A)<\delta \Rightarrow
   g=1\,.    
        $$

and let $B$ be an arbitrary point  
which belong to the disc $D_{\delta\over r3}(A)$
Consider orbits $\A$ and $\B$ of these points.
It is easy to see from triangle inequality that
for arbitrary points $A'\in\A$ and $B'\in \B$
the distance between these points is bigger or equal to
  $r={\delta\over 2}$. Indeed let $A"=A^g$, and 
$B'=B^h$. Denote $\tilde B=B^{hg^{-1}}$. Then
$d(A',B')=
d\left(A, \left(B^h\right)^{g^{-1}}\right)=
d\left(A, \tilde B\right)$, and by triangle inequality
     $$
d(A',B')=d(A,\tilde B)\geq \left|d(B,\tilde B)-d(A,B)\right|
>\delta
     $$
if $B\not=\tilde B$.


Thus we see that in the case if two points $A$
and $B$ are closer than $\delta\over 2$,
then the distance between orbits $\A$ and $\B$
is equal to the distance $d(A,B)$.
This proves that $M_\G$ is uniformly locally Euclidean
if $\G$ is uniformly discontinous.



the orbit of the point $B$ and the orbit of the point $A$
the  disc of radius


\subsection {Uniformly discontinuous subgroups of $\E(2)$}


  First of all recall the classic Theorem:
 
\begin{theorem} (Schalle+?)
  Any isometry of $\E^2$ is rotation, or translation or
  glided reflection.
\end{theorem}

  This Theorem possesses two statements. First that an arbitrary
(even non-linear map ) which is isometry has linear appearance
            $$
   F(\r)=A\r+\b\,,
            $$ 
where $A$ is linear operator, and the second statement
that this linear map is rotation (with respect to some centre)
or translation or glided reflection.

    We can prove the first statement under the assumption that
  $F(\r)$ is smooth map of $\E^2$ in $\E^2$.


     Suppose now  that $F(\r)=A(\r)+\b$.
   
\smallskip

  I-st case) orientation is preserved, i.e. $\det A=1$.
If $A=1$ then it is translation, if $A\not=1$ then operator
$A-1$ is invertible, and
              $$
\r=A(\r)+\b=\ac+\left(A\left(\r-\ac\right)\right)\,,
{\rm where}\,\,,  \ac= (1-A)^{-1}(\b)\,,
              $$
i.e. this affine transformation is a rotation around
the point $O-\ac$.

\smallskip

II-nd case) orientation is not preserved, i.e. $\det A=-1$.
  This operator has eigenvector  $\n$ 

One can see that we come to reflection with respect to the line
along the vector $\n$ and translation along $\n$, i.e. glided
reflection.
 


  Now using this Theorem classify 
   uniformly discontinuous subgroups of isometry group.

From no on we exclude the trivial case
when $G={e}$.


   Let $\G$ be a subgroup of $\E(2)$ which acts on
$\E^2$ uniformly discontinuous, i.e. there exists
$\delta>0$ such that for an arbitrary point $A\in \E^2$
and an arbitrary $g\in \G$
                  $$
                 d(A,A^g)\leq \delta \Rightarrow g=1\,.
                   $$
 It follows from proposition \ref{1} that the subgroup $\G$
contains only translations and non-trivial glided reflections.

  If group $\G=e$ this is trivial: $M_\G=\E^2$.

Denote by $\G_0$ the subgroup of $\G$
which preserve orientation, i.e. subgroup of translations.


\begin{proposition}\label{3}
The subgroup $\G_0$ of uniformly discontinuous
grroup $\G$
of orientation preserving transformations
is 

\begin{itemize}
\item the group of translations generated
  by arbitrary non-zero vector $\ac$
        $$
 \G_0=\G_0^{\ac}=\{T_{n\ac}\colon T_{n\ac}(\r)=\r+n\ac,\,,
{\rm where}\,\, n=0,\pm 1,\pm 2,\dots\}
        $$
\item the group of translations generated
  by arbitrary two non-zero linearly independent 
vectors $\ac,\b$
        $$
 \G_0=\G_0^{\ac,\b}=
\{T_{m\ac+n\b}\colon T_{m\ac+n\b}(\r)=
\r+m\ac+n\b,\,,
{\rm where}\,\, m,n=0,\pm 1,\pm 2,\dots\}
        $$
\end{itemize}


\end{proposition}

On the base of this proposition study the group $\G$.

\m
I-st case


Let   $\G_0=\{T_{n\ac}\}$. There are two possibilites.
First if $\G=\G_0=\{T_{n\ac}\}$,
and $M=\E^2\backslash \G$ is cylindre.

Now suppose  $\G\not=\G_0$,
i.e. $\G$ possesses glided reflections.
Let $S=S_{l,\b}\in \G$
(reflection with respect to the line 
$l$ directed along the vector
$\b$ and translation
on the vector $\b$ ). Then
$S^2=T_{2\b}$. Hence $\b={k\ac\over 2}$
for some integer $k$. This integer $k$ has to be odd,
since if $p=2p$ then the transformation $T_{-p\ac}S$
is the reflection, and it possesses fixed point
(see the porposition \ref{1}). 
We see that in this case $G$ is generated by translation
$T_\ac$ and glided reflection  $S_{l,{\ac\over 2}}$

In this case
$M=\E^2\backslash \G$ is twisted cylindre, 
(Mobius strip
with infinite sides)
\m

II-nd case


Let   $\G_0=\{T_{n\ac}\}$. Study again 
 two possibilites.

First:
$\G=\G_0=\{T_{n\ac}\}$. In this case

$M=\E^2\backslash \G$ is a torus.


Suppose now that $\G\not=\G_0$ and it possesses glided
reflections. One can see that this can happen
only if vectors $\ac$ and $\b$ are orthogonal to each
other. We will come to
$M=\E^2\backslash \G$ is a Klein bottle.


 \begin{theorem}  Let $\G$ be uniformly discontinuous
subgroup of isometries group. Then the following possibilities
may occur ({\it ca il faut dire mieux!!!})

\begin{itemize}

\item
I-st case (trivial)

$\G={e}$ has only identity element. Then $M=M\backslash \G=\E^2$ 

\item
II-nd case 


Group $\G=\{T_{m\ac}\}$
is generated by translation on vector $\ac$,
where $\ac$ is an arbitrary non-zero vector.
Then $M=M\backslash \G$ is cylindre 

\item
III-rd  case 


Group 
is generated by translation on vector $\ac$,
and glided reflection $S_{l,{\ac\over 2}}$,
where the line
$l$ goes along the vector $\ac$
Then $M=M\backslash \G$ is twisted cylindre 

\item
IV-th case 


Group $\G=\{T_{m\ac+\n\b}\}$
is generated by translation on vectors $\ac$
and $\b$,
where $\ac,\b$ are  arbitrary linearly independent vectors.
Then $M=M\backslash \G$ is a torus

\item
V-th case 


Group
is generated by translation on vectors $\ac$
and $\b$, and glided reflection $S_{l,{\ac\over 2}}$,
where the line
$l$ goes along the vector $\ac$
where $\ac,\b$ are  arbitrary non-zero vectors which are
{\it orthogonal!} to each other.
Then $M=M\backslash \G$ is a Klein bottle.

\end{itemize}
\end{theorem}

\section {The final classification}

We classified in the Theorem above all 
locally Euclidean surfaces which are generated
by uniformly discontinuous groups.

Is an arbitrary uniformly locally Euclidean surface
$M$ generated by uniformluy discontinuous group?
  Yes!


To prove this statement we consider the following covering

Take an arbitrary point $A\in \E^2$ and $\A\in M$,
and consider the map
   $$
  \begin{matrix}
        \E^2\cr
         p\downarrow\cr
          M\cr
     \end{matrix}
      $$
such that
  
\begin{itemize}

\item
$p(A)=\A$.


\item Let $B$ be an arbitrary point in $\E^2$.

If $d$


\end{itemize}
We come to 

\begin{theorem}  Let $M$ be an arbitrary 
uniformly locally Euclidean surface. Then there
exists uniformly discontinuous group
  $\G$ such that 
       $$
    M=M_\G
       $$
\end{theorem}

It follows from this Theorem and Proposition \ref{3}
the following corollary

\begin{corollary} Let  $M$
be an arbitrary uniformly locally Eucldean
surface. Then the following cases occur:

  \begin{itemize}
\item  $M$ is $\E^2$


\item $M$ is cylindre

\item $M$ is twisted cylindre

\item $M$ is torus

\item $M$ is Klein bottle
\item
  \end{itemize}
\end{corollary}


\section {Space of locally Eucldean geometries}

  Geometries on cylindres are similar to each other,
  the same about geometries of twisted cylindres and Klein 
bottles.

  Consider the space of geometries on tori.

Every lattice $T_{\ac,\b}$ defines
geometry $M_{\ac,b}=M\backslash {\{T_{m\ac+n\b}\}}$.

Two geometries coincide if lattices are isometric.

 We say that two geometries  $M_{\ac,b}$
and $M_{\ac',b'}$ are similar if 
$\ac'=\lambda\ac$ and $\b'=\lambda \b$.

\begin{proposition}
Two geometries $M_{\ac,\b}$ and $M_{\ac',\b'}$
are similar if and only if
            $$
  \begin{pmatrix}
      \ac'\cr
       \b'\cr
  \end{pmatrix}=  
     \begin{pmatrix}
      p & q\cr
        r & s\cr
  \end{pmatrix}
     \begin{pmatrix}
      \ac\cr
       \b\cr
  \end{pmatrix}
            $$
such that
      $$
    ps-qr=\pm 1
      $$
and
       $$
p,q,r,s \quad \hbox {are integers}
       $$
in other words if these lattices are related by
unimodular transformation in integers.


\end{proposition}


Now identify all similar tori and tori which are related
with each other by unimodular transfromation
  (arbitrary?)

Find distance between geometries.

Let $T_{\ac,\b}$ be a lattice.

Assign to this lattice the  complex number
         $$
z={a_x+ia_y\over b_x+ib_y}
         $$
in the case if $\ac\times \b$ is poisitve, and
the inverse complex number in the case if
  $\ac\times \b$ is negative.

We see that a class of similar geometries
define the point in upper-half plane.
 
One can say that set of points on the upper half plane
is the set of all geometries on tori.

Define distance on this set!

Recall that under transformation
transformation
     $$
{\ac\over \b}={a_x+ia_y\over b_x+ib_y}\leftrightarrow
  {p\ac+q\b\over r\ac+s\b}
       $$
geometry is no changed if $ps-rq=\pm 1$, and $p,q,r,s$
are integeres.  Now we find distance function which is
invariant with respect to the action of this group.
Here we consider instead integeres arbitrary real numbers.

 In terms of complex coordinates $z={a_x+ia_y\over b_x+ib_y}$
we come to transformations
       \begin{equation}\label{group}
 w={az+b\over cz+d}\,,\quad {\rm with}\,\,
   ad-bc=1\,, 
 w={a\bar z+b\over c\bar z+d}\,,\quad {\rm with}\,\,
   ad-bc=-1\,, 
       \end{equation}
This is so called extended $SL(2,\R)$ group

Find a geometry  which is invariant with respect to this action.

First find geodesics, "straight lines" of this geometry.


Notice that reflection with respect
to vertical axis, and inversion with a 
centre at real axis,  $z=a$
($a$ is real) are symmetry transformation:

         $$
w=2a-\bar z\,,\qquad  
w=a+{1\over \bar z-a}
         $$

The set of fixed points of this transformations are
vertical lines and half-circles with centre at real axis.
These are geodesics due to the following 

\begin{lemma}
Let $C$ be a locus of fixed points of isometry.
If it is a curve, then this curve is geodesic.
\end{lemma}

Consider points on the vertical line $z=it$,
and define the distance between these points.

Transformation $z\to\lambda z$, and 
$z\to -{\lambda\over z}$ are symmetry transfromations,
hence for arbitrary points $z=ia,ib$
and for arbitrary real $\l$
   $$
d\left(ia,ib\right)=
d\left(i\lambda,i\lambda b\right)=
d\left(i{\lambda\over a},i{\lambda\over b}\right)=
   $$
Moreover since vertical line is geodesic hence
for arbitrary three points $ia,ib,ic$
      $$
   d(ia,ic)=d(ia,ib)+d(ib,ic)\,,\qquad
{\rm if}\,\, a<b<c
      $$
It follows from these equations that 
$d(ia,ib)=\log {a\over b}$

\begin{theorem}
Riemannian metric which is invariant with resepct to the
action of the group \eqref{group} is the Lobachevsky metric
which up to multiplier is defined
by equation
         $$
G={dz d\bar z\over (z-\bar z)(\bar z-z)}
       $$

\end{theorem}


\end{document}

\bye
