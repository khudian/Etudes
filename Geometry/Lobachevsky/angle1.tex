

\magnification=1200 


\baselineskip=14pt
\def\vare {\varepsilon}
\def\A {{\bf A}}
\def\t {\tilde}
\def\a {\alpha}
\def\K {{\bf K}}
\def\N {{\bf N}}
\def\w {\omega}
\def\s {{\sigma}}
\def\S {{\Sigma}}
\def\s {{\sigma}}
\def\p{\partial}
\def\vare{{\varepsilon}}
\def\Q {{\bf Q}}
\def\D {{\cal D}}
\def\G {{\Gamma}}
\def\C {{\bf C}}
\def\L {{\cal L}}
\def\Z {{\bf Z}}
\def\U  {{\cal U}}
\def\H {{\cal H}}
\def\R  {{\bf R}}
\def\S  {{\bf S}}
\def\E  {{\bf E}}
\def\l {\lambda}
\def\degree {{\bf {\rm degree}\,\,}}
\def \finish {${\,\,\vrule height1mm depth2mm width 8pt}$}
\def \m {\medskip}
\def\p {\partial}
\def\r {{\bf r}}
\def\pt {{\bf pt}}
\def\v {{\bf v}}
\def\n {{\bf n}}
\def\t {{\bf t}}
\def\b {{\bf b}}
\def\c {{\bf c }}
\def\e{{\bf e}}
\def\ac {{\bf a}}
\def \X   {{\bf X}}
\def \Y   {{\bf Y}}
\def \x   {{\bf x}}
\def \y   {{\bf y}}
\def \G{{\cal G}}
\def\ss  {\sigma_{\rm sph}}
\def\grad {{\rm grad\,}}
% I began this file on 21 July 2017

    \centerline {\bf Angle function}


 { In this etude we will speak about one function
which can be defined in a very simple way, and it has relations
with such  beautiful problem of mathematics as  Dirichle problem for
domain, conformal  map. e.t.c}




  Let $C\colon \r=\r(t)$ be a curve in $\E^2$. 
Consider the function
            $$
W(\R)=\hbox{the angle at which one sees the curve
  $C$ from the point $\R$}
         \eqno (1.1)
            $$

What can we say about this function?

%  {\it
1) This function is harmonic function:
     $$ 
    \Delta W=
{\p^2 W(x,y)\over \p x^2}+
{\p^2 W(x,y)\over \p y^2}=0 \,,\quad  \hbox
{for all the points $\r\not\in C $}
         $$

($x,y$ are standard Cartesian coordinates in $\E^2$)

on the surface $C$ this function has jump of values

{\bf Example 1}
1. Let $AB$ be a segment 
of straight line between points
$A=(a,b)$
and $B=(1,1)$ on the axis $Ox=X$,
$A=(a,0)$, $B=(b,0)$.

Consider an arbitrary point $P=(X,Y)$ on $\E^2$.
 Let $N=(0,X)$ be a projection of the point $P$
on the axis $OX$. Then it 
is obvious that
       $$
W(P)=W(X,Y)=\angle PAN-\angle PBN=
  {\rm arctan\,}{X-a\over Y}-
  {\rm arctan\,}{X-b\over Y}\,.
       \eqno (1.Ex1.1)
       $$

{\bf Example 2}  Let $u(\varphi)$,
$\nu(\varphi)$ be functions
on unit circle $x^2+y^2=1$ then
              $$
  W(r,\theta)=\int_0^{2\pi}{1-r\cos(\theta-\varphi)\over 
      1+r^2-2r\cos(\theta-\varphi)}
      \nu(\varphi)d\varphi=
 {1\over 2\pi}\int_0^{2\pi}{1-r^2\over 
      1+r^2-2r\cos(\theta-\varphi)}
      f(\varphi)d\varphi
              $$
is a harmonic function in circle with ``wall'' on the circle.
the jump is equal to $\pi \nu(\varphi)$.


\m

{\bf Example 3} Let $u(x)$ be a function on axis $OX$. Then
                $$
   W(x,y)=\int {yu(t)\over (x-t)^2+y^2}dt
                $$
is harmonic function with wall on the axis $OX$

\m

One can easy to generalise this example considering
 instead $C$ hypersurface pf dimension $n-1$ in $\E^n$,
and instead angle cohomology  (see later)
For example for (1.1)
              $$
W(X,Y)=\int F^*_\R(xdy-ydx)=\int {(x-X)dy-(y-Y)dx\over (x-X)^2+(y-Y)^2}
                   $$
for the case $n>1$ one have to consider instead $1$-form -cohomology
    $xdy-ydx\over x^2+y^2$. the $n-1$-form
              $$
\w\colon\,   dx^1\wedge\dots\wedge dx^n=r^{n-1}dr \wedge \w
              $$    


3)This function is related wih the potential of double layer
  (see later)


4) This function is related with Dirichle problem:
 find a function $W$ which is harmonic in a domain $U$
  if its value on the boundary $\p U$ is equal to 
the given function


5)  Conformal mapping: Let $U$ be a function which is conjugate
to the function $W$, i.e.
               $$
        F(x,y)=W+iU
               $$
is meromorphic function

%}



\bye
