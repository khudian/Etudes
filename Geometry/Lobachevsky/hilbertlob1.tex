

\magnification=1200 


\baselineskip=14pt
\def\vare {\varepsilon}
\def\A {{\bf A}}
\def\t {\tilde}
\def\a {\alpha}
\def\K {{\bf K}}
\def\N {{\bf N}}
\def\V {{\cal V}}
\def\s {{\sigma}}
\def\S {{\Sigma}}
\def\s {{\sigma}}
\def\p{\partial}
\def\vare{{\varepsilon}}
\def\Q {{\bf Q}}
\def\D {{\cal D}}
\def\G {{\Gamma}}
\def\C {{\bf C}}
\def\M {{\cal M}}
\def\Z {{\bf Z}}
\def\U  {{\cal U}}
\def\H {{\cal H}}
\def\R  {{\bf R}}
\def\S  {{\bf S}}
\def\E  {{\bf E}}
\def\l {\lambda}
\def\ll {{\bf l}}
\def\degree {{\bf {\rm degree}\,\,}}
\def \finish {${\,\,\vrule height1mm depth2mm width 8pt}$}
\def \m {\medskip}
\def\p {\partial}
\def\r {{\bf r}}
\def\pt {{\bf p}}
\def\v {{\bf v}}
\def\n {{\bf n}}
\def\t {{\bf t}}
\def\b {{\bf b}}
\def\c {{\bf c }}
\def\e{{\bf e}}
\def\ac {{\bf a}}
\def \X   {{\bf X}}
\def \Y   {{\bf Y}}
\def \x   {{\bf x}}
\def \y   {{\bf y}}
\def \G{{\cal G}}

% I began this file on 12 August 2018


\centerline {\bf Hilbert Theorem of non-embedding of hyperbolic space, sine-Gorodon, e.t.c.}

{\it  Here I try to give some ideas of proof of the theorem, that Lobachevsky plane as a whole
cannot be immersed in $\E^3$.   I knew  this statement long time, 
  considering it as very important statement, which just has to be known.

I did not realise  how much it is related with such ``working''  questions
as assymptotic directions, Chebyshev net  and ...  sine-Gordon equation:
                 $$
      {\p^2F\over \p x \p y}=\sin F
         \eqno (01)
                 $$

}



   Suppose that $M\colon\quad \r=\r(u,v)$ is immersion of 
Lobachevsky plane in $\E^3$.
     Let  $\Pi(\x,\y)$  be second quadratic form.  
Since Gaussian curvture is equal to $-1$, hence in arbitrary point
  $\pt \in M$ there exist two directions $\ll_1, \ll_2$  such that
     the second quadratic form vanishes on every vector which is in one
of these directions.

    These directions define a net at least locally.
We may choose local coordinates $u,v$  such that second quadratic form
is equal to 
               $$
   \Pi= \Pi(u,v)dudv
           \eqno (1.a)
               $$            
in these local coordinates.

   {\bf Proposition}  One may choose local coordinates $u,v$
such that second quadratic form has appearance (1a)  and
the first quadratic form has appearnace
                    $$
         G=du^2+2\cos \Theta (u,v)dudv+dv^2\,.
                         \eqno (1b)
                    $$


To prove it consider basic coordinate vectors $\e_\a={\p \r\over \p u^\a}$.

We have  that
                    $$
     {\p \e_\pi\over \ \p u^a}=
(\p_\a \e_\pi)_{||}+
(\p_\a \e_\pi)_{\perp}=
\nabla_\a \e_\pi+\Pi (\e_\pi, \e_\a)\n=
\Gamma_{\a\pi}^\rho\e_\rho+\Pi_{\a\pi}\n\,.
                     $$      
(Here as usual $\n$ is normal unit vector).

We have that
                     $$
     {\p^2 \e_\pi\over \ \p u^a\p u^\beta}=
              {\p\over \p u^\beta}
         \left(
\Gamma_{\a\pi}^\rho\e_\rho+\Pi_{\a\pi}\n
        \right)=
    \p_\beta\Gamma_{\a\pi}^\rho\e_\rho+
    \Gamma_{\a\pi}^\rho\p_\beta\e_\rho+
      \p_\beta\Pi_{\a\pi}\n+\Pi_{\a\pi}\p_\beta \n=
                  $$
                  $$
    \p_\beta\Gamma_{\a\pi}^\rho\e_\rho+
         \Gamma_{\a\pi}^\rho
                \left(
\Gamma_{\beta\rho}^\sigma\e_\sigma+\Pi_{\beta\rho}\n
                  \right)+
 \p_\beta\Pi_{\a\pi}\n-\Pi_{\a\pi}\Pi_\beta^\rho\e_\rho\,.
                     $$
(We use equation  for  Shape operator:  $\p_\beta\n=-S_\beta^\rho\e_\rho$,
$\Pi_\beta^\rho=S_\beta^\rho=\Pi_{\beta\a}g^{\a\rho}$. )


We have that 
               $$
     {\p^2 \e_\pi\over  \p u^a\p u^\beta}=
     {\p^2 \e_\pi\over  \p u^\beta\p u^\alpha}=
                      $$
(so called Peterson-Codazzi integrability conditions)

give as that
           $$
     \p_\beta\Gamma_{\a\pi}^\rho\e_\rho+
         \Gamma_{\a\pi}^\rho
                \left(
\Gamma_{\beta\rho}^\sigma\e_\sigma+\Pi_{\beta\rho}\n
                  \right)+
 \p_\beta\Pi_{\a\pi}\n-\Pi_{\a\pi}\Pi_\beta^\rho\e_\rho
             =
                         $$
                         $$
      \p_\a\Gamma_{\beta\pi}^\rho\e_\rho+
         \Gamma_{\beta\pi}^\rho
                \left(
\Gamma_{\alpha\rho}^\sigma\e_\sigma+\Pi_{\alpha\rho}\n
                  \right)+
 \p_\alpha\Pi_{\beta\pi}\n-\Pi_{\beta\pi}\Pi_\alpha^\rho\e_\rho
           $$
Comparing the terms at $\e_\rho$ and $\n$ we come to:

             $$
   \p_\beta\Gamma_{\a     \pi}^\rho
   -\p_\a   \Gamma_{\beta \pi}^\rho
      +\Gamma_{\beta \sigma}^\rho\Gamma_{\a\pi}^\sigma
      -\Gamma_{\a \sigma}^\rho\Gamma_{\beta\pi}^\sigma
             =
             \Pi_\beta^\rho\Pi_{\a\pi}-
             \Pi_\a^\rho\Pi_{\beta\pi}\,,
             $$
i.e.
                     $$
                 R^\rho_{\, \pi\beta\a}=
            \Pi_\beta^\rho\Pi_{\a\pi}-
             \Pi_\a^\rho\Pi_{\beta\pi}\,,\,
            (\hbox {Gau\'ss conditions})
                     $$
and
                 $$
\p_\a\Pi_{\beta\pi}-\Gamma_{\a\pi}^\rho\Pi_{\rho\beta}=
\p_\beta\Pi_{\a\pi}-\Gamma_{\beta\pi}^\rho\Pi_{\rho\a}\,,
  \hbox {(Peterson Kodazzi conditions)}
                 $$ 
Taking traces of Gauss conditions we come to
the stadnard realtion between Gaussian and scalar curvature: 
                      $$
    R=R^\beta_{\, \pi\beta\a}g^{\pi\a}=
 \left[{\rm Tr\,}\Pi\right]^2-
 {\rm Tr\,}\Pi^2=2\det S=2K=2{\det \Pi\over \det g}
                      $$
(Here we are abusing little bit lower and upper indices, e.g.
   $S^\a_\rho=-\Pi_{\a\beta}g^{\beta\rho}$).

Consider Peterson Codazzi relations:

 for indices    $\a\beta\pi=1\, 1\, 1$ empty conditions 
(as well as for$\a\beta\pi=2\, 2\, 2$)
 
for indices    $\a\beta\pi=1\, 1\, 2$ empty conditions 
(as well as for$\a\beta\pi=2\, 2\, 1$) 


for indices    $\a\beta\pi=1\, 2\, 2$ we have:
             $$
\p_1 P=P(\Gamma^1_{11}-\Gamma^2_{21})
             $$

as well as for indices    $\a\beta\pi=2\, 1\, 1$ we have:
             $$
\p_2 P=P(\Gamma^2_{22}-\Gamma^1_{12})
             $$

Here   $\Pi= \pmatrix{0 & P\cr P & 0}$.


  These conditions do not look nice, but we are in dimension 2.


  Note that Gaussian curvature is constant. Choose it $1$:
   We see that
              $$
 \det \Pi=-P^2\cdot \det g=-1\Rightarrow   P=\sqrt {\det g}\,.
              $$ 

Note that 
           $$
  \delta \sqrt {\det g}={1\over 2}\sqrt {\det g} g^{ik}\delta g_{ik}\,,
           $$
 i.e.
                  $$
 {\p P\over \p u^\a}=\sqrt {\det g}{\p \sqrt {\det g}\over \p u^\a}=
  P\Gamma_{\a\pi}^\pi\,.
                  $$
We see that Petrson Codazzi conditions mean that
              $$
  \Gamma_{11}^1+\Gamma_{12}^2=
  \Gamma_{11}^1-\Gamma_{12}^2\Rightarrow\Gamma_{12}^2=0\,,\quad
  \Gamma_{21}^1+\Gamma_{22}^2=
  \Gamma_{22}^2-\Gamma_{12}^2\Rightarrow\Gamma_{12}^1=0\,,\quad
            $$
Thus we see that Peterson-Codazzi conditions with constant curvature
imply that
           $$
      \Gamma_{12}^1=\Gamma_{12}^2=0\Rightarrow
      \Gamma_{12;1}=\Gamma_{12;2}=0\Rightarrow
     {\p g_{11}\over \p u^2}=
     {\p g_{22}\over \p u^1}=0\,.
           $$

We see that first quadratic form is
              $$
G=A(u)du^2+2B(u,v)dudv+C(v)dv^2
               $$
Taking antiderivatives 
 $\tilde u=\int \sqrt {A(u)}du$
and 
 $\tilde v=\int \sqrt {C(v)}dv$

We come to coordinates $u=\tilde u,v=\tilde v$ such that in these coordinates
                $$
G=du^2+2\cos\Theta(u,v)dudv+dv^2
                $$
and second quadratic form in these coordinates
is equal to
                  $$
\Pi=2\sin\Theta (u,v)dudv 
                  $$

(On this metric see the blog in August 2018)


\bye


