

\magnification=1200 


\baselineskip=14pt
\def\vare {\varepsilon}
\def\A {{\bf A}}
\def\t {\tilde}
\def\a {\alpha}
\def\K {{\bf K}}
\def\N {{\bf N}}
\def\w {\omega}
\def\s {{\sigma}}
\def\S {{\Sigma}}
\def\s {{\sigma}}
\def\p{\partial}
\def\vare{{\varepsilon}}
\def\Q {{\bf Q}}
\def\D {{\cal D}}
\def\G {{\Gamma}}
\def\C {{\bf C}}
\def\L {{\cal L}}
\def\Z {{\bf Z}}
\def\U  {{\cal U}}
\def\H {{\bf H}}
\def\R  {{\bf R}}
\def\S  {{\bf S}}
\def\E  {{\bf E}}
\def\l {\lambda}
\def\degree {{\bf {\rm degree}\,\,}}
\def \finish {${\,\,\vrule height1mm depth2mm width 8pt}$}
\def \m {\medskip}
\def\p {\partial}
\def\r {{\bf r}}
\def\pt {{\bf pt}}
\def\v {{\bf v}}
\def\n {{\bf n}}
\def\t {{\bf t}}
\def\b {{\bf b}}
\def\c {{\bf c }}
\def\e{{\bf e}}
\def\ac {{\bf a}}
\def \X   {{\bf X}}
\def \Y   {{\bf Y}}
\def \x   {{\bf x}}
\def \y   {{\bf y}}
\def \G{{\cal G}}
\def\ss  {\sigma_{\rm sph}}
\def\grad {{\rm grad\,}}
% I began this file on 31 July 2017





\centerline  {\bf  30 July 2017}


Учу гипербол. геом.
Очень полезен Лаврентьев Шабат.

   Долгое время не понимал:
 Lioville Theorem prohibits conformal bijection
of plane on the disc, whilst there is no problem
of conformal bikjection half-plane to disc

********************************************************

\bigskip

\centerline  {\bf  31 July 2017}

      Study conformal mapings, Dirichle problem (Lavrentiev book)
      
            Изучаю.


       Метод сглаживания:

   Let $C$ be a curve with pieces (arcs) $C_\a$.
  (e.g. polygon)
   
  Let $f=f(t)$ be a function on $C$, 
which has jumps.
Let $D_i$ be internal points
of arcs $C_\a$, where $f$ has a jump.
Let $A_i$ be vertices $A_\a$: the arc $C_\a$ 
goes from the vertex $A_\a$
to the vertex $A_{\a+1}$.
  Note that at vertices we have a jump of 
angles: at the vertex $A_\a$ 
the jump $$
\delta\varphi_\a=
\varphi_\a^{(+)}-
\varphi_\a^{(-)}\,.
    $$
 
      Let  $F(z)$ be function with jumpes $h_k$
at the points $D_k$, where curve is smooth and
  with jumps $H_\a$ at the vertices $A_\a$

Consider (with Lavrentiev-Shabad) the new 
function
          $$
\tilde F(z)=F(z)+
{1\over \pi}\sum_k\arg (z-D_k)-
  \sum_\a {H_\a\over \delta\varphi_a}
\arg(z-A_\a )
          $$
This function has no jumps!!!



********************************************
\bigskip

\centerline  {\bf  1 August 2017}

How look linear factional maps which send
 disc $x^2+y^2=1$ onto itself:
this is representation of the group
  $SL(2,R)/??$ in Poincare disc.

(Linear fractional bijections of upper-half plane onto itself
are $SL(2,R)/Z_2$?)


  If $$
 w={Az+B\over Cz+D}
        $$ 
and $|w|=1$ if $|z|=1$ , then
             $$
z\bar z=1\to w\bar w=1, i.e. 
  (Az+B)(\bar A\bar z+\bar B)=
  (Cz+D)(\bar C\bar z+\bar D)
             $$
we come to
      $$
|A|^2+|B|^2+(A\bar B z+c.c.)=
|C|^2+|D|^2+(C\bar D z+c.c.)\,.
      $$
We see that for $g=\pmatrix{A & B\cr C & D}$,
         $$
 |A|^2+|B|^2=|C|^2+|D|^2\,,\quad
 {\rm and}\,,\quad 
 A\bar B-C\bar D=0
    \eqno (*)
    $$
This means that  $g$  up to dilation 
preserves hermitian
scalar product with signature $(1.1)$,
we come to the group $SU(1,1)/\{1.-1\}$.

Indeed
  Let
    $g=\pmatrix{A & B\cr C & D\cr}$ 
Then conditions (*)  means that
         $$
g^{+}\cdot 
\pmatrix{1 & 0\cr 0 & -1\cr}\cdot
g=
 \pmatrix{\bar A & \bar C\cr 
 \bar B & \bar D\cr}
\pmatrix{1 & 0\cr 0 & -1\cr}
\pmatrix{A & B\cr C & D\cr}
=
   \lambda
 \pmatrix{1 & 0\cr 0 & -1\cr}\,
         $$
i.e. removing scaling factor $\l$
and putting $\det g=1$ we come to 
               $$
    g^{+}=
    \pmatrix{\bar A & \bar C\cr 
        \bar B & \bar D}=
  \pmatrix{1 & 0\cr 0 & -1\cr}
 \pmatrix{A & B\cr C & D\cr}^{-1}
     \pmatrix{1 & 0\cr 0 & -1\cr}
         =
  \pmatrix{1 & 0\cr 0 & -1\cr}
 \pmatrix{D & -B\cr -C & A\cr}
\pmatrix{1 & 0\cr 0 & -1\cr}=
 \pmatrix{D & B\cr C & A\cr}\,,
           $$
i.e. $D=\bar A, C=\bar B$
  Thus  symmetries of Poincare disc
is $SU(1,1)/\{1,-1\}$, where
      $$
SU(1.1)=\left\{  g=
\pmatrix{A & B\cr \bar B & \bar A\cr}\colon
      |A|^2-|B|^2=1
          \right\}\,.
      $$

How look linear factional maps which send
 disc $x^2+y^2=1$ onto itself:
this is representation of the group
  $SL(2,R)/??$ in Poincare disc.

(Linear fractional bijections of upper-half plane onto itself
are $SL(2,R)/Z_2$?)


  If $$
 w={Az+B\over Cz+D}
        $$ 
and $|w|=1$ if $|z|=1$ , then
             $$
z\bar z=1\to w\bar w=1, i.e. 
  (Az+B)(\bar A\bar z+\bar B)=
  (Cz+D)(\bar C\bar z+\bar D)
             $$
we come to
      $$
|A|^2+|B|^2+(A\bar B z+c.c.)=
|C|^2+|D|^2+(C\bar D z+c.c.)\,.
      $$
We see that for $g=\pmatrix{A & B\cr C & D}$,
         $$
 |A|^2+|B|^2=|C|^2+|D|^2\,,\quad
 {\rm and}\,,\quad 
 A\bar B-C\bar D=0
    \eqno (*)
    $$
This means that  $g$  up to dilation 
preserves hermitian
scalar product with signature $(1.1)$,
we come to the group $SU(1,1)/\{1.-1\}$.

Indeed
  Let
    $g=\pmatrix{A & B\cr C & D\cr}$ 
Then conditions (*)  means that
         $$
g^{+}\cdot 
\pmatrix{1 & 0\cr 0 & -1\cr}\cdot
g=
 \pmatrix{\bar A & \bar C\cr 
 \bar B & \bar D\cr}
\pmatrix{1 & 0\cr 0 & -1\cr}
\pmatrix{A & B\cr C & D\cr}
=
   \lambda
 \pmatrix{1 & 0\cr 0 & -1\cr}\,
         $$
i.e. removing scaling factor $\l$
and putting $\det g=1$ we come to 
               $$
    g^{+}=
    \pmatrix{\bar A & \bar C\cr 
        \bar B & \bar D}=
  \pmatrix{1 & 0\cr 0 & -1\cr}
 \pmatrix{A & B\cr C & D\cr}^{-1}
     \pmatrix{1 & 0\cr 0 & -1\cr}
         =
  \pmatrix{1 & 0\cr 0 & -1\cr}
 \pmatrix{D & -B\cr -C & A\cr}
\pmatrix{1 & 0\cr 0 & -1\cr}=
 \pmatrix{D & B\cr C & A\cr}\,,
           $$
i.e. $D=\bar A, C=\bar B$
  Thus  symmetries of Poincare disc
is $SU(1,1)/\{1,-1\}$, where
      $$
SU(1.1)=\left\{  g=
\pmatrix{A & B\cr \bar B & \bar A\cr}\colon
      |A|^2-|B|^2=1
          \right\}\,.
      $$

********************************************


\bigskip

\centerline  {\bf  2 August 2017}




\def\w {\omega}

 
Let $\xi$ be a point in the unit disc,
then the trasformation
     $$
w=w(z)={z=\xi\over 1-\bar \xi z }
     $$
trasforms disc onto disc and the point $\xi$
on the centre.

  Mы знали какбы с детства (инверсия),  
но все таки приятно:


********************************************


\bigskip

\centerline  {\bf  3 August 2017}


Let $u=u(x,y)$ be harmonic function
in the disc,
and let $z_0$ be its singulr point.

Consider conjugaate function $v$:
       $$
v_x=-u_y\,,v_y=-u_x
       $$
it is antiderivative of $1$-form
$dv=u_ydx-u_xdy$. It defines
multivalued function
    $
V=\int (u_ydx-u_xdy)
    $
whic is defined up to a period
     $$
\Pi=\int_{C}(u_ydx-u_xdy)\,.
  $$
  We come to multivalued holomorphic function
$u+iv$ with period $\Pi$. On the other hand 
the function
     $$
{\Pi\over 2\pi}Log (z-z_0)
     $$
    is multivalued, with the same period.
Hence the function:
        $$
F(z)=u(z)+iv(z)-
{\Pi\over 2\pi}\log (z-z_0)
        $$
has disconnected leaves.  This is multivalued
function $f(z)\to f(z)+i\Pi$. Consider the function
        $$
G(z)=\exp\left(-{2\pi F(z)\over \Pi}\right)=
\exp\left( u(z)+iv(z)-
{\Pi\over 2\pi}\log (z-z_0)\right)=
 (z-z_0)\exp\left(-{2\pi F(z)\over \Pi}\right)
        $$
This is ONE-VALUED! holomorphic function:
        $$
   f(z)\to f(z)+i\Pi\,, G(z)\to 
   \exp{{2\pi\over \Pi}\left(F(z)+i\Pi\right)}=G(z)\,.
         $$
(here ther are slight inconveniences.....)

 
  This is holomorphic function in the disc.
Consider function
           $$
    G
           $$
   Почти элементарно, но так приятно,
и все это есть в Лаврентьеве!


********************************************


\bigskip

\centerline  {\bf  4 August 2017}
     Why the function multivalued function
$\arctan {y\over x}$
appeared often in calculations?

Because this function is conjugate to
$\log \sqrt {x^2+y^2}$:
           $$
\log z=\log (x+iy)\log \sqrt {x^2+y^2}
+\arctan {y\over x}
           $$



********************************************



\bigskip

\centerline  {\bf 5 August 2017}
     Why the function multivalued function
$\arctan {y\over x}$
appeared often in calculations?

Because this function is conjugate to
$\log \sqrt {x^2+y^2}$:
           $$
\log z=\log (x+iy)\log \sqrt {x^2+y^2}
+\arctan {y\over x}
           $$



********************************************



\bigskip

\centerline  {\bf   6 August 2017}


I know well angle function and its relation
with Green function:


In fact Green function generalises this concept.

   recall sketchly :
 Let $G_U==G(z_0,z)$  be a Green  function of 
the domain $U$, i.e. the function such that
 
1) $G\approx \log(z-z_0)$ in a vicnity of the point
$z_0$,

2) it is harmonic elswhere in $U$ except 
a point $z_0$

3) it vanishes at boundary $\p U$
  

    Using the identity:
        $$
\int_U  u\Delta v-\int_U v\Delta u=
\int_{\p U} u *dv-\int_{\p V} v * du=
     \eqno (identity)
        $$
we solve the boundary problem
   $$
\cases 
{\Delta F=f\cr 
F\big\vert_{\p U}=\mu\cr
}\,,\qquad
      F=\int_U G\circ f+\int_{\p U}*dG \mu
$$
where $\circ$ is convolution.
(this is called Dirichle problem 
if $f\equiv 0$.)


On the other hand one can come
to the Green function
$G_U$  trhough

  1)   Green function $G_\infty(z_0,z)=
-{1\over 2\pi}\log(z-z_0)$

2) and the ``angle function''
  defined on the boundary
solving (with Arsenin) the integral equation.
  Recall that angle function
defines for every curve the function
on plane 
which is equal to the angle
that we look at  this curve
\footnote{$^*$}{another name: 
double layer potential}.

{\bf Remark}  {\it 
We wrote identity (ident.)
in a way to emphasize as much as possible
the relation between `angle' function and
  Green function.
   This identity is written in
 standard way as following:
     $$
\int_U  u\Delta v-\int_U v\Delta u=
\oint_{\p U} u{\p v\over \p n}d{\bf S} -
 \oint_{\p V} v {\p u\over \p n}d{\bf S}=
     \eqno (identity')
        $$
(see any book).
In fact for every vector field $\A$
      $$
\underbrace
\int_C \A d{\bf s}_{\hbox{flux of $\A$ trhough the 
surface $C$}}=\int \Omega {\cal c}\A\,,
      $$
where $\Omega$ is volume $2$-form, and
      $$
{\p f\over \p n}={\grad}f
      $$
and $*$ is Hodge operation:
       $$
   * df=\Omega {\cal c}\grad f
       $$
Notice that for harmonic function $u$,
$d^{-1}(*du)$ is conjugate function:
      $$
\Delta u(x,y)=0
 \Leftrightarrow 
 u(x,y)+id^{-1}(*d u(x,y))=F(z)\,\,
 \hbox{is holomorphoc function}
         $$
, e.g.
        $$
  d^{-1}(*d\log\sqrt{x^2+y^2})={\rm arctan}{y\over x}\,,
    \log {x^2+y^2}+i{\rm arctan}{y\over x}=\log (x+iy)
        $$
it is how angle function relates with 
normal derivative of Green function.
}


\bigskip

 Thus on the space $C(\p U)$ 
of functions on boundary we 
have to linear operators:

  First operator:
    $$
  C(\p U)\ni \nu\mapsto   \mu\in C(\p U)\colon
    \quad
     s(t)=\int_{\p U} L(t,t')\nu(t')dt'
    $$
  Second operator
          $$
  C(\p U)\ni \nu\mapsto 
W\in C(U)\colon\quad
 W(\r)=\int_U L(\r,t')\nu(t')dt'      
          $$
On the other hand we know that
  for the angle function $L$
the difference of values of these operators
on the boundary $\p U$ is equal to 
$\pi \nu(t)$:
         $$
  C(\p U)\ni \nu\mapsto 
\left(\lim_{\r\to t}
int_U L(\r,t')\nu(t')dt'\right)-
 \int_{\p U} L(t,t')\nu(t')dt'=\pi\nu(t)
         $$
Thus to solve the Dirichle problem,
i.e. to reconstruct harmonic function $W$
by its value $\nu$ at the 
boundary $\p U_{0_-}$
we first solving integral equation
find a function  $\nu$
such that 
        $$
\pi\nu+\int_{\p U} L(t,t')\nu(t')dt'=\mu(t)
          $$
 then we reconstruct $W$ in terms of $\nu$.



$$ $$

 {\bf Example}


  Tro to recontruct
    harmonic function
$W$ in the disc $x^2+y^2<1$
using different methods.

First: Green function:
       $$
W=\int_{x^2+y^2=1} *d G
       $$
but here we need to know $G=G(z_0,0)$
 We already know the conformal map
    whoch transforms circle to circle,
and we know the Green fucntion for  $z_0=0$,
hence
     $$
G(z_0,z)=-{1\over 2\pi}\log
            \left|
   {z-z_0\over 1-\bar {z_0}z}
            \right|
     $$

Another way to calculate:
 



********************************************


\bigskip

\centerline  {\bf 7 August  2017}



\centerline {\bf Calculation for Green 
function of disc and half-plane.}



Recall that if $G_U(z_0,z)$ is Green function
of domain $U$ then 
the identity
      $$
  \int_U \left(u\Delta v-v\Delta v\right)=
\int_{\p U}\left(u *dv-v *du\right) 
 \eqno (0.1)
      $$
implies that
         the function
      $$
u(\zeta)=\G_U(\zeta,z)f(z)\Omega+
\int_{\p U} *d_{(z)}G(\zeta,z)\mu\,,\quad
   (\Omega \hbox{is area form on $z$-plane})
 \eqno (0.2)
      $$
is the solution of boundary problem
       $$
\cases{
   \Delta u=f\cr u\big\vert_{\p U}=\mu\cr
         }
   \eqno (0.3)
       $$

\m

   \centerline {\bf Green function for disc}

     Having a map
         $$
z\to w={z-\zeta\over 1-\bar\zeta z }
         $$
which maps unic disc onto unit disc and 
boundary on boundary, and
the point $\zeta$ on the centre we define
               $$
    G(z_0,z)=-{1\over 2\pi}\log
            \left|
   {z-z_0\over 1-\bar {z_0}z}
            \right|
               $$ 
  One can see that
              $$
  *dG_{z}=\Omega{\cal c}\grad G_{z}=
rdr\wedge d\varphi{\cal c}
\left({\p G\over \p r}{\p \over \p r}+
{1\over r^2}{\p G\over \p \varphi}{\p \over \p \varphi}\right)=
  r{\p G\over \p r}{\p \over \p r}d\varphi-{1\over r}{\p G\over \p \varphi}dr
              $$
($z=x+iy=r\cos\varphi_ir\sin\varphi$, $z_0=R\cos\theta+iR\sin\theta$)
   We need derivatives of Green function on boundary $r=1$:
              $$
  *dG_{z}\big\vert{|z|=1}=-\Omega{\cal c}
    \grad G_{z}\big\vert{|z|=1}=
  -r{\p G\over \p r}{\p \over \p r}\big\vert_{r=1}d\varphi=
              $$
      $$
-r{\p\over \p r}\left(
        -{1\over 2\pi}
\log \left|r^{i\varphi}-Re^{i\theta}\right|
        +{1\over 2\pi}
\log \left|1-Re^{-i\theta}r^{e^{i\varphi}}\right|
  \right)\big\vert_{r=1}d\varphi=
      $$
      $$
-r{\p\over \p r}\left(
        -{1\over 2\pi}
\log \sqrt{R^2-2Rr\cos(\theta-\varphi)r+r^2}
        +{1\over 2\pi}
\log \sqrt{1-2Rr\cos(\theta-\varphi)r+R^2r^2}
       \right)\big\vert_{r=1}d\varphi=
      $$
         $$
      -{1\over 2\pi} 
{1-R^2\over 1-2R\cos(\theta-\varphi)+R^2}d\varphi
            $$

{\bf Remark} (Peut etre un problem de sign?)


This we have the solution of Dirichle problme for the circle:
if $W$ harmonic function in the disc $D\colon\,x^2+y^2<1$
such that $W_{\p D}=\mu(\varphi)$
($r\to 1$????) then
         $$
W(R,\theta)={1\over 2\pi}
      \int_0^{2\pi}\mu(\varphi)
{1-R^2\over 1-2R\cos(\theta-\varphi)+R^2}d\varphi\,.
         $$

Look at this formula.
It is useful sometimes to write it:

         $$
W(R,\theta)={1\over 2\pi}
      \int_0^{2\pi}\mu(\varphi)
{1-R^2\over (1-Re^{ i(\theta-\varphi)})
            (1-Re^{-i(\theta-\varphi)})}d\varphi\,.
         $$

Notice also that 
If $\mu(\varphi)=C$ then obviously
          $$
W(R,\theta)=\cases{C \,\hbox{for $R<1$}\cr 
{C\over 2}\,\hbox{for $R=1$} }
          $$
if $\mu(\varphi)=e^{in\varphi}$, $n\geq 0$ then 
          $$
W(R,\theta)=\cases{z^n=R^n e^{in\varphi}\,\hbox{for $R<1$}\cr 
\,\hbox{for $R=1$} }
          $$
 then we come to harmonic polynomials.....

{\bf Remark} {\it In the formula (*) there is a jump,
so to be more precise we have to write:

         $$
W(R,\theta)=\lim_{R\to R_-}{1\over 2\pi}
      \int_0^{2\pi}\mu(\varphi)
{1-R^2\over (1-Re^{ i(\theta-\varphi)})
            (1-Re^{-i(\theta-\varphi)})}d\varphi\,.
         $$
sure this is important only for $R=1$:
         $$
W(1,\theta)=\lim_{R\to 1_-}{1\over 2\pi}
      \int_0^{2\pi}\mu(\varphi)
{1-R^2\over (1-Re^{ i(\theta-\varphi)})
            (1-Re^{-i(\theta-\varphi)})}d\varphi\,.
         $$
}

{\bf Remark 2} {\it All this can be done much more easilly 
using Fourier transform :
Indeed if $\mu(\varphi=e^{ik\theta})$
then $W=r^ke^{ik\varphi}$, hence 
           $$
W(R,\theta)=\sum\mu_k R^k e^{ik\theta}=
  {1\over 2\pi}\sum_k\left(
   \int_0^{2\pi}\mu(\varphi)e^{-ik\varphi}d\varphi\right)
         R^{|k|} e^{ik\theta}=
           $$
      $$
     {1\over 2\pi}\int_0^{2\pi}
\left(\sum_{k=-\infty}^{k=\infty}
   R^{|k|}e^{ik(\theta-\varphi)}\right)
      d\varphi
      = {1\over 2\pi}\int_0^{2\pi}
\left(
{1\over 1-Re^{i(\theta-\varphi)}}+
{1\over 1-Re^{-i(\theta-\varphi)}}-1
\right)
      d\varphi
      $$
}

\m

Sure we can solve Dirichle problem
just using angle function.
  Recall shortly the calculation of
 angle function
            $$
       Angle_C(\R)=\int_C{(x-X)dy-(y-Y)dx} 
             $$
(compare with $*d\log|z|={xdy-ydx\over (x-X)^2+(y-Y)^2}$).
Put $x=\cos\varphi, y=\sin\varphi$ we come to
           $$
   L(R,\theta,\varphi)d\varphi=
  {(x-X)dy-(y-Y)dx\over (x-X)^2+(y-Y)^2}
\big\vert_
{X=R\cos\theta, Y=R\sin\theta,x=\cos\varphi,y=\sin\varphi}=
           $$
       $$
    {1-R\cos(\theta-\varphi)\over 
1-2R\cos(\theta-\varphi)+R^2}d\varphi
       $$

Now we solve Dirichle problem via integral equation.

  Consider the distribution $\nu(\varphi)$ on the circle.
  It defines thefunction:
              $$
     U(R,\theta)=
\int_0^{2\pi}\nu(\varphi)
    L(R,\theta,\varphi)d\varphi=
\int_0^{2\pi}\nu(\varphi)
    {1-R\cos(\theta-\varphi)\over 
1-2R\cos(\theta-\varphi)+R^2}d\varphi\,.
              $$
This function is harmonic in the interior of the disc
, in the exterior of the dis, but it has the jump:
            $$
    U(R,\theta)\big\vert_{R=1-0_+}=
    U(R,\theta)\big\vert_{R=1}+\pi\nu(\theta)\,,\quad
    U(R,\theta)\big\vert_{R=1+0_+}=
    U(R,\theta)\big\vert_{R=1}-\pi\nu(\theta)\,.
            $$

In particular we come to the harmonic function
in the disc such that its values at the boundary are equal to
        $$
\mu(\varphi)=
    U(R,\theta)\big\vert_{R=1-0_+}=
    U(R,\theta)\big\vert_{R=1}+\pi\nu(\theta)=
       $$
       $$
\int_0^{2\pi}\nu(\varphi)
    {1-R\cos(\theta-\varphi)\over 
1-2R\cos(\theta-\varphi)+R^2}\big\vert_{R=1}d\varphi+
\pi\nu(\theta)=
   {1\over 2}\int_0^{2\pi}\nu(\varphi)
    d\varphi+
\pi\nu(\theta)
        $$
This is linear integral equation on the function  
$\nu(\theta)$:
       $$
\mu(\theta)=
     {1\over 2}\int_0^{2\pi}\nu(\varphi)
    d\varphi+
\pi\nu(\theta)\,.
        $$
Solving it we come to
        $$
\nu(\theta)={\mu(\theta)\over \pi}-{1\over 4\pi^2}
\int_0^{2\pi}\mu(\varphi)d\varphi\,.
         $$
Hence   $$
     U(R,\theta)=
\int_0^{2\pi}\nu(\varphi)
    L(R,\theta,\varphi)d\varphi=
        $$
        $$
\int_0^{2\pi}\left[
   {\mu(\varphi)\over \pi}-{1\over 4\pi^2}
\int_0^{2\pi}\mu(\tau)d\tau\right]
    L(R,\theta,\varphi)d\varphi=
       $$
       $$
\int_0^{2\pi}\mu(\varphi){L(R,\theta,\varphi)\over \pi}-
\int_0^{2\pi}{L(R,\theta,\varphi\over \pi}d\varphi\cdot
 {1\over 4\pi^2}\int_0^{2\pi}\mu(\tau)d\tau=
        $$
        $$
\int_0^{2\pi}\mu(\varphi){L(R,\theta,\varphi)\over \pi}-
2\pi\cdot
 {1\over 4\pi^2}\int_0^{2\pi}\mu(\varphi)d\varphi=
{1\over \pi}
  \int_0^{2\pi}\mu(\varphi)
  \left(L(R,\theta,\varphi)-{1\over 2}\right)d\varphi=
              $$
         $$
       {1\over \pi}
  \int_0^{2\pi}\mu(\varphi)
  \left({1-R\cos(\theta-\varphi)\over 
1-2R\cos(\theta-\varphi)+R^2}-{1\over 2}\right)d\varphi
       =
 {1\over 2\pi}
  \int_0^{2\pi}\mu(\varphi)
  \left({1-R^2\over 
1-2R\cos(\theta-\varphi)+R^2}-{1\over 2}\right)d\varphi
         $$
Thus we come to the solutioj of Dirichle problem.

{\bf Remark} {\it Sure using angle function we can reconstruct
not only $*dG$ but the Green fucntion also.
We take the ordinary Green function, $G_\infty(z_0,z)$
and using Dirichle problem solution we will find harmonic
function $W$ such that $G_\infty+W$ becomes the Green function:
  this harmonic function is the soluton of
Dirichle problem $W\big\vert_{boundary}=-
G_\infty\big\vert_{boundary}$
}

\bigskip

\centerline{Upper-half plane: Green function, Dirichle problem}
G
We know that $w={iz+1\over z+i}$ maps $\H\leftrightarrow D$
  (in fact this is not arbitrary: see the blog on 10-th August)

taking composition we come to the map:
        $$
w=w(\xi,z)={{iz+1\over z+i}-{i\zeta+1\over \zeta+i}
        \over 
 1-{-i\bar\zeta+1\over \bar\zeta-i}{iz+1\over z+i} }
        $$
maps upper half plane $H$ on the disc $x^2+y^2<1$
and a point $\zeta$ on the centre.

Hence Green function for half-plane is
        $$
  G(z_0,z)=\log|w(z_0,z)|
   \eqno (3.1)
        $$
{\bf Remark} {this is very stupid way to calculate
Green function. Much easier another way: see the blog on 10 August!}

 It seems that here it is much easier way to 
calculate Dirichle problem solution
straightforwardly, using angle function.
We will do it, but in tomorrow file you will see that
it is much easier to calculate straightforwardly Green function
using reciprocity method.
   
Indeed due to
       $$
   A_C(\R)=\int_C {(x-X)dy-(y-Y)dx\over (x-X)^2+(y-Y)^2}
       $$
Put $x=t,y=0$ we come to
          $$
L(\R,t)dt={Ydt\over (t-X)^2+Y^2}=
d\left(\arctan{t-X\over y}\right)
          $$
The potential of double layer with density $\nu(t)$
(compare with disc) is equal to
        $$
U(X,Y)=\int_{-\infty}^{\infty}\nu(t)
   {Ydt\over (t-X)^2+Y^2}
      eqno (2.1)
        $$
in the same way as for the circle
          $$
  U(X,0_+)=\pi\nu(x)+U(X,0),
          $$
but for the boundary of half plane this function vanishes:
the calculations are simpler, we do not need
to solve the integral equation. The function
       (2.1)  is the solution of Dirichle problem:
          $$
\cases {
  \Delta W=0\cr
  W\big\vert=\mu(x)
}
  \Rightarrow    W(X,Y)={1\over \pi}
   \int_{-\infty}^{\infty}\mu(t)
   {Ydt\over (t-X)^2+Y^2}
\eqno (2.2) 
          $$  

{\bf Remark} {\it One can see straightforwardly that
the function
           $$
 F(X)=\lim_{Y\to 0}{Ydt\over (t-X)^2+Y^2}=\pi\delta(t-x)\,.
           $$
Indeed   $F(X)=0$ for all  $X\not=t$ and $\int F(X)dX=\pi$

  This implies the boundary condition

}

Calculate Green function using solution of Dirichle problem:
  Let
             $$
G_\H(z_0,z)=G_{\rm classic}(z_0,z)+W=
          $$
        $$
     -{1\over 2\pi}\log|z-z_0|+W=
     -{1\over 2\pi}\log\sqrt {(x-X)^2+(y-Y)^2}+W(x,y)
              $$
The condition that $G_\H$ vanishes at absolute implies that
        $$
W(x,y)\big\vert_{y=0}=
     {1\over 2\pi}\log\sqrt {(x-X)^2+(y-Y)^2}\big\vert_{y=0}=
     {1\over 2\pi}\log\sqrt {(x-X)^2+Y^2}
     $$
thus we have that in Dirichle problem (2.2)
$\mu(t)={1\over 2\pi}\log\sqrt {(t-X)^2+Y^2}$, thus
     $$
W(x,y)={1\over 2\pi^2}
   \int_{-\infty}^{\infty}\log\sqrt {(t-X)^2+Y^2}
   {ydt\over (t-x)^2+y^2}
      $$
and
      $$
G_\H(z_0,z)=
     -{1\over 2\pi}\log\sqrt {(x-X)^2+(y-Y)^2}+
       {1\over 2\pi^2}
   \int_{-\infty}^{\infty}\log\sqrt {(t-X)^2+Y^2}
   {ydt\over (t-x)^2+y^2}\,.
      $$

It seems that calculation of the integral here are not  an easy task.
But instead,  note that the Green function can be immediately
    obtained using symmetry arguments:
         $$
  G_\H(z_0,z)=-{1\over 2\pi}\left(\log|z-z_0|-\log|z-\bar z_0|\right)=
     -{1\over 2\pi}\log\sqrt {(x-X)^2+(y-Y)^2}+
    {1\over 2\pi}\log\sqrt {(x-X)^2+(y+Y)^2}+
            $$
Thus:
                      $$      
   \int_{-\infty}^{\infty}\log\sqrt {(t-X)^2+Y^2}
   {ydt\over (t-x)^2+y^2}=
  ...\log|z-\bar z_0|=
    ...\log\sqrt {(x-X)^2+(y+Y)^2}+
                     $$
Beautiful, is not it???

********************************************

\bigskip

\centerline  {\bf 8 August  2017}

\centerline {\bf Calculation for half-plane, encore}

  Yesterday we did difficult caclulations for
half-plane. On the other hand it is obvious that
     $$
-{1\over 2\pi}\left(\log|z-z_0|-\log|z+\bar z_0|\right)
     $$
is a Green function!
This imeediately gives answer
for $*dG$ on the absolute and implies
many identities!
 inncluding the relation for the integral (see the end of yesterday file.)


********************************************

\bigskip

\centerline  {\bf 10 August  2017}


   {\bf Conformal map and
in terms of Green function}

  This is the topic why I began to read the Lavrentiev inspired
by French book. Now it is alright.
   If we know the Green function $G_U(z_0,z)$  
 then it defines in  $U\backslash \{z_0\}$ holomorphic function
$\hat G=G+iU$ where $U$ 
is defined up to period  $\Pi=\oint...$

Thus we come to  the map $e^{...G\over \Pi}$.

\m

  (comment ca marche: the question of bijection????)

\m

  {\bf Example}  How to construct $\H\to D$
We know that
             $$
  G_\H=...\left(\log|z-z_0|-\log|z-\bar z_0|\right)
             $$
(coefficient $-{1\over 2\pi}$)

  Hence  $G(z_0,z)=\approx {z-z_0\over z-\bar z_0}$

is it unique? Yes up to the coefficient $e^{i\varphi}$. 

 In  the blog on 7 August
we considered function
       $$
    {G(z_0,z)-G(z_0,\zeta)\over 1-\bar{G(z_0.\zeta)} 
G(z_0,z)}
       $$
Thus we see that this function is equal to
      $z-\zeta\over z-\bar\zeta$ up to a coefficient.

{\bf Remark} { \it We used here that every $F$ holomorphic map
of Disc $|z|<1$ onto itself such that $F(\zeta)=0$ 
is nothing but
${z-\zeta\over 1-\bar\zeta z}e^{i\varphi}$ }



\bye
