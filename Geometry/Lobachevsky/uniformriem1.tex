

\magnification=1200 


\baselineskip=14pt
\def\vare {\varepsilon}
\def\A {{\bf A}}
\def\t {\tilde}
\def\a {\alpha}
\def\K {{\bf K}}
\def\N {{\bf N}}
\def\w {\omega}
\def\s {{\sigma}}
\def\S {{\Sigma}}
\def\s {{\sigma}}
\def\p{\partial}
\def\vare{{\varepsilon}}
\def\Q {{\bf Q}}
\def\D {{\cal D}}
\def\G {{\Gamma}}
\def\C {{\bf C}}
\def\L {{\cal L}}
\def\Z {{\bf Z}}
\def\U  {{\cal U}}
\def\H {{\cal H}}
\def\R  {{\bf R}}
\def\S  {{\bf S}}
\def\E  {{\bf E}}
\def\l {\lambda}
\def\degree {{\bf {\rm degree}\,\,}}
\def \finish {${\,\,\vrule height1mm depth2mm width 8pt}$}
\def \m {\medskip}
\def\p {\partial}
\def\r {{\bf r}}
\def\pt {{\bf pt}}
\def\v {{\bf v}}
\def\n {{\bf n}}
\def\t {{\bf t}}
\def\b {{\bf b}}
\def\c {{\bf c }}
\def\e{{\bf e}}
\def\ac {{\bf a}}
\def \X   {{\bf X}}
\def \Y   {{\bf Y}}
\def \x   {{\bf x}}
\def \y   {{\bf y}}
\def \G{{\cal G}}
\def\ss  {\sigma_{\rm sph}}
\def\grad {{\rm grad\,}}
% I began this file on 21 July 2017

    \centerline {\bf Uniformisation of bounded domain}

This etude contains three topics:

  In the first topic we will discuss 
the following formula:
If $M$ is an hypersurface  surface in $\E^n$, then the function
       $$
   F_M(\r)=\hbox {the angle that a surface $M$ is seen 
   from the point $\r$}
       $$
obeys the equation
        $$
  \Delta F_M=0\,, F_M\big\vert_M=\pi   ????????
        $$
(in the case if $M$ is regular surface\footnote{$^*$}
{some technical conditions: Lyapunov surfaces????})

In the second topic we will discuss
There are two well know constructions: 
Dirichle integral,
which produces the harmonic function $h\colon U\to \R$ function on the 
domain $U$ of $\E^2$
obeying boundary conditions
                     $$
                h\big\vert_{\p U}=u\,.
                     $$
In the third topic we will consider 
holomorphic map $F\colon U\to D$ which maps
 domain $U$ to the disc  $|w|<1$ in ${\bf C}$.

    
There is a deep relation between these 
two constructions.
and the conformlal maps.  
This we will study in the third 
topic.


\bigskip
 \centerline {${\cal x} 1$}

  Let $C\colon \r=\r(t)$ be a curve in $\E^2$. 
Consider the function
            $$
W(\R)=\hbox{the angle at which one sees the curve
  $C$ from the point $\R$}=
      \int_C F^*\w_{\rm angle}\R\,.
         \eqno (1.1)
            $$
One can see that the function $W(\R)$
tends to zero at infinity 

{it obeys the boundary condition????
           $$
W\big\vert_{C}=\pi\,.
       \eqno (1.2)
           $$
no}
and it is harmonic function:
       $$
\Delta W=
{\p^2 W(x,y)\over \p x^2}+
{\p^2 W(x,y)\over \p y^2}=0\,.
      \eqno (1.3)
       $$
{Condition (1.2) is almost evident AND WRONG!}, 
We will prove (1.3)
later.

  To deal with function (1.1) we will express
it as an intergral.

 Consider  $1$-form $\w=\w_{\rm angle}$
        $$
\w=\w_{\rm angle}={xdy-ydx\over x^2+y^2}=d\varphi \,\,
      \hbox{in polar coordinates}\,,
         \eqno (1.4)
         $$     
      then one can see that
     $$
W(\R)=\int_{F_*(\R)C} \w=\int_{C} F^*(R)\w\,,
    \eqno (1.5)
      $$   
where $F\colon$ is a map
      $$
   F\colon\quad F(\r)=\r-\R
      $$

{\it Examples}


1. Let $AB$ be a segment 
of straight line between points
$A=(a,b)$
and $B=(1,1)$ on the axis $Ox=X$,
$A=(a,0)$, $B=(b,0)$.

Consider an arbitrary point $P=(X,Y)$ on $\E^2$.
 Let $N=(0,X)$ be a projection of the point $P$
on the axis $OX$. Then it 
is obvious that
       $$
W(P)=W(X,Y)=\angle PAN-\angle PBN=
  {\rm arctan\,}{X-a\over Y}-
  {\rm arctan\,}{X-b\over Y}\,.
       \eqno (1.Ex1.1)
       $$
We come to the same formula using (1.4):
     $$
 F^*(P)\w=F^*(X,Y)\left({xdy-ydx\over x^2+y^2}\right)=
 {(x-X)dy-(y-Y)dx\over (x-X)^2+(y-Y)^2}\,,
     $$
a curve $C\colon \cases {x=t\cr y=0\cr}\,, a\leq t\leq b$,
and
          $$
   W(P)= W(X,Y)=
   \int_C F^*(P)\w=
    \int_{\cases {x=t\cr y=0\cr} a\leq t\leq b}
 {(x-X)dy-(y-Y)dx\over (x-X)^2+(y-Y)^2}=
    \int_a^b {Ydt\over (X-t)+Y^2}
        \eqno (1, Ex.1.2)
          $$
Integrating we will come to formula
(1, Ex 1.1). 

\medskip

{\bf Example 2}

  Let $C$ be an arc of unit circle $x^2+y^2=1$:
           $$
    C\colon \cases {x=\cos t\cr y=\sin t\cr}\,
        \a\leq t\leq \beta
           $$
then equation (1.5) implies that
           $$
       W_C(P)=\colon\,\hbox
{angle such that  
one can see the arc $C$ from the point $P$
}=
           $$
       $$
\int_{\cases {x=\cos t\cr y=\sin t\cr}\,,
 \a\leq t\leq \beta}
 {(x-X)dy-(y-Y)dx\over (x-X)^2+(y-Y)^2}=
\int_\a^\beta
 {(x(t)-X)dy(t)-(y(t)-Y)dx(t)\over (x(t)-X)^2+(y(t)-Y)^2}=
          $$
          $$
\int_\a^\beta
 {(\cos t-r\sin\theta)\cos t-
(\sin t-r\sin\theta)(-\cos t)\over 
(\cos t-r\cos\theta)^2+(\sin t-\sin\theta)^2}dt=
\int_\a^\beta
 {1-r\cos(\theta-\varphi)\over 
1+r^2-2r\cos (\theta-\varphi)}dt\,.
       $$
Here we denote $X=r\cos\theta, Y=r\sin\theta$
coordinates of point $P$.

 Integrating we come to
          $$
        W_{S^1}(P)=
\int_\a^\beta
 {1-r\cos(\theta-\varphi)\over 
1+r^2-2r\cos (\theta-\varphi)}dt=
   {\rm arctan\,}\left({\sin (t-\theta)\over 
      \cos(t-\theta)-r}\right)\big\vert^\beta_\a\,.
       $$


(Sure one can guess antiderivative 
knowing that this is
the angle (1.1).)
 
\bigskip
The expressions calculated above are potentials
of the double layer.  We will use them later.

Consider the
 kernel $A_{S^1}(\r,\R)$ 
($\r=(\cos\varphi,\sin\varphi)$) 
is the point on the circle, 
$\R=(R\cos\theta,R\sin\theta)$ 
is the observation point,
and   $A_{S^1}(\r,\R)d\varphi$ is equal to 
the differential of the angle at which one sees the
arc $d\varphi$ from the point $\R$. We have
          $$
A_{S^1}(\r,\R)d\varphi=
{\cos\angle (\n,\R)d\varphi\over |\r-\R|}=
  {1-R\cos(\theta-\varphi)\over 
    1+R^2-2R\cos(\theta-\varphi)}d\varphi\,.
          $$
Compare with expressions above.

{\it Useful exercies  Show straightforwardly that
the integral over circle is equal to $2\pi$
or $\pi$ or $0$ depending on the position
of the point $\R$.}

  Write down also the angle function for unit sphere:
          $$
A_{S^2}=(\R,\r)=
 {\cos\angle (\n,\R)\over |\r-\R|}=
 {(\r-\R,\R)\over |\r-\R|^2}
          $$
here 
$\r=(\sin\theta\cos\varphi,
  \sin\theta\sin\varphi,\cos\theta)$
is a point on the unit sphere,
$\R=(R\Theta\cos\Phi,
  Rsin\Theta\sin\Phi,R\cos\Theta)$
is the observation point
We come to
          $$
A_{S^2}=(\R,\r)=
 {(\r-\R,\R)\over |\r-\R|^2}
  {1-R\cos(\Theta-\theta)-
R\sin\Theta\sin\theta
\left(\cos(\Phi-\varphi)-1\right)\over
1+^2-2R\cos\Theta-\theta)-
2R\sin\Theta\sin\theta
\left(\cos(\Phi-\varphi)-1\right)
}
          $$
{\bf Remark}  Sure one can write the formula 
for $1$-form like in Example 2, this is
the form
       $$
{dx\wedge dy\wedge dz\over r^2dr}
{xdy\wedge dz+ydz\wedge dx+zdx\wedge dy
\over r^3}=\sin\theta d\theta d\varphi
       $$
(see Appendix)


\bigskip




\centerline {\bf Dirichle problem}


Recall  Dirichle problem
Let $U$ be ball-like domain,
$\rho$ a functionon $U$ and
   $\mu$ a function on the boundary.
 We have to find a function  $F$
such that
       $$
    \cases
     {
\Delta F=\rho\cr
    F\big\vert_{\p U}=\mu
    }
\eqno (2.1)
       $$




To solve this problem we have to find Green function
 $G_U(\r,\R)$, the function such that
         $$
        \cases
       {
  \Delta_{(\r)} G(\r,\R)=\delta(\r-\R)\cr
      G(\r,\R)\big\vert_{\R\in \p U}=0
       }
\eqno (2.2)
         $$
If we find Green function then one can see that
              $$
f(R)=\int f(\r)G(\r,\R)+c\oint\mu(\r)\grad_{(\r)}G  
              $$
is a solution of (2.1)

We can go another scenario.

   We already know the fundamental solutions of
  equation $\Delta\Phi_n=\delta(\r)$
(see Appendix), and we know functions
potentials of double layer...

  We may try to express solutions
and Green function in this way.

W

   Study relation between these approaches
Let   $G_\infty$ be fundamental solution of
Laplacian in $\E^n$
       $$
     G^{(2)}_\infty=2\pi\log r, 
     G^{(3)}_\infty=-4\pi\log r,\dots, 
     G^{(n)}_\infty=
{\sigma_{n-1}\over n-2}\log r,\dots, 
       $$
(see Appendix)

Later we will show how to deduce
  $G$ knowing $G_\infty$ and using potential
of double layer.
  Now we just will see the relation between
these functions.
   We consider fundamental Green second identity:
   for arbitrary (generalised) functions $B,C$
            $$
\int_U (U\Delta V-V\Delta U)\Omega=
  \int_{\p U}\Omega{\cal c}
   \left(B\grad C-C\grad B\right)
            $$
where  $\Omega$ is volume form and
             $$
  \Delta F\Omega=\L_{\grad F}\Omega=
     d\left(\Omega{\cal c}\grad F\right)
         $$
in a more convenitional notations
             $$
\int_U (U\Delta V-V\Delta U)d^nx=
  \oint_{\p U}
\left(B\grad C-C\grad B\right)d{\bf S}
            $$
  
Let $G$ be a Green function of Dirichle
 

We perform considerations for $n=2$ and $n=3$.

\centerline {n=2}

Conisder unit circle $U\colon x^2+y^2<1$.
Find function harmonic in $U$ such that  
its value on the boundary is the function
   $F(\varphi)$
           $$
\Delta u=0\,, u\big\vert
           $$
Later we will discuss in detail what happens
at boundary.

Instead using Green function method we
will  use  functions considered in the 
first paragraph.

Table of functions:

  Double layer Potential of unit circle:
         $$
w(r,\theta=\int
     {1-r\cos(\theta-\varphi)
          \over
         1+r^2-2r\cos(\theta-\varphi)}
       \nu(\varphi)d\varphi
         $$ 

1. If $\nu(\varphi)=1$ then 
$w$ is angle (see the first paragraph).

2. For arbitrary (continuous) $\nu(\varphi)$
      $$
   w(r,\varphi)\big\vert_{r\to 1_-}=
   \pi\nu(\varphi)+w(r,\varphi)\big\vert_{r=1}\,,
           \quad
    w(r,\varphi)\big\vert_{r\to 1_+}=
   -\pi\nu(\varphi)+w(r,\varphi)\big\vert_{r=1}\,.
     $$

  If $w$ is harmonic function



\bigskip

In the previous paragraph we described the solution of
Dirichle problem. It seemed to be not so difficult.
Sure we were save because we knew that this problem
has solution and it is unique.
In fact the Dirichle problem for domain  $U$
in some sence is equivalent to
biholomorphme map

\bigskip

  $$ $$
\centerline {\bf Appendix. Green function}
   
   We consider $n$-dimensional Euclidean space $\E^n$.

    Let $\Omega=dx^1\wedge\dots\wedge dx^n$
be a volume form on $\E^n$.

Note that in `polar' coordinates
      $$
\Omega=\sigma_n r^{n-1}dr \ss\,,
      $$
where $r=\sqrt{{(x^1)}^2+\dots+{(x^n)}^2}$ and
       $$
   \ss=(-1)^kx^k{\Omega\over r^ndx^k}=\sum_{k=1}^n
       (-1)^k{x^k dx^1\dots dx^{k-1}\wedge 
      dx^{k+1}\wedge\dots \wedge dx^n\over r^n }\,. 
       $$
Namely:
           $$
r^{n-1}dr \ss=r^{n-2}
\sum_{i=1}^n\left(x^idx^i\right)\sum_{j=1}^n
          \left( 
        (-1)^k{x^k dx^1\dots dx^{k-1}\wedge 
      dx^{k+1}\wedge\dots \wedge dx^n\over r^n }
         \right)=\Omega\,.
        \eqno (App1.1)
           $$      
(Note that area $\sigma_n$
of $n$-dimensional
unit sphere $S^n$ is equal to 
       $$
 \s_n=\int_{S^n\subset \E^{n+1}}\ss=
{2\pi^{n+1\over 2}\over\Gamma
\left({n+1\over 2}\right)}
  \eqno (App1.1b) 
     $$
\footnote{$^*$}{Indeed in $$
\int_{\E^{n+1}}e^{-r^2}\Omega=
  \left(\int e^{-x^2}dx\right)^{n+1}=
      \pi^{n+1\over 2}\,,
   $$
and on the other hand in `polar' coordinates
          $$
\int_{\E^{n+1}}e^{-r^2}\Omega=
   \int_{\E^{n+1}}e^{-r^2}r^{n}dr\ss=
   \int_{r=1}\ss\cdot 
   \int_{\E^{n+1}}e^{-r^2}r^{n-1}dr\ss=
\sigma_n\Gamma\left({n+1\over 2}\right)
     $$
This implies (App1.1b).
   })

Consider on $\E^n\backslash 0$ spherically invariant
function $\Phi=\Phi(r)$ such that
$n-1$-form
               $$
     \w_\Phi=  \Omega {\cal c} {\rm grad\,}\Phi-
      \L_{{\rm grad\,}\Phi}
               $$
is closed form in $\E^n/0$.

Thus we come to the statement

We see that
          $$
  \w_\Phi=r^{n-1}dr{\cal c}
     \left(
       {d\Phi(r)\over dr}{\p\over \p r}
      \right)=r^{n-1}{d\Phi(r)\over dr}\ss
           $$
Hence we have that for $\Phi=r^{2-n}$
(and $\Phi=\log r$ for $\E^2$)
the form $\w_\Phi$ is closed non-exact form.


The function 
  $$
         \Phi_n=\cases 
                {
    r^{2-n} 
    {\rm if\,} n\not= 2\cr
    \log |r| {\rm if\,} n=2\cr
         }
           $$
defines closed non exact form,  
non-vanishing  charge (cohomology):
if $C$ is a sphere such that origin belong to it, then
         $$
   \int_C\Omega{\cal c}{\rm grad\,}\Phi_n=
    \oint_C {\rm grad \Phi}=
     \cases 
                {
    {\sigma_{n-1}\over 2-n} 
    {\rm if\,} n\not= 2\cr
     {2\pi}\,\,  {\rm if\,} n=2\cr
         }
       \eqno (1.2a)
         $$
Notice that condition that form 
$\Omega{\cal c}{\rm grad}\Phi_n$ is closed, means
that 
               $$
{\rm for\,\,}\r\not=0, \Delta\Phi=0\,.
               $$
Indeed by definition
        $$
\Delta\Phi_n={\rm div\,}{\grad }\Phi_n=
{\L_\grad\Phi_n\over \Omega}=
   {d\left(
 \Omega{\cal c}\grad\Phi_n
  \right)
\over \Omega}=0\,.
        $$
We come to 
{\bf Proposition}
Functions $\Phi_n$ are harmonic functions
out of the origin:



{\bf Theorem}  Laplacian of function $\Phi_n$,
considered in the sence of generalised functions
is eproportional to $\delta$-function:
           $$
\Delta\Phi_n={\sigma_{n-1}\over 2-n}\delta(\r)
\,,\quad (2\pi\delta(r)\,,\quad
\hbox{in the case if $n=2$}
       \eqno (App1.3)
           $$

One can say that closedness of the form
$\Omega{\cal c}\grad\Phi_n$ means that function
$\Phi_n$ is harmonic function, and
non-exactness of this form means that
$\Phi_n$ is proportional to $\delta$-
function\footnote{$^{**}$}
{Cohomology--- generalsed functions}


We will try to make the proof of this Theorem
`almost' proper.

   Let $\rho (\r)$ (density function)
vanishes for big $R$. 
We calculate derivatives in the sense
of generalised functions
          $$
(\Delta\Phi_n,\rho)=(\Phi_n,\Delta\rho)\,.
          $$
Calculate the right hand side explicitly:
            $$
(\Phi_n,\Delta\rho)=
\int_{\E^{n}}\Phi_n(\r)\Delta\rho(\r)d^nx=
\lim_{\vare\to 0+}
    \int_{r>\vare}\Phi_n(\r)\Delta\rho(\r)d^nx=
  \lim_{\vare\to 0_+}I_\vare
           $$
We have
             $$
  I_\vare=
\int_{r>\vare}\Phi_n(\r)\Delta\rho(\r)d^nx=
\int_{r<\vare<R}\left(
     \Phi_n(\r)\Delta\rho(\r)-
     \rho(\r)\Delta\Phi_n(\r)\right)d^nx\,.
         $$
since $\Delta\Phi_n(r)\equiv 0$ for $\r\not=0$ then
            $$
I_\vare=\int_{r<\vare<R}\left(
     \Phi_n(\r){\rm div\,}\grad\rho(\r)-
     \rho(\r){\rm div\,}\grad\Phi_n(\r)\right)d^nx=
             $$
             $$
\int_{r<\vare<R}\left(
     \Phi_n(\r)d
  \left(\Omega{\cal c}\grad\rho(\r)\right)-
     \rho(\r)d
      \left(
     \Omega{\cal c}
     \grad\Phi_n(\r)
         \right)
           \right)\,.
            $$        
Now using Stokes Theorem we rewrite this integral 
as integral over the boundary of the 
integration domain of $n-1$-form:
         $$
I_\vare=\int_{\p[{r<\vare<R}]=[{r=R}]-[{r=\vare}]}
\left(
     \Phi_n(\r)
  \left(\Omega{\cal c}\grad\rho(\r)\right)-
     \rho(\r)
      \left(
     \Omega{\cal c}
     \grad\Phi_n(\r)
         \right)
           \right)\,.
  \eqno(App1.4a)
         $$
This integral can be in other notations:
         $$
I_\vare=\oint_{\p[{r<\vare<R}]=[{r=R}]-[{r=\vare}]}
          \left(
     \Phi_n(\r)\grad\rho(\r)-
     \rho(\r)
      \grad\Phi_n(\r)
        \right)d{\bf S}\,.
  \eqno(App1.4b)
         $$
Now using the fact that function $\rho$
vanishes at big $R$ we rewrite (1.4a)
as
       $$
I_\vare=-\left[
  \int_{r=\vare}\left(
     \Phi_n(\r)
  \left(\Omega{\cal c}\grad\rho(\r)\right)-
     \rho(\r)
      \left(
     \Omega{\cal c}
     \grad\Phi_n(\r)
         \right)
           \right)\right]\,.
       $$
Look carefully on two integrals in this 
last expression. The first integral
        $$
  -\int_{r}\left(
     \Phi_n(\r)
  \left(\Omega{\cal c}\grad\rho(\r)\right)\right)
        $$
tends to zero if $\vare\to 0$ since
are of the sphere of radius $r$ is equal  
to  $\sigma_{n-1}\vare^{n-1}$.

For the second integral using the cohomology
formula (1.2a) we come to
$$
\lim_{\vare\to 0_+}
I_\vare=
     -\left[
  -\int_{r=\vare}
     \rho(\r)
  \left(\Omega{\cal c}\grad\Phi_n(\r)\right)
     \right]=
\rho(0)\int_{r}\Omega{\cal c}\grad\Phi_n=
\cases{
 \rho(0){\sigma_{n-1}\over 2-n}\,\,
 {\rm if}\,n\not=2\cr
 2\pi \rho(0)\,\,{\rm if}\, n=2\cr
}
$$  
Thus we calculated $(\Delta\Phi_n,\rho)$
and Theorem is proved.

\medskip

  This Theorem may be used to construct Green 
functions of $\Delta$ on $\E^n$.
      (see details in the second paragraph)
          Consider the function
        $$
G_n(\r,\R)=\cases
           {
       {2-n\over\sigma_{n-1}}\Phi_n(\r-\R)
          \,\,{\rm if}\, n\not=2\cr
       {1\over 2\pi}\log (\r-\R)
         \,\,{\rm if}\, n=2\cr
         }
        $$
Due to Theorem for every $\rho$ with compact support
the function      
 $$
U(\R)=\int \rho(\r)G(\r,\R)d^x
      $$

\bye
