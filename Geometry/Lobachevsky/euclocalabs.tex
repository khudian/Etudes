

Title: 
    Locally Euclidean geometries and hyperbolic geometry

Abstract: 
      Geometry on the surface of the cylinder is 
locally Euclidean. An "ant-mathematician" who lives 
on the cylinder will not distinguish the geometry 
of the surface at small distances from the Euclidean  
geometry; the Pythagorean Theorem will be almost the 
same, and for "not too large" triangles the sum of 
the angles will  be $\pi$.
     In the first part of talk, we will study locally 
Euclidean two-dimensional geometries. We will study  
these geometries by using discrete subgroups of the 
isometry group of the Euclidean plane $E^2$. 
The list of locally Euclidean geometries  is exhausted 
by the geometries  on the surface of the cylinder, 
on the surface of the torus, on the surface of the 
"twisted cylinder" (the Moebius band), and on the 
so-called Klein bottle.
     In the second part of the talk, we will consider 
the set of locally Euclidean  geometries, and will 
show that this set can be naturally parametrized by 
the points of the Lobachevsky (hyperbolic) plane.


