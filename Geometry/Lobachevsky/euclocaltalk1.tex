
  \magnification=1200 


\baselineskip=14pt
\def\vare {\varepsilon}
\def\AA {{\bf A}}
\def\A {{\cal A}}
\def\t {\tilde}
\def\bs {{\bf s}}
\def\a {\alpha}
\def\d {\delta}
\def\K {{\bf K}}
\def\N {{\bf N}}
\def\w {\omega}
\def\s {{\sigma}}
\def\S {{\Sigma}}
\def\s {{\sigma}}
\def\p{\partial}
\def\vare{{\varepsilon}}
\def\Q {{\bf Q}}
\def\D {{\cal D}}
\def\B {{\cal B}}
\def\G {{\Gamma}}
\def\C {{\bf C}}
\def\L {{\cal L}}
\def\F {{\cal F}}
\def\Z {{\bf Z}}
\def\U  {{\cal U}}
\def\H {{\bf H}}
\def\S  {{\bf S}}
\def\E  {{\bf E}}
\def\l {\lambda}
\def\degree {{\bf {\rm degree}\,\,}}
\def \finish {${\,\,\vrule height1mm depth2mm width 8pt}$}
\def \m {\medskip}
\def\p {\partial}
\def\r {{\bf r}}
\def\pt {{\bf pt}}
\def\v {{\bf v}}
\def\n {{\bf n}}
\def\t {{\bf t}}
\def\b {{\bf b}}
\def\c {{\bf c }}
\def\e{{\bf e}}
\def\ac {{\bf a}}
\def \X   {{\bf X}}
\def \Y   {{\bf Y}}
\def \x   {{\bf x}}
\def \y   {{\bf y}}
%\def \G{{\cal G}}
\def\ss  {\sigma_{\rm sph}}
\def \grad {{\rm grad\,}}
\def\e {{\bf e}}
\def\f {{\bf f}}
\def\g {{\bf g}}



  \centerline    {\bf Locally Euclidean geometries and 
       hyperbolic geometry}

  \centerline {H.M.Khudaverdian}

{\tt This talks is based on the book of S.Nikuin, R.Shafarevich}

\medskip

Abstract: {\it Geometry on the surface of the cylinder is 
locally Euclidean. An "ant-mathematician" who lives 
on the cylinder will not  distinguish the geometry of  
the surface at small distances from the Euclidean  
geometry; the Pythagorean Theorem will be almost the same, 
and for "not too large" triangles the sum of the angles will
 be $\pi$.
      In the first part of talk, we will study locally 
Euclidean two-dimensional geometries. We will study  
these geometries by using discrete subgroups of the 
isometry group of the Euclidean 
plane $E^2$. The list of locally Euclidean geometries is 
exhausted by the geometries  on the surface of the cylinder, 
on the surface of the torus, on the surface of the 
"twisted cylinder" (the Moebius band), 
and on the so-called Klein bottle.
         In the second part of the talk, we will 
consider the set of locally Euclidean  geometries, 
and will show that this set can be naturally 
parametrized by the points of the Lobachevsky 
(hyperbolic) plane.
}


  We consider locally Euclidean $2$-dimensional 
geometries.  An arbitrary $2$-dimensional  geometry 
can be considered ass
$2$-dimensional  Riemannian surface---$(M,G)$.
$M$ is a surface, and $G$ defines scalar product of
tangent vectors, i.e. length of an arbitrary curve.
For arbitrary curve
  $\x=\x(t), t_1\leq t\leq t_2$
             $$
 \hbox {length of the curve}=
   \int_{t_1}^{t_2} \sqrt{\left(\v(t),\v(t)\right)}dt\,,\,
\hbox{scalar product}\,\,  
\left(\v(t),\v(t)\right)=G(\v(t),\v(t))\,, 
             $$
where $\v(t)$ is velocity vector. 
 In local coordinates $x^i$,
the curve has appearance  $\x(t)=x^i(t)$,
    $G=g_{ik}(x)dx^idx^k$,
  $\v(t)=v^i(x(t))\p_i={dx^i(t)\over dt}{\p\over \p x^i}$,
and 
the scalar product of velocity vector on itself 
is equal to
      $$
\left(\v,\v\right)=
 v^i(x(t))g_{ik}(x(t))v^k(x(t))=
 {d x^i(x(t))\over dt}
   g_{ik}(x(t))
 {d x^k(x(t))\over dt}\,,
      $$
i.e.
      $$
 \hbox {length of the curve}=
   \int_{t_1}^{t_2} \sqrt{\left(\v(t),\v(t)\right)}dt
      =
    \int_{t_1}^{t_2} \sqrt{
  {d x^i(x(t))\over dt}
   g_{ik}(x(t))
 {d x^k(x(t))\over dt} \,.   
             }dt
      $$
We say that $M$ is (uniformly) locally Eucldean
if

1) in a vicinity of arbitrary points there are 
 Eucldean coordiantes, i.e. the coordinates  $u,v$
such that  $G=du^2+dv^2$ in these coordinates


2) This  neighborhood is enough large:
there exists $\delta>0$
   i.e. these coordinates $u,v$ are defined in
the circle of radius $\delta$


{\bf Remark} 
  Surface is called locally Euclidean if the first condition
is obeyed. In the case if the second condition is
obeyed also, 
the surface is called {\it uniformly locally  Euclidean.}


\medskip

{\bf Exercise}  Show that the surface of sphere
is not locally Eucldean.


{\bf Exercise}  Show that domain $a<x<b$ of $\E^2$
is locally Eucldean but it is not uniformly 
locally Eucldean.

(We suppose that metric on the surface
$M$ in $\E^3$
is  the metric induced from $\E^3$.)

In this talk we will consider only uniformly locally 
Euclidean surface, and we will call them just
locally Eucldean.

\m
   First of all examples of locally Euclidean surfaces.

   We will consider the following set of examples.


Let $\G$ be subgroup of isometries of $\E^2$.
We say that the subgroup $\G$ acts properly discontinuous
on $\E^2$ if there exists $\delta$ such that
for an arbitrary point $A\in \E^2$, and for an arbitrary
non-identity element $g\in \G$ 
        $$
      d(A,g(A))\geq \delta
        $$
 In other words it means that the distance between 
distinct points of an arbitrary orbit exceeds the  $\delta$.

Why these groups are interesting?  Because every such group
defines locally Eucldean manifold. 

Indeed, let $\G$ be an arbitrary subgroup
of group of isometries of $\E^2$.


Assign to the group $K$ a
space $M_\G$ of orbits of $G$-group action
on $\E^2$,
$M_\G=\E^2\backslash \G$, i.e. 
           
The points of the space $M_\G$, the orbits, 
 we denote by handwriting letters $\A,\B,C,\D,\dots$

Every  point $A\in \E^2$ produces the point
$\A\in \E^2\backslash \G$, the equivalence class
of a point $A$ with respect to the gropu $\G$:
         $$
\A=[A]_\G\,,\qquad  g\in \G\,,
A'=g(A)\in [A]\,. 
        $$
To establish the geometry on $M$
we define the distance between points $\A,\B$
as the minimal distance between the orbit
   $\{A^g\}$ and $\{B^g\}$:
            $$
\hbox{if $\A=[A]$ and $\B=[B]$ then}\,\,
d(\A,\B)=\min_{g,g'\in\G}d(A^g,B^{g'})
            $$ 

{\bf Exercise 1}  Let $\G$ be group of rweflection
with resepct to the line $l$.

  Describe the geometry $M_\G$
and show that this is not uniformly locally Euclidean manifold.

\medskip

{\bf Theorem}  $M_\G$ is uniformly locally Eucldean, 
if and only if the group $\G$ is uniformly discontinuous.  

 {\tt Sketch of the proof}.

Let $\G$ be uniformly discontinuous:
         $$
\exists\delta\geq 0 \,,,
\hbox{such that for an arbitrary}\,\,
 A\in \E^2\,, g\in \G\,, d(A^g,A)<\delta \Rightarrow
   g=1\,.    
        $$

and let $B$ be an arbitrary point  
which belong to the disc $D_{\delta\over r3}(A)$
Consider orbits $\A$ and $\B$ of these points.
It is easy to see from triangle inequality that
for arbitrary points $A'\in\A$ and $B'\in \B$
the distance between these points is bigger or equal to
  $r={\delta\over 2}$. Indeed let $A"=A^g$, and 
$B'=B^h$. Denote $\tilde B=B^{hg^{-1}}$. Then
$d(A',B')=
d\left(A, \left(B^h\right)^{g^{-1}}\right)=
d\left(A, \tilde B\right)$, and by triangle inequality
     $$
d(A',B')=d(A,\tilde B)\geq \left|d(B,\tilde B)-d(A,B)\right|
>\delta
     $$
if $B\not=\tilde B$.


Thus we see that in the case if two points $A$
and $B$ are closer than $\delta\over 2$,
then the distance between orbits $\A$ and $\B$
is equal to the distance $d(A,B)$.
This proves that $M_\G$ is uniformly locally Euclidean
if $\G$ is uniformly discontinous.



the orbit of the point $B$ and the orbit of the point $A$
the  disc of radius
\bye
