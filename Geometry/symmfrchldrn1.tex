\magnification=1200


  \centerline {\bf Symmetries}

\bigskip



  \centerline {\bf ${\cal x}1$Symmetries of triangles, quadrates, ....}

\bigskip

  We consider symmetries of polygons (triangles,....)

  Symmetry, it is a transformation  of polygon such that the new polygon coincides with former one.

\medskip

{\bf 1}  Let $ABC$ be an isoscales triangle ($AC=CB$).

We have transformation (reflection) $R=\pmatrix {AB\cr BA\cr}$. $R^2=I$:
                    $$
                    \pmatrix {AB\cr BA\cr}\circ \pmatrix {AB\cr BA\cr}=\pmatrix {AB\cr AB\cr}=I
                       $$

We come to {\it group} of transformations which possesses two elements $\{I, R\}$.
Multiplication table is:
             $$
             I\circ I=I, I\circ R=R,    R\circ R=I
             $$

We will denote this group by $D_2$.

\medskip

{\bf 2}  Let $ABC$ be an equilateral triangle ($AB=BC=AC$).


 We have much more symmetries:

   There are three reflections
                                  $$
   R_{AB}\colon \,\,R_{AB}=\pmatrix {AB\cr BA\cr},\,R_{AB}\colon \,\,
   R_{BC}=\pmatrix {BC\cr CB\cr}\,\,R_{AC}\colon \,\,R_{AC}=\pmatrix {CA\cr
   BA\cr}\,.
                                  $$
 All reflections of course obey the law: $R\circ R=I$


But there are also another symmetry transformations-- rotations. Consider rotation
  $S\colon\,\, \pmatrix{ABC\cr BCA\cr}$.
It is anticlock-wise rotation on the angle $120^\circ$. One can see that $S\circ S$ is
anticlock-wise rotation on the angle $240^\circ$, or
clock-wise rotation on the angle $120^\circ$:
   $$
S^2=I=\pmatrix{ABC\cr BCA\cr}\circ \pmatrix{ABC\cr BCA\cr}=\pmatrix{ABC\cr CAB\cr}
   $$
 So we have  $6$ symmetries (including identical)  $\{I,S,S^2, R_{AB}, R_{AC}, R_{BC}\}$:
                             $$
I=\pmatrix{ABC\cr ABC\cr},S=\pmatrix{ABC\cr BCA\cr},S^2=\pmatrix{ABC\cr CAB\cr},
                         $$
                         $$
R_{AB}=\pmatrix{ABC\cr BAC\cr},R_{AC}=\pmatrix{ABC\cr CBA\cr},R_{BC}=\pmatrix{ABC\cr ACB\cr}
                             $$

Look on multiplication table.
Now multiplication table is not so simple. In particular
$R_{AB}\circ S=\pmatrix{ABC\cr BAC\cr}\circ \pmatrix{ABC\cr BCA\cr}=\pmatrix{ABC\cr CBA\cr}=R_{AC}$,
but  $S\circ R_{AB}=\pmatrix{ABC\cr BCA\cr}\circ \pmatrix{ABC\cr BAC\cr}=\pmatrix{ABC\cr ACB\cr}=R_{BC}$.
We see that now multiplication is not commutative:
              $$
              R_{AB}\circ S\not=S\circ R_{AB}\,.
              $$

 We denote the group by $D_3$.


{\bf Exercise 1} (for sixthformers)

Calculate multiplication table.





 {\bf Exercise 2}  Consider subsets $\{I, S,S^2\}$, $\{I,R_{AB}\}$, $\{S, R_{AB}\}$. Analyse which sets are subgroups?


 {\bf Exercise 3} Describe all subgroups in $D_3$.


 \bigskip

  The group $D_3$ of symmetries of triangle possesses $6$ elements: Note that it is just group of {\it all permutations}
  of three elements (group $S_3$), say $\{A,B,C\}$ or say $\{pen, chair, table\}$....


{\bf Exercise 4}  Consider quadrat: quadrat $ABCD$. Find all symmetry transformations.

 {Exercise} Find all symmetry transformations.


   {\bf Exercise 5} How many permutations one can do if we have four elements $\{A,B,C,D\}$ or say
     $\{pen, pencil, chair, table\}$

 {Exercise} Is it TRUE or FALSE:  Any permutation of letters $\{A,B,C,D\}$ is a symmetry transformation.





\bigskip


\centerline {\bf ${\cal x}2$ Applications of symmetries to polynomial equations.}

  Consider now the problem from algebra.

\medskip

  Let $x_1,x_2$  be roots of quadratic polynomial $x^2+px+q=0$.

  By the reasons which will be clear later denote one root by $A$ and the second root by the letter $B$
  Consider the following expressions:

     a) $A-B$, $A+B$, $A^2+B^2$, $A^3-B^3$, $AB+B$, $AB+A+B$

  Apply to these expressions symmetry group $D_2$ of isoscales.

 We see that some of these expressions are  $D_2$-invariant, some not.

\smallskip

  {\bf Theorem } (Vi\`ete Theorem). $A+B=-p$, $AB=q$. This you know well. But Vi\`ete Theorem
  is  equivalent to another theorem

  \smallskip

  {\bf Theorem $^\prime$} An arbitrary polynomial on $A,B$ (recall that $A,B$ are roots of quadratic equation $x^2+px+q$)
      which is $D_2$-invariant can be expressed as a polynomial on $p,q$


You see the relation of group $D_2$ of isoscales with the problem of calculation expressions formed with the roots
of quadratic polynomial:

      \bigskip
\centerline {\bf
  $D_2$-symmetric expressions is easy to calculate!}


  {\bf Exercises} (for sixthformers)

    1. Let $A,B$ be roots of polynomial $x^2-x+1$.

    a) Calculate $a_n=A^n+B^n$ for $n=1,2,3,\dots$

     b) Calculate $A^2+B^2$ for the polynomial $x^2+x+10$.
     (The answer will be negative number. Explain it)

    c) something of this type


    d) {\it Formula for roots via Vi\`ete Theorem}

    Calculate $A-B=\pm \sqrt {(A-B)^2}=\pm \sqrt {(A+B)^2-4AB}$


    Note that $$
         A,B={(A+B)\pm (A-B)\over 2}={(A+B)\pm \sqrt {(A+B)^2-4AB}\over
         2}=\dots
            $$
We come to the formula for roots of quadratic polynomial.




\bigskip



 Above we see that $D_2$-invariant expressions of roots $A,B$ of quadratic polynomial $x^2+px+q$ is easy to calculate,
 i.e. you do not need to find roots themself.


   What about... $D_3$-invariant expressions of roots $A,B,C$ of cubic  polynomial $x^3+px^2+qx+r$


This question is much more tricker? Not only you do not know how to calculate rooots of the cubic polynomial,
but people who know it realise that formulae are much more complicated and out of practical use.

It turns out that in spite of the fact that we cannot calculate roots but Vi`ete Theorem still works:

  Let $x_1,x_2,x_3$ be roots. As before we denote them by $A,B,C$ Then comparing polynomials $x^3+px^2+qx+r$

We come to
            $$
         A+B+C=-p, AB+AC+BC=q, ABC=-r
            $$
Note that expressions in left hand side are $D_3$-invariant.

  We come to

 {\bf Theorem 2} (Vi\`ete Theorem). $A+B=-p$, $AB=q$. This you know well. But Vi\`ete Theorem
  is  equivalent to another theorem

  \smallskip

  {\bf Theorem $2^\prime$} An arbitrary polynomial on $A,B,C$ (recall that $A,B,C$
  are roots of cubic equation $x^3+px^2+qx+r=0$)
      which is $D_3$-invariant can be expressed as a polynomial on $p,q$


 {\bf Exercises} Consider expressions $A^2+B^2-C^2$, $A^2+B^2+C^2$... $A^2B+B^2C+C^2A$,
   $A^2B+B^2C+C^2A+AB^2+BC^2+CA^2$ where $A,B,C$ are roots of polynomial $x^3-x-1=0$

 where $A,B,C$ are roots of
   Find $D_3$-invariant expresisions

   Calculate $D_3$-invariant expression.


    $$ $$

    What next.


     One can try to consider group $D_4$. It is related with quadratic equation.

      May be you heard the name of Evarist Galois...

    Express




\bye
