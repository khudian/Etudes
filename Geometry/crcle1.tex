\magnification=1200

\def\r {\bf r}   \centerline {\bf Elementary Geometry and differential equations}
\def\a{\alpha}
   Consider the particle $P$ which moves along the circle
   of the radius $R$ with constant angular velocity $w$:
                        $$
         x=R\cos (wt+C),\quad y=y_0\sin (wt+C)
         \eqno (1)
                         $$
   The particle starts motion at the initial point  $A$ with coordinates
                 $$
   x=x_0=R\cos \a,\quad y=y_0=R\sin \a \eqno (2)
   $$
   and moves with constant speed
                         $$
                         v=wR
                         \eqno (3)
                         $$
   Does there exist another trajectory  for the point $P$
   starting  at the initial point $A$ such that
   such that angular velocity  around the
  initial point $O$ with coordinates $x=y=0$ is constant and
  speed is constant too?

   Yes there exists. To study this question apply differential equations.
   It is convenient to use polar coordinates $r,\varphi$.
   If $\bf r$ is the radius-vector with polar coordinates $r,\varphi$
   then $x=r\cos\varphi, y=y\sin\varphi$.

   Let $\r=\r(t)=(r(t),\varphi(t))$ be the trajectory of the particle $P$:
   $\r(0)=(R,C)$.  Express anglular and linear velocities in polar coordinates:
                    $$
                    w=\dot\varphi,\quad
                    v=\sqrt {\dot x^2+\dot y^2}=\dot r^2+r^2\dot\varphi^2
                        $$
  The condition that particle moves with constant angular velocity means that
  $w=\dot\varphi^2$ is constant. Hence the condition that linear velocity is constant
  too implies the differential equation:
                         $$
            \left({dr(t)\over dt}\right)+w^2r^2(t)=v^2
                           $$
We have to solve this differential equation with initial conditions (2)

  Using the standard technique one can solve this differential equation:
                            $$
              {dr\over dt}=\sqrt {v^2-wr^2 }
              \eqno ()
                            $$
                        $$
              \int {dr\over \sqrt {v^2-wr^2 }}=\int dt,\quad
                      {1\over w}\arcsin {wr\over v}=t+C
                           $$
                           $$
                    r(t)={v\over w}\sin (wt+C_1),\quad
                    \varphi(t)=wt+C_2,
                        $$
 where constants $C_1,C_2$ are defined by the initial conditions (2).
             $$
             \varphi(t)\big\vert_{t=0}=C_2=\a,\quad
               r(t)\big\vert_{t=0}={v\over w}\sin C_1=R
                        $$
Returning to cartesian coordinates we see that
               $$
      x(t)=r(t)\cos \varphi(t)={v\over w}\sin (wt+C_1)\cos (wt+C_2)
              $$
               $$
      y(t)=r(t)\sin \varphi(t)={v\over w}\sin (wt+C)\sin (wt+C_2)
      $$
Using the trigonometric formulae $2\sin x\cos y=\sin(x+y)+\sin(x-y)$,
$2\sin x\sin y=\cos(x-y)-\cos(x+y)$ we can rewrite these formulae in the following way:
           $$
x(t)=r(t)\cos \varphi(t)={v\over w}\sin (wt+C_1)\cos (wt+C_2)={v\over 2w}
          \left(2wt+C_1+C_2\right),
               $$
               $$
      y(t)=r(t)\sin \varphi(t)={v\over w}\sin (wt+C)\sin (wt+C_2)
           $$


Surprisingly we come to solution which is different from ()



  {\it Find the trajectory of  the point of the point $P$ which moves in the plane
  with constant angular velocity $w$ around the point $O$with constant
  pace $v$ and constant angular velocity $w$ around a fixed point
  $O$on the plane which moves with
   and with
  constant velocity $v$. What will be its trajectory?)=}

  \smallskip

  The trivial answer is: it will be the circle with centre $O$
  and with radius $R$ such that
                        $$
                                   v=wR\,.
                         $$
 But there is exactly one another case!!!
 The circle with radius ${R\over 2}$
 with centre in the point $O$.
 In fact trivial solution is "ogibajushaja" of these solutions!

 From the point of view of differential equations it is
 following:
 The diff.equation
                    $$
                      \cases
                       {
                       \dot x=\sqrt {1-x^2}\cr
                          x(0)=1,\quad o\leq t\leq {\pi\over 2}\cr
                          }
                          $$
               has exactly two solutions:
            the first one $x(t)=1$,
            the second one: $x(t)=sin t$.
            \bye
