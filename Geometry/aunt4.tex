
   \centerline {\bf A Tale on Differential Geometry}

  \bigskip


\magnification=1200

\def \he {He (She)$\,$}

   Once upon a time there was  an ant living on a sphere of radius $R$.
    One day
   he  asked himself some questions:
   What is the structure of the Universe (surface) where he
   lives?
   Is it a sphere? Is it a torus? Or may be something more
   sophisticated, e.g. pretzel (a surface with two holes)


  Three-dimensional human beings do not need to be mathematicians
  to distinguish between a sphere torus or pretzel.  They just have to look on the surface.
  But the ant living on
  two-dimensional surface cannot fly. He cannot look on the surface
   from outside. How can he judge about what surface he lives on
  \footnote{$^*$}{This is not very far from reality: For us human beings
   it is impossible to have a global look on three-dimensional  manifold.
 We need to develop local methods to understand global properties
 of our Universe. {\it Differential Geometry} allows to study
  global properties of manifold with local tools.}?

  Our ant loved mathematics and in particular {\it Differential Geometry}. He liked to draw various triangles,
   calculate their angles
  $\alpha,\beta,\gamma$, area
  $S(\Delta)$.
   He knew from geometry books that the sum of the angles of
   a triangle equals  $\pi$,
    but for triangles which he drew it was not right!!!!

       Finally he understood that the following formula is true:
        For every triangle
               $$
     {\left(\alpha+\beta+\gamma-\pi\right)\over S(\Delta)}=c
     \eqno (1)
               $$
  A constant in the right hand side depended neither on size of triangle
   nor the triangles location.  After hard research he came to conclusion
   that its Universe can be considered as a sphere embedded in three-dimensional
    Euclidean space  and a constant $c$ is related with
    radius of this sphere by the relation
                 $$
                 c={1\over R^2}
                 \eqno (2)
                 $$
  ...Centuries passed. Men have deformed the sphere of our old ant.
    They smashed it. It seized to be round,
  but the ant civilisation survived. Moreover
 old books survived. New ant mathematicians try to understand the
 structure of their Universe.
 They see
 that formula (1) of the Ancient Ant mathematician is not true.
 For triangles at different places  the right hand side
 of the formula above is different. Why? If ants could fly and look
 on the surface from the cosmos they could see how much the sphere has been
 damaged by humans beings,
 how much it has been deformed, But the ants cannot fly. On the other hand
 they adore mathematics and in particular
 {\it Differential Geometry}. One day considering for every point very small triangles they
introduce
 so called curvature for every point $P$ as a limit of right hand
 side of the formula (1) for small triangles:
                     $$
 K(P)=\lim_{S(\Delta)\to 0}{\left(\alpha+\beta+\gamma-\pi\right)\over S(\Delta)}
                     $$
Ants realise that curvature which can be calculated in every point
gives a way to decide where they live on sphere, torus, pretzel...
They  come to following formula \footnote{$^{**}$} {In human
civilisation this formula is called Gau\ss $\,$-Bonet formula. The
right hand side of this formula is called Euler characteristics of
the surface.} : integral of curvature over the whole Universe (the
sphere) has to equal  $4\pi $,  for torus it must equal  $0$, for
pretzel it equalts $-4\pi$...
             $$
{1\over 2\pi}\int K(P)dP=2\left(1-\hbox{number of holes}\right)
             $$





   \bye
