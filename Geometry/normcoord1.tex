\magnification=1200
\def\p{\partial}
\def\t {\tilde}
\def \m {\medskip}
\def\degree {{\bf {\rm degree}\,\,}}
\def \finish {${\,\,\vrule height1mm depth2mm width 8pt}$}





\def\a {\alpha}
\def\vare{{\varepsilon}}
\def\l {\lambda}
\def\s {{\sigma}}

\def\G {{\Gamma}}

\def\A {{\bf A}}
\def\C {{\bf C}}
\def\E  {{\bf E}}
\def\K {{\bf K}}
\def\N {{\bf N}}
\def\Q {{\bf Q}}
\def\R  {{\bf R}}
\def\V {{\cal V}}
\def \X   {{\bf X}}
\def \Y   {{\bf Y}}
\def\Z {{\bf Z}}



\def\ac {{\bf a}}
\def\e{{\bf e}}
\def\f {{\bf f}}
\def\n {{\bf n}}
\def\r {{\bf r}}
\def\v {{\bf v}}
\def \x   {{\bf x}}
\def \y   {{\bf y}}


\def\pt {{\bf p}}
\def \exer {{\sl Exercise$\,\,$}}

 \centerline   {\bf Normal coordinates}
  \bigskip

 {\it I was badly surprised when googling I could not find real information about normal coordinates. The level of contemporary information is very low.  in particularly
 the quotation of the fact that if local coordinates $\{x^i\}$
 are normal coordinates in a vicinity of the point in Riemannian manifold then
              $$
           g_{ik}(x)x^i=x^k
           \eqno (0.1)
              $$
Sure that this fact implies very beautiful features of Taylor series expansion for metric.

Thirty years ago when I was unexperienced student I red these and many other facts in the Appendices to the famous article of Atia Bott and Pathody about the index of elliptic operator and in the book of Petrov.
I will try in this text after many many years to reconstruct old constructions....}

\medskip



\centerline {\bf $\cal x$0. Preliminaries. Normal gauging for
Electromagnetic and Yang-Mills field.}

We say that $A_\mu(x)$ is in normal gauge ($A_\mu\to A_\mu+ \p_\mu F$) if

           $$
           A_\mu(x) x^\mu=0
           $$
One can do it in hundred different ways and it is obvious that Taylor expansion coefficients are
curvature and its derivatives. The most elegant way I know from the preprint of Fateyev-Tyupkin-Schwarz:
Consider the identity
                         $$
                    A_\mu(x)=\int_{0}^1 F_{\mu\nu}(tx)tx^\nu dt+\p_\mu \int_0^1 A_\nu(tx)x^\nu dt
                         $$
Sure this is related with Poincare lemma and the  fact that
radial vector field $E=x^i{\p\over \p x^i}$ contracts Euclidean space to the point.
We see that in normal gauge
        $$
A_\mu(x)=\int_{0}^1 F_{\mu\nu}(tx)tx^\nu
  \eqno (0.1)
  $$
Comparing the terms of the order $n$ in the left and right hand sides we see that Taylor series for $A_\mu$
are curvature and its derivatives:
         $$
      A_\mu^{(n)}=...F_{\mu\nu}^{n-1}x^\nu
         $$
        where we denoted by $f^{(n)}$ the terms of the order $n$: $f=\sum f^{n}$.
We can look on the expression (0.1) applying integration by part. We come to:
           $$
       A_\mu(x)=\int_{0}^1 F_{\mu\nu}(tx)tx^\nu dt={1\over 2}F_{\mu\nu}(x)x^\nu+\dots
           $$
        This can be applied to yang-Mills field too.

  {\bf Exercises}


  \exer   Consider the equation on the function $f$
           $$
x^\mu \p_\mu f= F(x)
           $$
Show that it has the unique solution.


Compare Taylor series.

\exer Wright down the explicit expression for the solution,
 for smooth functions.

 $$ $$

\centerline {\bf $\cal x$1. Normal coordinates for Riemannian manifold. Coordinate approach.}

\m

{\bf Definition of normal coordinates}
 Let $\Gamma$ be Christoffel symbols of Levi-Civita connection $\nabla$.

 We say that $\{x^i\}$ are normal coordinates in a vicinity of $p$ if lines $x^i(t)=a^it$ are geodesics.

Recall the equation of geodesics:
     $$
\nabla_t \v_t={d^2x^i(t)\over dt^2}+{d x^k(t)\over dt}\Gamma_{km}^i(x(t)){d x^k(t)\over dt}=0\,.     
     $$
 Consider an arbitrary vector $\v\in T_\pt M$. Assign to this vector the geodesic $\x_\v (t)$ such that
 $\v$ is its tangent vector at the point $\pt$.  Consider the point $\x_\v (1)$ of this curve we come to the map
 which maps a vicinity of zero of $T_\pt M$ on manifold $M$. Choosing an arbitrary coordinates
 in $M$ we come to normal coordinates.

\exer.  Using equation of geodesics show that for normal coordinates

          $$
      \Gamma^i_{mn}(x)x^mx^n\equiv 0
      \eqno (1.1)
          $$
 \exer.
 Normal coordinates are defined up to linear transformation.


\exer Show that the condition (1.1) implies that coordinates are normal.
\m
\m

We rewrite the condition (0.1) in a more wise way:
           $$
        g_{ik}(x)x^k=g_{ik}(\pt)x^k
        \eqno (1.2)
           $$
In the case if metric in origin is defined by Kronecker symbol we come to (0.1).

\exer Show that the condition (1.2) implies the condition (1.1)
(We suppose that the connection is Levi-Civita connection)

We choose coordinates in the tangent space such that $g_{ik}{_\pt}$ is Kronecker delta-Symbol:
      $g_{ik}{_\pt}=\delta_{ik}$

Now study the formula (1) for these normal  coordinates



For any point $\x=a^it$ consider covector
          $$
       a_i(t)=g_{ik}(ta^1,\dots,ta^n)a^k
       \eqno (3)
          $$
We have to prove that
    $$
   {da_i(t)\over dt}\equiv 0
    $$
 This implies that in arbitrary normal coordinates
                $$
           g_{ik}(x)x^k=g_{ik}(x)\big\vert_{x=\pt}x^k
           \eqno (4)
                $$
          and in particularly we come to (1) if metric at the origin is Euclidean.

How to prove it?


             \centerline {{$\cal x $2}}


One way (not the best) is the following: Let ${x^i}$ be normal coordinates in a vicinity of the given point $\pt$
and $T(x)$ an arbitrary tensor field considered as formal power series. Then one can prove that
for Taylor series expansion
          $$
T(x)=T(\p)+{\p T\over \p x^i}\big\vert_{x=\pt}x^i+{1\over 2}{\p^2 T\over \p x^k\p x^i}\big\vert_{x=\pt}x^kx^i+
{1\over 6}{\p^2 T\over \p x^k\p x^i\p x^m}\big\vert_{x=\pt}x^kx^ix^m+\dots
          $$
one can replace usual derivatives on covariant ones compensating this changing by adding curvature tensor and its
covariant derivatives:
             $$
      T^i(x)=T^i(\p)+\left(\nabla_k T^i\right)\big\vert_{x=\pt}x^k+
      {1\over 2}\left(\nabla_m\nabla_k T^i\pm {1\over 3} R^i_{mkp}T^p\right)\big\vert_{x=\pt}x^mx^k+
                    $$
                    $$
{1\over 6}\left(\nabla_m\nabla_n\nabla_k T+\dots R^i_{mnp}\nabla_kT^p+
\dots \nabla_kR^i_{mnp}T^p\right)\big\vert_{x=\pt}x^mx^nx^k+\dots
                    $$
Then one can apply this formula for tensor field $g_{ik}(x)$. Noting that covariant derivatives of metric
tensor vanish we come to:
               $$
     g_{ik}(x)=g_{ik}\big\vert_{x=\pt}-{1\over 3}R_{imkn}\big\vert_{x=\pt}x^mx^n-
     {1\over 3!}\nabla_kR_{imkn}\big\vert_{x=\pt}x^mx^nx^k+
                $$
                $$
        +{1\over 5!}\left(-6\nabla_p\nabla_kR_{imkn}\big\vert_{x=\pt}+
          {16\over 3} R_{mkn}^r\big\vert_{x=\pt}R_{kipr}\big\vert_{x=\pt}+\dots
        \right)x^mx^nx^kx^p
               $$
One can note that all terms in this expansion except the first one vanish when contracting with $x^k$. This implies
(1.2). (see the book of  Petrov). I am not satisfied by this "curled" proof.

Nevertheless go further

\bigskip

\centerline {$\cal x$3 \bf Normal gauge for connection in the vector bundle: general picture}

\m

In this paragraph we consider more general situation:
Let $E\to M$ be vector bundle equipped with connection. We will consider the "normal gauge" for
the connection such that in the normal gauge coefficients of Taylor expansion for connection are
curvature and its derivatives.



Let $E\to M$ be fibre bundle over manifold $M$ with connection $D$. (The dimension of the fibre--$n$).

The connection $D$ is the operation which maps local sections to the local 1-forms:
$D(fr)=dfr+fDr$ for an arbitrary function $f$ and local section $s$.


 Let $\{s_i(x)\}$ be the set of $n$-independent local sections in the vicinity of the point $\pt$
 and $\{\theta^k_i\}$ ($i.k=1,\dots,n$) the set of $n^2$ 1-forms which define the connection in this local basis (in this trivialisation):
              $$
         Ds_i=\theta^k_is_k, \nabla_\Y s_i=Ds_i(\Y)=\theta^k_i(\Y)s_i
              $$
If $\theta^k_i=\theta^k_i\nu dx^\nu$, $r(x)=r^i(x)s_i(x)$ be an arbitrary local section and vector field $\Y=Y^\mu\p_\mu$ then
              $$
   \nabla_\Y r(x)=\p_\Y r^i(x)s_i(x)+r^i(x)\theta_i^k(x)s_k(x)=
   Y^\mu\left(\p_\mu r^k(x)+\theta^k_i\mu(x)\right)s_k(x)\,.
              $$
The curvature two form $K$--is linear operator valued $2$- form. $K=D^2\theta$. In the basis of local sections $\{s_i(x)\}$
     $$
  D^2(r^is_i)=D\left(dr^is_i+r^iDs_i\right)=d^2r^is_i-dr^i\wedge Ds_i+dr^i\wedge Ds_i+r^iD^2s_i=
     $$
   $$
 r^iD(\theta_i^k s_k)=r^i(d\theta_i^k -\theta_i^p\wedge\theta_p^k )s_k,
   $$
i.e. curvature 2-form equals to
           $$
        K^i_k=d\theta_i^k -\theta_i^p\wedge\theta_p^k ,  \,\,\,
        K^i_{k\mu\nu}dx^\mu \wedge dx^\nu=
          \left(\p_\mu\theta^k_{i\nu}-\theta_{i\mu}^p\wedge \theta_{p\nu}^k\right)dx^\mu \wedge dx^\nu\,.
           $$



Now we will define very important notion of the
trivialisation which is  synchronised with the given coordinates.



Let $\{x^\mu\}$ be local coordinates in  vicinity of the point $\pt$.
We suppose that $x^i(\pt)=0$. The curve $x^i=a^it$ starting at the point $\pt$ is ray.

{\bf Definition}.
 We say that the trivialisation $\{s_i(x)\}$ is synchronised with the  coordinates  $\{x^\mu\}$
 if for an arbitrary point with coordinates $x^\mu=a^\mu$ any vector  $s_i(x)$ ($i=1,2,\dots,n$) is the parallel transport of the vector st the origin
  $s_i(\pt)$ along the  the ray $x^\mu(t)=a^\mu t$.


\exer {\it Consider one-dimensional vector bundle. Its connection ("electromagnetic field") is $1$-form $A_\mu dx^\mu$. Show exactly
how to find the trivialisation synchornised with the given coordinates.

Show that for this trivialisation we come to normal gauge: $A_\mu x^\mu=0$.
}


\exer {\it Show exactly the existence of synchronised trivialisation}

  Let  ${s_i(x)}$ be  trivialisation synchronised with coordinates $\{x^\mu\}$.

Consider radial vector field  $\R=x^\mu\p_\mu$. Since the bases $\{s_i\}$ are synchronized with coordinate systems
then $$
\theta_i^k(\R)= \theta^k_{i\mu}(x) x^\mu=0\,.
        $$


This is condition of parallel transport along the rays )

\exer {Show this}.


       \def \L {{\cal L}}
Using this identity and applying  Lie derivative $\L_\R$ to connection 1-forms  $\theta_i^k$ we come to
        $$
\L_\R \theta_i^k=d\theta^k_i\rfloor \R+d\theta(\R)=
d\theta^k_i\rfloor \R=\left(d\theta^k_i-\theta_i^m\wedge \theta_m^k\right)\rfloor \R=
      K_i^k\rfloor \R
        $$
Thus we come to the relation between connection forms and curvature.

Write down it in components:
      $$
 \L_\R (\theta_{i\mu}^kdx^\mu)=
 x^\nu\p_\nu\theta_{i\mu}^kdx^\mu+\theta_{i\mu}^kdx^\mu=x^\nu K^k_{i\nu\mu}dx^\mu
      $$
This gives us an explicit relation for Taylor series of $\theta$ and $K$.

{\bf Remark} very simple but important remark: $\L_\R dx^\mu=dx^\mu$.

The operator $x^\nu\p_\nu$ multiplies the monom of the order $n$ on the number $n$. Thus
       $$
       x^\nu\p_\nu\theta_{i\mu}^k+\theta_{i\mu}^k=x^\nu K^k_{i\nu\mu}
       \eqno (3.4)
       $$
 and
             $$
        (n+1){\theta^k_{i\mu}}^{(n)}=x^\nu {K^k_{i\nu\mu}}^{(n-1)}\,,
        \eqno (3.4a)
             $$
where we denote by $f^{(n)}$ the monom of $n$-th order in the expansion.

We have:
          $$
     \theta^k_{i\mu}\big\vert_\pt+2\p_\nu\theta^k_{i\mu}\big\vert_{\pt}x^\nu+
     {3\over 2}
     \p_\sigma\p_\nu\theta^k_{i\mu}\big\vert_{\pt}x^\nu x^\sigma
  +{4\over 3!}
     \p_\lambda\p_\sigma\p_\nu\theta^k_{i\mu}\big\vert_{\pt}x^\nu x^\sigma x^\lambda
      +{5\over 4!}
     \p_\rho\p_\lambda\p_\sigma\p_\nu\theta^k_{i\mu}\big\vert_{\pt}x^\nu x^\sigma x^\lambda x^\rho
     \dots
          $$
          $$
= K^k_{i\nu\mu}\big\vert_\pt x^\nu+\p_\sigma K^k_{i\nu\mu}\big\vert_\pt x^\nu x^\sigma+
    {1\over 2}\p_\lambda \p_\sigma K^k_{i\nu\mu}\big\vert_\pt x^\nu x^\sigma x^\lambda+
    {1\over 3!}\p_\lambda \p_\sigma K^k_{i\nu\mu}\big\vert_\pt x^\nu x^\sigma x^\lambda\p_\rho x^\rho+\dots
    \eqno (3.5)
          $$

We see that
          $$
          \matrix
          {
          \theta^k_{i\mu}\big\vert_\pt=0\cr
           \p_\nu\theta^k_{i\mu}\big\vert_{\pt}={1\over 2}K^k_{i\nu\mu}\big\vert_\pt\cr
           \p_\sigma\p_\nu\theta^k_{i\mu}\big\vert_{\pt}={2\over 3}\p_\sigma K^k_{i\nu\mu}\big\vert_\pt
           \dots
          }
          $$
1. Je croire que usual derivatives coudl be replaced by covariant derivatives (rightly understood), in the same way like in the previous paragraph.

\def\D {{\cal D\,}}
2. The equation (3.4) is valid for smooth fuinctions. The rest is for anbalytic functions. On the other hand oine can
solve the equation (3.4) in integrals:
    $$
   \D\theta_{i\mu}^k+\theta_{i\mu}^k=x^\nu K^k_{i\nu\mu},\,\, {\rm i.e.}\,\, \theta_{i\mu}^k=\int_0^1 tx^\nu K^k_{i\nu\mu}(tx)dt
    $$

(In general $\D F=G$, then $F=\D^{-1}G=\int_{0}^1{G(tx)\over t}dt$. This is Adamart lemma.
If $\D F+cF=G$, then $F=(\D+c)^{-1}G=\int_{0}^1 t^{c-1}G(tx)dt$).


All this stuff could be applied for  vector bundle which is just tangent space. Thus trivialsiation will be just



\centerline {Non-holonomial basis in affine space}




Let $M$ be space equipped with connection. The considerations of previous paragraph could be applied in this case.

 Trivialisation  synchronised with coordinate systems $\{x^\mu\}$ is non holonomic bases $\{\e_i(x)\}$
 which is obtained from the basis in the tangent space $T_\pt M$ thorugh the parallel transport along the rays.



Let $\theta^k_i=\theta^k_{i\mu}dx^\mu$ are connections forms of the connection $\nabla$ in non-holonomic synchronised basis $\{\e_i\}$
 and $\Gamma^\mu_{\nu\rho}$ are Christoffel symbols of this connection:
      $$
   \nabla_\mu \e_i=\theta^k_{i\mu}\e_k, \nabla_\mu\p_\nu=\Gamma_{\mu\nu}^\rho\p_\rho
      $$
and
        $$
D\e_i=dx^\nu\nabla_\nu(e_i^\mu\p_\mu)=dx^\nu\left(\p_\nu e_i^\mu+\Gamma^\mu_{\nu\sigma} e_i^\sigma\right)\p_\mu=
        dx^\nu\left(\p_\nu e_i^\mu+\Gamma^\mu_{\nu\sigma} e_i^\sigma\right)E^k_\mu\e_k\,,
        $$
where $\e_i=e_i^\mu\p_\mu$ and ${\bf E}^k=E^k_\mu dx^\mu$ are dual ${\bf E}^k(\e_i)=\delta^k_i$,
i.e. $Ee=I$.
Thus 1-forms $\{\theta_i^k\}$ are related with Christopher symbols by the relations:
          $$
      \theta_i^k=\theta^k_{i\nu} dx^\nu= dx^\nu\left(\p_\nu e_i^\mu+\Gamma^\mu_{\nu\sigma} e_i^\sigma\right)E^k_\mu
          $$

\bigskip

\centerline {\bf $\cal x$4 Normal coordinates on Riemannian   manifold and Tetrad formalism.}

\m

Let $(M,G)$ be Riemannian manifold. Define normal coordinates in vicinity of a given point $\pt$.

Let $\nabla$ be Levi-Civita connection on $M$.

  Consider normal coordinates of the Levi-Civita connection, i.e. coordinates such that the rays $x^i=a^it$ are geodesics.
  (See the paragraph 1.)   Consider the bases $\{\e_i(x)\}$ synchronised with these normal coordinates and the dual $1$-forms
  $\E^i$. (Basis vectors $\e_i$ at the point $x$ are the  parallel transport of the basis vector
  $\p_\i\big\vert_\pt$ along the ray and dual $1$-forms $E^i$. Dual forms are also the result of parallel transport along the rays.).
  As usual denote by  $\R=x^\mu\p_\mu$ the  radial vector field and denote by $\{\theta^k_i\}$ connection forms for this basis.


               $$
               \E^i(\R)\big\vert_x=x^i
               $$
        since dual forms are the result of parallel transport along the rays,
        $$
       \theta^k_i(\R)=0
        $$
since basis is syncronised with connection and for the metric $G$
       $$
   G=\E^i\otimes \E^i, i.e.      g_{\mu\nu}(x)=\sum_i E^i(x)\mu E^i\nu(x)
       $$
since metric is preserved via the parallel transport.
where as usual $\R=x^\mu\p_\mu$ is radial vector field. As usual denote connection components in the basis

   The condition that $\nabla$ is Levi-Civita condition means also that this is symmetric, i.e. it s torsion equals to zero:
                   We have for torsion
                     $$
             T(\e_i,\e_k)=\nabla_{\e_i}\e_k-\nabla_{\e_k}\e_i-[\e_i,\e_k]=\theta_k^r(\e_i)\e_r-\theta_i^r(\e_k)\e_r-[\e_i,\e_k]=
                     $$
                 $$
             \theta^r_p\wedge \theta^p(\e_i,\e_k)\e_r+?d\E^r(\e_i,\e_k)\e_r
                 $$
Hence for Levi-Civita connection
           $$
         d\E^r=-\theta^r_p\wedge \E^p
           $$
Now apply $\L_\R$. According the previous formulae we have:
               $$
 \L_\R \E^r=d(\E^r\rfloor \R)+d\E^r\rfloor \R=dx^i+d\E^r\rfloor \R=dx^i-\left(\theta^r_p\wedge \E^p\right)\rfloor \R=
               $$
               $$
    dx^i-\theta^r_p(\R)\E^p+\theta^r_p\E^p(\R)=dx^i+x^p\theta^r_p
               $$

Again:  Consider $r^2=x^1x^1+\dots+x^nx^n$. Then
           $$
r\circ\L_\R\circ {1\over r}\L_\R \E^r=r\circ\L_\R\circ {1\over r}\left(dx^i+x^p\theta^r_p\right)
           $$
Note that $\L_\R dx^\nu=dx^\nu$, $\L_\R f=nf$ if it has weight $f$. In particular $\L_\R{1\over r}=-{1\over r}$. Hence
                            $$
     \L_\R\left(dx^i\over r\right)=0,\,\,\,\,\,\L_\R\left(x^i\over r\right)=0
              $$
               and using the formula ()
              $$
 r\circ\L_\R\circ {1\over r}\L_\R \E^r=r\circ\L_\R\circ {1\over r}\left(dx^i+x^p\theta^r_p\right)=
 x^p\L_\R \theta^r_p.
              $$
On the other hand we already know that $\L_\R\theta^r_p$ is related with curvature:
              $$
      \L_\R \theta^r_p=d\theta^r_p\rfloor \R=
           \left(d\theta^r_p-\theta^r_q\wedge \theta^q_p\right)\rfloor \R=K^r_p\rfloor \R
              $$
where $K^r_p$ is curvature two-form, since $\theta^r_p\rfloor \R=0$. Hence we have that
                $$
                 r\circ\L_\R\circ {1\over r}\L_\R \E^r=
 x^p\L_\R \theta^r_p=x^pK^r_p\rfloor \R=x^\sigma K^r_{\sigma\nu\mu}x^\nu dx^\mu\,.
                $$
In the curvature tensor all indices are world indices except the index $r$.
Now calculate left hand side using the properties of $\L_\R dx^\mu=dx^\mu$:
      $$
r\circ\L_\R\circ {1\over r}\L_\R \E^r=r\circ\L_\R\circ {1\over r}\L_\R (E^r_\mu dx^\mu)=
r\circ\L_\R\circ {1\over r}\left(\p_\R E^r_\mu dx^\mu+E^r_\mu dx^\mu\right)=
      $$
      $$
r\circ\L_\R\circ \left(\p_\R E^r_\mu {dx^\mu\over r}+E^r_\mu {dx^\mu\over r}\right)=
r\circ\L_\R\circ \left(\left(\p_\R E^r_\mu +E^r_\mu \right){dx^\mu\over r}\right)=
(\p_\R^2E^r_\mu+\p_\R E^r_\mu)dx^\mu
      $$
We finally come to
         $$
         (\p_\R^2E^r_\mu+\p_\R E^r_\mu)dx^\mu=x^\sigma K^r_{\sigma\nu\mu}x^\nu dx^\mu\
         $$
The operator $\p_\R=x^\nu\p_\nu$ multiplies monom of the order $n$ on the number $n$. Hence
             $$
             (n^2+n){E^r_\mu}^{(n)}={K^r_{\sigma\nu\mu}}^{(n-2)}x^\sigma x^\nu
             $$
Before going further notice that because of anti-symmetricity of curvature tensor
for  $E^r_\mu x^\mu$ all higher orders terms vanish since

         $$
       E^r_\mu x^\mu
         $$
all higher order derivatives vanish! Thus we come to the formula 0.

We return to this later. Now
$K$ is nothing but Riemannian curvature tensor.  Thus we come to expansion formulae for tetrads:
(compare with (3.5))
  We have for the expansion of tetrad
          $$
2 \p_\nu E^r_\mu\big\vert_\p x^\nu+{6\over 2}\p_\sigma \p_\nu E^r_\mu\big\vert_\p x^\nu x^\sigma+
{12\over 3!}\p_\lambda \p_\sigma \p_\nu E^r_\mu\big\vert_\p x^\nu x^\sigma x^\lambda+
{20\over 4!}\p_\rho \p_\lambda \p_\sigma \p_\nu E^r_\mu\big\vert_\p x^\nu x^\sigma x^\lambda x^\rho+\dots=
          $$
          $$
R^r_{\sigma\nu\mu}\big\vert_\pt
          $$


 {\bf Remark} Purists told that the result is for analytic functions. On the other hand in the same way as for
 general case one can deduce the integral relation from the equation ()

{\bf Remark} Note that by the way we gave the proof that every connection has its metric locally
(we choose any basis in $T+\pt M$ and make its chosen parallel transport along some curves.)

One can see that connection with vanishing torsion is
Levi-Civita connection. One has to prove that the answer does not depend on gauging. (coordinate system. Est-ce que c'est vrai?).


          \bye
