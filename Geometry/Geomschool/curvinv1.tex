\magnification=1200

 We consider curves---  $x^i=x^i(t)$ in $d$-dimensional
eucledian space. Our goal is to construct local integrands
 over these curves which are

1) reparametrization invariant

2) invariant under global translation  and rotation of curves

3) are full divergence!---i.e. their Euler Lagrange equations are trivial

 If we omit condition (3) the answer is invariant densities--
length: $\sqrt {{dx^i\over dt}{dx^i\over dt}}$,...

But how to construct at least one invariant closed density?
In fact in $d$-dimensional case we restrict ourself by densities
whose rank (the order of derivatives) is not higher than $d$


1) $d=1$ . Answer is trivial:

               $$
        A=  x_t
                         \eqno (1)
                $$

2) $d=2$ Less trivial but evident too:
                 $$
 A= {x_ty_{tt}-y_tx_{tt}\over x_t^2+y_t^2}
                               \eqno (2)
                   $$
  One can see easily that (2) is total divergence
 (it is evident in polar coordinates):
                       $$
 A= {x_ty_{tt}-y_tx_{tt}\over x_t^2+y_t^2}=
              {d\varphi\over dt},\quad \varphi={\rm arctg}{y_t\over x_t}
                               \eqno (2^\prime)
                   $$


 3) Now I present closed (divergenceless)
 density in $3$-dimensional space:
                      $$
                      A=
                      {
                    \det
                   \pmatrix
                      {
           &x_t &y_t & z_t \cr
           &x_{tt} &y_{tt} &z_{tt} \cr
           &x_{ttt} &y_{ttt} &z_{ttt} \cr
                      }
                      \over
            (x_ty_{tt}-y_{t}x_{tt})^2+
            (y_tz_{tt}-z_{t}y_{tt})^2+
            (z_tx_{tt}-x_{t}z_{tt})^2
                      }
                      \cdot
                \sqrt
              {x_t^2+y_t^2+z_t^2}
                                             \eqno (3)
                        $$

  Do you want to check by straightforward calculation
that (3) is total divergence?
Note that "dimensionality"
 ("razmernostj") in (3) and (2) is $[{1\over sec}]$
I came to (3) using (2) (see in details the next Section)

In fact I think that in $d$-dimensional space all invariant densities
can be expressed in terms of determinant and its
first minors as in (3).
For example:

  Let we denote by $x^i_{(n)}$ the $n$-th derivative by $t$ of the
 coordinate $x^i$  in $I\!E^d$  $(i=1,...,d)$.
  We consider  the matrix
                        $$
          \pmatrix
                        {
             &x^1_{(1)}, &x^2_{(1)},  &x^3_{(1)}, ...,&x^d_{(1)} \cr
             &x^1_{(2)}, &x^2_{(2)},  &x^3_{(2)}, ...,&x^d_{(2)} \cr
             &x^1_{(3)}, &x^2_{(3)},  &x^3_{(3)}, ...,&x^d_{(3)} \cr
             &..., &...,  &..., ...,&... \cr
             &..., &...,  &..., ...,&... \cr
             &..., &...,  &..., ...,&... \cr
             &..., &...,  &..., ...,&... \cr
           &x^1_{(d)}, &x^2_{(d)},  &x^3_{(d)}, ...,&x^d_{(d)} \cr
                   }
                          \eqno (4)
                      $$
 We denote
                  $$
 \Delta_1= \sqrt {(x^1_{(1)})^2+(x^2_{(1)})^2+(x^3_{(1)})^2+
      (x^d_{(1)})^2}\,,
                    $$
                    $$
    \Delta_2=\sqrt
                        {
              \left(
                 \det
                             \pmatrix
                                    {
                       &x^1_{(1)}, &x^2_{(1)}\cr
                       &x^1_{(2)}, &x^2_{(2)} \cr
                                        }
                                  \right)^2
                               +
                 \dots+
              \left(
                 \det
                             \pmatrix
                                    {
                       &x^{d-1}_{(1)}, &x^d_{(1)}\cr
                       &x^{d-1}_{(2)}, &x^d_{(2)} \cr
                                        }
                                  \right)^2
                     }
                              $$

                    $$
    \Delta_3=\sqrt
                        {
              \left(
                 \det
                             \pmatrix
                                    {
                       &x^1_{(1)}, &x^2_{(1)},&x^3_{(1)}\cr
                       &x^1_{(2)}, &x^2_{(2)},&x^3_{(2)} \cr
                       &x^1_{(3)}, &x^2_{(3)},&x^3_{(3)} \cr
                                         }
                                  \right)^2
                               +
                 \dots+
              \left(
                 \det
                             \pmatrix
                                    {
                    &x^{d-2}_{(1)}, &x^{d-1}_{(1)}, &x^d_{(1)}\cr
                 &x^{d-2}_{(2)}, &x^{d-1}_{(2)}, &x^d_{(2)} \cr
                 &x^{d-2}_{(3)}, &x^{d-1}_{(3)}, &x^d_{(3)} \cr
                                      }
                                  \right)^2
                     }
                              $$
 e.t.c. $\Delta_d$ is nothing that the determinant of the matrix (4).

We consider in $d$-dimensional space the following invariant densities

               $$
 A_1=\Delta_1,\quad  A_2={\Delta_2\over\Delta_1^2},  \quad
         A_3={\Delta_3\cdot\Delta_1\over\Delta_2^2},\dots
           \eqno (5)
            $$
It is easy to see that $A_1$ is nothing but the lenghth of the
curve, and it is divergence in the case $d=1$, $A_2$ is the
density of second rank which coincides with (2) if $d=2$, $A_3$
is the density of third rank which coincides with (3) if $d=3$.
Sure one can continue this sequence: In $d$-dimensional case one
can consider the sequence of the invariant densities
   $\{A_1,...,A_d\}$ such that
the last one is total divergence...


                  \bigskip
         \centerline {\bf Section2}
                     \medskip
  How one can come to the expression (3)


 We recall at first the essence of the formula (2).
If $L$ is the curve in $I\!E^2$ then the tangent unit
 vector to this curve maps the curve $L$ to the curve in $S^1$:
            $$
             \varphi={\rm arctg}{y_t\over x_t}
              \eqno (2.1)
                $$
 and (2) corresponds to the pull back of the volume form $d\varphi$
  on the $S^1$. In the other word the density (2) is constructed via
 the canonical density on the $S^1$ and the density (2.1)
  {\bf which takes values in the unit vectors}---it is
 the composition of the densities.
                  $$
     A_2={d\over dt} {\rm arctg}{y_t\over x_t}
                $$
 Now bearing in mind this interpretation for (2)
we consider for the curve $L$ in the three-dimensional
space the map analogous to (2.1).---Considering
the unit vector to the every point of the curve
 $L$ we transform this curve to the curve
 on the sphere $S^2$:
                         $$
               \theta=
       {\rm arccos}{z_t\over\sqrt{x_t^2+y_t^2+z_t^2}},\quad
                 \varphi={\rm arctg}{y_t\over x_t}
           \eqno (2.2)
                    $$
 We come to the curve on the $2$-dimensional sphere and we apply
 the formula (2) for this curve. Of course the sphere is not
 euclidean  but one can guess that
      for the curves on the $S^2$ the analog of (2) will be something like
                $$
      {d\over dt} {\rm arctg}{\sin\theta \cdot\varphi_t\over \theta_t}
                  $$
Combining this density with previous one we come to
 the "answer"
                $$
  A_3 ={d\over dt} {\rm arctg}{\sin\theta \cdot\varphi_t\over \theta_t}=
     {d\over dt}\left (
        {\rm arctg}{\sin
                 \left(
{\rm arccos}{z_t\over\sqrt{x_t^2+y_t^2+z_t^2}}
           \right)
                       \cdot
        \left(
              {\rm arctg}{y_t\over x_t}
                     \right)_t
                 \over
                  \left(
         {\rm arccos}{z_t\over\sqrt{x_t^2+y_t^2+z_t^2}}
             \right)_t}
                 \right)
            \eqno (2.3)
                      $$

 Can you calculate (simplify) this? I cannot. It is here where I
began to think. (All previous constructions, you understand, do
no need paper in spite of their "artifficiel" complicatness...)

 For considering the awful  expression  (2.3) I consider first the case
 where  in the given point
                 $$
               z_t\big\vert_{t=t_0}=y_t\big\vert_{t=t_0}=0
               $$
 hence in the monstre (2.3)
                    $$
            \theta(t) \big\vert_{t=t_0}={\pi\over 2},\quad
            \varphi\big\vert_{t=t_0}=0
            $$
  Using  multiple times the fact known  that
               $$
            \left(f(x)x\right)^\prime_{x=0}=f(0)
                    $$
one comes to the result
                        $$
       A_2= {y_{tt}z_{ttt}-y_{ttt}z_{tt}\over y_{tt}^2+z_{tt}^2}\quad
                {\rm if}\quad
               z_t\big\vert_{t=t_0}=y_t\big\vert_{t=t_0}=0
                     $$
 Now I began to do tricks  considering
 the global rotations
\def\a{\alpha}
\def\b{\beta}

                $$
         \cases
          {
          x\rightarrow  x\cos\a+z\sin\a\cr
          z\rightarrow -x\sin\a+z\cos\a\cr
                      }\qquad
         \cases
          {
          x\rightarrow  x\cos\b+y\sin\b\cr
          y\rightarrow -x\sin\b+z\cos\b\cr
                      }\qquad,
             $$

where the angles $\a,\b$ are defined by the values
 of $z_t,y_t$ and I came to the (3).
 To be fair I have to tell that from the very beginning
 I noticed that determinant of the matrix (4) is the density
 and invariant density and I could not see nothing more--
 and everything have to be combined via this determinant?.
but I did not understand how the invariant combinations of minors arise.

 {\bf Conjecture.} Looking on the sequence (5)  I did the following
  conjecture:
     In the $d$-dimensional space one can consider the sequence
       $\{A_1,A_2,\dots,A_d\}$ where
                  $$
   A_1=\Delta_1,  A_2={\Delta_2\over A_1^2},
          A_3={\Delta_3\over A_2^2\cdot A_1^3},
          A_4={\Delta_4\over A_3^2\cdot A_2^3\cdot A_1^4},
                       \dots
         A_d={\Delta_d\over A_{d-1}^2\cdot A_{d-2}^3\cdot
              A_{d-3}^4\dots A_1^d}
                         $$
   In general the recurrent formula maybe is
                                $$
                        A_{n+1}=
      {\Delta_{n+1}\over\prod_{k=1}^{k=n} A_k^{n+2-k}}\quad???
              $$
 I have not any proof yet. Of course
   in the way how I constructed the (3)
  I can consider the sequence of the maps
     from the curve $L$ on the $S^{d-1}$ using the unit vectors
 but
 I afraid even to think about these calculations after my experience
for $S^2$ in (2.3).

    The  following hints make me to believe
    that this sequence may be is right:

  1) in the first three terms  is right

  2) the dimension is always the same $[{1\over sec}]$

   \medskip

 Sure this has to come from nineteen century. Indeed in classical mathematics
 for the curve in $d$-dimensional space there are the sequence of Frenet
  curvatures $\{k_1,...,k_{d-1}\}$. The curvature $k_1$ is proportional
 to $A_2$:
               $$
    k_1={\Delta_2\over\Delta_1^3}={A_2\over\Delta_1}
                $$
  it meausures in $[{1\over cm}]$.
My sequence have to be strictly connected with the sequence
 of curvatures but my goal was to come to divergence-like
 lagrangians. In some sence this means that the last
curvature is proportional to the divergence-like Lagrangian.

  One can consider the $d$-basis in every point of the curve.
The first vector is tangent vector, the first two-dimensional plane
is the plane spanned by the tangent vector and its acceleration
during the motion, e.t.c.--- Hence the movement on the
  $SO(d)$ of the solid particle can be considered.-- I think
it is just what people call rigid particle motion,

  I think that all these formulae
      have a sense  in rigid particle physics.

$$ $$

   \centerline {???????????}


            Do svidanija O.M.
\bye
