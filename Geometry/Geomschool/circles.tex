\magnification=1200
   \centerline {\bf Something very amazing about circles!?}

Thinking about Frenet curvature and performing {\bf fault
considerations} I cam eoccasionally to very elementar but amazing
statement.

  {\bf Question}. Consider the point on the plane which moves with
  constant angular velocity $w$ around the point $O$ and with
  constant velocity $v$. What will be its trajectory?

  \smallskip

  The trivial answer is: it will be the circle with centre $O$
  and with radius $R$ such that
                        $$
                                   v=wR\,.
                         $$
 But there is exactly one another case!!!
 The circle with radius ${R\over 2}$
 with centre in the point $O$.
 In fact trivial solution is "ogibajushaja" of these solutions!

 Frome the point of view of differential equations it is
 following:
 The diff.equation
                    $$
                      \cases
                       {
                       \dot x=\sqrt {1-x^2}\cr
                          x(0)=1,\quad o\leq t\leq {\pi\over 2}\cr
                          }
                          $$
               has exactly two solutions:
            the first one $x(t)=1$,
            the second one: $x(t)=sin t$.
            \bye
