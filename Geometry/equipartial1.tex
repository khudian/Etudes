\magnification=1200 %\baselineskip=14pt
\def\vare {\varepsilon}
\def\A {{\bf A}}
\def\t {\tilde}
\def\a {\alpha}
\def\K {{\bf K}}
\def\N {{\bf N}}
\def\V {{\cal V}}
\def\s {{\sigma}}
\def\S {{\Sigma}}
\def\s {{\sigma}}
\def\p{\partial}
\def\vare{{\varepsilon}}
\def\Q {{\bf Q}}
\def\D {{\cal D}}
\def\G {{\Gamma}}
\def\C {{\bf C}}
\def\M {{\cal M}}
\def\Z {{\bf Z}}
\def\U  {{\cal U}}
\def\H {{\cal H}}
\def\R  {{\bf R}}
\def\S  {{\bf S}}
\def\E  {{\bf E}}
\def\l {\lambda}
\def\degree {{\bf {\rm degree}\,\,}}
\def \finish {${\,\,\vrule height1mm depth2mm width 8pt}$}
\def \m {\medskip}
\def\p {\partial}
\def\r {{\bf r}}
\def\v {{\bf v}}
\def\n {{\bf n}}
\def\t {{\bf t}}
\def\b {{\bf b}}
\def\c {{\bf c }}
\def\e{{\bf e}}
\def\ac {{\bf a}}
\def \X   {{\bf X}}
\def \Y   {{\bf Y}}
\def \x   {{\bf x}}
\def \y   {{\bf y}}
\def \G{{\cal G}}

\centerline {\bf Cube and tetrahedron are not equipartial.}

  \bigskip

   {\bf Theorem 1} Two polygons of equal area
   are equipartial.

    This means that if polygons $\Pi_1$ and $\Pi_2$ have the same area then one can cut
     the polygon $\Pi_1$ on polygons $\pi_1,\dots,\pi_k$ and
     polygon  $\Pi_2$ on polygons $\pi'_1,\dots,\pi'_k$ such that polygons $\pi_k$
     are equal to polygons $\pi'_k$: $\pi_1=\pi'_1$,  $\pi_2=\pi'_2$, $\pi_3=\pi'_3,\dots, \pi_k=\pi'_k$.





  The proof is simple.   I give two hints to prove it.

\m

   {\it Hint 1. This was proved by amateur mathematician in XIX century.   This means
    that you can prove it! (Ja v svojo vremia  sdelal eto s udovoljstvijem!)}

    Hint 2. The proof immediately follows from the lemma.

\m

   {\bf Lemma}  Let $S_1$ be a triangle (with acute angles), and $S_2$ be an rectangtle
   such that they have the same area
      and one of the sides of triangle $S_1$ coincides with one of the sides of the rectangle  $S_2$.
     Then the triangle $S_1$ is equipartial with the rectangle $S_2$.

     {\it Proof}: Let $S_1$ be a $\triangle ABC$ with $a=BC$ and $S_2$ be rectangle with a side $a$.
      Consider the segment $MN$ joining midpoints $M,N$ of the sides $AB$ and $AC$,
      and the altitude (height) $AP$ of the triangle $AMN$. Then cut triangle $ABC$
      on $\triangle AMP$, $\triangle ANP$ and trapezoid $BMNC$. Puting triangles
      $ABC$, $AMP$ to the trapezoid we come to the rectangle.

\m
 Now it is easy to prove the Theorem. To see how the lemma helps consider


{\bf Example}. Show that rectangle $\Pi_1$ with sides $\{1,2\}$ and square $\Pi_2$ with sides
$\{\sqrt 2, \sqrt 2\}$ are equipartial.

{\it Solution:}  It follows from lemma that rectangle  $\Pi_1$ with sides $\{1,2\}$ and triangle
  with sides $\{2,2,2\sqrt 2\}$ are equipartial. Again applying lemma we see that triangle
  with sides $\{2,2,2\sqrt 2\}$ and rectangle $\Pi_3$ with sides  $\{2\sqrt 2,{\sqrt 2\over 2}\}$
  are equiaprtial. On the other hand rectangle $\Pi_3$ with sides  $\{2\sqrt 2,{\sqrt 2\over 2}\}$
  and square  $\Pi_2$ with sides $\{\sqrt 2, \sqrt 2\}$ are equipartial. Hence rectangle $\Pi_1$
  and square $\Pi_2$ are equiaprtial.





  \bigskip

{\it Now the most interesting part:}

\bigskip

   {\bf Theorem 2} The cub and tetrahedron of the same volume are not equipartial.

  It is one of the Hilbert's problem.

  The meaning of this theorems is following:  we know that area of the triangle is equal to
  $S={ah\over 2}$,
  where $h$--is the length of the altitude on the side $a$; and  the volume  of tetrahedron
   is equal to ${SH\over 3}$,
   where $S$ is the area of the base and
    $h$ is the length of the altitude on the base.
  The Theorem 2  means that one {\it cannot escape the Analysis} (consider integration) to define
  the volume of tetrahedron\footnote{$^*$}
 {The Theorem 1 claims that one comes to the formula for an area of triangle
 just by cutting rectangle
 and {\it without using Analysis}, i.e. without integration}.



   Few weeks ago I heard about wonderful proof of the second Theorem.
   (Davidik rasskazal mne eto dokazateljstvo, kogda ja vstretilsja s nim na Ukrajine.
   On priivjoz eto iz Moskvy)
    Here it is:

\m


  Consider cube with edge $1$ and regular tetrahedron of the same volume.
  Let $\theta$  be an angle between sides of the tetrahedron.
  One can see that $\theta\over \pi$ is irrational number.
(I think this follows easy from the fact that $\cos \theta=1/3$).

  For every polyhedron $C$ consider the function
              $$
           P_C=\sum_i |l_i| F(\varphi_i),\eqno (1)
              $$
where $\{l_i\}$ are edges of the polyhedron, $\varphi_i$ is 
the angle between sides adjusted
to the edge $l_i$ and $F(\varphi)$--a real valued function (v etoj funktsiji i vsja solj!).
The summation goes over all edges $l_i$.

Now the most interesting part:
 Consider an additive function $F$ on $\R$:
$F(a+b)=F(a)+F(b)$, i.e. the linear function on the real numbers,
considered as a vector space over rational numbers, such that
               $$
             F(\pi/2)=0, F(\theta)=1.
             \eqno (2)
               $$
 This function exists because $\theta\over \pi$ is irrational number, but
 this function is not linear in common sense, 
i.e. it is not {\it continuous} function.!
  To construct this function we need Hamel basis
  \footnote{$^{**}$}{The space $\R$ of real 
numbers is a vector space over rational numbers. The basis in this space
  is the set $\{e_\a\}$ of numbers such 
that for an  arbitrary  real number $\b$,
  $\b=\sum\gamma_\a e_\a$ where all $\{\gamma_\a\}$ 
are equal to zero except the finite set.
  The set $\{\gamma_\a\}$ is defined uniquely. The problem is that this vector space is "worse" than infinite-dimensional---
   its dimension is uncountable. To find a basis $\{\e_\a\}$ one needs to use transfinite induction, i.e.
    essentially use of Choice Axiom. A basis $\{e_\a\}$ is called {\it Hamel basis}}. Now the proof is in one line:

The function $P_C=\sum_i |l_i| F(\varphi_i)$ defined by relations (1) and  (2)
is equal to $0$ if $C$ is the cube of volume $1$
and it is equal to $z=6l$ if $C$ is regular tetrahedron, 
where $l$ is a length of the teathredon.
On the other hand the function  $P_C$ does not change under 
cutting of polyhedron because
the function $F$ is additive function of the angles,
and the condition $F(\pi)=0$ is obeyed.
 Contradiction.\finish



  I enjoyed so much this proof, but something is worrying:
  we use Choice Axiom for constructing additive not continuous function $F$ on all real numbers.
    Do we really need it?

  \bigskip

  I think one can escape the using of choice axiom.

\m


 Indeed  suppose one can cut cube on polyhedra $\gamma_1,\dots,\gamma_k$
  such that after putting with each other we come to tetrahedron.
   Consider the finite set of angles $\{\varphi_1,\dots,\varphi_N\}$
   which arise during cuttings.

   Let $V$ be the linear space spanned by the numbers $\{\varphi_1,\dots,\varphi_N\}$ with rational coefficients:
              $$
      V=\{a_1\varphi_1+\dots+a_N\varphi_N,\,\,\,{\rm where}\,\,\, a_1,\dots,a_n\in \Q\}
              $$
 Let $F$ be a linear function on $V$ which obeys the condition (2). One does not need Choice Axiom to
 construct this function (in spite of the fact that a function $F$ is not defined uniquely),
 since  $V$ is finite-dimensional vector space.
It  suffices to consider this function to come to contradiction.

   Krassivo nepravda li?








    \bye

          jonathan.bagley@manchester.ac.uk,
          Georgi.Boshnakov@manchester.ac.uk,
          David.Broomhead@manchester.ac.uk,
          victor.buchstaber@manchester.ac.uk,
          mark.coleman@manchester.ac.uk,
          francis.coghlan@manchester.ac.uk,
          carolyne.dean@manchester.ac.uk,
          kit.dodson@manchester.ac.uk,
          john.dold@manchester.ac.uk,
          peter.eccles@manchester.ac.uk,
          paul.glendinning@manchester.ac.uk,
          jerry.huke@manchester.ac.uk,
          gabor.megyesi@manchester.ac.uk,
          james.montaldi@manchester.ac.uk
          mark.muldoon@manchester.ac.uk,
          john.parkinson@manchester.ac.uk,
          mike.prest@manchester.ac.uk,
          nikita.sidorov@manchester.ac.uk,
          toby.stafford@manchester.ac.uk,
          peter.symonds@manchester.ac.uk,
          martin.taylor@manchester.ac.uk,
          markus.tressl@manchester.ac.uk,
          david.bell@manchester.ac.uk,
          alexandre.borovik@manchester.ac.uk,
          sergey.fedotov@manchester.ac.uk,
          theodore.voronov@manchester.ac.uk,
          grant.walker@manchester.ac.uk,
          reg.wood@manchester.ac.uk


         jonathan.bagley@manchester.ac.uk,
          Georgi.Boshnakov@manchester.ac.uk,
          David.Broomhead@manchester.ac.uk,
          victor.buchstaber@manchester.ac.uk,
          mark.coleman@manchester.ac.uk,
          francis.coghlan@manchester.ac.uk,
          carolyne.dean@manchester.ac.uk,
          kit.dodson@manchester.ac.uk,
          john.dold@manchester.ac.uk,
          peter.eccles@manchester.ac.uk,
          paul.glendinning@manchester.ac.uk,
          jerry.huke@manchester.ac.uk,
          gabor.megyesi@manchester.ac.uk,
          james.montaldi@manchester.ac.uk
          mark.muldoon@manchester.ac.uk,
          john.parkinson@manchester.ac.uk,
          mike.prest@manchester.ac.uk,
          nikita.sidorov@manchester.ac.uk,
          toby.stafford@manchester.ac.uk,
          peter.symonds@manchester.ac.uk,
          martin.taylor@manchester.ac.uk,
          markus.tressl@manchester.ac.uk,
          david.bell@manchester.ac.uk,
          alexandre.borovik@manchester.ac.uk,
          sergey.fedotov@manchester.ac.uk,
          theodore.voronov@manchester.ac.uk,
          grant.walker@manchester.ac.uk,
          reg.wood@manchester.ac.uk
