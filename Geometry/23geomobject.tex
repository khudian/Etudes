
\magnification=1200
\baselineskip=14pt

\def\A {{\bf A}} 
\def\B {{\cal B}}
\def\C {{\bf C}}
\def\CC {{\cal C}}
\def\Cl {{\tt \hbox{Cliff}}}
\def\E {{\bf E}}
\def\EE {{\cal E}}
\def\F {{\cal F}}
\def\FF {{\cal F}}
\def\G {\Gamma}
\def\GG {{\cal G}}
\def\H {{\bf H}}
\def\K {{\bf K}}
\def\L {{\cal L}}
\def\M {{\cal M}}
\def\N {{\bf N}}
\def\R {{\bf R}}
\def\Sb {{\bf S}}
\def\SS {{\cal S}}
\def\Tr {{\rm Tr\,}}
\def\V {{\cal V}}
\def\X {{\bf X}}
\def\XX {{\cal X}}
\def\Y {{\bf Y}}
\def\Z {{\bf Z}}

\def\a {\alpha}
\def\ac {{\bf a}}
\def\b {{\bf b}}
\def\bs {{\bf s}}
\def\c {{\bf c}}
\def\d {\delta}
\def\dist {{\tt \hbox{distance}}}
\def\e {{\bf e}}
\def\f {{\bf f}}
\def\finish {\blacksquare}
\def\g {{\bf g}}
\def\grad {{\rm grad\,}}
\def\h {\hbar}
\def\k {{\bf k}}
\def\l {{\bf l}}
\def\m {{\bf m}}
\def\n {{\bf n}}
\def\nameofthefile {\centerline}
\def\p {\partial}
\def\pb {{\bf p}}
\def\pt {{\bf pt}}
\def\q {{\bf q}}
\def\r {{\bf r}}
\def\s {\sigma}
\def\ss {{\bf s}}
\def\t {{\bf t}}
\def\tS {{\tilde \Sigma}}
\def\td {\tilde}
\def\v {{\bf v}}
\def\vare {\varepsilon}
\def\x {{\bf x}}
\def\y {{\bf y}}
\def\w {\omega}

\nameofthefile {\bf Geometrical object (look from outside)}


{\it Long ago I red the definiiton of geometrical object
in Beklemishev (Manuel on Analytical geometry).
 I was very happy to know how physisists define vectors,
 tensors,...
 and honestly long time did not understand the question:
 "does it exist"...., and the fact that it is the cocycle condition
 (see equation (3) below which guarantees it!))
 
 I struggled with this during all my life.
 My friend Ted Voronov always  emphasized  the  importance
 of cocycle condition for the existence of object.. 
 The proverb: "vek zhivi, vek uchisj, 
 durakom pomrjoshj" works fine in this case.}



   {\bf Definition}  We say that at the points of
   manifold $M$ we define {\tt geometrical object}
   if for every point $\pt\in M$ and for every local coordinates 
  $x^i_{(\a)}$ which are defined in the vicinity
  of this points we define the sequence
  $\Sigma^I$ of numbers 
        $$
\{x^i_{(\a)}\}\mapsto\quad
\Sigma^I=\Sigma^I\left(x^i_{(\a)}\right)
\eqno (1)
	$$
   such that
  the following conditions are obeyed:

1) Let $x^i_{(\a)}$ and $x^{i'}_{(\beta)}$
be two arbitrary coordinates which are defined in
the vicinity of the point $\pt$, then 
         $$
\Sigma^{I'}\left(x^{i'}_{(\beta)}\right)=
\Psi^{I'}_{\beta\a}\left(
\Sigma^{I}\left(x^{i}_{(\alpha)}\right)
\right)\,.
\eqno (2)
	 $$

2)  Let $x^i_{(\a)}$, $x^{i'}_{(\beta)}$,
$x^{i''}_{(\gamma)}$
be three  arbitrary coordinates which are defined in
the vicinity of the point $\pt$, 
and let object $\Sigma$ is defined 
              $$
	      \matrix
	      {
\hbox {in coordinates $\{x^i_{(\a)}\}$   by  the sequence 
    $\Sigma^I_\a\left(x^i_{(\a)}\right)$}\cr
\hbox {in coordinates $\{x^{i'}_{(\beta)}\}$   by  the sequence 
    $\Sigma^{I'}_\beta\left(x^{i'}_{(\beta)}\right)$}\cr
\hbox {in coordinates $\{x^{i''}_{(\gamma)}\}$   by  the sequence 
    $\Sigma^{I''}_\gamma\left(x^{i''}_{(\gamma)}\right)$}\cr
              }
        $$
	Then
          $$
\Psi^{I}_{\a\gamma}
    \left(
\Sigma^{I''}_\gamma
\left(
x^{i''}_{(\gamma)}
\right)
\right)=
\Psi^{I}_{\a\beta}
    \left(
\Psi^{I'}_{\beta\gamma}
\left(
\Sigma^{I''}
\left(
x^{i''}_{(\gamma)}
\right)
\right)
\right)=
\eqno (3)
  $$

Equation (2) defines transformation of components of
geometrical object when we transform coordinates.

Equations (3)  (cocycle conditions ) guarantee
that if you transform coordinates in different ways, the answer is
idependent. For example if we have  five different local coordinates
      $$
      x_{(\a)}\,,
      x_{(\beta)}\,,
      x_{(\gamma)}\,,
      x_{(\delta)}\,,
      x_{(\rho)}\,,
      x_{(\omega)}\,,
         $$
     such that
       $$
     x_{(\a)}\left(x_{(\beta)}
     \left(
     x_{(\gamma)}
     \left(
     x_{(\delta)}
     \right)
     \right)
     \right)
     =
     x_{(\a)}\left(x_{(\rho)}
     \left(
     x_{(\omega)}
     \left(
     x_{(\delta)}
     \right)
     \right)
     \right)  
       $$
then the coycle condition (3) guqrantees that
answer will be the same if you go from coordinates
$x_{(\delta)}$ to coordinates $x_{(\a)}$
through coordinates  $x_{(\gamma)}$
and $x_{(\beta)}$ or if you go from coordinates
$x_{(\delta)}$ to coordinates $x_{(\a)}$
through coordinates  $x_{(\omega)}$
and $x_{(\rho)}$ or if you go.


Cocycle conditions guarantees existence of the object.

\bigskip

Examples? It is crazy easy.   Remember how physisists define  vectors.
The define vector as array of components which transforms in a given
way if you change a basis.  It is only sometimes very curious students
ask question: does at least one vector  exist
\footnote{$^*$} {the existence is guaranteed by the
condition...[try to write it!!!] (Usually 
this question makes angry a lecturer if his (her)
level of mathematical culture is not too high)
}?

   \bye
  
 
