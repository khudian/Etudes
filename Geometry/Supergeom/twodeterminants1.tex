\def\vare {\varepsilon}
\def\B {{\mathcal B}}
\def\C {{\mathcal C}}
\def\t {\tilde}
\def\a {\alpha}
\def\K {{\bf K}}
\def\N {{\bf N}}
\def\V {{\cal V}}
\def\s {{\sigma}}
\def\S {{\Sigma}}
\def\s {{\sigma}}
\def\p{\partial}
\def\vare{{\varepsilon}}
\def\Q {{\bf Q}}
\def\D {{\cal D}}
\def\G {{\Gamma}}
\def\M {{\cal M}}
\def\Z {{\bf Z}}
\def\U  {{\cal U}}
\def\H {{\cal H}}
\def\R  {{\bf R}}
\def\S  {{\bf S}}
\def\E  {{\bf E}}
\def\l {\lambda}
\def\degree {{\bf {\rm degree}\,\,}}
\def \finish {${\,\,\vrule height1mm depth2mm width 8pt}$}
\def \m {\medskip}
\def\p {\partial}
\def\r {{\bf r}}
\def\v {{\bf v}}
\def\n {{\bf n}}
\def\e{{\bf e}}
\def\w{{\omega}}
\def\ac {{\bf a}}
\def \X   {{\bf X}}
\def \Y   {{\bf Y}}
\def \x   {{\bf x}}
\def \y   {{\bf y}}
\def \G{{\cal G}}
\def\Ber {{\rm Ber\,}}
\def\Tr {{\rm Tr\,}}
\def\Pf {{\rm Pf\,}}
\def\B {{\cal B}}
\def\DT {{\cal D}}
 \centerline {\bf Three identities on determinants}

{\bf 1.} Let $B$ be $m\times n$ matrix and $D$ be $n\times m$ matrix. Then
           $$
        \det(1+BD)=\det (1+DB)\,.
        \eqno (1)
           $$
 {\sl Proof}. $\,\,\,\Tr (BD)^k=\Tr (DB)^k$. Hence characteristic polynomials $\det(1+zBD)$ and
 $\det (1+zBD)$ coincide. Thus we come to (1).
 How to prove it in another way?


 It follows from (1) that if $D=B^+=(x^1,x^2,\dots,x^n)$ is $1\times n$ matrix then
            $$
       \det(\delta^{ik}+x^ix^k)=\det(1+BD)=\det(1+DB)=\det(1+x^ix^i)=1+(x^1)^2=\dots+(x^n)^2
            $$
   \bigskip
        The relation (1) has two very amusing  "generalisation":
\m

 {\bf 2.}  Assume that entries of matrices $B$ and $D$ {\it are "odd numbers", i.e.
anticommuting elements of some $\bf Z_2$-algebra (e.g.  odd elements of Grassmann algebra).}
In this case we have instead (1) the following elation:
            $$
            \det(1+\B\D)={1\over \det(1+\D\B)}
            \eqno (2)
            $$
   (Here and later denoting $\B$ instead $B$ and $\D$ instead $D$ I would like to emphasize
   the odd nature of matrices.)


     To prove (2) we note that   $\Tr (\B\D)^k=-\Tr(\D\B)^k$. Hence
               $$
          {d\over dz}\log\left[\det(1+zBD)\det(1+zDB)\right]=
          \Tr\left[(1+zBD)^{-1}BD+(1+zDB)^{-1}DB\right]=
               $$
               $$
              \sum_kz^k\left[\Tr(BD)^{k+1}+\Tr(DB)^{k+1}\right]=0\,\Rightarrow
              \det(1+zBD)\det(1+zDB)\equiv 1\,.
               $$
\m

  {\bf 3.}. If $\B$ and $\D$ are $n\times n$ matrices such that  $\B$ is symmetrical matrix and $\D$ is antisymmetrical matrix ($\B_{ik}=\B_{ki}$, $\D_{ik}=-\D_{ki}$) then
                       $$
                       \det(1+{\cal B}{\cal D})=1\,\,,
                       \eqno (3)
                       $$
              This is very important identity\footnote{$^*$}{In particular it follows form this identity
               that square root of Berezinian (superdeterminant) of linear canonical transformation is equal to the
               determinant of its boson-boson sector}.


               This identity can be considered as a special case of identity (2).
                But it can be proved independently.
             Indeed to prove the identity (2) it is enough to show that
                         $$
                     \Tr \left({\cal B}{\cal D}\right)^k=0\,,
                     \eqno (4)
                       $$
           since in the same way as above  the relation (4) implies that characteristic polynomial
           $\det(1+z{\cal B}{\cal D})$ equals to $1$.
           We have:
              $$
  \Tr (\B\D)^k=-\Tr(\D\B)^k=\Tr(\D^+\B^+)^k=-\Tr\left((\B\D)^+\right)^k=-\Tr(\B\D)^k.            
              $$
      Hence we come to (4)\finish        


           \bye 