

\magnification =1200

\baselineskip=14pt


{\it 14 May 2016}

  We say that $L^{ab}$ is $\pmatrix {0\cr2\cr}$ tensor of type $t=0,1$
  if under changing of coordinates it transforms as
           $$
    L^{ab}=(-1)^Z x^a_m L^{mn} x^b_n\,,\quad
   Z=Z(a,m)=am+(p(L)+t)a+(p(L)+t+1)m
          \eqno (1)
           $$
Sure to see that this is well-defined
one has to check that we do not fail in contradiction
changing coordinates (we have to consider three coordinate systems... )

\medskip


  {\bf Fact}
   If  tensor $L$ is of the type $t=0$ then the decomposition
         $$
       L^{ab}=\underbrace
    {{1\over 2}(L^{ab}+(-1)^{ab}L^{ba})}_{\hbox{symmetric}}+
    \underbrace{{1\over 2}(L^{ab}-(-1)^{ab}L^{ba})}_{\hbox{antisymmetric}}
         $$
is invariant under changing of coordinates.

In other words if type $t=0$ tensor $L^{ab}$ is symmetric 
in given coordinates 
                   $$
        L^{ab}=(-1)^{ab}L^{ba}
                  $$
then it
remains symmetric in all coordinate systems, 
and the same for antisymmetric.


Respectively


  If  tensor $L$ is of the type $t=1$ then the decomposition
         $$
       L^{ab}=\underbrace
    {{1\over 2}(L^{ab}-(-1)^{(a+1)(b+1)}L^{ba})}_{\hbox{``symmetric''}}+
    \underbrace{{1\over 2}(L^{ab}+(-1)^{(a+1)(b+1)}L^{ba})}_
         {\hbox{``antisymmetric''}}+
         $$
is invariant under changing of coordinates.

In other words if type $t=1$ tensor $L^{ab}$ is 
  sheefted antisymmetric=``symmetric'' 
in given coordinates
          $$
        L^{ab}=-(-1)^{(a+1)(b+1}L^{ba}=(-1)^{ab+a+b}L^{ba}
           $$
then it
remains sheefted antisymmetric=``symmetric'' 
in all coordinate systems, and the same for ``antisymmetric''.


  \medskip
{\bf Fact 2.}
  If tensor  $L^{ab}$ belongs to class $t$ then tensor
     ${\cal L}^{ab}=(-1)^aL^{ab}$ belongs to the class $t+1$.


\medskip


  One can formulate the analogous statements for covaraint tensors:

\medskip
  

  We say that $E_{ab}$ is $\pmatrix {2\cr 0\cr}$ tensor of type $t=0,1$
  if under changing of coordinates it transforms as
           $$
    E_{ab}=(-1)^Z x^m_aE_{mn} x^n_b\,,\quad
   Z=Z(a,m)=am+(p(E)+t)a+(p(E)+t+1)m
          \eqno (4)
           $$

  {\bf Fact 1'}
 If tensor $E_{ab}$ is of type $t=0$, and it  is symmetric 
in given coordinates 
                   $$
        E_{ab}=(-1)^{ab}E_{ba}
                  $$
then it
remains symmetric in all coordinate systems, and the same for antisymmetric.


Respectively


  If  tensor $E$ is of the type $t=1$, and it is 
  sheefted antisymmetric=``symmetric'' 
in given coordinates
          $$
        E_{ab}=-(-1)^{(a+1)(b+1)}(-1)^{ab+a+b}E_{ba}
           $$
then it
remains ``symmetric'' in all coordinate systems, and the 
same for antisymmetric.


   The transformation  $E_{ab}\to (-1)^a E_{ab}$
chanes the type of the tensor.

  \medskip


And finally the 
{\bf Fact 3}


  If tensor  $L^{ab}$ belongs to class $t$ 
and it is invertible, then the inverse tensor belongs to the class
   $t+p(L)$.
   Symmetrical tensor goes to symmetrical
if it is even and to shifted symmetrical if it is odd)
    
In other words 
the tensor $E^{ab}$ defining odd symplectic structure 
is an odd tensor of type $t=0$. Its inverse is an odd ``antisymmetrical'' 
tensor
of the type $t=1$


$$ $$

  Now revenons a nos moutons.


  In our considerations we in fact cosnidered for Riemannian and 
sympelcti structure the tensor which looked the same,
but the type was different!!!



We can  to define Riemannian metric by the tensor
$\pmatrix {0 & I\cr -I& 0 }$ of the type $t=1$


You was right when you told that
 `you believe to formulae and do not believe to concrete
  expression $\pmatrix {0 & I\cr I& 0 }$.

For $\kappa_{ab}$ we may take  ``symmetrical''=shifted antisymmetrical
tensor   $\pmatrix {0 & I\cr -I& 0 }$ of the type $t=1$
and using transformation $\kappa\to (-1)^a\kappa $ make from this
the tensor  $\pmatrix {0 & I\cr I& 0 }$ of the type $t=0$.


  We come to the fact that both
symplectic and Riemannian structure are defined by the tensor
which looks the same in special coordinates.
For Riemannian structure it is the tensor of the type $t=0$
and for ymplectic structure it is the tensor of the type $t=1$.


  Good?
 

\bye



