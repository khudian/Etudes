Nijenhuis 
\magnification=1200 

\baselineskip=14pt


\def\vare {\varepsilon}
\def\A {{\bf A}}
\def\B {{\bf B}}
\def\t {\tilde}
\def\a {\alpha}
\def\K {{\bf K}}
\def\N {{\bf N}}
\def\V {{\cal V}}
\def\L {{\cal L}}
\def\s {{\sigma}}
\def\S {{\Sigma}}
\def\s {{\sigma}}
\def\p{\partial}
\def\vare{{\varepsilon}}
\def\Q {{\bf Q}}
\def\O {{\bf O}}
\def\D {{\cal D}}
\def\G {{\Gamma}}
\def\C {{\bf C}}
\def\N {{\cal N}}
\def\Z {{\bf Z}}
\def\U  {{\cal U}}
\def\H {{\cal H}}
\def\R  {{\bf R}}
\def\S  {{\bf S}}
\def\E  {{\bf E}}
\def\l {\lambda}
\def\degree {{\bf {\rm degree}\,\,}}
\def \finish {${\,\,\vrule height1mm depth2mm width 8pt}$}
\def \m {\medskip}
\def\p {\partial}
\def\r {{\bf r}}
\def\v {{\bf v}}
\def\n {{\bf n}}
\def\t {{\bf t}}
\def\b {{\bf b}}
\def\c {{\bf c }}
\def\e{{\bf e}}
\def\ac {{\bf a}}
\def \X   {{\bf X}}
\def \Y   {{\bf Y}}
\def \x   {{\bf x}}
\def \y   {{\bf y}}
\def \G{{\cal G}}
\def\w{\omega}
\def\finish {${\,\,\vrule height1mm depth2mm width 8pt}$}


\centerline {\bf Nijenhuis bracket of differential 
     forms valued vector fields}


{\it Let $A(x)$ be a linear operator on tangent vectors: 
$A(x)\colon\quad T_xM\to T_x(M)$. Then one can define $[A,A]$
 which is linear operator from $T_xM\wedge T_x M\to T_xM$. 
This is a special case
of Nijenhuis bracket. We first do it in straightforward way, then
  comne to this formula using general formalism.}

  Let  $A(x)$ be an operator-valued function on manifold $M$. Consider
   the following function on vector fields:
                      $$
    \X,\Y\mapsto
 \N(\X,\Y)=         \left[L(\X),\,L(\Y)\right]-
                    L\left(\left[\X,\,L(\Y)\right]\right)-
                   L\left(\left[L(\X),\,\Y\right]\right)+
                      L\left(
                    L\left(
              \left[\X,\,\Y\right]
                    \right)
                     \right)\,.
                      $$
where $[\,\,,\,\,]$ is commutator of vector fields:
  $[\A,\B]=[\A,\B]^i\p_i=(A^r\p_r B^i-B^r\p_r A^i)\p_i$.

   $\N(\X,\Y)$ is vector field on $M$.  One can see  which that:
              $$
 \N(\X,\Y)=-\N(\Y,\X)
              $$
 {\bf Fact}   $\N(\X,\Y)$ is not only linear over vector fields, it is
linear over  algebra of functions on $M$:
  in particular for arbitrary function $f$
               $$
\N(f\X,\Y)=f\N(\X,\Y)\,,
               $$
(this implies linearity over functions),

\m

This statement means that at every point $x_0$,
 $\N$ is linear function
on vectors $\X,\Y$ tangent to $M$ at this point. 


 Show it,  Note that 
  $[f\X,Y]=f[\X,\Y]-(\Y f)\X$. Hence
         $$
  \N(f\X,\Y)=\left[L(f\X),\,L(\Y)\right]-
                    L\left(\left[f\X,\,L(\Y)\right]\right)-
                   L\left(\left[L(f\X),\,\Y\right]\right)+
                      L\left(
                    L\left(
              \left[f\X,\,\Y\right]
                    \right)
                     \right)
=
               $$ 
               $$
          f\left[L(\X),\,L(\Y)\right]-\left(L(\Y)f\right)L(\X)
      -fL\left(\left[\X,\,L(\Y)\right]\right)+\left(L(\Y)f\right)L(\X)+
            $$
            $$
          -fL\left(
             \left[L(\X),\Y\right]
                    \right)+
                \left(\Y f\right)L\left(L\left(\X\right)\right)
                +f L\left(
                    L\left(
              \left[\X,\,\Y\right]
                    \right)
                     \right)-
     \left(\Y f\right)L\left(L\left(\X\right)\right)
             =f\N(\X,\Y)\,.
                    $$
In components
        $$
\N(\X,\Y)=N^i_{kp}X^kY^p, 
        $$
where $$
N^m_{kp}\p_m=\N(\p_k,\p_p)=\left[L(\p_k),\,L(\p_p)\right]-
                    L\left(
              \left[\p_k,\,L(\p_p)\right]-\left[\p_p,\,L(\p_k)\right]
                    \right)+
                      L\left(
                    L\left(
              \left[\p_k,\,\p_p\right]
                    \right)
                     \right)=
           $$
            $$
     L^i_k\p_i L^m_p-L^i_p\p_i L^m_k-
      L^m_r\left(\p_kL_p^r-\p_p L_k^r\right)
           $$
{\bf Theorem (Neveinhuisen)} Operator valued function $L(x)=$
  vector valued differential $1$-form  defines
  vector valued differential $2$-form:
  $$
  L\colon L=dx^kL_k^i\p_\mapsto [L,L]=dx^p\wedge dx^k
      \left(L^i_k\p_i L^m_p-
      L^m_r\p_kL_p^r
\right)\,. 
            $$ 
  This is  bracket of $L$ with itself.
In fact Nijenhuis defines bracket for vector fields valued 
in differential forms of arbitrary rank.
  We will describe them using supermathematics.

     \centerline  {\bf General approach}

  For  manifold $M$ of dimension $n$ consider $n|n$-dimensional
supermanifold $\Pi TM$, i.e. nothing that tangent bundle $TM$ with
changing parity of fibers.

    Note that usual $k$-form on $M$ 
$dx^{i_1}\wedge\dots\wedge dx^{i_k}\w_{i_1\dots i_k}$ defines a function
            $$
    \w(x,dx)=dx^{i_1}\dots dx^{i_k}\w_{i_1\dots i_k}(x)
            $$
 which is even if $k$ is even and odd if $k$ is odd
($dx^i$ are odd variables: $dx^idx^j=-dx^jdx^i$.)

  Differential form-valued vector field  $\X(x,dx)=X^i(x,dx)\p_i$ 
is nothing but
 vector field on $\Pi TM$ such that its vertical components
vanishes. Sure the vanishing of vertical components is not
covariant condition.  Better to say, that form-valued vector field
defines differentiation of functions on $M$ with values in differential
forms. One can  define canonical lifting of this object on 
vector field on whole $\Pi TM$:
                 $$
\X(x,dx)=X^i(x,dx)\p_i\mapsto \widehat{\X(x,dx)}=
 X^i(x,dx)\p_i+(-1)^{p(\X)}dx^k\p_k X^i(x,dx){\p\over \p dx^i}
                 $$ 
$p(\X)=0$ if it values in forms of even rank and it is equal to $1$
if it values in forms of an odd rank.

 This canonical lifting is uniquely defined by the condition that 
veector field $\widehat \X$ commutes wioth de Rham differential $d=dx^k{\p_k}$.


  {{\bf Definition} of Nijenuis bracket}
Let $\A=A^i(x,dx)\p_i, \B=B^i(x,dx)\p_i$ 
be vector valued differential forms
Then
         $$
  [\A,\B]_{\rm N}\colon\,\,
   \widehat {[\A,\B]}_{\rm N}=[\hat \A,\hat \B]\,,
         $$
where on the right hand side is usual commutato of vector fields.
 This is all\finish
\bye
