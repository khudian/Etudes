
   \magnification=1200
   \baselineskip 14 pt
%  21 November 2012.
%I began this file on the base of the previous file caonisomorph1.tex

\def\V {{\cal V}}
\def\s {{\sigma}}
\def\Q {{\bf Q}}
\def\D {{\cal D}}
\def\G {{\Gamma}}
\def\C {{\bf C}}
\def\M {{\cal M}}
\def\Z {{\bf Z}}
\def\U  {{\cal U}}
\def\H {{\cal H}}
\def\R  {{\bf R}}
\def\l {\lambda}
\def\w {\omega}
\def\p {\partial}
\def\r {{\bf r}}
\def\v {{\bf v}}
\def\n {{\bf n}}
\def\t  {\tilde}
\def\b {{\bf b}}
\def\ac {{\bf a}}
\def \X   {{\bf X}}
\def \Y   {{\bf Y}}
\def \E   {{\bf E}}
\def \N   {{\bf N}}
\def \finish {${\,\,\vrule height1mm depth2mm width 8pt}$}

\centerline {\bf On canonical isomorphisms $T^*E=T^*E^*$ for vector bundle $E$}



  Kirill Mackenzie told and explained many many times the construction of
remarkable isomorphism between cotangent bundles of 
vector bundle and its dual: for an arbitrary vector bundle $E$
 $T^*E=T^*E^*$,
where $E^*$ is a bundle dula to $E$.

The first thing that you try to apply this construction to it is to consider
  tangent bundle: $E=TM$ or cotangent bundle $E=T^*M$ and consider  
  isomorphism $T^*TM=T^*T^*M$.
On the other hand in this special case
the canonical symplectic structure on the cotangent bundle $T^*M$ implies
the canonical isomorphism between tangent and cotangent bundles of
the manifold $T^*M$:
              $T^*T^*M=TT^*M$.
Hence in the special case of $E=TM$ the ''Mackenzie'' isomorphism 
combined with isomorphism  induced by cnonical symplectic structure 
implies the canonical isomorphisms $T^*TM=TT^*M$.

In this etude we would like to reconstruct 
the ''Mackenzie'' isomorphism $T^*E=T^*E^*$ 
and its special case, the isomorphism
  $TT^*M=T^*TM$ using pedestrian's arguments.  
 In the first section we conisder the special case
$E=TM$ and establish the isomorphism$TT^*M=T^*TM$.
In the second question we will establish  
the ''Mackenzie'' isomorphism $T^*E=T^*E^*$.

Our notations are little bit inconsistent: in the first paragraph we
denote coordintes by $x^i,....$ and new ones by
 $\tilde x^\mu,...$.
  In the second paragraph our notations are much more traditional:
indices of new coordinates are denoted by the same letters with ''prime'' indices($x^i\to x^{i'}$). 

\medskip

\centerline{ \bf Canonical isomorphism $TT^*M=T^*TM$}

  Let $M$ be manifold. Establish and study canonical isomorphisms $TT^*M=T^*TM=T^*T^*M$.

   Perform calculations in local coordinates.
It may sounds surprising but calculations in local coordinates are transparent and illuminating.


 First consider local coordinates on $TM$ and $T^*M$ corresponding to
local coordinates $(x^i)$ on $M$.
 Local coordinates for $TM$ are $(x^i,t^j)$: every vector $\r\in TM$ is a vector
  $t^i{\p\over \p x^i}$, $t^i(\r)=dx^i(\r)$.
         If   $\tilde x^\mu=\tilde x^\mu(x^i)$  are new local coordinates  on $M$
then
                  $$
            d\t x^\mu\left(t^i{\p\over \p x^i}\right)={\p \tilde x^\mu(x^i)\over \p x^i}dx^i
            \left(t^i{\p\over \p x^i}\right)={\p \tilde x^\mu(x^i)\over \p x^i}t^i\,.
              \eqno(1.1)
                  $$
Hence changing of local coordinates in $TM$ is
                    $$
                    (x^i,t^j)\mapsto (\t x^\mu, \t t^\nu),\quad \t x^\mu=\t x^\mu(x^i),\,\,
                 \tilde t^\mu=\pmatrix{\mu\cr i\cr}t^i,
                 \eqno(1.2)
                    $$
where we denote ${\p \tilde x^\mu(x^i)\over \p x^i}$ by  $\pmatrix{\mu\cr i\cr}$.



  Respectively local coordinates for $T^*M$ are $(x^i,p_j)$. For every $1$-form $\w\in T^*M$
  $p_i=\w\left({\p\over \p x^i}\right)$. Under changing of local coordinates on $M$
             $\tilde x^\mu=\tilde x^\mu(x^i)$, coordinates $(p_i)$ change to new coordinates $(p_\mu)$:
                           $$
p_\mu=w\left({\p\over \p \t x^\mu}\right)=
\w\left({\p x^i(\tilde x^\mu)\over \p \tilde x^\mu }
{\p\over \p  x^i}\right)={\p x^i(\tilde x^\mu)\over \p \tilde x^\mu }p_i
                           $$
Hence changing of local coordinates in $T^*M$ is
                        $$
                        (x^i,p_k)\mapsto (\tilde x^\mu,\tilde p_\nu),
                        \quad \t x^\mu=\t x^\mu(x^i),\,\,
                           p_\mu=\pmatrix{i\cr \mu\cr}p_i,
                           \eqno (1.3)
                        $$
where we denote ${\p x^i(\tilde x^\mu)\over \p \tilde x^\mu }$ by  $\pmatrix{i\cr \mu\cr}$

   The space  $TT^*M$ is tangent space to the space $T^*M$. 
The local coordinates on $TT^*M$
corresponding to local coordinates $(x^i,p_j)$
on $T^*M$ are coordinates $(x^i,p_j;\xi^k,\rho_m)$; $\xi^k=dx^i(\r),\rho_m=dp_m(\r)$.
Under changing of local coordinates  $(x^i)$ to coordinates 
$\t x^\mu=\tilde x^\mu(x^i)$
coordinates $(\xi^i)$ and $(\rho_m)$ transform to new coordinates 
$(\t \xi^\mu)$ and $(\t \rho_\nu)$ respectively.
It follows from (1.1) that
                       $$
                    \t\xi^\mu={\p \t x^\mu\over \p x^i}\xi^i+{\p \t x^\mu\over \p p_i}\rho_i=\pmatrix{\mu\cr i\cr}\xi^i\,.
                       $$
  because  ${\p \t x^\mu\over \p p_i}=0$.
  For transformation of coordinates $(\rho_m)$ calculations are longer:
            $$
  \t\rho_\mu={\p \t p_\mu\over \p x^i}\xi^i+
   {\p \t p_\mu\over \p p_i}\rho_i\,.
            $$
We see that   ${\p \t p_\mu\over \p p_i}={\p\over \p p_i}
\left(\t p_\mu=\pmatrix {k\cr \mu\cr}p_k\right)=\pmatrix {i\cr \mu\cr}$ and
                           $$
{\p \t p_\mu\over \p x^i}={\p\over \p x^i}\left(\t p_\mu=\pmatrix {k\cr \mu\cr}p_k\right)=
 \pmatrix {\nu\cr i\cr} \pmatrix {k\cr\nu \mu\cr}p_k,
                           $$
where we denote as always by  $\pmatrix {\nu\cr i\cr}$ the partial derivative ${\p \t x^\mu\over \p x^i}$
and by $\pmatrix {k\cr\nu \mu\cr}$ the partial derivative ${\p^2 x^k\over \p \t x^\nu \p \t x^\mu}$.
The summation over repeated indices is assumed.  Finally we come to
                  $$
  \t\rho_\mu={\p \t p_\mu\over \p x^i}\xi^i+{\p \t p_\mu\over \p p_i}\rho_i=
  \xi^i\pmatrix {\nu\cr i\cr} \pmatrix {k\cr\nu \mu\cr}p_k+\pmatrix {i\cr \mu\cr}\rho_i\,.
                  $$


     Summarising:

    {\bf Proposition 1} {\it  Let $(x^i,p_j;\xi^k,\rho_m)$ be local coordiantes
on $TT^*M$ described above.  
    Under changing of coordinates $(x^i)\mapsto (\t x^\mu)$ on $M$ 
these coordinates transform in the following way}
                                $$
            \t p_\mu=\pmatrix{j\cr\mu\cr}p_j, \quad
            \t\xi^\mu=\pmatrix{\mu\cr i\cr}\xi^i,\quad
            \t\rho_\mu=
            \xi^i\pmatrix {\nu\cr i\cr} \pmatrix {k\cr\nu \mu\cr}p_k+
\pmatrix {i\cr \mu\cr}\rho_i\,.
\eqno (*)
                                $$
\bigskip
Now consider analogously coordinates  on $T^*TM$ and their 
transformation rules.
 If $(x,t)$ coordinates on $TM$ (see (1)) and $(x,t,\pi,\tau)$ corresponding coordinates on $T^*TM$
 ($\pi_k=\w\left({\p\over \p x^k}\right)$,
 $\tau_m=\w\left({\p\over \p t^m}\right)$) then according to (2)
 under changing of coordinates $(x^i)\mapsto (\t x^\mu)$, the coordinates $(\pi_m)$ transform to coordinates
 $(\t \pi_\mu)$, the coordinates $(\tau_k)$ transform to coordinates
 $(\t \tau_\nu)$ such that
                  $$
\t\pi_\mu={\p x^i\over \p \t x^\mu}\pi_i+{\p t^k\over \p \t x^\mu}\tau_k,\,\,\,\,
\t\tau_\nu={\p x^i\over \p \t t^\nu}\pi_i+{\p t^k\over \p \t t^\nu}\tau_k
                  $$
Since ${\p x^i\over \p \t t^\nu}=0$
and  ${\p t^k\over \p \t t^\nu}={\p x^k\over \p \t x^\mu}$
then $$\t\tau_\nu=\pmatrix {k\cr\nu}\tau_k$$.  For $\t\pi_\mu$ we have
                  $$
\t\pi_\mu=\pmatrix {i\cr\mu}\pi_i+{\p\over \p \t x^\mu}\left({\p x^k\over \p\t x^\nu}t^\nu\right)\tau_k=
\pmatrix {i\cr\mu}\pi_i+
\pmatrix {k\cr\mu\nu}\pmatrix {\nu\cr i}t^i\tau_k\,.
                  $$

     Summarising:

    {\bf Proposition 2} {\it Let $(x^i,t^j;\pi^k,\tau^m)$ be 
     local coordinates
on $T^*TM$ described above.  
 Under changing of coordinates $(x^i)\mapsto (\t x^\mu)$ on $M$ 
these coordinates transform in the following way}
                                $$
            \t \tau_\mu=\pmatrix{j\cr\mu\cr}\tau_j, \quad
            \t t^\mu=\pmatrix{\mu\cr i\cr}t^i,\quad
            \t\pi_\mu=
            t^i\pmatrix {\nu\cr i\cr} \pmatrix {k\cr\nu \mu\cr}\tau_k+\pmatrix {i\cr \mu\cr}\pi_i
            \eqno (**)
                                $$


Comparing  Propositions 1 and 2 we come to 

{\bf Observation }

  As it was described above  
assign to local coordiantes $x^i$ on a manifold $M$ local coordinates
$(x^i,t^j;\pi_k,\tau_j)$ on $T^*TM$ and local coordiantes
 $(x^i,p_i;\xi^k,\rho_m)$ on $T^*TM$  
 which are described above. The map
                         $$
           t^i=\xi^i,\,\,\, \tau_j=p_j,\,\,\, \pi_k=\rho_k
                      \eqno (1.***)
                      $$
 establishes the isomorphism between the spaces $T^*TM$ and $TT^*M$ 
which does not depend on
             the choice of local coordinates $x^i$ on $M$\footnote{$^*$}
{In fact one can consider the {\it pencil}  of maps 
          $
 t^i={\ac}\xi^i,\,\,\, \tau_j={\bf b}p_j,\,\,\, \pi_k={\bf ab}\rho_k
           $
where ${\bf a,b}\not=0$.}.


   Note that canonical symplectic structure
$\Omega=dp_i\wedge dx^i$ establishes canonical 
isomorphism  between spaces $TT^*M$ and $T^*T^*M$:
                     $$
  dx^i\leftrightarrow {\p\over \p p_i}\,,\quad
dp_i\leftrightarrow -{\p \over \p x^i }
                    $$ 
Combining this isomorphism with isomorphism (1.$***$) we come to the  
isomoprphism $T^*TM=T^*T^*M$ which looks like
                            $$
            (x^i,t^i,\pi_k,-p_i)\leftrightarrow (x^i,p_i;\pi_k,t^i)
                            $$ 
Of course there is two-parametric freedom (see the footnote).

\medskip

\centerline  {${\cal x}2 $ \bf 
 Canoncial isomorphism in general case: $T^*E=T^*E^*$ }


 In the previous paragraph we constructed canonical isomorphism 
$T^*E=T^*E^*$ in the case where vector bundle $E$ is tangent (cotangent bundle).

  Now consdier general case. Let $(x^\mu,s^i)$ be local coordiantes on 
bundle $E$.  Let under changing of coordinates
   $x^{\mu'}=x^{\mu'}(x)$  fibre coordiantes $s^i$ 
transform in the following way: $s^{i'}=\Psi^{i'}_k(x)s^k$. Then
Respectively coordinates $s_k$ of dual fibre will transform as
                       $s_{k'}=\Phi_{k'}^is_i$,
  where transition matrices $\Psi$ and $\Phi$ are inverse to each other: 
 $\Psi^{i'}_k\Phi^k_{j'}=\delta^{i'}_{j'}$.

 Let $(x^\mu,s^i; \rho_\mu,\pi_i)$ be local coordinates in $T^*E$ and respectively let $(x^\mu,s_i; \zeta_\mu,t^i)$ be local coordinates in $T^*E^*$. 
  Reccall that as usual we suppose that 
$1$-form  $\w\in T^*E$ has local coordinates 
$(x^\mu,s^i;\rho_\mu,\pi_i)$ if it is the function
on vectors tangent to the manifold $E$ at the point $(x^\mu, s^i)$ such that
              $$
     \w\left({\p\over \p x^\mu}\right) =\rho_\mu,\quad 
        \w\left({\p\over \p s^i}\right)=\pi_i
             $$
  Respectively we suppose that $1$-form 
$\w\in T^*E^*$ has local
coordinates $(x^\mu,s_i;\zeta_\mu,t_i)$ if it is the function
on vectors tangent to the manifold $E^*$ at the point 
$(x^\mu, s_i)$ such that
              $$
     \w\left({\p\over \p x^\mu}\right)=\zeta_\mu, 
\quad \w\left({\p\over \p s_i}\right)=t_i\,.
             $$

 Write down transformation for fields under coordiante trasnformation $x^{\mu'}=x^{\mu'}(x^\mu)$. We have
                    $$
                    \cases
                    {
               s^{i'}=\Psi^{i'}_k(x)s^k\cr
            \rho_{\mu'}={\p x^\mu\over \p x^{\mu'}}\rho_\mu+
            {\p s^i\over\p x^{\mu'}}\pi_i\cr
                   \pi_{i'}={\p x^\mu\over \p s^{i'}}\rho_\mu+
          {\p s^i\over \p s^{i'}}\pi_i\cr
                      },\qquad
                    \cases
                    {
               s_{i'}=\Phi_{i'}^k(x)s_k\cr
            \zeta_{\mu'}={\p x^\mu\over \p x^{\mu'}}\zeta_\mu+
           {\p s_i\over\p x^{\mu'}}t^i \cr
                   t^{i'}={\p x^\mu\over \p s_{i'}}\zeta_\mu+
         {\p s_i\over \p s_{i'}}t^i\cr
                      }
                      $$
Note that

                     $$
    {\p x^\mu\over \p s^{i'}}={\p x^\mu\over \p s_{i'}}=0,\,\,  
      {\p s^i\over \p s^{i'}}=\Phi^i_{i'},\,\,
    {\p s_i\over \p s_{i'}}=\Psi^{i'}_i\,,     
                     $$
and
                    $$
{\p s^i\over \p x^{\mu'}}\pi_i={\p \Phi^i_{i'}\over \p x^{\mu'}}s^{i'}\pi_i=
   {\p x^\mu\over \p x^{\mu'}}{\p \Phi^i_{i'}\over \p x^{\mu}}s^{i'}\pi_i=
   {\p x^\mu\over \p x^{\mu'}}
     {\p \Phi^i_{i'}\over \p x^{\mu}}
        \Psi^{i'}_ks^{k}\pi_i\,,
             \eqno (2.2a)
                   $$
                  $$
{\p s_i\over \p x^{\mu'}}t^i=
 {\p \Psi_i^{i'}\over \p x^{\mu'}}s_{i'}t^i=
   {\p x^\mu\over \p x^{\mu'}}{\p \Psi_i^{i'}\over \p x^{\mu}}
         s_{i'}t^i=
   {\p x^\mu\over \p x^{\mu}}{\p \Psi_i^{i'}\over \p x^{\mu}}
        \Phi_{i'}^ks_{k}t^i\,.
             \eqno (2.2b)
                   $$
Introducing  $L_{\mu k}^i$ such that
                 $$
         L_{\mu k}^i=
       {\p \Psi_k^{i'}\over \p x^{\mu}}
        \Phi_{i'}^i=
            -
       {\p \Phi^i_{i'}\over \p x^{\mu}}
        \Psi^{i'}_k
             ,\,\, (\Psi\circ \Phi=1)
                 $$
we see that the 
transformations (2.1) have the following nice appearance:
                   $$
                   \cases
                    {
               s^{i'}=\Psi^{i'}_k(x)s^k\cr
            \rho_{\mu'}={\p x^\mu\over \p x^{\mu'}}
                     \left(
                      \rho_\mu-
                     L_{\mu k}^is^k\pi_i
                      \right)
                      \cr
                   \pi_{i'}=\Phi^i_{i'}\pi_i\cr
                      },
                    \qquad
                    \cases
                    {
               s_{i'}=\Phi_{i'}^k(x)s_k\cr
 \zeta_{\mu'}={\p x^\mu\over \p x^{\mu'}}
        \left(
           \zeta_\mu+
               L_{\mu k}^it^k s_i
               \right)\cr
    t^{i'}=\Psi^{i'}_it^i\cr
                      }
\eqno (2.3)
                      $$
Put
         $$
       \cases
         {
      \pi_i=\alpha s_i\cr
      t^i=\beta s^i\cr
      \zeta_\mu=\gamma \rho_\mu\cr
        }
         $$
  We see that this map is invariant with respect to changing of coordinates
if $\beta=-\gamma\alpha$. In particular we can put $\alpha=-1$, $\beta=\gamma=1$
We come to isomorphism $T^*E=T^*E^*$ defined in local coordinates by condition
that
             $$
    (x^\mu,s^i;\rho_\mu,\pi_i)\leftrightarrow(x^\mu,s_i;\zeta_\mu,t^i),
   \quad\hbox{such that}\,\,\rho_\mu=\zeta_\mu,
      \pi_i=-s_i,t^i=s^i
             $$ 
(Compare with (1.***))

We constructed the isomorphism \finish
\bye
