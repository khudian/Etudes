

\magnification=1200
\baselineskip=14pt


\def\vare {\varepsilon}
\def\A {{\bf A}}
\def\B {{\bf B}}
\def\t {\tilde}
\def\a {\alpha}
\def\K {{\bf K}}
\def\k {{\bf k}}
\def\x {{\bf x}}
\def\y {{\bf y}}
\def\V {{\cal V}}
\def\L {{\cal L}}
\def\s {{\sigma}}
\def\S {{\Sigma}}
\def\s {{\sigma}}
\def\p{\partial}
\def\vare{{\varepsilon}}
\def\Q {{\bf Q}}
\def\O {{\bf O}}
\def\D {{\cal D}}
\def\G {{\Gamma}}
\def\C {{\bf C}}
\def\N {{\cal N}}
\def\Z {{\bf Z}}
\def\E {{\bf E}}
\def\U  {{\cal U}}
\def\H {{\cal H}}
\def\R  {{\bf R}}
\def\S  {{\bf S}}
\def\E  {{\bf E}}
\def\l {\lambda}
\def\degree {{\bf {\rm degree}\,\,}}
\def \finish {${\,\,\vrule height1mm depth2mm width 8pt}$}
\def \m {\medskip}
\def\p {\partial}
\def\r {{\bf r}}
\def\v {{\bf v}}
\def\n {{\bf n}}
\def\t {{\bf t}}
\def\b {{\bf b}}
\def\c {{\bf c }}
\def\e{{\bf e}}
\def\ac {{\bf a}}
\def \X   {{\bf X}}
\def \Y   {{\bf Y}}
\def \x   {{\bf x}}
\def \y   {{\bf y}}
\def \G{{\cal G}}
\def\w{\omega}
\def\finish {${\,\,\vrule height1mm depth2mm width 8pt}$}


  \centerline  {\bf Huigens principle}
  ({Here I will present my calculations based on memories and textbooks...})

   Consider in $\E^n$ differential equation
         $$
       \cases
          {
         u_{tt}-\Delta u=0\cr
           u(t,\x)\big\vert_{t=0}=\varphi(\x)\cr
           {\p u(t,\x)\over \p t}\big\vert_{t=0}=\psi(x)\cr
            }
\eqno (1)
           $$
   Using Fourrier transformation
 one can see that formal solution in Fourrier series will be
              $$
     C_n\int e^{i\k (\x-\y)}\left(\varphi(y)\cos kt+
       \psi(y){\sin kt\over k}\right)d^n\k d^n \y\,, (),
           \eqno (1a)
              $$
where $C_n={2\pi}^{-{n\over 2}}$, $\k,\x$ are vectors, $k$ is modulus
of vector $\k\,$, $k=|\k|$,
where coefficients $A(\k), B(\k)$

 
  {\it Symmetry of solutions with respect to initial data}

Let $u=u^{\{\varphi,\psi\}}$ be the solution (1a) of wave
 equation with boundary conditions (1). Then it
is easy to see that for function $v=u_t$
       $$
  v=u^{\{\varphi',\psi'\}} \,\, {\rm with}\,\,\varphi'=\psi\,,{\rm and}\,\,
          \psi'=\Delta\varphi\,.
      \eqno (2a)
       $$
since
     $$
v\big\vert_{t=0}=u_t\big\vert_{t=0}\,,{\rm and}\,\,
v_t\big\vert_{t=0}=u_{tt}\big\vert_{t=0}=\Delta u\big\vert_{t=0}=
\Delta\varphi\,.
   \eqno (2b)
     $$
In particular if the function $u$ is the solution of wave equation 
with boundary conditions 
$u(t,\x)\big\vert_{t=0}=0$ and
           ${\p u(t,\x)\over \p t}\big\vert_{t=0}=\varphi(x)$,
then the function $v=u_t$ is the solution of wave equation 
with boundary conditions 
$v(t,\x)\big\vert_{t=0}=\varphi$ and
           ${\p v(t,\x)\over \p t}\big\vert_{t=0}=0$, and
       $$
 u^{\{\varphi,\psi\}}=
u^{\{0,\psi\}}+u_t^{\{0,\varphi\}}.
\eqno (2c)
       $$
  
\smallskip

   We calculate this integral, show that
for odd $n$ it implies Huigens and try to reveal some geometrical reasons
of this fact based on the article of Roger Howe
\footnote{$^*$}
{Roger Howe. On the role of the Heisenberg group in harmonic analysis.
 {Bulletin (New Series) OF THE AMERICAN MATHEMATICAL SOCIETY, Volume 3, Number 2, September 1980}}.
   \medskip


  Preliminary calculation:  Calculate preliminary the average
of the function   $e^{i\k(\x-y)}$ over unit $n-1$-dimensional 
 sphere of the radius $k$.
 
 Denote by $\s_n$ area of $n$-dimensional unit sphere:
         $$
 \s_0=2,\s_1=2\pi\,,\s_2=4\pi\,\dots,  
\s_n={2\pi^{n+1\over 2}\over \Gamma\left(n+1\over 2\right)}\,.
    \eqno (3)
                $$
(It is funny to note that volume of $0$-dimensional sphere $\s_0=2$ 
is given by the general formula.)

   Function $\k\x$  is not constant on $n-1$ dimensional sphere $kx=1$,
but it is constant on $n-2$ dimensional spheres $\k\x\cos\theta=c$
($\theta$ is angle between $\k$ and $\x$ and $|c|\leq 1$).
  Using this fact we calculate average of function $f(\k\x)=e^{i\k \x}$
over the sphere of radius $k$:
    $$
<f{\k \x}>_k
  ={1\over k^{n-1}\s_{n-1}}
  \int_0^\pi f(kx\cos\theta)\sigma_{n-2}(k\sin\theta)^{n-2}kd\theta=
                $$
                $$
   ={\sigma_{n-2}\over \s_{n-1}}\int_0^\pi f(kx\cos\theta)
           \sin\theta)^{n-3}d\sin\theta=
   {\sigma_{n-2}\over \s_{n-1}}\int_{-1}^1 f(kxu)
           (1-u^2)^{n-3\over 2}du
\eqno (3a)
      $$
  (If we put $f\equiv 1$ we come to the formula (3)
for volumes of sphere\footnote{$^*$}{Indeed we have
             $
1=<1>= {\sigma_{n-2}\over \s_{n-1}}\int_{-1}^1 
           (1-u^2)^{n-3\over 2}du\,.
             $. Hence
  ${\sigma_{n-2}\over \s_{n-1}}=\int_{-1}^1 
           (1-u^2)^{n-3\over 2}du=$
               $$
            \int_0^1 
           (1-x)^{n-3\over 2}x^{-{1\over 2}}dx=
                B\left({n-1\over 2},{1\over 2}\right)=
                          {
                \Gamma\left({n-1\over 2}\right)
               \Gamma\left({1\over 2}\right)
                      \over
               \Gamma\left({n\over 2}\right)
                                   }={
                \sqrt\pi\Gamma\left({n-1\over 2}\right)
                      \over
               \Gamma\left({n\over 2}\right)
                                   }\,.
                 $$
This implies the formula (3).

}
One can see that answers for even and odd will be different.
   For odd $n$ it is just elementary function, and 
   for even $n$ they are expressed via special function
   $$
J(x)=\int_0^\pi e^{ix\cos\theta}d\theta\,
  \eqno (4)
  $$


   
In more details.
 First consider special cases
(we often omit later all the coefficients....)
         $$
n=2, J(x)=F_2(x)=\s_0\int_0^{\pi} e^{ix\cos\varphi}d\varphi=
        \int_0^{2\pi} e^{ix\cos\varphi}d\varphi\,,
     \eqno (5a)
         $$ 
         $$
n=3, F_3(x)=\s_1\int_0^{\pi} e^{ix\cos\varphi}\sin \varphi d\varphi=
        2\pi\int_{-1}^{1} e^{ix u}du=2i{\sin x\over x}\,.
         \eqno (5b)
         $$ 
Now one can  see that the answer for $n=2$ 
produces all the answers
for even $n$ and the answer for $n=3$ produces all the answers for
odd $n$: all fnctions $F_n(x)$ can be produced from function
$J(a)$ for even $n$ and function $f(a)={\sin a\over a}$ 
by differentiation, e,g,
      $$
F_6(x)=\int e^{ix\cos\theta}\sin^4\theta d\theta=
\int e^{ix\cos\theta}(1-\cos^2\theta)^2 d\theta=
\left(1+{d^2\over dx^2}\right)^2J(x)\,
     \eqno (6a)
      $$
                     $$
      F_7(x)=
   \int_0^\pi e^{ix\cos\theta}\sin^{5}\theta d\theta=
  \int_0^\pi e^{ix\cos\theta}\sin^{4}\theta d\cos\theta=
   \int_0^\pi e^{ix u}(1-u^2)^2du=
\eqno (6b)
         $$
            $$
\left(1-\left(i {d\over dx}\right)^2\right)^2
        \int_0^\pi e^{ix u}du=
\left(1+{d^2\over dx^2}\right)^2
        {\sin x\over x}\,.
          \eqno (6b)
           $$
(We often omit coefficients proportional to volumes of spheres.)

One can see that in general case
{\bf PROPOSITION}

$F_n(x)=P\left({d\over dx}\right)J(x)$ for even $n$ and
 $F_n(x)=P\left({d\over dx}\right){\sin x\over x}$. 


Now we return to the integral (*). 
Using functions $F_n(a)$  which are averaging
of exponent over spere we come to
               $$
  u(t,\x)=C_n\int e^{i\k (\x-\y)}\left(\varphi(y)\cos kt+
       \psi(y){\sin kt\over k}\right)d^n\k d^n\y\,.
           =
           $$
           $$
  =\int F_n(k|\x-\y|)\left(\varphi(y)\cos kt+
       \psi(y){\sin kt\over k}\right)k^{n-1}dk d^n \y=
     \eqno (7)
              $$
We denote
             $$
     G^{(0)}_n(\x,\y,t)=
   \int e^{i\k(\x-\y)}\cos kt\, d^n\k =
   \int F_n(k|\x-\y|)\cos kt\, k^{n-1}dk =
      \eqno (8a)
             $$
and           $$
     G^{(1)}_n(\x,\y,t)=
   \int e^{i\k(\x-\y)}
       {\sin kt\over k}\,d^n\k\,.
   \int F_n(k|\x-\y|)
       {\sin kt\over k}k^{n-1}dk\,.
     \eqno (8b)
             $$

We see that
        $$
   u(\x,t)=
    \int G^{(0)}(\x,\y,t)\varphi(\y)d^n\y+
    \int G^{(1)}(\x,\y,t)\psi(\y)d^n\y
      \eqno (8c)
        $$
Sometimes we use functions 
 $P^{(0)}(\x,t)$ and $P^{(1)}(\x,t)$: 
      $$
 G^{(0)}(\x,\y,t)=
 P^{(0)}(\x-\y,t)\,,
 G^{(1)}(\x,\y,t)=
 P^{(1)}(\x-\y,t)\,.
     \eqno (8d)
        $$
We may rewrite (8c) as
        $$
u=P^{(0)}\circ \varphi
+P^{(1)}\circ \psi\,,
        $$
where $``\circ''$ is convolution.



Now one can see literally that for odd $n$ integral is localised
on the light cone. This is easy to check for propagators $P_1,P_2$:
 Let $n=2m+1$ be odd then according to previous calculations for
for function $F_n(u)$ and elementary formulae of differentiation of
trigonometric functions we have:
            $$
P_0(\x,t)=
   \int e^{i\k\x}\cos kt\, d^{2m+1}\k =
   \int F_{2m+1}(kx)\cos kt\, k^{2m}dk =
            $$
           $$
\left(\int_0^{\pi\over 2} 
e^{iu\cos\theta}\sin^{2m-1}\theta d\theta\right)\big\vert_{u=kx}
      \cos kt\, k^{2m}dk=
\left(\int_0^{\pi\over 2} 
e^{iu\cos\theta}\sin^{2m-2}\theta d\cos\theta\right)\big\vert_{u=kx}
      \cos kt\, k^{2m}dk
          $$
           $$
=\left(\int_{-1}^1 e^{iuz}(1-z^2)^{m-1\over 2}dz\right)\big\vert_{u=kx}
      \cos kt\, k^{2m}dk=
=\left(\left(1-\left(i{d\over du}\right)^2\right)^{m-1}
\int_{-1}^1 e^{iuz}dz\right)\big\vert_{u=kx}
      \cos kt\, k^{2m}dk
             $$
             $$
=\left(\left(1+{d^2\over du^2}\right)^{m-1}
    {\sin u\over u}\right)\big\vert_{u=kx}
      \cos kt\, k^{2m}dk=
           $$


\centerline {Comparing with paper of Howe}

Try to compare propagators $P_1,P_2$ (see equations 8d) with
those in paper of Howe.
   Denote as in Howe Fourrier image of $F$ by $\hat F$. We have
        $$
\hat {P_1}(\k,w)=
\int e^{i\k'\x}\cos k't e^{-i(\k\x-wt)}d^n\k'd^n\x dt=
        $$
        $$
\left(\delta(k-w)+\delta(k-w)\right)d^nk dw=
k^2\delta(k^2-w^2)d\mu
        $$
where
      $$
d\mu={d^nk\over k}\,,\hbox{is an invariant measure in $\E^{n+1}$ with
  Minkovsku metric}
     $$
and      $$
\hat {P_2}(\k,w)=
\int e^{i\k'\x}{\sin k't\over k'} e^{-i(\k\x-wt)}d^n\k'd^n\x dt=
        $$
        $$
\left(\delta(k-w)+\delta(k-w)\right)d^nk dw=
k^2\delta(k^2-w^2)d\mu
        $$

\centerline {Action of dilation on propogators.}

   We consider with Howe, Heisenberg  group $H_n$:
               $$
     H_n=\{(\y,\x,z), \x,\y\in \R^n, z\}
               $$
The action on function:
                $$
   (h\circ \Psi)(\r)=ze^{2\pi i \y\cdot \x}\Psi(\r-\x)
                $$




Let  $g\GL(n,\R)$ be an arbitrary linear transformation.
It defines automorphism of $H$
           





 \centerline { Action of group $SL(2)$}



Consider on $\E^n$ operators
         $$
\Delta=g^{ik}\p_i\p_k,   \rho^2=g_{ik}x^ix^k\,, E={n+1\over 2}+D
         $$
here $g_{ik}={\rm diag\,}(1,-1,\dots,-1)$
We see that
        $$
[\Delta,\rho^2]=[g^{ik}\p_i\p_k, g_{pq}x^px^q]=2(n+1)+4x^i\p_i=4E
        $$
         $$
[E,\rho^2]=2E\,,\quad [E,\Delta]=2\Delta 
         $$

Now notice that


\bye
