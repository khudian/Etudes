\magnification=1200 %\baselineskip=14pt
\def\vare {\varepsilon}
\def\A {{\bf A}}
\def\t {\tilde}
\def\a {\alpha}
\def\K {{\bf K}}
\def\N {{\bf N}}
\def\V {{\cal V}}
\def\s {{\sigma}}
\def\S {{\Sigma}}
\def\s {{\sigma}}
\def\p{\partial}
\def\vare{{\varepsilon}}
\def\Q {{\bf Q}}
\def\D {{\cal D}}
\def\G {{\Gamma}}
\def\C {{\bf C}}
\def\M {{\cal M}}
\def\Z {{\bf Z}}
\def\U  {{\cal U}}
\def\H {{\cal H}}
\def\R  {{\bf R}}
\def\S  {{\bf S}}
\def\E  {{\bf E}}
\def\l {\lambda}
\def\degree {{\bf {\rm degree}\,\,}}
\def \finish {${\,\,\vrule height1mm depth2mm width 8pt}$}
\def \m {\medskip}
\def\p {\partial}
\def\r {{\bf r}}
\def\v {{\bf v}}
\def\n {{\bf n}}
\def\t {{\bf t}}
\def\b {{\bf b}}
\def\c {{\bf c }}
\def\e{{\bf e}}
\def\ac {{\bf a}}
\def \X   {{\bf X}}
\def \Y   {{\bf Y}}
\def \x   {{\bf x}}
\def \y   {{\bf y}}
\def \G{{\cal G}}
\def\w{\omega}
\def\finish {${\,\,\vrule height1mm depth2mm width 8pt}$}

\centerline  {\bf Killings of sphere in stereographic projection 
          and Geodesics,}
  \centerline {\bf (Runge-Lents-Laplace-like vectors)}

\m

{\sl 28 February 2013}

\m
Consider sphere of radius $1$ in stereographic projection 
trough North Pole:
         $$
     u={x\over 1-z}, v={y\over 1-z}\,.
         $$
The metric is 
                  $$
   G={4(du^2+dv^2)\over (1+u^2+v^2)^2}\,.
                  $$ 
Volume form of the metric  is 
              $$
    dv=\det G={4du\wedge dv\over (1+u^2+v^2)^2}\,.
              $$
Recall that the nicest way to find Killings, 
rotations of sphere, it is to consider
them as linear Hamiltonians 
restricted on the sphere , 
        $$
H=ax+by+cz=a\sin\theta\cos\varphi+b\sin\theta\sin\varphi+c\cos\theta\,,
        $$
which is provided with symplectic structure
$\w=dv=\sin\theta\wedge d\varphi$.
(Symplectic structure on the sphere
is the Kirillov symplectic structure on orbit 
($=S^2$) in coalgebra $so(3)=E^3$;
linear Hamiltonians are elements of this coadjoint algebra.)
Do it in stereographic coordinates:
         $$
H_x=x={2u\over 1+u^2+v^2},\, H_y=y={2v\over 1+u^2+v^2},\, 
H_z=z={u^2+v^2-1\over 1+u^2+v^2}\,,
         $$
Using that $D_H=\w^{-1}dH$ we come to
        $$
D_x=\w^{-1}dH_x={1+v^2-u^2\over 2}\p_v+uv\p_u\,,
         $$
        $$
D_y=\w^{-1}dH_x={1+u^2-v^2\over 2}\p_u+uv\p_v
         $$
        $$
D_z=\w^{-1}dH_x=\pm(v\p_u-u\p_v)\,.
         $$
It is representation of $so(3)$: 
        $$
 [D_x,D_y]=D_z,\,\,
 [D_y.D_z]=D_x,\,\,
 [D_z.D_x]=D_y\,.
       $$

We call them Runge-Lentz since under
suitable transformation free particle on sphere becomes particel in Coulomb
field and these inegrals are just Runge-Lentz-Laplace -vectors. 

It is interesting to find geodesics of sphere (great circles)
via these Runge-Lentz vector fields
in stereographic coordinates. Of course we know already answer:
  Image of circles are circles 
($u+iv=z\mapsto z'={Az+B\over -\bar Bz+\bar A }$ is $SU(2)$transformation.)

Great circle passes through two antipodal points: 
images of antipodal points 
are points $z,z'$, such that $z'=-{1\over \bar z}$. Hence geodesics
are circles such that images of antipodal points are edges of
their diameters.

But try to claculate them trough geodesics. Geodesics of sphere are 
equation of motion of Lagrangian
            $$
   L=L_G=G_{ik}\dot x^i\dot x^k=
  {2(\dot u^2+\dot v^2)\over (1+u^2+v^2)^2}\,.
            $$ 
Instead solving second order equations consider three
integrals of motion
 $I=K^i{\p L\over \p \dot x^i}$ 
corresponding to Killing vector fields $\{D_x,D_y,D_z\}$:
              $$
I_x={2\over (1+u^2+v^2)^2}\left(
    2uv\dot u+(1+v^2-u^2)\dot v\right)=C_1\,,
              $$
          $$
I_y={2\over (1+u^2+v^2)^2}\left(
   (1+u^2-v^2)\dot u+2uv\dot v\right)=C_2\,,
              $$
        $$
I_z={2\over (1+u^2+v^2)^2}\left(
   v\dot u-u\dot v\right)=C_3\,.
              $$
We have three integrals of motion which are equal to constants
$C_1,C_2,C_3$ on geodesics (paths of free particle).

Above there are three equations on two unknowns $\dot u$ and $\det v$. 
Hence we come from these three equations on 
variables $\dot u,\dot v$ to equation
       $$
\det 
\pmatrix
   {
  2uv &1+v^2-u^2 &C_1\cr
  1+u^2-v^2 & 2uv  &C_2\cr
  v     & -u        & C_3\cr
     } =0
       $$
(Vectors of integrals of motions belong to the span to two vectors.)

Calculating determinant we come to 
       $$
\det 
\pmatrix
   {
  2uv &1+v^2-u^2 &C_1\cr
  1+u^2-v^2 & 2uv  &C_2\cr
  v     & -u        & C_3\cr
     } =(1+u^2+v^2)\left[C_2v-C_1u+C_3(u^2+v^2-1)\right]=0\,,
       $$
i.e.
      $$
C_2v-C_1u+C_3(u^2+v^2-1)=0\,.
      $$
These are trajectories of free particle----geodeisics.


Three cases  

1st case: $C_3=0$ we come to straight lines $C_2v-C_1u=0$, images of meridians.

2-nd case:  $C_1=C_2=0$ we come to $u^2+v^2=1$, the
 image of Equator

3-rd case $C_1\not=0$ or $C_2\not=0$ and $C_3\not=0$ we come to circles
such that ''antipodal points'' are inversed.\finish 

\bye
