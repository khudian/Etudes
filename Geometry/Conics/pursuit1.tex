   \magnification=1200
   \baselineskip 17 pt

\def\V {{\cal V}}
\def\s {{\sigma}}
\def\Q {{\bf Q}}
\def\D {{\cal D}}
\def\G {{\Gamma}}
\def\C {{\bf C}}
\def\M {{\cal M}}
\def\Z {{\bf Z}}
\def\U  {{\cal U}}
\def\H {{\cal H}}
\def\R  {{\bf R}}
\def\l {\lambda}
\def\p {\partial}
\def\r {{\bf r}}
\def\v {{\bf v}}
\def\n {{\bf n}}
\def\t {{\bf t}}
\def\b {{\bf b}}
\def\ac {{\bf a}}
\def \X   {{\bf X}}
\def \Y   {{\bf Y}}
\def \E   {{\bf E}}
\def\vare {\varepsilon}
\def\A {{\bf A}}
\def\t {\tilde}
\def\a {\alpha}
\def\K {{\bf K}}
\def\N {{\bf N}}
\def\V {{\cal V}}
\def\s {{\sigma}}
\def\S {{\Sigma}}
\def\s {{\sigma}}
\def\p{\partial}
\def\vare{{\varepsilon}}
\def\Q {{\bf Q}}
\def\D {{\cal D}}
\def\G {{\Gamma}}
\def\C {{\bf C}}
\def\M {{\cal M}}
\def\Z {{\bf Z}}
\def\U  {{\cal U}}
\def\H {{\cal H}}
\def\R  {{\bf R}}
\def\E  {{\bf E}}
\def\l {\lambda}
\def\degree {{\bf {\rm degree}\,\,}}
\def \finish {${\,\,\vrule height1mm depth2mm width 8pt}$}
\def \m {\medskip}
\def\p {\partial}
\def\r {{\bf r}}
\def\v {{\bf v}}
\def\n {{\bf n}}
\def\t {{\bf t}}
\def\b {{\bf b}}
\def\c {{\bf c}}
\def\e{{\bf e}}
\def\f{{\bf f}}
\def\g{{\bf g}}
\def \X   {{\bf X}}
\def \Y   {{\bf Y}}
\def \x   {{\bf x}}
\def \y   {{\bf y}}
\def\w {{\omega}}
\def\A{{\bf A}}
\def\B{{\bf B}}
\def\pt{{\bf p}}


\centerline {Conic sections and pursuit problem}

14 April 2018

 Consider the following pursuit problem:

   The point $A$ moves in vertical direction
     starting at origin with constant speed,
  and the point   $B$
   moves from the point
    (L,0)) with the same speed, 
   such that velocity
  is directed along the segment $BA$. 
  One can see that the path of the point
  $B$ will be assimptotically tending to vertical. 
      What will be the distance between
points $A$ and $B$ at $t\to \infty$?


  This problem has the following nice but artifical 
solution.

Denote by $\theta$ the angle between
the segment $|BA|$ and $OY$ axis.
The length of the segment  $BA$ 
is decreasing in time with the 
speed $v(1-\cos\theta)$,  
and the projection of
this segment on the axis $OY$ is increasing in time with
the same speed $v(1-\cos\theta)$. Hence the
sum of the length  of the segment $BA$ and its projection
on the axis $OY$ remains invariant in time.
At $t=0$, this sum was equal to $L$, and
  If $t\to \infty$, when  the segment $BA$ tends to
vertical segment, this sum is tending
to $2|AB|$,  Hence the distance between the points
is $A$ and $B$, the length of the segment
$BA$  tending to $L\over 2$. 
      
Nice and unexpected solution.

***************


Many many years ago in 1971 when I was in school,
I desperately was trying to solve this problem
(it was in March 1971 on the Republican
Olimpiad in Physics in Yerevan).
Later in my student times,
I  wasy trying  to find another solution,
which would be  not so artificail, but
still elegant. Once I realied that
conic sections lead to the beaitufil and
illuminating solution. Here it is.
  
\m

  Consider the frame of reference such that 
the point $A$ stands still in this frame. 
In this frame the point
$B$ simultaneously moves to the origin with the
same speed it 
moves vertically down.
It means that in this frame of reference,
the distance between point $B$
and origin, is the same than the distance
between the point $B$, and the line$y=-L$: this means
that the trajectory of the point $B$ 
in this frame is a PARABOLA
with focus at origin, and with directrix $y=-L$.
At $t\to \infty$ the point will be at the midpoint
of the segment, which is orthogonal to directrix
and which is between the origin, the focus,
 and the directrix, i,e, it will be
at the distance  $L\over 2$.
\finish

\bye
