
   \magnification=1200
   \baselineskip=17 pt

\def\V {{\cal V}}
\def\s {{\sigma}}
\def\Q {{\bf Q}}
\def\M {{\bf M}}
\def\D {{\cal D}}
\def\G {{\Gamma}}
\def\C {{\bf C}}
%\def\M {{\cal M}}
\def\Z {{\bf Z}}
\def\U  {{\cal U}}
\def\H {{\cal H}}
\def\R  {{\bf R}}
\def\l {\lambda}
\def\p {\partial}
\def\r {{\bf r}}
\def\v {{\bf v}}
\def\n {{\bf n}}
\def\t {{\bf t}}
\def\b {{\bf b}}
\def\ac {{\bf a}}
\def \X   {{\bf X}}
\def \Y   {{\bf Y}}
\def \E   {{\bf E}}
\def\vare {\varepsilon}
\def\A {{\bf A}}
\def\t {\tilde}
\def\a {\alpha}
\def\K {{\bf K}}
\def\N {{\bf N}}
\def\V {{\cal V}}
\def\s {{\sigma}}
\def\S {{\Sigma}}
\def\s {{\sigma}}
\def\p{\partial}
\def\vare{{\varepsilon}}
\def\Q {{\bf Q}}
\def\D {{\cal D}}
\def\G {{\Gamma}}
\def\C {{\bf C}}
\def\Z {{\bf Z}}
\def\U  {{\cal U}}
\def\H {{\cal H}}
\def\R  {{\bf R}}
\def\E  {{\bf E}}
\def\l {\lambda}
\def\degree {{\bf {\rm degree}\,\,}}
\def \finish {${\,\,\vrule height1mm depth2mm width 8pt}$}
\def \m {\medskip}
\def\p {\partial}
\def\r {{\bf r}}
\def\v {{\bf v}}
\def\n {{\bf n}}
\def\t {{\bf t}}
\def\b {{\bf b}}
\def\c {{\bf c}}
\def\e{{\bf e}}
\def\f{{\bf f}}
\def\g{{\bf g}}
\def \X   {{\bf X}}
\def \Y   {{\bf Y}}
\def \x   {{\bf x}}
\def \y   {{\bf y}}
\def\w {{\omega}}
\def\A{{\bf A}}
\def\B{{\bf B}}
\def\pt{{\bf p}}

\centerline {\bf Conic sections and Kepler law. 
Newton---Lagrange----A.Giventale}


\m

 
{\it  This etude is written on the base of
my conversations with Alexander Giventale in
   May 2014, and his letters to me,  
  in April 2017
}
      
\m

\centerline {${\cal x}1$ \bf Introduction}

Conic section is an intersection of 
plane with surface of cone.
Let $M\colon k^2x^2+k^2y^2=z^2$ be 
conic surface, and let $\pi\colon Ax+By+Cz=1$
be a plane\footnote{$^{3}$}{we do not 
conisder degenerate case when plane 
intersects origin.}

Conic section
        $$
C\colon \cases
     {
        k^2x^2+k^2y^2=z^2\cr
         Ax+By+Cz=1\cr
       }
     \eqno (1.0)
        $$ 
          $$
\hbox {{\it 1.  Conic sections are ellipses or parabolas or hyperbolas}}\,,
       \eqno (1.1)
       $$
Kepler Law: 
         $$
    \matrix
     {
\hbox {trajectories of particle in
gravitational field are conic section}\cr
    \hbox{ In particular
oribits of planets are ellipses}\cr
          }
           \eqno (1.2)
       $$
These two statements  are common place for everybody.

   Almost everybody who has learnt a bit of higher
mathematics knows how to prove statement (1.1) 
(exercise in Analytical Geometry).
Everybody who learnt little bit calculus knows
that using the gravitational law, that 
              $$
F=\gamma{mM_{\rm sun}\over R^2}\,
  \eqno (1.3)
              $$   
one can prove statement (1.2), showing that  
 the solution of differential equation
      $$
  {d^2\over dt^2}\R={{\bf F}\over m}=
     -\gamma{M_{\rm sun}\R\over R^3}\,,
      \eqno (1.3a)
           $$
is a conic section($m$ is a mass of planet, 
and $M_{\rm sun}$ is Solar mass).
Standard proofs
are based on calculus: solving equation
(1.3a) in polar coordinates   
and calculating the
integral  we come to  conic section
\footnote{$^{1)}$}
{Usually equation (1.3) is called Newton Graviational
Law, and equation (3a) Newton Second law. 
There is an alternative point of view that
Robert Hook formulated statement (1.3) 
in the letter to Newton,
and Newton who invented Calculus, 
deduced the Kepler law, statement (2),
solving differential equation (1.3a).}
(see any book in Theoretical Mechanics or Appendix 1
in the end of this text).


  Here we discuss how to come to the statement
(1.2) avoiding caclulations of integras.

Instead conic section (1.0) one can consider
its orthogonal projection on the plane
   $z=0$. For example if
        $$
C\colon \cases
     {
        k^2x^2+k^2y^2=z^2\cr
         Ax+By+z=1\cr
       } 
\hbox {then its orthogonal projection}\,\,
C_{proj}\colon \cases
     {
        k^2x^2+k^2y^2=(1-Ax-By)^2\cr
         z=0\cr
       } 
         \eqno (1.0)
         $$ 

The following statement which is almost 
evident plays cruciar role in this etude:
        $$
\hbox{{\it orthogonal projections of conic sections
on the plane is also a conic sections.}}
\eqno (4)
        $$
  In fact orbits of planets are these projections.... 

\bigskip



\centerline {${\cal x}$2}



{
  This etude has the following history.
Few months ago I was preparing the new lecture for 
Geometry students about conic sections.  
I had a task to explain
that sections of conic surface with plane are conic sections
  (ellipses, parabolas or hyperbolas).    
The problem was 
 that I could not  find a beautiful and 
short explanation
of the statement (1) without using extracurricular 
material.  Another trouble was that at that time I had
 serious problems with eyes,
and my  access to books was very restricted.
Finally I prepared the following  explanation:
Instead statement (1) I proved to students
the statement (4), then I deduced  
statement (1) as the corollary of the statement (4).
Honeslty  considerations looked very bulky, 
I did not like them, but I had no choice.   
 
   I never forget early morning 30 March.  
 In five  hours the lecture will begin, 
where I have to explain to students the statement
(1) on conic sections.
I feel me very unhappy and unsatisfied with  
the way how I want to deliver the lecture, 
since the geometry of my considerations on the lecture
look very vague.
 Suddenly I ask me a question:  
  where are the foci of the projected conic section? 
(The foci of initial conic sections are at 
the points  where
 Dandellen spheres touch the plane which sects the conic surface.)  
   When the question was put, 
the answer come almost immediately:
  It is the vertex of the cone 
that is one of  the foci of projected
  conic section. This was crucial:
I immediately remembered, the May 2014.  


{\tt I am in the  
 Davis University
on 80 years celebration of my teacher 

 Albert S. Schwarz.
I meet there Alexander Givental. Alexander 

Givental is famous mathematician, 
but he is also
very much  engaged  in 

teaching mathematics to keeds
(see his homepage in Berkeley University).

During conference 
dinner we are sharing the same table, and Alexander

is explaining me some beautiful 
properties of conic section.

In  particular he is telling  
me  the sentence: 

 if you consider  a projection of
ellipse on the horisontal plane, 

you come naturally to
Kepler law...
}
 
  {\sl Recalling this phraze in 
the morning 30 March 2017,
 I immediately realised the geometrical meaning
of my construction (1.4), which I made preparing the lecture, and I understood what Alexander Givental wanted
to tell me three years ago.   
} 


   Next day  I contacted
Alexander iby e-mail. He immediately sent me  
the detailed answer 
%on my question,
%article [3], and the 
%letter about Lagrange considerations; all the material
%that this etude is based on.

   First, Alexander sent me the article [3]. 
In this article he  gives detailed
geometrical explanation why trajectories
of planets are conic section, and he does not use
the caclulus. You can find this article 
on the homepage
of  A.Givental. Thirty years ago, an abstract of this 
article was included by V. I. Arnold into 
his joint with V. V. Kozlov and A. I. Neishtadt 
survey of classical mechanics [2]. 

   In this letter  A. Givental also told me that
Alain Chensiner noted him that his (A.G.) 
considerations are very close to Lagrange's proof
(see the paper [1]), and Alexander sent me the 
Lagrange article [1], and the letter
where he explained this. 

I do not want here to retell the work [3]
of A.Givental. I just 
try based on the letter of A.Givental
to  retell the simple proof of
statement (1.2) avoiding calculus.
This construction can be traced to the work [1]
of Lagrange.  Of course the work [3]
contains complete picture of these considerations.




My modest contribution to this etude 
is related with the fact that trying
to find a  simple proof of classical statement
(1.1), I realised the importance
of the statement (4),a nd on the base of
it I try to retell the constuctions
of the papers [1] and [3].
 
\bigskip

\centerline {${\cal x}3$ \bf  Simple exlanation of
  Kepler Law.}





Consider differential equation (1.3a)
of particle in gravitational field
                    $$
              {\buildrel ..\over \R}=
            -{k\R\over R^3}\,.
                      \eqno (3.1)
                        $$
let $\R=\R(t)$ be a solution of this equation.
In Appendix we explicitly solved the equation
and showed that it is conic section.
Now we will do something much simpler.
Without calculating explicitly integrals 
(see (App1.9), (App1.9a)) we just will show that
   $\R(t)$ is a conic section, that is it obeys
   equation (App1.10).
Further we will use considerations from Appendix which has nothing to do with calculus
(equations (App1.1)---(App1.8))


The preservation of angular momentum
implies that the solution $\R=\R(t)$ is a curve in 
the plane. Choose Cartesian coordinates
$x,y,z$ such that
   $$
 \R(t)=x(t)\e_x+y(t)\e_z
\eqno (3.2)
    $$, and 
angular momentum
      $$
\M=(x(t)\dot y(t)-y(t)\dot x(t))\e_z=\mu\e_z\,.
    \eqno (3.3)
      $$
(see for details Appendix).

Choose an arbitrary solution of equation (3.1)
Let  $x_{_0}(t)$, $y=y_{_0}(t)$ be components (3.2) 
of this solution of equation (3.1):
           $$
  \pmatrix{{\buildrel ..\over x_{_0}(t)}\cr 
  {\buildrel ..\over y_{_{0}}(t)}\cr}=
      -{k\over\left(x^2_{_0}(t)+
     y^2_{_0}(t)\right)^{3\over 2}}
\pmatrix{x_{_0}(t)\cr y_{_0}(t)\cr}\,.
     \eqno (3.4) 
      $$
Now consider the pair of differential equationts
on function $f=f(t)$

1) homogeneous linear differential equation
     $$
{d^2 f\over dt^2}=-k{f\over r^3_{_0}(t)}\,,
\eqno (3.5)
     $$ 
and  non-homogeneous linear differential equation
             $$
{d^2 f\over dt^2}=-k{f\over r^3_{_0}(t)}+
    {\mu^2\over r^3_{_0}(t)}\,,
    \eqno (3.5a)
             $$
where function $\r_{_0}(t)$ is defined by
solutions $x_{_0}(t)$, $y_{_0}(t)$ (3.4):
           $$
r_{_0}(t)=\sqrt{x^2_{_0}(t)+y^2_{_0}(t)}\,.
           $$

We see that $x_{_0}(t)$, $y_{_0}(t)$
are two independent solutions of second
order homogeneous equation
(we suppose that angular momentum 
$\mu=x\dot y-y\dot x$ does not vanish).

Let $g=g_0(t)$ be 
an  arbiitrary solution of linear 
not homogeneous equation (3.5a).
Then  the space of all  solutions
of  equation (3.5a) is
         $$
f(t)=Ax_{_0}(t)+By_{_0}(t)+g_0(t)
     \eqno (3.6)
         $$


also that differential equation (3.1) implies 
that the functio
                   $$
             r_0(t)=\sqrt{x_0^2(t)+y_0^2(t)}
                   $$
also obeys the differential equation
                    $$
    {d^2 r\over dt^2}=-{k\over }+{\mu\over r^3}\,.
                     \eqno (3.5)
                    $$
(see formulae )
We consider the solution  $x_0(t), y_0(t)$
of the equation (1)
Consider the function 
            $$
   F\colon\,, F(t)=\left(x_0^2(t)+y_0^2(t)\right)^{-{3\over 2}}\,.
            $$
This function is defined by the solution of the equation (1).
$r(t)=\sqrt{x^2(t)+y^2(t)}$.

Consider also the following two linear differential equations
(homogeneous and non-homogeneous)
                 $$
  {\buildrel ..\over u}=-kF_0 u\,,\qquad
     {\buildrel ..\over u}=-kF_0 u+CF_0\,.
                 $$
 Solutions of the first equation form $2$-dimensional linear space,
solutions of the second equation form $2$-dimensional
 affine space.  The improtant observation is that


  One can see that equation (1) implies that
for the function $r(t)$ 
         $$
{d^2r(t)\over dt^2}=
          {d^2\over dt^2}
      \left(\sqrt{x^2(t)+y^2(t)}\right)
         = {d\over dt}
      \left(
    x(t){dx(t)\over dt}+
    y(t){dy(t)\over dt}
         \over 
\sqrt{x^2(t)+y^2(t)}
      \right)=
         $$
           $$
         {
    x(t){d^2x(t)\over dt^2}+
    y(t){d^2y(t)\over dt^2}
         \over 
\sqrt{x^2(t)+y^2(t)}
            }+
       \left(
    \left({dx(t)\over dt}\right)^2+
    \left({dy(t)\over dt}\right)^2
         \over 
\sqrt{x^2(t)+y^2(t)}
      \right)-
         \left(
    x(t){dx(t)\over dt}+
    y(t){dy(t)\over dt}
         \over 
\sqrt{x^2(t)+y^2(t)}
      \right)
           $$

\bigskip

   \centerline {\bf Appenidix 1}


{\it In this Appnedix we will give
the standard proof based on the calculus, that
 trajectory of particle in gravitational filed is
a conic section}

  Let $\R=\R(t)$ is a solution of differential equation
(1.3a):
      $$
  {d^2\over dt^2}\R={{\bf F}\over m}=
     -\gamma{M_{\rm sun}\R\over R^3}\,,
      \eqno (Ap1.1)
           $$
 This implies that angular momentum
      $$
\M=m\dot \R\times\R
      \eqno (App1.2)
        $$
is conserved: 
$\dot M=m\dot\R\times \dot \R+
m{\buildrel ..\over\R }\times\R=
0+\left( 
 -\gamma{mM_{\rm sun}\R\over R^3}
\right)\times\R=0$.
Thus vector $\R(t)$ is orthogonal to the vector
$\M$.
Consider Cartesian coordinates $x,y,z$
such that $\M$ is directed along axis $OZ$,
and $\R(t)=x(t)\e_x+y(t)\e_y$ belongs to the plane
$OXY$:
          $$
\R(t)=x(t)\e_x+y(t)\e_y\,,\quad
 \M(t)=\mu\e_z\,, (\mu\,\, \hbox{is constant})\,.
           \eqno (App1.5)
           $$  
Let $\varphi,r$ be polar coordinates
in the plane $OXY$, $x=r\cos\varphi,y=r\sin\varphi$.
Then equation (1.3a) and (1.5) imply that
        $$
\mu=m(x\dot y-y\dot x)=
r\cos\varphi(\dot r\sin\varphi+r\dot\varphi\cos\varphi)-
r\sin\varphi(\dot r\cos\varphi-r\dot\varphi\sin\varphi)=
  mr^2\dot\varphi\,,
        (App1.6a)
        $$
and
       $$
 {\buildrel ..\over r}={d\over dt}\dot r=
  {d\over dt}\left({x\dot x+y\dot y\over r}\right)=
{x{\buildrel ..\over x}+y{\buildrel ..\over y}\over r}+
    {\dot x^2+\dot y^2\over r}-
    {\left(x\dot x+y\dot y\right)^2\over r^3}=
       $$
      $$
{x\left(-\gamma{M_{\rm sun}x\over r^3}\right)+
y\left(-\gamma{M_{\rm sun}y\over r^3}\right)\over r}+
{(\dot x^2+\dot y^2)(x^2+y^2)-
 (x\dot x+y\dot y)^2\over r^3}
        =
        $$
       $$
-\gamma{M_{\rm sun}\over r^2}+
{(x\dot y-y\dot x)^2\over r^3}=
-\gamma{M_{sun}\over r^2}+{\mu^2\over m^2r^3}\,,
\eqno(App1.7a)          
       $$
i.e.
      $$
{d\over dt}\left({\dot r^2\over 2}\right)=
\dot r{\buildrel ..\over r}=
-\gamma{M_{\rm sun}\over r^2}\dot r+
{\mu^2\over m^2 r^3}\dot r=
   {d\over dt}\left(\gamma{M\over r}-
        {\mu^2\over 2m^2r^2}\right)\,.
      $$
Thus we come to integral of motion:
     $$
{d\over dt}\left({m\dot r^2\over 2}-
\gamma{mM_{\rm sun}\over r}+
{\mu^2\over 2mr^2}\right)=0
 \Rightarrow
{m\dot r^2\over 2}-
\gamma{mM_{\rm sun}\over r}+
{\mu^2\over 2mr^2}=E\,.
\eqno (App1.8) 
     $$
We come to first order differential equations
\footnote{$^{2)}$}{
     $E={m\dot r^2\over 2}-
\gamma{mM_{\rm sun}\over r}+
{\mu^2\over 2mr^2}$ is energy, 
${m\dot r^2\over 2}$ is kinetik enery, and
  $U_{\rm eff}=-
\gamma{mM_{\rm sun}\over r}+
{\mu^2\over 2mr^2}$ is effective potential energy}:
            $$
\cases
               {
 {dr(t)\over dt}=\sqrt {2\over m}\sqrt{
  E+\gamma{mM_{\rm}\over r}-{\mu^2\over 2mr^2}}\cr
   {d\varphi(t)\over dt}={\mu\over mr^2}\cr   
              }\Rightarrow
   {d\varphi\over dr}=
        {
{\mu\over r^2}\over
\sqrt{
  2mE+\gamma{2mM_{\rm sun}\over r}-{\mu^2\over r^2}}
       }
        \eqno (App1.8a)
            $$
thus
        $$
\varphi=\int
           {
{\mu dr\over r^2}\over
\sqrt{
  2mE+\gamma{2m^2M_{\rm sun}\over r}-{\mu^2\over r^2}}
       }
         \eqno (App1.9)
        $$
Integrating we come to 
$$\int
           {
{\mu dr\over r^2}\over
\sqrt{
  2mE+\gamma{2m^2M_{\rm sun}\over r}-{\mu^2\over r^2}}\,,
       }=
       -\int 
         {\mu  d\left({1\over r}\right)\over
\sqrt{
  \gamma^2{m^2M^2_{\rm sun}\over \mu^2}+2mE-
\mu^2\left({1\over r}-
    \gamma{m^2M_{\rm sun}\over \mu^2}\right)^2
             }
       }=
$$
      $$
{\rm arcsin\,}
   \left(
     {\mu\over 
         \sqrt{
   \gamma^2{m^2M^2_{\rm sun}\over \mu^2}+2mE}
           }
       \right)
    \left({1\over r}-
    \gamma{m^2M_{\rm sun}\over \mu^2}\right)\,
          \eqno (App1.9a)
      $$ 
i.e.
     $$
\sin\varphi=
     {\mu\over 
         \sqrt{
   \gamma^2{m^2M^2_{\rm sun}\over \mu^2}+2mE}
           }
    \left({1\over r}-
    \gamma{m^2M_{\rm sun}\over \mu^2}\right)=
{1\over \sqrt{1+{2E\mu^2\over \gamma^2M^2_{\rm sun}m^3}}}
        \left(
          {\mu^2\over\gamma m^2 M_{\rm sun}}
             {1\over r}-1
        \right)
\eqno (App1.9b)
     $$
Introduce notations:
         $$
e=\sqrt{1+{2E\mu^2\over \gamma^2M^2_{\rm sun}m^3}}\,,
   k={\mu^2\over\gamma m^2 M_{\rm sun}}\,,
          $$
then (App1.9b) has appearance:
        $$
e\sin\varphi={k\over r}-1
  \eqno (App1.10)
        $$
This is conic section with excentricet $\vare$,
and one of the foci at the origin.

{\bf Remark} {\tt in fact it has to be $\cos\varphi$,
not $\sin\varphi$}




[1] Lagrange  {\it DES PERTURBATIONS DES COMETES.}---
            SECTION DDEUXIEME.  Integrations des equations differentielles
de l'orbite non-altere. pp.419---430,  1785


[2] V. I. Arnold, V. V. Kozlov, A. I. Neishtadt
{\it  Mathematical aspects of classical 
and celestial mechanics.}
 Dynamical systems – III 
(Encyclopaedia of Mathematical Sciences), 
Springer, 1987.



[3] Alexander Givental 
{\it Kepler’s Laws and Conic Sections}
  Arnold Mathematical Journal March 2016, 
Volume 2, Issue 1, pp 139–148

see in homepage of the author:
math.berkeley.edu$\backslash$
$~$
giventh$\backslash$
kepler$\_$091615.pdf


\bye

://math.berkeley.edu/~giventh/kepler_091615.pdf

