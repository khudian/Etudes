
   \magnification=1200
   \baselineskip 17 pt

\def\V {{\cal V}}
\def\s {{\sigma}}
\def\Q {{\bf Q}}
\def\D {{\cal D}}
\def\G {{\Gamma}}
\def\C {{\bf C}}
\def\M {{\cal M}}
\def\Z {{\bf Z}}
\def\U  {{\cal U}}
\def\H {{\cal H}}
\def\R  {{\bf R}}
\def\l {\lambda}
\def\p {\partial}
\def\r {{\bf r}}
\def\v {{\bf v}}
\def\n {{\bf n}}
\def\t {{\bf t}}
\def\b {{\bf b}}
\def\ac {{\bf a}}
\def \X   {{\bf X}}
\def \Y   {{\bf Y}}
\def \E   {{\bf E}}
\def\vare {\varepsilon}
\def\A {{\bf A}}
\def\t {\tilde}
\def\a {\alpha}
\def\K {{\bf K}}
\def\N {{\bf N}}
\def\V {{\cal V}}
\def\s {{\sigma}}
\def\S {{\Sigma}}
\def\s {{\sigma}}
\def\p{\partial}
\def\vare{{\varepsilon}}
\def\Q {{\bf Q}}
\def\D {{\cal D}}
\def\G {{\Gamma}}
\def\C {{\bf C}}
\def\M {{\cal M}}
\def\Z {{\bf Z}}
\def\U  {{\cal U}}
\def\H {{\cal H}}
\def\R  {{\bf R}}
\def\E  {{\bf E}}
\def\l {\lambda}
\def\degree {{\bf {\rm degree}\,\,}}
\def \finish {${\,\,\vrule height1mm depth2mm width 8pt}$}
\def \m {\medskip}
\def\p {\partial}
\def\r {{\bf r}}
\def\v {{\bf v}}
\def\n {{\bf n}}
\def\t {{\bf t}}
\def\b {{\bf b}}
\def\c {{\bf c}}
\def\e{{\bf e}}
\def\f{{\bf f}}
\def\g{{\bf g}}
\def \X   {{\bf X}}
\def \Y   {{\bf Y}}
\def \x   {{\bf x}}
\def \y   {{\bf y}}
\def\w {{\omega}}
\def\A{{\bf A}}
\def\B{{\bf B}}
\def\pt{{\bf p}}

\centerline {\bf Conic sections and Kepler law. 
Newton---Lagrange----A.Giventale}
      
Conic section is intersection of plnae with surface of 
cone.
         $$
\hbox {{\it 1.  Ellipses are conic sections}}
       \eqno (1)
       $$
          $$
\hbox {Kepler Law:
  {\it 2. The orbit of Planets are ellipses.}}
       \eqno (2)
       $$
These two statements  are common place for everybody.

   Almost everybody who have learnt a bit of higher
mathematics knows how to prove statement (1) 
(exercise in Analytical Geometry).
Everybody who learnt little bit calculus knows
that using the fact that
              $$
F=\gamma{mM\over R^2}\,
  \eqno (3)
              $$   
one can prove statement (2), showing that  
 the solution of differential equation
      $$
  {d^2\over dt^2}\R={{\bf F}\over m}=
     -\gamma{M\R\over R^3}\,,
      \eqno (3a)
           $$
is a conic section($m$ is a mass of planet, 
and $M$ is Solar mass).   
On the other hand all standard proofs
are based on calculus: solving equation
(3a) in polar coordinates  come to 
and calculating the
integral  we come to  conic section
 (see any book in Theoretical Mechanics)
\footnote{$^{1)}$}
{Usually equation (3) is called Newton Graviational
Law, and equation (3a) Newton Second law. 
There is an alternative point of view that
Robert Hook formulated statement (3) in the letter to Newton,
and Newton who invented Calculus, based
on (3) and (3a) deduced Kepler law, statement (2).}
 

The following statement which is almost 
evident plays cruciar role in this etude:
        $$
\hbox{{\it orthogonal projections of conic sections
on the plane which are conic sections also.}}
\eqno (4)
        $$
  In fact orbits of planets are these projections.... 

\bigskip

  This etude has the following history.
 In May 2014 I was in Davis University
on 80 years celebration of my Teacher Albert S. Schwarz.
I met there Alexander Givental. 
  Alexander is famous mathematician, but he is also
very much engaged in teaching mathematics
(see his homepage in Berkeley University)
During conference 
dinner we were sharing the same table, and Givental
was explaining me some beautiful 
properties of conic section.
In  particular he told me  
the sentence: if you consider  a projection of
ellipse on the horisontal plane, you come naturally to
Kepler law.  At that 
moment I did not understand the meaning of this phraze...   
About three years passed.   
Few months ago I preparing the lecture for 
Geometry students about conic sections, realised that
I cannot find a beautiful and short explanation
of the statement (1) without using extracurricular 
material.  Another trouble was that because of problems
of my eyes my  access to books was very restricted.
Finally I prepared the lecture where to prove the 
statement (1) I was forced to use the statement (4).
Honeslty I did not like this way of considerations,
but I had no choice.   
 30 March, morning,  in four hours will begin the lecture,
I am very unsatisfied, since I do not understand the menaing
of the statement (4). Suddenly I decided just to check,
  where are the foci of the projected cone. (The foci
of initial conic sections are related with Dandellen 
spheres.)  In a moment  I realised that the vertex of the
cone is one of the foci!. The next second I recalled the
the converstaion with Alexander Givental, his phraze:
this is projected conic secion which is related with Kepler
law.  Sure I was very happy, I immediately contacted
Alexander and he sent me his article all the material
that this etude is based on.
My modest contribution to this etude 
is related with the fact that I was desperately trying
to find as much as possible easy proof of classical statement
(1).
 




{\it  This etude is written on the base of
my conversations with Alexander Giventale in
   may 2014, his paper [3], and
his letter to me,  where he explained me that
 main idea of his paper can be traced in 
 to the manuscript [1] of Lagrange.
My modest contribution is that few months ago
preparing
lectures for students I noted that explaining 
conic sections,
this is useful to consider projection of conics
on the plane.  Thinking about the geometrical meaning
of the projection of conic, I realised that,
the vertex of the cone will be one of the 
foci of the projected ellipse.  
Immediately I remembered that three years ago when
I was in US Davis Univerrsity (celebretation of
80 years of my teacher A.S.Schwarz)
 I met A.Givental who told me  many beautiful stories
about ellipses, and I recalled his sentence:
To see Kepler's Law  you have to consider the projection
of ellipse. The rest was easy.}
 



Considee differential equation
of particle in gravitational field
                    $$
                       {\buildrel ..\over \r}=k{\r\over r^3}\,.
                      \eqno (1)
                        $$
The solution $\r=\r(t)$ is a curve in 
the plane (preservation of 
angular momentum), and we know that this is conic section
(ellipse, hyperbola, parabola) with focus at the origin
(Kepler law)  The standard proof of this fact 
(see any book in Mechanics)
contains caluclation of integrals.   A. 
Giventale had a beatiful point of
view on the geometrical origin of the Kepler first law.
(This was anounced in [2] and 
published in [3] after 30 years).
On the other hand A.Givental told me that
  Alain Chensiner noted him that 
this idea can be traced in the work [1] of Lagrange
`
  Here is the exposition of this result.
 (This is due to results of A.Givental
in his paper, and in his letter to me.)



  Let particle moves in the plane, and let $x(t),y(t)$ be components
  of $\r(t)$ ($x,y$ are Cartesian coordinates of the plane)
      $$
  \pmatrix{{\buildrel ..\over x(t)}\cr {\buildrel\over y(t)}\cr}=
      -{k\over\left(x^2(t)+y^2(t)\right)^{3\over 2}}
\pmatrix{x(t)\cr y(t)\cr}\,.
     \eqno (1a) 
      $$
 
Note also that differential equation (1) implies that the function 
                   $$
                 r_0(t)=\sqrt{x_0^2(t)+y_0^2(t)}
                   $$
obeys the differential equation
                    $$
        {d^2 r\over dt^2}=-{k\over r}+{C\over r^3}\,.
                     \eqno (2)
                    $$

We consider the solution  $x_0(t), y_0(t)$
of the equation (1)
Consider the function 
            $$
   F\colon\,, F(t)=\left(x_0^2(t)+y_0^2(t)\right)^{-{3\over 2}}\,.
            $$
This function is defined by the solution of the equation (1).
$r(t)=\sqrt{x^2(t)+y^2(t)}$.

Consider also the following two linear differential equations
(homogeneous and non-homogeneous)
                 $$
  {\buildrel ..\over u}=-kF_0 u\,,\qquad
     {\buildrel ..\over u}=-kF_0 u+CF_0\,.
                 $$
 Solutions of the first equation form $2$-dimensional linear space,
solutions of the second equation form $2$-dimensional
 affine space.  The improtant observation is that


  One can see that equation (1) implies that
for the function $r(t)$ 
         $$
{d^2r(t)\over dt^2}=
          {d^2\over dt^2}
      \left(\sqrt{x^2(t)+y^2(t)}\right)
         = {d\over dt}
      \left(
    x(t){dx(t)\over dt}+
    y(t){dy(t)\over dt}
         \over 
\sqrt{x^2(t)+y^2(t)}
      \right)=
         $$
           $$
         {
    x(t){d^2x(t)\over dt^2}+
    y(t){d^2y(t)\over dt^2}
         \over 
\sqrt{x^2(t)+y^2(t)}
            }+
       \left(
    \left({dx(t)\over dt}\right)^2+
    \left({dy(t)\over dt}\right)^2
         \over 
\sqrt{x^2(t)+y^2(t)}
      \right)-
         \left(
    x(t){dx(t)\over dt}+
    y(t){dy(t)\over dt}
         \over 
\sqrt{x^2(t)+y^2(t)}
      \right)
           $$
                 



[1] Lagrange  {\it DES PERTURBATIONS DES COMETES.}---
            SECTION DDEUXIEME.  Integrations des equations differentielles
de l'orbite non-altere. pp.419---430,  1785


[2] V. I. Arnold, V. V. Kozlov, A. I. Neishtadt
{\it  Mathematical aspects of classical 
and celestial mechanics.}
 Dynamical systems – III 
(Encyclopaedia of Mathematical Sciences), 
Springer, 1987.



[3] Alexander Givental 
{\it Kepler’s Laws and Conic Sections}
  Arnold Mathematical Journal March 2016, 
Volume 2, Issue 1, pp 139–148




\bye
