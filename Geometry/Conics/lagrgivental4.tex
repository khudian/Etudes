
   \magnification=1200
   \baselineskip=17 pt

\def\V {{\cal V}}
\def\s {{\sigma}}
\def\Q {{\bf Q}}
\def\M {{\bf M}}
\def\D {{\cal D}}
\def\G {{\Gamma}}
\def\C {{\bf C}}
%\def\M {{\cal M}}
\def\Z {{\bf Z}}
\def\U  {{\cal U}}
\def\H {{\cal H}}
\def\R  {{\bf R}}
\def\l {\lambda}
\def\p {\partial}
\def\r {{\bf r}}
\def\v {{\bf v}}
\def\n {{\bf n}}
\def\t {{\bf t}}
\def\b {{\bf b}}
\def\ac {{\bf a}}
\def \X   {{\bf X}}
\def \Y   {{\bf Y}}
\def \E   {{\bf E}}
\def\vare {\varepsilon}
\def\A {{\bf A}}
\def\t {\tilde}
\def\a {\alpha}
\def\K {{\bf K}}
\def\N {{\bf N}}
\def\V {{\cal V}}
\def\s {{\sigma}}
\def\S {{\Sigma}}
\def\s {{\sigma}}
\def\p{\partial}
\def\vare{{\varepsilon}}
\def\Q {{\bf Q}}
\def\D {{\cal D}}
\def\G {{\Gamma}}
\def\C {{\bf C}}
\def\Z {{\bf Z}}
\def\U  {{\cal U}}
\def\H {{\cal H}}
\def\R  {{\bf R}}
\def\E  {{\bf E}}
\def\l {\lambda}
\def\degree {{\bf {\rm degree}\,\,}}
\def \finish {${\,\,\vrule height1mm depth2mm width 8pt}$}
\def \m {\medskip}
\def\p {\partial}
\def\r {{\bf r}}
\def\v {{\bf v}}
\def\n {{\bf n}}
\def\t {{\bf t}}
\def\b {{\bf b}}
\def\c {{\bf c}}
\def\e{{\bf e}}
\def\f{{\bf f}}
\def\g{{\bf g}}
\def \X   {{\bf X}}
\def \Y   {{\bf Y}}
\def \x   {{\bf x}}
\def \y   {{\bf y}}
\def\w {{\omega}}
\def\A{{\bf A}}
\def\B{{\bf B}}
\def\pt{{\bf p}}

\centerline {\bf Conic sections and Kepler's law. 
Newton---Lagrange---Givental}


\m

 
{\it  This etude is written on July 2017 on the base of
some story happened with me when I was preparing
my lectures, and on the base of 
my conversations with Alexander Givental in
   May 2014, and his letters to me,  
  in April 2017
}
      
\m

\centerline {${\cal x}1$ \bf Introduction}

Conic section is an intersection of 
plane with surface of cone.
Let $M\colon k^2x^2+k^2y^2=z^2$ be 
conic surface, and let $\pi\colon Ax+By+Cz=1$
be a plane.
Conic section
        $$
C\colon \cases
     {
        k^2x^2+k^2y^2=z^2\cr
         Ax+By+Cz=1\cr
       }
     \eqno (1.0)
        $$ 
          $$
\hbox {{\it 1.  Conic sections are ellipses or parabolas or hyperbolas}}\,,
       \eqno (1.1)
       $$
(We do not 
conisder degenerate case when plane 
intersects origin.)

Kepler's  first law: 
         $$
    \matrix
     {
\hbox {trajectories of particle in
gravitational field are conic section}\cr
    \hbox{ In particular
orbits  of every planet  is an  ellipse}\cr
   \hbox{Sun is one of the foci of this ellipse}
          }
           \eqno (1.2)
       $$
These two statements  are common place for everybody.

   Almost everybody who has learnt a bit of higher
mathematics knows how to prove statement (1.1) 
(exercise in Analytical Geometry).
Everybody who learnt little bit calculus knows
that using the gravitational law, that 
              $$
F=\gamma{mM_{\rm sun}\over R^2}\,
  \eqno (1.3)
              $$   
one can prove statement (1.2), showing that  
 the solution of differential equation
      $$
  {d^2\over dt^2}\R={{\bf F}\over m}=
     -\gamma{M_{\rm sun}\R\over R^3}\,,
      \eqno (1.3a)
           $$
is a conic section( $m$ is a mass of planet, 
and $M_{\rm sun}$ is Solar mass, $\gamma$--gravitational constant).
Standard proofs
are based on calculus: solving equation
(1.3a) in polar coordinates   
and calculating the
elementary integrals  we come to  conic section
\footnote{$^{1)}$}
{Usually equation (1.3) is called Newton Graviational
Law, and equation (1.3a) Newton Second law. 
There is an alternative point of view that
Robert Hook formulated statement (1.3) 
in the letter to Newton,
and Newton who invented Calculus, 
deduced Kepler's first  law, statement (1.2),
solving differential equation (1.3a).}
(see calculations in the next paragraph, or in
any book on Theoretical Mechanics).


  How to come to the statement
(1.2) avoiding caclulations of integrals?

Instead conic section (1.0) one can consider
its orthogonal projection on the plane
   $z=0$. For example if
        $$
C\colon \cases
     {
        k^2x^2+k^2y^2=z^2\cr
         Ax+By+z=1\cr
       } 
        $$
then for its orthogonal projection
          $$
C_{proj}\colon \cases
     {
        k^2x^2+k^2y^2=(1-Ax-By)^2\cr
         z=0\cr
       } \,.
         \eqno (1.0)
         $$ 
The following statement which is almost 
evident plays crucial role in this etude:
        $$
\hbox{{\it orthogonal projections of conic sections
on the plane is also a conic sections.}}
\eqno (4)
        $$
In fact
          $$
  \hbox{orbits of planets are these projections....} 
            $$
\bigskip






  This etude has the following history.
Few months ago I was preparing the new lecture for 
Geometry students about conic sections.  
I had a task to explain statement (1.1),
that sections of conic surface with plane are conic sections
  (ellipses, parabolas or hyperbolas).    
The problem was 
 that I could not  find a beautiful and 
short explanation
of the statement (1.1) without 
using extracurricular 
material.  Another trouble was that at that time I had
 serious problems with eyes,
and my  access to books was very restricted.
Finally I prepared the following  explanation:
Instead statement (1.1) I can prove to students
the statement (1.4), then I can deduced  
statement (1.1) as the corollary of 
the statement (1.4).
Honeslty  considerations looked very bulky, 
I did not like them, but I had no choice.   
 
   I never forget early morning 30 March.  
 In five  hours the lecture will begin, 
where I have to explain to students the statement
(1.1) on conic sections.
I feel me very unhappy and unsatisfied with  
the way how I want to deliver the lecture, 
since the geometry of my considerations on the lecture
look very vague.
 Suddenly I ask me a question:  
  where are the foci of the projected conic section? 
(The foci of initial conic sections are at 
the points  where
 Dandellen spheres touch the plane which sects the conic surface.)  
   When the question was put, 
the answer come almost immediately:
  It is the vertex of the cone 
that is one of  the foci of projected
  conic section. This was crucial:
I immediately remembered, the May 2014.  


\medskip


\centerline {*****}

{\tt ...I am in the  
 Davis University,  California,
on 80 years celebration 

of my teacher 
 Albert S. Schwarz.
I meet there Alexander Givental. 

Alexander 
Givental is famous mathematician, 
but he is also
very much  

engaged  in 
teaching mathematics to kids


(see his homepage in Berkeley University).

During conference 
dinner we are sharing the same table, and Alexander

is explaining me some beautiful 
properties of conic section.

In  particular he is telling  
me  the sentence: 

 if you consider  a projection of
ellipse on the horisontal plane, 

you come naturally to
Kepler's law...
}
 
\centerline {*****}


  {\sl Recalling this phraze in 
the morning 30 March 2017,
 I immediately realised the geometrical meaning
of my construction (1.4), which I made preparing the lecture, and I understood what Alexander Givental wanted
to tell me three years ago.   
} 


   Next day  I contacted
Alexander by e-mail. He immediately sent me  
the detailed answer. 
%on my question,
%article [3], and the 
%letter about Lagrange considerations; all the material
%that this etude is based on.

   First, Alexander sent me the article [3]. 
In this article he  gives detailed
geometrical explanation why trajectories
of planets are conic section, and he does not use
the caclulus. You can find this article 
on the homepage
of  A.Givental. Thirty years ago, an abstract of this 
article was included by V. I. Arnold into 
his joint with V. V. Kozlov and A. I. Neishtadt 
survey of classical mechanics [2]. 

   In this letter  A. Givental also told me that
Alain Chensiner noted him that his (A.G.) 
considerations are very close to Lagrange's proof
(see the paper [1]), and Alexander sent me the 
Lagrange article [1], and the letter
where he explained this. 

I do not want here to retell the work [3]
of A.Givental. (See however the remark at the end of the text.) 
I just recommend all of you 
to read this wwonderful paper (it is on his homepage).
  Here I just  
try based on the letter of A.Givental
to  retell the simple proof of
statement (1.2) avoiding calculus,
and compare this proof with the sandard one.
This simple proof can be traced to the work [1]
of Lagrange.  Of course the work [3]
contains complete picture of these considerations.
 



My modest contribution to this etude 
is related with the fact that trying
to find a  simple proof of classical statement
(1.1), I realised the importance
of the statement (1.4), and on the base of
it I try to retell here the constuctions
of the papers [1] and [3].

 
 
\bigskip


\centerline {${\cal x}2$ \bf  Standard 
exlanation of
  Kepler's Law.}

{\it We will give
the standard proof based on the calculus, that
 trajectory of particle in gravitational filed is
a conic section}

  Let $\R=\R(t)$ is a solution of differential equation
(1.3a):
      $$
  {d^2\over dt^2}\R={{\bf F}\over m}=
     -\gamma{M_{\rm sun}\R\over R^3}\,,
      \eqno (2.1)
           $$
 Consider the  angular momentum
      $$
\M=m\dot \R\times\R
      \eqno (2.2)\,.
        $$
It is preserved: 
$\dot M=m\dot\R\times \dot \R+
m{\buildrel ..\over\R }\times\R=
0+\left( 
 -\gamma{mM_{\rm sun}\R\over R^3}
\right)\times\R=0$.

   Vector $\R(t)$ is orthogonal to the constant
vector $\M$.
Consider Cartesian coordinates $x,y,z$
such that $\M$ is directed along axis $OZ$,
and $\R(t)=x(t)\e_x+y(t)\e_y$ belongs to the plane
$OXY$:
          $$
\R(t)=x(t)\e_x+y(t)\e_y\,,\quad
 \M(t)=\mu\e_z\,, (\mu\,\, \hbox{is constant})\,.
           \eqno (2.3)
           $$  
Let $\varphi,r$ be polar coordinates
in the plane $OXY$, $x=r\cos\varphi,y=r\sin\varphi$.

Notice that in polar coordinates
      $$
\mu=m(x\dot y-y\dot x)=
r\cos\varphi(\dot r\sin\varphi+r\dot\varphi\cos\varphi)-
r\sin\varphi(\dot r\cos\varphi-r\dot\varphi\sin\varphi)=
  mr^2\dot\varphi\,,
        (2.4)
        $$

This is integral of motion.

Equations (2.1) imply that
  for coordinate $r=\sqrt{x^2+y^2}$ 
       $$
 {\buildrel ..\over r}={d\over dt}\dot r=
  {d\over dt}\left({x\dot x+y\dot y\over r}\right)=
{x{\buildrel ..\over x}+y{\buildrel ..\over y}\over r}+
    {\dot x^2+\dot y^2\over r}-
    {\left(x\dot x+y\dot y\right)^2\over r^3}=
       $$
      $$
{x\left(-\gamma{M_{\rm sun}x\over r^3}\right)+
y\left(-\gamma{M_{\rm sun}y\over r^3}\right)\over r}+
{(\dot x^2+\dot y^2)(x^2+y^2)-
 (x\dot x+y\dot y)^2\over r^3}
        =
        $$
       $$
-\gamma{M_{\rm sun}\over r^2}+
{(x\dot y-y\dot x)^2\over r^3}=
-\gamma{M_{sun}\over r^2}+{\mu^2\over m^2r^3}\,,
\eqno(2.5)          
       $$
i.e.
      $$
{d\over dt}\left({\dot r^2\over 2}\right)=
\dot r{\buildrel ..\over r}=
-\gamma{M_{\rm sun}\over r^2}\dot r+
{\mu^2\over m^2 r^3}\dot r=
   {d\over dt}\left(\gamma{M\over r}-
        {\mu^2\over 2m^2r^2}\right)\,.
      $$
Thus we come to the second integral of motion:
     $$
{d\over dt}\left({m\dot r^2\over 2}-
\gamma{mM_{\rm sun}\over r}+
{\mu^2\over 2mr^2}\right)=0
 \Rightarrow
{m\dot r^2\over 2}-
\gamma{mM_{\rm sun}\over r}+
{\mu^2\over 2mr^2}=E\,.
\eqno (2.6) 
     $$
We come to first order differential equations
\footnote{$^{2)}$}{
     $E={m\dot r^2\over 2}-
\gamma{mM_{\rm sun}\over r}+
{\mu^2\over 2mr^2}$ is energy, 
${m\dot r^2\over 2}$ is kinetik enery, and
  $U_{\rm eff}=-
\gamma{mM_{\rm sun}\over r}+
{\mu^2\over 2mr^2}$ is effective potential energy}:
            $$
\cases
               {
 {dr(t)\over dt}=\sqrt {2\over m}\sqrt{
  E+\gamma{mM_{\rm}\over r}-{\mu^2\over 2mr^2}}\cr
   {d\varphi(t)\over dt}={\mu\over mr^2}\cr   
              }\Rightarrow
   {d\varphi\over dr}=
        {
{\mu\over r^2}\over
\sqrt{
  2mE+\gamma{2mM_{\rm sun}\over r}-{\mu^2\over r^2}}
       }
        \eqno (2.7)
            $$
thus
        $$
\varphi=\int
           {
{\mu dr\over r^2}\over
\sqrt{
  2mE+\gamma{2m^2M_{\rm sun}\over r}-{\mu^2\over r^2}}
       }
         \eqno (2.8)
        $$
Integrating we come to 
$$\int
           {
{\mu dr\over r^2}\over
\sqrt{
  2mE+\gamma{2m^2M_{\rm sun}\over r}-{\mu^2\over r^2}}\,,
       }=
       -\int 
         {\mu  d\left({1\over r}\right)\over
\sqrt{
  \gamma^2{m^2M^2_{\rm sun}\over \mu^2}+2mE-
\mu^2\left({1\over r}-
    \gamma{m^2M_{\rm sun}\over \mu^2}\right)^2
             }
       }=
$$
      $$
{\rm arcsin\,}
            \left\{
   \left(
     {\mu\over 
         \sqrt{
   \gamma^2{m^2M^2_{\rm sun}\over \mu^2}+2mE}
           }
       \right)
    \left({1\over r}-
    \gamma{m^2M_{\rm sun}\over \mu^2}\right)\,
            \right\}
          \eqno (2.9)
      $$ 
i.e.
     $$
\sin\varphi=
     {\mu\over 
         \sqrt{
   \gamma^2{m^2M^2_{\rm sun}\over \mu^2}+2mE}
           }
    \left({1\over r}-
    \gamma{m^2M_{\rm sun}\over \mu^2}\right)=
{1\over \sqrt{1+{2E\mu^2\over \gamma^2M^2_{\rm sun}m^3}}}
        \left(
          {\mu^2\over\gamma m^2 M_{\rm sun}}
             {1\over r}-1
        \right)
\eqno (2.9b)
     $$
Introduce notations:
         $$
e=\sqrt{1+{2E\mu^2\over \gamma^2M^2_{\rm sun}m^3}}\,,
   k={\mu^2\over\gamma m^2 M_{\rm sun}}\,,
          $$
then (App1.9b) has appearance:
        $$
e\sin\varphi={k\over r}-1
  \eqno (2.10)
        $$
This is conic section with excentricet $e$,
and one of the foci at the origin.

{\bf Remark} {\tt in fact it has to be $\cos\varphi$,
not $\sin\varphi$}

\bigskip

\centerline {${\cal x}3$ \bf  Simple exlanation of
  Kepler's Law (avoiding calculations of integrals)}

In the previous paragraph we 
performed geometrical considerations
(equations (2.1)---(2.5)).
On the base of these equations
we formulated first order differential equations
(2.7),
solved these differential equations
using calculus (see (2.8), (2.9))
and come to equation (2.10) of conic section.

Now  we will return
to the considerations 
(2.1)---(2.5) of the previous
paragraph, and on the base of these
considerations we will come to
equation (2.10) avoiding the differential
equations and integrals (2.6)---(2.9). 
Without calculating explicitly integrals 
(2.9), (2.9a)) we just will show that
 the solution
  $\R(t)$ is a conic section, i.e.  it obeys
   equation (2.10).



Choose an arbitrary solution $\R_{_0}(t)$ 
of equation (2.1).
In coordinates (2.3)
         $$
\R_{_0}(t)=x_{_0}(t)\e_x+y_{_0}(t)\e_y
                 \eqno(3.3)
         $$
We suppose that angular momentum 
$\M=\mu\e_z\not=0$.

   Writing the equation (2.1)
in components we come to
            $$
  \pmatrix{{\buildrel ..\over x_{_0}(t)}\cr 
  {\buildrel ..\over y_{_{0}}(t)}\cr}=
      -\gamma{M_{\rm sun}\over\left(x^2_{_0}(t)+
     y^2_{_0}(t)\right)^{3\over 2}}
\pmatrix{x_{_0}(t)\cr y_{_0}(t)\cr}\,.
     \eqno (3.4) 
             $$
Now consider the pair of differential equationts
on function $f=f(t)$

1) homogeneous second order 
linear differential equation
     $$
{d^2 f\over dt^2}=-\gamma{M_{\rm sun}
         f\over r^3_{_0}(t)}\,,
\eqno (3.5)
     $$ 
and  associated with it,
non-homogeneous second order 
linear differential equation
             $$
{d^2 f\over dt^2}=-\gamma{M_{\rm sun}
    f\over r^3_{_0}(t)}+
    {\mu^2\over m^2r^3_{_0}(t)}\,,
    \eqno (3.5a)
             $$
where function $r_{_0}(t)$ is defined by
solutions $x_{_0}(t)$, $y_{_0}(t)$ (3.4):
           $$
r_{_0}(t)=\sqrt{x^2_{_0}(t)+y^2_{_0}(t)}\,.
             \eqno (3.6)
           $$

Notice that $x_{_0}(t)$, $y_{_0}(t)$
are two independent solutions of second
order homogeneous equation
(we suppose that angular momentum 
$\mu=x\dot y-y\dot x$ does not vanish),].

   Notice also that equation
  (2.5) implies that function 
    (3.6) is the solution of 
non-homogeneous equation
       (3.5a). Indeed due to equation (2.5)
we have:
           $$
{d^2 r_{_0}(t)\over dt^2}=
{d^2 \over dt^2}
     \left(
    \sqrt{x^2_{_0}(t)+y^2_{_0}(t)}
      \right)=
     -\gamma{M_{\rm sun}\over r_{_0}^2(t)}+
    {\mu^2\over m^2 r^3_{_0}(t)}
             =\,.
     -\gamma{M_{\rm sun}r_{_0}(t)\over r_{_0}^3(t)}+
    {\mu^2\over m^2 r^3_{_0}(t)}\,.
           $$
Non-homogeneous equation (3.5a)
has also the following trivial solution:
constant function 
       $$
k=g_0={\mu^2\over \gamma m^2 M_{\rm sun}}\,.
\eqno (3.7)
        $$
The difference of solutions (3.6)
and (3.7) is a solution of homogeneous
equation (3.5). Thus we come to the equation:
        $$
r_{_0}(t)-g_0=
Ax_{_0}(t)+By_{_0}(t)\,.
        \eqno (3.8)
    $$
where $A,B$ are constants, i.e.
        $$
 r_{_0}(t)=k+
Ar_{_0}(t)\cos\varphi_{_0}(t)+
Br_{_0}(t)\sin\varphi_{_0}(t)=
 k+er_{_0}(t)\sin(\varphi_{_0}(t)+\delta)\,,
\quad e=\sqrt{A^2+B^2}\,,\tan\delta={B\over A}
\eqno (3.8)
        $$
Thus $r_{_0}(t), \varphi_{_0}(t)$ obeys equation 
        (2.10) for conic
with focus at the origin,

Notice that the curve (3.9) is the projection
of conic section
    $$
C\colon\quad \cases
            {
      x^2+y^2=z^2\cr
      z=Ax+By
          }
    $$
on the plane $z=-k$.

\m


{\bf Remark}[3].  Geometrical meaning of considerations (3.5)-(3.8)
is the following:  Consider
                    $$
\R(t)=\R_{_0}(t)+(r_{_0}(t)-g_{_0})\e_z
                    $$
Then previous calculations imply that
 $\R(t)$ also obeys equation (2.1).
 `momentum' $\R\times \dot \R$ is preserved.
We see that $\R(t)$ is intersection of plane
with cone surface $x^2+y^2-z^2=0$,
and $\r_{_0}(t)$ its projection, the projection
of conic. (See for details [3]).
       
\bigskip






[1] Lagrange  {\it DES PERTURBATIONS DES COMETES.}---
            SECTION DDEUXIEME.  Integrations des equations differentielles
de l'orbite non-altere. pp.419---430,  1785


[2] V. I. Arnold, V. V. Kozlov, A. I. Neishtadt
{\it  Mathematical aspects of classical 
and celestial mechanics.}
 Dynamical systems – III 
(Encyclopaedia of Mathematical Sciences), 
Springer, 1987.



[3] Alexander Givental 
{\it Kepler’s Laws and Conic Sections}
  Arnold Mathematical Journal March 2016, 
Volume 2, Issue 1, pp 139–148

see in homepage of the author:
math.berkeley.edu$\backslash$
$~$
giventh$\backslash$
kepler$\_$091615.pdf


\bye

://math.berkeley.edu/~giventh/kepler_091615.pdf

