% I commenced  this version on 22 February 2009
\magnification=1200 %\baselineskip=14pt
\def\vare {\varepsilon}
\def\A {{\bf A}}
\def\t {\tilde}
\def\a {\alpha}
\def\K {{\bf K}}
\def\N {{\bf N}}
\def\V {{\cal V}}
\def\s {{\sigma}}
\def\S {{\Sigma}}
\def\s {{\sigma}}
\def\p{\partial}
\def\vare{{\varepsilon}}
\def\Q {{\bf Q}}
\def\D {{\cal D}}
\def\G {{\Gamma}}
\def\C {{\bf C}}
\def\M {{\cal M}}
\def\Z {{\bf Z}}
\def\U  {{\cal U}}
\def\H {{\cal H}}
\def\R  {{\bf R}}
\def\E  {{\bf E}}
\def\l {\lambda}
\def\degree {{\bf {\rm degree}\,\,}}
\def \finish {${\,\,\vrule height1mm depth2mm width 8pt}$}
\def \m {\medskip}
\def\p {\partial}
\def\r {{\bf r}}
\def\v {{\bf v}}
\def\n {{\bf n}}
\def\t {{\bf t}}
\def\b {{\bf b}}
\def\c {{\bf c }}
\def\e{{\bf e}}
\def\ac {{\bf a}}
\def \X   {{\bf X}}
\def \Y   {{\bf Y}}
\def \x   {{\bf x}}
\def \y   {{\bf y}}
\def \G{{\cal G}}

\centerline  {\bf  Afiine and linear connections}

%"Vek zhivi vek uchish---durakom pomrjoshj"---how often I need to
%remember this Russian expression learning mathematics.
  Linear connection are known to everybody in theoretical
  physics under the name of so called "Christofells symbols". Here I will try to explain what is it  affine connection and its relations to
  linear  connections.  Respectively we will try to consider relations between
   Euclidean connection and Levi-Chivita connection.

  \bigskip
         \centerline {${\cal x} 0\,\,$\bf  Some standard stuff}


 Let $(P,\pi,M, G)$ be a principal bundle over manifold $M$  with structure group $G$.
 $M$ is a base, $\pi$ is a projection on the base
 $M$, Lie group $G$ is a structure group.  Structure group $G$ acts on the right transitively and free on all fibres
 $\pi^{-1}(x)$. The base can be covered by open sets $\{V_\a\}$ $M=\cup_\a V_\a$ such that
 all $\pi^{-1}V_\a$ are homeomorphic to $V_\a\times G$:
                            $$
                          \pi^{-1}(V_\a)\,
                          \longleftrightarrow\, V_\a\times G  \eqno (0.1)
                            $$
Let $\{p_\a(x)\}$ ($p_\a(x)\colon V_\a\rightarrow P$ such that $\pi\circ p_\a={\bf id}\vert_{V_\a}$)
  be the collection of local sections \footnote{$^*$}{One can define local section over $V_\a$ by formulae $p_\a(x)=\psi_\a(x,g_\a(x))$ where $\psi_\a$
establish homeomorphism (0.1) and $g_\a(x)$ are arbitrary (smooth) functions, e.g. $g_\a(x)\equiv 1$.}.
  Thus we come to local coordinates $\{(x^\mu_{(\a)},g_{(\a)})\}$ on the domains $\{\pi^{-1}(V_\a)\}$ of the bundle:
                         $$
              (x^\mu_{(\a)},g_{(\a)}){\buildrel\varphi_\a\over \longmapsto} p_\a(x)\circ g
              \eqno (0.2)
                         $$
where
$x^\mu_{(\a)}$ are local coordinates of points in $V_\a$ (in some
atlas on $M$), $\mu=1,2,\dots,n$, where $n$ is a dimension of $M$
and $g_{(\a)}\in G$, (index $\a$ later will be often omitted)\footnote{$^{**}$}{the homeomorphisms $\{\psi_\a\}$
in (0.1) provide coordinatisation for the choice of local sections $p_\a(x)=\psi(x,e)$ (see the first footnote)}.
With some abuse of notations
we often identify points $\bf x$ in $V_\a$ with there local coordinates $x^\mu_{(\a)}$.


Consider local transition functions
$\Psi_{\a\beta}\colon \,\,(V_\a\cap V_\beta)\times G\rightarrow (V_\a\cap V_\beta)\times G$:
$\Psi_{\a\beta}=\varphi^{-1}_\a\circ \varphi_\beta$.
      One can see that they following:
                   $$
               \Psi_{\a\beta}(x,g)=(x,h_{\a\beta}(x)\circ g),
               \quad \hbox {where $h_{\a\beta}(x)\colon\,\, p_\beta(x)=p_\a(x)\circ h_{\a\beta}$}
               \eqno (0.3)
                   $$
Indeed
                  $$
   \Psi_{\a\beta}(x,g)=\varphi^{-1}_\a\circ \varphi_\beta(x,g)=\varphi^{-1}_\a(p_\beta(x)g)=
        \varphi^{-1}_\a(p_\a(x)h_{\a\beta}(x)g)=(x,h_{\a\beta}(x)g)
                      \eqno (0,3)
                            $$

{\bf Remark} Please note that in the equation (0.3) the action of group on fibre coordinates is
left multiplication.---
It is passive action of changing coordinates of the fibre.
 In the definition of the fibre bundle there is right action of the group: it is an active action-- the point
 is changed.

              \centerline {\it Connection in the principal bundle}

  One can define a connection---one-form
  $\Omega$
  on $P$ with values in the Lie algebra $\G(G)$ of the group $G$
  such that


  \item {1.} $\Omega(\tilde \xi)=\xi$ if the vertical vector $\tilde \xi$ is tangent to the curve
                    $p\circ (1+t\xi+\dots)$ at the point $t=0$

  \item{2.} It is invariant with respect to the action of group $G$:   $R^*_g\Omega=Ad_{g^{-1}}\Omega$


    We say that the vector ${\bf X}\in TP$ is {\it horisontal} vector if $\Omega({\bf X})=0$.



      The tangent space $T_pP$ for an arbitrary point $p$ in the bundle $P$
      is a direct sum of vertical subspace and horisontal subspace:
                     $$
            T_pP=T^{^{\cal c}}_pP\oplus T^{^{||}}_pP\qquad {\rm and}\,\,\, R_g T^{^{||}}_p=T^{^{||}}_{pg}
                     $$

 In local coordinates
                       $$
              \Omega=g^{-1}dg-g^{-1}A^{(\a)}_{\mu}dx^ug
                       $$
where $A^{(\a)}_{\mu}dx^u$ is local one-form with values in the Lie algebra $\G(G)$.
($A^{(\a)}_{\mu}(x)$ is sometimes called  Yang Mills field.)


         \centerline {${\cal x} 1\,\,$\bf  Linear connection on the linear frames bundle }


  Let $M$ be a manifold and $P(M)$ the principal bundle of linear frames, i.e.
        $$
     \e^\mu_{_{i}}(x(t))=e^\mu_{0_{i}}g(t)
        $$
be a parallel transport along a curve $x(t)$ on the base, where $g(t)$ obeys the differential equation:
           $$
      {d\over dt}g(x(t))={dx^\mu(t)\over d t}A_\nu(x(t))g
           $$
(We mean that group is matrix group, thus $A_\nu(x(t))g=A^i_{\nu k}(x(t))gk_m$)

    The solution of this equation is
            $$
        g(t)=P\exp \left(\int_0^t {dx^\mu(t)\over d t}A_\nu(x(t))dt)\right)
            $$
(Explain what it means.)

    Now


\bigskip


\bye
