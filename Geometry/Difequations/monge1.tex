
% I began this file on 18 July 2016



\magnification=1200

\baselineskip=17pt



\def\vare {\varepsilon}
\def\A {{\bf A}}
\def\t {\tilde}
\def\a {\alpha}
\def\K {{\bf K}}
\def\N {{\bf N}}
\def\V {{\cal V}}
\def\s {{\sigma}}
\def\S {{\bf S}}
\def\s {{\sigma}}
\def\bs {{\bf s}}
\def\p{\partial}
\def\vare{{\varepsilon}}
\def\Q {{\bf Q}}
\def\D {{\cal D}}
\def\P {{\cal P}}
\def\S {{\cal S}}
\def\L {{\cal L}}
\def\G {{\Gamma}}
\def\C {{\bf C}}
\def\M {{\cal M}}
\def\Z {{\bf Z}}
\def\U  {{\cal U}}
\def\H {{\cal H}}
\def\R  {{\bf R}}
\def\E  {{\bf E}}
\def\l {\lambda}
\def\degree {{\bf {\rm degree}\,\,}}
\def \finish {${\,\,\vrule height1mm depth2mm width 8pt}$}
\def \m {\medskip}
\def\p {\partial}
\def\r {{\bf r}}
\def\v {{\bf v}}
\def\n {{\bf n}}
\def\t {{\bf t}}
\def\b {{\bf b}}
\def\e{{\bf e}}
\def\f{{\bf f}}
\def\ac {{\bf a}}
\def \X   {{\bf X}}
\def \D   {{\cal D}}
\def \Y   {{\bf Y}}
\def\diag {\rm diag\,\,}
\def\pt {{\bf p}}
\def\w {\omega}
\def\la{\langle}
\def\ra{\rangle}
\def\x{{\bf x}}
\def\m {\medskip}
\def\thick {{\buildrel \to\over \to}}

  \centerline{\bf Monge cone and characteristic for equation
        $F(x,u,p,q)=0$}

  Consider partial differential equation
             $$
          F(x,y,u,p,q)=0
             $$
on function $u=u(x,y)$, where $p=u_x, q=u_y$.

Function $u$, solution of this equation defines surface $u_F$.

Differential equation is a function on space of $1$-forms


  Pick an arbitrary point $\pt$ at the surface $M_u$.

The set of all $1-forms$ at the space  $T^*_\pt R^3$
which obey the equation $F\big\vert_\pt=0$ defines
one-parametric curve in projective space $P^{*2}$.
  This curve is Monge cone.

The dula of this curve, the curve in $P^2$
defined by vectors, directions, which 
annihilate these forms.

   This is {\it dual Monge cone}.




    Calculations:
    
Consider projective curve  of $1$-forms $[a(s):b(s):c(s)]=[p(s):q(s):-1]$
such that $F(p,q)=0$ (at the point $\pt$, i.e.) 

  It annihilates tangent vectors. A curve
   $[\a(s):\beta(s):\gamma(s)]$ dual to the curve
   $[a(s):b(s):c(s)]$ can be defined by relation
                 $$
  \det\pmatrix
         {
         a(s) &\dot a(s)  &\alpha (s)\cr
         b(s) &\dot b(s)  &\beta (s)\cr
         c(s) &\dot c(s)  &\gamma (s)\cr
          }=0
                 $$ 
  For the Monge curve $[p(s):q(s):-1]$ we have
                  $$
\det\pmatrix
         {
         p(s) &\dot p(s)  &\alpha (s)\cr
         q(s) &\dot q(s)  &\beta (s)\cr
         -1 & 0  &\gamma (s)\cr
          }=0\Rightarrow 
    \left[\matrix{\a(s)\cr\beta(s)\cr\gamma(s)\cr}\right]=
    \left[
           \pmatrix{ p(s)\cr q(s)\cr -1\cr}
                   \times
         \pmatrix{\dot p(s)\cr\dot q(s)\cr 0\cr}
               \right]=
    \left[\matrix{ \dot q(s)\cr-\dot p(s)\cr p(s)
        \dot q(s)-q(s)\dot p(s)\cr}\right]\,.
                 $$        
Recalling that
                 $$
           F_p\dot p+F_q\dot q=0
                 $$
we come to
               $$
[\a(s):\beta(s):\gamma(s)]=
    \left[ \dot q(s): -\dot p(s): 
         p(s)\dot q(s)-q(s)\dot p(s)\right]=
        [F_p:F_q:pF_p+qF_q]
               $$
This is characteristic, more precisely the projection of
bicharacteristic on the points space.

  For every point $\pt=(x,y,u)$ and for every point 
 $[p:q:-1]$ on the Monge cone over point $\pt$
  (Monge cone is one-parametric curve in projective space)
   there is a direction 
            $$
   [dx:dy:du]=[F_p:F_q:pF_p+qF_q]\,,
            $$
which is a point
on the dula curve corresponding to the point $[p:q:-1]$.

This direction is nothing is defined by the projection
of bicharacteristic 
         $$
          \X=K_F=F_p
          $$
(see the file on bicharacteristics) 
on the points space.
  
  We defined in the previous file $\X$
as a field sympelctorthogonal to tnagent vector fields
which annihilate the contact strucutre.

 Here we will calculate the bicharacteristics $\X$
direction (i.e. vectgor field (up to multiplier)
just using Monge cone considerations.

For every point $(x,y,u, p,q)$ on the manifold
  $M_f$  we will define
       $$
        \pmatrix {\dot x\cr\dot y\cr\dot u\cr \dot p \cr \dot q\cr}
       $$

We consider curve 
       $$
        \pmatrix {x(s)\cr y(s)\cr u(s)\cr p(s) \cr q(s)\cr}
       $$
We have already that
         $$
   [\dot x:\dot y:\dot u]=[F_p:F_q: pF_p+qF_q]
          $$
It is convenient to choose  $\dot x=F_p, \dot y=F_q$
(we calculate up to a multiplier.)
   Now caclulate $\dot p, \dot q$. 
          $$
{dp(s)\over ds}={d\over ds}u_x(x(s,y(s)))=p_x x_s+p_y y_s
          $$
Use the fact that 
             $$
p_y=q_x\,,\, \quad\hbox {Lagrangian surface}
              $$
We have
    $$
{dp(s)\over ds}={d\over ds}u_x(x(s,y(s)))=p_x x_s+p_y y_s=
        p_xF_p+p_yF_q=p_xF_p+q_x F_q=
          $$
              $$
(F_x+F_uu_x+F_pp_x+F_qq_x)-F_x-F_uu_x\,.
               $$
Now note that 
             $$
 (F_x+F_uu_x+F_pp_x+F_qq_x)={d\over dx}\left(F(x,y,u,p,q)\right)_
{u=u(x,y)\,,p=p(x,y)\,,q=q(x,y)}=0\,.
             $$
Hence we come to
                    $$
{dp(s)\over ds}=-F_x+F_uu_x\,.
               $$
In the same way we calculate
                 $$
{dq(s)\over ds}=-F_y+F_uu_y\,.
               $$
nd we come to bicharacteristic direction
              $$
  dx:dy:du:dp:dq=F_p:F_q:pF_p+qF_q:-(F_x+u_xF_u):-(F_x+u_xF_u)
              $$
            This is in accordance with the fact that bicharacteristic 
vector field is equal to  
             $$
\X=K_F=F_p{\p \over \p x}+F_q{\p \over \p y}-
   \left(F_x+u_xF_u\right){\p\over \p p}+
   \left(F_y+u_yF_u\right){\p\over \p q}+
      \left(pF_p+qF_q\right){\p \over \p u}
             $$ 
If we ingnore $u$ we come to Hamiltonian vector field.

{\bf Example}  Consider linear case
      $$
au_x+bu_y=c, 
      $$
i.e. $F=ap+bq-c$.
In this case answer is obvious without any calculation:
Characteristic vector direction is  $$
a:b:c=F_p:F_q:pF_p+F_qF_q=p:q:ap+bq=p:q:c
       $$
The equation defines $1$-parametric set of $1$-forms
$pdx+qdy-du$ such that $ap+bq=c$. They define the line
in projective space $P^{*2}$. The dual to this line
will be a point in $P^2$ defined by direction
 $a:b:c$.

\bye
