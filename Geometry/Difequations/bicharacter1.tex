


\magnification=1200

\baselineskip=17pt



\def\vare {\varepsilon}
\def\A {{\bf A}}
\def\t {\tilde}
\def\a {\alpha}
\def\K {{\bf K}}
\def\N {{\bf N}}
\def\V {{\cal V}}
\def\s {{\sigma}}
\def\S {{\bf S}}
\def\s {{\sigma}}
\def\bs {{\bf s}}
\def\p{\partial}
\def\vare{{\varepsilon}}
\def\Q {{\bf Q}}
\def\D {{\cal D}}
\def\P {{\cal P}}
\def\S {{\cal S}}
\def\L {{\cal L}}
\def\G {{\Gamma}}
\def\C {{\bf C}}
\def\M {{\cal M}}
\def\Z {{\bf Z}}
\def\U  {{\cal U}}
\def\H {{\cal H}}
\def\R  {{\bf R}}
\def\E  {{\bf E}}
\def\l {\lambda}
\def\degree {{\bf {\rm degree}\,\,}}
\def \finish {${\,\,\vrule height1mm depth2mm width 8pt}$}
\def \m {\medskip}
\def\p {\partial}
\def\r {{\bf r}}
\def\v {{\bf v}}
\def\n {{\bf n}}
\def\t {{\bf t}}
\def\b {{\bf b}}
\def\e{{\bf e}}
\def\f{{\bf f}}
\def\ac {{\bf a}}
\def \X   {{\bf X}}
\def \D   {{\cal D}}
\def \Y   {{\bf Y}}
\def\diag {\rm diag\,\,}
\def\pt {{\bf p}}
\def\w {\omega}
\def\la{\langle}
\def\ra{\rangle}
\def\x{{\bf x}}
\def\m {\medskip}
\def\thick {{\buildrel \to\over \to}}

  \centerline{\bf Characteristics and bicharacteristics.
      Solution of equation $\Phi(x,u,p)=0$}

\centerline {Introduction: Characteristic}

  Let us recall simple things.

Characteristic is vector field (direction) tangent to the surface.
It appears naturally when we solve linear differential equation
         $$
      A^a\p_a u=C
         \eqno (1)
         $$
Solution $u=u(x)$ of this equation defines hypersurface
  $M_u$ in $\R^{n+1}$ (with coordinates $(x^1,\dots,x^n,u)$).
   Equation (1) can be rewritten as
          $$
M\colon\quad\L_\X M_u=0, \quad {\rm where} \,\, \X=A^a\p_a-C\p_u\,
\hbox{is {\it characteristic}}\,.
          $$
    Let $N=N^{n-1}$ be $n-1$-dimensional surface in $\R^{n+1}$.
  which is transversal to vector field $\Y$. Then taking
  exponents of vector field $\X$ we come to the surface
  $M$  defined in a vicinity
  of the surface $N$.

Typical Cauchy problem: Let  $\phi$ be a function defined
on the $n-1$-dimensional subspace $L=L^{n-1}\subset \R^n$. Find function
$u$ on domain in $\R^{n+1}$, solution of equation (1) 
such that $u=\varphi$ on this subspace. 
{\sl Solution} Take a $n-1$-dimensional surface
 $Y$ defined by the function $\varphi$ on $L$. If this surface
is transversal to characterisic  

\smallskip

Now we go to more delicate matter--bicharacteristic.

First two very important definition.

Let $M=M^n$ be an arbitrary $n$-dimensional manifold.
Consider $2n+1$ dimensional space of first jets $J_1=J_1(M^n,\R)$
with local coordinates $(x^a,p_b,u)$.
Usually we consider  $M^n=\R^n$, $n$-dimensional affine space,

One can define on $J^1(M,\R)$ the contact $1$-form
$\a$:
        $$
   \a=du-p_adx^a
        $$
It is is easy to see the invariant meaning of this form:
 value of $\a$ at every vector $\xi$ attached
at the point $\pt$ is equal to the value of the form $\pt$
at the projection of this vector on $M^n$.

 
 Let $\Phi=\Phi(x^a,p_a,u)$ be an arbitrary (smooth function)
in the $2n+1$ dimensional space of first jets $J_1=J_1(\R^n,\R)$
with coordinates $(x^a,p_b,s)$.
We assign to this function the following ector field on $J^1$:
            $$
   D_\Phi=
  {\p \Phi(x,p,u)\over \p p^a}{\p\over \p x^a}-
       \left(
  {\p \Phi(x,p,u)\over \p x^a}+p_a
  {\p \Phi(x,p,u)\over \p u}
        \right)
      {\p\over \p u}+
         p_a
  {\p \Phi(x,p,u)\over \p p^a}{\p\over \p u}\,.
     \eqno (2.0)
            $$
We explain the geometrical meaning of this formula.
At every point $\pt=(x,p,u)$ of Jets space the function
$\Phi$ defines the $2n$-dimensional
tangent space $D_\pt$ of vectors which
annihilate the form $d\Phi$ and the
$2n$-dimensional tangent space $K_\pt$ of vectors
which annihilate the contact form $\a$. 
The intersection of these vector spaces
$K_\pt$ and $D_\pt$ is $2n-1$-dimensional
vector space $P_\pt$
of vectors which annihilate 1-form $d\Phi$ and 
contact $1$-form $\a$:
         $$
  P_\pt=D_\pt\cap K_\pt\,.
         $$
Now notice that $P_\pt$ is hyperspace of symplectic space  $K_\pt$.
Vector $\X$ is vector in $K_\pt$ which is symplectoorthogonal
to $K_\pt$. It is defined up to multiplier.
In the special case if $2n$-dimensional planes are not transversal,
then $\X=0$. 
 
 Consider non-linear first order partial differential equation
       $$
  \Phi\left(x^a,u, {\p u\over \p x^b}\right)=0
      \eqno (2.1)
       $$
on function $u(\x)$ ($\x\in \R^n$).

  Solution of this equation $u=S(x)$ defines $n$-dimensional
surface, integral of two distributions----distribution
$\D$ of $2n$-dimensional planes which annihilate $d\Phi_u$
and distribution $\K$ of planes which annihilate  contact form $\a$.

Thus at every point (except special points) it is defined
vector field $D_\Phi$ tangent to manifold $M_\Phi$.

  This vector field is bicharacteristics.

It is useful write down bicharacterstic in the special
case when  function $\Phi(x,,p,u)=H(x,p)$ does not depend
explicitly on $u$:
                $$
                \matrix
        {
\hbox {Equation $\Phi(x,{\p S\over \p x^i}, S)=0$}&
\hbox {Equation $H(x,{\p S\over \p x^i})=0$}\cr
\hbox { hypersurface  $\Phi(x,p, u)=0$ }&
\hbox {Equation $H(x,p)=0$ }\cr
\hbox {  in $2n+1$-dim. jets space $J^1(\R^n,R)$}&
\hbox { in $2n$-dimensional space}\cr
\hbox {  contact $1$-form $\a=du-p_a dx^a$}&
\hbox {symplectic $2$-form $\w=dp_a\wedge dq^a$}\cr
   \hbox{for an arbitr. gen.  point $\pt$ on a surface $\Phi=0$}&
   \hbox{for an arbitr. gen. point $\pt$ on a surface $H=0$}\cr
\hbox {$D_p$ is $2n$-dim. plane orthogonal to $d\Phi$  }&
\hbox {$D_p$ is $2n-1$-dim. plane orthog. to $dH$  }\cr
\hbox {$K_\pt$ is $2n$-dim.plane orthog. to contact form $d\a$}&
\hbox { all the tangent plane to $H=0$ }\cr
\hbox{bicharacteristic $\X_\Phi$}&
\hbox {bicharacteristic $D_H$}\cr
\hbox {symplectoorthog. to $K_\pt$ and $D_\pt$  }&
\hbox {symplectoortog. to $D_\pt$}\cr
\hbox {  }&
\hbox {}\cr
       }
        $$
  Properties of bicharacteristic field.

 \smallskip
      $$
\L_{D_\Phi}\a=d\Phi+\Phi_u\a\,.
      \eqno (***)
      $$
  


  Let $Y^{n-1}$ be a surface in $V^n$ and $\varphi$
function on $Y^{n-1}$

  We have to find function $u$ such that
       $$
    \Phi(x,u,p)=0\,,, u\big\vert_{Y}=\varphi
       $$

`{\sl Solution}

Characteristic vector $\X=D_\Phi$ field,
       $$
  \X=-{\p \Phi\over \p p_m}{\p\over \p_m }+
   \left({\p \Phi\over \p x^m}+p_m{\p \Phi\over \p y}\right)
       {\p\over \p }
       +
    p_r{\p \Phi\over \p p_r}{\p\over \p u}
       $$
is

  i) tangent to the surface $M$

 ii) it vanishes contact form

  iii) it is symplectoorthogonal to all vectors which are contact and
tangent

  Due to relation (***) the flux of this vector field
preserves all this stuff, i.e. applied to initial surface
$Y^{n-1}$ we come to $n$-dimensional surface.



\bye 
