% I begin this version on 20 january 2007
\magnification=1200 %\baselineskip=14pt
\def\vare {\varepsilon}
\def\A {{\bf A}}
\def\t {\tilde}
\def\a {\alpha}
\def\K {{\bf K}}
\def\N {{\bf N}}
\def\V {{\cal V}}
\def\s {{\sigma}}
\def\S {{\Sigma}}
\def\s {{\sigma}}
\def\p{\partial}
\def\vare{{\varepsilon}}
\def\Q {{\bf Q}}
\def\D {{\cal D}}
\def\G {{\Gamma}}
\def\C {{\bf C}}
\def\M {{\cal M}}
\def\Z {{\bf Z}}
\def\U  {{\cal U}}
\def\H {{\cal H}}
\def\R  {{\bf R}}
\def\E  {{\bf E}}
\def\l {\lambda}
\def\degree {{\bf {\rm degree}\,\,}}
\def \finish {${\,\,\vrule height1mm depth2mm width 8pt}$}
\def \m {\medskip}
\def\p {\partial}
\def\r {{\bf r}}
\def\v {{\bf v}}
\def\n {{\bf n}}
\def\t {{\bf t}}
\def\b {{\bf b}}
\def\c {{\bf c }}
\def\e{{\bf e}}
\def\ac {{\bf a}}
\def \X   {{\bf X}}
\def \Y   {{\bf Y}}
\def \x   {{\bf x}}
\def \y   {{\bf y}}

\centerline  {\bf  Jacobi identity and intersection of altitudes}

\bigskip

  It is many years that I know the expression 
which belongs to V.Arnold and which sounds something like
  that: {\sl ''Altitudes (heights) of triangle intersect in 
one point because of Jacoby identity''}
  or may be even more aggressive: 
{\sl "The geometrical meaning of Jacoby identity 
is contained in the fact that
    altitudes of triangle are intersected in the one point".}
  Today preparing exercises for students I suddenly understood a 
meaning of this sentence. Here it is:

\m

  Let $ABC$ be a triangle. Denote by $\ac$ vector $BC$, 
by $\b$ vector $CA$ and by $\c$ vector $AB$:
  $\ac+\b+\c=0$.
   Consider vectors $\N_{\ac}=[\ac,[\b,\c]]$, 
$\N_{\b}=[\b,[\c,\ac]]$ and $\N_{\c}=[\c,[\ac,\b]]$.
(We denote by $[\,,\,]$ vector product).
   Vector $\N_{\ac}$ applied at the point $A$ of the 
triangle $ABC$ belongs to the plane of triangle, it
   is perpendicular to
   the side $BC$ of this triangle.  Hence the altitude (height) 
$h_A$ of the triangle
    which goes via the vertex $A$
   is egment on the line given by equation
 $\r_A(t)=A+t\N_{\ac}$. 
The same is for vectors $\N_\b,\N_\c$:
    Altitude (height) $h_B$ is on the  line which goes via 
the vertex $B$ along the vector $\N_b$ and
    altitude $h_C$ (height) is a line which goes via the 
vertex $C$ along the vector $\N_c$.

    Due to Jacobi identity sum of vectors 
$\N_\ac, \N_\b, \N_\c$ is equal to zero:
                      $$
         \N_{\ac}+\N_{\b}+\N_\c=[\ac,[\b,\c]]+[\b,[\c,\ac]]+[\ac,[\b,\c]]=0
         \eqno (1)
                      $$
 To see that altitudes $h_A\colon A+t\N_{\ac}$, 
$\,\,h_B\colon B+t\N_\b$ and $h_C\colon C+t\N_\c$
intersect  at  a  point it is enough to show that
 the sum of torques (angular momenta) of 
vector $\N_\ac$ attached at the point $A$,
 vector $\N_\b$ attached at the line $B$, and vector
$\N_\c$ attached at the line $C$  
vanishes with respect to some  $M$:
                 $$
          [MA,\N_\ac]+[MB,\N_\b]+[MC,\N_\c]=0\,.
                 \eqno (2)
                  $$
 Indeed  it is easy to see that equation (1) implies that 
relation (2) obeys for  an arbitrary   
point $M'$ if and only if  it obeys for a given  point $M$.
 
 We  prove now  equation (2) for an arbitrary  point $M$. 
   Denote $MA=\x$ then
  using equation (1)
we see that
for  left hand side of the equation (2) 
          $$ 
[MA,\N_\ac]+[MB,\N_\b]+[MC,\N_\c]= 
[\x,\N_\ac]+[\x+\c,\N_\b]+[\x+\c+\ac,\N_\c]=
          $$      
          $$      
  =[\c,\N_\b]+[\c+\ac,\N_\c]=
   [\c,[\b,[\c,\ac]]]+[\c+\ac,[\c,[\ac,\b]]]=
           $$
         $$
   [\ac+\b,[\b,[\ac+\b,\ac]]]+[\b,[\ac+\b,[\ac,\b]]]=\,
    \left(\hbox{here we used that $\ac+\b+\c=0$}\right)
        $$
         $$
   [\ac,[\b,[\b,\ac]]]+[\b,[\b,[\b,\ac]]]+
 [\b,[\ac,[\ac,\b]]]+[\b,[\b,[\ac,\b]]]=
   [\ac,[\b,[\b,\ac]]]+
 [\b,[\ac,[\ac,\b]]]=
        $$
        $$
   \underbrace
       {
  [\ac,[\b,[\b,\ac]]]+
 [\b,[[\b,\ac]]]+
   [[\b,\ac],[\ac,\b]]
          }_{\hbox{Jacobi identity}}-
   [[\b,\ac],[\ac,\b]]=
   [[\ac,\b],[\ac,\b]]=0\,.
         $$
In the last relation we again use Jacobi identity: 
We see that equation (2) holds, hence altitudes 
of triangle intersect in one point!  Zabavno, da?\finish

\medskip

$\qquad\qquad\qquad\qquad$  {\it Hovik Khudaverdian (24.01.07)}



  \bye
