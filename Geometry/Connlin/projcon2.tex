\magnification=1200 %\baselineskip=14pt
\def\vare {\varepsilon}
\def\A {{\bf A}}
\def\t {\tilde}
\def\a {\alpha}
\def\K {{\bf K}}
\def\N {{\bf N}}
\def\V {{\cal V}}
\def\s {{\sigma}}
\def\S {{\Sigma}}
\def\s {{\sigma}}
\def\p{\partial}
\def\vare{{\varepsilon}}
\def\Q {{\bf Q}}
\def\D {{\cal D}}
\def\G {{\Gamma}}
\def\C {{\bf C}}
\def\M {{\cal M}}
\def\Z {{\bf Z}}
\def\U  {{\cal U}}
\def\H {{\cal H}}
\def\R  {{\bf R}}
\def\S  {{\bf S}}
\def\E  {{\bf E}}
\def\l {\lambda}
\def\degree {{\bf {\rm degree}\,\,}}
\def \finish {${\,\,\vrule height1mm depth2mm width 8pt}$}
\def \m {\medskip}
\def\p {\partial}
\def\r {{\bf r}}
\def\v {{\bf v}}
\def\n {{\bf n}}
\def\t {{\bf t}}
\def\b {{\bf b}}
\def\c {{\bf c }}
\def\e{{\bf e}}
\def\ac {{\bf a}}
\def \X   {{\bf X}}
\def \Y   {{\bf Y}}
\def \x   {{\bf x}}
\def \y   {{\bf y}}
%\def \G{{\cal G}}
\def \F {{\cal F}}
\def\s {\sigma}
\def\o {\omega}
\def \ggb {\Gamma_{_{\bullet}}}
\def \gb {\Gamma_{_{\bullet}}}


\centerline{    {\bf Some facts in projective Geometry}}

\m

\centerline   {\sl Lecture 1. (August 2011)}


{\it We try to expose here some common facts of projective geometry about geodesics and
projective class of connections.
 First we define the projective class of connections which can be put in one-one correspondence
with non-parameterised geodesics.  Then on the base of it we describe the class of all projective transformations
and give a variant of a proof of the "old fashioned" statement about that all transformations preserving lines
are projective.


Everything sure is well-known, but on the other hand well-forgotten too.
}



$$ $$



           \centerline {\bf ${\cal x} 1$ Projective class of connections and non-parameterised geodesics}

\m



Let $M$ be a manifold with affine connection. We consider only symmetric connections.

   We say that two symmetric connections $\nabla$ and $\tilde \nabla$ belong to the same {\it projective}
   class if
   there exists $1$-form $\o$ such that
             $$
       \left(\tilde\nabla-\nabla\right)(\X,\Y)=\X \o(\Y)+\Y\o(\X)\,,
       \eqno (1.1a)
          $$
   i.e. for Christoffel symbols    $$
     \tilde \Gamma^i_{km}-\Gamma^i_{km}=\delta^i_k\o_m+\delta^i_m\o_k\,,
     \eqno (1.1b)
           $$

   {\bf Proposition} {\it Two symmetric connections belong to the same projective class if and only if
   their all geodesics coincide up to reparameterisation.}
\m

     Prove it. Let   $\x(t)$ be a geodesic of
     connection $\nabla$, i.e.
                   $$
                {d^2 x^i\over d t^2}+ \Gamma^i_{km}(\x(t)){d x^k\over d t} {d x^m\over d t}=0\,.
                \eqno (1.2a)
                   $$

     Let a connection $\tilde \nabla$ belong to the projective class of the connection $\nabla$.
     Show that there exists reparameterisation  $\tau=\tau(t)$ such that
                  $$
               {d^2 \tilde x^i\over d \tau^2}+
     \tilde\Gamma^i_{km}(\tilde \x(\tau)){d \tilde x^k\over d \tau} {d \tilde x^m\over d \tau}=0   \,.
       \eqno (1.2b)
                  $$
     for $\tilde \x(\tau)=\x(t)$ for $\tau=\tau(t)$.

     We rewrite  (1.2a) using that $\x_t=\x_\tau \tau_t$:
                     $$
                     {d^2 \tilde x^i\over d \tau^2}\tau_t^2+
                     {d \tilde x^i\over d \tau}\tau_{tt}+\Gamma^i_{km}(\tilde \x(\tau))
                     {d \tilde x^k\over d \tau} {d \tilde x^m\over d \tau}\tau_t^2=0\,.
                     \eqno (1.3)
                     $$
 Dividing this equation on  $\tau_t^2$ and substracting from the equation (1.2b) we come to
                 $$
          -{d \tilde x^i\over d \tau}{\tau_{tt}\over \tau_t^2}+\left(\tilde\Gamma^i_{km}-\Gamma^i_{km}\right)
          {d \tilde x^k\over d \tau} {d \tilde x^m\over d \tau}=
          -{d \tilde x^i\over d \tau}{\tau_{tt}\over \tau_t^2}+
 2 {d \tilde x^i\over d \tau}\o_m\left(\tilde \x(\tau)\right){d \tilde x^m\over d \tau}=
                 $$
                 $$
                 -{d \tilde x^i\over d \tau}
                 \left[{\tau_{tt}\over \tau_t^2}-2\o_m(\tilde \x (t)){d \tilde x^m(t)\over d \tau}\right]=
                 -{d x^i(t)\over d t}
                 \left[{d\log \tau_t\over dt}-2\o_m(\x (t)){d  x^m(t)\over d }\right]t_\tau^2=
                 0\,.
                 \eqno (1.4)
                 $$
      Thus we see that if  $\x(t)$ is a geodesic of the connection $\nabla$ then
      $\tilde \x(\tau)$ is a geodesic of the connection $\tilde\nabla$ if
        $\tau=\tau (t)$ obeys the differential equation:
               $$
               {d\over dt}\log \tau_t-2\o_m(\x (t)){d x^m(t)\over d t}\,.
               \eqno (1.4b)
               $$
   i.e. if
                    $$
         \tau(t)=\exp \left(\int^t F(t')dt'\right)\,\, {\rm where}\,\,\,
          F(t)=\int^t \left[2\o_m(\x (t')){d x^m(t')\over d t'}\right]dt'
          \eqno (1.5)
                    $$

Now prove the converse implication.

Suppose that an arbitrary geodesic $\x(t)$ of the connection $\nabla$
becomes the geodesic of the connection $\nabla$ after suitable reparameterisation
$\tau(t)\colon \tilde \x(\tau)=\x(t)$. Then comparing the equations (1.2a) and (1.2b) we come
instead the equation (1.4) to the equation
                  $$
-{d \tilde x^i\over d \tau}{\tau_{tt}\over \tau_t^2}+S^i_{km}
          {d \tilde x^k\over d \tau} {d \tilde x^m\over d \tau}=0, \,\,{\rm where}\,\,\,
             S^i_{km}=\tilde\Gamma^i_{km}-\Gamma^i_{km}
             \eqno (1.6)
                  $$
We have to prove that in this case there exists $1$-form $\o$ such that
  $S^i_{km}=\delta^i_k\o_m+\delta^i_m\o_k$.

  This holds for the arbitrary velocity vector $\v={d\x \over d\tau}$. Hence this follows from the following algebraic statement:


  {\it Let $\S=\S(\X,\Y)$ be symmetric bilinear form on the vector space $V$ with values in this space such that
   for arbitrary vector $\X$ the quadratic form   $\S(\X,\X)$ is collinear  to $\X$. Then
   there exists $1$-form (covector)  $\t$ such that
                        $$
               \S(\X,\Y)=\X \t(\Y)+\Y\t(\x), \qquad (t_i={1\over n+1}S^k_{ki})\,.
                        $$


    }

(See the proof in the Appendix.)

\m

We introduced by (1.1.a) a {\it projective class of connections}.

{\bf Definition} {\it projective connection} is the equivalence class $[\nabla]$ of connections. 
We define the Christoffel symbols $\Pi([\Gamma])$ of projective connection as
                   $$
                 \Pi^i_{km}=\Gamma^i_{km}+{1\over n+1}\left(\delta^i_k\gamma_m+\delta^i_m\gamma_k\right)
                 \eqno (1.7a)
                   $$
where
                 $$
              \gamma_k=-\Gamma_{ik}^i
              \eqno (1.7b)
                 $$
is a connection on volume form induced by the connection $\Gamma$.


One can see that  $\Pi^i_{km}$ in (1.7a) does not change if Christoffel symbols   $\Gamma^i_{km}$
change in projective class: 
               $$
   \Gamma^i_{km}\mapsto \Gamma^i_{km}+\delta^i_k t_m+\delta^i_m t_k,\,\hbox {then $\Pi^i_{km}$ does not change}            
               $$
  Thus Christoffel symbols of projective class of connection are well. 
               
We see that non-parameterised connection is defined by projective class of symmetric connections
and vice versa.

\bigskip


\centerline {${\cal x}2$.  \bf Projective transformations.}

The projective transformations are
           $$
      x^i\mapsto y^i={A^i_k x^k\over 1+a_kx^k}
      \eqno (2.1)
           $$
They transform straight lines to straight lines (some lines go to infinity).

Is it true that any transformation of $\R^n$ that transform straight lines to straight lines
is projective transformation?  Yes it is (up to infinity), but I never could find the proof of this fact in modern books.  Here I will show the elementary proof in the analytical case.

It is evident that if Christoffels symbols of connection vanish in coordinates
$x^i$ then straight lines $x^i=x^i_0+v^i t$ are geodesic of this connection. Moreover it follows from the
considerations of the previous paragraph that straight lines in coordinates $\{x\}$ are geodesics of connection iff
 Christoffel symbols of projective class of connection vanish in these coordinates, i.e.
           $$
      [\Gamma]=0 \Leftrightarrow \Pi^i_{km}=\Gamma^i_{km}+{1\over n+1}
      \left(\gamma_k\delta^i_m+\gamma_m\delta^i_k\right)=0\,.
      \eqno (2.2)
           $$


On the other hand it is easy to see that the following statement is valid:

\m

{\bf Proposition} {\it If Christoffel symbols vanish in coordinates $\{x\}$ then
for an arbitrary projective transformation (2.1) the  Christoffel symbols have the appearance
         $$
     ^{^{{(y)}}}\Gamma^a_{bc}={\p y^a(x)\over \p x^r}{\p^2 x^r(y)\over \p y^b \p y^c}=
     \delta^a_b p_c\Delta+\delta^a_c p_b\Delta
     \eqno (2.3)
         $$
where $p_a$ are constants and $\Delta={1\over 1-p_m x^m}$.}





\m

Indeed let Christoffel symbols vanish in coordinates $\{x\}$.  It is evident that Christoffel symbols vanish in any new affine coordinates.
On the other hand  one can see that every projective transformation (2.1) is a composition
of affine transformation $x^i\mapsto c^i+A^i_k x^k$ and the special projective transformation
              $$
         y^i={x^i\over 1+ p_kx^k}, i.e. x^i={y^i\over 1- p_ky^k}
         \eqno (2.3)
              $$
The direct calculations show that for this transformation
Christoffels are defined by the expression  (2.2).

(To say more nice: let  $x$ be Cartesian coordinates
and $\Gamma$ define canonical connection, then under arbitrary projective transformation
the difference of $\Gamma$ is a connection (2.2) defined by $1$-form $p_m dy^m\over 1-p_ky^k$)


{\bf Remark} It is very useful for calculations to note that
  ${\p \Delta\over \p y^k}=p_k\Delta^2$.

\m

One can easy to see that we proved little bit more: that
The Proposition states that if coordinates $x$ and $y$ are related with projective transformation then
 Christoffel symbols $\Pi^i_{km}$ of projective connection
vanish in coordinates $y$ if they vanish in coordinates $x$.


\bigskip

Now we are ready to prove that any transforamtion  transforming straight lines to straight lines
is a projective transformations. 

From previous considerations one can see that this statement is equivalent
to the following 

{\bf Theorem.} Let $\nabla$ be canonical connection in $\R^n$ (with Cartesian coordinates $x$),
(Christoffel symbols of the canonical connection vanish in Cartesian coordinates $x$).
Consider the projective connection $[\nabla]$.
If  the Christoffel symbols  of projective connections $[\nabla]$ vanish in local coordinates $y$,
$\Pi^i_{km}=0 $, i.e. in coordinates $y$
         $$
       ^{^{{(y)}}}\!\Gamma^a_{bc}={\p y^a(x)\over \p x^r}{\p^2 x^r(y)\over \p y^b \p y^c}=
     \delta^a_b \o_c+\delta^a_c \o_b
     \eqno (2.4)
         $$
then the transformation $y=y(x)$ is a projective transformation (2.1) and due to  Proposition above
in particular there exist
constant $p_k$ such that $\o_k=p_k\Delta={p_k\over 1-p_ry^r}$.

\m


 Indeed let  under transformation $y=y(x)$ straight lines remain straight lines.
  Then according to previous paragraph
 the projective class of connection $[\nabla]$ has  vanishing Chrsitoffel symbols in coordinates  $y$ 
 in the case if  connection $\nabla$ has vanishing Christoffel symbols in coordinates $x$.
 Hence it follwos from the Theorem
  that $y=y(x)$ it is a projective transformation.

\m

{\sl Proof of Theorem}.

  {\it We prove this Theorem under the following simplyfying assumption: the transformation $y=y(x)$
  is analytical transformation. More precisely we prove the statement for formal power series and it will be valid
  for analytical functions.}

   Without loss of generality we may assume that
    for $y=y(x)$
                      $$
                 y^a=x^a+\hbox {terms of the order more than equal 3, i.e.}\,\,\,
                 y^a=x^a+L^a_{bcd}x^bx^cx^d+\dots
                 \eqno (2.5)
                      $$
   Indeed suppose that
                 $$
                 y^a=c^a+x^a+A^a_bx^b+K^a_{bc}x^bx^cx^a+L^a_{bcd}x^bx^cx^d+\dots
                 $$
Then making  suitable affine transformation $y^a\mapsto B^i_k\left(y^a-c^a\right)$ with $B=A^{-1}$
 we can consider new coordinates $y'$ that in these coordinates
                   $$
                   y'^a=x^a+K^a_{bc}x^bx^cx^a+L^a_{bcd}x^bx^cx^d+\dots
                   \eqno (2.5b)
                   $$
  Now one can suppose that in (2.5b) $K^b_{ba}=0$. Indeed if $K^b_{ba}=s_a\not=0$ one can consider the
  special projective transformation (2.3) which eliminates the trace $K^b_{ba}$.
  Finally if the transformation has the appearance
                   $$
                    y^a=x^a+K^a_{bc}x^bx^cx^a+\dots \,\,{\rm with}\,\, K^b_{ba}=s_a=0
                   \eqno (2.5c)
                   $$
 then according to  (2.4) and (2.5c)
                   $$
                   ^{^{{(y)}}}\Gamma^a_{bc}\big\vert_{x=0}=
                   {\p y^a(x)\over \p x^r}{\p^2 x^r(y)\over \p y^b \p y^c}\big\vert_{x=0}
      =L^a_{bc}=
     \left(\delta^a_b \o_c+\delta^a_c \o_b\right)\big\vert_{x=0}=0
                   $$
  since $\o_b\big\vert_{x=0}={K^b_{ba}\over n+1}=0$.

  Hence we see that under suitable "tuning" we come to the transformation $y=y(x)$ such that
                   $$
   y^a=x^a+L^a_{bcd}x^bx^cx^d+\dots
               $$
               $$
   y^a_rx^r_{bc}=\delta^a_b t_c+\delta^a_c t_b\,\,{\rm with} t_c\big\vert_{x=0}=0
   \eqno (2.6)
                   $$
       Now one can easy prove by induction that all higher derivatives vanish too.

   {\it First step}. Prove that $L^a_{bcd}=0$. Differentiating the equation
   $y^a_rx^r_{bc}=\delta^a_b t_c+\delta^a_ct_b$ at the $x=0$
   and using the fact that $y^a=x^a+\dots$ and second derivatives vanish we come to
                           $$
       \left(y^a_rx^r_{bc}\right)_d\big\vert_{x=0}=L^a_{bcd}=\delta^a_b t_{c,d}+\delta^a_ct_{b,d}
                           $$
        We have symmetricity condition: $L^a_{bcd}=L^a_{bdc}$. Hence
             $$
\delta^a_b t_{c,d}+\delta^a_ct_{b,d}=\delta^a_b t_{d,c}+\delta^a_dt_{b,c}
\eqno (2.7a)
             $$
  Taking trace with respect to indices $a,b$ we come to $(n+1)t_{d,c}=(n+1)t_{c,d}$. Hence
$t_{d,c}=t_{c,d}$ we have in (2.7) that
                      $$
\delta^a_ct_{b,d}=\delta^a_dt_{b,c}
\eqno (2.7b)
                      $$
Taking trace with respect to indices $a,c$ we come to $nt_{b,d}=t_{b,d}$, i.e. $t_{b,d}=0$. Hence we see that
       $$
        \left(y^a_rx^r_{bc}\right)_d\big\vert_{x=0}=L^a_{bcd}=\delta^a_b t_{c,d}+\delta^a_ct_{b,d}=0
       $$
i.e. $L^a_{bcd}=0$. It is easy to see that under assumption that the first $k$ derivatives equal to zero
the $k+1$-th derivative equals to zero also. \finish


$$ $$












\def\K {{\bf K}}
\def \App {{\rm App\,}}

\centerline {\bf Appendix }

  Here we prove the following simple statement:

  {\it Let $\S=\S(\X,\Y)$ be symmetric bilinear form on the vector space $V$ with values in this space such that
                   $$
    \S(\X,\X)\,\, \hbox{ is  collinear to $\X$ for arbitrary vector $\X$ }\,.
   \eqno (\App 1a)
                    $$ Then
   there exists $1$-form (covector)  $\t$ such that
                        $$
               \S(\X,\Y)=\X \t(\Y)+\Y\t(\x),\qquad (t_i={1\over n+1}S^k_{ki})\,.
               \eqno (\App 1b)
                        $$


    }


Proof. We have that  $S^i_{km}=S^i_{mk}$ and
$S^i_{km}a^ka^m=a^i\lambda(\ac)$ for arbitrary vector $\ac=a^i$. This means that for arbitrary $i,j$
$a^jS^i_{km}a^ka^m=a^{i}S^j_{km}a^ka^m $.

Consider the set vectors $\ac$ with first component $a^1=0$.  Then for $i=2,3,\dots,n$ we come
                      $$
a^1S^i_{km}a^ka^m=S^d_{bc}a^ba^c+2S^d_{1c}a^c+S^d_{11}=a^{i}S^{1}_{km}a^ka^m=
        a^d\left(S^1_{bc}a^ba^c+2S^1_{1c}a^c+S^1_{11}\right)\
        \eqno (\App 2a)
                      $$
         where indices $b,c,d$ run over $\{2,3,\dots,n\}$.
Comparing these quadratic polynomials we see that
                  $$
          S^d_{bc}=\delta^d_b S^1_{1c}+\delta^d_cS^1_{1b},\quad
           S^d_{1c}=S^d_{c1}=\delta^d_c S^1_{11},\quad
                     S^d_{11}=0\,.
                     \eqno (\App 2b)
                  $$
Consider the object $(t_1,t_2,\dots,t_n)$ such that
          $
       t_1={1\over 2}S^1_{11},\,\,\, t_a=S^1_{1a}\, (a=2,\dots,n)
          $.
(We still do not know is it a vector!)
      It is easy to see from equations (App 4) that
             $$
         S^i_{km}=\delta^i_k t_m+\delta^i_mt_k
             $$
We come to the equation ($\App 2$): $t_m={1\over n+1}{\rm Tr}S= {1\over n+1}S^k_{km}$ is a covector.
\bye



Denote by ${\rm tr}_1 S$ the contraction of the tensor $S$ with respect to upper and low indices:
                 $
                {\rm tr}_1 S(\x)=\sum_i {\e}^i\left(s(\e_i,\x)\right)
                 $,
  where $\{\e_i\}$ is an arbitrary basis and $\{\e^i\}$ is a dual one. In components
  ${\rm tr}_1 S_i=S^m_{mi}$.

  Consider projector
                $$
  S\to PS=S-{1\over n+1}\left({\rm {\bf id}}\otimes {\rm tr}_1 S+{\rm tr}_1 S\otimes {\rm {\bf id}}\right)
  \eqno (App 3)
                              $$
  In components
                   $$
             \left(PS\right)^i_{km}=S^i_{km}-{1\over n+1}\left(\delta^i_k S^n_{nm}+\delta^i_m S^n_{nk}\right)
\eqno (App 2a)
                   $$
In what follows all formulae will be written in components.

It is easy to see that under projection $P$ trace of $S$ vanishes:
                  $$
                  (PS)^n_{nm}=0.
                  $$
    To prove the statement (App2) it suffices to show that $S=0$ if the condition (App 1) holds and  $S=PS$.              Indeed it follows from this fact that  the tensor $S-PS$ vanishes. Hence $S=PS$ and $\t={\rm tr}_1 S$.


\bye


Date: Fri, 19 Aug 2011 13:30:29 +0000
From: Samuel Danagoulian <danagous@ncat.edu>
To: H Khudaverdian <khudian@manchester.ac.uk>
Subject: Re: Barevner Safarian Stjopajits ev Khudaverdian
     Hovikts
Resent-From: <Hovhannes.Khudaverdyan@manchester.ac.uk>
Parts/Attachments:
   1   OK     ~83 lines  Text (charset: ISO-8859-1)
   2 Shown   ~125 lines  Text (charset: ISO-8859-1)
----------------------------------------

Dear Hovik,
I was looking over my mails and discovered that I
did not answer to your mail.
Though, when I received your mail (It was the day
before the last day of our stay
in Yerevan, two days later we took a flight to US,
home), it was very pleasant
surprise. We were reading your mail with Minassian
Martin (now he is our neighbor
in Yerevan) and Sveta, then we went to your
homepage, looked at you.
It was a pleasure to find Stepa as well. He was
one of our (me and Sveta) best
friends when we were students. How often do you
see Stepa? Please say
hello to him for us. What Stepa does
in  Gottingen?
We were in Yerevan with our two grand kids during
July 13 - August 10. We kept
our apartment. Yerevan was great, but expensive
this time. We saw Sedrakian
Araik with his wife, Gurzadian Vahagn and Pasha
with his wife. We went to
Ashtarak river gorge and spent time together with
barbecue.

Now we are back, our classes started. Svetlana's
classes start this coming
Monday.  We live in Greensboro NC. Ones a year we
go to Yerevan, Paris
or London (in London we stay just a day in between
two flights. Our daughter
lived there years ago, she graduated from LSE with
M.S.). We have good
friends in Paris since we worked in Orsay in 90s.
We have four grand kids.
How about you? How is your life? Give me more
information please.

Svetlana says hello to you.

Take care,

Samvel



Samuel Danagoulian
Professor of Physics,
North Carolina A&T State University
1601 E. Market St. Greensboro, NC 27411
Tel. (336) 285-2111, Fax: (336) 334-7423





On Mon, Aug 8, 2011 at 6:09 AM, H Khudaverdian
<khudian@manchester.ac.uk> wrote:

        Barev Samvel dzhan!

       Erek Stjopaji het hishetsink kez u
      Svetajin

       Es hima Manchesterumn em
      (see home page
      "www.maths.manchester.ac.uk/khudian")
      Kins hima indz het e. Tornikner es
      Moskvajumn en.

       Stjopa n aprum e Gottingenum ira
      @ntaniki het

       Menk urakh enk and erdhznik enk (at
      least when we meet)
        Il faut vivre!

        Barevnerov
             Hovik, Stjopa
