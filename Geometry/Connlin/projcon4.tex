\magnification=1200 %\baselineskip=14pt
\def\vare {\varepsilon}
\def\A {{\bf A}}
\def\t {\tilde}
\def\a {\alpha}
\def\K {{\bf K}}
\def\N {{\bf N}}
\def\V {{\cal V}}
\def\s {{\sigma}}
\def\S {{\Sigma}}
\def\s {{\sigma}}
\def\p{\partial}
\def\vare{{\varepsilon}}
\def\Q {{\bf Q}}
\def\D {{\cal D}}
\def\G {{\Gamma}}
\def\C {{\bf C}}
\def\M {{\cal M}}
\def\Z {{\bf Z}}
\def\U  {{\cal U}}
\def\H {{\cal H}}
\def\R  {{\bf R}}
\def\S  {{\bf S}}
\def\E  {{\bf E}}
\def\l {\lambda}
\def\degree {{\bf {\rm degree}\,\,}}
\def \finish {${\,\,\vrule height1mm depth2mm width 8pt}$}
\def \m {\medskip}
\def\p {\partial}
\def\r {{\bf r}}
\def\v {{\bf v}}
\def\n {{\bf n}}
\def\t {{\bf t}}
\def\b {{\bf b}}
\def\c {{\bf c }}
\def\e{{\bf e}}
\def\ac {{\bf a}}
\def \X   {{\bf X}}
\def \Y   {{\bf Y}}
\def \x   {{\bf x}}
\def \y   {{\bf y}}
%\def \G{{\cal G}}
\def \F {{\cal F}}
\def\s {\sigma}
\def\o {\omega}
\def \ggb {\Gamma_{_{\bullet}}}
\def \gb {\Gamma_{_{\bullet}}}


\centerline{    {\bf Some facts in projective Geometry}}

{\it We try to expose here some common facts of projective geometry about geodesics and
projective class of connections.}


\centerline {\sl { Lecture 2}}

\m
{\it

Differential equations for parameterised geodesics define symmetric connection. Unparameterised geodesics define
projective class of connections. We come using  to the notion of Thomas bundle when projective class of connections
on the manifold define the connection on speciall one-dimensional bundle over $M$.

Everything sure is well-known, but on the other hand well-forgotten too.
}





$$ $$

\centerline {\bf ${\cal x}1$.  \bf First steps to Thomas bundle.}


  A symmetric connection on manifold $M$ define differential equations of parameterised geodesics:
                 $$
                {d^2 x^i\over dt^2}+\Gamma^i_{km}{d x^k\over dt}{d x^m\over dt}=0\,. 
                \eqno (1.1)
                 $$
Let $[\nabla]$  be a projective connection, and $\Gamma^i_{km}$ be Christoffel symbols
of an arbitrary connection in this class.   We have
                 $$
                  \Gamma^i_{km}=\Pi^i_{km}-{1\over n+1}\left(\delta^i_k\gamma_m+\delta^i_m\gamma_k\right)\,,
                  $$
                  where
                  $$
         \Pi^i_{km}=\Gamma^i_{km}+{1\over n+1}\left(\delta^i_k\gamma_m+\delta^i_m\gamma_k\right),\,\,
         {\rm with} \,\,\gamma_k=-\Gamma^i_{ik}
                  \eqno (1.2)
                 $$ 
where $\Pi^i_{km}$ are Christoffel symbols for projective class of connections $[\nabla]$.


 We come to differential equations:
                  $$
         {d^2 x^i\over dt^2}+\Gamma^i_{km}{d x^k\over dt}{d x^m\over dt}=
         {d^2 x^i\over dt^2}+\Pi^i_{km}{d x^k\over dt}{d x^m\over dt}+2{d x^i\over dt}\gamma_k(x) {d x^k\over dt}\,.
                \eqno (1.3)         
                  $$
Changing $\Gamma$ in projective class changes the {\it parameterisation} of geodesics. As it was explained
in the lecture 1 this equation defines {\it non-parameterised} geodesics. 
The equation (1.3) assigns to non-parameterised geodesics the family of differential equations
depending on covector field $\o_k(x)$:
changing $\gamma_k\to \gamma_k+\o_k$ does not change non-parameterised geodesics.
 We try to write one equation instead this family of equation.


Consider first the simple case.  

 Let $\nabla$ be a connection such that its Christoffel symbols vanish in coordinates $x$.
 
 Let $y=y(x)$ be new coordinates such that Christoffel symbols vanish up to a term $\delta t+t \delta$,
 i.e. Christoffel symbols of projective connection  $[\nabla]$ vanish in coordinates $y$:
            $$
    ^{^{(y)}}\!\Gamma^a_{bc}={\p y^a\over \p x^r}{\p^2 x^r\over \p y^b \p y^c}=\delta^a_b t_c+\delta^a_c t_b        
            $$
            We know from the Theorem in the previous lecture that this means that
           $$
   ^{^{(y)}}\!\Gamma^a_{bc}={\p y^a\over \p x^r}{\p^2 x^r\over \p y^b \p y^c}=\delta^a_b 
   {p_c\over 1+p_k y^k}+\delta^a_c {p_b\over 1+p_k y^k}
   \eqno (1.4)
           $$
where constants $p_b$ are parameters of special projective transformation. 

 Sure geodesic is straight lines. Look again on differential equations for geodesic in coordinates $x$
 and $y$:
                         $$
               {d^2 x^i\over dt^2}=0,\,,\, {d^2 y^a\over dt^2}+
               ^{^{(y)}}\!\Gamma^a_{bc}{d y^b\over dt}{d y^c\over dt}=
               {d^2 y^a\over dt^2}+2{dy^a\over dt}p_c\Delta {dy^c\over dt}\,,
               \eqno (1.5)
                         $$
where we denote by $\Delta={1\over 1-p_ky^k}=1+p_k x^k$
\m

Now very important step: Add a  new coordinate $x^0$  to the set of coordinates $x^i$ and respectively $y^0$ 
to the set of coordinates $y^a$ and consider instead equation above the following equations:
                     $$
                   \cases
                     {
                     {d^2 x^i\over dt^2}+2{dx^i\over dt}{dx^0\over dt}=0\cr
                     {d^2 x^0\over dt^2}+ {dx^0\over dt}{dx^0\over dt}=0  \cr
                     } \qquad \qquad
                     \cases
                     {
                     {d^2 y^a\over dt^2}+2{dy^a\over dt}{dy^0\over dt}=0\cr
                     {d^2 y^0\over dt^2}+ {dy^0\over dt}{dy^0\over dt}=0  \cr
                     }\,.
                     \eqno (1.5a)
                     $$
We want these equations to be equivalent to previous ones.
  Thus we need to put
                  $$
                  {dy^0\over dt}=p_c \Delta {d y^c\over dt} ={p_c \over 1-p_ky^k}{d y^c\over dt}\,.
                  \eqno (1.6)
                  $$
  In this case we have that
                   $$
 {d^2 y^0\over dt^2}=p_c {d\Delta\over dt} {d y^c\over dt}+p_c \Delta {d^2 y^c\over dt^2}=
           \left( p_c\Delta {d y^c\over dt}\right)^2+p_c\Delta \left(-2{dy^c\over dt}{dy^0\over dt}\right)=-
           \left({dy^0\over dt}\right)^2\,.
           \eqno (1.7)                 
                   $$
  Note that  in (1.5 a) ${dx^0\over dt}=0$ and using (1.6) we come to 
                      $$
             0  ={dx^0\over dt}=  \left({dy^0\over dt}-p_c {d\Delta\over dt} {d y^c\over dt}\right)=
                    {d\over dt}\left(y^0-\log \Delta\right)     
                      $$
What is it geometrical meaning of $\Delta$? For special projective transformation
                            $$
           y^a={x^a\over \Delta}={x^a\over 1+p_kx^k}\,, x^k=y^k\Delta
                            $$                     
            one can easy to see that  $\Delta$ is an exponent of jacobian 
                       $$
      J=\det \left({\p y\over \p x}\right)=\det\left( {\delta ^i_k\over \Delta}-{p_k x^i\over \Delta^2} \right)=
                \Delta^{-n}\det \det\left( {\delta ^i_k}-{p_k x^i\over \Delta} \right)=\Delta^{-n-1} 
                       $$                
 Thus                      
               $$
  0  ={dx^0\over dt}=
                    {d\over dt}\left(y^0-\log \Delta\right)=
                    {d\over dt}\left(y^0+{1\over n+1}\log \det 
                    \left({\p y\over \p x}\right)\right)              
               $$
 We come to 
 
 \m
 
 {\bf Observation}  Let $x,y$ be two local coordinate systems related with the transformation
  projective transformation. Add additional coordinate $x^0$ to coordinates $x^i$ and respectively
     coordinate $y^0$ to coordinates $y^i$ such that 
                     $$
          y^0=x^0+{k\over n+1}\log \det \left({\p y \over \p x}\right)           
                     $$
 Then the system of differential equations                     
          $$
          \cases
                     {
                     {d^2 x^i\over dt^2}+2{dx^i\over dt}{dx^0\over dt}=0\cr
                     {d^2 x^0\over dt^2}+ {dx^0\over dt}{dx^0\over dt}=0  \cr
                     } 
          $$           
  remains the same in coordinates $y$ and solutions of these equations are straight lines.
  
  These differential equations describe non-parameterised straight lines in arbitrary projective coordinates.              
               
Let $M$ be a projective manifold, i.e. smooth manifold equipped with an atlas such that
all transition maps are projective transformations.

One can consider on the projective manifold the projective class of connections $[\nabla]=0$. It is well-defined due
the fact that projective transformations do not change the class. (See the previous section.)
  Let $\{x^i\}$ be an arbitrary local coordinates belonging to the projective atlas.
  Consider the bundle $\hat M$ with coordinate transformation for coordinate $x^0$
  defined by the equation
  Consider differential equations


          $$
         \Gamma^i_{km}\to  \Gamma^i_{km}+\delta^i_k\o_m+\delta^i_m\o_k
          $$














\centerline {\bf ${\cal x}2$.  \bf Thomas bundle.}


    Let $M$ be a manifold equipped with projective class of connections, i.e.
    a class $[\nabla]$ of connections is defined on it.
     (We say that two connections belong to the same class if the condition (1.1) is obeyed.)
    E.g. if $M$ is $\R^n$ then consider connection which Christoffel symbols vanish in cartesian coordinates
    The projective class of connections have Christoffel symbols
      $\G^i_{km}=\delta^i_k \o_m+\delta^i_m \o_k$ where $\o_k$ is an arbitrary covector.
      Geodesics of these connections are straight lines (in Cartesian coordinates.)



    Add a new coordinate $t$, i.e. consider expanded manifold $\hat M$ with coordinates
    $x^\mu=(t,x^i), t=x^0.$
     We come to so called  so called Thomas one-dimensional bundle $\hat M$.

     Under changing of coordinates
       $x^{i'}=x^{i'}(x^i)$ the coordinate $x^0$ changes in the following way:
                   $$
                x^{0'}=\log \left(\det\left({\p x^i\over \p x^{i'} }\right)\right)+x^0
                  $$
    {\bf Theorem.} Let $[\nabla]$ be a projective class of connections on $M$. One can assign to this class the connection
$\hat \nabla$ in $\hat M$ with the following Christoffel symbols:
  $\Pi^\mu_{\nu\rho}$  ($\mu,\nu,\rho=0,1,2,\dots,n$) are such that
                              $$
                       \Pi^i_{km}=\Gamma^i_{km}+{1\over n+1}\left(\delta^i_k\gamma_m+\delta^i_k\gamma_m \right)
                              $$
for $i=1,2,3,\dots,n$ where $\gamma_i=-\Gamma^k_{ik}$,
               $$
            \Gamma^i_{k0}=\Gamma^i_{0k}=-{\delta^i_k\over n+1}
               $$
and so on.

To understand this so called so on, let us consider in detail the special case:


{\bf Remark} Please pay attention that in the formulae above the right hand side is defined
up to





\bye


$$ $$



