\magnification=1200 

\baselineskip=14pt


\def\vare {\varepsilon}
\def\A {{\bf A}}
\def\t {\tilde}
\def\a {\alpha}
\def\K {{\bf K}}
\def\N {{\bf N}}
\def\V {{\cal V}}
\def\s {{\sigma}}
\def\S {{\Sigma}}
\def\s {{\sigma}}
\def\p{\partial}
\def\vare{{\varepsilon}}
\def\Q {{\bf Q}}
\def\O {{\bf O}}
\def\D {{\cal D}}
\def\G {{\Gamma}}
\def\C {{\bf C}}
\def\M {{\cal M}}
\def\Z {{\bf Z}}
\def\U  {{\cal U}}
\def\H {{\cal H}}
\def\R  {{\bf R}}
\def\S  {{\bf S}}
\def\E  {{\bf E}}
\def\l {\lambda}
\def\degree {{\bf {\rm degree}\,\,}}
\def \finish {${\,\,\vrule height1mm depth2mm width 8pt}$}
\def \m {\medskip}
\def\p {\partial}
\def\r {{\bf r}}
\def\v {{\bf v}}
\def\n {{\bf n}}
\def\t {{\bf t}}
\def\b {{\bf b}}
\def\c {{\bf c }}
\def\e{{\bf e}}
\def\ac {{\bf a}}
\def \X   {{\bf X}}
\def \Y   {{\bf Y}}
\def \x   {{\bf x}}
\def \y   {{\bf y}}
\def \G{{\cal G}}
\def\w{\omega}
\def\finish {${\,\,\vrule height1mm depth2mm width 8pt}$}


16 June, 2020

\centerline  {\bf Orthocentre of triangle and related problems}


  Three heights (altitudes) of triangle intersect at the point.  
     This is  well-known statement \footnote{$^*$}
     {Arnold makes it famous
     claiming  that this happens due to Jacobi identity.
     (see the etude in my homepage)
     However we will speak here about other topic.}
     
     
     I remember how I was surprised in school
     when I realised that this happens since
     orthocentre of $\triangle ABC$ coincides
     with centre of  circumstribed circle of 'double' triangle 
     $\triangle A'B'C'= 2\times \triangle ABC$!

  {\bf Remark}   
     We will use notation 
$\triangle A'B'C'= 2\times \triangle ABC$
for triangle such that 
                   $$
         \matrix
	 {
\hbox {side $A'B'$  passes via the point $C$ and $A'B'||AB$}\cr
\hbox {side $B'C'$  passes via the point $A$ and $B'C'||BC$}\cr
\hbox {side $A'C'$  passes via the point $B$ and $A'C'||AC$}\cr
    }
    \eqno (1)
                 $$

\medskip

We consider another remarkable point of $\triangle ABC$:
                    $$
\hbox {intersection of heights  (altitudes) of double triangle} 
     \triangle A'B'C'= 2\times \triangle ABC 
                 \eqno (2)
                 $$
 or in other words
     the centre of circumscribed circle  of the 'quatre' triangle
   $\triangle\tilde A\tilde B\tilde C=2\times \triangle A'B'C'$,
      where   $\triangle A'B'C'= 2\times \triangle ABC$.





     Many years ago I was solving the following problem: 
     Let $ABCD$ be tetraedron such
     that all its faces are four equal triangles (with sides $a,b,c$).
     Calculate its volume\footnote{$^{**}$}
     {This problem comes from 1984 when I was tutoring Vahagn Minasian...
     {\tt Ou es-tu maintenant, Vahagn?}}
      At that time I came to the following very beautiful 
      solution of this problem:

      Consider right parallelipiped $ABCDA'B'C'D'$  with sides $x,y,z$
         such that this tetrahedron is inscribed in this parallelepiped:
	           $$
		    \cases
		     {
		   AD=BC=A'D'=B'C'=x\cr
	    AB=CD=A'B'=C'D'=y\cr
           AA"=BB'=CC"=DD"=z\cr
	          }\,,\quad
		  {\rm where}\,\,
               \cases
		   {
		   x^2+y^2=a^2\cr
		   y^2+z^2=b^2\cr   
		   z^2+x^2=c^2\cr
		   }
                  \,\,i.e.
		  \cases
		       {
           x=\sqrt {a^2+c^2-b^2\over 2}\cr
           y=\sqrt {b^2+a^2-c^2\over 2}\cr
           z=\sqrt {b^2+c^2-a^2\over 2}\cr
			       }
			       \eqno (3)
		 		   $$
      We see that triangle which forms tetrahedron has to be acute
      (not obtuse),
      since $x,y,z$ have to be positive (or non zero)
     
     Now we see that  volume of our tetrahedron is equal to
             $$
	  {\rm Vol\, }(AB'CD')=
	  {\rm Vol \,}(ABCDA'B'C'D')-
	{\rm Vol\, }(BACB')-
	{\rm Vol \,}(C'B'D'C)-
	        $$
		$$
	{\rm Vol\,}(DACD')-
	{\rm Vol\,}(A'B'D'A)=
	xyz-{xyz\over 6}\cdot 4={xyz\over 3}=
	     $$
	     $$
	     {\sqrt 2\over 12}
	     \sqrt
	     {
		     (a^2+b^2-c^2)     
		     (b^2+c^2-a^2)     
		     (c^2+a^2-b^2)     
		     }
		     \eqno (4)
		     $$
	 Yes, this is beautiful.  However there is also another solution.
Thirty years ago trying to construct this tetrahedron, I came the equations
                   $$
		   \cases
		   {
	         {a'}^2+h^2=a^2\cr		   
	         {b'}^2+h^2=b^2\cr		   
	         {c'}^2+h^2=c^2\cr		   
			   }\,.
			   \eqno (5)
		   $$
	  Here $a,b,c$  are edges of the $\triangle ABC$,
	       $AB=c,BC=a$ and $AC=b$.
	       For an arbitrary point $P$ on the plane denote by
	       $x_a=PA$, $x_b=PB$ and $x_c=PC$.
	    If you find a point $P$ such that these equations
	    are fullfilled then, 
	             $$
		 {\rm Vol\,} (tetrahedron)=
		 {h\cdot {\rm Area\,  of\, the\,}  \triangle ABC\over 3}=
		    \eqno (6)
		     $$
I could not solve these equations thirty years ago.  A week ago 
I told this problem to my friend, Hovik Nersessian.
He suggested that a point $P$ is related with orthocentre...
This solves the puzzle!
I realised that the following statements are obeyed:


	       
	       {\bf Theorem}

{\it	Let $ABC$ be a triangle on the plane with edges,
	$a=BC$, $b=AC$ and $c=AB$.   For every point $P$ on the plane
   consider
              $$
	      \cases
	      {
	  x_a=PA=\hbox{the length of the 
	   segment from point $P$ to the point $A$}\cr
	   x_b=PB=\hbox{the length of the 
	   segment from point $P$ to the point $B$}\cr
	   x_c=PC=\hbox{the length of the 
	   segment from point $P$ to the point $C$
	    }
	    }\,.
	    \eqno (7)
	      $$
	  Then 
	  
1)	  there exist a unique point $P=M_1$
	  such that at this point
	            $$
		   \cases
		   {
	    x_a^2-x_b^2=b^2-a^2\cr		   
	    x_b^2-x_c^2=c^2-b^2\cr		   
	     x_c-x_a^2=a^2-c^2\cr		   
			   }
			\eqno (8)   
			$$   

and this point is orthocentre of the triangle $ABC$.			

2)	  there exist a unique point $P=M_2$
	  such that at this point
	            $$
		   \cases
		   {
	    x_a^2-x_b^2=a^2-b^2\cr		   
	    x_b^2-x_c^2=b^2-c^2\cr		   
	     x_a-x_c^2=a^2-c^2\cr		   
			   }
			 \eqno (9)  
			$$   
and this point is orthocentre of the double triangle triangle $ABC$.		

({\tt 
Note, please that RHS in equations (8) and (9) differ by sign!})

3) the transformation
   $$
\cases
{
x_a^2\mapsto 2x_b^2+2x_c^2-a^2\cr	
x_b^2\mapsto 2x_a^2+2x_c^2-b^2\cr	
x_c^2\mapsto 2x_a^2+2x_b^2-c^2\cr	
}
\eqno (10)
   $$
 transforms  solutions of the equation (8) to solutions of 
  equation (9); and vice versa 
  $$
\cases
{
	x_a^2\mapsto 
{
x_b^2+x_c^2-x_a^2+b^2+c^2-a^2\over 4}\cr
	x_b^2\mapsto 
{
x_a^2+x_c^2-x_b^2+a^2+c^2-b^2\over 4}\cr
	x_c^2\mapsto 
{
x_a^2+x_b^2-x_c^2+a^2+b^2-c^2\over 4}\cr
	}
	\eqno (11)
   $$
 transforms  solutions of the equation (9) to solutions of 
  equation (8)\footnote{$^\dagger$} {One can see
  that formal transformation 
  $x_a\mapsto ix_a$
  $x_b\mapsto ix_b$
  $x_c\mapsto ix_c$
 does this!}; 

4) If $DABC$ is tetrahedron such that all its faces are equal to
the triangle $ABC$, then it is acute triangle,
if  $M_2$ is the pointin $\triangle ABC$
which is the orthocentre of the double triangle,
then $DM_2$ is the height.
}
	    The proof
	    is almost evident.

Notice that points on the line  which passes via the
point $C$ and is orthogonal to the edge $AB$
is the locus of the points $M$ such that
         $$
	d(M,A)^2-d(M,B)^2=b^2-a^2\,.
	 $$
We see that points on the height obey this property.	 
The same is true for the line orthogonal to $BC$
which passes through the vertex $A$ and for the line
orthogonal to $AC$ which passes via vertex $B$.

Thus we see that three heights of triangle intersect 
at the point and equation (8) is obeyed.

Equation (9) looks almost the same, but...!!!
here we come to a different point.

  Points  on the line  which passes via the
point $C'$ (vertex of the double triangel)
and is orthogonal to the edge $AB$
is the locus of the points $M$ such that
         $$
	d(M,A')^2-d(M,B')^2=a^2-b^2\,,
	 $$
and respectively points on the line  which passes via the point $B$
and is orthogonal 
to the edge $AC$
is the locus of the points $M$ such that
         $$
	d(M,A)^2-d(M,C)^2=a^2-c^2\,,
	 $$
and respectively
points on the line  which passes throug the point
$A$ and is orthogonal to the edge $BC$
is the locus of the points $M$ such that
         $$
	d(M,B)^2-d(M,C)^2=b^2-c^2\,.
	 $$	 
	 These equations imply that $M_2$
	 is the intersection point of 
	 heights of double trianfgle.
    
    Now about symmetry between equations (8) and (9)
   
It is convenient to denote by $x_a,x_b,x_c$ solutions of equation
(8) and by $y_a,y_b,y_c$
solutions of equation (9).

First consider transformation (10)
Consider double triangle $A'B'C'$
and triangle $C'M_2B'$, where $M_2$ is orthocentre of the double triangle.
We see that $M_2A=y_a$. Point $M_2$ is orthocentre of the big triangle.
Hence by similarity
with edges $C'M_2=2x_c$ and $B'M_2=2x_b$, and  $y_a$ is median.
We come to
       $$
      y_a^2=2x_b^2+2x_c^2-a^2\,. 
       $$
       This proves (10). To prove transformation
       (11) consider in triangle $ABC$  half-triangle
       $A'B'C'$. We will come to (11).
    \bye
