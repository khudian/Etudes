\magnification=1200 

\baselineskip=14pt


\def\vare {\varepsilon}
\def\A {{\bf A}}
\def\t {\tilde}
\def\a {\alpha}
\def\K {{\bf K}}
\def\N {{\bf N}}
\def\V {{\cal V}}
\def\s {{\sigma}}
\def\S {{\Sigma}}
\def\s {{\sigma}}
\def\p{\partial}
\def\vare{{\varepsilon}}
\def\Q {{\bf Q}}
\def\O {{\bf O}}
\def\D {{\cal D}}
\def\G {{\Gamma}}
\def\C {{\bf C}}
\def\M {{\cal M}}
\def\Z {{\bf Z}}
\def\U  {{\cal U}}
\def\H {{\cal H}}
\def\R  {{\bf R}}
\def\S  {{\bf S}}
\def\E  {{\bf E}}
\def\l {\lambda}
\def\degree {{\bf {\rm degree}\,\,}}
\def \finish {${\,\,\vrule height1mm depth2mm width 8pt}$}
\def \m {\medskip}
\def\p {\partial}
\def\r {{\bf r}}
\def\v {{\bf v}}
\def\n {{\bf n}}
\def\t {{\bf t}}
\def\b {{\bf b}}
\def\c {{\bf c }}
\def\e{{\bf e}}
\def\ac {{\bf a}}
\def \X   {{\bf X}}
\def \Y   {{\bf Y}}
\def \x   {{\bf x}}
\def \y   {{\bf y}}
\def \G{{\cal G}}
\def\w{\omega}
\def\finish {${\,\,\vrule height1mm depth2mm width 8pt}$}

\centerline  {\bf Orthocentre of triangle and related problems}


  Three heights of triangle intersect at the point.  
     This is  well-known statement \footnote{$^*$}
     {Arnold makes it famous
     claiming  that this happens due to Jacobi identity.
     (see my etude:)
     However we will speak here about other topic.}
     
     
     I remember how I was surprised
     when I realised that this happens since
     orthocentre of $\triangle ABC$ coincides
     with orthocentre of 'double' triangle 
     $\triangle A'B'C'= 2\times \triangle ABC$!

  {\bf Remark}   
     We will use notation 
$\triangle A'B'C'= 2\times \triangle ABC$
for triangle such that 
                   $$
         \matrix
	 {
\hbox {side $A'B'$  passes via the point $C$ and $A'B'||AB$}\cr
\hbox {side $B'C'$  passes via the point $A$ and $B'C'||BC$}\cr
\hbox {side $A'C'$  passes via the point $B$ and $A'C'||AC$}\cr
    }
    \eqno (1)
                 $$

\medskip

We consider another remarkable point of $\triangle ABC$:
                    $$
\hbox {intersection of heights  of double triangle} 
     \triangle A'B'C'= 2\times \triangle ABC 
                 \eqno (2)
                 $$
 or in other words
     the orthocentre of the 'quatre' triangle
   $\triangle\tilde A\tilde B\tilde C=2\times \triangle A'B'C'$,
      where   $\triangle A'B'C'= 2\times \triangle ABC$.





     Many years ago I was solving the following problem
     : Let $ABCD$ be thetraedron such
     that all its fsces are four euqal triangles with sides $a,b,c$.
     Calculate its volume\footnote{$^{**}$}
     {THis problem comes from 1984 when I was tutoring Vahagn Minasian...}
      At that time I came to the following very beautiful 
      solution of this problem:

      Consider right parallelipiped $ABCDA'B'C'D'$  with sides $x,y,z$
         such that this tetrahedron is inscribed in this parallelepiped:
	           $$
		    \cases
		     {
		   AD=BC=A'D'=B'C'=x\cr
	    AB=CD=A'B'=C'D'=y\cr
           AA"=BB'=CC"=DD"=z\cr
	          }\,,\quad
		  {\rm where}\,\,
               \cases
		   {
		   x^2+y^2=a^2\cr
		   y^2+z^2=b^2\cr   
		   z^2+x^2=c^2\cr
		   }
                  \,\,i.e.
		  \cases
		       {
           x=\sqrt {a^2+c^2-b^2\over 2}\cr
           y=\sqrt {b^2+a^2-c^2\over 2}\cr
           z=\sqrt {b^2+c^2-a^2\over 2}\cr
			       }
		 		   $$
      We see that triangle which forms tetrahedron has to be acute
      (not obtuse),
      since $x,y,z$ have to be positive (or non zero)
     
     Now we see that  volume of our tetrahedron is equal to
             $$
	  {\rm Vol\, }(AB'CD')=
	  {\rm Vol \,}(ABCDA'B'C'D')-
	{\rm Vol\, }(BACB')-
	{\rm Vol \,}(C'B'D'C)-
	        $$
		$$
	{\rm Vol\,}(DACD')-
	{\rm Vol\,}(A'B'D'A)=
	xyz-{xyz\over 6}\cdot 4={xyz\over 3}=
	     $$
	     $$
	     {\sqrt 2\over 12}
	     \sqrt
	     {
		     (a^2+b^2-c^2)     
		     (b^2+c^2-a^2)     
		     (c^2+a^2-b^2)     
		     }
		     \eqno (*)
		     $$
	 Yes, this is beautiful.  However there is also another solution.
Thirty years ago trying to construct this tetrahedron, I came the equations
                   $$
		   \cases
		   {
	         {a'}^2+h^2=a^2\cr		   
	         {b'}^2+h^2=b^2\cr		   
	         {c'}^2+h^2=c^2\cr		   
			   }\,.
			   \eqno (**)
		   $$
	  Here $a,b,c$  are edges of the $\triangle ABC$,
	       $AB=c,BC=a$ and $AC=b$.
	       For an arbitrary point $P$ on the plane denote by
	       $a'=PA$, $b'=PB$ and $c'=PC$.
	    If you find a point $P$ such that these equations
	    are fullfilled then, 
	             $$
		 {\rm Vol\,} (tetrahedron)=
		 {h\cdot {\rm Area of the\,}  \triangle ABC\over 3}=
		     $$
I could not solve these equations.  A week ago 
i told this problem to my friend, Hovik Nersessian.
He suggested that a point $P$ is related with orthocentre...
Due to hom I realised that the following statements are obeyed:


	       
	       {\bf Theorem}
	       There is unique point $P$ such that
	           $$
		   \cases
		   {
	    {a'}^2-{b'}^2=b^2-a^2\cr		   
	    {b'}^2-{c'}^2=c^2-b^2\cr		   
	    {c'}^2-{a'}^2=a^2-c^2\cr		   
			   }
		   $$
and this point is {\it the orthocentre of triangle $ABC$.}

\bigskip


There is unique point $P$ such that
	           $$
		   \cases
		   {
	    {a'}^2-{b'}^2=a^2-b^2\cr		   
	    {b'}^2-{c'}^2=b^2-c^2\cr		   
	    {c'}^2-{a'}^2=c^2-a^2\cr		   
			   }
		   $$
and this point is {\it the orthocentre of 'double'
triangle $ABC$.}



	    
	    
     
     \bye
