\magnification=1200 \baselineskip=14pt
\def\vare {\varepsilon}
\def\A {{\bf A}}
\def\t {\tilde}
\def\a {\alpha}
\def\K {{\bf K}}
\def\N {{\bf N}}
\def\V {{\cal V}}
\def\s {{\sigma}}
\def\S {{\Sigma}}
\def\s {{\sigma}}
\def\p{\partial}
\def\vare{{\varepsilon}}
\def\Q {{\bf Q}}
\def\D {{\cal D}}
\def\G {{\Gamma}}
\def\C {{\bf C}}
\def\M {{\cal M}}
\def\Z {{\bf Z}}
\def\U  {{\cal U}}
\def\H {{\cal H}}
\def\R  {{\bf R}}
\def\E  {{\bf E}}
\def\l {\lambda}
\def\degree {{\bf {\rm degree}\,\,}}
\def \finish {${\,\,\vrule height1mm depth2mm width 8pt}$}
\def \m {\medskip}
\def\p {\partial}
\def\r {{\bf r}}
\def\v {{\bf v}}
\def\n {{\bf n}}
\def\t {{\bf t}}
\def\b {{\bf b}}
\def\e{{\bf e}}
\def\ac {{\bf a}}
\def \X   {{\bf X}}
\def \Y   {{\bf Y}}
\def \x   {{\bf x}}
\def \y   {{\bf y}}
\def\f {{\bf f}}
\def\pt {{\bf pt}}
\centerline  {\bf Shape operator $=$ square root of curvature two-form}

\centerline {\bf and}

\centerline {\it Teorema Egregium}

\bigskip



{\it We give here very simple proof of the fact why for an even-dimensional surface
curvature defined by shape operator is invariant of isometries and
show why this does  not work for odd-dimensional case\footnote{$^*$}
{\noindent {\sl Thedya Voronov
suggested to make these calculations using moving frames formalism.  This
works very nice.}}.}



\bigskip

  Let $M$ be a surface of codimension $1$ in $n+1$-dimensional Riemannian space.
   One can consider unit normal vector field $\n$. Since $(\n,\n)=1$ then
  the derivative of vector field $\n$ along an arbitrary vector $\x$ tangent to the
  surface is tangent vector also. We come to
  Shape operator $S$ defined on the vectors tangent to the surface:
             $$
           S(\x)=-\p_\x \n
           \eqno (0.1)
             $$
One can see that $\det S$ is equal to the density of volume form induced on the surface
 by the volume form on $n$-dimensional sphere in $\E^n$ through Gauss map
 ($M\ni \pt\to \n\vert_{\pt}$)., i.e.
      an integral
           $$
         \int \det S d\sigma
         \eqno (0.2)
           $$
is equal to the degree of Gaussian map.
($d\sigma$ is volume form on the surface induced by the Riemannian metric:$d\sigma=\sqrt {\det g}$)

 The integrand in this formula is total divergence. ({\tt Explain why?})

  \smallskip

  In the case if surface $M$ is {\it even}-dimensional then one can say more:
  $\det S$ is proportional to Gaussian curvature, it is invariant of isometries, i.e. it depends
  on the induced metric and does not change if we change embedding and does not change induced metric.
  It is Gauss Egregium Teorema. In this case $\det S d\sigma$ is local expression for Euler class.




  We would like to show this by simple calculations. Our plan is following:
  we show that $\det S$ is equal to the pfaffian of the determinant
  of curvature-two form.


\m
{\bf Remark} Note that   Note that shape operator (0.1) is defined up to a sign as well as normal unit vector.
Hence the density $\det S$ in (0.2) is defined up to a sign if dimension $M$ is odd and it is well-defined
if dimension of $M$ is even: $\det (-S)=(-1)^n\det S$.


\bigskip

\centerline {\it General stuff}

\m
 In his paragraph we consider relations between curvatures of tangent and normal bundle for an arbitrary
 $k$-dimensional surface in $n$-dimensional Euclidean space $\E^n$.

  Let $x^i(\xi^\a)$ ($i=1,\dots,n,\a=1,\dots,k$) be a
  parameterisation of the given surface $M$. The trivial
  (canonical) connection in $\E^n$ induces connection on surface and
  on normal bundle to the surface.
   Consider an orthonormal (in general non-holonomic) basis $\{\e_a,\n_p\}$, $a=1,\dots,k,p=1,\dots,n-k$
   on the points of the surface $M$ such that vectors $\e_1,\dots,\e_k$ are tangent to the surface
   and vectors $\n_1,\dots,\n_{n-k}$ are normal vectors:
          $$
        \e_a(\xi)=e_a^\a x^i_\a {\p\over \p x^i}\big\vert_{x^i_\a={\p x^i(\xi)\over \p \xi^\a}}
        ,\,\, (a=1,\dots,k),\,\,
        (\e_a,\e_b)=\delta_{ab},(\e_a,\n_p)=0,(\n_p,\n_q)=\delta_{pq}
          $$
  One  can consider  basis vectors $\{\e_a,\n_p\}$ as functions, $0$-forms on the points of the manifold $M$
  with values in the vectors of $\E^n$,
  $\{\e_a(\xi)=\e_a(x(\xi)),\n_p(\xi)=\n_p\left(x\left(\xi\right)\right)\}$.

 Consider $1$-forms $d\e_a$, $d\n_p$ on $M$ with values in vectors in $\E^n$. Expanding them over the basis
$\{\e_a,\n_p\}$ we come to the equations:
              $$
              d
              \pmatrix{\e_a\cr\n_p\cr}=
              \pmatrix {A_{ab} &L_{aq}\cr S_{pb} &T_{pq}}
              \pmatrix{\e_b\cr\n_q\cr}
              $$
            $$
            d\e_a=A_{ab}d\e_b+L_{aq}\n_q,\,\,\qquad d\n_p=S_{pb}\e_b+T_{pq}\n_q,
            \eqno (1)
            $$
where $(A_{ab},L_{aq},S_{pb},R_{pq})$ are one-forms.
The matrix  $\pmatrix {A_{ab} &L_{aq}\cr S_{pb} &T_{pq}}$ is antisymmetric since $(\e_a,\n_p)$ is an arthonormal
 basis at any point of $M$:
               $$
            A_{ab}=-A_{ba},\,\,   L_{ap}=-S_{pa},\,\,\, T_{pq}=-T_{qp}
            \eqno (2)
               $$

 The condition that $d^2\e_a=0, d^2 \n_b=0$ leads to identities:
                $$
        0=d^2 \pmatrix{\e_a\cr\n_p\cr}=
        \pmatrix {dA_{ab} &dL_{aq}\cr dS_{pb} &dT_{pq}}\pmatrix{\e_b\cr\n_q\cr}-
        \pmatrix {A_{ab} &L_{aq}\cr S_{pb} &T_{pq}}\pmatrix{d\e_b\cr d\n_q\cr}=
                $$
                $$
               \left(
        \pmatrix {dA_{ab} &dL_{aq}\cr dS_{pb} &dT_{pq}}-
        \pmatrix {A_{ac} &L_{ar}\cr S_{pc} &T_{pr}}
        \pmatrix {A_{cb} &L_{cq}\cr S_{rb} &T_{rq}}
        \right)
        \pmatrix{\e_b\cr\n_q\cr}
                $$
Hence we have four conditions (structure equations)
             $$
             \matrix
             {
         dA_{ab}-A_{ac}\wedge A_{cb}-L_{ar}\wedge S_{rb}=0\cr
         d L_{aq}-A_{ac}\wedge L_{cq}-L_{ar}\wedge T_{rq}=0\cr
          dS_{pb}-S_{pc}\wedge A_{cb}-T_{pr}\wedge S_{rb}=0\cr
          dT_{pq}-S_{pc}\wedge L_{cq}-T_{pr}\wedge T_{rq}\cr=0
               }
             $$
In a view of (2) we come to the following three conditions on one-forms $A_{ab}, S_{pb}, T_{pq}$ :
                    $$
                     \matrix
             {
         dA_{ab}-A_{ac}\wedge A_{cb}+S_{ra}\wedge S_{rb}=0\qquad\qquad &(*)\cr
          dS_{pb}-S_{pc}\wedge A_{cb}-T_{pr}\wedge S_{rb}=0\qquad\qquad &(**)\cr
          dT_{pq}+S_{pc}\wedge S_{qc}-T_{pr}\wedge T_{rq}=0\qquad\qquad &(***)\cr
               }
               \eqno (3)
                    $$
Clarify geometric meaning of these conditions.

1. One-forms $\{A_{ab}\}$ in (1) define parallel transport of tangent vectors and
connection in the tangent bundle such that $\e_b$-component of covariant derivative of
vector field $\e_a(x(\xi))$
along tangent vector $\X$ is equal to  $A_{ab}(\X)$.
   For any section $u^a\e_a$  the one-form  with values in tangent vectors to $M$
                $$
                D(u^a\e_a)=du^a\e_a+u^aD\e_a=du^a\e_a+u^aA_{ab}\e_b,
                  $$
  where $D$ is covariant differential.  Respectively
                  $$
            \nabla_\X \left({u^a\e_a}\right)=D(u^a\e_a)\left(\X\right)=
            du^a\left(\X\right)\e_a+u^aD\e_a\left(\X\right)=
            \p_\X u^a \e_a+u^aA_{ab}(\X)\e_b\,.
                $$


 One can show that this is Levi-Chivita connection of the metrics $g_{\a\b}$ ($\e g\e^+=1$) ({\tt Show it!}).
       The action of $D^2$ defines curvature operator:
                $$
         D^2(u^a\e_a)=D\left(du^a\e_a+u^aA_{ab}\e_b\right)=-du^aD\e_a+d(u^aA_{ab})\e_b-u^aA_{ab}D\e_b=
            u^a{\cal R}^{\rm tang}_{ab}\e_b
               $$
where for curvature-form  ${\cal R}^{\rm tang}_{ab}$
             $$
{\cal R}^{\rm tang}_{ab}=dA_{ab}-A_{ac}\wedge A_{cb}
             $$
Hence the first condition in (3) means
           $$
   {\cal R}^{\rm tang}_{ab}=-S_{ra}\wedge S_{rb}
   \eqno (4)
           $$
This is our central statement.

 Analogously we can consider connection and curvature of normal bundle.
 In particular we see that for curvature ${\cal R}^{\rm norm}_{pq}$ of normal bundle
            $$
            {\cal R}^{\rm norm}_{ab}=dT_{pq}-T_{pr}\wedge T_{rq}
            $$
and third condition means
          $$
  {\cal R}^{\rm norm}_{ab}=--S_{pc}\wedge S_{qc}
  \eqno (5)
            $$
            We see that both curvatures of normal and tangent bundle are defined via the one forms
  $S_{pr}$.



  \bigskip
            \centerline  {\it Surfaces of codimension 1}
 Now return to one-codimensional case. All becomes much simpler: Derivation equation (1) have now the
 following appearance:
         $$
          d\e_a=A_{ab}d\e_b-S_a\n,\,\,\qquad d\n_p=S_{b}\e_b,
         $$
        The one-form $S_a$ is nothing but shape operator (0.1)
  Indeed if $S_\a^{\beta}$ are components of shape operator in coordinate basis
        and one-form $S_a=d\xi^\a S_{\a a}$ then
               $$
               {\p \n\over \p \xi^\a}=-S_\a^\beta {\p\r(\xi)\over \p\xi^\beta}=S_{\a b}\e_b=
               S_{\a b}e_b^\beta {\p\r(\xi)\over \p\xi^\beta},
               $$
i.e.
                   $$
         =-S_\a^\beta=S_{\a b}e^\beta_b,\qquad  (\e_b=e_b^\beta x^i_\beta)
                   $$
condition (4) for hypersurface becomes:
            $$
               {\cal R}^{\rm tang}_{ab}=-S_{a}\wedge S_{b},
               \eqno (2.1)
            $$
`This formula contains all the answer!


\bye
 To understand it first consider two examples

 {\bf Example 1} Two-dimensional surface $x^i=x^i(\xi^1,\xi^2)$ in three dimensional Euclidean space.

 Curvature two form ${\cal R}_{ab}$ has one essential component
 ${\cal R}_{12}=R_{12}(\xi^1,\xi^2)d\xi^1\wedge d\xi^2$
 (${\cal R}_{21}=-{\cal R}_{12}, {\cal R}_{11}={\cal R}_{22}=0$).
  For one-forms $S_a$ ($a=1,2$) and shape operator $S$ we have
   $S({\bf u})=S_a({\bf u})\e_a=S_1({\bf u})\e_1+S_2({\bf u})\e_2$

 According to
 (2.1)
                  $$
{\cal R}_{12}=R_{12}(\xi^1,\xi^2)d\xi^1\wedge d\xi^2=-S_1\wedge S_2=()
                  $$
\bye
