\magnification=1200 %\baselineskip=14pt
\def\vare {\varepsilon}
\def\A {{\bf A}}
\def\t {\tilde}
\def\a {\alpha}
\def\K {{\bf K}}
\def\N {{\bf N}}
\def\V {{\cal V}}
\def\s {{\sigma}}
\def\S {{\Sigma}}
\def\s {{\sigma}}
\def\p{\partial}
\def\vare{{\varepsilon}}
\def\Q {{\bf Q}}
\def\D {{\cal D}}
\def\G {{\Gamma}}
\def\C {{\bf C}}
\def\M {{\cal M}}
\def\Z {{\bf Z}}
\def\U  {{\cal U}}
\def\E  {{\cal E}}
\def\H {{\cal H}}
\def\R  {{\bf R}}
\def\S  {{\bf S}}
\def\l {\lambda}
\def\degree {{\bf {\rm degree}\,\,}}
\def \finish {${\,\,\vrule height1mm depth2mm width 8pt}$}
\def \m {\medskip}
\def\p {\partial}
\def\r {{\bf r}}
\def\v {{\bf v}}
\def\n {{\bf n}}
\def\t {{\bf t}}
\def\b {{\bf b}}
\def\c {{\bf c }}
\def\e{{\bf e}}
\def\ac {{\bf a}}
\def \X   {{\bf X}}
\def \Y   {{\bf Y}}
\def \x   {{\bf x}}
\def \y   {{\bf y}}
%\def \G{{\cal G}}
\def \F {{\cal F}}
\def\s {\sigma}
\def\o {\omega}
\def \ggb {\Gamma_{_{\bullet}}}
\def \gb {\Gamma_{_{\bullet}}}
\def\pt {{\bf p}}

\centerline{    {\bf One fundamental fact}}


\bigskip


\m

{\bf Theorem}  {\it Let $M$ be an $n$-dimensional manifold equipped with parallelism structure, i.e. there are globally
defined $n$ vector fields $\e_1(x),\dots,\e_n(x)$ such that they are linearly 
independent at any point  of $M$. (These vector
fields define so called $\{1\}-structure$.).
Consider the group $\E$ of all diffeomorphisms which preserve this  structure: diffeomorphism   $F\in \S$
if it preserves all $n$ vector fields defining structure: $F^*\e_k=\e_k$. 

Then  $\E$ is a Lie group.  This Lie group has dimension less or equal to $n$. One can say more:

Consider for every point $x\in M$ the  map
                           $$
                   L_x\colon\qquad \E\ni F\mapsto F(x)
                   \eqno (1)        
                           $$ 
Then $L_x$ is injective image of this Lie group in the manifold $M$. The image is closed  manifold.
The structure of manifold on image is the manifolds structure on Lie group. 
}

This is standard statement (see Kobayashi. I like to study it since we see her the mnifestation of theisis
that sometimes better proof is more clear than "intuitive")
 
 
 
 
 
  Denote by $M_F$ the set of fixed points of the diffeomorphism $F\in \E$. 
    
    If $F={\bf id}$ then $M_F=M$. 
    
    {\bf Lemma}  If $M$ is connected manifold and diffeomorphism $F\in \E$ is not identity  then $M_F$ is empty set.
     
{\sl Proof}   The condition that diffeomorphism $F$ preserves basic fields $\{\e_1,\dots,\e_n\}$ means
that  for an arbitrary point $x\in M$
                    $$
                  dF\circ \e_k\big\vert_x=\e_k\big\vert_{F(x)},\,\, k=1,2,\dots,n 
                  \eqno (2a)   
                    $$  
This holds also for an arbitrary linear combination of basic vectors with constant coefficients: 
                 $$
        dF\circ \R\big\vert_x=\R\big\vert_{F(x)},\,\, k=1,2,\dots,n\,,
                  \eqno (2b)
                           $$
if $\R=c_1\e_1+\dots+c_n\e_n$
(with $c_1,c_2,\dots,c_n$ constants).

One can integrate these equations considering the fluxes of vector fields:
                              $$
         F\left(\,{\rm exp}\, t\R|\x\rangle\right) =\,{\rm exp}\, t\R|F(\x)\rangle
         \eqno (3)                    
                              $$
(for enough small times.)


Consider the set $M_F$ of fixedd points. On on hand this is a closed set, since every point
in $M_F$ is defined by equation $F(x)=x$. On the other hand it follows from the equation (3)
that this is open set. Indeed suppose that  $\pt\in M_\F$. 
Then one can see that for enough small $t$
  $\pt'={\rm exp}\, t\R|\pt\rangle$ too:
                 $$
   F(\pt')=F\left({\rm exp}\, t\R|\pt\rangle\right)={\rm exp}\, t\R|F(\pt)\rangle=
   {\rm exp}\, t\R|\pt\rangle=\pt\,.              
                 $$
We see that the set $M_F$ is open set and closed set. Hence since $M$ is connected and $F\not={\bf id}$
then $M_F$ is empty set\footnote{$^1$}{It is here where we see that exact considerations are better than empiric.
Taking many derivatives  of equation
(2a) one will come to the same result but in terms of formal power series.
(Sure the "trouble under the cover"
is in the definition of $exp t\R$).
}.\finish.

The second step is to see that...the image is closed manifold. 



\bye

{\sl Proof}