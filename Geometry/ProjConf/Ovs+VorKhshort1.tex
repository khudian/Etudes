\magnification=1200 %\baselineskip=14pt
\def\vare {\varepsilon}
\def\A {{\bf A}}
\def\t {\tilde}
\def\a {\alpha}
\def\K {{\bf K}}
\def\N {{\bf N}}
\def\V {{\cal V}}
\def\s {{\sigma}}
\def\S {{\Sigma}}
\def\s {{\sigma}}
\def\p{\partial}
\def\vare{{\varepsilon}}
\def\Q {{\bf Q}}
\def\D {{\cal D}}
\def\G {{\Gamma}}
\def\C {{\bf C}}
\def\M {{\cal M}}
\def\Z {{\bf Z}}
\def\U  {{\cal U}}
\def\H {{\cal H}}
\def\R  {{\bf R}}
\def\S  {{\bf S}}
\def\E  {{\bf E}}
\def\l {\lambda}
\def\degree {{\bf {\rm degree}\,\,}}
\def \finish {${\,\,\vrule height1mm depth2mm width 8pt}$}
\def \m {\medskip}
\def\p {\partial}
\def\r {{\bf r}}
\def\v {{\bf v}}
\def\n {{\bf n}}
\def\t {{\bf t}}
\def\b {{\bf b}}
\def\c {{\bf c }}
\def\e{{\bf e}}
\def\ac {{\bf a}}
\def \X   {{\bf X}}
\def \Y   {{\bf Y}}
\def \x   {{\bf x}}
\def \y   {{\bf y}}
\def \G{{\cal G}}

           \centerline {\bf Geometry of differential operators on $\R$}

Let
        $$
    A=t^\delta
           \underbrace{
           \left(
    s{\p^{n+1}\over \p x^{n+1}}
        +a\hat w {\p^{n}\over \p x^{n}}+b\hat w^2 {\p^{n-1}\over \p x^{n-1}}
          \right)}_
          {\hbox{terms of the order $n+1$}}+
          $$
          $$
          +
          \underbrace{\left(
    p{\p^{n}\over \p x^{n}}
        +c\hat w {\p^{n-1}\over \p x^{n-1}}
          +\dots\right)}_{{\hbox{terms of the order $n$}}}+
         \underbrace{
                \left(
    q{\p ^{n}\over \p x^{n-1}}
        +\dots\right)
        }_{{\hbox{terms of the order $n-1$}}}+\dots
        $$
be differential operator of the order $n+1$ and of the weight $\delta$ on the algebra $\cal F$
of densities on $\R$.  Here $\hat w =t{\p\over \p t}$, $a=a(x),\dots$

E.g. if $\Psi(x,t)=\varphi(x)t^5$ is a density $\varphi(x)|dx|^5$ of the weight $5$,
and

\noindent $A=t^{-3}\left({\p^3\over \p x^3}+
    p(x){\p^2\over \p x^2}
        +b(x)\hat w
        {\p\over \p x}\right)$
then
$$
 A\Psi=t^{-3}\left(
 {\p^3\Psi\over \p x^3}+ p(t){\p^2\Psi\over \p x^2}+
 b(x)t{\p\over \p t}
 {\p\over \p x}
 \right)=
 \left({d^3\varphi\over dx^3}+p(t){d^2\varphi\over dx^2}+
 4b(x){d\varphi\over dx}\right)|dx|^2\,.
 $$
The canonical scalar product in the space $\cal F$ is defined by condition that
for densities $\Psi=\Psi(x)t^\lambda$ and $\Phi=\Phi(x)t^\mu$
            $$
          \langle \Psi, \Phi
          \rangle=\cases
          {0\,\,\hbox {if $\lambda+\mu\not=1$}\cr
 \int\Psi(x)\Phi(x)dx \,\,\,\,\hbox {if $\lambda+\mu\not=1$}\cr
          }\,.
            $$

 Thus we have the conjugation of derivatives:
           $$
     x^\dagger=x,  \left({\p\over \p x}\right)^\dagger=-{\p\over \p x}, t^\dagger=t,
    \left({\p\over \p t}\right)^\dagger={2\over t}-{\p\over \p t},
           $$
In particular:
            $$
   \hat w^\dagger=\left(t{\over d\p }\right)^\dagger=1-\left(t{\p\over \p t}\right)^\dagger=1-\hat w^+
            $$
            and
            $$
        \hat w^\dagger
         (t^\sigma \Psi)=
         t^\sigma
         (1-\sigma-\hat w)
         \Psi\,.
            $$

 {$\cal x$1. \sl Subprincipal symbol}

\medskip

Now we find the restrictions on operator $A$ posed by the condition that it is self conjugate operator (up to  sign),
i.e. $$
A^\dagger=(-1)^{n+1} A
   $$,
Then we discuss the geometrical nature of coefficients.

\noindent We check the condition of self-conjugality  step by step for lower and lower derivatives
\footnote{$^1$}{All that we will do here could be compared with our calculations with T.Voronov for second order operators. The motto is that we descend from higher derivatives to lower: for $n$-th order operators
 the terms of $n$-th order behave like the term of second order of Laplacian,
 the terms of $n-1$-th order behave like the term of first order of Laplacian,
 the terms of $n-2$-th order behave like the term of zeroth order of Laplacian.}.

In this paragraph we consider only the terms which are proportional to derivatives of the order $n+1$ and $n$
 with respect to $x$:
           $$
       A=t^\delta\
          \left(
    s{\p^{n+1}\over p x^{n+1}}
        +a\hat w {\p^{n}\over \p x^{n}}+
    p{\p^{n}\over \p x^{n}}
               +
            \dots
          \right)\,,
           $$
             $$
           A^\dagger=
             \left[
            t^\delta\
          \left(
    s{\p^{n+1}\over \p x^{n+1}}
        +a\hat w {\p^{n}\over \p x^{n}}+
    p{\p^{n}\over \p x^{n}}
               +
            \dots
          \right)
          \right]^\dagger=
             $$
        $$
        (-1)^{n+1}
    t^\delta {\p^{n+1}\over \p x^{n+1}}
    \left(s(x) \,\cdot\,\right)+
    \hat w^\dagger\left[t^\delta (-1)^n{\p^{n}\over \p x^{n}} \left(a(x) \,\cdot\,\right)\right]+
    t^\delta (-1)^n{\p^{n}\over \p x^{n}} \left(p(x) \,\cdot\,\right)+\dots=
        $$
        $$
=   (-1)^{n+1}
            t^\delta s{\p^{n+1}\over \p x^{n+1}}
    +
            (-1)^nt^\delta
            \left(-a\hat w
            -(n+1) {ds\over dx}
            +(1-\delta) a+
             p\right){\p^{n}\over \p x^{n}}
        +\dots
                $$
Hence
                  $$
      0=  (-1)^{n+1}A^\dagger-A=
                   $$
                   $$
       t^\delta s{\p ^{n+1}\over \p x^{n+1}}
                  +
            t^\delta
            \left(a\hat w
            +(n+1) {ds\over dx}
            -(1-\delta) a-
             p\right){d^{n}\over dx^{n}}
             -t^\delta\
          \left(
    s{\p^{n+1}\over \p x^{n+1}}
        +a\hat w {\p^{n}\over \p x^{n}}+
    p{\p^{n}\over \p x^{n}}
               +
            \dots
          \right)
                  $$
                  $$
      =t^\delta
            \left((n+1) {ds\over dx}
            +(\delta-1) a-2p
               \right){\p^{n}\over \p x^{n}}+\dots\,.
                  $$
Hence we come to the condition:
               $$
               {n+1\over 2}{ds\over dx}
            +(\delta-1) a-2p=0\,,,\,{\rm i.e.}\,\, p={n+1\over 2}\left({ds\over dx}\right)+
{1\over 2}(1-\delta)\,.
               $$
We see that self-conjugate operator $A$ in the terms proportional to the order $n+1$ and $n$ looks like

         $$
       A=t^\delta
          \left(
    s{\p^{n+1}\over \p x^{n+1}}
        +a\hat w {\p^{n}\over \p x^{n}}+
    p{\p^{n}\over \p x^{n}}
               +
            \dots
          \right)=
          $$
          $$
          A=t^\delta
    s{\p^{n+1}\over \p x^{n+1}}
        +{1\over 2}
        \left(
        (n+1){ds\over dx}+
        (2\hat w+1-\delta)a(x)
               \right) {\p^{n}\over \p x^{n}}
            \dots
          $$
Now we study how transform $s$ and $a$ under coordinate transformation.

\medskip


{$\cal x$2. \sl Geometric meaning of Subprincipal symbol}

\medskip

  We consider how $a$ transforms with respect to an arbitrary coordinate transformation.



Consider arbitrary coordinate transformation: $(x,t)\mapsto (y,\tau)$\footnote{$^1$}
{it would be enough to fix the weight of the densities, i.e.
consider the action of the operator on the subspace
${\cal F}_\lambda$ of the densitites of the fixed weight $\lambda$ and consider only coordinate transformations $y=y(x)$ but we prefer to consider the general case.}.
         :
                $$
\cases {x=x(y)\cr t=x_y\tau}\,,\, \cases {y=y(x)\cr \tau=y_xt}
              $$

Then to calculate how operator $A$ will transform we note that $t=x_y\tau$,
         $$
      \hat w_{(t)}=t{\p\over \p t}=t=x_y\tau {\p \tau\over \p t}{\p\over \p\tau}=
      x_y\tau y_x {\p\over \p\tau}=\tau {\p\over \p\tau}=\hat w_{(\tau)}
         $$
and
             $$
       {\p\over \p x}=y_x{\p \over \p y}+{\p\tau\over \p x}{\p \over \p \tau}=
         y_x{\p \over \p y}+ty_{xx}{\p \over \p \tau}=
         y_x{\p \over \p y}+{y_{xx}\over y_x}w_{(\tau)}
             $$
Little bit work and we come to the following formula:
                   $$
   {\p^k\over \p x^k}=\left(y_x{\p \over \p y}+{y_{xx}\over y_x}\hat w_{(\tau)}\right)^k=
             $$
             $$
             y_x^k{\p^k\over \p y^k}+
   \left((1+2+\dots+(k-1))y_x^{k-2}y_{xx}+k\left({y_{xx}\over y_x}\right)y_x^{k-2}y_{xx}\hat w\right)
   {\p^{k-1}\over \p y^{k-1}}+\dots=
                   $$
                   $$
                   y_x^k{\p^{k}\over \p y^{k}}+
               \left({k(k-1)\over 2}+k\hat w\right)y_x^{k-2}y_{xx}
               {\p^{k-1}\over \p y^{k-1}}+\dots\,.
                   $$
 (This can be easily calculated by induction.)

           Now we are ready to calculate transformation of coefficients:
                  $$
                  A=t^\delta
    s{\p ^{n+1}\over \p x^{n+1}}
        +t^\delta{1\over 2}
        \left(
        (n+1){ds\over dx}+
        (2\hat w+\delta-1)a(x)
               \right) {\p^{n}\over \p x^{n}}+
            \dots=
                  $$
                  $$
                  \tau^\delta x_y^\delta
                           s
    \left(y_x {\p\over \p y}+{y_{xx}\over y_x}\hat w\right)^{n+1}
        +{1\over 2}
        \tau^\delta x_y^\delta
        \left(
        (n+1){ds\over dx}+
        (2\hat w+\delta-1)a(x)
               \right) \left(y_x {\p\over \p y}+{y_{xx}\over y_x}\hat w\right)^n+
            \dots=
                  $$
                  $$
                 \tau^\delta y_x^{-\delta}
                           s
                          \left(
                          y_x^{n+1}{\p ^{n+1}\over \p y^{n+1}}+
                          \left({n(n+1)\over 2}+(n+1)\hat w\right)y_{xx}y^{n-1}_x
                          {\p ^{n}\over \p y^{n}}
                          \right)
        +
        $$
        $$
        +{1\over 2}
        \tau^\delta y_x^{-\delta}
        \left(
        (n+1){ds\over dx}+
        (2\hat w+\delta-1)a(x)
               \right)
               y^n_x {\p^n\over \p y^n}+
            \dots
                  $$
Dots means terms which are of the order $\leq n-1$ with respect to variables $x,y$

Denote by $\tilde s=y_x^{n+1-\delta}s$ (principal symbol in new coordinates).
We come to 
              $$
    A=                 \tau^\delta
                          \left(
                          \tilde s{\p ^{n+1}\over \p y^{n+1}}+
                          \left({n(n+1)\over 2}+(n+1)\hat w\right)\tilde sy_{xx}y^{-2}_x
                          {\p ^{n}\over \p y^{n}}
                          \right)+
                        $$
                         $$
                                                {1\over 2}
        \tau^\delta
        \left(
        (n+1){ds\over dy}+
        (n+1)(\delta-n-1)y_{xx}y_x^{-2}\tilde s+
        (2\hat w+\delta-1)a(x)y_x^{n-\delta}
               \right)
         {\p^n\over \p y^n}
                 +\dots=
                         $$
                         $$
                         \tau^\delta
                          \tilde s{\p ^{n+1}\over \p y^{n+1}}+
        \tau^\delta {n+1\over 2}{ds\over dy}{\p^n\over \p y^n}+
                         $$
                         $$
                         {1\over 2}
        \tau^\delta
        (2\hat w+\delta-1)
        \left(
        a(x)y_x^{n-\delta}+(n+1)\tilde s{y_{xx}\over y_x^2}
        \right)
         {\p^n\over \p y^n}
                 +\dots=
                         $$
{\bf Claim}  We see that if in coordinates $x,t$
                 $$
A=t^\delta
    s{d^{n+1}\over dx^{n+1}}
        +t^\delta{1\over 2}
        \left(
        (n+1){ds\over dx}+
        (2\hat w+\delta-1)a(x)
               \right) {d^{n}\over dx^{n}}+
            \dots=
                 $$
then in new coordinates  $y=y(x)$, $\tau=y_xt$ ($t\sim |dx|, \tau\sim |dy|$)
              $$
          A=\tau^\delta
    \tilde s{d^{n+1}\over dy^{n+1}}
        +\tau^\delta{1\over 2}
        \left(
        (n+1){ds\over dy}+
        (2\hat w+\delta-1)\tilde a(x)
               \right) {d^{n}\over dy^{n}}+
            \dots=
              $$
  where
             $$
         \tilde s=sy_x^{n+1-\delta}\,,\,
               \tilde a=a(x)y_x^{n-\delta}+(n+1)\tilde s{y_{xx}\over y_x^2}=
               y_x^{-\delta}\left(a+(n+1)s{\p \log y_x\over \p x}\right)y_x^n\,.
                   $$

{\bf Remark} In the case $n=2$ this is just the connection on the volume forms. (Principal symbol equals to $2s$)

\medskip

{\bf Resum\'e }


In the general case $2a\over n+1$ is a "connection"\footnote{$^3$}{It transforms as upper connection. Here
 may be a coefficient $n+1\over 2$ is chosen wrong}. We denote
              $$
            \gamma={2a\over n+1}, a={(n+1)\gamma\over 2}
              $$
Then we can rewrite the operator
            $$
        A=t^\delta
    s{d^{n+1}\over dx^{n+1}}
        +t^\delta{1\over 2}
        \left(
        (n+1){ds\over dx}+
        (2\hat w+\delta-1)a(x)
               \right) {d^{n}\over dx^{n}}+
            \dots=
            $$
            $$
            t^\delta
      {d^{n+1}\over dx^{n+1}}
        +t^\delta{n+1\over 4}
        \left(
        2{ds\over dx}+
        (2\hat w+\delta-1)\gamma(x)
               \right) {d^{n}\over dx^{n}}+
            \dots=
                $$
\medskip



One can consider the canonical pencil of $n$-th order operators of the degree $\delta$
which send the density of the weight $\lambda$ to the density of the weight $\lambda+\delta$
               $$
          \Psi(x)|dx|^\lambda\mapsto
           \left(
    s{d^{n+1}\Psi(x)\over dx^{n+1}}+
        {n+1\over 4}\left(
        2{ds\over dx}+
        (2\lambda+\delta-1)\gamma(x)
               \right) {d^{n}\Psi\over dx^{n}}+
            \dots\right)|dx|^{\lambda+\delta}=
                          $$
where $\gamma$ is a connection


{\bf Exercise} Consider the previous construction for the case $n=1$ ("laplacian") and $n=0$ (vector field)

The next step is to consider the subsubprincipal symbol, which is of highly interest\footnote{$^4$}
{We have also to construct the operator globally, but this we can do defining a connection.}

 Before going to the next step note that for the canonical pencil above if we put:
                 $$
        \cases{\delta=n+1\cr 2\gamma+\delta=1}\leftrightarrow \cases {\lambda=-{n\over 2}\cr \delta=2}         
                 $$
then we come to the constructions of the book Ovsienko=Tabachnikov.

\medskip

{$\cal x$3. \sl Subsubprincipal symbol}


Now we check the condition of self-conjugancy ($A=(-1)^{n+1}A^\dagger$) up to the order $n-1$ with respect to $x$:

From previous considerations it follows that the operator
          $$
  A=t^\delta
           \underbrace{
           \left(
    s{\p^{n+1}\over \p x^{n+1}}
        +a\hat w {\p^{n}\over \p x^{n}}+b\hat w^2 {\p^{n-1}\over \p x^{n-1}}
          \right)}_
          {\hbox{terms of the order $n+1$}}+
          \underbrace{\left(
    p{\p^{n}\over \p x^{n}}
        +c\hat w {\p^{n-1}\over \p x^{n-1}}
          +\dots\right)}_{{\hbox{terms of the order $n$}}}+
         \underbrace{
                \left(
    q{\p ^{n}\over \p x^{n-1}}
        +\dots\right)
        }_{{\hbox{terms of the order $n-1$}}}+\dots=
          $$

         $$
         =t^\delta
          \left(
    s{\p^{n+1}\over \p x^{n+1}}
        +a\hat w {\p^{n}\over \p x^{n}}+
    p{\p^{n}\over \p x^{n}}
               +
            \dots
          \right)=
          $$
          $$
          A=t^\delta
    s{\p^{n+1}\over \p x^{n+1}}
        +{1\over 2}
        \left(
        (n+1){ds\over dx}+
        (2\hat w+1-\delta)a(x)
               \right) {\p^{n}\over \p x^{n}}+
               (b\hat w^2+c \hat w+q){\p^{n-1}\over \p x^{n-1}}+\dots
            \dots
                 $$

Now find the restrictions which are imposed by the condition $A=(-1)^{n+1}A^\dagger$:
       We have that
         $$
    (-1)^{n+1}A^\dagger\Psi=
    {\p^{n+1}\over \p x^{n+1}}(t^\delta s\Psi)-
    {n+1\over 2}{\p^{n}\over \p x^{n}}\left(t^\delta {ds\over dx}\Psi\right)-
             $$
             $$
    {n+1\over 2}{\p^{n}\over \p x^{n}}\left(t^\delta \left(\hat w+{\delta-1\over 2}\right)\gamma\Psi\right)+
    {\p^{n-1}\over \p x^{n-1}}\left(t^\delta (\hat w^2 b+\hat w c+q)\psi\right)=
         $$
         $$
     t^\delta
    s{\p^{n+1}\over \p x^{n+1}}
        +{1\over 2}
        \left(
        (n+1){ds\over dx}+
        (2\hat w+1-\delta)a(x)
               \right) {\p^{n}\over \p x^{n}}+
               $$
               $$
        {n(n+1)\over 2}t^\delta \left(\hat w+{\delta-1\over 2}\right){d\gamma\over dx}{\p ^{n-1}\Psi\over \p x^{n-1}}+
        t^\delta\left[b(1-\delta-\hat w)^2+c(1-\delta-\hat w)+q\right]{\p ^{n-1}\Psi\over \p x^{n-1}}+\dots
            \dots
         $$
         Comparing with operator $A$ we see that the condition $A=(-1)^{n+1}A^\dagger$ implies that
                     $$
                     {n(n+1)\over 2}\left(\hat w+{\delta-1\over 2}\right){d\gamma\over dx}+
                     b(1-\delta-\hat w)^2+c(1-\delta-\hat w)+q=
                     b \hat w^2+c\hat w+q
                     $$
Thus
                $$
             c={n(n+1)\over 2}
             {d \gamma\over dx}+b(\delta-1)
                $$

We denote $b={\theta\over 2}$. (To compare with second order operators.)

We come to the following statement:

  {\bf Theorem} The self-conjugate operator of the order $n+1$ on the algebra of densitites has the following appearance:
                    $$
            A=t^\delta
    s{\p^{n+1}\over \p x^{n+1}}
        +{1\over 2}
        \left(
        (n+1){ds\over dx}+
        (2\hat w+\delta-1)a(x)
               \right) {\p^{n}\over \p x^{n}}+
               $$
               $$
               {1\over 2}\left[
               \theta\hat w^2+\left(n(n+1){d\gamma\over dx}+\theta(\delta-1)\right) \hat w+q
               \right]{\p^{n-1}\over \p x^{n-1}}+\dots
            \dots
                                      $$

Here $s$ is the density of the weight $\delta-n-1$, $\gamma$ is connection
and Brans-Dicke scalar is related to Schwarzian.

Considering the restriction of this operator on the space  $\cal F_\lambda$
we come to the following pencil of operators.
Any density of the weight $\lambda$  $\Psi|dx|^\lambda$ is transformed to the density
of the weight $\mu\lambda+\delta$:
              $$
 \Psi|dx|^\lambda\mapsto \Phi(x)|dx|^{\lambda+\delta}
    $$
    where
              $$
              \Phi=
    s{d^{n+1}\Psi\over d x^{n+1}}
        +{1\over 2}
        \left(
        (n+1){ds\over dx}+
        (2\lambda+\delta-1)a(x)
               \right) {d^{n}\Psi\over d x^{n}}+
               $$
               $$
               {1\over 2}\left[
               \theta \lambda^2+\left(n(n+1){d\gamma\over dx}+\theta(\delta-1)\right) \lambda+q
               \right]{d^{n-1}\over d x^{n-1}}+\dots
              $$

\medskip


{$\cal x$4. \sl Special case}



Consider operator of the weight $\delta$ on the densitites o the weight $\lambda$ such that
             $$
             \cases
             {
           \delta-n-1=0\cr\,2\lambda+\delta-1=0\,.\cr }
             $$
i.e.
            $$
          \cases
             {
           \delta=1+n\cr\,\,\,\lambda=-{n\over 2} }
            $$
The principal symbol $s$ becomes the scalar we put it $s=1$, subprincipal symbol vanishes. We come to the operator
               $$
 \Psi|dx|^{-{n\over 2}}\mapsto \Phi(x)|dx|^{1+{n\over 2}}
    $$
    where
              $$
              \Phi(x)=
     {d^{n+1}\Psi\over d x^{n+1}}
        +
               {1\over 2}\left[
               \theta {n^2\over 4}+\left(n(n+1){d\gamma\over dx}+\theta n\right) \left({-n\over 2}\right)+q
               \right]{d^{n-1}\over d x^{n-1}}+\dots=
                             $$

                             $$
                             {d^{n+1}\Psi\over d x^{n+1}}+
                             \dots \theta
                             {d^{n-1}\Psi\over d x^{n+1}}+
                             \dots {d\gamma\over dx}
                             $$


The next step is to consider how $\theta$ transform under coordinate transformations.,,


\bye
