
\magnification=1200 %\baselineskip=14pt
\def\vare {\varepsilon}
\def\A {{\bf A}}
\def\t {\tilde}
\def\a {\alpha}
\def\K {{\bf K}}
\def\N {{\bf N}}
\def\V {{\cal V}}
\def\s {{\sigma}}
\def\S {{\Sigma}}
\def\s {{\sigma}}
\def\p{\partial}
\def\vare{{\varepsilon}}
\def\Q {{\bf Q}}
\def\D {{\cal D}}
\def\G {{\Gamma}}
\def\C {{\bf C}}
\def\M {{\cal M}}
\def\Z {{\bf Z}}
\def\U  {{\cal U}}
\def\H {{\cal H}}
\def\R  {{\bf R}}
\def\S  {{\bf S}}
\def\E  {{\bf E}}
\def\l {\lambda}
\def\degree {{\bf {\rm degree}\,\,}}
\def \finish {${\,\,\vrule height1mm depth2mm width 8pt}$}
\def \m {\medskip}
\def\p {\partial}
\def\r {{\bf r}}
\def\v {{\bf v}}
\def\n {{\bf n}}
\def\t {{\bf t}}
\def\b {{\bf b}}
\def\c {{\bf c }}
\def\e{{\bf e}}
\def\ac {{\bf a}}
\def \X   {{\bf X}}
\def \Y   {{\bf Y}}
\def \x   {{\bf x}}
\def \y   {{\bf y}}
\def \G{{\cal G}}
\def \F {{\cal F}}

\centerline {\bf Linear Differential Operators on the Real Line}




Consider the space $Dens({\bf R})$ as an inner product space with canonical inner product and the space of linear differential operators over this space. We then have that a (n-)self adjoint operator of weight n ($L^{\dagger} = (-1)^{n}L$), with top term invertable, has the following form to second order,
                  $$
L  = t^{n+\delta} \left[ s \p_{x}^{n} + {n\over 2}(\partial_{x} s + 2 s \gamma \widehat{w}_{n+\delta})\partial_{x}^{n-1} + {n(n-1)\over 2}(\partial_{x}(s\gamma) + s(\gamma^{2} + \tau)\widehat{w}_{n+\delta})\widehat{w}_{n+\delta}\partial_{x}^{n-2} \right.
          $$ \bye

\[ \left. + \frac{n(n-1)}{2} \left( \frac{n-2}{6}\partial_{x}(s)\gamma - \frac{n+1+3\delta}{6}s \partial_{x}(\gamma) - \frac{n+1 + 3\delta(\delta+1)}{12}s\gamma^{2} + s \rho \right) \partial_{x}^{n-2} + ... \right] \]

where $s$ is a density of weight $\delta$, $\gamma$ is a connection (on densities if weight 1), and $\tau$ and $\rho$ are both densities of weight 2. A particularly simplified structure becomes apparent when we consider s = 1, this also enforces the condition $\delta = 0$. Then if we act on the kernel of the operator $\widehat{w}_{n}$, i.e. densities of weight (1-n)/2 we come to the expression,

\[ L = t^{n} \left[ \partial_{x}^{n} - \frac{(n+1)n(n-1)}{12}( \partial_{x} \gamma + \frac{1}{2} \gamma^{2} + \tau ) \partial_{x}^{n-2} \right. \]

\[ \left. - \frac{(n+1)n(n-1)(n-2)}{24}\partial_{x}(\partial_{x}\gamma + \frac{1}{2}\gamma^{2} + \tau)\partial_{x}^{n-3} + ... \right] \]

Note that the coefficient of $\partial_{x}^{n-2}$ in this equation transforms in the same way as a multiple of a Schwarzian.


It is a classical result that to a non-degenerate projective curve, $f:\mathbb{R} \rightarrow \mathbb{R}P^{1}$, we can associate a second order \emph{self adjoint} differential operator (by non-degenerate we mean that our curve has non-zero speed). We shall now perform this construction - the language of densities is essential and a solution that is coordinate independant would be impossible without them. Firstly consider our linear operator above in the special case when n=2 acting on $-\frac{1}{2}$ densities;

\[ L = t^{2}( \partial^{2} + Q) \]

Then take two linearly independant solutions, $\phi_{i} \in Dens^{-\frac{1}{2}}(\mathbb{R})$ i =1, 2, and define the map $\Phi:\mathbb{R} \rightarrow \mathbb{R}P^{1}$ by $x \mapsto (\phi_{1}(x) : \phi_{2}(x))$. As The function $\Phi$ is well defined despite the fact that the $\phi$s are densities, this is because their ratio is a function and so is independant of frame. However the map, $L \mapsto \Phi$, is not completely well defined as we may choose another pair of solutions, but the dimension of the kernel of a Sturm-Liouville operator is 2 and therefore any solution may be written as a linear combination of $\phi_{1}$ and $\phi_{2}$. In particular if $\phi_{i}' \in Dens^{-\frac{1}{2}}(\mathbb{R})$ are another pair of linearly independant then

\[ (\phi_{1}' , \phi_{2}') = (\phi_{1} , \phi_{2}) g \quad \quad g \in GL^{2}(\mathbb{R}) \]

Therefore $\Phi' = \Phi \cdot g$, where $g$ is constant and the action is the standard one of projective transformations. This is equivalent to saying that the \emph{projective class} of the curve $\Phi$ is well defined.


Conversely let $f: \mathbb{R} \rightarrow \mathbb{R}P^{1}$ be a projective curve, we can then get a lift of this curve $F: \mathbb{R} \rightarrow \mathbb{R}^{2}$ \footnote[1]{That this lift exists is a standard result of covering space theory. Namely $S^{n}$ can be considered as the double cover of $\mathbb{R}P^{n}$ and therefore any map $f:X \rightarrow \mathbb{R}P^{n}$ can be lifted to $S^{n}$ if $f_{\ast}(\pi_{1}(X)) \subseteq p_{\ast}(\pi_{1}(S^{n}))$, where p is the covering map. In our case $X$ is contractable and hence $f_{\ast} = 0$.}


\end{document} 

\bye