\magnification=1200 %\baselineskip=14pt
\def\vare {\varepsilon}
\def\A {{\bf A}}
\def\t {\tilde}
\def\a {\alpha}
\def\K {{\bf K}}
\def\N {{\bf N}}
\def\V {{\cal V}}
\def\s {{\sigma}}
\def\S {{\Sigma}}
\def\s {{\sigma}}
\def\p{\partial}
\def\vare{{\varepsilon}}
\def\Q {{\bf Q}}
\def\D {{\cal D}}
\def\G {{\Gamma}}
\def\C {{\bf C}}
\def\M {{\cal M}}
\def\Z {{\bf Z}}
\def\U  {{\cal U}}
\def\H {{\cal H}}
\def\R  {{\bf R}}
\def\S  {{\bf S}}
\def\E  {{\bf E}}
\def\l {\lambda}
\def\degree {{\bf {\rm degree}\,\,}}
\def \finish {${\,\,\vrule height1mm depth2mm width 8pt}$}
\def \m {\medskip}
\def\p {\partial}
\def\r {{\bf r}}
\def\v {{\bf v}}
\def\n {{\bf n}}
\def\t {{\bf t}}
\def\b {{\bf b}}
\def\c {{\bf c }}
\def\e{{\bf e}}
\def\ac {{\bf a}}
\def \X   {{\bf X}}
\def \Y   {{\bf Y}}
\def \x   {{\bf x}}
\def \y   {{\bf y}}
%\def \G{{\cal G}}
\def \F {{\cal F}}
\def\s {\sigma}
\def\o {\omega}
\def \ggb {\Gamma_{_{\bullet}}}
\def \gb {\Gamma_{_{\bullet}}}
\def\pt{{\bf }}

\centerline{    {\bf Linear projective and... connections}}


    {\sl   When I was young I liked so much general constructions.
    Now I realize that special ("singular") cases are often most important. More: 
     When we try to construct an object that describes a "general stuff" it is very useful to fix everything
     that could be fixed \it {\bf canonically.
     
     Here after explaining general nonsense we will consider one example of the application of this motto.
   We try to define affine connection then after "tiding it up" we will come to linear connection, which is
   expressed in terms of Chrsitoffel symbols.}


\bigskip


    First of all general non-sense. Let $P\to M$ be a principal connection with structure group $G$. (We suppose that group $G$ has a right action.)   $G$-invariant connection is defined by the $G$-invariant distribution
    of $n$-planes which are transversal to fibres.

    $G$-invariant connection  defines global $1$-form $\Omega$ which is well-defined by the following relations:
                    
  $\bullet$   Let $\bf R$ be a vector attached at the point $\pt$
  is a vertical vector
      $\gamma_\R=pg(t)$
                    $$
               \omega(\pt ,\hat\X)=\X     
                    $$  
                    
  if vector $\X(\pt)$ is the value of fundamental vector field 




\bye 