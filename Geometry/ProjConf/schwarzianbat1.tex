          \magnification=1200
\def\p{\partial}
\def\t {\tilde}
\def \m {\medskip}
\def\degree {{\bf {\rm degree}\,\,}}
\def \finish {${\,\,\vrule height1mm depth2mm width 8pt}$}





\def\a {\alpha}
\def\vare{{\varepsilon}}
\def\l {\lambda}
\def\s {{\sigma}}

\def\G {{\Gamma}}

\def\A {{\bf A}}
\def\C {{\bf C}}
\def\E  {{\bf E}}
\def\K {{\bf K}}
\def\N {{\bf N}}
\def\Q {{\bf Q}}
\def\R  {{\bf R}}
\def\V {{\cal V}}
\def \X   {{\bf X}}
\def \Y   {{\bf Y}}
\def\Z {{\bf Z}}



\def\ac {{\bf a}}
\def\e{{\bf e}}
\def\f {{\bf f}}
\def\n {{\bf n}}
\def\r {{\bf r}}
\def\v {{\bf v}}
\def \x   {{\bf x}}
\def \y   {{\bf y}}


\def\pt {{\bf p}}
\def \exer {{\sl Exercise$\,\,$}}

 \centerline   {\bf Orbits of Batalin-Vilkovisky groupoid and... Schwarzian}
  \bigskip

 Many years ago with Ted we consider the {\it canonical} pencil of second
 order operators acting on densities.
                  $$
 \Delta_w({\bf \Psi})={1\over 2}
              \left[
              \p_a\left(S^{ab}\p_b \Psi\right)
                  +
                  (2w+\lambda-1)\gamma^a\p_a\Psi
                  +
                  \left(
                  w\p_a\gamma^a+w(w+\gamma-1)\theta
                  \right)|Dx|^{w+\lambda}
              \right]
              \eqno (1)
                  $$
 It is the operator of the weight $\lambda$ which acts on the
 density ${\bf \Psi}=\Psi(x)|Dx|^w$ of the weight $w$.

This is self-adjoint operator (with respect to canonical scalar product on the space of densitites which
is normalised by the condition $\Delta 1=0$.) (See the details in [1].)


 ${\bf S}=S^{ab}|Dx|^\lambda\p_a\p_b$ is density valued symmetric tensor field of the rank 2.

 $\gamma^a\p_a|Dx|^\lambda, \theta|Dx|^\lambda$ are connection and Brans-Dicke "scalar"
 (See the details in [1].),

We omit here the signs related with the fact that some of
coordinates are odd.



Here we will be interesting in the case when ${\bf S}=S^{ab}|Dx|^\lambda\p_a\p_b$ is {\it non-degenerate.}


Let $\gamma_a$ be Christoffel symbols of an arbitrary connection on volume forms. Then one can see that
             $$
          \gamma^a|Dx|^\lambda=S^{ab}\gamma_b|Dx|^\lambda,\,\,\, \theta=\gamma_aS^{ab}\gamma_b|Dx|^\lambda
             $$
are required density valued upper connection and Brans-Dicke "density".

We know already very well the geometrical meaning of this pencil for $\lambda=0$.

To understand this constructions for $\lambda\not=0$ it is illuminating to do the exercises:

 {\bf Exercise 1} Let $\Delta$ be a self-adjoint operator on densities $\Delta=\delta(\p_x, \hat w)$ on densities
  of the weight $\lambda=0$. Consider the volume form and the operator:
                      $$
                  \Delta'={1\over 2}\left(\rho (x)|Dx|^\lambda\Delta+\Delta \rho (x)|Dx|^\lambda\right)
                      $$
This is self-adjoint operator.  Its value for fixed $w$ gives the pencil/
 \medskip

{\bf Exercise 2}  Let us try to construct by hand an operator
$\Psi|dx|^w\,\,{\buildrel \Delta\over \longrightarrow}\,\, \Psi|dx|^{w+\lambda}$


Consider an arbitrary volume form $\rho(x)|dx|$ and the arbitrary connection $\gamma_a$.

Then consider density valued vector field
             $$
  {\bf \rm grad}{\bf \Psi}={\bf T}^a=T^a|Dx|^w=
  {\bf S}^{ab}\nabla_b {\bf \Psi}=\left(S^{ab}|Dx|^\lambda\right)\left(\p_a\Psi+w\gamma_a\Psi^a\right)|Dx|^w
             $$

  In the case if
              $$
             w+\lambda-1=0\,.
              $$
This vector field takes values in the density of the weight $1$ and one can consider its divergence:
  $\p_a{\bf T}^a$.
    In the general case we use volume form
    and we come to the operator
             $$
          \Delta {\bf \Psi}= \rho^{w+\lambda-1}{\bf \rm div} \left(\rho^{1-w-\lambda}{\bf \rm grad}{\bf \Psi}\right)=
            \rho^{w+\lambda-1}\p_a
            \left[
               \rho^{1-w-\lambda}
               \left[
              S^{ab}
              \left(
                \p_b\Psi+w\gamma_b\Psi
              \right)
               \right]
               \right]|Dx|^{w+\lambda}
             $$
             $$
             =\left[\p_a(S^{ab}\p_b\Psi)+
             \left(w+\gamma-1\right))\gamma^{(0)}_aS^{ab}\left(\p_b\Psi+w\gamma_b\Psi\right)+
                 w\p_a\left(\gamma^a\Psi\right)\right]|Dx|^{w+\l}=
             $$
             $$
\left[\p_a(S^{ab}\p_b\Psi)+
             \left( w\gamma^a+
             \left(w+\gamma-1\right)){\gamma^{(0)}}^a\right)\p_a\Psi+
                 w\p_a\gamma^a\Psi+
            w(w+\l-1){\gamma^{(0)}}^a\gamma_a
                 \Psi\right]|Dx|^{w+\l}
             $$
Here $\gamma^a, \gamma^{(0)a}$ are components of upeer valued connection-density:
 (e.g.$\gamma^a=S^{ab}\gamma_b$)

To simplify if   $\gamma^a=\gamma^{(0)a}$ then we come to
              $$
 \Delta{\bf\Psi}=\left[\p_a(S^{ab}\p_b\Psi)+
             \left( 2w+\gamma-1\right)\gamma^a\p_a\Psi+
                 w\p_a\gamma^a\Psi+
            w(w+\l-1)\gamma^a\gamma_a
                \Psi \right]|Dx|^{w+\l}
              $$

We see that

1)  we can put $\gamma^a=S^{ab}\gamma_b$ where $\gamma$ is usual connection and
$\theta=\gamma^a\gamma_a$.

 \medskip

 {\it Schwazian}

In what follows we assume that the symbol-density ${\bf S}$ is non-degenerate,
$\gamma_a$ are Christoffel symbols of connection , $\gamma^a={\bf S}^{ab}\gamma_b$ and  $\theta=\gamma^a\gamma_a$.


 Now consider the case if $2w+\l-1=0$, i.e.
         $$
         w={1-\l\over 2}\,\,.
         $$
  In this case
        $$
        \Delta{\bf\Psi}=\Delta\left(\Psi|Dx|^w\right)=\left[\p_a(S^{ab}\p_b\Psi)+
                    +
                 w\p_a\gamma^a\Psi+
            w(w+\l-1)\gamma^a\gamma_a
                 \Psi\right]|Dx|^{w+\l}
        $$
For a given connecion $\nabla$ with Christoffel symbol $\gamma_a$the orbit of Batalin-Vilkovisky groupoid is given nu differential equation:
             $$
          w\p_a\gamma^a+
            w(w+\l-1)\gamma^a\gamma_a=w\p_a\gamma'^a+
            w(w+\l-1)\gamma'^a\gamma'_a
             $$
i.e.
            $$
   w\p_a(S^{ab}\gamma_b)+w(w+\l-1)\gamma_aS^{ab}\gamma_b=w\p_a(S^{ab}\gamma'_b)+w(w+\l-1)\gamma'_aS^{ab}\gamma'_b
            $$
 The equation of vector field $\X=\gamma'-\gamma$ is
(C`est important: nous arrivons a invariant operator on vector densitites. Do it!)

In the special case $\lambda=0$ it is usual Batalin-Vilkoviskyu groupoid.

I consider here another special case:
     $$
     \lambda=2, 2w+\lambda-1=0, w=-{1\over 2}
     $$
In this case if the space is one-dimensional one can consider invariant vector density
        $$
      \V={\p\over \p x}|dx|
        $$
     and respectively the symbol-density of the weight 2
             $$
            {\bf S}=|dx|^2{\p\over \p x}{\p\over \p x}
              $$
The Laplacian becomes:
       $$
          \Delta{\bf\Psi}=
          \Delta\left(\Psi|Dx|^{-{1\over 2}}\right)=\left[\Psi''
                    -
                 {1\over 2}\gamma'\Psi
             -{1\over 4}\gamma^2
                 \Psi\right]|Dx|^2
       $$
let $\gamma$ be an arbitrary connection. Since $n=1$ there are coordinates such that
in these coordinates $\gamma=0$ and
 $$
 \Delta{\bf\Psi}=\Delta\left(\Psi|Dx|^{-{1\over 2}}\right)=\Psi''|Dx|^2
  $$


Find all coordinates such that the appearance of operator be the same.
(In the case of Batalin-Vilkovisky it was Darboux coordinates:)

        If $\gamma=0$ in given coordinates, then in coordinates $y=y(x)$, then $|dx|=x_y|dy|$ and
        $\gamma=-\p_y\log x_y=-{x_{yy}\over x_y}$. We see
         that
             $$
      \gamma'+{1\over 2}\gamma^2=\left(-{x_{yy}\over x_y}\right)_y+{x^2_{yy}\over x_y^2}=
 {3\over 2}{x^2_{yy}\over x_y^2}-{x_{yy}\over x_y}=-\hbox{Schwarzian of the function $y(x)$}
             $$

Hence we see that operator will have the form $ \Delta{\bf\Psi}=\Delta\left(\Psi|Dx|^{-{1\over 2}}\right)=\Psi''|Dx|^2$
 after arbitrary Mobius transformation.

 We see that Mobius transformations is the orbit of Batalin-Vilkovisky groupoid as
 in the odd sympelctic case the Darboux coordinates are the orbit of groupoid.



\bye
