

\magnification=1200
\def\vare {\varepsilon}
\def\a {\alpha}
\def\b {{\bf b}}
\def\c {{\bf c }}
\def\d {{\delta}}
\def\g{{\gamma}}
\def\p{\partial}
\def\l {{\lambda}}
\def\s {{\bf s }}
\def\t {\tilde}
\def\degree {{\bf {\rm degree}\,\,}}
\def \m {\medskip}
\def\v {{\bf v}}
\def\ac {{\bf a}}
\def\A {{\bf A}}
\def\D {{\cal D}}
\def\F{{\cal F}}
\def\G {{\Gamma}}
\def\L {{\cal L}}
\def\Z {{\bf Z}}
\def\U  {{\cal U}}
\def\R  {{\bf R}}
\def\S  {{\bf S}}
\def \X   {{\bf X}}
\def \Y   {{\bf Y}}
\def \F {{\cal F}}
\def\hl{{\widehat\lambda}}
\def\pr{{\prime}}
\def\ad {{\rm ad}}
\def \gb {\Gamma_{_{\bullet}}}
\def\pt {{\bf p}}
\def\0 {{_0}}
\def \finish {${\,\,\vrule height1mm depth2mm width 8pt}$}



   {\it Valya Ovsienko likes to say that Schwarzian has more than 600 different manifestations.
   Yesterday we discussed with my student Adam Biggs one of them.
   Here is the result of our discussions.}
   
   $\qquad$ $\qquad$$\qquad$$\qquad$$\qquad$$\qquad$ {\it HMK 26 June 2012.}



\m
\centerline{    \bf Schwarzian and ... normal gauging conditions }


   Normal gauging is tremendously powerfull tool in geometry.
   E.g. the quickest way to define invariants of gauge field (connection)
   is to consider connection $A_\mu(x)$ in so called "normal gauge":
                $$
A'_\mu(x)\colon \qquad A_\mu(x)(x^\mu-x_{_0}^\mu)=0\,,\quad
(A'_\mu=g A_\mu g^{-1}+g^{-1}\p_\mu g)\,.
\eqno (1)
                $$
(Here $A_\mu$ takes values in the Lie algebra Lie group $\cal G$, $g(x)$ is the function in $G$.)
Coefficients of Taylor series expansion are curvature of connection
               and its covariant derivatives at the point $\pt$.
               (Here $x^\mu$ are local coordiantes in the vicinity of the point $\pt$ 
               with coordinates $x^\mu_{_0}.$)
  E.g. if $A_\mu(x)$ is electromagnetic field given in a vicinity 
  of the point $x^\mu_{_0}=0$ in normal gauge
then $A_\mu(x)=F_{\mu\nu}x^\nu+\dots$ where $F_{\mu\nu}$ is the value of electromagnetic field tensor.
 Another example: if Riemannian metric is given in normal coordinates:
$
g_{\mu\nu} x^\nu=\delta_{\mu_\nu} x^\nu$ then
            $$
  g_{\mu\nu}(x)=\delta_{\mu\nu}+\dots R_{\mu \a\nu\beta} 
  (x^\a-x_{_0}^\a)(x^\beta-x_{_0}^\beta)+\dots
            $$
where $R_{\mu\a\nu\beta}$ is curvature tensor at the point $\pt$.
The normal gauging condition $g_{\mu\nu}x^\nu=\delta_{\mu_\nu} x^\nu$ is strictly related
with condition $\Gamma^\a_{\mu\nu}x^\mu x^\nu=0$ for geodesic coordinates.
You can read about normal gauge in different textbooks
\footnote{$^*$}{One can find the excellent exposition of
the geometry of normal gauging  in one of  Appendices of the famous
 article: "On the heat equation and the index theorem" of M. Atiyah, R. Bott and V.K. Patodi)}. 
 
   Now  {\it revenons a nos moutons}.   Let $x$ be a local coordinate on projective line $P \R^1$
   which fixes projective structure in a vicinity of the point $\pt$:
   one admits the changing of coordinates $x\mapsto {ax+b\over cx+d}$.
   Let $F(x)$ be a local expression for diffeomorphism of $P\R^1$.
   
   Recall that Shwarzian equals to the following density of the weight $2$:
                           $$
        {\cal S}^F=\left({F_{xxx}\over F_x}-{3\over 2}{F^2_{xx}\over F^2_x}\right)|Dx|^2\,.                   
        \eqno (2)
                           $$
     This is non-trivial $1$-cocycle of diffeomorphisms which vanishes on projective transformations.                      
   Projective transformations have three degrees of freedom.  Consider
   ''normal'' gauging of the diffeomorphism: the new diffeomorphism $F'$ such that
   it difffers form $F$ on projective transformation and $F'$ is identity in a vicinity of $\pt$
   up to the third order terms:
                             $
                F'\colon F=G\circ F'$ where $G={ax+b\over cx+d}$ is projective transformation such that
                                              $$
                             F'(x)=(x-x_{_0})+\hbox{\it O}((x-x_{_0})^3), \colon\qquad 
                          F'\big\vert_{x=\pt}=
                          {dF'(x)\over dx}\big\vert_{x=\pt}=
                          {d^2F'(x)\over dx^2}\big\vert_{x=\pt}=0\,.
                                              $$
  The value of gauged diffeomorphsim $F'$ at the point $\pt$ is the Schwarzian of  $F$ at the point $\pt$.
 Calculate  it.  
 
 If $
   F(x)=a+b(x-x_{_0})+c(x-x_{_0})^2+d(x-x_{_0})^3+\hbox{\it o}((x-x_{_0})^3)$
then consider composition of projective tansformations such that they ''kill'' derivatives:
$G_1\colon x\mapsto x-a$ (translation),
$G_2\colon x\mapsto x\mapsto {x\over b}$ and special projective transformation 
$G_3\colon x\mapsto {x\over 1+px}$ with $p={c\over b}$. Then we come to 
                    $$
F'(x)=G_3\circ G_2\circ G_1\circ F=
{(x-x_{_0})+{c\over b}(x-x_{_0})^2+{d\over b}(x-x_{_0})^3+\hbox{\it o}((x-x_0)^3)\over 1+
p((x-x_{_0})+{c\over b}(x-x_{_0})^2+\hbox{\it o}((x-x_{_0})^2))}=
                    $$
                    $$
      (x-x_0)^2+d'(x-x_0)^3+\hbox{\it o}((x-x_0)^3,
                    $$
where               $$
 d'={d\over b}-{c^2\over b^2}=\left({6F_{xxx}\over F_x}-{(2F_{xx})^2\over F^2_x}\right)\big\vert_{x=x_{_0}}=
 6\left({F_{xxx}\over F_x}-{3\over 2}{F^2_{xx}\over F^2_x}\right)\big\vert_{x=x_{_0}}\,.
                    $$
We come (up to a multiplier) to the Schwarzian (2).
Schwarzian appears in the way as curvature of gauge field....



\bye