\magnification=1200 %\baselineskip=14pt
\def\vare {\varepsilon}
\def\A {{\bf A}}
\def\t {\tilde}
\def\a {\alpha}
\def\K {{\bf K}}
\def\N {{\bf N}}
\def\V {{\cal V}}
\def\s {{\sigma}}
\def\S {{\Sigma}}
\def\s {{\sigma}}
\def\p{\partial}
\def\vare{{\varepsilon}}
\def\Q {{\bf Q}}
\def\D {{\cal D}}
\def\G {{\Gamma}}
\def\C {{\bf C}}
\def\M {{\cal M}}
\def\Z {{\bf Z}}
\def\U  {{\cal U}}
\def\H {{\cal H}}
\def\R  {{\bf R}}
\def\S  {{\bf S}}
\def\E  {{\bf E}}
\def\l {\lambda}
\def\degree {{\bf {\rm degree}\,\,}}
\def \finish {${\,\,\vrule height1mm depth2mm width 8pt}$}
\def \m {\medskip}
\def\p {\partial}
\def\r {{\bf r}}
\def\v {{\bf v}}
\def\n {{\bf n}}
\def\t {{\bf t}}
\def\b {{\bf b}}
\def\c {{\bf c }}
\def\e{{\bf e}}
\def\ac {{\bf a}}
\def \X   {{\bf X}}
\def \Y   {{\bf Y}}
\def \x   {{\bf x}}
\def \y   {{\bf y}}
\def \G{{\cal G}}
\def \F {{\cal F}}
\def\s {\sigma}
\def \ggb {\Gamma_{_{\bullet}}}
\def \gb {\Gamma_{_{\bullet}}}
           \centerline {\bf Geometry of differential operators on $\R$. II}


   {\bf Example of operator}

(All notations from the previous file)


We already know the canonical pencil of $n+1$-th order operators on
$\R$.

Now we will write down an example of $n+1$-th order operator on $\R$
provided with volume form $\rho\in \F_{1}$.








   Consider the operator:
              $$
              D\colon \qquad   \Psi |dx|^\s \mapsto
              D_\sigma\Psi=|dx|^{\s+1}\rho^\s{d\over dx}\left({\Psi\over \rho^\s}\right)=
              |dx|^{\s+1}\left({d\Psi\over dx}+\gb\Psi\right)
              $$
           which sends $\F_\sigma$ to $\F_{\sigma+1}$.

      (Here $\gb=-{d\over dx}\log \rho$ is a flat connection corresponding to the volume form $\rho$.)

One can assign to this operator the operator
              $$
              \hat D=t\left({\p\over \p x}+\gb\hat w\right)
              $$
on the whole space $\F$ of all the densities.

Operator $\hat D$ is self-adjoint operator.


Then we come to the following operator which sends $\F_\lambda$ to $\F_{\lambda+n+1}$:


 Consider $n$-th order operator
          $$
      L_n=\hat D^n=\left[t\left({d\over dx}+\gb\hat w\right)\right]^n=
      t^n\left({\p^n\over \p x^n}+A_n{\p^{n-1}\over \p x^{n-1}}+B_n{\p^{n-2}\over \p x^{n-2}}+\dots\right)
          $$
       One can see that
            $$
L_{n+1}=\hat D L_n=t\left({\p\over \p x}+\gb\hat w\right)t^n
\left({\p^n\over \p x^n}+A_n{\p^{n-1}\over \p x^{n-1}}+B_n{\p^{n-2}\over \p x^{n-2}}+\dots\right)=
            $$
            $$
            t^{n+1}\left({\p\over \p x}+\gb(n+\hat w)\right)
\left({\p^n\over \p x^n}+A_n{\p^{n-1}\over \p x^{n-1}}+B_n{\p^{n-2}\over \p x^{n-2}}+\dots\right)=
            $$
            $$
            t^{n+1}\left({\p^{n+1}\over \p x^{n+1}}+A_{n+1}{\p^{n}\over \p x^{n}}+
            B_{n+1}{\p^{n-1}\over \p x^{n-1}}+\dots\right)
            $$
Thus we come to recurrent relations. For ${A_n}$,
            $A_1=\gb \hat w,\,\, A_{n+1}=A_n+\gb(n+\hat w)$, i.e.
           $$
        A_n=n\gb \hat w+(1+2+\dots+n-1)\gb={n\over 2}\left(2\hat w+n-1\right)\gb
            $$
For $B_n$: $B_1=0$, $$B_{n+1}=B_n+\left(\p_x+(n+\hat w)\gb\right)A_n,
           =B_n+\left(\p_x+(n+\hat w)\gb\right){n+1\over 2}\left(2\hat w+n\right)\gb,\,\,{\rm i.e.}$$,
           $$
   B_{n+1}=B_n+{(n+1)(n+2\hat w)\over 2}\p_x\gb+{(n+1)(n+\hat w)(n+2\hat w)\over 2}\gb^2\,.
           $$
One can see (this  is long calculations) that
            $$
            B_{n}={n(n-1)\over 6}
            \left[
            \left(
            3\hat w+n-2
            \right)
            \gb^\prime+
           \left(
           3\hat w^2+3(n-1)\hat w+
            {(n-2)(3n-1)\over 4}
           \right)\gb^2
           \right]
            $$
({\it Tu a calculer ca...})

We see
         $$
   L_{n+1}= t^{n+1}\left[{\p^{n+1}\over \p x^{n+1}}+
          {n+1\over 2}\left(2\hat w+n\right)\gb{\p^{n}\over \p x^{n}}+
                B_{n+1}
            {\p^{n-1}\over \p x^{n-1}}+\dots\right]
         $$
({\it Coefficient $A_n$ is right here? ...})

 This operator defines the pencil:  It transforms the density $\Psi|dx|^\l$ of the weight $\l$ into the density
 $\Phi(x)|dx|^{\l+n-1}$ of the weight $\l+n-1$, where
          $$
   \Phi(x)={\p^{n+1}\Psi\over \p x^{n+1}}+
   {n+1\over 2}\left(2\l+n\right)\gb {\p^{n}\Psi\over \p x^{n}}+
              B_{n+1}
            {\p^{n-1}\Psi\over \p x^{n-1}}+\dots
          $$
If $\l=-{n\over 2}$ then
            $$
            \Phi(x)={\p^{n+1}\Psi\over \p x^{n+1}}+B_{n+1}
            {\p^{n-1}\Psi\over \p x^{n-1}}+\dots
            $$
One can see that it is $\sim$ to Schwarzian...
           $$
         B_{n+1}=c_{n+1}B_2
           $$
({\it Nous avons calculer hier})
\bigskip

     {\bf Second example}

     Go further. One can see that for $D=t(\p_x+\hat w\gb)$
             $$
             \hat wD-D\hat w=D
             $$
     This commutation relation implies that $n+1$-th order operator
            $$
         K_{n}=
         D^n\hat w+(-1)^{n+1}\hat w^\dagger
         \left(D^\dagger\right)^n=
         D^n\hat w+(-1)^{n+1}(1-\hat w)
         \left((-1)^nD\right)^=
            $$
            $$
         D^n(2\hat w+n-1)=t^n(2\hat w+n-1)\left(\p_x^n+\dots\right)
            $$
     is self-conjugate operator.

     Let $r=r(x)t$ be a density of the weight $1$. it is useful to consider also the  operator
     which is anticommutator:
          $$
       {\cal K}_{n}={1\over 2}\left(K_n\circ tr(x)+tr(x)\circ K_n \right)=
            {1\over 2}\left(D^n(2\hat w+n-1)\circ tr(x)+tr(x)\circ D^n(2\hat w+n-1) \right)
        $$
        $$
       =t^{n+1}(2\hat w+n)r(x)\left(\p_x^n+\dots\right)
          $$

{\bf Proposition }           Let $\Delta_{n+1}=t^{n+1}\p_x^{n+1}+\dots$ be an arbitrary
self-adjoint operator. It has the following
appearance:
                        $$
                    \Delta_{n+1}=t^{n+1}
                    \left(
                   \p_x^{n+1}
        +{n+1\over 2}
        (2\hat w+n)\ggb\p_x^{n}
                      +
                 \beta_{n+1}\p_x^{n-1}+\dots
            \dots
            \right)\,,
                        $$
                        where
                        $$
                     \beta_{n+1}={1\over 2}\left[
               \theta\hat w^2+\left(n(n+1)\ggb^\prime+n\theta\right) \hat w+q
               \right]
                        $$
(see the previous file. {\it I am not sure about the term $\beta_n$, recalculate it, please.})

We come to decomposition:
            $$
 \Delta_{n+1}=D^{n+1}+t^{n+1}{n+1\over 2}(2\hat w+n)r\p_x^n+\dots
            $$
           where $r=\ggb-\gb$.  On the other hand
                   $$
   t^{n+1}{n+1\over 2}(2\hat w+n)r\p_x^n+\dots=...{\cal K}_n+...\hbox{operator of the order $\leq n-1$ w.r.s. to $x$}
                   $$
where $\cal K$ is self-adjoint operator defined above.

\bigskip

     \centerline {\bf Another useful formulae}

  It is useful to rewrite self-conjugality conditions in terms of the corresponding pencil.
  Let $A$ be an operator on $\cal F$ of the weight $\delta$ and $A_\l$ the corresponding pencil:
             $$
         A_\l\colon \Psi(x)|dx|^\l\mapsto \left(A(x,\p_x,\hat w)\Psi(x)t^\l\right)\big\vert_{t=|dx|^\l+\delta}\,.     
             $$ 
           or : $A_\l=A(x,\p_x,\hat w)\big\vert_{\hat w=\l}$
           
One can see that 
       $$
     A_\l^\dagger = (A^\dagger)_{1-\delta-\l} 
       $$
and in particular 
         $$
          A_\l^\dagger=A_{1-\delta-\l}
         $$
if $A$ is self-adjoint operator.

\bigskip


It is useful to write down the "test-operator" which has the form $t^n\p_x\hat w^k$
   Note that 
               $$
            \hat w^m D^n=D^n(\hat w+n)^m\,.   
               $$
  Consider the following (anti)self-adjoint operator produced via the operator $\hat w^r D^n$ of the order $r+n$:
              $$
     K_n^m=\hat w^m D^n+(-1)^{n+m}\left(\hat w^m D^n\right)=
     \hat w^m D^n+(-1)^m\left(D^n\hat w^{\dagger m}\right)=
     \hat w^m D^n+(-1)^m\left(D^n\left(1-\hat w\right)^m\right)=
              $$
             $$
             \hat w^m D^n+\left(D^n\left(\hat w-1\right)^m\right)=
             D^n\left(
             \left(\hat w+n\right)^m+(\hat w-1)^m
             \right)
             $$
We come to self-adjoint operator of the weight $n$
              $$
   K_n=D^n\left((\hat w+n)^m+(\hat w-1)^m\right)=t^n\p_x^n
   \left(2\hat w^m+(n-2)\hat w^{m-1}+\dots
   \right)+\dots
                  $$
                  If $r$ is a density of the weight $l$ then calculating anticommutator
                  we come to the self-adjoint operator
                $$
   {\cal K}_n^{m}={1\over 2}\left[K_n^m, s\right]=t^{n+l}\p_x^n \left(\hat w^r+\dots\right)             
                $$           

it is a basic operator.
\bye
