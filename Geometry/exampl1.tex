\magnification=1200


  \centerline {\bf Symmetries}

\bigskip

    \centerline{{\bf Questions} (First working session)}
\bigskip

You know about points and lines of symmetries.

We are going to discuss symmetries of polygons: triangles,
squares....

\bigskip


{\bf 1.}  How many symmetries are there of an isosceles triangle $\triangle ABC$ ($AC=BC$)?

\medskip

{\it Note: In Maths, we also include as a symmetry, the operation of "leaving it
alone'' --- called the identity operation, and written $I$. So in fact every
object has at least one symmetry, the identity.}



\bigskip

{\bf 2}
  How many symmetries are there of an equilateral triangle  $\triangle ABC$ ($AC=BC=AB$)?

\medskip

{\bf 3} Find at least two symmetries of equilateral triangle and combine them,
   i.e. if $R$ and $S$ are two symmetries consider $R\circ S$, i.e. first do  $S$
   then $R$.  It will be also a symmetry?



\bigskip


Symmetries arise in Algebra as well as Geometry.  For example, if you have an
expression in 2 variables then some expression are symmetric when you
exchange the 2 variables.  For example  $F(A,B)=A+B=F(B,A)=B+A$ is symmetric expression.
  An expression  $G(A,B)=A^3+B$ is not symmetric, because $G(B,A)=B^3A\not =G(B,A)=A^3B$

\medskip

{\bf 4} Which of the following expressions are symmetric?

$A-B$

$(A-B)^2$

$A^2+B^2$

$A^2-B^2$



$A^2+B^2+C^2$

$(A-B)(B-C)(C-A)$





\bye
