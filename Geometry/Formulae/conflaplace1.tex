\magnification=1200 

\baselineskip=14pt
\def\vare {\varepsilon}
\def\A {{\bf A}}
\def\a {\alpha}
\def\N {{\bf N}}
\def\s {{\sigma}}
\def\bs {{\bf s}}
\def\S {{\Sigma}}
\def\s {{\sigma}}
\def\d {{\delta}}
\def\p{\partial}
\def\vare{{\varepsilon}}
\def\Q {{\bf Q}}
\def\D {{\cal D}}
\def\g {{\Gamma}}
\def\C {{\bf C}}
\def\M {{\cal M}}
\def\Z {{\bf Z}}
\def\U  {{\cal U}}
\def\H {{\cal H}}
\def\R  {{\bf R}}
\def\S  {{\bf S}}
\def\E  {{\bf E}}
\def\l {\lambda}
\def\degree {{\bf {\rm degree}\,\,}}
\def \finish {${\,\,\vrule height1mm depth2mm width 8pt}$}
\def \m {\medskip}
\def\p {\partial}
\def\g {{\Gamma}}
\def\v {{\bf v}}
\def\n {{\bf n}}
\def\t {{\tilde}}
\def\b {{\bf b}}
\def\c {{\bf c }}
\def\e{{\bf e}}
\def\ac {{\bf a}}
\def \X   {{\bf X}}
\def \Y   {{\bf Y}}
\def \x   {{\bf x}}
\def \y   {{\bf y}}
\def \G{{\cal G}}
\def\w{\omega}
\def\pt {{\bf pt}}
\def\finish {${\,\,\vrule height1mm depth2mm width 8pt}$}


  It is well-known that Laplacian is 
        $$
\Delta+c_nR
  \eqno (1)
        $$
\def\t {\tilde}
\def\hDelta {\hat\Delta}
 invariant with respect to conformal transformations
     $$
   \t g_{ik}=e^{2\s}g_{ik}
       \eqno (2)
     $$
Here  $c_n$ is conastand depending on dimension of the space, $R$
is a scalar curvature.

What is the exact statement?

  Why $R$ appears?


Let $\bs=s|Dx|^\l$ be a density of weight $\l$. Consider our 
operator
          $\hat\Delta$ acting on this density:
          $$
 \hDelta_{g,\g}=\hDelta\bs= \left(
   \p_m(g^{mn}\p_ns)+(2\l-1)\g^i\p_i s
         \right)|Dx|^\l+\l\p_i\g^i \bs+
   \l(\l-1)\g^i\g_i\bs
      \eqno (3)
           $$
here $\g_i$ is a connection on densities. We know that
          $$
   \hDelta\bs =\rho^\l\Delta(\rho^{-\l}\bs )\,,
      \eqno (4)
          $$
where $\rho=\sqrt {\det g}|Dx|$ is volume form on Riemannian manifold,
   $\g_i=-\p_i\log\rho$ and $\Delta$ is Laplace-Beltrami Laplacian
 on functions:
             $$
\Delta F={1\over\rho}\p_m\left(\rho g^{mn}\p_nF\right)\,
           \eqno (5)
             $$


Now study how operator $\hDelta$ changes under conformal transformations:
         $$
g\to\t g=e^{2\s}g\,,
  \rho\to\t\rho=\rho(\sqrt{\det g})^{n\over 2}=\rho e^{n\s}\,, 
\g_i\to \t\g_i=-\p_i\log\t\rho=\g_i-n\p_i\s\,.
         \eqno (6)
         $$
Then we come to operator
       $$
\hDelta'=\hDelta'_{\t g,\t \g}= 
   \p_m(\t g^{mn}\p_n)+(2\l-1)\t\g^i\p_i+\l\p_i\t\g^i +
   \l(\l-1)\t\g^i\t\g_i
        \eqno (7)
       $$
   Study the relation between operators $\hDelta'$ and $\hDelta$.

Using equations (6) we come to
         $$
\hDelta'=\hDelta_{\t g,\t \g}=
   \p_m\left( e^{-2\s}g^{mn}\p_n\,\right)+
 (2\l-1)\t\g^i\p_i 
         +\l\p_i\t\g^i +
   \l(\l-1)\t\g^i\t\g_i=
         $$
         $$
   \p_m\left( e^{-2\s}g^{mn}\p_n\,\right)+
    (2\l-1)e^{-2\s}g^{ij}(\g_j-n\p_j\s)\p_i 
         +\l\p_i\left(e^{-2\s}g^{ij}
       \left(\g_j-n\p_j\s\right)\right) +
          $$
          $$
   \l(\l-1)e^{-2\s}g^{ij}(\g_i-n\p_i\s)
    (\g_j-n\p_j\s)=
         $$
         $$
         e^{-2\s}
       \left[
        \p_m\left(g^{mn}\p_n\,\,\right)
        -2\p_m\s
      g^{mn}\p_n\,\,\right]+
            $$
            $$+
    e^{-2\s}(2\l-1)g^{ij}(\g_j-n\p_j\s)\p_i+
            $$
            $$ 
         +e^{-2\s}\l
              \left[
                 -2\p_i\s g^{ij}(\g_i-n\p_j\s)+
           \p_i\g^i-
               n\p_i \left(g^{ij}\p_j\s\right)
            \right] +
              $$
              $$
   e^{-2\s}\l(\l-1)g^{ij}(\g_i-n\p_i\s)
    (\g_j-n\p_j\s)
       =
         $$
      $$
e^{-2\s}\left[ 
  \p_m( g^{mn}\p_n\,\,)+(2\l-1)\g^i\p_i+\l\p_i\g^i +
   \l(\l-1)\g^i\g_i
\right]
     $$
     $$
   -e^{-2\s}\left[2+n(2\l-1)\right](\nabla\s)^m\p_m
+
   $$
      $$
\l e^{-2\s}
 \left[-2\p_i\s\g^i+2n(\nabla\s)^m\p_m\s-
 n\p_i(\nabla^i\s)\right]+       
     $$
     $$
-2e^{-2\s}\l(\l-1)n\g^i\p_i\s+
+e^{-2\s}\l(\l-1)n^2\nabla^i\s\p_i\s\,,
     \eqno (8)
        $$
where $\nabla^i\s={\rm grad\,}_g \s=g^{ij}\p_j\s$.  
Use also that
      $$
{\cal L}_\A\bs=
(A^m\p_m s+\l\p_mA^m)|Dx|^\l
    \eqno (9)
      $$
and the fact that:
       $$
\Delta\s=\p_n\nabla^m\s=\p_n(g^{mn}\p_m\s)-\g^m\p_m\s=
(\hDelta)_{\l=0}\s\,.
\eqno (10)  
       $$
Hence we have
        $$
\hDelta'\bs=\hDelta_{\t g=e^{2\s}g}\bs=e^{-2\s}\hDelta_{g}\bs
        $$
       $$
-\left[2+n(2\l-1)\right]{\cal L}_{\nabla\s}\bs+
\l\left[2+n(2\l-1)\right]\p_m\nabla^m\s\bs+
       $$
       $$
\l e^{-2\s}
 \left[-2\p_i\s\g^i+2n(\nabla\s)^m\p_m\s-
 n\p_i(\nabla^i\s)\right]\bs+       
     $$
     $$
-2e^{-2\s}\l(\l-1)n\g^i\p_i\s
+e^{-2\s}\l(\l-1)n^2\nabla^i\s\p_i\s\bs=
    $$
     $$
e^{-2\s}
    \left(
       \Delta\bs-
\left[2+n(2\l-1)\right]{\cal L}_{\nabla\s}\bs+
   2\l[1+n(\l-1)]\left(\nabla^i\s\p_i\s-\g_i\s\right)\bs+
    n\l\left[2+n(\l-1)\right]\nabla^i\s\p_i\s\bs
    \right)
     $$
  $$
=e^{-2\s}\left(
       \Delta\bs-
\left[2+n(2\l-1)\right]{\cal L}_{\nabla\s}\bs+
   2\l[1+n(\l-1)]\left(\Delta\s\right)\bs+
    n\l\left[2+n(\l-1)\right]\nabla^i\s\p_i\s\bs
    \right)
     $$
or in other way:
        $$
e^{2\sigma}\hDelta'=\hDelta-
   \left(2+n(2\l-1)\right){\cal L}_{\nabla\s}+
   2\l\left(1+n(\l-1)\right)\Delta\s+
    n\l\left(2+n(\l-1)\right)\nabla^i\s\p_i\s\,.
        \eqno (11)
           $$
or in other way:
             $$
     \hDelta'=e^{-2\sigma}
         \left(\hDelta+a(n,\l){\cal L}_{\nabla\s}+\Phi\right)\,,
                $$
where
                 $$
 a(n,\l)=\left(2+n(2\l-1)\right)\,,\,\,
  \Phi=
   2\l\left(1+n(\l-1)\right)\Delta\s+
    n\l\left(2+n(\l-1)\right)\nabla^i\s\p_i\s\,.
        \eqno (11a)
             $$

  We see that symbols of operator pencils
$\hDelta'$ and $\hDelta$ coincide up to multiplier
$e^{2\s}$. The pencil $\hDelta$ defines self-adjoint operator
in algebra of densities. The
 operator $e^{2\s}\hDelta'$ differs from operator $\Delta$
on antiself-adjoint operator $\cal L$ and scalar function.

Hence one can find a value of $\l$ such that operator is almost the same
up to scalar:
        $$
2+(2\l-1)n=0\,\, {\rm i.e.}\,\,\l_{_0}={1\over 2}-{1\over n}
        $$
In this case we have that for equation (11)
     $$
  2\l(1+n(\l-1))\big\vert_{\l=\l_0}={n-2\over 2}\,,\, 
  n\l(2+n(\l-1))\big\vert_{\l=\l_0}=-{(n-2)^2\over 4}\,.
       \eqno (12) 
     $$
 We see that on the densities of weight 
$\l_0={1\over 2}-{1\over n}$ the following realtion holds:
        $$
\hDelta'={e^{-2\s}}
        \left(
          \hDelta-
   {n-2\over 2}\Delta\s-
    {(n-2)^2\over 4}\p_m\s\nabla^m\s
        \right)\,.
    \eqno (13)
        $$

Thus we come to construction of the operator:
       $$
L=\hDelta-{n-2\over 4(n-1)}R\,\,
\hbox{acting on densities of weight $\l={1\over 2}-{1\over n}$}.
       $$

\bigskip

  Now we want to go further. Return to the formula (11)

             $$
     \hDelta'=e^{-2\sigma}
         \left(\hDelta+a(n,\l){\cal L}_{\nabla\s}+\Phi\right)\,,
                $$
where
                 $$
 a(n,\l)=\left(2+n(2\l-1)\right)\,,\,\,
  \Phi=
   2\l\left(1+n(\l-1)\right)\Delta\s+
    n\l\left(2+n(\l-1)\right)\nabla^i\s\p_i\s\,.
        \eqno (11a)
             $$
ecall that operator pencil $\hDelta$ is constructed vial Beltrami-Laplace,
and we know that $\hDelta$ is self-adjoint operator.
       Consider operator
                $$
    \hDelta_{\delta}=\rho^\delta\hDelta=(\det g)^{n\over 2}|Dx|^\delta\hDelta
                $$
 This operator sends $\l$-densities to $\l+\delta$ dentisites.
This is easy exercise to check that this is alos self-adjoint operator!
   We denote by $\hDelta$ the Beltrami-Laplac pencil 
defined by Riemannian ,metric $g_{ik}$
  $\hDelta'$ the pencil cooresponding to the metric
$\tilde{ g_{ik}}=e^{2\sigma}g_{ik}$,
and we denote by $\hDelta_{(\delta)}$
the weighted pencil $\hDelta_{(\delta)}=\rho^\delta\hDelta=
\rho^\d\hDelta_{(0)}$.

  The relation (11a) will take the form:
              $$
\hDelta'_{(\d)}=\tilde\rho^\delta\hDelta'=
(\det \tilde g)^{n\delta\over 2}|Dx|^\d\hDelta'=
 e^{n\s\delta}\rho^\delta\hDelta'=
e^{n\s\delta}e^{-2\sigma}\rho^\delta
         \left(\hDelta+a(n,\l){\cal L}_{\nabla\s}+\Phi\right)=
                 $$
                 $$
e^{(n\delta-2)\s}\rho^\delta
         \left(\hDelta+a(n,\l){\cal L}_{\nabla\s}+\Phi\right)=
e^{(n\delta-2)\s}
         \left(\hDelta_{(\d)}+
a(n,\l)\rho^\d{\cal L}_{\nabla\s}+\rho^\d\Phi\right)\,.
$$

  Now we put
               $$
              \delta={2\over n}\,,\quad
 {\rm then}\,\,\, \rho^\d=\det g\,.
                $$
We come to the statement:

     If weight $\d={2\over n}$ then the weighted pencil
  obeys the transformation:
                $$
\hDelta'_{{2\over n}}=
\hDelta_{{2\over n}}+\det g\left(a(n,\l){\cal L}\nabla\s+\Phi\right)
                $$
Note  that
              $$
              a(n,\hat\l)^*=-a(n,\l)
              $$
We come to

{\bf Proposition} I
Weighted Operators $\hDelta,\hDelta'$ 
both are self-adjoint weighted operators
with the same principal symbol in the case if $\delta={1\over 2n}$.

Using Vornov-Khudaverdian Theorem we come to the conclusion that
          $$
  \hDelta=t^\d\left(S^{ab}\p_b\p_a+\p_bS^{ba}\p_a+
  +\left(2\hat\l+\d-1\right)\g^a\p_a+\hat\l\p_a\g^a+
\hat\l(\hat\l+\d-1)\g^a\g_a\right)\,
          $$ 
and        $$
  \hDelta=t^\d\left(S^{ab}\p_b\p_a+\p_bS^{ba}\p_a+
  +\left(2\hat\l+\d-1\right)\g^a\p_a+\hat\l\p_a\g^a+
\hat\l(\hat\l+\d-1)\g^a\g_a\right)\,
          $$ 
where $S^{ab}=\rho^\delta g^{ab}$
\bye

