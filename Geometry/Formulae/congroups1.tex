




\magnification=1200 

\baselineskip=14pt
\def\vare {\varepsilon}
\def\A {{\bf A}}
\def\a {\alpha}
\def\N {{\bf N}}
\def\s {{\sigma}}
\def\bs {{\bf s}}
\def\S {{\Sigma}}
\def\s {{\sigma}}
\def\d {{\delta}}
\def\p{\partial}
\def\vare{{\varepsilon}}
\def\Q {{\bf Q}}
\def\D {{\cal D}}
\def\g {{\Gamma}}
\def\C {{\bf C}}
\def\M {{\cal M}}
\def\Z {{\bf Z}}
\def\U  {{\cal U}}
\def\H {{\cal H}}
\def\R  {{\bf R}}
\def\S  {{\bf S}}
\def\E  {{\bf E}}
\def\l {\lambda}
\def\degree {{\bf {\rm degree}\,\,}}
\def \finish {${\,\,\vrule height1mm depth2mm width 8pt}$}
\def \m {\medskip}
\def\p {\partial}
\def\g {{\Gamma}}
\def\v {{\bf v}}
\def\n {{\bf n}}
\def\t {{\tilde}}
\def\b {{\bf b}}
\def\c {{\bf c }}
\def\e{{\bf e}}
\def\ac {{\bf a}}
\def \X   {{\bf X}}
\def \Y   {{\bf Y}}
\def \x   {{\bf x}}
\def \y   {{\bf y}}
\def \G{{\cal G}}
\def\w{\omega}
\def\pt {{\bf pt}}
\def\finish {${\,\,\vrule height1mm depth2mm width 8pt}$}

In mathematical physics if one says `conformal structure of 
Riemannian manifold $M$'
it means or

{\bf A}  Transformations of manifold $M$ which preserve
  metric $g$.   

{\bf B}  It is considered Equivalence class, conformal class  $[g]$
of Riemannian metric. Metric $g'$ belongs to conformal class $[g]$
if 
           $$
     g'=e^{\s(x)}g
           $$  

If dimension of $M$ is greater than $2$ then conformal structure {\bf A}
possesses at most finite parametric family of transofrmations 
  For example all conformal transformations of Euclidean space $V^n$
are exhausted by orthogonal transformations, homothety,
translations and inversion
if $n\geq 3$ (Lioville Theorem). 
The algebra of conformal transofrmations
$c0(n)$ is nothing but $so(n+1.1)$. 
Infinitesimal conformal transformations are: 
        $$
\hbox {infinitesimal rotations}:  x^i\mapsto x^i+\vare B^i_kx^k, \quad
     (B^i_k=-B^k_i)\,
        $$ 
        $$
\hbox {infinitesimal translations}:  x^i\mapsto x^i+\vare t^i, \quad
        $$
dilations (homothety):
             $$
          x^i\mapsto x^i+\vare x^i\,,
             $$ 
and special conformal transformations generated by 
inversion and translations:
             $$
  K^i\colon\qquad x^i\mapsto x^i+\vare K^i=
       O\left(Ox^i+\vare t^i\right)\,,
             $$
where $O$ is inversion with respect to origin: $Ox^i={x^i\over |x|^2}$,
             $$
x^i\mapsto x^i+\vare K^i=
       O\left(Ox^i+\vare t^i\right)=
      {{x^i\over |x|^2}+\vare t^i
          \over
       \left|{x^i\over |x|^2}+\vare t^i
\right|^2  }=
             $$
            $$
      {{x^i\over |x|^2}+\vare t^i
          \over
       {1\over |x|^2}+{2\vare t^ix^i\over |x|^2}
 }={x^i+\vare t^i|x|^2
          \over
       1+2\vare t^ix^i
 }=x^i+\vare t^i(x,x)-2\vare x^i(t,x)
          $$
Thus we see that generators of algebra $co(n)$ are
               $$
\underbrace
 {x^i\p_j-x^j\p_i}_{\hbox {inf.rotations}}
      \,,
\underbrace
 {\p_i}_{\hbox {inf.translations}}
      \,,
\underbrace
 {x^m\p_m}_{\hbox {inf.homothety}}
      \,,
\underbrace
 {|x|^2\p_i-2x^ix^m\p_m}_{\hbox {inf.special conf.transform.}}
      \,.
               $$
We have
                   $$
|co(n)|={n(n-1)\over 2}+n+1+n={(n+2)(n+1)\over 2}=|so(n+1.1)|
                   $$

One can show that conformal transformations of $R^n$ are orthogonal
of conic in $R^{n+1.1}$


Now consider second group.

It is related with the group $CO_0(n)=O(n)\otimes \R^*$
(i.e. orthogonal transformations and homothety)

Its algebra is $co_0(n)\oplus \R$. 

{\bf Lemma} The algebra $co(n)=so(n+1.1)$
is Cartan prolongaion of the algbera $co_0(n)$.

 {\sl Proof of the Lemma}

First recall Cartan prolongation

Let $\G=\G_0$ be an arbitrary Lie algebra.

   Consider graded Lie algebra
             $$
    \G=  \G_{-1}\oplus \G_{0}\oplus\G_{1}\oplus\dots\,,
             $$
where  
        $$
\G_{0}\quad \hbox{is Lie algebra $\G_{0}$}\,,
        $$
        $$
\G_{-1}=V\quad \hbox{is the vector space,
such that $\G_0(n)$ faithfully acts on $V$}\,,
        $$
We assume:
       $$
[v_1,v_2]=0\quad \hbox{for every two elements $v_1,v_2\in \G_{-1}$}
       $$
and
        $$
    [h,v]=-[v,h]=h(v) \quad 
\hbox {for arbitrary $h\in \G_{0}$ and $v\in \G_{-1}$}\,
        $$
 ($h(v)$ is the actioh of element $h$ on vector $v$.)
   Now we construct  the spaces   $\G_{i}$ ($i=1,2,3,\dots$)
 {\it $i$-th Cartan prolongations of Lie algebra $\G_{0}$}
in the following way:  

   Consider the space  
            $$
               T_i(V)=
  Hom\left(
     \underbrace{V\times V\times\dots\times V}_{\hbox{$i+1$ times}}\,,
                V
             \right)
     \underbrace{V\otimes V\otimes\dots\otimes V}_{\hbox{$i+1$ times}}\otimes V
            $$      
of tensors of rank $i+2$ of valency $\pmatrix{1\cr i+1\cr}$.

   Now we define 
vector space $\G_{i}$ as the subspace 
of tensors in $T_i(V)$ which are symmetric:
                    $$
 T_i(V)\ni  
          t\left(
        \dots 
    \underbrace {u}_{\hbox{$m$-th place}}
    \dots
    \underbrace {v}_{\hbox{$n$-th place}}
       \dots
      \right)
             =
              t\left(
        \dots 
    \underbrace {v}_{\hbox{$m$-th place}}
    \dots
    \underbrace {u}_{\hbox{$n$-th place}}
       \dots
      \right)
                    $$
and obey the following condition:

  for arbitrary $i$ vectors
 $v_1,\dots,v_i\in V$ there exists an element $h\in \G_{0}$
such that
            $$
    h(u)=t\left(u,v_1,\dots,v_i\right)
            $$
for arbitrary vector $u\in \R^n=\G_{-1}$.

   The Lie commutator in spaces $\G_{i}$ (i=1,2,3,\dots) is the following:
for every $t\in \G_i$, $v\in \G_{-1}$
                    $$
         \G_{i-1} \ni [t,v]\colon\quad
            [t,v](u_1,\dots,u_{i})=t(v,u_1,\dots,u_i)\,,
                    $$
        
for every $t\in \G_i$, $h\in \G_{0}$
                    $$
         \G_{i} \ni [h,t]\colon\quad
            [h,t](u_1,\dots,u_{i+1})=
          $$
          $$
           t\left([h,u_1],u_2,\dots,u_{i+1}\right)+
           t\left(u_1,[h,u_2],u_3,\dots,u_{i+1}\right)
           +\dots+
           t\left(u_1,u-2,\dots, u_i, [h,u_{i+1}]\right)
    \hbox{(up to a constant)}                  \,,
                    $$
(up to a constant)
of symmetric tensors in the space 
$Hom(V\times V,V)=V^*\times V^*\times V$ such that for every vectors
   $v_1,v_2\in V$
          $$
  t(v)
          $$


      $$
\G_{0}\quad \hbox{is Lie algebra $co_0(n)$}\,,
        $$

\bye
