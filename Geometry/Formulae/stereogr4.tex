\magnification=1200 

\baselineskip=14pt
\def\vare {\varepsilon}
\def\A {{\bf A}}
\def\t {\tilde}
\def\a {\alpha}
\def\K {{\bf K}}
\def\N {{\bf N}}
\def\V {{\cal V}}
\def\s {{\sigma}}
\def\S {{\Sigma}}
\def\s {{\sigma}}
\def\p{\partial}
\def\vare{{\varepsilon}}
\def\Q {{\bf Q}}
\def\D {{\cal D}}
\def\G {{\Gamma}}
\def\C {{\bf C}}
\def\M {{\cal M}}
\def\Z {{\bf Z}}
\def\U  {{\cal U}}
\def\H {{\cal H}}
\def\R  {{\bf R}}
\def\S  {{\bf S}}
\def\E  {{\bf E}}
\def\l {\lambda}
\def\degree {{\bf {\rm degree}\,\,}}
\def \finish {${\,\,\vrule height1mm depth2mm width 8pt}$}
\def \m {\medskip}
\def\p {\partial}
\def\r {{\bf r}}
\def\v {{\bf v}}
\def\n {{\bf n}}
\def\t {{\bf t}}
\def\b {{\bf b}}
\def\c {{\bf c }}
\def\e{{\bf e}}
\def\ac {{\bf a}}
\def \X   {{\bf X}}
\def \Y   {{\bf Y}}
\def \x   {{\bf x}}
\def \y   {{\bf y}}
\def \G{{\cal G}}
\def\w{\omega}
\def\pt {{\bf pt}}
\def\finish {${\,\,\vrule height1mm depth2mm width 8pt}$}




We know that the standard way
 to calculate metric in stereographic projection
is annuyingly long exercise.

   Consider another way to do it.
  
    We use the following fact:

   If function $f$ vanihes at the given point, then
  derivative of function $fg$ at this point is equal to $f'(x_0)g(x_0)$.

This trivial statement very often facilitates calculations.


 Let $S^2$ be a sphere of radius $x^2+y^2+z^2=1$ with 
stereographic coordinates  $u,v$ 
                           $$
\cases{
  u={2x\over 1-z}\cr
  v={2y\over 1-z}\cr
    }\,,\qquad
     \cases
 {
   x={2u\over 1+u^2+v^2}\cr
   y={2v\over 1+u^2+v^2}\cr
   z={u^2+v^2-1\over 1+u^2+v^2}\cr
  }
          $$
  To calculate metric in stereographic coordinates we have to calculate:
                  $$
   \left(dx^2+dy^2+dz^2\right)_{x=x(u,v),y=y(u,v),z=z(u,v)}
                  $$
If you did these calculations you remember that
straightforward caclulations are not difficult,
but annuying, moreover the final answer appears something as
as a ``rabbit of the hat''

    We consider here much more illuminating calculations.

   First of all calculate (1c) at the point $u=v=0$
   ($x=y=0,z=-1$).
   We see that due to the vanishing of 
   functions $u$ and $v$ at this point we have:
                  $$
   \left(dx^2+dy^2+dz^2\right)_{x=0, y=0,z=-1}=
                  $$
                  $$
d\left({2u\over 1+u^2+v^2}\right)^2_{u=v=0}+
d\left({2v\over 1+u^2+v^2}\right)^2_{u=v=0}+
d\left({u^2+v^2-1\over 1+u^2+v^2}\right)^2_{u=v=0}=4du^2+4dv^2\,.
                  $$

Now use the fact that metric on the sphere is invariant with respect to 
rotations.

   Metric of sphere in stereographic coordinates is invariant
with respect to rotations----i.e. rotations of sphere around
axis $OZ$. To calculate metric at arbitrary point of the plae $(u,v)$
we use the invariance of the metric with respect to rotations around
other axis.
 Let $\pt--(u_0,0)$ be an arbitrary point of the line $(u,0)$.
Consider the rotation of the sphere around axis $0Y$:
                   $(x,y,z)\to (\tilde x,y,\tilde z)$ 
which transforms the point $\pt$ to the origin
(respectively the point $(x_0,0,z_0)$) corresponding to the point $\pt$
to the point $(\tilde x,\tilde y,\tilde z)=(0,0,-1)$.
                    $$
   \tilde u={\tilde x\over 1-\tilde z}=
 {x\cos\varphi+z\sin \varphi_0\over 1-(-x\sin\varphi_0+z\cos\varphi_0)}=
 {{2u\over 1+u^2}\cos\varphi_)+{u^2r-1\over u^2+1}\sin \varphi_0
         \over 
   1-\left(-{2u\over 1+u^2}\sin\varphi_0+{u^2-1\over u^2+1}
        \cos\varphi_0\right)}\,,
                    $$
                    $$
   \tilde v={ y\over 1-\tilde z}=
 {y\over 1-(-x\sin\varphi_0+z\cos\varphi_0)}=
 {{2v\over 1+u^2}
         \over 
   1-\left(-{2u\over 1+u^2}\sin\varphi_0+{u^2-1\over u^2+1}
\cos\varphi_0\right)}\,,
                    $$
where angle $\varphi_0$ is chosen such that 
$\tilde x=0, \tilde z=-1$:
              $$
   {\rm ctg\,}\varphi_0={1-u_0^2\over 2u_0}
              $$
Now one can easy  calculate  differentials $d\tilde u$ and $d\tilde v$
at the origin---i.e. differentials $du,dv$ at the point $\pt$:
Since for the point $\pt$ $x=0, u=0$, and $\tilde z=-1$, $\tilde x=0$
we can very quickly calculate $d\tilde v$ at give point:          
             $$
d\tilde v\big\vert_{(u,v)=(u_0,0)}=
      d\left(
           {{2v\over 1+u^2}
         \over 
   1-\left(
   -{2u\over 1+u^2}\sin\varphi+{u^2-1\over u^2+1}\cos\varphi
\right)}
\right)=
          {d\left({2v\over 1+u^2}\right)\over 2}\big\vert_{v=0}=
              ={d v\over 1+u^2}\,,
                     $$
and little bit more  difficult (since we need to calculate 
   $\sin \varphi$)
  but still quick we can caluclate the differential 
   $d\tilde u$ (at the given point):
   $$ 
d\tilde u\big\vert_{(u,v)=(u_0,v)}=d\left( 
 {{2u\over 1+u^2}\cos\varphi_0+{u^2-1\over u^2+1}\sin \varphi_0
         \over 
   1-\left(-{2u\over 1+u^2}\sin\varphi_0+{u^2-1\over u^2+1}\cos\varphi_0
\right)}
\right)=
                       $$
                       $$
                          {
          d\overbrace{\left(
    2u\cos\varphi_0+(u^2+1)\sin \varphi_0
\right)}^{\hbox{vanishes at $u=u_0$}}
                  \over 2(1+u^2)
                  }=
                     {
          d\left(\overbrace{\left(
    2u\left({1-u_0^2\over 2u_0}\right)+(u^2-1)
\right)}^{\hbox{vanishes at $u=u_0$}}\sin\varphi_0\right)
                      \over 2(1+u^2)
                    }=
                   $$
                   $$
                   {
          d\left(\overbrace{\left(
    2u\left({1-u_0^2\over 2u_0}\right)+(u^2+1)
\right)}^{\hbox{vanishes at $u=u_0$}}\sin\varphi_0\right)
              \over 2(1+u^2)
                 }=
                   $$
              $$
       {\sin\varphi_0du\over (1+u^2)^2}\left({1-u^2\over 2u}+u\right)
               \big\vert_{u=u_0}={du\over 1+u^2}\big\vert_{u=u_0}
              $$

  The result of calculation: we see that at the origin
     metric is equal to $4du^2+4dv^2$. It follows from the previous calculations
that at the point $\pt\colon\,\, (u,v)=(u_0,v)$
               $$
    G=4(d\tilde u)^2+4d{\tilde v)^2=4du^2+4dv^2\over (1+u^2)^2}\,.
               $$Metric is invariant with rotation (corresponding to rotation
along axis $OZ$). Hence we come to the answer:
                   $$
          G={4du^2+4dv^2\over (1+u^2+v^2)^2)}\,.
                   $$
This answer can be easily generalised for $n$-dimensional sphere:
       

\bye
