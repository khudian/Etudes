\magnification=1200 

\baselineskip=14pt
\def\vare {\varepsilon}
\def\A {{\bf A}}
\def\a {\alpha}
\def\N {{\bf N}}
\def\s {{\sigma}}
\def\bs {{\bf s}}
\def\S {{\Sigma}}
\def\s {{\sigma}}
\def\d {{\delta}}
\def\p{\partial}
\def\vare{{\varepsilon}}
\def\Q {{\bf Q}}
\def\D {{\cal D}}
\def\g {{\Gamma}}
\def\C {{\bf C}}
\def\M {{\cal M}}
\def\Z {{\bf Z}}
\def\U  {{\cal U}}
\def\H {{\cal H}}
\def\R  {{\bf R}}
\def\S  {{\bf S}}
\def\E  {{\bf E}}
\def\l {\lambda}
\def\degree {{\bf {\rm degree}\,\,}}
\def \finish {${\,\,\vrule height1mm depth2mm width 8pt}$}
\def \m {\medskip}
\def\p {\partial}
\def\g {{\Gamma}}
\def\v {{\bf v}}
\def\n {{\bf n}}
\def\t {{\tilde}}
\def\b {{\bf b}}
\def\c {{\bf c }}
\def\e{{\bf e}}
\def\ac {{\bf a}}
\def \X   {{\bf X}}
\def \Y   {{\bf Y}}
\def \x   {{\bf x}}
\def \y   {{\bf y}}
\def \G{{\cal G}}
\def\w{\omega}
\def\pt {{\bf pt}}
\def\finish {${\,\,\vrule height1mm depth2mm width 8pt}$}

  \centerline  {\bf Second attempt}

  It is well-known that on Riemannian manfiold, the Laplacian  
        $$
\Delta_{g}=\Delta^{(B)}_{g}+c_nR
  \eqno (1)
        $$
\def\t {\tilde}
\def\hDelta {\hat\Delta}
 is invariant with respect to conformal transformations
     $$
   \t g_{ik}=e^{\s}g_{ik}
       \eqno (2)
     $$
Here  $\Delta^{(B)}_g$ is standard Beltrami-Laplace:
               $$
\Delta^{(B)}_g={1\over \rho_g}{\p \over \p x^i}
                 \left(\rho_gg^{ik}{\p \over \p x^k}\right)\,,\quad
\rho_q={\det g}^{1\over 2}\,.
                $$
is a scalar curvature of metric $g$,
$c_n$ is conast  depending on the dimension of the space  $M$.

What is the exact statement?

 The standard answer is: 
              $$
\Delta_{\t g}=e^{a_n\s}\Delta_ge^{b_n\s}\,, \quad
{\rm if}\,\,\g_{ik}=e^\s g_{ik}
                  $$
where $a_n,b_n$ are constants. They measure so called conformal weight.

   Sure this is much better to writhe down invartiant operator,
without mentioning conformal weight. 
 I prefer to tell this in the following way:
    Consider on Riemannian manifold, a tensorial density:

                   $$
         S_g^{ik}\p_i\otimes \p_k=
          \rho_g^{1\over n}g^{ik}\p_i\otimes \p_k=
             \left(|Dx|\sqrt{\det g}\right)^{1\over n}g^{ik}\p_i\otimes\p_k
                   $$
This tensorial density depends on {\it conformal class}
of Riemannian metric.  In particular it {\it does not change}
under Weyl transformations (2):
                            $$   
             \left(|Dx|\sqrt{\det \t g}\right)^{1\over n}\t 
                 g^{ik}\p_i\otimes\p_k=
             \left(e^{n\s}|Dx|\sqrt{\det g}\right)^{1\over n}
               e^{-\s}g^{ik}\p_i\otimes\p_k=
             \left(|Dx|\sqrt{\det g}\right)^{1\over n}g^{ik}\p_i\otimes\p_k\,.
                                  $$
Consider pencil of self-adjoint operators in the space of densities
  corresponding to this tensorial density:
                          $$
\hatDelta=
                          $$


\bye 
