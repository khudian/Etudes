\magnification=1200 %\baselineskip=14pt
\def\vare {\varepsilon}
\def\A {{\bf A}}
\def\t {\tilde}
\def\a {\alpha}
\def\K {{\bf K}}
\def\N {{\bf N}}
\def\V {{\cal V}}
\def\s {{\sigma}}
\def\S {{\Sigma}}
\def\s {{\sigma}}
\def\p{\partial}
\def\vare{{\varepsilon}}
\def\Q {{\bf Q}}
\def\D {{\cal D}}
\def\G {{\Gamma}}
\def\C {{\bf C}}
\def\M {{\cal M}}
\def\Z {{\bf Z}}
\def\U  {{\cal U}}
\def\H {{\cal H}}
\def\R  {{\bf R}}
\def\S  {{\bf S}}
\def\E  {{\bf E}}
\def\l {\lambda}
\def\degree {{\bf {\rm degree}\,\,}}
\def \finish {${\,\,\vrule height1mm depth2mm width 8pt}$}
\def \m {\medskip}
\def\p {\partial}
\def\r {{\bf r}}
\def\v {{\bf v}}
\def\n {{\bf n}}
\def\t {{\bf t}}
\def\b {{\bf b}}
\def\c {{\bf c }}
\def\e{{\bf e}}
\def\ac {{\bf a}}
\def \X   {{\bf X}}
\def \Y   {{\bf Y}}
\def \x   {{\bf x}}
\def \y   {{\bf y}}
\def \G{{\cal G}}

\centerline  {\bf  Stereographic projection and around}

\bigskip
            \centerline {$\cal x$ Twisted stereographic projection}

  Let $S^1$ be a circle $x^2+y^2=1$ and $l$ be $x$-axis, a line $y=0$ in $\E^2$.

   Let $\ac=(a_x,a_y)$ be an arbitrary point on $\S^1$ ($a_x^2+a_y^2=0$), such that $a_x\not=0$.

   For an arbitrary $t$, consider the line $[(t,0),(a,b)]$,
   i.e. the line which passes through the points  $(t,0)$ on the $x$-axis and the points $\ac=(a_x,a_y)$.
   This line intersects the circle in general in two points--the point $\ac=(a_x,a_y)$ and the point
           $S_\ac(t)$ with coordinates $(x_\ac(t),y_\ac(t))$, where
                    $$
  S_\ac(t)\colon\quad           x_\ac(t)={2t-a_xt^2-a\over 1-2a_xt+t^2},\qquad  y_\ac(t)={a_y(t^2-1)\over 1-2a_xt+t^2}
                             \eqno (1)
                     $$

To derive this formula is a good exercise for students: The coordinates of intersection points $\S-\ac(t)$
and $\ac$ of circle with line are roots of quadratic equation. Hence applying Vieta Theorem we come to (1).


 In the case if $\ac=(0,1)$ or $\ac=(0,-1)$ the map (1) is nothing but stereographic projection:
              $$
              x_0(t)={2t\over 1+t^2},\qquad  y_0(t)={t^2-1\over 1+t^2},\quad \hbox{(for the
              north pole $\ac=(0,1)$)}
                             \eqno (1a)
               $$

 Of course it can be considered as "twisted" stereographic projection in the general case (1).

   Curious student notes that the map (1a), as well as (1)
 is a bijection of projective line $\R P^1$ on the circle $S^1$. In the general case (1)
 the image of the $\infty$ point is the point $(-a_x,a_y)$ on the circle.

  The more curious student will note that the map (1) establishes one-one correspondence between
  rational points on the line\footnote{$^1$}{we consider $\infty$ as rational point too} and points on the
  circle $S^1$ with rational coordinates in the case if coordinates of point $\ac$ are rational:
   $a_x,a_y\in \Q$. The inverse map to $(1)$ is:
                          $$
               t=\S_{\ac'}^{-1}(x,y)={a_xy-a_yx\over y-a_y}
                     \eqno (2)
                          $$
   One can rewrite this map in the following way:
                     $$
                      t=\S_{\ac'}^{-1}(x,y)={1+a_xx+a_yy\over x+a_x}
                            \eqno (2a)
                     $$
   {\it What is the difference between these two representations?  Good question}

  The map (2a) has only essential singularity--at the point $(-a_x,a_y)$. The map (2) has non-essential
  singularity too at the point $(a_x,a_y)$ where line $[(t,0), (x,y)]$ becomes tangent to the circle.
  (See in details the next section)


 Of course the fact that the maps (1),(2) establish one-one correspondence between points of line and circle
  is not just the consequence of occaisonal beauty of formulae (1,2). The reason of this phenomenon and of the fact
  that maps (1), (2) are rational is that quadratic polynomial with rational coefficients has both rational roots
  if one root is rational.



Now using the map (1) and (2) consider the map:
                $$
          \S_{\ac}^{-1}\S_{\ac'}\colon\quad \R P^1\to \R P^1
                $$
for two different $\ac',\ac$.

E.g. consider $\ac'=(0,1)$ (usual stereographic projection). Then one can see that
            $$
            u(t)=
   \S_{\ac}^{-1}\S_{\ac'}(t)=\S_{\ac}^{-1}
          \left(
          x_0(t),y_0(t)
          \right)=
          \S_{\ac}^{-1}
          \left(
          {2t\over 1+t^2},  {t^2-1\over 1+t^2}
          \right)\,.
          \eqno (3)
            $$
    ($\S_{\ac'}=S_0$ is given by equation (1a))
 Using (2) we come to
           $$
   u(t)=
   \S_{\ac}^{-1}\S_{\ac'}(t)=\S_{\ac}^{-1}\S_{0}(t)={a_xt^2-2a_yt-a_x\over (1-a_y)t^2-(1+a_y)}
   \eqno (4)
           $$
In the case if $a_x,a_y$ are rational ($a_x^2+a_y^2=1$) the map (3a) establishes one-one correspondence between
rational points of $\R P^1$.

{\it Stop! How comes. Something is wrong: here is ratio of quadratic polynomials. }.

We see that these polynomials must have common divisor. Apply geometry to find it.

  Consider the point of intersection of the line $(\ac,\ac')$ and $x$-axis. Denote this point
  as $\bf\Delta$.  We have that  $\S_\ac({\bf\Delta})=\ac'$ and vice versa  $\S_\ac'({\bf\Delta})=\ac$

  What about the value of the map $\S_{\ac'}^{-1}$ at the point $\S_\ac({\bf\Delta})=\ac'$, respectively

  what about the value of the map $\S_{\ac}^{-1}$ at the point $\S_\ac({\bf\Delta})=\ac$


{\it Answer: Numerator and denominator of the fraction (4) have common divisor $t-\bf\Delta$, where
  $\ac=\S({\bf\Delta})$}
$$|$$
  \centerline{Avoiding not-essential singularities}


  Consider a curve $P(x,y)=0$ and a point $(a,b)$ on it. Consider the function
                  $$
                F(x,y)={y-b\over x-a}
                  $$

   The point $(a,b)$ is not-essential singularity for the function $F$,
   if $P_y(x,y)\not=0$ at the point $(a,b)$ ($F$ is equal to the derivative of the implicit function
   $y(x)\colon\, P(x,y(x))=0$ at the point $x=a$).

   Due to the Theorem one can find a fraction (polynomial) such that denominator is not zero at $a$
   (More precisely denominators vanish only at essential singularities. ) E.g. if $P(x,y)=y-x^2$
   and $b=a^2$ then
                  $$
            F(x,y)={y-b\over x-a}={(y-b)(x+a)\over (x-a)(x+a)}={(y-b)(x+a)\over x^2-a^2}=x+a
                  $$
This is in accordance with Theorem that the function is defined on the whole algebraic curve,
hence it is equivalent to the polynomial.


Another example:  $F(x,y)=x^2+y^2-1$, $a=0,b=1$. Then
                    $$
            F(x,y)={y-a\over x-b}={y-1\over x}={(y-1)x\over  x^2}={(y-1)x\over 1-y^2}={-x\over 1+y}
                    $$
This function has only essential singularity at the point $(0,-1)$

\bye
