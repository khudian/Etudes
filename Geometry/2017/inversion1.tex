% I begin this version on 29 january 2017
\magnification=1200 %\baselineskip=14pt
\def\vare {\varepsilon}
\def\A {{\bf A}}
\def\t {\tilde}
\def\a {\alpha}
\def\K {{\bf K}}
\def\N {{\bf N}}
\def\V {{\cal V}}
\def\s {{\sigma}}
\def\S {{\Sigma}}
\def\s {{\sigma}}
\def\p{\partial}
\def\vare{{\varepsilon}}
\def\Q {{\bf Q}}
\def\D {{\cal D}}
\def\G {{\Gamma}}
\def\C {{\bf C}}
\def\M {{\cal M}}
\def\Z {{\bf Z}}
\def\U  {{\cal U}}
\def\H {{\cal H}}
\def\R  {{\bf R}}
\def\E  {{\bf E}}
\def\l {\lambda}
\def\degree {{\bf {\rm degree}\,\,}}
\def \finish {${\,\,\vrule height1mm depth2mm width 8pt}$}
\def \m {\medskip}
\def\p {\partial}
\def\r {{\bf r}}
\def\v {{\bf v}}
\def\n {{\bf n}}
\def\t {{\bf t}}
\def\b {{\bf b}}
\def\c {{\bf c }}
\def\e{{\bf e}}
\def\ac {{\bf a}}
\def \X   {{\bf X}}
\def \Y   {{\bf Y}}
\def \x   {{\bf x}}
\def \y   {{\bf y}}
{29 January 2017}

\centerline  {\bf  Inversion and stereographic projection}



\noindent {\it Vek zhivi, vek uchisj,}

\noindent { \it durakom pomrjoshj}   

\medskip

      You know how I like stereoraphic projection, it has so 
much beautiful properties, in particular 
the property of transforming of rational points
to rational ones. Today I learned that stereographic projection 
is just the restriction of inversion!

   In a more details:

  Let $\varphi\colon S^n\backslash N\leftrightarrow \R^{n}$
be a stereographic projection of unit sphere 
without North pole $N=(0,.\dots,0,1)$,
              $$
     S^n\backslash N=\left\{x^i\colon\quad
     \sum_{i=0}^n x^ix^i=1, x^0\not=1\right\}
             $$
 on the hyperplane 
       $$
   \R^n=\left\{x^i=u^i, i=1,\dots,n;\,\, x^0=0   \right\}\,.
       $$ 
  A point $\varphi(A)\in \R^n$ is stereopgraphic image of the point
$A\in S^n$ if three points $N,A,\varphi(A)$ are incident 
(belong to the same line).

   This is remarkable map, in particular it has the property:
               $$
\hbox{point $\varphi(A)$ has rational coordinates if and only if
the point $A$ has rational coordinates}
          \eqno (\dagger)
               $$
\medskip



Now consider the inversion  $\hat\varphi$ 
of $\R^{n+1}$ with respect to the sphere
$\tilde S$, such that the centre of this sphere is
the north pole $N$ and 
     equatorial circle  $S^{n-1}_{\rm equator}$
   (points $x^0=0$ on $S^n$) belongs
to this sphere:
    One can see that the sphere $\tilde S$  is defined 
by the equation
             $$
   (x^0-1)^2+\sum_{i=1}^n x^i x^i=2\,.
\eqno (*)
        $$
The map $\hat \varphi$ has the following appearance:
            $$
\R^{n+1}\backslash \n\ni \x\,\,\,  
    \hat\varphi(\x)=\n+{2\x-2\n\over |\x-\n|^2}\,,     
           $$
where $\n$ is a vector attaching origin to the North pole.
The image of $S^n$ under inversion $\hat\varphi$
is hyperplane, since
$\hat\varphi$ is not define at vector $\n$.
It is easy to check also that equatorial circle remains invariant.
Using properties of inversion\footnote{$^{1)}$}
{inversion maps circles and planes to circles
and planes}  we come to the 

  {\bf Fact}{\it
             $$
\hat\varphi_{_{S^n\backslash N}}=\varphi\,,
             $$
i.e. the restriction of  inversion  with respect 
to the sphere (*) is stereographic
projection $\varphi$}.


Notice that for inversion $\hat\varphi$ condition ($\dagger$) holds.

\bigskip



  {\sl I always was little bit unhappy  that
  there is always a problem when we consider
stereographic map  $\varphi$ in Cartesian coordinates of embedding space.
 The embedding of $\varphi$
in inversion, $\hat \varphi$ makes us little bit happier.

}
\bye
