\def\t {\tilde}
\def\a {\alpha}
\def\K {{\bf K}}
\def\A {{\bf A}}
\def\B {{\bf B}}
\def\N {{\bf N}}
\def\V {{\cal V}}
\def\s {{\sigma}}
\def\S {{\bf S}}
\def\s {{\sigma}}
\def\p{\partial}
\def\Q {{\bf Q}}
\def\D {{\cal D}}
\def\G {{\Gamma}}
\def\C {{\bf C}}
\def\H {{\bf H}}
\def\M {{\cal M}}
\def\Z {{\bf Z}}
\def\U  {{\cal U}}
\def\H {{\bf H}}
\def\R  {{\bf R}}
\def\E  {{\bf E}}
\def\l {\lambda}
\def\hl {{\hat \lambda}}
\def\degree {{\bf {\rm degree}\,\,}}
\def \finish {${\,\,\vrule height1mm depth2mm width 8pt}$}
\def \m {\medskip}
\def\p {\partial}
\def\grad {{\rm grad\,}}
\def\div {{\rm div\,}}
\def\r {{\bf r}}
\def\v {{\bf v}}
\def\n {{\bf n}}
\def\t {{\bf t}}
\def\b {{\bf b}}
\def\e{{\bf e}}
\def\f{{\bf f}}
\def\ac {{\bf a}}
\def \X   {{\bf X}}
\def \Y   {{\bf Y}}
\def\diag {\rm diag\,\,}
\def\pt {{\bf p}}
\def\w {\omega}
\def\la{\langle}
\def\ra{\rangle}
\def\x{{\bf x}}

\def\s{\sigma}
\def\vare{\varepsilon}
\def\O{\Omega}
\def\G {\Gamma}
\def\GG {{\cal G}}
\def\A  {{\bf A}}
\def\o{\omega}
\def\F {{\cal F}}
\def\L {{\cal L}}
\def\D {{\cal D}}
\def\S {{\mathbf S}}
\renewcommand{\SS}{{\Sigma}}
%\renewcommand{\r}{{\rho}}
\renewcommand{\l}{{\lambda}}
%\newcommand{\x}{{\xi}}
\def\d{\delta}
\def\t{\theta}
\def\pr {\prime}
\def\hl{{\widehat\lambda}}
\def\ad {{\rm ad}}

\def\diff {{\rm diff}}
\def\proj {{\rm proj}}
\def\affin {{\rm proj}}
\def\vertL {{\hat L^{\rm vert}}}
\def\C {{\widehat C}}

\newcommand{\bs}{{\boldsymbol{s}}}

\newcommand{\rh}{{\boldsymbol{\rho}}}
\newcommand{\Ps}{{\boldsymbol{\Psi}}}

\newcommand{\der}[2]{{\frac{\partial {#1}}{\partial {#2}}}}
\newcommand{\lder}[2]{{\partial {#1}/\partial {#2}}}
\newcommand{\dder}[3]{{\frac{\partial^2 {#1}}{\partial {#2}\partial {#3}}}}
%\newcommand{\R}[1]{{\mathbb R}^{#1}}
\newcommand{\RR}{\mathbb R}
%\newcommand{\Z}{{\mathbb Z_{2}}}
\newcommand{\ZZ}{{\mathbb Z}}

\newcommand{\CC}{\mathbb C}
\newcommand{\NN}{{\mathbb N}}



\documentclass[12pt]{article}
\usepackage{amsmath,amsthm}


\usepackage{amsmath,amssymb,amsfonts,amsthm}


\theoremstyle{theorem}
\newtheorem{thm}{Khimera}

\numberwithin{equation}{section}


\title{Geometry of Fadeev-Popov trick and BV formalism}
\date{}



\begin{document}
\maketitle

 \centerline {H.M. Khudaverdian}

\centerline {\it The University of Manchester. School of Mathematics.}

\bigskip

  \centerline {Manhester  February 2018---2 }

\bigskip

\begin{abstract}
{
 I try to return to considerations in
papers \cite{KhNers}, \cite{Khcmp2}... 

  
}

\end{abstract}
\bigskip
   
\tableofcontents


\section {\bf Lecture I.  Elements of Fadeev -Popov trick}


\subsection {Very elementary example}


   Let $\rho(\r)$ be a function on $\E^2$ such that

\begin{itemize}

\item it is invariant with respect to rotation:
             $$
                \rho(\r)=f(r)\,.
                 $$

\item it increases rapidly:
                  $$
\int\rho(x,y)dx\wedge dy=2\pi\int_0^\infty f(t)tdt<\infty\,.
                  $$
\end{itemize}


   Let $C$ be a curve beginning at origin, and
going to infinity such that  for arbitrary positive number
  $R$, every circle $x^2+y^2=R^2$ intersects the curve $C$
exactly at one point.

    Suppose the curve $C$ is defined by equation $F(x,y)=0$.
We say that function $F$ is a gauge function.

             Define function $B(x,y)$ such that
                     \begin{equation*}
         B(x,y)
         \int_0^{2\pi}
  \delta \left(\left(F(x^\varphi,y^\varphi)\right)\right)d\varphi=1\,.
                     \end{equation*}
 where
               $$
 \begin{pmatrix}
         x^\varphi\cr
         y^\varphi\cr
 \end{pmatrix}
         =
 \begin{pmatrix}
         \cos\varphi &-\sin\varphi\cr
         \cos\varphi &\cos\varphi\cr
 \end{pmatrix}
 \begin{pmatrix}
         x\cr
         y\cr
 \end{pmatrix}
       =
 \begin{pmatrix}
     x\cos\varphi-y\sin\varphi\cr
     x\sin\varphi+y\sin\varphi\cr
 \end{pmatrix}
               $$
We see that function $B(x,y)$ is invariant with respect to rotation:
          \begin{equation*}
          B(x^\varphi,y^\varphi)=B(x,y)\,,\quad {\rm i.e.}\,\,
      B(x,y)=B(r)
          \end{equation*}
Thus we have:
              $$
\int_{\bf E^2} \rho(x,y)dx\wedge dy=
\int_{\bf E^2} 1\cdot  \rho(x,y)dx dy=
\int_{-\infty}^{\infty}
\int_{-\infty}^\infty 
B(x,y) \rho(x,y)dx dy=
                 $$
                 $$
         \int_0^{2\pi}
\int_{-\infty}^{\infty}
\int_{-\infty}^\infty d\varphi dx dy
       \rho(x,y)
             B(x,y) 
  \delta \left(\left(F(x^\varphi,y^\varphi)\right)\right)d\varphi=
              $$ $$
         \int_0^{2\pi}
\int_{-\infty}^{\infty}
\int_{-\infty}^\infty d\varphi dx^\varphi dy^\varphi
       \rho(x^\varphi,y^\varphi)
             B(x^\varphi,y^\varphi) 
  \delta \left(\left(F(x^\varphi,y^\varphi)\right)\right)=
              $$
              $$
         \int_0^{2\pi}
\int_{-\infty}^{\infty}
\int_{-\infty}^\infty d\varphi dx dy
       \rho(x,y)
             B(x,y) 
  \delta \left(\left(F(x,y)\right)\right)=
              $$
               $$
         2\pi
\int_{-\infty}^{\infty}
\int_{-\infty}^\infty  dx dy
       \rho(x,y)
             B(x,y) 
  \delta \left(\left(F(x,y)\right)\right)=
              $$


  here we have to recognise the density since
it depends on function.......

\end{document}
