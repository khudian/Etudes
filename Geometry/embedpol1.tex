
   \magnification=1200
   \baselineskip 17 pt

\def\V {{\cal V}}
\def\s {{\sigma}}
\def\Q {{\bf Q}}
\def\D {{\cal D}}
\def\G {{\Gamma}}
\def\C {{\bf C}}
\def\M {{\cal M}}
\def\Z {{\bf Z}}
\def\U  {{\cal U}}
\def\H {{\cal H}}
\def\R  {{\bf R}}
\def\l {\lambda}
\def\p {\partial}
\def\r {{\bf r}}
\def\v {{\bf v}}
\def\n {{\bf n}}
\def\t {{\bf t}}
\def\b {{\bf b}}
\def\ac {{\bf a}}
\def \X   {{\bf X}}
\def \Y   {{\bf Y}}
\def \E   {{\bf E}}
\def \N   {{\bf N}}

    \centerline {\bf On polynomials on hypersurfaces }

 \bigskip

   Let $M$ be $n$-hypersurface $C\subset \E^{n+1}$ in $\E^n$.
   One can assign  a (formal) polynomial to this surface in the following way:
                  $$
               P_M(z)=\int \det
               \left(
               \delta_{\beta}^\alpha+zg^{\alpha\gamma}
               {
               \left(
         \r_{\gamma\beta},\N\right)
               \over
             \sqrt{\det g}}
               \right)d^{n}\xi
                  $$
Here $x^i=x^i(\xi)$ is a parameterisation of surface, $x^i_\alpha
x^i\beta$ induced metric;
                         $$
      {\bf N}=*\underbrace{{\bf r}\wedge\dots\wedge {\bf r}} _{n-{\rm times}},
      \quad {{\bf N}\over \sqrt {\det g}}=\n\,\,
      \hbox {is unit normal vector to the surface}
                            $$

 Integral has meaning if $M$ is compact. In general case one have to multiply integrand
on function something like $e^{-r^2}$ which tends to zero at
infinity)

Expansion in power series give invariants.In particular if $n=2$
then the term proportional to $z$ is mean curvature, the next one
Gaussian.


  It seems to be interesting to consider this polynomial in dual approach:

Suppose a surfaces is given by the equation $\Phi=0$.

 Long ago I knew a part of an answer:
   The expression
                    $$
   T=\Delta\Phi-{\nabla_i\Phi\nabla_k\Phi\nabla_{ik}\Phi\over \nabla_r^2\Phi}
                    $$
is a dual density corresponding to mean curvature: if $\Phi\to f\Phi$ then $T\to fT$ on
$\Phi=0$.
                   $$
   \int \left(\Delta\Phi-{\nabla_i\Phi\nabla_k\Phi\nabla_{ik}\Phi\over \nabla_r^2\Phi}\right)
   \delta(\Phi)d^{n+1}x
                     $$
is integral of mean curvature.


It is just integral corresponding to term linear by $z$.

  The detailed  answer is following:
            $$
            \tilde P_M(z)=\int \det(\delta_{ik}+zM_{ik}(\Phi))\delta(\Phi)d^{n+1}x\,,
            $$
where
            $$
M_{ik}(\Phi)=\p_i\p_k\Phi+{\p_i\Phi\p_k\Phi\Delta\Phi\over
\p_r\Phi\p_r\Phi}
   -{{\p_i\Phi\p_m\Phi\p_m\p_k\Phi+\p_k\Phi\p_m\Phi\p_m\p_i\Phi
   \over \p_r\Phi\p_r\Phi}}
            $$

      {\bf Statement} Integral above does not depend on the function $\Phi$
      defining the surface.

      {\it Proof}. It follows from the identity:
                 $$
             M_{ik}\left(\lambda(x)\Phi(x)\right)=\lambda(x)
             M_{ik}(\Phi(x))+G(\lambda(x),\Phi(x))\Phi(x)
                    $$
                    which can be easily checked.
Note that this integral has power on $z$ $n+1$. E.g. the term proportional to $z^{}n+1$
is a product of mean and Gaussian curvatures.

The detailed analysis lead to the following answer:
Consider the linear operator corresponding to the matrix $M_{ik}$
$M_{ik}$ is bilinear form,
$G^{ik}M_{kn}$ is linear operator where $G$ is Riemannian
tensor in embedded $\E$.
{\it This linear operator is a direct sum of two
operators: Weingerten shape operator and the one-dimensional operator which multiplies
normal vector on mean curvature!!!!
                 $$
           \det (1+zM)=(1+zH)\cdot P_M(z)
                 $$


 }



 \bye
