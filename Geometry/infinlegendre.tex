 

\magnification=1200
\baselineskip=14pt
\def\vare {\varepsilon}
\def\t {\tilde}
\def\a {\alpha}
\def\K {{\bf K}}
\def\N {{\bf N}}
\def\C {{\cal C}}
\def\L {{\cal L}}
\def\E {{\cal E}}
\def\s {{\sigma}}
\def\S {{\Sigma}}
\def\p{\partial}
\def\vare{{\varepsilon}}
\def\Q {{\bf Q}}
\def\D {{\cal D}}
\def\G {{\Gamma}}
\def\Z {{\bf Z}}
\def\R  {{\bf R}}
\def\l {\lambda}
\def\ll {{\bf l}}
\def\degree {{\bf {\rm degree}\,\,}}
\def \finish {${\,\,\vrule height1mm depth2mm width 8pt}$}
\def \m {\medskip}
\def\p {\partial}
\def\r {{\bf r}}
\def\pt {{\bf p}}
\def\v {{\bf v}}
\def\n {{\bf n}}
\def\t {{\bf t}}
\def\b {{\bf b}}
\def\c {{\bf c }}
\def\e{{\bf e}}
\def\f{{\bf f}}
\def\ac {{\bf a}}
\def \X   {{\bf X}}
\def \Y   {{\bf Y}}
\def \x   {{\bf x}}
\def \y   {{\bf y}}
\def\w {{\omega}}
\def \Tr  {{\rm Tr\,}}
\def\dim {{\rm dim\,\,}}
% I wrote this file on 16 January
\def\t {{\tilde}} 
\def\dist {{\hbox{\tt "distance"}}}


\centerline {\bf Symmetries of Clairaut equation}

\centerline {\bf or}
\centerline {\bf how looks infinitesimal Legendre transformations}

\def\t {\tilde}

{\it Recall that ordinary first order 
differential equation $f(x,y',y)=0$ has 
the following geometrical interpretation:  
The difequation defines surface $M_f$ of codimension 
$1$ in the contact space $J^1(\R)$ with coordinates
$(x,p,y)$ ($p=y'$),
      $$
M_f=\{(x,p,y)\colon f(x,p,y)=0\}
  \eqno (0.1)
      $$
 and the (generalised) solution of the 
equation is a curve $l=(x(t),p(t),y(t))$ belonging to this
surface such that the contact form
       $$
\w=pdx-dy
\eqno (0.2)
        $$
vanishes on the curve
\footnote{$^{1)}$}{The contact form (0.1) defines
the distribution (not-integrable) of the planes 
which vanish the form.  
The solution $l$ is the integral of this distribution.} 

The Legendre transformation
    $$
\t x=p,\t p=-x, \t y=y-px\,,
   \eqno (0.3)
    $$
of contact space $J^1(\R)$
preserves the contact form.


We study here wow lookz infinitesimal transformation 
which generates
the Legendre transformation. This maybe applied to
Clairaut equation.
}

Our aim to find the  one-parametric famiily
  of contact trasnformations which include the 
Legendre transformation (0.3) 


Recall that infinitesimal transformations---vector
fields on $J^1(\R)$ which preserves the 
distribution of the planes which vanish the contact form (0.2)
are in one-one correspondence with Hamiltonians on $J^1(\R)$:
           $$
C(J^1(\R))\ni H(x,p,y)\leftrightarrow  \X\colon\quad
  {\cal L}_\X (pdx-dy)=\lambda(x,p,y)(pdx-dy)\,.
        \eqno (1.1)
           $$
The correspondence is the following:
        $$
H(x,p,y)\mapsto \X_H= 
   {\p H\over \p p}{\p\over \p x}-
   {\p H\over \p x}{\p\over \p p}+
   \left(p{\p H\over \p p}-H\right){\p\over \p y}\,,\quad
\X\mapsto H=\w {\cal c}\X\,,\left(\lambda=-H_y\right)\,.
 \eqno (1.2)
            $$
Consider on $J^1(\R)$ 
the Hamiltonian $H={p^2\over 2}+{x^2\over 2}$
of harmonic oscillator.  This Hamiltonian 
defines due to (1.2) on $J^1(\R)$ the contact vector field
        $$
   \X_H=p{\p\over \p x}-
   x{\p\over \p p}+
   \left(
 {p^2\over 2}-
 {x^2\over 2}
    \right){\p\over \p y}\,,
\quad
  \L_{\X_H}\w=0\,,
\eqno (1.3)
        $$
and it induces
the contact transformation
   $$
  \pmatrix {x\cr p\cr y\cr}\to
  \pmatrix {\t x_\tau\cr \t p_\tau\cr \t y_\tau\cr}\,,
  $$
such that 
        $$
\cases {
{d {\t x}\over d\tau}={\p H\over \p {\t p}}=\t p\cr
{d {\t p}\over d\tau}=-{\p H\over \p {\t x}}=-\t x\cr
{d {\t y}\over d\tau}= \left(
    {{\t p}^2\over 2}
    -{{\t x}^2\over 2}
          \right) 
        \cr
}\quad\hbox {with boundary conditions}
  \pmatrix {\t x_\tau\cr \t p_\tau\cr \t y_\tau\cr}
      \big\vert_{\tau=0}=
  \pmatrix { x\cr  p\cr  y\cr}\,.
        $$
Solving this equation we come to
     $$
 \cases
  {
  x_\tau=x\cos \tau+p\sin \tau\cr
  p_\tau=-x\sin \tau+p\cos \tau\cr
  y_\tau=y+
 {1\over 4}\left(p^2-x^2\right)\sin 2\tau
 +{1\over 2}px\left(\cos 2\tau-1\right)\cr
       }
       $$
We see that for $\tau=0$ this is the identity 
transformation,
and for $\tau={\pi\over 2}$ this the Legendre trasnformation
 $$
 \cases
  {
  x_{\pi\over 2}=p\cr
  p_{\pi\over 2}=-x\cr
  y_{\pi\over 2}=y-px\cr
       }
       $$


\centerline {\bf Application to Clairaut equation}

The Clairaut equation
       $$
 y-xy'=f(y')
\eqno (2.1)
       $$
has the one-parametric family of lines
       $$
     y=kx+f(k)\,,\quad k\in\R, -\infty<k<\infty\,,
       \eqno (2.2a)
       $$
and the special soluction, their envelope: $\varphi(x)$,
     $$
\varphi(x)=k(x)x+f(k(x))\colon\quad k(x)\,\,
  \hbox{ is such that}
    {\p\varphi\over \p k}=x+f'(k)=0\,.
      \eqno (2. 2b) 
     $$
E.g. the solutions of Clairaut equation $y-xy'={y'}^2$
are the lines $y=kx+k^2$, $k\in \R$ and their envelope, the 
function $\varphi(x)=-{x^2\over 2}$.
This is standard.


Apply Legendre transformation  (0.3) 
to Clairaut equation (2.1).
The Clairaut equation (2.1)
transforms to the algebraic equation 
        $$
   \t y=f(\t x)
        \eqno (2.3)
        $$,

The solutions (2.2a), the lines
will transform to (generalised) solutions,
vertical lines:

Every line (2.2a) $l_k=\pmatrix {x=t\cr p=k\cr y=kt+f(k)}$
is transformed
to vertical line
   $\cases {\t x=k\cr \t p= t\cr
          \t y=f(k)\cr}$, 
$-\infty<t<\infty$, and this vertical line is the generalised
solution of equation (2.3).

     The special envelop solution (2b), the curve
  $$
l_\varphi=\pmatrix {x=t\cr p=k(t)\cr y=k(t)t+f(k(t))}\,,
{\rm where}\,\,
   k(t)\colon t+f(k(t))=0
       $$
is transformed to curve
   $\cases {\t x= t \cr 
            \t p=  f'(t)\cr
             \t y=f(t)\cr}$,
which comes from the solution of algebraic equation.



\smallskip

{\bf Remark} it is improtant to note that Legendre transformation
is the quasiclassic of Fourier trasnformaation.  This is why this
etude may be relevant.
\bye


