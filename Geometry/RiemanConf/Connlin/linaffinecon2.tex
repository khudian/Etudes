\magnification=1200 %\baselineskip=14pt
\def\vare {\varepsilon}
\def\A {{\bf A}}
\def\t {\tilde}
\def\a {\alpha}
\def\K {{\bf K}}
\def\N {{\bf N}}
\def\V {{\cal V}}
\def\s {{\sigma}}
\def\S {{\Sigma}}
\def\s {{\sigma}}
\def\p{\partial}
\def\vare{{\varepsilon}}
\def\Q {{\bf Q}}
\def\D {{\cal D}}
\def\G {{\Gamma}}
\def\C {{\bf C}}
\def\M {{\cal M}}
\def\Z {{\bf Z}}
\def\U  {{\cal U}}
\def\H {{\cal H}}
\def\R  {{\bf R}}
\def\S  {{\bf S}}
\def\E  {{\bf E}}
\def\l {\lambda}
\def\degree {{\bf {\rm degree}\,\,}}
\def \finish {${\,\,\vrule height1mm depth2mm width 8pt}$}
\def \m {\medskip}
\def\p {\partial}
\def\r {{\bf r}}
\def\v {{\bf v}}
\def\n {{\bf n}}
\def\t {{\bf t}}
\def\b {{\bf b}}
\def\c {{\bf c }}
\def\e{{\bf e}}
\def\ac {{\bf a}}
\def \X   {{\bf X}}
\def \Y   {{\bf Y}}
\def \x   {{\bf x}}
\def \y   {{\bf y}}
\def \GG{{\cal G}}
\def \F {{\cal F}}
\def\s {\sigma}
\def\o {\omega}
\def \ggb {\Gamma_{_{\bullet}}}
\def \gb {\Gamma_{_{\bullet}}}
\def\pt{{\bf }}

\centerline{    {\bf Linear projective and... connections}}

\medskip


    {\sl   When I was young I liked so much general constructions.
    Now I realize that special ("singular") cases are often most important. More:
     When we try to construct an object that describes a "general stuff" it is very useful to fix everything
     that could be fixed \it {\bf canonically.}

 \m

     Here after explaining general nonsense we will consider one example of the application of this motto.
   We try to define affine connection then after "tiding it up" we will come to linear connection, which is
   expressed in terms of Chrsitoffel symbols.}


\bigskip
\centerline {${\cal x} 0$ \bf General definitions.}

    First of all general non-sense. Let $P\to M$ be a principal fibre bundle over $M$
    with structure group $G$. (We suppose that group $G$ has a right action.)   $G$-invariant connection is defined by the $G$-invariant distribution
    of $n$-planes (horizontal planes) which are transversal to fibres.

    $G$-invariant connection  defines global $1$-form $\Omega$. This $1$-form  by the following relations:

   $\bullet$  it vanishes on horizontal plane:
                        $$
                      \Omega (p, \X)=0
                      \eqno (0.1a)
                        $$
if  $\X$ is a horisontal vector.


  $\bullet$
    Let $\X\in \GG$ be an element of Lie algebra of the group $G$ and
     let $\hat\X(p)$ be a vertical vector which is tangent to the curve $\gamma(t)=\pt (1+tX+dots)$
    starting at the point $p$. Then
                     $$
               \Omega(p ,\hat\X)=\X\,.
                   \eqno (0.1b)
                    $$
In other words it means that the value of the $1$-form $\Omega$ on the fundamental vector field $\hat \X(p)$ corresponding to vector $X\in \GG$ equals to the vector $X$.

This $\GG$-valued $1$-form is ${\rm Ad\,}\G$-invariant since connection is $G$-invariant.
                       $$
                  \Omega(pg, R_g\X)=g\Omega(p,\X)g^{-1}\,.
                       $$
 We say that $p(t)$ is lifting of the curve  $x(t)$ if $\pi p(t)=x(t)$ and tangent vectors $\dot p(t)$
are horizontal liftings of the tangent vectors $x(t)$.

  How this looks in local coordinates:

  Let $p_\a(t)$ be a local section of the fibre bundle $P\to M$ over domain $U_\a$
  It defines trivialisiation, local coordinates
                       $$
  \varphi_\a\, U_\a\times G\to P\colon(x,g)\mapsto p_\a(x)g\,.
                        $$
  One can consider $\GG$ valued local $1$-form on the base $M$ corresponding to this coordinatisation,
  such that
                                             $$
  \omega_\a(x,\X)=\Omega(p_\a(x),d p_\a(x)\X)
                                             $$

\bye
