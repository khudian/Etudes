

\magnification=1200 


\baselineskip=14pt
\def\vare {\varepsilon}
\def\A {{\bf A}}
\def\t {\tilde}
\def\a {\alpha}
\def\K {{\bf K}}
\def\N {{\bf N}}
\def\V {{\cal V}}
\def\s {{\sigma}}
\def\S {{\Sigma}}
\def\s {{\sigma}}
\def\p{\partial}
\def\vare{{\varepsilon}}
\def\Q {{\bf Q}}
\def\D {{\cal D}}
\def\G {{\Gamma}}
\def\C {{\bf C}}
\def\M {{\cal M}}
\def\Z {{\bf Z}}
\def\U  {{\cal U}}
\def\H {{\cal H}}
\def\R  {{\bf R}}
\def\S  {{\bf S}}
\def\E  {{\bf E}}
\def\l {\lambda}
\def\degree {{\bf {\rm degree}\,\,}}
\def \finish {${\,\,\vrule height1mm depth2mm width 8pt}$}
\def \m {\medskip}
\def\p {\partial}
\def\r {{\bf r}}
\def\pt {{\bf pt}}
\def\v {{\bf v}}
\def\n {{\bf n}}
\def\t {{\bf t}}
\def\b {{\bf b}}
\def\c {{\bf c }}
\def\e{{\bf e}}
\def\ac {{\bf a}}
\def \X   {{\bf X}}
\def \Y   {{\bf Y}}
\def \x   {{\bf x}}
\def \y   {{\bf y}}
\def \G{{\cal G}}

 I began this file on 19 June 2017


\centerline {Hidden hyperbolicity}


  Let $C$ be a circle in the plane and $A$, a point. Consider the locus $K$ 
of the points $K$ such that the length of the tangent from $K$ to circle
$C$ is equal to the length of the segment $KA$:
                     $$
   M=\{K\colon\quad KA=\hbox {lenght of the tangent from $K$ to the circle.}\}
                     $$

Then $M$ is the line which is in the distance 
$\rho={1\over 2}\left(a+{1\over a}\right)$ from the centre of the circle $C$.
(We suppose that the radius of the circle is equal to $1$,
and the point $K$ is at the distance $a$ and)


This is elementary problem, and it possesses the hidden hyperbolicity.



 
   Consider point $A'$ the inversion of the point $A$
with respect to the circle $C$.
  The locus of these points is nothing butthe line $l$ which is on the equal
 distance between these points. 

 {\it Meaning in hyperbolic geometry}

One of he points $A$ or $A'$ is in the circle $C$. 
Without loss of generality suppose that this is $A$.
Consider the Lobachevsky plane formed by uer half plane, with
  $l$ is absolute.  
  The circle $C$ will be the circle in the hyperbolic plnae also.
One can see that the point $A$ is the centre
of this hyperbolic circle. 


   Indeed  due to lemma all the geodesics interesect 
the circle $C$ under the right angle, i.e. the points of $l$
belong to this locus.

 
\bye

