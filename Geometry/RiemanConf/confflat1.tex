\magnification=1200 %\baselineskip=14pt
\def\vare {\varepsilon}
\def\A {{\bf A}}
\def\t {\tilde}
\def\a {\alpha}
\def\K {{\bf K}}
\def\N {{\bf N}}
\def\V {{\cal V}}
\def\s {{\sigma}}
\def\S {{\Sigma}}
\def\s {{\sigma}}
\def\p{\partial}
\def\vare{{\varepsilon}}
\def\Q {{\bf Q}}
\def\D {{\cal D}}
\def\G {{\Gamma}}
\def\C {{\bf C}}
\def\M {{\cal M}}
\def\Z {{\bf Z}}
\def\U  {{\cal U}}
\def\H {{\cal H}}
\def\R  {{\bf R}}
\def\S  {{\bf S}}
\def\E  {{\bf E}}
\def\l {\lambda}
\def\degree {{\bf {\rm degree}\,\,}}
\def \finish {${\,\,\vrule height1mm depth2mm width 8pt}$}
\def \m {\medskip}
\def\p {\partial}
\def\r {{\bf r}}
\def\v {{\bf v}}
\def\n {{\bf n}}
\def\t {{\bf t}}
\def\b {{\bf b}}
\def\c {{\bf c }}
\def\e{{\bf e}}
\def\ac {{\bf a}}
\def \X   {{\bf X}}
\def \Y   {{\bf Y}}
\def \x   {{\bf x}}
\def \y   {{\bf y}}
\def \G{{\cal G}}

           \centerline {\bf ${\cal x}1$ Conformally flat manifolds, and conformal class of metric}

\m

  We say that manifold $M$ is provided with conformal flat structure
  if it is provided by atlas $\{x^i_{(\alpha)}\}$ (which is compatible with smooth structure)
  such that transition functions $\Psi_{\a\beta}$ are conformal transformations.

   Conformal transformations are:

   i)  translation
                $$
                x^i_{(\beta)}=x^i_{(\a)}+a^i
                \eqno ({\rm Conf. trans. 1})
                 $$
   ii) Orthogonal transformations
            $$
          x^i_{(\beta)}=P^i_k x^i_{(\a)},\,\,\,\hbox {where $P^i_k$ is an orhogonal matrix}\,,
             \eqno ({\rm Conf. trans. 2})
            $$
 iii) Inversion
                $$
                x^i_{(\beta)}={x^i_{(\a)}\over \sum_i x^i_{(\a)}x^i_{(\a)}}
                \eqno ({\rm Conf. trans. 3})
                $$

   and any composition of these transformations.


   We say that Riemannian metric is conformally flat if in a vicinity of an arbitrary point there
   exist coordinates $x^i$ such that in these coordinates $g_{ik}=e^\sigma(x)\delta_{ik}$.

   Very important exercise:

   {\bf Proposition}

  Let metric be flat in coordinates $x^i$, i.e. it has an appearance $g_{ik}=\delta_{ik}$ in this coordinates.
    Then

   1) This metric has an an appearance $g_{ik}=e^\sigma(x)\delta_{i'k'}$ in new coordinates
   $x^{i'}$ if and only if changing of coordinates is conformal, i.e. it is a composition of transformations
   $ ({\rm Conf. trans. })$.

In other words transformations of local cartesian coordinates (Conf.trans.1---Conf.trans.3) and only these transformations
and their compositions preserve the angles.

   2)  The following statement {\it is not true}:

     Let metric has an appearance $g_{ik}=e^\sigma(x)\delta_{ik}$ in coordinates $x^i$ where
     $\sigma(x)$ is an arbitrary smooth function. Then there exist new coordinates
     such that in these new coordinates  metric is flat, i.e. it has an appearance $g_{ik}=e^\sigma(x)\delta_{ik}$


   Show counterexample.

 \medskip




  {\bf Theorem}   A manifold $M$ can be endowed with flat conformal structure if and only if
   there exist conformally flat metric on this manifold.


\medskip


   Sketch of the proof.

   Let atlas $\{x^i_{(\a)}\}$ provides $M$ with flat conformal structure, i.e. all transition functions are
   conformal (See equations (Conf.trans.)). Consider  locally defined Riemannian metrics
   $g_{ik}^{(\a)}=\delta_{ik}$ and $0$-cochain $\{\sigma_{(\a)}\}$ such that
                $$
             \sigma_{(\a)}-\sigma_{(\beta)}=t_{\a\beta}\,,
                $$
where $t_{\a\beta}$ is logarifm of Jacobian of conformal transformations from coordinates $x^i_{(\a)}$
to coordinates $x^i_{(\beta)}$. One can see that the right hand side defined closed $1$-cochain, hence
$0$-cochain $\sigma_{(\a)}$ is a coboundary. Using partition we see that
                 $$
               \sigma_{(\a)}=\dots ?? \sum_\gamma t_{\a\gamma}\varphi_\gamma
                 $$
and $g_{ik}=\e^{\sigma_(\a)}\delta_{ik}$ defines globally conformally flat Riemannian metric on $M$.

 Exercise:  Reconstruct exactly the right hand side of  the right hand side of the lasst expression.

\medskip


Now prove the converse implication

  Let $\{x^i_{\a}\}$ be an atlas such that in this atlas Riemannian metric has an appearance
              $$
              g_{ik}^{(\a)}=\e^{\sigma_{(\a)}}\delta_{ik}
              $$
Under change of coordinates $x^i_{(\beta)}=x^i_{(\beta)}(x^i_{(\alpha)})$
metric is changed on the scalar function, i.e. angles do not change. Thus this follows from
Proposition 1 that transition functions are conformal \finish




\centerline {\bf ${\cal x}2$ Conformally flat metrics , and Weil tensor}



   {\bf 1Exercise} Calculate variation of curvature tensor $R^i_{klm}$ under conformal variation of metrics.

{\bf 1Exercise}   Write down the tensor (linear combination of $R^i_{klm}$, $R^i_k$ and $R$ with coefficients
 formed by the tensor $g_{ik}$ such that this tensor is invariant with respect to conformal transformations.


  Thus we will write Weyl tensor:
                           $$
                C^i_{klm}=R^i_{klm}+\dots
                           $$


  Theorem: Weyl curvature vanishes if and only if the metric is conformally flat, i.e. flat conformal structure exists.



\bye
