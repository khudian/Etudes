\magnification=1200 


\baselineskip=14pt
\def\vare {\varepsilon}
\def\A {{\bf A}}
\def\t {\tilde}
\def\a {\alpha}
\def\K {{\bf K}}
\def\N {{\bf N}}
\def\V {{\cal V}}
\def\s {{\sigma}}
\def\S {{\Sigma}}
\def\s {{\sigma}}
\def\p{\partial}
\def\vare{{\varepsilon}}
\def\Q {{\bf Q}}
\def\D {{\cal D}}
\def\G {{\Gamma}}
\def\C {{\bf C}}
\def\M {{\cal M}}
\def\Z {{\bf Z}}
\def\U  {{\cal U}}
\def\H {{\cal H}}
\def\R  {{\bf R}}
\def\S  {{\bf S}}
\def\E  {{\bf E}}
\def\l {\lambda}
\def\degree {{\bf {\rm degree}\,\,}}
\def \finish {${\,\,\vrule height1mm depth2mm width 8pt}$}
\def \m {\medskip}
\def\p {\partial}
\def\r {{\bf r}}
\def\pt {{\bf pt}}
\def\v {{\bf v}}
\def\n {{\bf n}}
\def\t {{\bf t}}
\def\b {{\bf b}}
\def\c {{\bf c }}
\def\e{{\bf e}}
\def\ac {{\bf a}}
\def \X   {{\bf X}}
\def \Y   {{\bf Y}}
\def \x   {{\bf x}}
\def \y   {{\bf y}}
\def \G{{\cal G}}

 I began this file on 16 June 2017

   The following lemma is useful.


{\it     Let $C$ be a small circle in Riemannian manifold. Then
it is orthogonal to geodesics starting at the centre of this circle. 
 }


  Let $M$ be $n$-dimensional Riemannian manifold.
  Consider the spray of the geodesics $x^i=
x^i(t,\n)$, (%|\n|=1$) which are solutions of 
differential equation:
                      $$
x^i(t,\tau)\colon\quad
           \cases
             {
           x^i{tt}+\Gamma^i_{km}x^k_tx^m_t=0\cr
             x^i(t)\big\vert_{t=0}=0\cr
             x^i(t,\n)\big\vert{t=0}=
        \n\,.
               }
\hbox {$\n$ is unit vector }
                      $$
Thus we have the exponential map which assigns
to every unit vector $\n\in T_\pt M$, the geodesics.

    The natural question can be asked.
Consider the disc of the radius $R$, i.e. 
 the set of the points
  $x^i(t,\n)\colon t\in [0,R]$. Does the boundary
of thedisc  is perpendicular to the geodesics, i.e.
is it true that
                 $$
\langle x_t(t,\n), \delta x\rangle=0
                 $$

Let $\tau_\a$ ($\a=1,\dots,n-1$) be cooridnates
on unit sphere $\n=\n(\tau_\a)$.

Denote
              $$
   L=\langle \x_t,\x_\tau\rangle=g_{ij}x^i_t x^j_\tau\,.
               $$
It is evident that $L(t,\tau)\big\vert_{t=0}=0$
since this is scalar product of unit 
vector on the tangent vector to the unit sphere.
Prove that it vanishes for all $t$.

Let $E={1\over 2}|x^i_t|^2$. This is equal to constant. Hence
                 $$
0=\left({1\over 2}g_{mn}x^m_t x^n_t\right)_\tau=
 {1\over 2}x^p_\tau \p_p g_{mn}x^m_t x^n_t+g_{mn}x^m_{t\tau}x^n_t=
 {1\over 2}x^p_\tau \p_p g_{mn}x^m_t x^n_t+
\left(g_{mn}x^m_{\tau}x^n_t\right)_t=
                            $$
                             $$
= {1\over 2}x^p_\tau \p_p g_{mn}x^m_t x^n_t+
L_t-x^r_t\p_r g_{mn}x^m_{\tau}x^n_t-g_{mn}x^m_{\tau}x^n_{tt}=
                          $$
                           $$
 L_t-{1\over 2}\left(\p_mg_{np}+\p_n g_{mp}-\p_p g_{mn}\right)
          x^p_\tau x^m_t x^n_t-g_{mn}x^m_{\tau}x^n_{tt}=
L_t-g_{mn}x^m_{\tau}\left(x^n_{tt}+\Gamma^i_{mn}x^m_tx^n_t\right)=L_t=0\,.
                 $$

Invariant proof.

  Consider two vector fields $\v,\Y$ which commute
     Then 
   We have
                 $$
0=\p_\Y\left({1\over 2}\langle\v,\v\rangle\right)=
  \langle\nabla\Y\v],\v\rangle=\langle\nabla_\v\Y,\v\rangle=
\p_\v\langle\Y,\v\rangle-\langle\Y,\nabla_\v\v\rangle\,.
                 $$



\bye
