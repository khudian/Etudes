

\magnification=1200 


\baselineskip=14pt
\def\vare {\varepsilon}
\def\A {{\bf A}}
\def\t {\tilde}
\def\a {\alpha}
\def\K {{\bf K}}
\def\N {{\bf N}}
\def\V {{\cal V}}
\def\s {{\sigma}}
\def\S {{\Sigma}}
\def\s {{\sigma}}
\def\p{\partial}
\def\vare{{\varepsilon}}
\def\Q {{\bf Q}}
\def\D {{\cal D}}
\def\G {{\Gamma}}
\def\C {{\bf C}}
\def\M {{\cal M}}
\def\Z {{\bf Z}}
\def\U  {{\cal U}}
\def\H {{\cal H}}
\def\R  {{\bf R}}
\def\S  {{\bf S}}
\def\E  {{\bf E}}
\def\l {\lambda}
\def\degree {{\bf {\rm degree}\,\,}}
\def \finish {${\,\,\vrule height1mm depth2mm width 8pt}$}
\def \m {\medskip}
\def\p {\partial}
\def\r {{\bf r}}
\def\pt {{\bf pt}}
\def\v {{\bf v}}
\def\n {{\bf n}}
\def\t {{\bf t}}
\def\b {{\bf b}}
\def\c {{\bf c }}
\def\e{{\bf e}}
\def\ac {{\bf a}}
\def \X   {{\bf X}}
\def \Y   {{\bf Y}}
\def \x   {{\bf x}}
\def \y   {{\bf y}}
\def \G{{\cal G}}

% I began this file on 19 June 2017


\centerline {\bf Hidden hyperbolicity}

{\it In this \'etude we consider one geometrical problem,
which have two manifestations. It seems to be
the standard Euclidean problem,  but it
 possesses the hidden hyperbolicity.
We first explain how the example is arised, then consider
its solution in terms of Euclidean and Hyperbolic Geometry}

\m

\centerline {\it Source}

Consider realisation of hyperbolic (Lobachevsky) plane
as upper half-plane. Let $C$ be a circle in $H$,

  The circle in $H$ is the circle Euclidean
circle also. Let $A$ be a centre of hyperbolic circle 
and $O$ be a centre of $C$ considered 
as the Eucldean circle, and 
let $A$ be a centre of $C$ considered 
as the hyperbolic  circle:
This means that if two curves $\gamma,\gamma'$ ar  
arbitrary hyperbolic geodesics
passing through the point $A$, and 
$L_\gamma,L_{\gamma'} $ are points
of the intersections of these geodesics 
with circle $C$, 
$L_\gamma=\gamma\times C$,  
$L_{\gamma'}=\gamma'\times C$  
then the lengths (hyperbolic) of these geodesics 
coincide. In particular this implies
that all these geodesics intersect the circle $C$
under the angle $\pi\over 2$.

Now look at this picture from the point of view 
of Euclidean geometry. 
   All gedesics are half-circles
with centres on the absolute, the line $x=0$,
(except the geodesicis which are the vertical lines)
Angles are the same. If geodesic $\gamma$ is represented
by the half-circle with the centre at the point 
$K$, then the segment $KL_\gamma$, radius of this half-circle
is tangent to the circle $C$.
 Thus we come to conclusion
that for the arbitrary  point $K$ on the absolute
the (Euclidean) length of the tangent from the point $K$ to the
circle $C$ is equal to the (Euclidean) length of the segment $KA$.

\bigskip



The considerations above make us to 
formulate the following problem of Eucldean geometry.

  Let $C$ be a circle in the Euclidean 
plane and let $A$, be an arbitrary point on this plane.

 Consider the locus $M_{C,A}$ of the points $K$ such that the 
length of the tangent from $K$ to the circle
$C$ is equal to the length of the segment $KA$:
                     $$
   M_{C,A}=
\{K\colon\quad KA=
\hbox {lenght of the tangent from $K$ to the circle.}\}\,.
     \eqno (1)
                     $$

Find the locus  $M_{C,A}$.

This problem looks as standard geometrical question (in Euclidean gepmetry).
 Temporary forgetting where this problem comes from, we will
discuss first solution of this problem just in terms of  Eucldean
geometry


\bigskip


{\sl Solution}

Let $A'$ be  a point which is inverse to the point
$A$ with respect to the circle $C$:
Points $O$ (centre of the circle), $A$ and $A"$
belong to the same ray $r_{OA}$ and
             $$
|OA|\cdot |OA'|=1\,.
   \eqno (2)
                 $$ 
(We suppose that  the circle $C$ has unit length.)


Consider the set of circles which are passing throug the points
   $A$ and $A'$. One can see that for every such a circle,
                     $$
 \hbox {it intersects the circle $C$ under the right angle}
             \eqno(3)
                     $$
 This means that for centre $K$ of such circle, the
tangent to the circle $C$ is the radius of this circle.
   We see that the locus  $M_{C,A}$ $=$ the locus
 of the centres of circles passing through the points 
     $A$ and $A'$  $=$
  the locus of the points which are on the
same (Euclidean) distance from the points $A$ and $A'$.
    This is the line $l$ which is ortogonal to the
$AA'$ and 
passes through th emiddle point $P$ of the segment
  $AA'$ ($P\in AA', |AP|=|PA'|$. 

    So we proved that
                   $$
        M_{C,A}= {l\colon \quad d(l,A)=d(l,A')}\,.
                      \eqno (4)
                   $$



   Without loss of generality  suppose that 
point $A$ belongs to the interior of the
circle, then the point $A'$ is out of the circle.


\m



 Let $M$ be a point of intersection of the ray $r_{OA}$
with the circle $M$, and $N$ be a point on the
circle $C$m which is on the continuation of
the ray $r_{OA}$. Denote by $a$ 
the length of the segment $OA$, then due
to (2) $|OA'|={1\over a}$, and
   $|OP|={1\over 2}\left(a+{1\over a}\right)$.
  One can see that
          $$
 |MA|\cdot |MA'|=(1+a)\left(1+{1\over a}\right)=
2+a+{1\over a}=
2\left(1+{1\over 2}\left(a+{1\over a}\right)\right)
     =|MN|\cdot |MP|\,.
     \eqno (3)
          $$
The relation (2) means that 
inversion with centre at the point
$O$ with radius $r=1$ transforms point $A$
to the point $A'$ and vice versa.
We call this inversion the first inversion,
the inversion $I_I$.
  The relation (3) means that the inversion
with centre at the point $N$ and with radius
   $r'=\sqrt{(1+a)\left(1+{1\over a}\right)}$
also ransforms point $A$
to the point $A'$ and vice versa.
This inversion also transforms the circle $C$
to the line $l$ and vice versa, since
   a point $M$ of the circle is the centre of the
inversion.
We call this inversion the second inversion,
the inversion $I_{II}$:
          $$
I_{I}\colon\quad A\leftrightarrow A'\,,\quad
I_{II}\colon\quad A\leftrightarrow A'\,,
{\rm and}\,\, C\leftrightarrow l\,.
         \eqno (4)
          $$

We are ready to prove the Claim. Let $K$
be an arbitrary point on the line $l$.

Consider the circle $C_K$ with the centre at the point
$K$ and with the radius $|KA|$.
   This circle passes trhough the points $A, A'$m
hence according to equation (4) its inverse
   $I_{II}(C_K)$,
is passinig through these points also.  Since
  the curve $I_{II}(C_K)$ is the circle (or line)
passing via the points $A,A'$
hence it is orthogonal to the line $l$.
  Hence the circle $C_K$ is orthogonal to the
curve $C=I_{II}(l)$.   This means
that tangent from the point $K$ to the circle $C$
is the radius, thus length of the
tangent  is equal to the $|KA|$.  

\bigskip


 \centerline {\it Meaning in hyperbolic geometry}

The considerations of the first paragraph
show that the fact that 
  $l=M_{C,A}$ has explanations in hyperbolicity.


Indeed the line $l$  divides the plane on
two half-planes. Cconsider the half-plane 
the  circle $C$ belogs to, as a model
of jyperbolic plane
\footnote{$^*$}
         {
   Recall shrotly what is it. 
One can consider 
Cartesian coordinates $(x,y)$ such that
 the line $l$ is $y=0$ the half-plane 
is $y\geq 0$.
Then hyperbolic plane $H$ can be defined as 
as this half-plane with Riemannian metric
         $
       G={dx^2+dy^2\over y^2}
         $.
The geodesics of this metric 
(lines of hyperbolic plane) 
are vertical lines $x=a$
and upper half-circles with centre on the absolute
   $l$: $\cases{(x-a)^2+y^2=R^2\cr y>0 \cr}$.
The distance  
between two points $A_1=(x_1,y_1)$
and $A_2=(x_2,y_2)$, the length of the geodesic
passing via points $A_1,A_2$
 can be defined alternatively
by cross-ratio of the points
        $$
d(A_1,A_2)=\left|
  \log\left(A_1, A_2,A_0,A_\infty)\right)\right|=
\left|\log\left(
 {z_1-z_0\over z_1-z_\infty}:
 {z_2-z_0\over z_1-z_\infty}\right)\right|\,,
        $$  
where points $A_0,A_\infty$
are points of intersection of the half-circle with absolute, $z_1=x_1+iy_1$,$z_2=x_2+iy_1$.
E.g. for two points $A_1=(0,a_1)$, $A_2=(o,a_2)$
on the vertical line (this is geodesic)
 $$
d(A_1,A_2)=
    \left|\log
  \left|(A_1, A_2,0,\infty)\right)\right|=
            \left|
             \log
             \left(
 {ia_1-0\over ia_1-\infty}:
 {ia_2-0\over ia_1-\infty}\right)\right|=
            \left|
             \log
             \left(
 {a_1\over a_2}\right)\right|\,.
        $$ 
}.


One of he points $A$ or $A'$ is in the circle $C$. 
Consider the Lobachevsky plane formed by uer half plane, with
  $l$ is absolute.  
  The circle $C$ will be the circle in the hyperbolic plnae also.
One can see that the point $A$ is the centre
of this hyperbolic circle. 


   Indeed  due to lemma all the geodesics interesect 
the circle $C$ under the right angle, i.e. the points of $l$
belong to this locus.

 
\bye

