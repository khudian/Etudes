
\magnification=1200 
\baselineskip=14pt
\def\vare {\varepsilon}
\def\A {{\bf A}}
\def\t {\tilde}
\def\a {\alpha}
\def\K {{\bf K}}
\def\N {{\bf N}}
\def\V {{\cal V}}
\def\L {{\cal L}}
\def\s {{\sigma}}
\def\S {{\Sigma}}
\def\s {{\sigma}}
\def\p{\partial}
\def\vare{{\varepsilon}}
\def\Q {{\bf Q}}
\def\D {{\cal D}}
\def\G {{\Gamma}}
\def\C {{\bf C}}
\def\M {{\cal M}}
\def\Z {{\bf Z}}
\def\U  {{\cal U}}
\def\H {{\cal H}}
\def\R  {{\bf R}}
\def\S  {{\bf S}}
\def\E  {{\bf E}}
\def\l {\lambda}
\def\degree {{\bf {\rm degree}\,\,}}
\def \finish {${\,\,\vrule height1mm depth2mm width 8pt}$}
\def \m {\medskip}
\def\p {\partial}
\def\r {{\bf r}}
\def\v {{\bf v}}
\def\n {{\bf n}}
\def\t {{\bf t}}
\def\b {{\bf b}}
\def\c {{\bf c }}
\def\e{{\bf e}}
\def\ac {{\bf a}}
\def \X   {{\bf X}}
\def \Y   {{\bf Y}}
\def \x   {{\bf x}}
\def \pt   {{\bf p}}
\def \y   {{\bf y}}
\def\w{\omega}
\def\finish {${\,\,\vrule height1mm depth2mm width 8pt}$}

\centerline  {\bf Killing vectors and Levi-Ciovita conenction}
 
{\sl This lecture is the addition to the subsection 1.6 
``Infinitesimal isometries
of Riemannian manifold''}
  

  Let $M$ be Riemannian manifold with Riemannian metric $G$.
   Recall that a vector field $\K$ is Killing vector field,
i.e. it defines infinitesimal isometry, 
  (given in local coordinates by the formula 
$x^i\mapsto x^i+\vare K^i$, $\vare^2=0$) if  Lie derivative
of metric with respect to vector field vanishes:
                       $$
\L_\K G=K^i{\p g_{mn}(x)\over \p x^i}+
  {\p K^i(x)\over\p \p x^m}g_{in}(x)+
  {\p K^i(x)\over\p \p x^n}g_{im}(x)=0\,.
                \eqno (1)
                       $$
(see  subsection 
``Infinitesimal isometries
of Riemannian manifold'' of Lecture notes.)

   Let $\nabla$ be Levi-Civita connection of Riemannian metric:
in local coordinates Christoffel symbols of this connection are
                  $$
   \G^i_{km}={1\over 2}g^{ij}(x)
\left(
{\p g_{jk}\over \p x^m}+
{\p g_{jm}\over \p x^k}-
{\p g_{km}\over \p x^j}
\right)\,,
\eqno (2)
            $$
 where $g^{ij}$ is tensor inverse to $g_{ik}$.
                 
  Consider  the following construction.

  Let $\K$ be an arbitrary vector fields 
 (not necessarily Killing vector field) on manifold $M$,
  Consider the following operation on vector fields:
   to every vector field $\X$ we assign the vector field
                 $$
\left(\nabla_\K-\L_\K\right)\X, 
                      \eqno (3)
                 $$
where  $\nabla$ is Levi-Civita connection (2) and
w$\L_\K$ is Lie derivative of $\X$ with respect to $\K$:
                 $$
\L_\K\X=[\K,\X],    \quad
       [\K,\X]^i=K^m{x}{\p X^i(x)\over \p x^m}-
        X^m{x}{\p K^i(x)\over \p x^m},
                 $$ 
Hence we have that equation  (3) can be written as
                 $$
\left(\nabla_\K-\L_\K\right)\X=
                  $$
                  $$
                   \left(
                  \left(
                K^m
                 \left({\p X^i\over \p x^m}+\G^i_{mn}X^n
                  \right)
                   \right)
                    -
                \left(
    K^m{\p X^i\over \p x^m}-
        X^m{\p K^i\over \p x^m} 
            \right)
                   \right){\p \over \p x^i}=
                 $$
                $$
                  X^m
                {\p K^i\over \p x^m}+
                   K^m\G^i_{mn}X^n=
                   X^m{\p K^i\over \p x^m}+
                   X^n\G^i_{mn}K^M=
                X^m{\p K^i\over \p x^m}+
                   X^n\G^i_{nm}K^m=\nabla_\X\K
                \eqno (4)
               $$
since Levi-Civita connection is symmetric: $\G^i{mn}=\G^i_{nm}$.
   We see that operation (3) defines linear operator $A_\K$
 in the following way:
  to every vector $\X$ tangent to manifold at the point $\pt$,
   $\X\in T_{\pt}M$ operator $A_\K$ assigns vector $\nabla_\X \K$:
                   $$
T_\pt M\ni \X\mapsto A_\K(\X)=\nabla_\X\K=
                 \left(
      \left({\p K^i\over \p x^m}+\G^i_{mn}K^n\right)X^m
           \right){\p \over \p x^i}
                        \eqno (4a)
                   $$

It follows from formulae (3) and (4) that for every vector
field $\X$ on $M$
                $$
   A_\K(\X)=\left(\nabla\X-\L_\K\right)\X\,.
               \eqno (4b)
                $$
In the left hand side of this  formula $\X$ 
is the value of vector field at the given point $\pt$
and on the right hand side covarinat derivative and Lie derivative
act on vector  field. More accurate we have to write this formula
in the following way: 
  $$
   A_\K\left(\X\big\vert_\pt\right)=
         \left(
      \left(\nabla_\K-\L_\K\right)\X
            \right)
         \big\vert_\pt
               \eqno (4c)
                $$
Now we formulate and prove the Proposition

\medskip

{\bf Proposition} Let $M$ be Riemannian manifold
 and $\nabla$ be its Levi-Civita connection. 
Vector field $\K$ is Killing vector field 
of Riemannina manifold $M$ if and only if
  the linear operator
  $A_\K$ defined by (4) is antisymmetric operator, i.e.
if for arbitrary vector fields $\X,\Y$
                    $$
\langle A_\K(\X),\Y\rangle=-\langle \X,A_\K(\Y)\rangle\,.
                  \eqno (5)
                    $$
where $\langle \,\,,\,\, \rangle$ is Riemannian scalar product:
                   $$
  \langle\X,\Y\rangle= X^i(x)g_{ik}(x)Y^k(x)\,.
                   $$

{\sl Proof}  



Let $\X,\Y$ be two arbitrary vector fields.

The condition that $\nabla$ is Levi-Civita connection means
that
              $$
     \L_\K\langle\X,\Y\rangle=
     \p_\K\langle\X,\Y\rangle=
     \langle\nabla_\K\X,\Y\rangle+
     \langle\X,\nabla\Y\rangle\,.
          \eqno (6)
              $$
(See the subsection in Lecture notes about Levi-Civita connection.)
(Lie derivative of function is just directional derivative:
 $\L_\K F=\p_\K F$).

The condition that $\K$ is Killing vector field
is the condition that $\K$ preserves Riemannian metric,
scalar product (see (1)). Hence
                  $$
     \L_\K\langle\X,\Y\rangle=
     \p_\K\langle\X,\Y\rangle=
     \langle\L_K\X,\Y\rangle+
     \langle\X,\L_\X\Y\rangle
                  \eqno (6a)
                  $$
 Now substract an equation (6a) from the equation (6).
Using relations (4a,4b,4c) we come to
                $$
\langle\left(\nabla_\K-\L_\K\right)\X,\Y\rangle+
\langle\X, \left(\nabla_\K-\L_\K\right)\Y\rangle=
\langle \nabla_\X\K,\Y\rangle+
\langle\X,\nabla_\Y\K\rangle=
\langle A_\K(\X),\Y\rangle+
\langle\X,A_\K(\Y)\rangle=0\,.
                $$
Thus relation (5) is proved \finish
\bye

