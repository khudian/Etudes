\magnification=1200
\baselineskip=14pt
\def\vare {\varepsilon}
\def\A {{\bf A}}
\def\t {\tilde}
\def\a {\alpha}
\def\K {{\bf K}}
\def\N {{\bf N}}
\def\V {{\cal V}}
\def\s {{\sigma}}
\def\S {{\Sigma}}
\def\s {{\sigma}}
\def\p{\partial}
\def\vare{{\varepsilon}}
\def\Q {{\bf Q}}
\def\D {{\cal D}}
\def\G {{\Gamma}}
\def\C {{\bf C}}
\def\M {{\cal M}}
\def\Z {{\bf Z}}
\def\U  {{\cal U}}
\def\H {{\cal H}}
\def\R  {{\bf R}}
\def\E  {{\bf E}}
\def\l {\lambda}
\def\degree {{\bf {\rm degree}\,\,}}
\def \finish {${\,\,\vrule height1mm depth2mm width 8pt}$}
\def \m {\medskip}
\def\p {\partial}
\def\r {{\bf r}}
\def\v {{\bf v}}
\def\n {{\bf n}}
\def\t {{\bf t}}
\def\b {{\bf b}}
\def\e{{\bf e}}
\def\ac {{\bf a}}
\def \X   {{\bf X}}
\def \Y   {{\bf Y}}
\def \x   {{\bf x}}
\def \y   {{\bf y}}
\def\f {{\bf f}}
\def\la{\langle}
\def\ra{\rangle}
\centerline  {\bf Everybody has to know this}


   Symmetric matrices.
  
  
  
  Let $A=||a_{ik}||$ be $n\times n$ symmetrical matrix with real entries. 
   
   
   
      Matrix $A$ defines symmetric bilinear form $A(\X,\Y)$ in $\R^n$:
                      $$
                      A(\X,\Y)=A_{ik}X^iY^k\,,\qquad \hbox{   (summation over repeated indices)}
                      \eqno (0.1)
                      $$
 where $\X,\Y$ are vectors from $\R^n$.
   
          \m
          
             \centerline {{${\cal x} 0$. } Geometrical meaning of $\R^n$}         
   
   \m
   
   $\R^n$ is a vector space of $n$-tuples: $\R^n=(x^1,x^2,\dots,x^n)$.  
    One can consider canonical basis $\{\e_1,\e_2,\dots,\e_n\}$ in $\R^n$, where
                        $$
                        \matrix
                         {
                     \e_1=& (1,0,0\dots,0,0) \cr
                    \e_2=& (0,1,0\dots,0,0) \cr
                    \dots & \dots \cr
                    \e_n=& (0,0,0\dots,0,1) \cr
                    }
                             $$
Then for an arbitrary vector   $\R^n\ni \ac=(a^1,a^2,a^3,\dots,a^n)$
                     $$
            \ac=(a^1,a^2,a^3,\dots,a^n)=a^i\e_i
                  $$
        One can consider canonical scalar product $\la\,,\,\ra_0$ in $\R^n$ such that
                             $$
                 \la \e_i,\e_j\ra_0=\delta_{ij}, 
                 \quad \la\X,\Y\ra_0=X^1Y^1+X^2Y^2+\dots X^nY^n\,.
                             $$          
 The canonical basis $\{\e_i\}$ is orthonormal basis with respect to the canonical scalar product.
 
   We say that an arbitrary basis $\{\f_1,\dots\f_n\}$ is orthonormal (with respect to canonical scalar product)
    if
             $$
             \la\f_i,\f_j\ra_0=\delta_{ij}
              $$ 
   One can see that the basis $\{\f_i\}$ is orthonormal if and only if the transition matrix $O$
    from the canonical basis $\{\e_i\}$ to the basis $\{\f_i\}$ is orthonormal matrix:
                   $$
                   \f_i=O_{i}^j\e_{j}, \qquad     O^{^T}O=I\,.
                   \eqno (1.1)
                   $$
   
              
                \centerline {{${\cal x} 1$. } Geometrical meaning of symmetrical matrix.}
   Now return to the formula (0.1). It defines bilinear form on $\R^n$.
   
   Lete  $\{\f_1,\dots,\f_n\}$ be an arbitrary  basis, (not necessarily orthonormal!)
    then  $\f_i=P_i^m\e_m$, where $P$ is non-degenerate.  We see that
                 $$
                A(\f_i,\f_j)=A(P_i^m\e_m,P_j^n\e_n)=P_i^m A_{mn}P_j^n=\left(PAP^{^T}\right)_{ij} 
                \eqno (1.2)
                 $$
     Entries of the matrix $A=||a_{ij}||$ are components of bilinear form  $A(\X,\Y)$ with respect to the canonical basis $\{\e_i\}$.
     Entries of the matrix  $A'=PAP^{^T}=||P_i^ma_{mn}P_j^n||$ are components of the bilinear form
     $A(\X,\Y)$ with respect to the  new basis $\f_i=P_i^m\e_m$.
   
 
   
 Note also that symmetric bilinear form $A$ together with canonical scalar product $\la\,,\,\ra_0$
 defines a linear operator    which  we denote by $\hat A$:
                    $$
                   \hat A\colon\quad \la \hat A \X, \Y\ra_0=A(\X,\Y) 
                   \eqno (1.3)
                    $$
 
 
   \m

1.There are two  absolutely fundamental questions: at what extent we can simplify matrix components  of bilinear form  $A(\X,\Y)$ by basis  transformation?
   
2.  What can we say about eigenvalues of operator $S_A$?   

\bigskip
   
   
          \centerline {{${\cal x} 2$. } Answer on the first question}   
 
 \m
   
At beginning   we answer the first question in the special but very important case.
   
   {\bf Definition}  Symmetric bilinear form $A(X,Y)$ or corresponding quadratic form
   $A(X,X)$ is called positive-definite if the quadratic form takes positive values on all non-zero vectors:
                $$
              A(X,X)>0\,\,\, {\rm if} X\not=0\,.  
              \eqno (2.1)
                $$
   Respectively matrix $||a_{ik}||$ is called positive-definite if corresponding quadratic form $A(X,X)=a_{ik}X^iX^k$
   is positive-definite.

\m
   
   {\bf Remark}
   {\it One can see that in this case $A(\X,\Y)$ defines a scalar product
    (In general another scalar product , not canonical one)}.

\m

  
   
   We prove the following 
   
   {\bf Theorem} Let $A(X,Y)$ be symmetric bilinear such that it is positive-definite.
   Then there exists a new basis $\{\f_i\}$ (not necessarily orthonormal!) such that
   in this basis           
                    $$
            A(\f_i,\f_j)=\delta_{ij}\,.
            \eqno (2.2)
                    $$
     It means that for every vector $\x=x^i\f_i$ ($x^i$ are components of vector in the new basis $\{\f_i\}$)
                  $$
  A(\x,\x)=(x^1)^2+(x^2)^2+\dots+\dots+(x^{n-1})^2+(x^{n})^2
                  $$
   In other words the new scalar product $\la\,,\,\ra_A=A(X,Y)$
   has a canonical appearance and the new basis $\{\f_1,\dots,\f_n\}$
   is a orthonormal basis with respect to this new scalar product (see (2.2)):
   
   $$ $$
   
   {\sl Proof}  We will  construct explicitly the new basis $\{\f_i\}$ such that (2.2) is obeyed.
   
   Before doing this note that
    if $A(X,X)$ is positive definite then all main minors are not equal to zero.
    
   
      Construct $f_1,f_2,\dots,f_n$ by induction using triangular linear transformations.
      
        Inductive hypothesis:
        
       For every integer $k=1,2,\dots,n$,  there exists a set of vectors  $\{\f_1,\f_2,\dots,\f_k\}$
       such that
                           $$
                           \cases
                           {
                           \f_1= S_1^1 \e_1\cr
                           \f_2=S_2^1\e_1+S_2^2\e_2\cr
                           \f_3=S_3^1\e_1+S_3^2\e_2+S_3^3\e_3\cr
                           \dots \cr
                           \f_k=S_k^1\e_1+S_k^2\e_2+S_k^3\e_3+\dots+S_k^k\f_k\cr
                            }
                           $$
       and       
                     $$
               A(\f_i,\f_j)=\delta_{ij},\quad \hbox{for $i,j\leq k$}      
                     $$
       
      I $k=1$.  $A(\e_1,\e_1)=a_{11}>0$ due to positive definiteness. 
      Consider 
                 $$
               f_1={1\sqrt {a_11}}\e_1.  
                   $$
  We see that $A(\f_1,\f_1)=1$ 
  
     II $k=m$ Suppose that  there exists a set of vectors  $\{f_1,\f_2,\dots,f_m\}$ such that
                $ A(\f_i,\f_j)=\delta_{ij}$, for all $i,j\leq m$.
                
                
   
   
  
  
  In the general case the answer is given by the following Theorem 
   
   {\bf Theorem 2} Let $A=A(\X,\Y)$ be a symmetric bilinear form.
    Assume that it is non-degenerate. Then there exists a new basis $\{\f_1,\dots,\f_n\}$ such that in this basis
                  $$
                   A(\f_i,\f_j)=\cases 
                   {
                  \delta_{ij} \qquad \hbox{for $i,j \leq p$}\cr
                  -\delta_{ij} \qquad \hbox{for $i,j >p$}\cr
                   }   
                  $$
   It means that every vector $\x=x^i\f_i$ ($x^i$ are components of vector in the new basis $\{\f_i\}$)
                  $$
  A(\x,\x)=(x^1)^2+(x^2)^2+\dots+(x^p)^2-(x^{p+1})^2-(x^{p+2})^2-\dots-(x^{n-1})^2-(x^{n})^2
                  $$
     $A(\X,\X)$ is quadratic form corresponding to symmetric bilinear form $A(\X,\Y)=A(\Y,\X)$.
   
   
  In matrix notations this means that there exist matrix $P$ such that for the matrix
                                 $$
                     PAP^{^T}=||P_{i}^ma_{mn}P_{j}^n||=
                            \pmatrix
                                 {
                            I_{p} &0\cr
                              0     & -I_{n-p}\cr
                              }={\rm diag}[\underbrace{1,1,\dots,1}_{\hbox{$p$ times}} \underbrace{-1,-1,\dots,-1}_{\hbox{$n-p$ times}}]
                                 $$
where we denote by $I_k$ unit $k\times k$ matrix.
   A number $p$ is called the index of the form $A$. It is an invariant (does not depend on basis $\{\f_i\}$.)
   
   
   We first will answer the special (very important case) of this question.
   
   
   Suppose a matrix $A$ is such that 
   
   To answer this question we will describe the explicit construction of the basis $\{\f_i\}$
   which diagonalizes  the  bilinear form.
   
   Consider $\f_1=c_1\e_1$
   
   
   
   
   $$ $$


          \centerline {{${\cal x} 3$. } Answer on the second question}

   
   {\bf Theorem 2} There exist a new orthonormal basis $\{\f_1,\dots,\f_n\}$ such that in this orthonormal 
   basis
                  $$
                   A(\f_i,\f_j)=\lambda_i \delta_{ij} \quad \hbox{(no summation)}
                  $$
   
This means that the  vectors $\{\f_1,\dots,\f_n\}$
are eigenvectors of the operator $\hat A$ corresponding to bilinear form $A(\X,\Y)$ and canonical scalar product
$\la\,,\,\ra_0$ (see 1.3):
                          $$
                   \A \f_i=\lambda_i \f_i      
                          $$


  In matrix notations this means that there exist orthogonal matrix $0$ such that for the matrix
                                 $$
                     PAP^{^T}=OAO^{-1}
                                $$
              is a diagonal matrix.
              
     
    
              
                               $$ 
                            \pmatrix
                                 {
                            I_{p} &0\cr
                              0     & -I_{n-p}\cr
                              }={\rm diag}[\underbrace{1,1,\dots,1}_{\hbox{$p$ times}} \underbrace{-1,-1,\dots,-1}_{\hbox{$n-p$ times}}]
                                 $$
where we denote by $I_k$ unit $k\times k$ matrix.
   
   
      
   
   
   1) If matrix $A$ is symmetric and non-degenerate then there is a basis 
   
   1)  
  
  
  
  
  
  {\bf Definition} We say that matrix $A=||a_{ik}||$ is positive definite if
                $$
        A(\X,\X)>0\, \qquad {\rm for}\quad \X\not=0.
                $$

   {\it One can see that in this case $A(\X,\Y)$ defines a scalar product
    (In general another scalar product , not canonical one)}.

   
  
  \bye
  