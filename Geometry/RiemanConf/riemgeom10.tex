% It is a new course. I began this document on 31-st January
\def\vare {\varepsilon}
\def\A {{\bf A}}
\def\t {\tilde}
\def\a {\alpha}
\def\K {{\bf K}}
\def\N {{\bf N}}
\def\V {{\cal V}}
\def\s {{\sigma}}
\def\S {{\bf S}}
\def\s {{\sigma}}
\def\p{\partial}
\def\vare{{\varepsilon}}
\def\Q {{\bf Q}}
\def\D {{\cal D}}
\def\G {{\Gamma}}
\def\C {{\bf C}}
\def\M {{\cal M}}
\def\Z {{\bf Z}}
\def\U  {{\cal U}}
\def\H {{\cal H}}
\def\R  {{\bf R}}
\def\E  {{\bf E}}
\def\l {\lambda}
\def\degree {{\bf {\rm degree}\,\,}}
\def \finish {${\,\,\vrule height1mm depth2mm width 8pt}$}
\def \m {\medskip}
\def\p {\partial}
\def\r {{\bf r}}
\def\v {{\bf v}}
\def\n {{\bf n}}
\def\t {{\bf t}}
\def\b {{\bf b}}
\def\e{{\bf e}}
\def\f{{\bf f}}
\def\ac {{\bf a}}
\def \X   {{\bf X}}
\def \Y   {{\bf Y}}
\def\diag {\rm diag\,\,}
\def\pt {{\bf p}}
\def\w {\omega}
\def\la{\langle}
\def\ra{\rangle}
\def\x{{\bf x}}

\documentclass[12pt]{article}
\usepackage{amsmath,amsthm}


\usepackage{amsmath,amssymb,amsfonts,amsthm}


\theoremstyle{theorem}
\newtheorem{thm}{Khimera}

\numberwithin{equation}{section}


\title{Riemannian Geometry}
\date{}
\begin{document}
\maketitle

  \centerline {it is a draft of Lecture Notes of H.M. Khudaverdian.}

  \centerline { Manchester, 20 May 2010}





\tableofcontents


\section {Riemannian manifolds}


\subsection { Manifolds. Tensors. (Recalling)}


I recall briefly basics of manifolds and tensor fields on manifolds.

An $n$-dimensional manifold is a space such that in a vicinity of any point
one can consider local coordinates $\{x^1,\dots,x^n\}$ (charts). One can consider different local coordinates.
If both coordinates $\{x^1,\dots,x^n\}$, $\{y^1,\dots,y^n\}$ are defined in a vicinity of the given point
then they are related by  bijective transition functions (functions defined on domains in $\R^n$ and taking values in $\R^n$).
              $$
             \begin{cases}
             x^{1'}=x^{1'}(x^1,\dots,x^n)\cr
             x^{2'}=x^{2'}(x^1,\dots,x^n)\cr
                \dots\cr
            x^{{n-1}'}=x^{{n-1}'}(x^1,\dots,x^n)\cr
              x^{{n}'}=x^{{n}'}(x^1,\dots,x^n)\cr

             \end{cases}
              $$
 We say that manifold is {\it differentiable} or {\it smooth} if transition functions are diffeomorphisms,
i.e. they are smooth and rank of Jacobian is equal to $k$, i.e.
              \begin{equation}\label{transfunctions---diffeomorphisms}
      \det
          \begin{pmatrix}
          {\p x^{1'}\over \p x^1} &{\p x^{1'}\over \p x^2}\dots &{\p x^{1'}\over \p x^n}\cr
         {\p x^{2'}\over \p x^1} &{\p x^{2'}\over \p x^2} \dots &{\p x^{2'}\over \p x^n}\cr
                           \dots\cr
      {\p x^{n'}\over \p x^1} &{\p x^{n'}\over \p x^2}\dots &{\p x^{n'}\over \p x^n}\cr
          \end{pmatrix}\not=0
              \end{equation}




    A good example of manifold is an open domain  $D$ in  $n$-dimensional vector space $\R^n$.
    Cartesian coordinates on $\R^n$ define global coordinates on $D$.
    On the other hand one can consider an arbitrary local coordinates in different domains in $\R^n$.

    E.g. one can consider polar coordinates $\{r,\varphi\}$ in a domain $D=\{x,y\colon y>0\}$ of $\R^2$
    (or in other domain of $\R^2$) defined by standard formulae:
              \begin{equation*}
                \begin{cases}
                x=r\cos\varphi\cr y=r\sin\varphi\cr
                \end{cases}\,,
              \end{equation*}
            \begin{equation}\label{jacobianofpolartransform}
                \det
          \begin{pmatrix}
          {\p x\over \p r} &{\p x\over \p \varphi}\cr
          {\p y\over \p r} &{\p y\over \p \varphi}\cr
          \end{pmatrix}=
                \det
          \begin{pmatrix}
          \cos\varphi &-r\sin\varphi\cr
          \sin\varphi &r\cos\varphi\cr
          \end{pmatrix}=r
            \end{equation}
             or one can consider spherical coordinates  $\{r,\theta,\varphi\}$ in a domain
  $D=\{x,y,z\colon x>0,y>0, z>0\}$ of $\R^3$ (or in other domain of $\R^3$)
  defined by standard formulae
  \begin{equation*}
                \begin{cases}
                x=r\sin\theta\cos\varphi\cr y=r\sin\theta\sin\varphi\cr z=r\cos\theta
                \end{cases}\,,
              \end{equation*}\,,
              \begin{equation}\label{jacobianofspherictransform}
                \det
          \begin{pmatrix}
          {\p x\over \p r} &{\p x\over \p \theta} &{\p x\over \p \varphi}\cr
          {\p y\over \p r} &{\p y\over \p \theta} &{\p y\over \p \varphi}\cr
           {\p z\over \p r} &{\p z\over \p \theta} &{\p z\over \p \varphi}\cr
          \end{pmatrix}=
                \det
          \begin{pmatrix}
          \sin\theta\cos\varphi &r\cos\theta\cos\varphi &-r\sin\theta\sin\varphi\cr
          \sin\theta\sin\varphi &r\cos\theta\sin\varphi &r\sin\theta\cos\varphi\cr
           \cos\theta &-r\sin\theta &0\cr
                 \end{pmatrix}=r^2\sin\theta
            \end{equation}

 Choosing domain where polar (spherical) coordinates are well-defined we have to be award
that coordinates have to be well-defined and transition functions \eqref{transfunctions---diffeomorphisms}
  have to be diffeomorphisms.
\m


Examples of manifolds: $\R^n$, Circle $S^1$,  Sphere $S^2$, in general sphere $S^n$, torus $S^1\times S^1$,
  cylinder, cone, \dots.

\m

We also have to recall briefly what are tensors on manifold.
\m


        \centerline {\it Tensors on Manifold}

For every point $\pt$ on manifold $M$
one can consider tangent vector space $T_\pt M$---the space of vectors tangent to the manifold at the point $M$.


 Tangent vector $\A=A^i{\p\over \p x^i}$. Under changing of coordinates it transforms as follows:
             $$
         \A=A^i{\p\over \p x^i}=A^i{\p x^{m'}\over \p x^i}{\p\over \p x^{m'}}=A^{m'}{\p\over \p x^{m'}}
             $$
  Hence

            \begin{equation}\label{transformofvector}
                   A^{i'}={\p x^{i'}\over \p x^i}A^i
            \end{equation}
              Consider also cotangent space $T^*_\pt M$ (for every point $\pt$ on manifold $M$)---
              space of linear functions on tangent vectors, i.e. space of $1$-forms which sometimes are called
              {\it covectors.}:

         One-form (covector) $\w=\w_idx^i$ transforms as follows
            $$
      \w=\w_m dx^m=\w_m {\p x^m\over \p x^{m'}}dx^{m'}=\w_{m'}dx^{m'}\,.
        $$
      Hence
         \begin{equation}\label{transformofoneform}
            \w_{m'}={\p x^{m}\over \p x^{m'}}\w_m\,.
         \end{equation}

\m

          {\it Tensors}:
\smallskip

One can consider contravariant tensors  of the rank $p$
                   $$
             T=  T^{i_1i_2\dots i_p}{\p\over \p x^{i_1}}\otimes
             {\p\over \p x^{i_2}}\otimes \dots \otimes {\p\over \p x^{i_k}}
                   $$
with components $\{T^{i_1i_2\dots i_k}\}$.


One can consider covariant tensors  of the rank $q$
                   $$
             S=  S_{j_1j_2\dots j_q}dx^{j_1}\otimes
              dx^{j_2}\otimes \dots dx^{j_q}
                   $$
with components $\{S_{j_1j_2\dots j_q}\}$.

One can also consider mixed tensors:
          $$
        Q=Q^{i_1i_2\dots i_p}_{j_1j_2\dots j_q}{\p\over \p x^{i_1}}\otimes
             {\p\over \p x^{i_2}}\otimes \dots \otimes {\p\over \p x^{i_k}}\otimes
             dx^{j_1}\otimes
              dx^{j_2}\otimes \dots dx^{j_q}
          $$
with components $\{Q^{i_1i_2\dots i_p}_{j_1j_2\dots j_q}\}$. We call these tensors
{\it tensors of the type  $\begin{pmatrix} p\cr q\cr\end{pmatrix}$}.
Tensors of the type $\begin{pmatrix} p\cr 0\cr\end{pmatrix}$ are called
{\it contravariant tensors of the rank p}.

Tensors of the type $\begin{pmatrix} 0\cr q\cr\end{pmatrix}$ are called
{\it covariant tensors of the rank q}.


Having in mind \eqref{transformofvector} and \eqref{transformofoneform} we come  to
the rule of transformation for tensors of the type  $\begin{pmatrix} p\cr q\cr\end{pmatrix}$:
\begin{equation}\label{ruleoftransformationofarbitrarytensors}
    Q^{i'_1i'_2\dots i'_p}_{j'_1j'_2\dots j'_q}=
    {\p x^{i'_1}\over \p x^{i_1}}
    {\p x^{i'_2}\over \p x^{i_2}}
    \dots
    {\p x^{i'_p}\over \p x^{i_p}}
    {\p x^{j_1}\over \p x^{j'_1}}
    {\p x^{j_2}\over \p x^{j'_2}}
    \dots
    {\p x^{j_q}\over \p x^{j'_q}}
    Q^{i_1i_2\dots i_p}_{j_1j_2\dots j_q}
\end{equation}



E.g. if $S_{ik}$ is a covariant tensor
of rank $2$  (tensor of the type $\begin{pmatrix} p\cr q\cr\end{pmatrix}$) then
            \begin{equation}\label{ruleoftransformfor 2-tensor}
                S_{i'k'}={\p x^i\over \p x^{i'}}{\p x^k\over \p x^{k'}}S_{ik}\,.
            \end{equation}
 If $A^i_k$ is a tensor of rank $\begin{pmatrix} 1\cr 1\cr\end{pmatrix}$ (linear operator on $T_\pt M$)
 then

             $$
    A^{i'}_{k'}=        {\p x^{i'}\over \p x^{i}}{\p x^k\over \p x^{k'}}A^i_k
             $$



{\bf Remark} Transformations formulae \eqref{transformofvector}---\eqref{ruleoftransformfor 2-tensor}  define
vectors, covectors and in generally any tensor fields in components. E.g. covariant tensor (covariant tensor field)
of the rank 2 can be defined as matrix $S_{ik}$ (matrix valued function $S_{ik}(x)$) such that
under changing of coordinates $\{x^1,x^2,\dots,x^n\}\mapsto \{x^{1'},x^{2'},\dots,x^{n'}\}$
 $S_{ik}$ change by the rule \eqref{ruleoftransformfor 2-tensor}.


{\bf Remark} {\it Einstein summation rules}


 In our lectures we always use so called {\it Einstein summation convention}.
 it  implies that when an index occurs more than once in the same expression in upper and in.. postitions..,
 the expression is implicitly summed over all possible values for that index.
  Sometimes it is called dummy indices summation rule.


    \subsection {Riemannian manifold---manifold equipped with Riemannian metric}

{\bf Definition} The Riemannian manifold is a manifold equipped with a Riemannian metric.




  The Riemannian metric on the manifold $M$ defines the
  length of the tangent vectors and the length of the curves.

{\bf Definition}
  Riemannian metric $G$ on n-dimensional manifold $M^n$
  defines for every point $\pt\in M$ the scalar product
  of tangent vectors in the tangent space $T_\pt M$
  smoothly depending on the point $\pt $.

  It means that in every coordinate system $(x^1,\dots,x^n)$
  a metric $G=g_{ik}dx^idx^k$ is defined by a matrix valued smooth function $g_{ik}(x)$ ($i=1,\dots,n;k=1,\dots n$)
  such that for any two vectors
       $$
  {\bf A}=A^i(x){\p\over \p x^i},\,\, {\bf B}=B^i(x){\p\over \p x^i},
      $$
tangent to the manifold $M$ at the point $\pt$ with coordinates $x=(x^1,x^2,\dots,x^n)$ ($\A,{\bf B}\in T_{\pt}M$)
the scalar product is equal to:
              $$
              \langle\A,{\bf B}\rangle_G\big\vert_\pt= G({\bf A},{\bf B})\big\vert_\pt=
A^i(x)g_{ik}(x)B^k(x)=
            $$
            \begin{equation}\label{scalarproduct}
  \begin{pmatrix}
   A^1 \dots A^n\\
   \end{pmatrix}
  \begin{pmatrix}
     g_{11}(x)&\dots &g_{1n}(x)\\
      \dots &\dots & \dots \\
         g_{n1}(x)&\dots &g_{nn}(x)\\  \\
   \end{pmatrix}
\begin{pmatrix}
   B^1\\
     \cdot \\
   \cdot\\
   \cdot\\
   B^n\\
   \end{pmatrix}
\end{equation}

where
\begin{itemize}

  \item  $G({\bf A},{\bf B})=G({\bf B},{\bf A})$, i.e.  $g_{ik}(x)=g_{ki}(x)$ (symmetricity condition)

    \item
       $G({\bf A},{\bf A})>0$ if $\bf A\not=0$, i.e.

    $g_{ik}(x)u^iu^k\geq 0$, $g_{ik}(x)u^iu^k=0$ iff $u^1=\dots=u^n=0$  (positive-definiteness)

   \item  $G({\bf A},{\bf B})\big\vert_{\pt=x}$, i.e. $g_{ik}(x)$ are smooth functions.


\end{itemize}


{\it One can say that Riemannian metric is defined by  symmetric covariant smooth tensor field $G$ of the rank 2
which defines scalar product in the tangent spaces $T_{\pt}M$ smoothly depending on the point $\pt$.
 Components of tensor field  $G$ in coordinate system are matrix valued functions $g_{ik}(x)$}:
\begin{equation}\label{symb}
    G=g_{ik}(x)dx^i\otimes dx^k\,.
\end{equation}

  \m

 The matrix $||g_{ik}||$ of components of the metric $G$ we also sometimes denote by $G$.

    {\it Rule of transformation for entries of matrix $g_{ik}(x)$}

  $g_{ik}(x)$-entries of the matrix $||g_{ik}||$  are components of tensor field $G$ in a given coordinate system.

    How do these components transform under transformation of coordinates $\{x^{i}\}\mapsto \{x^{i'}\}$?
    \begin{equation*}
        G=g_{ik}dx^i\otimes dx^k=g_{ik}
        \left({\p x^i\over \p x^{i'}}dx^{i'}\right)
        \otimes
        \left({\p x^k\over \p x^{k'}}dx^{k'}\right)=
    \end{equation*}
    \begin{equation*}\label{ruleoftransformation}
         {\p x^i\over \p x^{i'}}g_{ik}{\p x^k\over \p x^{k'}}
        dx^{i'}
        \otimes dx^{k'}=
        g_{i'k'}
        dx^{i'}
        \otimes dx^{k'}
            \end{equation*}
    Hence
    \begin{equation}\label{transformationlaw}
     g_{i'k'}={\p x^i\over \p x^{i'}}g_{ik}{\p x^k\over \p x^{k'}}\,.
       \end{equation}
     One can derive transformations formulae also
     using general formulae \eqref{ruleoftransformfor 2-tensor} for tensors.



Important remark

     \begin{equation}\label{imporatantremark2}
 g_{ik}=\left\langle{{\p \over \p x^i},{\p\over \p x^k}}\right\rangle
     \end{equation}
(See \eqref{scalarproduct}).

Later by some abuse of notations we sometimes omit the sign of tensor product
and write a metric just as
\begin{equation*}\label{symb1}
    G=g_{ik}(x)dx^idx^k\,.
\end{equation*}


{\bf Examples}









   \begin{itemize}
\item


   $\R^n$ with canonical coordinates $\{x^i\}$ and with metric
               $$
           G=(dx^1)^2+(dx^2)^2+\dots+(dx^n)^2
               $$
           $G=||g_{ik}||=\diag [1,1,\dots,1]$

Recall that this is a basis example of $n$-dimensional Euclidean space, where scalar product
is defined by the formula:
              $$
     G(\X,\Y)=\la\X,\Y\ra=g_{ik}X^iY^k=X^1Y^1+X^2Y^2+\dots+X^nY^n\,.
              $$
In the general case if $G=||g_{ik}||$ is an arbitrary symmetric positive-definite metric then
                $$
            G(\X,\Y)=\la\X,\Y\ra=g_{ik}X^iY^k\,.
                $$
One can show that there exist a new basis $\{\e_i\}$ such that in this basis
              $$
              G(\e_i,\e_k)=\delta_{ik}\,.
               $$
This basis is called orthonormal basis. (See the Lecture notes in Geometry)

Scalar product in vector space defines the {\it same} scalar product at all the points. In general case
for Riemannian manifold scalar product depends on the point.


\item  $\R^2$ with polar coordinates in the domain $y>0$ ($x=r\cos\varphi, y=r\sin\varphi$):

   $dx=\cos\varphi dr-r\sin\varphi d\varphi, dy=\sin\varphi dr+r\cos\varphi d\varphi$.
  In new coordinates the Riemannian metric $G=dx^2+dy^2$ will have the following appearance:
             $$
        G=(dx)^2+(dy)^2=(\cos\varphi dr-r\sin\varphi d\varphi)^2+(\sin\varphi dr+r\cos\varphi d\varphi)^2=
         dr^2+r^2(d\varphi)^2
            $$
   We see that for matrix    $G=L||g_{ik}||$
             \begin{equation*}
                 \underbrace{G=
           \begin{pmatrix}        g_{xx} &g_{xy}\cr  g_{yx} &g_{yy} \end{pmatrix}=
      \begin{pmatrix}    1 &0\cr  0 &1\cr  \end{pmatrix}}_{\hbox{in cartesian coordinates}},\qquad
                 \underbrace
                 {G=
      \begin{pmatrix}g_{rr} &g_{r\varphi}\cr g_{\varphi r} &g_{\varphi\varphi}\end{pmatrix}=
      \begin{pmatrix}1 &0\cr 0 &r\end{pmatrix}}_ {\hbox{in polar coordinates}}
             \end{equation*}



    \item   Circle

     Interval $[0,2\pi)$ in the line $0\leq x< 2\pi$ with Riemannian  metric
           \begin{equation}\label{circle1}
          G=  a^2dx^2
           \end{equation}
Renaming $x\mapsto \varphi $ we come to habitual formula for
metric
for circle of the radius $a$: $x^2+y^2=a^2$ embedded in the Euclidean space $\E^2$:
           \begin{equation}\label{circle2}
          G= a^2d\varphi^2\qquad
          \begin{cases}
          x=a\cos\varphi\cr
          y=a\sin\varphi
          \end {cases},
          0\leq \varphi <2\pi,
           \end{equation}




    \item Cylinder surface

     Domain in $\R^2$ $D=\{(x,y)\colon,\,\, 0\leq x< 2\pi$ with Riemannian  metric
           \begin{equation}\label{domainascylinder1}
          G=  a^2dx^2+dy^2
           \end{equation}
We see that renaming variables $x\mapsto \varphi $, $y\mapsto h$ we come to habitual, familiar formulae for
metric in standard polar coordinates
for cylinder surface of the radius $a$ embedded in the Euclidean space $\E^3$:
           \begin{equation}\label{domainascylinder2}
          G= a^2d\varphi^2+dh^2\qquad
          \begin{cases}
          x=a\cos\varphi\cr
          y=a\sin\varphi\cr
          z=h\cr
          \end {cases},
          0\leq \varphi <2\pi, -\infty<h<\infty
           \end{equation}


 \item  Sphere

   Domain in $\R^2$,  $0<x<2\pi$, $0<y<\pi$ with metric $G=dy^2+\sin^2 y dx^2$
We see that renaming variables $x\mapsto \varphi $, $y\mapsto h$ we come to habitual, familiar formulae for
metric in standard spherical coordinates
for sphere $x^2+y^2+z^2=a^2$ of the radius $a$ embedded in the Euclidean space $\E^3$:
           \begin{equation}\label{domainascylinder2}
          G= a^2d\theta^2+a^2\sin^2\theta
          d\varphi^2\qquad
          \begin{cases}
          x=a\sin\theta\cos\varphi\cr
          y=a\sin\theta\sin\varphi
          z=a\cos\theta
          \end {cases},
          0\leq \varphi <2\pi, -\infty<h<\infty
           \end{equation}


\end{itemize}



\subsection {Length of the curve}


Let $\gamma\colon\,\, x^i=x^i(t), (i=1,\dots,n))$
 $(a\leq t\leq b)$ be a curve on the Riemannian manifold $(M,G)$.

  At the every point of the curve the velocity vector (tangent vector)
  is defined:
\begin{equation}\label{velvector}
  \v(t)=\begin{pmatrix}
       \dot x^1 (t)\\
             \cdot\\
             \cdot\\
             \cdot\\
             \dot x^n(t)
         \end{pmatrix}
\end{equation}

The length of velocity vector $\v\in T_xM$ (vector $\v$ is tangent to the manifold $M$ at the point $x$)
equals to
     \begin{equation}\label{speedforaunt}
       |\v|_x=\sqrt {\la \v,\v\ra_G\big\vert_{x}}=
       \sqrt{g_{ik}v^iv^k}\big\vert_{x}=
       \sqrt{g_{ik}{dx^i(t)\over dt}{dx^k(t)\over dt}_x}\big\vert_{x}
     \end{equation}

  The length of
 the curve is defined by the integral of the length of velocity vector:
\begin{equation}\label{lengthofthecurve}
  L_\gamma=\int_a^b \sqrt {\langle\v,\v\rangle_G\big\vert_{x(t)}}dt=
  \int_a^b \sqrt {g_{ik}(x(t))\dot x^i(t) \dot x^k(t)}dt
\end{equation}

Bearing in mind that metric \eqref{symb} defines the length
we often write metric in the following form
\begin{equation}\label{metric}
  ds^2=g_{ik}dx^idx^k
\end{equation}

       \bigskip

       For example consider $2$-dimensional Riemannian manifold with metric
                    $$
                 ||g_{ik}(u,v)||=
                     \begin{pmatrix}
                     g_{11}(u,v) &g_{12}(u,v)\cr
                     g_{21}(u,v) &g_{22}(u,v)\cr
                    \end{pmatrix}\,.
                    $$
        Then
                           $$
            G=ds^2=g_{ik}du^idv^k=g_{11}(u,v)du^2+2g_{12}(u,v)dudv+g_{22}(u,v)dv^2
                         $$
                The length of the curve $\gamma\colon u=u(t),v=v(t)$, where $t_0\leq t\leq t_1$
                according to \eqref{lengthofthecurve} is equal to
\begin{equation}\label{lengthofthecurve4}
  L_\gamma=\int_{t_0}^{t_1} \sqrt {\langle\v,\v\rangle}=
  \int_{t_0}^{t_1} \sqrt {g_{ik}(x)\dot x^i \dot x^k}=
  \end{equation}
  \begin{equation}
\int_{t_0}^{t_1}
\sqrt {{g_{11}}\left(u\left(t\right),v\left(t\right)\right)u_t^2
+2{g_{12}}\left(u\left(t\right),v\left(t\right)\right)u_tv_t+
{g_{22}}\left(u\left(t\right),v\left(t\right)\right)v_t^2}dt
\end{equation}

The length of the curve defined by the formula\eqref{lengthofthecurve}
sure obeys the natural conditions

\begin{itemize}

\item It coincides with the usual length in the Euclidean space $\E^n$
($\R^n$ with standard metric  $G=(dx^1)^2+\dots+(dx^n)^2$
in cartesian coordinates). E.g. for $3$-dimensional Euclidean space
\begin{equation}\label{coincidewithlength}
      L_\gamma=
  \int_a^b \sqrt {g_{ik}(x(t))\dot x^i(t) \dot x^k(t)}dt=
  \int_a^b \sqrt{(\dot x^1(t))^2+(\dot x^2(t))^2+(\dot x^3(t))^2}dt
\end{equation}


\item It does not depend on parameterisation of the curve

\begin{equation}\label{independenceon parameterisation}
    L_\gamma=  \int_a^b \sqrt {g_{ik}(x(t))\dot x^i(t) \dot x^k(t)}dt=
        \int_{a'}^{b'} \sqrt {g_{ik}(x(\tau))\dot x^i(\tau) \dot x^k(\tau)}d\tau,
\end{equation}
where $x^i(\tau)=x^i(t(\tau))$, $a'\leq \tau\leq b'$ while $a\leq t\leq b$.


\item It {\it does not depend on coordinates on Riemannian manifold $M$}
 \begin{equation}\label{independence on coordinates}
    L_\gamma=  \int_a^b \sqrt {g_{ik}(x(t))\dot x^i(t) \dot x^k(t)}dt=
    \int_a^b \sqrt {g_{i'k'}(x'(t))\dot x^{i'}(t) \dot x^{k'}(t)}dt
 \end{equation}

\item  It is additive: if a curve  $\gamma=\gamma_1+\gamma$, i.e. $\gamma\colon x^i(t), a\leq t\leq b$,
$\gamma_1\colon x^i(t), a\leq t\leq c$ and $\gamma_2\colon x^i(t), c\leq t\leq b$ where
a point $c$ belongs to the interval $(a,b)$ then
                 $$
              L_{\gamma}=L_{\gamma^1}+L_{\gamma^2},\,\, {\rm i.e.}
                 $$
                 $$
                 \int_a^b \sqrt {g_{ik}(x(t))\dot x^i(t) \dot x^k(t)}dt=
                 $$
             \begin{equation}\label{additive}
             \int_a^c \sqrt {g_{ik}(x(t))\dot x^i(t) \dot x^k(t)}dt+
             \int_c^b \sqrt {g_{ik}(x(t))\dot x^i(t) \dot x^k(t)}dt
             \end{equation}
             \end{itemize}

Conditions \eqref{coincidewithlength} and \eqref{additive} evidently are obeyed.

Condition \eqref{independenceon parameterisation} follows from the fact that
  $$
  \dot x^i(\tau)={dx(t(\tau))\over d\tau}=
  {dx(t(\tau))\over d t}{dt\over d\tau}=
  \dot x^i(t){dt\over d\tau}\,.
  $$
Bearing in mind the above formula we have
           $$
        \int_{a'}^{b'} \sqrt {g_{ik}(x(\tau))\dot x^i(\tau) \dot x^k(\tau)}d\tau=
        \int_{a'}^{b'} \sqrt {g_{ik}(x(t(\tau)))\dot x^i(t) \dot x^k(t)\left({dt\over d\tau}^2\right)}d\tau=
           $$
           $$
           \int_{a'}^{b'} \sqrt {g_{ik}(x(t(\tau)))\dot x^i(t) \dot x^k(t)}\left|{dt\over d\tau}\right|d\tau=
           \int_a^b \sqrt {g_{ik}(x(t))\dot x^i(t) \dot x^k(t)}dt=L_\gamma
           $$
Condition \eqref{independence on coordinates} follows from the condition \eqref{transformationlaw}:
  $$
  \int_a^b \sqrt {g_{i'k'}(x'(t))\dot x^{i'}(t) \dot x^{k'}(t)}dt=
  \int_a^b \sqrt {g_{ik}(x(t)){\p x^i\over \p x^{i'}}{\p x^k\over \p x^{k'}}\dot x^{i'}(t) \dot x^{k'}(t)}dt=
   $$
   \begin{equation}\label{transformlaw2}
     \int_a^b \sqrt {g_{ik}(x(t))
     \left({\p x^i\over \p x^{i'}}\dot x^{i'}(t)\right)\left({\p x^k\over \p x^{k'}}\dot x^{k'}(t)\right) }dt=
     \int_a^b \sqrt {g_{ik}(x(t))\dot x^{i}(t)\dot x^{k}(t) }dt\,.
   \end{equation}


\subsection{Riemannian structure on the surfaces embedded in Euclidean space}

Let $M$ be a surface embedded in Euclidean space. Let $G$ be Riemannian structure on the manifold $M$.

  Let $\X, \Y$ be two vectors tangent to the surface
$M$ at a point $\pt\in M$. An External Observer calculate this scalar product viewing
these two vectors as vectors in $\E^3$ attached at the point $\pt\in \E^3$
using scalar product in   $\E^3$.  An Internal Observer will calculate the scalar product
viewing these two vectors as vectors  tangent to the surface $M$
using the Riemannian
metric $G$ (see the formula \eqref{scalarproduct}).  Respectively





If $L$ is a curve in $M$ then an External Observer consider this curve as a curve in $\E^3$,
calculate the modulus of velocity  vector (speed) and the length of the curve using Euclidean scalar
product of ambient space. An Internal Observer ("an aunt") will define the modulus of the velocity vector and
the length of the curve using Riemannian metric.

\subsubsection {Induced Riemannian metric}

 {\bf Definition}  We say that Riemannian metric on the embedded surface is induced by the Euclidean structure of the ambient space
    if External and Internal Observers come to the same results calculating scalar product of tangent vectors.

    In this case modulus of velocity vector (speed) and the length of the curve is the same for
    External and Internal observer.



    Let $\r=\r(u,v)$ be parameterisation of the surface $M$ embedded in the Euclidean space:
                           \begin{equation*}
                            \r(u,v)=\begin{pmatrix}x(u,v)\cr
                              y(u,v)\cr
                              z(u,v)\cr
                              \end{pmatrix}
                           \end{equation*}
       Tangent vectors at any point $(u,v)$
   are
              \begin{equation*}
                \r_u={\p \r(u,v)\over \p u}=\begin{pmatrix}
                              {\p x(u,v)\over \p u}\cr
                              {\p y(u,v)\over \p u}\cr
                              {\p z(u,v)\over \p u}\cr
                              \end{pmatrix}=
                              {\p x(u,v)\over \p u}{\p\over \p x}+
                              {\p y(u,v)\over \p u}{\p\over \p y}+
                              {\p z(u,v)\over \p u}{\p\over \p z}
              \end{equation*}
Here as always $x,y,z$ are Cartesian coordinates in $\E^3$. We denote by
${\p\over \p x},{\p\over \p y},{\p\over \p z}$ basic vectors $\e_x,\e_y,\e_z$.  Scalar product
in $\E^3$ is defined as usual
\begin{equation*}
    \la\e_i,\e_j\ra=\delta_{ij}\,,\qquad\hbox{(orthonormal basis)},
    \end{equation*}
\begin{equation}\label{scalarproductinE^3}
    \la\X,\Y\ra=\la X^1\e_x+X^2\e_y+X^3\e_z, Y^1\e_x+Y^2\e_y+Y^3\e_z\ra=
    X^1Y^1+X^2Y^2+X^3Y^3\,.
\end{equation}


  {\bf Remark}   It is convenient sometimes to denote
    parameters $(u,v)$ as $(u^1,u^2)$ or $u^\a$ ($\a=1,2$)
   and to write $\r=\r(u^1,u^2)$ or $\r=\r(u^\a)$ ($\a=1,2$)
   instead $\r=\r(u,v)$

For example consider surface defined by the equation $z-F(x,y)=0$.
It can be parameterised:
\begin{equation}\label{generalsurface}
  \r(u,v)=
  \begin{pmatrix}
  x(u,v)\\
  y(u,v)\\
  z(u,v)\\
  \end{pmatrix}=
\begin{pmatrix}
       u\\
        v\\
       F(u,v)\\
  \end{pmatrix}
  \end{equation}
  \begin{equation}
       \r_u =
\begin{pmatrix}
                  1\\
                   0\\
                   F_u(u,v)\\
  \end{pmatrix},
  \quad
  \r_v=
\begin{pmatrix}
            0\\
           1\\
            F_v(u,v)\\
  \end{pmatrix}
\end{equation}

{\bf Example } Consider sphere of radius $a$ ($a>0$):
\begin{equation}\label{generalsurface}
  \r(\theta,\varphi)=
  \begin{pmatrix}
       x(\theta,\varphi)\\
        y(\theta,\varphi)\\
       z(\theta,\varphi)\\
  \end{pmatrix}
  =
  \begin{pmatrix}
  a\sin\theta\cos\varphi\\
   a\sin\theta\sin\varphi\\
  a\cos\theta\\
  \end{pmatrix}\,.
  \end{equation}


  \begin{equation}
       \r_\theta =
\begin{pmatrix}
a\cos\theta\cos\varphi\\
   a\cos\theta\sin\varphi\\
  -a\sin\theta\\
\end{pmatrix},
  \quad
  \r_\varphi=
\begin{pmatrix}
-R\sin\theta\sin\varphi\\
   R\sin\theta\cos\varphi\\
         0\\
  \end{pmatrix}
\end{equation}





Recall the description of tangent vectors by External and Internal Observers.

\subsubsection {Internal and external coordinates of tangent vector}


\centerline {\it Tangent plane}

Let $\pt$ be an arbitrary point on the surface $M$.
Consider the plane formed by the vectors which are adjusted to the point $\pt$
and tangent to the surface $M$. We call this plane
{\it plane tangent to $M$ at the point $\pt$ } and denote it by $T_\pt M$.

Let $\r=\r(u,v)$ be a parameterisation of the surface $M$. For a point  $\pt\in M$ one can consider a basis
in the tangent plane $T_pM$ adjusted to the parameters $u,v$. Every vector $\X\in T_pM$
can be expanded over this basis:
\begin{equation}\label{expansion}
  \X=X_u\r_u+X_v\r_v,
\end{equation}
where $X_u, X_v$ are coefficients, components of the vector $\X$.

Internal Observer views  the basis vector $\r_u\in T_pM$, as a  velocity vector
for the curve $u=u_0+t,v=v_0$, where $(u_0,v_0)$ are coordinates of the point $p$.
Respectively the basis vector $\r_v\in T_pM$ for an Internal Observer, is velocity vector
for the curve $u=u_0,v=v_0+t$, where $(u_0,v_0)$ are coordinates of the point $p$.

Let $\r=\r(t)$ be a curve belonging to the surface $C$, which passes through the point $\pt$,
$\r(t)=\r(u(t),v(t))$ and $\pt=\r(t_0)$. Then vector
\begin{equation}\label{tangenttocurveonthesurface}
  \r_t={d\r\over dt}={d\r(u(t),v(t))\over dt}
\end{equation}
belongs to the tangent plane $T_pM$.


Note that for the vector \eqref{tangenttocurveonthesurface}
components $X_u, X_v$ are equal to  $X_u=u_t, X_v=v_t$  because
\begin{equation}\label{tangenttocurveonthesurface2}
  \r_t={d\r\over dt}={d\r(u(t),v(t))\over dt}=u_t\r_u+v_t\r_v
\end{equation}
  An  External Observer describes the vector $\r_t$ as a vector in $\E^3$
attached at the point $\pt$.
The Internal Observer describes this vector
as a vector which has components
$(u_t,v_t)$ in the basis $\r_u,\r_v$ according to the formula
\eqref{tangenttocurveonthesurface2}.

In general consider an arbitrary tangent vector $\X\in T_\pt M$.
Denote $X_u=a$, $X_v=b$

\begin{equation}\label{expansion}
  \X=X_u\r_u+X_v\r_v=a\r_u+b\r_v
         =a
\begin{pmatrix}
              x_u(u,v)\\
               y_u(u,v)\\
                z_u(u,v)\\
  \end{pmatrix}+b
\begin{pmatrix}
         x_v(u,v)\\
           x_v(u,v)\\
           x_v(u,v)\\
  \end{pmatrix}=
\begin{pmatrix}
              ax_u(u,v)+bx_v(u,v)\\
               ay_u(u,v)+by_v(u,v)\\
                az_u(u,v)+bz_v(u,v)\\
  \end{pmatrix}
\end{equation}
The last column in this formula represents the three components of the vector $\X$ in the ambient space.

The pair $(a,b)$ can be considered as {\it internal coordinates of the tangent vector $\X$}.
An Internal Observer, Ant living on the surface, deals with the vector $\X$ in terms of coordinates $(a,b)$.
External observer which contemplates the surface embedded in three-dimensional ambient space
 deals with vector $\X$ as with vector with three external coordinates
 (see the last right column in the formula \eqref{expansion}.)





\m

 In condensed notation instead denoting
 coordinates by $(u,v)$
we often denote them by
$u^\a=(u^1,u^2)$. Respectively we denote by
     $$
     \r_\a={d\r\over du^\a},\quad  \r_u=\r_1, \r_v=\r_2
     $$



 The formula \eqref{expansion} for
tangent vector field will have the following appearance:
\begin{equation}\label{expansioncondensed}
  \X=X^\a\r_\a=X^1\r_1+X^2\r_2,\quad (X^1=X_u, X^2=X_v)
\end{equation}



\medskip


When using condensed notations we usually omit explicit summation symbols.
E.g. we write $u^\a\r_\a$ instead $\sum_{i=1}^2 u^\a\r_\a$ or $u^1\r_1+u^2\r_2$

One can consider also differentials $du^\a=(du^1,du^2)$:
\begin{equation}\label{difonvector}
  du^\a(\r_\beta)=\delta^\a_\beta\colon\quad
  du^1(\r_1)=du^2(\r_2)=1,\,du^1(\r_2)=du^2(\r_1)=0
\end{equation}




  \subsubsection {Explicit formulae for induced Riemannian metric (First Quadratic form)}

Now we are ready to define the Riemannian metric on the surface induced by metric (scalar product)
in ambient Euclidean space.

     Let $M\colon \r=\r(u,v)$ be a surface embedded in $\E^3$.  Then
     since the scalar product of vectors $\r_u,\r_v$ has to be the same in the ambient space
     and in the Riemannian metric $G$ we have due to  \eqref{imporatantremark2} that

\begin{equation*}
||g||_M=
\begin{pmatrix}
   g_{11} & g_{12} \\
   g_{12}& g_{22} \\
   \end{pmatrix}=
   \begin{pmatrix}
   \la\r_u,\r_u\ra & \la\r_u,\r_v\ra \\l
   \la\r_u,\r_v\ra & \la\r_v,\r_v\ra \\
   \end{pmatrix},\quad g_{\a\beta}=\la\r_\a,\r_\beta\ra\,,
\end{equation*}
             \begin{equation}\label{firstquadraticform}
          G_M=g_{\a\beta}du^\a du^\beta=g_{11}du^2+2g_{12}dudv+g_{22}dv^2
             \end{equation}

where $(\,,\,)$ is a scalar product in Euclidean space.


The formula \eqref{firstquadraticform} is the formula for induced Riemannian metric on the surface---First Quadratic Form.

  If ${\X,\Y}$ are two tangent vectors in the tangent plane $T_pC$ then $G(\X,\Y)$
  at the point $p$ is equal to scalar product of vectors $\X,\Y$:
\begin{equation}\label{scalarproduct}
(\X,\Y)=(X^1\r_1+X^2\r_2, Y^1\r_1+Y^2\r_2)=
\end{equation}
                $$
X^1 (\r_1,\r_1)Y^1+X^1 (\r_1,\r_2)Y^2+X^2 (\r_2,\r_1)Y^1+X^2 (\r_2,\r_2)Y^2=
$$
$$X^\a (\r_\a,\r_\beta)Y^\beta=
X^\a g_{\a\beta}Y^\beta=G(\X,\Y)
$$





We can come to this formula just transforming differentials.
 In carteisan coordinates $\la\X,\Y\ra=X^1Y^1+X^2Y^2+X^3Y^3$,
i.e. the  Euclidean metric in cartesian coordinates is given by
           \begin{equation}\label{RiemEuclid}
            G_{\E^3}=(dx)^2+(dy)^2+(dz)^2\,.
           \end{equation}
The condition that Riemannian metric \eqref{firstquadraticform} is induced by Euclidean scalar product means
that

       \begin{equation}\label{intermsofdifferentials1}
        G_{\E^3}\big\vert_{\r=\r(u,v)}=\left((dx)^2+(dy)^2+(dz)^2\right)\big\vert_{\r=\r(u,v)}=
        G_M=g_{\a\beta}du^\a du^\beta
       \end{equation}
i.e.  $\left((dx)^2+(dy)^2+(dz)^2\right)\big\vert_{\r=\r(u,v)}=$
          $$
\left({\p x(u,v)\over \p u}du+{\p x(u,v)\over \p u}du\right)^2+
\left({\p x(u,v)\over \p u}du+{\p x(u,v)\over \p u}du\right)^2+
\left({\p x(u,v)\over \p u}du+{\p x(u,v)\over \p u}du\right)^2=
          $$
\begin{equation}\label{firstquadraticform1}
G_M=g_{\a\beta}du^\a du^\beta=g_{11}du^2+2g_{12}dudv+g_{22}dv^2=
\end{equation}
We come to the same formula.

(See the examples of calculations in the next subsection.)

Check explicitly again that  length of the tangent vectors
and of the curves calculating by External observer (i.e. using Euclidean metric \eqref{RiemEuclid})
{\it is the same} as calculating by Internal Observer, aunt (i.e. using the induced Riemannian metric \eqref{firstquadraticform})


  Consider a vector $\X=X^\a\r_\a=a\r_u+b\r_v$ tangent to the surface $M$.

  Calculate its length by External and Internal Observer.

  The square of the length $|\X|$ of this vector calculated by External observer
  (he calculates using the scalar product in $\E^3$) equals to
\begin{equation}\label{lengthforexternalobserver}
 |\X|^2=\la\X,\X\ra=\la\r_u+b\r_v,a\r_u+b\r_v\ra=a^2\la\r_u,\r_u\ra+
   2ab\la\r_u,\r_v\ra+b^2\la\r_v,\r_v\ra
\end{equation}
where $\la\,,\,\ra$ is a scalar product in $\E^3$.

The internal observer will calculate the length using Riemannian metric \eqref{firstquadraticform}:
\begin{equation}\label{thevalueoffirstquadraticform}
  G(\X,\X)=
  \begin{pmatrix}
   a, &b
  \end{pmatrix}
   \cdot
   \begin{pmatrix}
   g_{11} & g_{12} \\
   g_{21}& G_{22} \\
   \end{pmatrix}\cdot
  \begin{pmatrix}
   a\\ b\\
  \end{pmatrix}=
  g_{11}a^2+2g_{12}ab+g_{22}b^2
\end{equation}

{\it External observer (person living in ambient space $\E^3$) calculate the length
of the tangent vector  using formula
\eqref{lengthforexternalobserver}. An ant living on the surface
calculate length of this vector in internal coordinates using formula
\eqref{thevalueoffirstquadraticform}. External observer deals
with external coordinates of the vector, ant on the surface with internal coordinates.
 They come to the same answer.}


\smallskip



  Let $\r(t)=\r(u(t),v(t))$ $a\leq t\leq b$ be a curve on the surface.


  Velocity of this curve at the  point $\r(u(t),v(t))$ is equal to
  $$
  \v=\X=\xi \r_u+\eta\r_v
  \hbox
  {where
  $\xi=u_t,\eta=v_t\colon\quad \v={d\r(t)\over dt}=
    u_t\r_u+v_t\r_v$}
  \,.
  $$


The length of the curve is equal to
\begin{equation}\label{lengthofthecurveexternal}
L=\int_a^b |\v(t)|d t= \int_a^b\sqrt
{\la \v(t),\v(t)\ra_{\E^3}}d t = \int_a^b\sqrt
{\la u_t\r_u+v_t\r_v,u_t\r_u+v_t\r_v\ra_{\E^3}}dt=
\end{equation}
           $$
\int_a^b\sqrt {\la\r_u,\r_u\ra_{\E^3}u_t^2+2\la\r_u,\r_v\ra_{\E^3}u_tv_t+\la\r_v,\r_v\ra_{\E^3}v_t^2}d\tau=
           $$
\begin{equation}\label{lengthofthecurveinternal}
  \int_a^b\sqrt {g_{11} u_t^2+2g_{12}u_tv_t+g_{22}v_t^2}dt
\end{equation}

{\it An external observer will calculate the length of the curve using
\eqref{lengthofthecurveexternal}.  An ant living on the surface calculate
length of the curve using
\eqref{lengthofthecurveinternal} using  Riemannian
metric on the surface:
\begin{equation}\label{riem}
  ds^2=g_{ik}du^idu^k=g_{11}du^2+2g_{12}dudv+g_{22}dv^2
\end{equation}
 They will come to the same answer.}
\smallskip

\subsubsection{Induced Riemannian metrics. Examples.}

We consider here examples of calculating induced Riemanian metric
on some quadratic surfaces in $\E^3$.
using calculations for tangent vectors (see \eqref{firstquadraticform}) or
explicitly in terms of differentials (see \eqref{intermsofdifferentials1} and \eqref{firstquadraticform1}).




\m

First of all  consider  the general case when
 a surface $M$ is defined by the
 equation $z-F(x,y)=0$. One can consider the following parameterisation
 of this surface:
\begin{equation}\label{surface}
  \r(u,v)\colon\quad
  \begin{cases}
  x=u\\
  y=v\\
  z=F(u,v)
  \end{cases}
\end{equation}

Then

  \begin{equation}\label{c}
  \r_u=\begin{pmatrix}
        1\\
        0\\
        F_u\\
   \end{pmatrix}
\quad
  \r_v=\begin{pmatrix}
        0\\
        1\\
        F_v\\
   \end{pmatrix}
 \end{equation},
            $$
     (\r_u,\r_u)=1+F_u^2,\quad
     (\r_u,\r_v)=F_uF_v,\quad
     (\r_v,\r_v)=1+F_v^2
            $$
and induced Riemannina metric (first quadratic form) \eqref{firstquadraticform} is equal to
\begin{equation}\label{formula forfirstform}
   ||g_{\a\beta}||=
\begin{pmatrix}
   g_{11} & g_{12} \\
   g_{12}& g_{22} \\
   \end{pmatrix}=
   \begin{pmatrix}
   (\r_u,\r_u) & (\r_u,\r_v) \\
   (\r_u,\r_v) & (\r_v,\r_v) \\
   \end{pmatrix}=   \begin{pmatrix}
   1+F_u^2 & F_uF_v \\
   F_uF_v& 1+F_v^2 \\
   \end{pmatrix}
\end{equation}

\begin{equation}\label{formula forfirstformgeneral}
   G_M=ds^2=(1+F_u^2)du^2+2F_uF_vdudv+(1+F_v^2)dv^2
\end{equation}
  and the length of the curve $\r(t)=\r (u(t), v(t))$ on $C$  $(a\leq t\leq b)$
  can be calculated by the formula:
               \begin{equation*}
             L=\int
             \int_a^b\sqrt{(1+F_u^2)u_t^2+2F_uF_vu_tv_t+(1+F_v)^2v^2_t}dt
               \end{equation*}
One can calculate \eqref{formula forfirstformgeneral} explicitly using \eqref{intermsofdifferentials1}:
                     $$
                                    G_M=\left(dx^2+dy^2+dz^2\right)\big\vert_{x=u,y=v,z=F(u,v)}=
               (du)^2+(dv)^2+(F_udu+F_vdv)^2=
                     $$
               \begin{equation}\label{formula forfirstformgeneral}
=(1+F_u^2)du^2+2F_uF_vdudv+(1+F_v^2)dv^2\,.
               \end{equation}



\medskip

         \medskip


       \centerline  {\it Cylinder}


  Cylinder is given by the equation $x^2+y^2=a^2$. One can consider the following
parameterisation
 of this surface:
\begin{equation}\label{surface1}
  \r(h,\varphi)\colon\quad
  \begin{cases}
  x=a\cos\varphi\\
  y=a\sin\varphi\\
  z=h\\
  \end{cases}
\end{equation}

\medskip

We have   $G_{cylinder}=\left(dx^2+dy^2+dz^2\right)\big\vert_{x=a\cos\varphi,y=a\sin\varphi,z=h}=$
        \begin{equation}\label{firtsquadraticformcylinder(diff)}
               =(-a\sin\varphi d\varphi)^2+(a\cos\varphi d\varphi)^2+dh^2=a^2d\varphi^2+dh^2
        \end{equation}

The same formula in terms of scalar product of tangent vectors:



  \begin{equation}\label{cyl1}
  \r_h=\begin{pmatrix}
        0\\
        0\\
        1\\
   \end{pmatrix}
\quad
  \r_\varphi=\begin{pmatrix}
        -a\sin\varphi\\
        a\cos\varphi\\
          0\\
   \end{pmatrix}
 \end{equation},
            $$
     (\r_h,\r_h)=1,\quad
     (\r_h,\r_\varphi)=0,\quad
     (\r_\varphi,\r_\varphi)=a^2
            $$
and
\begin{equation*}\label{formula forfirstformcyl}
||g_{\a\beta}||=
   \begin{pmatrix}
   (\r_u,\r_u) & (\r_u,\r_v) \\
   (\r_u,\r_v) & (\r_v,\r_v) \\
   \end{pmatrix}=   \begin{pmatrix}
   1 & 0 \\
   0& a^2 \\
   \end{pmatrix}\,,
\end{equation*}
\begin{equation}\label{formula forfirstformcyl}
   G=dh^2+a^2d\varphi^2
\end{equation}
  and the length of the curve $\r(t)=\r (h(t), \varphi(t))$ on the cylinder
    $(a\leq t\leq b)$
  can be calculated by the formula:
               \begin{equation}
             L=
             \int_a^b\sqrt{h_t^2+a^2\varphi_t}dt
               \end{equation}



\medskip

  \centerline {\it Cone}
 Cone is given by the equation $x^2+y^2-k^2z^2=0$. One can consider the following
parameterisation
 of this surface:
\begin{equation}\label{surfacecone}
  \r(h,\varphi)\colon\quad
  \begin{cases}
  x=kh\cos\varphi\\
  y=kh\sin\varphi\\
  z=h\\
  \end{cases}
\end{equation}

\medskip

   Calculate induced Riemannian metric:

We have               $$
             G_{conus}=\left(dx^2+dy^2+dz^2\right)\big\vert_{x=kh\cos\varphi,y=kh\sin\varphi,z=h}=
                      $$
                      $$
                      (k\cos\varphi dh-kh\sin\varphi d\varphi)^2+
             (k\sin\varphi dh+kh\cos\varphi d\varphi)^2+dh^2
                      $$
        \begin{equation}\label{firtsquadraticformforconus(diff)}
            G_{conus} =k^2h^2d\varphi^2+(1+k^2)dh^2,\,\,
                        ||g_{\a\beta}||=
   \begin{pmatrix}
   1+k^2 & 0 \\
   0&  k^2h^2 \\
   \end{pmatrix}
                       \end{equation}
The length of the curve $\r(t)=\r (h(t), \varphi(t))$ on the
  cone
    $(a\leq t\leq b)$
  can be calculated by the formula:
               \begin{equation}
             L=\int_a^b
             \sqrt{(1+k^2)h_t^2+k^2h^2\varphi_t^2}dt
               \end{equation}

\medskip

 \centerline {\it Sphere}


\medskip


  Sphere is given by the equation $x^2+y^2+z^2=a^2$. Consider the following
(standard ) parameterisation
 of this surface:
\begin{equation}\label{surfacesphere}
  \r(\theta,\varphi)\colon\quad
  \begin{cases}
  x=a\sin\theta\cos\varphi\\
  y=a\sin\theta\sin\varphi\\
  z=a\cos\theta\\
  \end{cases}
\end{equation}

\medskip

   Calculate induced Riemannian metric (first quadratic form)
              $$
              G_{S^2}=\left(dx^2+dy^2+dz^2\right)\big\vert_{x=a\sin\theta\cos\varphi,y=a\sin\theta\sin\varphi,
              z=a\cos\theta}=
                      $$
                      $$
                      (a\cos\theta\cos\varphi d\theta-a\sin\theta\sin\varphi d\varphi)^2+
                      (a\cos\theta\sin\varphi d\theta+a\sin\theta\cos\varphi d\varphi)^2+
                         (-a\sin\theta d\theta)^2=
                      $$
                      $$
          a^2\cos^2\theta d\theta^2+a^2\sin^2\theta d\varphi^2+a^2\sin^2\theta d\theta^2=
                      $$
        \begin{equation}\label{firtsquadraticformforsphere(diff)}\,,\qquad
             =a^2d\theta^2+a^2\sin^2\theta d\varphi^2\,,\qquad
                        ||g_{\a\beta}||=
   \begin{pmatrix}
   a^2 & 0 \\
   0&  a^2\sin^2\theta \\
   \end{pmatrix}
                       \end{equation}

One comes to the same answer calculating scalar product of tangent vectors:
  \begin{equation}\label{c}
  \r_\theta=\begin{pmatrix}
        a\cos\theta\cos\varphi\\
        a\cos\theta\sin\varphi\\
        -a\sin\theta\\
   \end{pmatrix}
\quad
  \r_\varphi=\begin{pmatrix}
        -a\sin\theta\sin\varphi\\
        a\sin\theta\cos\varphi\\
          0\\
   \end{pmatrix}
 \end{equation},
            $$
     (\r_\theta,\r_\theta)=a^2,\quad
     (\r_h,\r_\varphi)=0,\quad
     (\r_\varphi,\r_\varphi)=a^2\sin^2\theta
            $$
and
\begin{equation*}\label{formula forfirstform3*}
   ||g||=
   \begin{pmatrix}
   (\r_u,\r_u) & (\r_u,\r_v) \\
   (\r_u,\r_v) & (\r_v,\r_v) \\
   \end{pmatrix}=
\end{equation*}
\begin{equation*}\label{formula forfirstformsphere}
   \begin{pmatrix}
   a^2 & 0 \\
   0&  a^2\sin^2\theta \\
   \end{pmatrix}, \quad
   G_{S^2}=ds^2=a^2d\theta^2+a^2\sin^2\theta d\varphi^2
\end{equation*}
  The length of the curve $\r(t)=\r (\theta(t), \varphi(t))$ on the
  sphere of the radius $a$
    $(a\leq t\leq b)$
  can be calculated by the formula:
               \begin{equation}
             L=\int_a^b
             a\sqrt{\theta_t^2+\sin^2\theta\cdot \varphi_t^2}dt
               \end{equation}


We considered all quadratic surfaces except paraboloid $z=x^2-y^2$.

It can be rewritten as $z=xy$ () and is call sometimes "saddle""

\centerline {\it Saddle (paraboloid)}


\medskip


  Saddle is given by the equation $z-xy=0$.

Why paraboloid?

{\bf Exercise} Show that the equation of saddle can be rewritten as $z=x^2-y^2$

\m

  (This surface is a ruled surface containing lines...)

  Consider the following
(standard ) parameterisation
 of this surface:
\begin{equation}\label{surface3}
  \r(u,v)\colon\quad
  \begin{cases}
  x=u\\
  y=v\\
  z=uv\\
  \end{cases}
\end{equation}

\medskip

Calculate induced metric:
                       $$
 G_{saddle}=\left(dx^2+dy^2+dz^2\right)\big\vert_{x=u\cos\varphi,y=v\sin\varphi,z=uv}=
                      $$
                      $$
                      du^2+dv^2+(udv+vdu)^2=
                      $$
        \begin{equation}\label{firtsquadraticformforsaddle(diff)}
            G_{saddle} =(1+v^2)du^2+2uvdudv+(1+u^2)dv^2\,.
            \end{equation}

 The length of the curve $\r(t)=\r (u(t), v(t))$ on the
  sphere of the radius $a$
    $(a\leq t\leq b)$
  can be calculated by the formula:
               \begin{equation}
             L=\int_a^b
             a\sqrt{(1+v^2)u^2_t+2uvu_tv_t+(1+u^2)v^2_t}dt
               \end{equation}

\medskip

    \centerline {\it One-sheeted and two-sheeted hyperboloids.}


    Consider surface given by the equation
                $$
              x^2+y^2-z^2=c
                $$
    If $c=0$ it is a cone. We considered it already above.

     If $c>0$ it is  one-sheeted hyperboloid---connected surface in $\E^3$
     If $c<0$ it is two-sheeted hyperboloid--- a surface with two sheets $z>0 $ and $z<0$
L

    \m
     Consider these cases separately.

     \m

    1) {\sl One-sheeted hyperboloid}: $x^2+y^2-z^2=a^2$. It is ruled surface. It contains two family of lines....



One-sheeted hyperboloid is given by the equation $x^2+y^2-z^2=a^2$.  it is convenient to
choose parameterisation:
\begin{equation}\label{surfaceonesheeteed}
  \r(\theta,\varphi)\colon\quad
  \begin{cases}
  x=a\cosh\theta\cos\varphi\\
  y=a\cosh\theta\sin\varphi\\
  z=a\sinh\theta\\
  \end{cases}
\end{equation}
   $$
 x^2+y^2-z^2=a^2\cosh^2\theta-a^2\sinh^2\theta=a^2
   $$
Compare the calculations with calculations for sphere! We changed functions $\cos, \sin$ on $\cosh,\sinh$
\medskip

Induced Riemannian metric (first quadratic form).


              $$
              G_{Hyperbol I}=\left(dx^2+dy^2+dz^2\right)\big\vert_{x=a\cosh\theta\cos\varphi,y=a\cosh\theta\sin\varphi,
              z=a\sinh\theta}=
                      $$
                      $$
                      (a\sinh\theta\cos\varphi d\theta-a\cosh\theta\sin\varphi d\varphi)^2+
                      (a\sinh\theta\sin\varphi d\theta+a\cosh\theta\cos\varphi d\varphi)^2+
                         (a\cosh\theta d\theta)^2=
                      $$
                      $$
          a^2\sinh^2\theta d\theta^2+a^2\cosh^2\theta d\varphi^2+a^2\cosh^2\theta d\theta^2=
                      $$
        \begin{equation}\label{firtsquadraticformforsphere(diff)}\,,\qquad
             =a^2(1+2\sinh^2\theta) d\theta^2+a^2\cosh^2\theta d\varphi^2\,,\qquad
                        ||g_{\a\beta}||=
   \begin{pmatrix}
   1+2\sinh^2\theta & 0 \\
   0&  \cosh^2\theta \\
   \end{pmatrix}
                       \end{equation}
 \m
 For two-sheeted hyperboloid calculatiosn will be very similar.

1) {\sl Two-sheeted hyperboloid}: $z^2-x^2-y^2=a^2$. It is not ruled surface!



  it is convenient to
choose parameterisation:
\begin{equation}\label{surfacehyperbol2}
  \r(\theta,\varphi)\colon\quad
  \begin{cases}
  x=a\sinh\theta\cos\varphi\\
  y=a\sinh\theta\sin\varphi\\
  z=a\cosh\theta\\
  \end{cases}
\end{equation}
   $$
 z^2-x^2-y^2=a^2\cosh^\theta-a^2\sinh^2\theta=a^2
   $$
Compare the calculations with calculations for sphere! We changed functions $\cos, \sin$ on $\cosh,\sinh$
\medskip

Induced Riemannian metric (first quadratic form).


              $$
              G_{Hyperbol I}=\left(dx^2+dy^2+dz^2\right)\big\vert_{x=a\sinh\theta\cos\varphi,y=a\sinh\theta\sin\varphi,
              z=a\cosh\theta}=
                      $$
                      $$
                      (a\cosh\theta\cos\varphi d\theta-a\sinh\theta\sin\varphi d\varphi)^2+
                      (a\cosh\theta\sin\varphi d\theta+a\sinh\theta\cos\varphi d\varphi)^2+
                         (a\sinh\theta d\theta)^2=
                      $$
                      $$
          a^2\cosh^2\theta d\theta^2+a^2\sinh^2\theta d\varphi^2+a^2\sinh^2\theta d\theta^2=
                      $$
        \begin{equation}\label{firtsquadraticformforsphere(diff)}\,,\qquad
             =a^2(1+2\sinh^2\theta)d\theta^2+a^2\sinh^2\theta d\varphi^2\,,\qquad
                        ||g_{\a\beta}||=
   \begin{pmatrix}
   1+2\sinh^2\theta & 0 \\
   0&  \sinh^2\theta \\
   \end{pmatrix}
\end{equation}


\bigskip

  We calculated examples of induced Rieamnnina structure embedded in Euclidean space almost for all quadratic surfaces.

Quadratic surface is a surface defined by the equation
              $$
             Ax^2+By^2+Cz^2+2Dxy+2Exz+2Fyz+ex+fy+dz+c=0
              $$
One can see that any quadratic surface by affine transformation can be transformed to one of these surfaces

  \begin{itemize}
  \item  cylinder (elliptic cylinder) $x^2+y^2=1$

  \item  hyperbolic cylinder: $x^2-y^2=1$)

  \item  parabolic cylinder  $z=x^2$

  \item  paraboloid   $x^2+y^2=z$

  \item      hyperbolic paraboloid $x^2-y^2=z$

  \item  cone $x^2+y^2-z^2=0$

  \item  sphere $x^2+y^2+z^2=1$

  \item  one-sheeted hyperboloid

  \item   two-sheeted hyperboloid

  \end{itemize}
(We exclude degenerate cases such as "point" $x^2+y^2+z^2=0$, planes, e.t.c.)



  \subsubsection {$^*$Induced metric on two-sheeted hyperboloid embedded in pseudo-Euclidean space.}

{\small
   Consider two-sheeted hyperboloid embedded in pseudo-Euclidean space with pseudo-scalar product defined by
   bilinear form
   \begin{equation}\label{pseudoscalar produc}
    \la\X,\Y\ra_{pseudo}= X^1Y^1+X^2Y^2-X^3Y^3
   \end{equation}
 The "pseudoscalar" product is bilinear, symmetric. It is defined by non-degenerate matrix.
  But it is not positive-definite: The "pseudo-length" of vectors $\X=(a\cos\varphi,a\sin\varphi,\pm a)$ equals to zero:
       \begin{equation}\label{pseudolength}
     \X=(a\cos\varphi,a\sin\varphi,\pm a)\Rightarrow   \la\X,\X\ra_{pseudo}=0,
       \end{equation}
This is not scalar product. The pseudo-Riemannian metric is:
              \begin{equation}\label{pseudoriemannianmetric}
                G_{pseudo}=dx^2+dy^2-dz^2
              \end{equation}



it turns out that the  following remarkable fact occurs:

 {\bf Proposition} {\it The pseudo-Riemannian metric \eqref{pseudoriemannianmetric} in the ambient $3$-dimensional
 pseudo-Euclidean space induces Riemannian metric on two-sheeted hyperboloid $x^2+y^2-z^2=1$.}

 Show it.  repeat the calculations above for two-sheeted hyperboloid changing in the ambient space
 Riemannian metric $G=dx^2+dy^2+dz^2$ on pseudo-Riemannian $dx^2+dy^2-dz^2$:

\m


Using \eqref{surfacehyperbol2} and  \eqref{pseudoriemannianmetric}
we come now to


              $$
              G=
              \left(dx^2+dy^2-dz^2\right)\big\vert_{x=a\sinh\theta\cos\varphi,y=a\sinh\theta\sin\varphi,
              z=a\cosh\theta}=
                      $$
                      $$
                      (a\cosh\theta\cos\varphi d\theta-a\sinh\theta\sin\varphi d\varphi)^2+
                      (a\cosh\theta\sin\varphi d\theta+a\sinh\theta\cos\varphi d\varphi)^2-
                         (a\sinh\theta d\theta)^2=
                      $$
                      $$
          a^2\cosh^2\theta d\theta^2+a^2\sinh^2\theta d\varphi^2-a^2\sinh^2\theta d\theta^2
                      $$
        \begin{equation}\label{firtsquadraticformforsphere(diff)}\,,\qquad
           G_L  =a^2d\theta^2+a^2\sinh^2\theta d\varphi^2\,,\qquad
                        ||g_{\a\beta}||=
   \begin{pmatrix}
   1 & 0 \\
   0&  \sinh^2\theta \\
   \end{pmatrix}
\end{equation}


The two-sheeted hyperboloid equipped with this metric is called hyperbolic or Lobachevsky plane.

 \m

Now express Riemannian metric in stereographic coordinates.

Calculations are very similar to the case of stereographic coordinates of $2$-sphere
 $x^2+y^2+z^2=1$. (See homework 1). Centre of projection $(0,0,-1)$:
 For stereographic coordinates $u,v$ we have ${u\over x}={y\over v}={1\over 1+z}$.  We come to
                   $$
                    \begin{cases}
             u={x\over 1+z}\cr
             v={y\over 1+z}\cr
                   \end{cases},
                    \qquad
                  \begin{cases}
                 x={2u\over 1-u^2-v^2}\cr
                 y={2v\over 1-u^2-v^2}\cr
                 z={u^2+v^2+1\over 1-u^2-v^2}
                    \end{cases}
                 \eqno (4)
                     $$

The image of upper-sheet is an open disc $u^2+v^2=1$ since
$u^2+v^2={x^2+y^2\over (1+z)^2}={z^2-1\over (1+z)^2}={z-1\over z+1}$.
Since for upper sheet $z>1$ then $0\leq {z-1\over z+1}<1$.
\m


    $$
                         G=(dx^2+dy^2-dz^2)\big\vert_{x=x(u,v),y=y(u,v),z=z(u,v)}=
                     \left(d\left(2u\over 1-u^2-v^2\right)\right)^2+
                     $$
                     $$
                     \left(d\left(2v\over 1-u^2-v^2\right)\right)^2-
         \left(d\left(u^2+v^2+1\over 1-u^2-v^2\right)\right)^2=
            $$
(Compare with calculations for sphere $x^2+y^2+z^2=1$). We have $G=$
                 $$
               \left(
              {2du\over 1-u^2-v^2}+
              {2u(2udu+2vdv)\over (1-u^2-v^2)^2}
              \right)^2
              +
              \left(
              {2dv\over 1-u^2-v^2}+
              {2v(2udu+2vdv)\over (1-u^2-v^2)^2}
              \right)^2
              $$
              $$
              -
                            \left(
              {2udu+2vdv\over 1-u^2-v^2}+
              {(u^2+v^2+1)(2udu+2vdv)\over (1-u^2-v^2)^2}
              \right)^2=
              {4(du)^2+4(dv)^2\over (1+u^2+v^2)^2}
    $$
To continue calculations it is convenient to denote by $s=1-u^2-v^2$. We come to
     $$
G={4\over s^4}\left[\left((1+u^2-v^2)du+2uvdv\right)^2+\left((1+v^2-u^2)du+2uvdu\right)^2-
4(udu+vdv)^2\right]=
     $$
     $$
          {4\over s^4} \left[
        \left(\left(1+u^2-v^2\right)^2+4u^2v^2-4u^2\right)du^2+
           (u\leftrightarrow v)+
        \left(
          4uv\left(1+u^2-v^2\right)+
          4uv\left(1+v^2-u^2\right)-8uv
        \right)dudv
        \right]=
     $$
      $$
     {4\over s^4} \left[
             s^2du^2+
               s^2dv^2+
                     0
        \right]={4du^2+4dv^2\over s^2}={4du^2+4dv^2\over (1-u^2-v^2)^2}\,.
      $$




}



\subsection {Isometries of Riemanian manifolds.}


{\bf Definition}  Let $(M_1,G_{(1)})$, $(M_2,G_{(2)})$ be two Riemannian manifolds---manifolds equipped
with Riemannian metric $G_1$ and $G_2$ respectively.




We say that these Riemannian manifolds are {\it isometric} if  there exists a diffeomorphism $F$
(one-one smooth map) such  that
              $$
            F^*G_{(2)}=G_{(1)}\,,
              $$
which  means the following:


      Let $\pt_1$ be an arbitrary point on manifold $M_1$ and $\pt_2\in M_2$ be its image:$F(\pt_1)=\pt_2$.
      Let $\{x^i\}$ be coordinates in a vicinity of a point $\pt_1\in M_1$
      and $\{y^a\}$ be coordinates in a vicinity of a point $\pt_2\in M_2$.
      Let  Riemannian  metrics  $G_1$ on $M_1$ has local expression
      $G_{(1)}=g_{(1)ik}(x)dx^idx^k$ in coordinates $\{x^i\}$   and respectively
      Riemannian  metrics  $G_{(2)}$ has local expression
      $G_2=g_{(2)ab}(y)dy^ady^b$ in coordinates $\{y^i\}$ on $M_2$.
                 Then
                 \begin{equation}\label{transformationlaw}
   g_{(1)ik}(x)dx^idx^k=g_{(2)ab}(y)dy^ady^b=g_{(2)ab}(y(x))
   {\p y^a(x)\over \p x^i}dx^i{\p y^b(x)\over \p x^k}dx^k\,
                 \end{equation}
    i.e.  \begin{equation}\label{transformationlaw2}
   g_{(1)ik}(x)={\p y^a(x)\over \p x^i}g_{(2)ab}(y(x)){\p y^b(x)\over \p x^k}\,.
                 \end{equation}
 where $y^i=y^i(x)$ is local expression for diffeomorphism  $F$.

\m

  {\bf Definition}  We say that two Riemannian manifolds $(M_1,G_{(1)})$, $(M_2,G_{(2)})$
  are {\it locally isometric} if  the following conditions hold:

  For  arbitrary point $\pt_1$ on the manifold $M_1$ there exists a point $\pt_2$ on the manifold
  $M_2$ such that there exist coordinates $\{x^i\}$ in a vicinity of a point $\pt_1\in M_1$
      and coordinates $\{y^a\}$ in a vicinity of a point $\pt_2\in M_2$
      such that local expression  for metric $G_{(1)}$ on $M_1$ in coordinates
      $\{x^i\}$ and  local expression  for metric $G_{(2)}$ on $M_2$ in coordinates
      $\{y^a\}$ are related via formuale \eqref{transformationlaw}, \eqref{transformlaw2}.


      \m

      Locally isometric Riemannian manifolds have not to be diffeomorphic. E.g. Euclidean plane
      is locally isometric to cylinder, but they are not diffeomorphic.


      \subsubsection {Examples of local isometries}

   Consider examples.

\m

      {\bf Example 1} Cylinder-Cone---Plane,

     Riemannian metric on cylinder is $G_{cylinder}=a^2d\varphi^2+dh^2$ and on the cone
       $G_{conus}=k^2h^2d\varphi^2+(1+k^2)dh^2$ (see formulae \eqref{firtsquadraticformcylinder(diff)}
       and \eqref{firtsquadraticformforconus(diff)}).

       To show that cylinder is isometric to Euclidean plan we have to find
       new local coordinates  $u,v$ on the cylinder such that in these coordinates the metric on the cylinder
       equals to $du^2+dv^2$. If we put $u=a\varphi$, $v=h$ then
                 \begin{equation}\label{localisometryofcylindertoplane}
            du^2+dv^2=d(a\varphi)^2+dh^2=a^2d\varphi^2+dh^2=G_{cylinder}\,.
                 \end{equation}
      Thus  we prove the local isometry. Of course the coordinate $u=\varphi$ is not global coordinate on the surface
      of cylinder. It is evident that cylinder and plane {\it are not globally isometric} since there
      are no diffeomorphism of cylinder on the plane. (They are different non-homeomorphic topological spaces.)


  \m

   Now show that cone is locally isometric to the plane.


     This means that we have to find local coordinates $u,v$ on the cone such that in these coordinates
  induced metric $G\vert_c$ on cone would have the appearance $G\vert_c=du^2+dv^2$.

  First of all calculate the metric on cone in natural coordinates $h,\varphi$ where
             $$
          \r(h,\varphi)\colon
          \begin{cases}
          x=kh\cos\varphi\cr
          y=kh\sin\varphi\cr
          z=h\cr
          \end{cases}.
             $$
($x^2+y^2-k^2z^2=k^2h^2\cos^2\varphi+k^2h^2\sin^2\varphi-k^2h^2=k^2h^2-k^2h^2=0$.

Calculate  metric $G_c$ on the cone in coordinates $h,\varphi$  induced with
the Euclidean metric $G=dx^2+dy^2+dz^2$:
                $$
            G_c=\left(dx^2+dy^2+dz^2\right)\big\vert_{x=kh\cos\varphi, y=kh\sin\varphi, z=h}=
            (k\cos\varphi dh-kh\sin\varphi d\varphi)^2+
            $$
            $$
            (k\sin\varphi dh+kh\cos\varphi d\varphi)^2+dh^2=
            (k^2+1)dh^2+k^2h^2d\varphi^2\,.
                $$
In analogy with polar coordinates try to find new local coordinates $u,v$
such that $\begin{cases} u=\a h\cos \beta\varphi\cr v=\a h\sin \beta\varphi\end{cases}$,
where $\alpha, \beta$ are parameters. We come to  $du^2+dv^2=$
             $$
    \left(\a\cos\beta\varphi dh-\a\beta h\sin\beta\varphi d\varphi\right)^2+
  \left(\a\sin\beta\varphi dh+\a\beta h\cos\beta\varphi d\varphi\right)^2=
  \alpha^2 dh^2+\a^2\beta^2 h^2d\varphi^2.
             $$
Comparing with the metric on the cone $G_c=(1+k^2)dh^2+k^2h^2d\varphi^2$  we see that if we put $\alpha=k$ and
$\beta={k\over \sqrt{1+k^2}}$
then $du^2+dv^2=\alpha^2 dh^2+\a^2\beta^2 h^2d\varphi^2=(1+k^2)dh^2+k^2h^2d\varphi^2$.

Thus in new local coordinates
                  $$
                  \begin{cases}
             u= \sqrt {k^2+1}h\cos {k\over \sqrt {k^2+1}}\varphi\cr
             v=  \sqrt {k^2+1} h\sin {k\over \sqrt {k^2+1}}\varphi\cr
                  \end{cases}
                  $$
induced metric on the cone becomes
$G\vert_c= du^2+dv^2$, i.e. cone locally is isometric to the Euclidean plane \finish

Of course these coordinates are local.---  Cone and plane are not homeomorphic, thus they are not globally isometric.

\m

    One can show also that

      2) Plane with metric $a(dx^2+dy^2)\over (1+x^2+y^2)^2$ is isometric to the sphere with radius $R(a)=...$

    3) Disc with metric $du^2+dv^2\over (1-u^2-v^2)^2$ is isometric to half plane with metric $dx^2+dy^2\over 4y^2$.

(see exercises in Homeworks and Coursework.)


    \subsection {Volume element in Riemannian manifold}

   The volume element   in $n$-dimensional Riemannian manifold with metric $G=g_{ik}dx^i
    dx^k$ is defined by the formula
     \begin{equation}\label{volumelement}
  \sqrt {\det g_{ik}}\,dx^1dx^2\dots dx^n
\end{equation}
    If $D$ is a domain in the $n$-dimensional Riemannian manifold with metric
    $G=g_{ik}dx^i$
    then its volume is equal to to the integral of volume element over this domain.
          \begin{equation}\label{volumeofriemanianmanifold}
  V(D)=\int_D \sqrt {\det g_{ik}}\,dx^1dx^2\dots dx^n
\end{equation}

{\bf Remark} Students who know the concept of exterior forms
 can read the volume element as
     \begin{equation}\label{volumelement2}
  \sqrt {\det g_{ik}}\,dx^1\wedge dx^2\wedge \dots \wedge dx^n
\end{equation}

\bigskip

     Note that in the case of $n=1$ volume is just the length, in the case
     if $n=2$ it is area.

\subsubsection {Volume of parallelepiped}

  Note that the formula \eqref{volumelement} gives the volume of $n$-dimensional parallelepiped.
  Show this.  Let $\E^n$ be Euclidean
  vector space with orthonormal basis $\{\e_i\}$. Let
  ${\v_i}$ be an arbitrary basis in this vector space (vectors $\v_i$ in general have not unit length and
  are not orthogonal to each other). Consider $n$-parallelepiped spanned by vectors $\{\v_i\}$:
                 $$
     \Pi_{\v_i}\colon  \r=t^i\v_i, 0\leq t^i\leq 1.
                 $$
  The volume of this parallelepiped equals to
             \begin{equation}\label{parallelepipinn-dimspace1}
                Vol(\Pi_{\v_i})=\det ||a_i^m||\,,
             \end{equation}
where $A=||a_i^m||$ is  transition matrix, $\v_i=\e_m a_i^m$.
On the other hand
             $$
\r=x^i\e_i=t^m\v_m, \,
\hbox{hence $x^i=a^i_m t^m$, where $\v_m=\e_ia^i_m$.}
             $$
Let $G=(dx^1)^2+\dots+(dx^n)^2=g_{ik}dt^idt^k$ be usual Euclidean metric in new coordinates $t^i$. The
          $$
     G=(dx^1)^2+\dots+(dx^n)^2=dx^i\delta_{ik}dx^k=
     dt^{i}{\p x^{i'}\over \p t^{i}}\delta_{i'k'}{\p x^{k'}\over \p t^{k}}dt^{k}.
          $$
Since   ${\p x^{i'}\over \p t^{i}}=a^{i'}_i$ then
                          $$
g_{ik}=\sum_{i'}a^{i'}_i a^{i'}_k\Rightarrow \det g=(\det A)^2, \det g=\sqrt {\det A}.
  $$
and according to the formula \eqref{volumeofriemanianmanifold}
          $$
      Vol(\Pi_{\v_i})=\int_{0\leq t^i\leq 1} \sqrt{\det g} dt^1dt^2\dots dt^n=\det A\,.
          $$
We come to \eqref{parallelepipinn-dimspace1}.
\m

Perform these calculations in detail for $3$-dimensional case.




E.g. if Euclidean space is $3$-dimensional then the parallelepiped spanned by basis vectors
$\{\ac,\b,{\bf c}\}$
                 $$
                 \Pi_{\ac,\b,{\bf c}}=t^1\ac+t^2\b+t^3{\bf c}\,,
                     0\leq t^1,t^2,t^3\leq 1.
                 $$
Volume of parallelepiped equals to
\begin{equation}\label{volparallelepindim3}
Vol \Pi_{\ac,\b,{\bf c}}=\det
\begin{pmatrix}
a_x&b_x&c_x\cr
a_y&b_y&c_y\cr
a_z&b_z&c_z\cr
\end{pmatrix},
\end{equation}
and
          $$
\begin{pmatrix}
x\cr
y\cr
z\cr
\end{pmatrix}=
\begin{pmatrix}
a_x&b_x&c_x\cr
a_y&b_y&c_y\cr
a_z&b_z&c_z\cr
\end{pmatrix}
\begin{pmatrix}
t^1\cr
t^2\cr
t^3\cr
\end{pmatrix}\,.
          $$
 The Riemannian metric in new coordinates $(t^1,t^2,t^3)$ equals to
               $$
    dx^2+dy^2+dz^2=\begin{pmatrix}dt^1&dt^2&dt^3\end{pmatrix}
    \begin{pmatrix}
a_x&a_y&a_z\cr
b_x&b_y&b_z\cr
c_x&c_y&c_z\cr
\end{pmatrix}
\begin{pmatrix}
a_x&b_x&c_x\cr
a_y&b_y&c_y\cr
a_z&b_z&c_z\cr
\end{pmatrix}
\begin{pmatrix}
dt^1\cr
dt^2\cr
dt^3\cr
\end{pmatrix},
    $$
i.e. in coordinates $t^i$ Riemannian metric $G=g_{ik}dt^idt^k$ where
           $$
\begin{pmatrix}
g_{11}&g_{12}&g_{13}\cr
g_{21}&g_{22}&g_{23}\cr
g_{31}&g_{32}&g_{33}\cr
\end{pmatrix}=
           \begin{pmatrix}
a_x&a_y&a_z\cr
b_x&b_y&b_z\cr
c_x&c_y&c_z\cr
\end{pmatrix}
\begin{pmatrix}
a_x&b_x&c_x\cr
a_y&b_y&c_y\cr
a_z&b_z&c_z\cr
\end{pmatrix}
           $$
i.e.
              \begin{equation}\label{volparapll3-2}
\sqrt {\det g_{ik}}=\det \begin{pmatrix}
a_x&b_x&c_x\cr
a_y&b_y&c_y\cr
a_z&b_z&c_z\cr
\end{pmatrix}=Vol \Pi_{\ac,\b,{\bf c}}\,.
              \end{equation}




{\small      \medskip
\subsubsection{ Invariance of volume element under changing of coordinates}
\medskip

    Prove that volume element is invariant under coordinate transformations,
    i.e. if $y^1,\dots,y^n$ are new coordinates:
    $x^1=x^1(y^1,\dots,y^n), x^2=x^2(y^1,\dots,y^n)...$,
                      $$
          x^i=x^i(y^p), i=1,\dots,n\quad, p=1,\dots,n
                      $$
and   $\tilde g_{pq}(y)$ matrix of the metric in new coordinates:
                       \begin{equation}\label{changingofcoord2}
    \tilde g_{pq}(y)={\p x^i\over \p y^{p}}g_{ik}(x(y))
             {\p x^k\over \p y^{q}}\,.
\end{equation}
Then
                    \begin{equation}\label{invarianceofvolumeelement}
 \sqrt {\det g_{ik}(x)}\,dx^1 dx^2 \dots dx^n=
 \sqrt {\det \tilde g_{pq}(y)}\,dy^1 dy^2 \dots dy^n
\end{equation}
This follows from \eqref{changingofcoord2}. Namely
                             $$
\sqrt {\det g_{ik}(y)}\,dy^1 dy^2 \dots dy^n=
\sqrt {\det \left({\p x^i\over \p y^{p}}g_{ik}(x(y))
             {\p x^k\over \p y^{q}}\right)}\,dy^1 dy^2 \dots dy^n
                             $$
Using the fact that $\det (ABC)=\det A\cdot \det B\cdot \det C$
and  $\det \left({\p x^i\over \p y^{p}}\right)=
\det \left({\p x^k\over \p y^{q}}\right)$\footnote{determinant of matrix does not change
if we change the matrix on the adjoint, i.e. change columns on rows.} we see that
from the formula above follows:
           $$
           \sqrt {\det g_{ik}(y)}\,dy^1 dy^2 \dots dy^n=
\sqrt {\det \left({\p x^i\over \p y^{p}}g_{ik}(x(y))
             {\p x^k\over \p y^{q}}\right)}
          dy^1 dy^2 \dots dy^n=
           $$
           $$
\sqrt {\left(\det
        \left(
    {\p x^i\over \p y^{p}}
    \right)\right)^2}
     \sqrt {\det g_{ik}(x(y))}
     dy^1 dy^2 \dots dy^n=
           $$
\begin{equation}\label{transformofvolumeform4}
\sqrt {\det g_{ik}(x(y))}
         \det
        \left(
    {\p x^i\over \p y^{p}}
    \right)
    dy^1 dy^2 \dots dy^n=
\end{equation}


Now note that   $$
\det \left({\p x^i\over \p y^{p}}\right)dy^1 dy^2 \dots dy^n=dx^1\dots dx^n
                $$
according to the formula for changing coordinates in $n$-dimensional integral
\footnote{Determinant of the matrix $\left({\p x^i\over \p y^{p}}\right)$ of changing of coordinates
is called sometimes Jacobian. Here we consider the case if Jacobian is positive.
If Jacobian is negative then formulae above remain valid just the symbol of modulus appears.}.
Hence
\begin{equation}\label{transformofvolumeform5}
\sqrt {\det g_{ik}(x(y))}\det
      \left(
 {\p x^i\over \p y^{p}}
 \right) dy^1 dy^2 \dots dy^n=
\sqrt {\det g_{ik}(x(y))}dx^1 dx^2 \dots dx^n
\end{equation}
Thus we come to \eqref{invarianceofvolumeelement}.}

\subsubsection { Examples of calculating volume element}

Consider first very simple example: Volume element of plane in cartesian coordinates,
metric $g=dx^2+dy^2$. Volume element  is equal to
                      $$
         \sqrt {\det g}dxdy=
         \sqrt
           {
         \det
         \left(
    \begin{array}{cc}
  1 & 0 \\
  0&  1 \\
\end{array}
\right)}dxdy=dxdy
                      $$
  Volume of the domain $D$ is equal to
              $$
          V(D)=\int_D\sqrt {\det g}dxdy=\int_D dxdy
              $$

If we go to polar coordinates:
 \begin{equation}\label{polarcoord1}
    x=r\cos\varphi, y=r\sin\varphi
\end{equation}
Then we have for metric:
              $$
G=dr^2+r^2d\varphi^2
              $$
because
               \begin{equation}\label{polarknow}
       dx^2+dy^2=(dr\cos\varphi-r\sin\varphi d\varphi)^2+
    (dr\sin\varphi+r\cos\varphi d\varphi)^2
       =dr^2+r^2 d\varphi^2
              \end{equation}
Volume element in polar coordinates is equal to
  $$
           \sqrt {\det g}drd\varphi=
         \sqrt
           {
         \det
         \left(
    \begin{array}{cc}
  1 & 0 \\
  0&  r^2 \\
\end{array}
\right)}drd\varphi=drd\varphi\,.
  $$

\m

{\it Lobachesvky plane.}

 In coordinates $x,y$ ($y>0$) metric $G={dx^2+dy^2\over y^2}$,
 the corresponding matrix $G=\begin{pmatrix}
      {1/y^2} & _0 \\
   _0 & {1/y^2} \
 \end{pmatrix}$. Volume element is equal to
 $\sqrt {\det g} dxdy={dxdy\over y^2}$\,.

\medskip


 {  \it Sphere in stereographic coordinates}
Consider the two dimensional plane with Riemannian metrics
                     \begin{equation}\label{stereographic}
    G={4(du^2+dv^2)\over (1+u^2+v^2)^2}
\end{equation}

(It is isometric to the sphere without North pole in stereographic coordinates)

   Calculate its volume element and volume.
            It is easy to see that:
                    \begin{equation}\label{}
    G=\left(
    \begin{array}{cc}
  {1\over (1+u^2+v^2)^2} & 0 \\
  0&  {1\over (1+u^2+v^2)^2} \\
\end{array}
\right)
\qquad
  \det g= {1\over (1+u^2+v^2)^4}
\end{equation}
and volume element is equal to  $\sqrt {\det g}dudv= {dudv\over (1+u^2+v^2)^2}$

The volume (area) of plane will be:
                 $$
      \int {dudv\over (1+u^2+v^2)^2}=\int_{-\infty}^{\infty} du \int_{-\infty}^{\infty}
         {du\over (1+u^2+v^2)^2}={\pi\over 2}\int_{-\infty}^{\infty} {du\over (1+u)^{3/2}}=4\pi
                 $$
We see that in coordinates $(u,v)$ calculation of the integral is not very easy.

One can consider volume form in polar coordinates
$u=r\cos\varphi, v=r\sin\varphi$.
Then it is easy to see that according to \eqref{polarknow}
 we have for the metric $G={du^2+dv^2\over (1+u^2+v^2)^2}={dr^2+r^2d\varphi^2\over (1+r^2)^2}$
and volume form is equal to
$\sqrt {\det g}drd\varphi={rdr d \varphi\over (1+r^2)^2}$

Now calculation of integral becomes easy:
                  $$
    V=\int   {rdr d \varphi\over (1+r^2)^2}=2\pi \int_0^\infty {rdr\over (1+r^2)^2}=
                  \pi \int_0^\infty {du\over (1+u)^2}=\pi\,.
                  $$

                  \medskip
  {\it Segment of the sphere.}

    Consider sphere of the radius $a$ in Euclidean space with standard Riemanian metric
                       $$
            a^2 d\theta^2+a^2\sin^2\theta d\varphi^2
                       $$
  This metric is nothing but first quadratic form   on the sphere (see \eqref{formula forfirstformsphere}).
   The volume element is
                             $$
           \sqrt {\det g}d\theta  d\varphi=
         \sqrt
           {
         \det
         \left(
    \begin{array}{cc}
  a^2 & 0 \\
  0&  a^2\sin \theta \\
\end{array}
\right)}d\theta d\varphi=a^2\sin \theta d\theta d\varphi
                             $$
 Now calculate the volume of the segment of the sphere between two parallel planes,
 i.e. domain restricted  by parallels $\theta_1\leq \theta \leq \theta_0$:
   Denote by  $h$ be the height of this segment. One can see that
                 $$
               h=a\cos\theta_0-a\cos\theta_1=a(\cos\theta_0-a\cos \theta_1)
                $$
   There is remarkable formula which express the area of segment via the height $h$:
       $$
   V= \int_{\theta_1\leq \theta \leq \theta_0}\left (a^2\sin \theta \right)d\theta d\varphi=
         \int_{\theta^0}^{\theta^1}\left(\int_0^{2\pi}\left(a^2\sin \theta\right) d\varphi\right)d\theta=
                $$
\begin{equation}\label{areaoftheseqment}
       \int_{\theta^1}^{\theta^0} 2\pi a^2\sin\theta d\theta=2\pi a^2 (\cos \theta_0-\cos\theta_1)=
            2\pi a (a\cos \theta_0-a\\cos \theta_1)=2\pi a h
\end{equation}
 E.g. for all the sphere $h=2a$. We come to $S=4\pi a^2$.
It is remarkable formula: area of the segment is a polynomial function of radius of the sphere
 and height (Compare with formula for length of the arc of the circle)


\section {Covariant differentiaion. Connection. Levi Civita  Connection on Riemannian manifold}

\subsection {Differentiation of vector field along the vector field.---Affine connection}

How to differentiate vector fields on a (smooth )manifold  $M$?


Recall  the differentiation  of functions on a (smooth )manifold  $M$.







Let $\bf X=X^i(x){\bf e}_i(x)={\p\over \p x^i}$ be a vector field on $M$.
Recall that vector field
\footnote{here like always we suppose by default the summation over repeated indices.
E.g.$\X=X^i{\bf e}_i$ is nothing but
$\X=\sum_{i=1}^nX^i{\bf e}_i$}
 $\bf X=X^i{\bf e}_i$ defines at the
every point $x_0$ an infinitesimal curve: $x^i(t)=x^i_0+tX^i$
(More exactly the equivalence class $[\gamma(t)]_\X $
of curves $x^i(t)=x^i_0+tX^i+\dots$).


Let $f$ be an arbitrary (smooth) function on $M$ and $\X=X^i{\p\over \p x^i}$.
 Then derivative
of function $f$ along vector field $\X=X^i{\p\over \p x^i}$ is equal to
                              $$
             \p_{\bf X}f= \nabla_{\bf X}f
                  =X^i{\p f\over \p x^i}
                                $$
The geometrical meaning of this definition is following:
If $\X$ is a velocity vector of the curve $x^i(t)$ at the point $x^i_0=x^i(t)$ at the "time"
$t=0$ then the value of the derivative $\nabla_{\bf X}f$ at the point $x^i_0=x^i(0)$
is equal just to the derivative by $t$ of the function $f(x^i(t))$ at the "time" $t=0$:
\begin{equation}\label{meaningofderivative}
{\rm if}\quad
    X^i(x)\big\vert_{x_0=x(0)}={dx^i(t)\over dt}\big\vert_{t=0},\quad
    {\rm then}\quad
\nabla_\X f\big\vert_{x^i=x^i(0)}=
{d\over dt}f\left(x^i\left(t\right)\right)\big\vert_{t=0}
\end{equation}


{\bf Remark} In the course of Geometry and Differentiable Manifolds the operator
 of taking derivation of function along the vector field
was denoted by "$\p_{\bf X}f$". In this course we prefer to denote it by "$\nabla_{\bf X}f$"
to have the uniform notation for both operators of taking derivation of functions and vector fields
along the vector field.


One can see that the operation $\nabla_X$ on the space $C^\infty(M)$ (space of smooth functions on the manifold)
 satisfies the following conditions:

\begin{itemize}
\item
   $\nabla_{\X}\left(\lambda f+\mu g\right)=
   a\nabla_{\X}f+b\nabla_{\X}g$
 where $\lambda,\mu\in \R$ (linearity over numbers )

\item
 $\nabla_{h\X+g\bf Y}(f)=h\nabla_{\X}(f)+g\nabla_{\bf Y}(f)$
                   (linearity over the space of functions)


\item

  $\nabla_\X(\lambda fg)=f\nabla_\X(\lambda g)+g\nabla_\X(\lambda f)$                       (Leibnitz rule)

\begin{equation}\label{conditionsforderivative}
\end{equation}

\end{itemize}

{\bf Remark}
One can prove that these properties characterize  vector fields:operator on smooth functions
obeying the conditions above is a vector field.
(May be you remember a statement similar  to this from the course of Differentiable Manifolds.)

\m

How to define differentiation of vector fields along vector fields.

The formula \eqref{meaningofderivative} cannot be generalised straightforwardly because
vectors at the point $x_0$ and $x_0+t X$ are vectors from different vector spaces.
(We cannot substract the vector from one vector space from the
vector from the another vector space, because {\it apriori}
we cannot compare vectors from different vector space.
One have to define an operation of transport of vectors from the space
$T_{x_0}M$ to the point $T_{x_0+tX}M$ defining the transport
from the point $T_{x_0}M$ to the point $T_{x_0+tX}M$).

Try to define the operation $\nabla$ on vector fields such
that conditions \eqref{conditionsforderivative} above  be satisfied.



\subsubsection{Definition of connection. Christoffel symbols of connection}

{\bf Definition} Affine connection on $M$ is the {\it operation} $\nabla$
which assigns to every vector field $\bf X$ a linear map, (but not $C(M)$-linear map!)
(i.e. a map which is linear over numbers not over functions)
 $\nabla_{\bf X}$
 on the space ${\cal O}(M)$ of
vector fields:
\begin{equation}\label{defofconnection}
  \nabla_{\bf X}\left(\lambda{\bf Y}+\mu{\bf Z}\right)=
   \lambda\nabla_{\bf X}{\bf Y}+\mu\nabla_{\bf X}{\bf Z},\qquad
   \hbox{for every $\lambda,\mu\in \R$}
\end{equation}
(Compare the first condition in \eqref{conditionsforderivative}).

\noindent which satisfies the following conditions:


\begin{itemize}

\item

for arbitrary (smooth) functions $f,g$ on $M$
\begin{equation}\label{linearityonfunctions}
  \nabla_{f\bf X+g\bf Y}\left({\bf Z}\right)=
   f\nabla_{\bf X}\left({\bf Z}\right)+
   g\nabla_{\bf Y}\left({\bf Z}\right)\qquad
   \hbox {($C(M)$-linearity)}
\end{equation}


(compare with second condition in \eqref{conditionsforderivative})

\item
for arbitrary function $f$


\begin{equation}\label{leibnitzrule}
  \nabla_{\bf X} \left( f\bf Y \right)=
   \left(\nabla_{\bf X}f\right){\bf Y}+
   f\nabla_{\bf X}\left({\bf Y}\right)\qquad
   \hbox {(Leibnitz rule)}
\end{equation}
Recall that $\nabla_\X f$ is just usual derivative of a function $f$ along vector field:
$\nabla_\X f=\p_\X f$.

(Compare with Leibnitz rule in \eqref{conditionsforderivative}).

{\it The operation $\nabla_\X\Y$ is called  covariant derivative
of vector field $\Y$ along the vector field $\X$.}


\end{itemize}



  Write down explicit formulae in a given local coordinates $\{x^i\}$ ($i=1,2,\dots,n$) on manifold $M$.

Let
         $$
      \X=X^i\e_i=X^i{\p\over \p x^i}\,\quad \Y=Y^i\e_i=Y^i{\p\over \p x^i}\,\quad
         $$
The basis vector fields ${\p\over x^i}$ we denote sometimes by $\p_i$ sometimes by $\e_i$


  Using properties above one can see that
\begin{equation}\label{explicitexpression}
  \nabla_{\bf X}{\bf Y}=\nabla_{X^i\p_i}{Y^k \p_k}
  =X^i\left(\nabla_i\left(Y^k\p_k\right)\right),\qquad
  \hbox{where $\nabla_i=\nabla_{\p_i}$}
\end{equation}
Then  according to \eqref{linearityonfunctions}
                 $$
           \nabla_i
            \left(
             Y^k \p_k
            \right)=
              \nabla_i
              \left(Y^k\right)\p_k+
            Y^k \nabla_i \p_k
                   $$

 Decompose the vector field  $\nabla_i \p_k$ over the basis $\p_i$:
             \begin{equation}\label{cristoffelinlocalcoordiantes1}
                \nabla_i \p_k=\Gamma_{ik}^m\p_m
              \end{equation}
             and
\begin{equation}\label{covariantderivative}
    \nabla_i\left({Y^k \p_k}\right)=
    {\p Y^k(x)\over \p x^i}{\p}_k+Y^k\Gamma_{ik}^m\p_m,
\end{equation}
   \begin{equation}\label{covariantderivative2}
    \nabla_{\bf X} {\bf Y}=
    X^i{\p Y^m(x)\over \p x^i}\p_m+X^iY^k\Gamma_{ik}^m\p_m,\quad
    \end{equation}

    In components
           \begin{equation}\label{covderivincomponents}
             \left( \nabla_{\bf X} {\bf Y}\right)^m=
    X^i\left({\p Y^m(x)\over \p x^i}+Y^k\Gamma_{ik}^m\right)
           \end{equation}
    Coefficients $\{\Gamma_{ik}^m\}$ are called {\it Christoffel symbols} in coordinates $\{x^i\}$.
These coefficients define covariant derivative---{\bf connection}.


If operation of taking covariant derivative is given we say that the connection is given on the manifold.
Later it will be explained why we us the word "connection"


We see from the formula above that to define covariant derivative of vector fields, connection,
we have to define Christoffel symbols in local coordinates.


\medskip

\subsubsection {Canonical flat affine connection }

  It follows from the properties of connection that it is suffice
   to define connection at vector fields which form basis at the every point
   using \eqref{cristoffelinlocalcoordiantes1}, i.e. to define Christoffel symbols of this connection.

   {\bf Example} Consider $n$-dimensional Euclidean space $\E^n$ with cartesian coordinates
   $\{x^1,\dots,x^n\}$.

   Define connection such that all Christoffel symbols
    are equal to zero in these cartesian coordinates $\{x^i\}$.
\begin{equation}\label{flatconnection1}
  \nabla_{\e_i}\e_k=\Gamma_{ik}^m\e_m=0,\quad   \Gamma_{ik}^m=0
\end{equation}
Does  this mean that Christoffel symbols are equal to zero in
an arbitrary cartesian coordinates if they equal to zero in given cartesian coordinates?

 Does this mean that  Christoffel symbols of this connection equal to zero in arbitrary coordinates system?



 To answer these questions note that the relations \eqref{flatconnection1} mean that
 \begin{equation}\label{flatconnection2}
    \nabla_\X \Y=X^m{\p Y^i\over \p x^m}
    {\p\over \p x^i}
 \end{equation}
in coordinates $\{x^i\}$

Consider  an arbitrary new coordinates $x^{i'}=x^{i'}(x^1,\dots,x^n)$.
Recall the transformation rule  for an arbitrary vector field (see subsection 1.1)
            $$
   {\bf R}=R^{m}{\p \over \p x^{m}}=R^{m}{\p x^{m'}\over \p x^{m}}{\p \over \p x^{m'}}
   \,,\quad {\rm i.e.} R^{m'}={\p x^{m'}\over \p x^{m}}R^m \,,{\rm and}\,,
              R^{m}={\p x^{m}\over \p x^{m'}}R^{m'}\,.
            $$
Hence we have from \eqref{flatconnection2} that
            $$
 \nabla_\X \Y=X^{m}{\p Y^{i}\over \p x^{m}}{\p\over \p x^i}=
 X^{m}{\p \over \p x^{m}}\left(Y^{i}\right){\p\over \p x^i}=
 X^{m}{\p x^{m'}\over \p x^{m}}{\p \over \p x^{m'}}\left({\p x^{i}\over \p x^{i'}}Y^{i'}\right){\p\over \p x^i}=
                     $$
                     $$
                     X^{m'}{\p \over \p x^{m'}}\left({\p x^{i}\over \p x^{i'}}Y^{i'}\right)
                     {\p\over \p x^i}=
         X^{m'}{\p \over \p x^{m'}}\left(Y^{i'}\right)
         {\p x^{i}\over \p x^{i'}}{\p\over \p x^i}+
          X^{m'}{\p^2 x^{i}\over \p x^{m'}\p x^{i'}}
          \left(Y^{i'}\right)
          {\p\over \p x^i}=
                        $$
                        \begin{equation}\label{flatconnecctiontransformation}
         X^{m'}{\p Y^{i'}\over \p x^{m'}}{\p\over \p x^{i'}}+
         \underbrace {X^{m'}{\p^2 x^{i}\over \p x^{m'}\p x^{i'}}Y^{i'}
          {\p\over \p x^i}}_{\hbox {an additional term}}=
                        \end{equation}
We see that an additional term  equals to zero for arbitrary vector fields $\X,\Y$ if and only if
the relations between new and old coordinates are linear:
                    \begin{equation}\label{additionalterm}
                    {\p^2 x^{i}\over \p x^{m'}\p x^{i'}}=0, \,\,{\rm i.e.}\,\,
                    x^i=b^i+a^i_kx^k
                    \end{equation}
 Comparing formulae \eqref{additionalterm} and \eqref{flatconnection2}
 we come to simple but very important

\m

 {\bf Proposition} {\it
   Let all Christoffel symbols of a given connection be equal to zero
 in a given coordinate system $\{x^i\}$.

  All Christoffel symbols of this connection are equal to zero in an arbitrary
 coordinate system  $\{x^{i'}\}$ such that the relations between new and old coordinates are linear:
               \begin{equation}\label{additionalterm1}
                   x^i=b^i+a^i_kx^k
                    \end{equation}
 If transformation to new coordinate system is not linear, i.e.  ${\p^2 x^{i}\over \p x^{m'}\p x^{i'}}\not=0$
 then Christoffel symbols of this connection are not equal to zero in new
 coordinate system  $\{x^{i'}\}$.}

\m

{\bf Definition} We call connection $\nabla$ flat if there exists coordinate system such that
all Christoffel symbols of this connection are equal to zero in a given coordinate system.

\m

In particular connection \eqref{flatconnection1} has zero Christoffel symbols in arbitrary cartesian coordinates.



{\bf Corollary} Connection has zero Christoffel symbols in arbitrary cartesian coordinates
if it has zero Christoffel symbols in a given cartesian coordinates.


Hence the following definition is correct:

\m


{\bf Definition}  A connection on $\E^n$ which Christoffel symbols vanish in cartesian coordinates
is called {\it canonical flat connection.}


{\bf Remark} {\small   Canonical flat connection in Euclidean space is uniquely defined, sincce cartesian coordinates
are defined globally.  On the other hand
on arbitrary manifold one can define flat connection locally just choosing any arbitrary {\it local}
coordinates and define {\it locally flat connection} by condition that Christoffel symbols
vanish in these local coordinates.  We will study this question after learning transformation law for Christoffel symbols.}




\subsubsection {Transformation of Christoffel symbols for an arbitrary connection}

 Let $\nabla$ be a connection on manifold $M$.
  Let $\{\Gamma^i_{km}\}$ be Christoffel symbols of this connection in  given local coordinates $\{x^i\}$.
  Then according \eqref{cristoffelinlocalcoordiantes1} and \eqref{covariantderivative} we have
                 $$
              \nabla_\X\Y=X^m{\p Y^i\over \p x^m}{\p\over \p x^i}+X^m\Gamma^i_{mk}Y^k{\p\over \p x^i},
                 $$
and in particularly
               $$
               \Gamma^i_{mk}\p_i=\nabla_{\p_m}\p_k
               $$
  Use this relation to calculate Christoffel symbols in new coordinates $x^{i'}$
               $$
            \Gamma^{i'}_{m'k'}\p_{i'}=
            \nabla_{\p_m'}\p_{k'}
               $$
  We have that $\p_m'={\p\over \p x^{m'}}={\p x^m\over \p x^{m'}}{\p\over \p x^{m}}= {\p x^m\over \p x^{m'}}\p_m $.
  Hence due to properties \eqref{linearityonfunctions}, \eqref{leibnitzrule} we have
    $$
          \Gamma^{i'}_{m'k'}\p_{i'}=
            \nabla_{\p_m'}\p_{k'}=\nabla_{\p_m'}\left({\p x^k\over \p x^{k'}}\p_{k}\right)=
            \left({\p x^k\over \p x^{k'}}\right)\nabla_{\p_m'}\p_{k}+
            {\p\over \p x^{m'}}\left({\p x^k\over \p x^{k'}}\right)\p_{k}=
    $$
              $$
    \left({\p x^k\over \p x^{k'}}\right)\nabla_{{\p x^m\over \p x^{m'}}\p_m}\p_{k}+
            {\p^2  x^k\over \p x^{m'}\p x^{k'}}\p_{k}=
                      {\p x^k\over \p x^{k'}}
          {\p x^m\over \p x^{m'}}
          \nabla_{\p_m}\p_{k}+
           {\p^2  x^k\over \p x^{m'}\p x^{k'}}\p_{k}
          $$
          $$
     {\p x^k\over \p x^{k'}}
          {\p x^m\over \p x^{m'}}
          \Gamma_{mk}^i\p_i+{\p^2  x^k\over \p x^{m'}\p x^{k'}}\p_{k}=
          {\p x^k\over \p x^{k'}}
          {\p x^m\over \p x^{m'}}\Gamma_{mk}^i{\p x^{i'}\over \p x^i}\p_{i'}+
          {\p^2  x^k\over \p x^{m'}\p x^{k'}}{\p x^{i'}\over \p x^k}\p_{i'}
          $$
Comparing the first and the last term in this formula we come to the transformation law:

\m

If $\{\Gamma^i_{km}\}$ are Christoffel symbols of the connection $\nabla$ in  local coordinates $\{x^i\}$
and $\{\Gamma^{i'}_{k'm'}\}$ are Christoffel symbols of this connection  in  new local coordinates $\{x^{i'}\}$
then
 \begin{equation}\label{formualfortransformationofconnection}
    \Gamma^{i'}_{k'm'}=
          {\p x^k\over \p x^{k'}}
          {\p x^m\over \p x^{m'}}
          {\p x^{i'}\over \p x^i}
          \Gamma_{mk}^i+
    {\p^2  x^k\over \p x^{m'}\p x^{k'}}{\p x^{i'}\over \p x^k}
 \end{equation}

{\bf Remark}  Christoffel symbols do not transform as tensor.
If the second term is equal to zero, i.e. transformation of coordinates are linear
(see the Proposition on flat connections)  then the transformation rule above is the the same as a
transformation rule for tensors of the type $\begin{pmatrix} 1\cr 2\cr\end{pmatrix}$
(see the formula \eqref{ruleoftransformationofarbitrarytensors}).
 In general case this is not true. Christoffel symbols does not transform as tensor
 under arbitrary non-linear coordinate transformation: see the second term in the formula above.


 \m

 \m



  {\bf Example} Consider a connection \eqref{flatconnection1} in $\E^2$. It is a flat connection.
            Calculate Christoffel symbols of this connection in polar coordinates
                       \begin{equation}\label{polarcoordinates}
                        \begin{cases}
                        x=r\cos\varphi\cr
                        y=y\sin\varphi\cr
                        \end{cases}
                        \qquad
                        \begin{cases}
                        r=\sqrt {x^2+y^2}\cr
                        \varphi={\rm arctan\,} {y\over x}\cr
                           \end{cases}
                                \end{equation}
Write down Jacobians of transformations---matrices of partial derivatives:
                      \begin{equation}\label{polarcoordinates2}
                      \begin{pmatrix}
                      x_r &y_r\cr
                      x_\varphi &y_\varphi\cr
                      \end{pmatrix}=
                      \begin{pmatrix}
                      \cos\varphi &\sin\varphi\cr
                      -r\sin\varphi & r\cos\varphi\cr
                      \end{pmatrix},\qquad
                      \begin{pmatrix}
                      r_x &\varphi_x\cr
                      r_y &\varphi_y\cr
                      \end{pmatrix}=
                      \begin{pmatrix}
                      {x\over \sqrt{x^2+y^2}} &-{y\over {x^2+y^2}}\cr
                      {y\over \sqrt{x^2+y^2}} & {x\over {x^2+y^2}}\cr
                      \end{pmatrix}
                      \end{equation}
According \eqref{formualfortransformationofconnection} and since Chrsitoffel symbols are equal to zero in cartesian coordinates
$(x,y)$ we have
             \begin{equation}\label{christoffelsymbolsinpolarcoordinates}
\Gamma^{i'}_{k'm'}=
    {\p^2  x^k\over \p x^{m'}\p x^{k'}}{\p x^{i'}\over \p x^k},
                             \end{equation}
           where $(x^{1},x^{2})=(x,y)$ and $(x^{1'},x^{2'})=(r,\varphi)$. Now using \eqref{polarcoordinates2}
           we have
            $$
         \Gamma^r_{rr}={\p^2 x\over \p r\p r}{\p r\over \p x}+{\p^2 y\over \p r\p r}{\p r\over \p y}=0
            $$
              $$
        \Gamma^r_{r \varphi}=\Gamma^r_{\varphi r}={\p^2 x\over \p r\p \varphi}{\p r\over \p x}+
        {\p^2 y\over \p r\p \varphi}{\p r\over \p y}=-\sin\varphi\cos\varphi+\sin\varphi\cos\varphi=0\,.
              $$
               $$
           \Gamma^r_{\varphi \varphi}={\p^2 x\over \p \varphi\p \varphi}{\p r\over \p x}+
        {\p^2 y\over \p r\p \varphi}{\p r\over \p y}=-x{x\over r}-y{y\over r}=-r\,.
               $$
                $$
                 \Gamma^\varphi_{rr}=
        {\p^2 x\over \p r\p r}{\p \varphi\over \p x}+{\p^2 y\over \p r\p r}{\p \varphi\over \p y}=0\,.
                $$
                $$
     \Gamma^\varphi_{\varphi r}=
     \Gamma^\varphi_{r\varphi}=
        {\p^2 x\over \p r\p \varphi}{\p \varphi\over \p x}+{\p^2 y\over \p r\p \varphi}{\p \varphi\over \p y}=
        -\sin\varphi{-y\over r^2}+\cos\varphi{x\over r^2}={1\over r}
                $$
             \begin{equation}\label{clhristoffelsymbolsinpolarcoordinates2}
                \Gamma^\varphi_{\varphi\varphi}=
        {\p^2 x\over \p \varphi\p \varphi}{\p \varphi\over \p x}+
        {\p^2 y\over \p \varphi\p \varphi}{\p \varphi\over \p y}=
        -x{-x\over r^2}-y{y\over r^2}=0\,.
                     \end{equation}
       Hence  we have that the covariant derivative \eqref{flatconnection2}
        in polar coordinates has the following appearance
                  \begin{equation*}
                  \nabla_r \p_r=
                  \Gamma_{rr}^r\p_r+
                  \Gamma_{rr}^\varphi\p_\varphi=0\,,
                  , \qquad
                  \nabla_r\p_\varphi=\Gamma_{r\varphi}^r\p_r+
                  \Gamma_{r\varphi}^\varphi\p_\varphi={\p_\varphi\over r}
                  \end{equation*}
                  \begin{equation}\label{covariantderivativeinpolarcoordinates}
                  \nabla_\varphi \p_r=
                  \Gamma_{\varphi r}^r\p_r+
                  \Gamma_{\varphi r}^\varphi\p_\varphi={\p_\varphi\over r}, \qquad
                  \nabla_\varphi\p_\varphi=\Gamma_{\varphi\varphi}^r\p_r+
                  \Gamma_{\varphi\varphi}^\varphi\p_\varphi=-r\p_r
                  \end{equation}
{\bf Remark}  Later when we study geodesics we will learn a very quick method to calculate
Christoffel symbols.

\subsubsection {$^*$ Global aspects of existence of connection}

{\small We defined connection as an operation on vector fields obeying the special axioms (see the subsubsection 2.1.1).
Then we showed that in a given coordinates
connection is defined by Christoffel symbols. On the other hand we know that in general
coordinates on manifold are not defined  globally. (We had not this trouble in Euclidean space where there are globally
defined cartesian coordinates.)

\begin{itemize}

\item How to define connection globally using local coordinates?

\item Does there exist at least one globally defined connection?

\item   Does there exist globally defined flat connection?

  \end{itemize}

  These questions are not  naive questions.   Answer on first and second questions is "Yes".
   It sounds bizzare but answer on the first question is not "Yes" \footnote{
    Topology of the manifold can be an obstruction to existence
of global flat connection.  E.g. it does not exist on sphere $S^n$ if $n>1$.}


     \centerline{Global definition of connection}

  The formula  \eqref{formualfortransformationofconnection} defines the transformation for
  Christoffer symbols if we go from one coordinates to another.


  Let $\{(x^i_\a), U_\a\}$ be an atlas of charts on the manifold $M$.

  If connection $\nabla$ is defined on the manifold $M$ then
  it defines in any chart (local coordinates) $(x^i_\a)$ Christoffer symbols
  which we denote by $_{(\a)}\Gamma^{i}_{km}$.
  If $(x^i_\a)$, $(x^{i'}_{(\beta)})$ are different local coordinates in a vicinity of a given point then
  according to \eqref{formualfortransformationofconnection}
     \begin{equation}\label{symbolsinlocalchart}
                        _{(\beta)}\Gamma^{i'}_{k'm'}=
          {\p x^k_{(\a)}\over \p x^{k'}_{(\beta)}}
          {\p x^m_{(\a)}\over \p x^{m'}_{(\beta)}}
          {\p x^{i'}_{(\beta)}\over \p x^i_{(\a)}}
          _{(\beta)}\Gamma_{mk}^{(\alpha)i}+
      {\p^2  x^{k}_{(\a)}\over \p x^{m'}_{(\beta)}\p x^{k'}_{(\beta)}}
                  {\p x^{i'}_{(\beta)}\over \p x^{k}_{(\a)}}
                 \end{equation}


  {\bf Definition}  Let $\{(x^i_\a), U_\a\}$ be an atlas of charts on the manifold $M$

  We say that the collection of Christoffel symbols  $\{\Gamma^{(\a)i}_{km}\}$
  defines globally a connection on the manifold $M$ in this atlas if for every two local coordinates
    $(x^i_{(\a)})$, $(x^i_{(\beta)})$ from this atlas the transformation rules  \eqref{symbolsinlocalchart}
    are obeyed.



  \m

  Using partition of unity one can prove the existence of global connection constructing it in explicit way.
  Let $\{(x^i_\a), U_\a\}$ $(\a=1,2,\dots,N)$ be a finite  atlas on the manifold $M$
  and let $\{\rho_\a\}$ be a partition of unity adjusted to this atlas.
    Denote by  $^{(\a)}\Gamma^i_{km}$ local connection defined in domain $U_\a$
    such that its components in these coordinates are equal to zero.
  Denote by $_{(\beta)}^{(\a)}\Gamma^i_{km}$ Christoffel symbols of this local connection
  in coordinates $(x^i_{(\beta)})$ ($_{(\beta)}^{(\a)}\Gamma^i_{km}=0$).
  Now one can define globally the connection by the formula:
                      \begin{equation}\label{formulaforglobalconnection}
                      {}_{(\beta)}\Gamma^i_{km}({\bf x})=
                                   \sum_\a \rho_\a(\x)\,\,{}^{(\a)}_{(\beta)}\Gamma^{i}_{km}(\x)=
           \sum_\a \rho_\a(\x) {\p x^{i}_{(\beta)}\over \p x^{i'}_{(\a)}}
              {\p^2 x^{i'}_{(\a)}(\x)\over \p x^{k}_{(\beta)}\p x^{m}_{(\beta)}}\,.
                      \end{equation}

 This connection in general is not flat connection\footnote{
 See for detail the text: "Global affine connection on manifold"
 " in my homepage: "www.maths.mancheser.ac.uk/khudian" in subdirectory Etudes/Geometry}

}


\subsection {Connection induced on the surfaces}

   Let $M$ be a manifold (surface) embedded in Euclidean space\footnote{We know that every $n$-dimensional manifodl can be embedded
    in $2n+1$-dimensional Euclidean space}.
    Canonical flat connection on $\E^N$  induces the connection on surface in the following way.

    Let $\X,\Y$ be tangent vector fields to the surface $M$ and $\nabla^{\rm can.flat}$ a canonical flat
    connection in $\E^N$.
    In general
            \begin{equation}\label{nottangent!}
                \Z=    \nabla^{\rm can.flat}_\X\Y\quad \hbox{is not tangent to manifold $M$}
                \end{equation}
                Consider its decomposition on two vector fields:
                          \begin{equation}\label{nottangent!2}
             \Z=\Z_{tangent}+\Z_\bot,       \nabla^{\rm can.flat}_\X,\Y=
                    \left(\nabla^{\rm can.flat}_\X\Y\right)_{tangent}+
                    \left(\nabla^{\rm can.flat}_\X\Y\right)_{\bot}\,,
                \end{equation}
where $\Z_{\bot}$ is a component of vector which is orthogonal to the surface $M$ and $\Z_{||}$ is a component which
is tangent to the surface. Define an induced connection $\nabla^{M}$ on the surface $M$ by the following formula
\begin{equation}\label{inducedconnection1}
\nabla^{M}\colon\qquad \nabla^{M}_\X\Y \colon =\left(\nabla^{\rm can.flat}_\X\Y\right)_{tangent}
\end{equation}

{\bf Remark} One can imply this construction for an arbitrary connection in $\E^N$.

\subsubsection {Calculation of induced connection on surfaces in $\E^3$.}

Let $\r=\r(u,v)$ be a surface in $\E^3$. Let $\nabla^{\rm can.flat}$ be a flat connection in $\E^3$.
   Then
\begin{equation}\label{inducedconnection1}
\nabla^{M}\colon\qquad \nabla^{M}_\X\Y \colon =\left(\nabla^{\rm can.flat}_\X\Y\right)_{||}=
\nabla^{\rm can.flat}_\X\Y-
\n(\nabla^{\rm can.flat}_\X\Y,\n),
\end{equation}
where $\n$ is normal unit vector field to $M$. Consider a special example

  {\bf Example} (Induced connection on sphere)
   Consider a sphere of the radius $R$ in $\E^3$:
              $$
           \r(\theta,\varphi)\colon \begin{cases}
           x=R\sin\theta \cos\varphi\cr
           y=R\sin\theta \sin\varphi\cr
           z=R\cos\theta\cr
           \end{cases}
              $$
then
           $$
      \r_{\theta}=\begin{pmatrix}
            R\cos\theta\cos\varphi\cr
            R\cos\theta\sin\varphi\cr
            -R\sin\theta\cr
                  \end{pmatrix},
       \r_{\varphi}=\begin{pmatrix}
            -R\sin\theta\sin\varphi\cr
            R\sin\theta\cos\varphi\cr
            0\cr
                  \end{pmatrix},
                  \n=\begin{pmatrix}
            sin\theta\cos\varphi\cr
            sin\theta\sin\varphi\cr
            \cos\theta\cr
                  \end{pmatrix},
           $$
where $\r_\theta={\p \r(\theta,\varphi)\over \p \theta}$,
$\r_\varphi={\p \r(\theta,\varphi)\over \p \varphi}$ are basic tangent vectors and $\n$ is normal unit vector.

Calculate an induced connection  $\nabla$ on the sphere.

First calculate $\nabla_{\p_\theta}\p_\theta$.
\begin{equation*}
    \nabla_{\p_\theta}\p_\theta=\left({\p \r_\theta\over \p\theta}\right)_{tangent}=
    \left(\r_{\theta\theta}\right)_{tangent}.
\end{equation*}
On the other hand one can see that $\r_{\theta\theta}=\begin{pmatrix}
            -Rsin\theta\cos\varphi\cr
            -Rsin\theta\sin\varphi\cr
            -R\cos\theta\cr
                  \end{pmatrix}=-R\n$ is proportional to normal vector, i.e. $\left(\r_{\theta\theta}\right)_{tangent}=0$.
                  We come to
\begin{equation}\label{inducedonsphere4}
    \nabla_{\p_\theta}\p_\theta=\left(\r_{\theta\theta}\right)_{tangent}=0 \Rightarrow
    \Gamma^\theta_{\theta\theta}=\Gamma^\varphi_{\theta\theta}=0\,.
\end{equation}

Now calculate $\nabla_{\p_\theta}\p_\varphi$ and $\nabla_{\p_\varphi}\p_\theta$.
\begin{equation*}
    \nabla_{\p_\theta}\p_\varphi=\left({\p \r_\varphi\over \p\theta}\right)_{tangent}=
    \left(\r_{\theta\varphi}\right)_{tangent},\,
    \nabla_{\p_\varphi}\p_\theta=\left({\p \r_\theta\over \p\varphi}\right)_{tangent}=
    \left(\r_{\varphi\theta}\right)_{tangent}
\end{equation*}
 We have
           $$
           \nabla_{\p_\theta}\p_\varphi=\nabla_{\p_\varphi}\p_\theta=\left(\r_{\varphi\theta}\right)_{tangent}=
           \begin{pmatrix}
            -R\cos\theta\sin\varphi\cr
            R\cos\theta\cos\varphi\cr
            0\cr
                  \end{pmatrix}_{tangent}.
           $$
We see that the vector $\r_{\varphi\theta}$ is orthogonal to $\n$:
\begin{equation*}
     \langle\r_{\varphi\theta},\n\rangle=-R\cos\theta\sin\varphi \sin\theta\cos\varphi+
     R\cos\theta\cos\varphi\sin\theta\sin\varphi=0.
\end{equation*}
Hence
\begin{equation*}
 \nabla_{\p_\theta}\p_\varphi=\nabla_{\p_\varphi}\p_\theta=
 \left(\r_{\varphi\theta}\right)_{tangent}=\r_{\varphi\theta}=
 \begin{pmatrix}
            -R\cos\theta\sin\varphi\cr
            R\cos\theta\cos\varphi\cr
            0\cr
                  \end{pmatrix}=
                  {\rm cotan\,}\theta  \r_\varphi\,.
\end{equation*}
We come to

\begin{equation}\label{inducedonsphere8}
    \nabla_{\p_\theta}\p_\varphi=\nabla_{\p_\varphi}\p_\theta={\rm cotan\,}\theta  \p_\varphi \Rightarrow
    \Gamma^\theta_{\theta\varphi}=\Gamma^\theta_{\varphi\theta}=0,\,\,
    \Gamma^\varphi_{\theta\varphi}=\Gamma^\varphi_{\varphi\theta}={\rm cotan\,}\theta
\end{equation}

Finally calculate $\nabla_{\p_\varphi} \p_\varphi$
                $$
            \nabla_{\p_\varphi} \p_\varphi=\left(\r_{\varphi\varphi}\right)_{tangent}=
            \left(
            \begin{pmatrix}
            -R\sin\theta\cos\varphi\cr
            -R\sin\theta\sin\varphi\cr
            0\cr
                  \end{pmatrix}
                  \right)_{tangent}
                $$
Projecting on the tangent vectors to the sphere (see \eqref{inducedconnection1}) we have
     $$
\nabla_{\p_\varphi} \p_\varphi=\left(\r_{\varphi\varphi}\right)_{tangent}=
\r_{\varphi\varphi}-\n\langle\n,\r_{\varphi}\rangle=
         $$
         $$
     \begin{pmatrix}
            -R\sin\theta\cos\varphi\cr
            -R\sin\theta\sin\varphi\cr
            0\cr
                  \end{pmatrix}-
                  \begin{pmatrix}
            \sin\theta\cos\varphi\cr
            \sin\theta\sin\varphi\cr
            \cos\theta\cr
                  \end{pmatrix}\left(-R\sin\theta\cos\varphi\sin\theta\cos\varphi-
                  R\sin\theta\sin\varphi\sin\theta\sin\varphi\right)=
     $$
         $$
      -\sin\theta\cos\theta
      \begin{pmatrix}
            R\cos\theta\cos\varphi\cr
            R\cos\theta\sin\varphi\cr
            -R\sin\theta\cr
                  \end{pmatrix}=-\sin\theta\cos\theta\r_\theta,
         $$
i.e.
\begin{equation}\label{connectiononsphere12}
   \nabla_{\p_\varphi}\p_\varphi=-\sin\theta\cos\theta\r_\theta \Rightarrow
    \Gamma^\theta_{\varphi\varphi}=-\sin\theta\cos\theta,\,\,
    \Gamma^\varphi_{\varphi\varphi}=\Gamma^\varphi_{\varphi\varphi}=0\,.
\end{equation}



\subsection {Levi-Civita connection}


  \subsubsection {Symmetric connection}

 {\bf definition}. We say that connection is symmetric if its Christoffel symbols $\Gamma^i_{km}$ are symmetric with respect to
  lower indices
  \begin{equation}\label{symmetricconnection}
    \Gamma^i_{km}=\Gamma^i_{mk}
  \end{equation}
  The canonical flat connection and induced connections considered above are symmetric connections.

  \m

  {\small

   \centerline {\it Invariant definition of symmetric connection}

  A connection $\nabla$ is symmetric if for an arbitrary vector fields $\X,\Y$

  \begin{equation}\label{symmetricconnectioninvariant}
    \nabla_\X \Y-\nabla_\Y \X-[\X,\Y]=0
  \end{equation}
  If we apply this definition to basic fields $\p_k,\p_m$ which commute: $[\p_k,\p_m]=0$
  we come to the
  condition
          $$
     \nabla_{\p_k}\p_m-\nabla_{\p_m}\p_k=\Gamma_{mk}^i\p_i-\Gamma_{km}^i\p_i=0
          $$
 and this is the condition  \eqref{symmetricconnection}.
  }



  \subsubsection {Levi-Civita connection. Theorem and Explicit formulae}
Let $(M, G)$ be a Riemannian manifold.

{\bf Definition. Theorem}

{\it A symmetric connection $\nabla$ is called Levi-Civita connection if it is compatible with metric, i.e.
if it preserves scalar product:
        \begin{equation}\label{connpreservingmetric}
  \p_\X\langle\Y,\Z\rangle=\langle\nabla_\X\Y,\Z\rangle+\langle\Y,\nabla_\X\Z\rangle
        \end{equation}
        for arbitrary vector fields $\X,\Y,\Z$.

  There exists unique levi-Civita connection on the Riemannian manifold.

  In local coordinates Christoffel symbols of Levi-Civita connection are given by the following formulae:
 \begin{equation}\label{levi-civitaformula}
    \Gamma^i_{mk}={1\over 2}g^{ij}\left({\p g_{jm}\over\p x^k}+
    {\p g_{jk}\over\p x^m}-{\p g_{mk}\over \p x^j}\right)\,.
 \end{equation}
 where $G=g_{ik}dx^idx^k$ is Riemannian metric in local coordinates and
 $||g^{ik}||$ is the matrix inverse to the matrix $||g_{ik}||$.

 }

 {\sl Proof}

{\small Suppose that this connection exists and $\Gamma^i_{mk}$ are its Christoffel symbols.
 Consider vector fields $\X=\p_m,\Y=\p_i$ and $\Z=\p_k$ in \eqref{connpreservingmetric}.
  We have that
            \begin{equation}\label{conditionforallindices}
         \p_mg_{ik}=\langle\Gamma_{mi}^r\p_r,\p_k\rangle+\langle\p_i,\Gamma^r_{mk}\p_r\rangle=
              \Gamma^r_{mi}g_{rk}+g_{ir}\Gamma^r_{mk}\,.
            \end{equation}
   for arbitrary indices $m,i,k$.

    Denote by $\Gamma_{mik}=\Gamma^r_{mi}g_{rk}$ we come to
                $$
          \p_mg_{ik}=\Gamma_{mik}+\Gamma_{mki}, \,\,{\rm i.e.}
                $$
    Now using the symmetricity $\Gamma_{mik}=\Gamma_{imk}$ since $\Gamma_{mi}^k=\Gamma_{im}^k$ we have
                $$
    \Gamma_{mik}=\p_mg_{ik}-\Gamma_{mki}=\p_mg_{ik}-\Gamma_{kmi}=\p_mg_{ik}-\left(\p_kg_{mi}-\Gamma_{kim}\right)=
                $$
                $$
\p_mg_{ik}-\p_kg_{mi}+\Gamma_{kim}=\p_mg_{ik}-\p_kg_{mi}+\Gamma_{ikm}=
\p_mg_{ik}-\p_kg_{mi}+\left(\p_ig_{km}-\Gamma_{imk}\right)=
                $$
                $$
       \p_mg_{ik}-\p_kg_{mi}+\p_ig_{km}-\Gamma_{mik}\,.
                $$
 Hence
                \begin{equation}\label{levi-vivitaforlowerindices}
   \Gamma_{mik}={1\over 2}(\p_mg_{ik}+\p_ig_{mk}-\p_kg_{mi})\Rightarrow \Gamma^k_{im}=
   {1\over 2}g^{kr}\left(\p_mg_{ir}+\p_ig_{mr}-\p_rg_{mi}\right)
                \end{equation}
  We see that if this connection exists then it is given by the formula\eqref{levi-civitaformula}.

  On the other hand one can see that \eqref{levi-civitaformula} obeys the condition
  \eqref{conditionforallindices}. We prove the uniqueness and existence.

since $\nabla_{\p_i}\p_k=\Gamma_{ik}^m\p_m$.}

\m

Consider examples.

\m

\subsubsection {Levi-Civita connection on $2$-dimensional Riemannian
manifold with metric $G=adu^2+bdv^2$.}

{\bf Example} Consider $2$-dimensional manifold with Riemannian metrics
                  $$
                  G=a(u,v)du^2+b(u,v)dv^2, \qquad
                  G=\begin{pmatrix}
                       g_{11} &g_{12}\\
                       g_{21}  &g_{22}\\
                     \end{pmatrix}=
         \begin{pmatrix}
                       a(u,v) &0\\
                       0  &b(u,v)\\
                     \end{pmatrix}
                  $$
Calculate Christoffel symbols of Levi Civita connection.

 Using  \eqref{levi-vivitaforlowerindices} we see that:
                       \begin{equation}\label{explicitcalculationofchristoffel}
                       \begin{matrix}
        \Gamma_{111}&={1\over 2}\left(\p_{1} g_{11}+\p_{1} g_{11}-\p_{1} g_{11}\right)=
        {1\over 2}\p_{1} g_{11}=&{1\over 2}a_u\\
             &\\
        \Gamma_{211}=\Gamma_{121}&={1\over 2}\left(\p_{1} g_{12}+
        \p_{2} g_{11}-\p_{1} g_{12}\right)=
        {1\over 2}\p_{2} g_{11}=&{1\over 2}a_v\\
           &\\
\Gamma_{221}&={1\over 2}\left(\p_{2} g_{12}+\p_{2} g_{12}-\p_{1} g_{22}\right)=
        -{1\over 2}\p_{1} g_{22}=-&{1\over 2}b_u\\ &\\
        \Gamma_{112}&={1\over 2}\left(\p_{1} g_{12}+\p_{1} g_{12}-\p_{2} g_{11}\right)=
        -{1\over 2}\p_{2} g_{11}=-&{1\over 2}a_v\\&\\
        \Gamma_{122}=\Gamma_{212}&=
        {1\over 2}\left(\p_{2} g_{21}+\p_{1} g_{22}-\p_{2} g_{21}\right)=
        {1\over 2}\p_{1} g_{22}=&{1\over 2}b_u\\&\\
        \Gamma_{222}&={1\over 2}\left(\p_{2} g_{22}+\p_{2} g_{22}-\p_{2} g_{22}\right)=
        {1\over 2}\p_{2} g_{22}=&{1\over 2}b_v\\
                           \end {matrix}
                       \end{equation}
To calculate $\G^i_{km}=g^{ir}\Gamma_{kmr}$ note that for the metric
$a(u,v)du^2+b(u,v)dv^2$
        $$
          G^{-1}=\begin{pmatrix}
                       g^{11} &g^{12}\\
                       g^{21}  &g^{22}\\
                     \end{pmatrix}=
         \begin{pmatrix}
                       {1\over a(u,v)} &0\\
                       0  &{1\over b(u,v)}\\
                     \end{pmatrix}
              $$
Hence
\begin{equation}\label{christophelfortwodimensionalmetric}
      \begin{matrix}
  \Gamma^{1}_{11}=&g^{11}\Gamma_{111}=&{a_u\over 2a},\qquad
  \Gamma^1_{21}=\Gamma^{1}_{12}=&g^{11}\Gamma_{121}=&{a_v\over 2a},\qquad
  \Gamma^{1}_{22}=&g^{11}\Gamma_{221}=&{-b_u\over 2a}\\&&&&&&\\
  \Gamma^{2}_{11}=&g^{22}\Gamma_{112}=&{-a_v\over 2b},\qquad
  \Gamma^2_{21}=\Gamma^{2}_{12}=&g^{22}\Gamma_{122}=&{b_u\over 2b},\qquad
  \Gamma^{2}_{22}=&g^{22}\Gamma_{222}=&{b_v\over 2b}\\
  \end{matrix}
\end{equation}

\subsubsection {Example of the sphere again}

 {Calculate Levi-Civita connection on the sphere.

 On the sphere first quadratic form (Riemannian metric)
 $G=R^2d\theta^2+R^2\sin^2\theta d\varphi^2$.  Hence we
 use calculations from the previous example
 with $a(\theta,\varphi)=R^2, b(\theta,\varphi)=R^2\sin^2 \theta$
  ($u=\theta, v=\varphi$).
 Note that $a_\theta=a_\varphi=b_\varphi=0$.
  Hence only non-trivial components
 of $\Gamma$ will be:
\begin{equation}\label{Christoffelforsphere:}
  \Gamma^{\theta}_{\varphi\varphi}={-b_\theta\over 2a}={-\sin 2\theta\over 2},
  \qquad
  \left(\Gamma_{\varphi\varphi\theta}={-R^2\sin 2\theta\over 2}\right),
       \end{equation}
  \begin{equation}\label{connectionforsphere}
  \Gamma^{\varphi}_{\theta\varphi}=\Gamma^{\varphi}_{\varphi\theta}=
  {b_\theta\over 2b}={\cos\theta \over \sin\theta}\,
  \qquad
  \left(\Gamma_{\theta\varphi\varphi}={R^2\sin 2\theta\over 2}\right)
\end{equation}
All other components are equal to zero:
                 $$
  \Gamma^{\theta}_{\theta\theta}=\Gamma^{\theta}_{\theta\varphi}=\Gamma^{\theta}_{\varphi\theta}=
  \Gamma^{\varphi}_{\theta\theta}=\Gamma^{\varphi}_{\varphi\varphi}=0
             $$

{\bf Remark} Note that Christoffel symbols of Levi-Civita connection on the sphere coincide
with  Christoffel symbols of induced connection calculated
in the subsection "Connection induced on surfaces".
later we will understand the geometrical meaning of this fact.






   \section {Parallel transport and geodesics}

   \subsection{Parallel transport}

\subsubsection {Definition}

   Let $M$ be a manifold equipped with affine connection $\nabla$.

   {\bf Definition} Let $C\colon \x(t)$ with coordinates
   $x^i=x^i(t),\,\, t_0\leq t\leq t_1$ be a curve on the manifold $M$.
   Let $\X=\X(t_0)$ be a vector attached at the initial point
  $\x_0$ with coordinates $x^i(t_0)$ of the curve $C$, i.e.
  $\X(t_0)\in T_{\x_0}M$ is a vector tangent to the manifold $M$
  at the point $\x_0$ with coordinates $x^i(t_0)$.

  We say that $\X(t)$, $t_0\leq t\leq t_1$ is a parallel transport of the
  vector $\X(t_0)\in T_{\x_0}M$ along the curve $C\colon x^i=x^i(t), t_0\leq t\leq t_1$ if

\begin {itemize}

\item For an arbitrary $t$, $t_0\leq t\leq t$, vector $\X=\X(t)$, ($\X(t)\vert_{t=t_0}=\X(t_0)$)
 is a vector attached at the point
  $\x(t)$ of the curve $C$,
 i.e. $\X(t)$ is a vector tangent to the manifold $M$
  at the point $\x(t)$  of the curve $C$.

  \item The covariant derivative of $\X(t)$ along the curve $C$ equals to zero:

  \begin{equation}\label{conditofparalleltransport}
    {\nabla \X\over dt}=\nabla_\v \X=0\,.
  \end{equation}
 In components: if $X^m(t)$ are components of the vector field $\X(t)$
 and   $v^m(t)$ are components of the velocity vector $\v$ of the curve  $C$ ,
                $$
                \X(t)=X^m(t){\p \over \p x^m}\vert_{\x(t)}\,,\quad
                \v={d\x(t)\over dt}={dx^i\over dt}{\p \over \p x^m}\vert_{\x(t)}
                $$
  then the condition \eqref{conditofparalleltransport} can be rewritten as
  \begin{equation}\label{conditofparalleltransportcomponents}
    {dX^i(t)\over dt}+v^k(t)\Gamma^i_{km}(x^i(t))X^m(t)\equiv 0\,.
  \end{equation}


\end{itemize}

\m

{\bf Remark}    We say sometimes that $\X(t)$ is {\it covariantly constant along the curve $C$}
   If $\X(t)$ is parallel transport of the vector $\X$ along the curve $C$.
  If we consider Euclidean space with canonical flat connection then in Cartesian coordinates
  Christoffel symbols vanish and parallel transport is nothing but
   ${d \X\over dt}=\nabla_\v \X=0$, $\X(t)$ is constant vector.



  \m


{\bf Remark} Compare this definition of parallel transport with the definition which we consider
in the course of "Introduction to Geometry" where we consider parallel transport of the vector along the
curve on the surface embedded in $\E^3$ and define parallel transport by the condition, that only orthogonal component
of the vector changes during parallel transport, i.e.
${d\X(t)\over dt}$ is a vector orthogonal to the surface.

\subsubsection {$^*$Parallel transport is a linear map }

{\small Consider two different points $\x_0,\x_1$ on the manifold  $M$ with connection $\nabla$.
Let $C$ be a curve $\x(t)$ joining these points.
The parallel transport \eqref{conditofparalleltransport}  $\X(t)$ defines the map between  tangent vectors
at the point $\x_0$ and tangent vectors at the point $\x_1$. This map depends on the curve $C$. Parallel
transport along different curves joining the same points is in general different (if we are not in Euclidean space).

On the other hand parallel transport is a linear map of tangent spaces which {\it does not depend on the
parameterisation of the curve} joining these points.

\m

{\bf Proposition}  {\it Let $\X(t), t_0\leq t\leq t_1$ be a parallel transport of the vector $\X(t_0)\in T_{\x_0}M$
along the curve $C\colon \x=\x(t), t_0\leq t\leq t_1$, joining the points $\x_0=\x(t_0)$ and $\x_1=\x(t_1)$.
 Then the map
      \begin{equation}\label{paralleltransportalongthecurve}
 \tau_{C}\colon \quad   T_{\x_0}M \ni  \X(t_0)\longrightarrow  \X(t_1)\in T_{\x_1}M
               \end{equation}
 is a linear map from the vector space  $T_{\x_0}M$ to the vector space $T_{\x_1}M$
   which does not depend on the parametersiation of the curve $C$.}

The fact that it is a linear map follows immediately from the construction.

The fact that the map \eqref{paralleltransportalongthecurve} does not depend on the parameterisation
follows from the differential equation \eqref{conditofparalleltransportcomponents} also.}




\subsection {Geodesics}

\subsubsection {Definition.  Geodesic on Riemannian manifold.}

Let $M$ be manifold equipped with connection $\nabla$.

{\bf Definition} A parameterised curve  $C\colon\quad x^i=x^i(t)$ is called geodesic if velocity vector
$\v(t)\colon v^i(t)={dx^i(t)\over dt}$ is covariantly constant along this curve, i.e.
it remains parallel along the curve:
\begin{equation}\label{geodesdef1}
    \nabla_\v \v={\nabla\v\over dt}={dv^i(t)\over dt}+v^k(t)\Gamma^i_{km}(x(t))v^m(t)=0,\,\,\,{i.e.}
\end{equation}
\begin{equation*}
    {d^2x^i(t)\over dt^2}+{d x^k(t)\over dt}\Gamma^i_{km}(x(t)){d x^m(t)\over dt}=0\,.
\end{equation*}
  These are linear second order differential equations. One can prove that
  this equations have solution and it is unique\footnote{this is true under additional technical conditions which we do not
  discuss here}for an arbitrary initial data
  ($x^i(t_0)=x^i_0, \dot x^i(t_0)=\dot x^i_0$. )


\m

Geodesics defined with Levi-Civita connection on the Riemannian manifold is called geodesic on Riemannian manifold.
We mostly consider geodesics on Riemannian manifolds.


{\bf Proposition}   {\it If $C \colon\,\, \x(t)$ is a geodesics on Riemannian manifold then the length of
velocity vector is preserved along the geodesic.}


{\sl Proof}   Since the connection is Levi-Civita connection then it preserves scalar product of tangent vectors,
(see \eqref{connpreservingmetric}) in particularly the length of the velocity vector $\v$:

\begin{equation}\label{preservationofthelength}
  \v={d\x(t)\over dt}\Rightarrow {d\over dt}\langle\v,\v\rangle=
  \left\langle{\nabla\v\over dt},\v\right\rangle+\left\langle\v,{\nabla\v\over dt}\right\rangle=0\Rightarrow
  |\v|=\sqrt {\langle\v,\v\rangle}=0\,.
\end{equation}

\m

In fact one can see that if $\X(t)$ is a parallel transport of an arbitrary vector (not necessary tangent to the curve)
along an arbitrary  curve
$C$ on the Riemannian manifold then the length of the vector $\X(t)$ does not change,
since the connection is Levi-Civita
connection preserves scalar product:
     \begin{equation}\label{lengthdoesnotchangeduringparalleltransport}
{d\over dt}\langle\X(t),\X(t)\rangle=
2\left\langle{\nabla \X(t)\over dt}, \X(t)\right\rangle=0\Rightarrow
   |\X(t)|=\sqrt{\langle\X(t),\X(t)\rangle}=const.
     \end{equation}


\subsubsection {Un-parameterised geodesic}

We defined a geodesic as a parameterised curve such that the velocity vector
is covariantly constant along the curve.

\m

What happens if we change the parameterisation of the curve?


\m

  Another  question:  Suppose a tangent vector to the curve remains tangent during parallel transport.
  Is it true that this curve (in a suitable parameterisation) becomes geodesic?

\m

 {\bf Definition}  We call un-parameterised curve geodesic if under suitable parameterisation it obeys
   the equation \eqref{geodesdef1} for geodesics.

\m


   Let $C$--be un-parameterised geodesic.
   Then the following statement is valid.

\m

   {\bf Proposition} {\it A curve $C$ (un-parameterised) is geodesic if an only if a tangent non-zero vector remains tangent
   during parallel transport.}

   \m
   {\sl Proof}. Let $C$ be un-parameterised geodesic.
     Let $x^i=x^i(t)$ be an arbitrary parameterisation of this curve (not necessary the parametrisation such that
   equations \eqref{geodesdef1} obey!).   Since $C$ is geodesic
   there exists a reparameterisation $t=t(\tau)$
such that  $x_0^i(\tau)=x^i(t(\tau))$ obeys the equations \eqref{geodesdef1} for geodesics.

Consider velocity vector $v^i(t)={dx^i(t)\over dt}$. In general the equations \eqref{geodesdef1}
are not obeyed for the velocity vector. (They are obeyed for the velocity vector in the right parameteristaion
$x_0^i(\tau)=x^i(t(\tau))$. )

 Prove that nevertheless the velocity vector $v^i(t)={dx^i(t)\over dt}$
 remains tangent during parallel transport.
 Consider differential equation \eqref{geodesdef1} of   parameterised geodesic $x_0^i(\tau)=x^i(t(\tau))$:
                  \begin{equation}\label{geodesdef2}
               {d\over d\tau}\left(v_0^i(\tau)\right)+v_0^k(\tau)\Gamma^i_{km}v_0^m(\tau)=0\,.
                    \end{equation}
Express in this diff. equation the  velocity $v^i_0(\tau)$ via the velocity $v^i(t)$
in the initial parameterisation:
                     $$
v^i_0(\tau)={dx_0^i(\tau)\over d\tau}={dt(\tau)\over d\tau} {dx(t)\over dt}=t_\tau v^i(t)\,.
                    $$
We have
 $$
{d\over d\tau}\left(v_0^i(\tau)\right)+v_0^k(\tau)\Gamma^i_{km}v_0^m(\tau)=
 t_\tau{d\over d t}\left(t_\tau v^i(t)\right)+
 t_\tau v^k(t)\Gamma^i_{km}t_\tau v^m(t)=0\,.
 $$
 Dividing by $t_\tau$ and denoting by $a(t)$ the function $t_\tau={dt\over d\tau}$: $a(t)={dt\over d\tau}$ we come to
       \begin{equation}\label{paralleltransportofunparametrcurve}
 {d\over d t}\left(a(t) v^i(t)\right)+
  v^k(t)\Gamma^i_{km} \left(a(t) v^m(\t)\right)={\nabla \left((a(t)\v(t)\right)\over dt}=0\,.
   \end{equation}
  We see that the vector field $a(t)\v(t)$ is covariantly  constant on the  geodesic, i.e it is parallel transport
  of the initial vector $a(t_0)\v(t_0)$, where $t_0$ is an initial point of the curve.
  Hence the vector field  $a(t)\v(t)\over a(t_0)$ is a a parallel transport of the vector $\v(t_0)$.


Since velocity vector remains tangent to the curve during parallel transport, hence
 an arbitrary tangent vector,  $\X=c\v(t)$ remains tangent  during parallel transport.



 \m

 One can easy prove the converse implication too: if velocity vector of the curve $x^i(t)$
remains tangent vector during parallel transport then there exists a parameterisation such
that equations \eqref{geodesdef1} are obeyed.
Indeed suppose that the velocity vector $v^i(t)$ remains tangent vector during parallel transport,
i.e. there exists a function $a(t)$ such that the vector field $\X(t)=a(t)\v(t)$
obeys the equations \eqref{paralleltransportofunparametrcurve}.  Then choosing a parameter $\tau$
such that ${dt(\tau)\over d\tau}=a(t)$ we come to the parameterisation such that the equations
\eqref{geodesdef2} are obeyed. \finish

\m

{\bf Remark} One can consider more compact proof using the fact that the map \eqref{paralleltransportalongthecurve}
of parallel transport does not depend on parameterisation.

\m

{\bf Remark} One can see that if $x^i=x^i(t)$ is geodesic in an arbitrary parameterisation and
$s=s(t)$ is a natural parameter (which defines the length of the curve) then
$x^i(t(s))$ is parameterised geodesic.



   \subsection {Geodesics and Lagrangians of "free" particle on Riemannian manifold.}



     \subsubsection {Lagrangian and Euler-Lagrange equations}

   A function $L=L(x,\dot x)$ on points and velocity vectors on manifold $M$
   is a {\it Lagrangian} on manifold $M$.


   We assign  to  Lagrangian $L=L(x,\dot x)$ the following second order differential equations
\begin{equation}\label{ELequations1}
    {d\over dt}\left({\p L\over \p \dot x^i}\right)={\p L\over \p  x^i}
\end{equation}
In detail
\begin{equation}\label{ELequationsindetail}
    {d\over dt}\left({\p L\over \p \dot x^i}\right)={\p^2 L\over \p x^m \p \dot x^i}\dot x^m+
    {\p^2 L\over \p \dot x^m \p \dot x^i}{\buildrel \cdot\cdot\over x^m}
    =
    {\p L\over \p  x^i}\,.
\end{equation}

  These equations are called {\it Euler-Lagrange equations} of the Lagrangian $L$.
  We will explain later the variational origin of these equations
  \footnote{To every mechanical system one can put in correspondence  a Lagrangian on configuration space.
      The dynamics of the system is described by Euler-Lagrange equations.
      The advantage of Lagrangian approach is that it works in an arbitrary coordinate system:
      Euler-Lagrange equations are invariant
      with respect to changing of coordinates since they arise from variational principe.}.

    \subsubsection {Lagrangian of "free" particle}


     Let $(M,G)$, $G=g_{ik}dx^idx^k$ be a Riemannian manifold.


     {\bf Definition} We say that {\it Lagrangian} $L=L(x,\dot x)$ is the Lagrangian of a "free" particle on the
     Riemannian manifold $M$ if
      \begin{equation}\label{frepaticlelagrangian}
        L={g_{ik}\dot x^i\dot x^k\over 2}
      \end{equation}

{\bf Example} "Free" particle in Euclidean space.
    Consider $\E^3$ with standard metric $G=dx^2+dy^2+dz^2$
      \begin{equation}\label{freeparticleineuclideanspace}
        L={g_{ik}\dot x^i\dot x^k\over 2}={\dot x^2+\dot y^2+\dot z^2\over 2}
      \end{equation}

Note that this is the Lagrangian that describes the dynamics od free particle.

\m

{\bf Example} "Free" particle on sphere.

The metric on the sphere of radius $R$ is
$G=R^2d\theta^2+R^2\sin^2\theta d\varphi^2$.
Respectively for the Lagrnagian of "free" particle we have
\begin{equation}\label{freeparticleonthesphere}
        L={g_{ik}\dot x^i\dot x^k\over 2}={R^2\dot\theta^2+R^2\sin^2\theta\dot \varphi^2\over 2}
      \end{equation}


\m
\subsubsection {Equations of geodesics and Euler-Lagrange equations}

{\bf Theorem}. {\it Euler-Lagrange equations of Lagrangian of free particle are equivalent
to the second order differential equations
for geodesics.}

This Theorem makes very easy calculations for Christoffel indices.

\m

 This Theorem can be proved by direct calculations.

Calculate Euler-Lagrange equations \eqref{ELequations1} for the Lagrangian \eqref{frepaticlelagrangian}:
               $$
    {d\over dt}\left({\p L\over \p \dot x^i}\right)=
    {d\over dt}\left({\p \left({g_{mk}\dot x^m\dot x^k\over 2}\right)\over \p \dot x^i}\right)=
     {d\over dt}\left(g_{ik}\dot x^k\right)=
    g_{ik}{\buildrel \cdot\cdot\over x{^k}}+{\p g_{ik}\over \p x^m}\dot x^m\dot x^k
               $$
and
            $$
            {\p L\over \p  x^i}={\p \left({g_{mk}\dot x^m\dot x^k\over 2}\right)\over \p  x^i}=
            {1\over 2}{\p g_{mk}\over \p x^i}\dot x^m\dot x^k.
            $$
Hence we have
           $$
 {d\over dt}\left({\p L\over \p \dot x^i}\right)=
 g_{ik}{\buildrel \cdot\cdot\over x{^k}}+{\p g_{ik}\over \p x^m}\dot x^m\dot x^k=
    {\p L\over \p  x^i}={1\over 2}{\p g_{mk}\over \p x^i}\dot x^m\dot x^k\,,
           $$
  i.e.
       $$
 g_{ik}{\buildrel \cdot\cdot\over x{^k}}+\p_mg_{ik}\dot x^m\dot x^k=
 {1\over 2}\p_i g_{mk}\dot x^m\dot x^k\,.
       $$
 Note that    $\p_mg_{ik}\dot x^m\dot x^k={1\over 2}\left(\p_mg_{ik}\dot x^m\dot x^k+\p_kg_{im}\dot x^m\dot x^k\right)$.
Hence we come to equation:
               $$
               g_{ik}{d^2 x^k\over dt^2}+
               {1\over 2}\left(
 \p_mg_{ik}+\p_kg_{im}-\p_ig_{mk}
                  \right)\dot x^m\dot x^k
               $$
 Multiplying on the inverse matrix $g^{ik}$ we come
\begin{equation}\label{ELequationforgeodesic1}
    {d^2 x^i\over dt^2}+{1\over 2}g^{ij}\left({\p g_{jm}\over\p x^k}+
    {\p g_{jk}\over\p x^m}-{\p g_{mk}\over \p x^j}\right){dx^m\over dt}{dx^k\over dt}=0\,.
\end{equation}
We recognize here Christoffel symbols of Levi-Civita connection (see \eqref{levi-civitaformula}) and we rewrite
this equation as
   \begin{equation}\label{ELequationforgeodesic2}
    {d^2 x^i\over dt^2}+{dx^m\over dt}\Gamma^i_{mk}{dx^k\over dt}=0\,.
\end{equation}
This is nothing but the equation \eqref{geodesdef1}.

Applications of this Theorem: calculation of Christoffel symbols of Levi-Civita connection.

\subsubsection{Examples of calculations of Christoffel symbols and geodesics using Lagrangians.}

Consider two examples: We calculate Levi-Civita connection on sphere in $\E^3$ and on Lobachevsky plane
using Lagrangians and find geodesics.

1) {\it Sphere of the radius $R$ in $\E^3$}:


   Lagrangian of "free" particle on the sphere is given by \eqref{freeparticleonthesphere}:
    $$
    L={R^2\dot \theta^2+R^2\sin^2\theta\dot \varphi^2\over 2}
    $$
Euler-Lagrange equations defining geodesics are
          \begin{equation}\label{EL eqforsphere}
  {d\over dt}\left({\p L\over \p  \dot\theta}\right)-
  {\p L\over \p \theta}=
  {d\over dt} \left(R^2{\buildrel \cdot\over \theta}\right)-
  R^2\sin\theta\cos\theta\dot \varphi^2\Rightarrow  {\buildrel \cdot\cdot\over \theta}-
  \sin\theta\cos\theta\dot \varphi^2=0\,,
          \end{equation}
              $$
{d\over dt}\left({\p L\over \p  \dot\varphi}\right)-
  {\p L\over \p \varphi}=
  {d\over dt} \left(R^2{\sin^2\theta\dot\varphi}\right)=0\Rightarrow
  {\buildrel \cdot\cdot\over \varphi}+
  {\rm cotan\,}\theta\dot\theta\dot\varphi=0\,.
                            $$
Comparing Euler-Lagrange equations with equations for geodesic in terms of Christoffel symbols:
         $$
   {\buildrel \cdot\cdot\over \theta}+\Gamma^\theta_{\theta\theta}\dot\theta^2+
   2\Gamma^\theta_{\theta\varphi}\dot\theta\dot\varphi+\Gamma^\theta_{\varphi\varphi}\dot\varphi^2=0,
         $$
    $$
  {\buildrel \cdot\cdot\over \varphi}+\Gamma^\varphi_{\theta\theta}\dot\theta^2+
   2\Gamma^\varphi_{\theta\varphi}\dot\theta\dot\varphi+\Gamma^\varphi_{\varphi\varphi}\dot\varphi^2=0
    $$
    we come to
 \begin{equation}\label{christsymbolsthroughlagrangiansforsphere1}
    \Gamma^\theta_{\theta\theta}=
    \Gamma^\theta_{\theta\varphi}=\Gamma^\theta_{\varphi\theta}=0\,,
    \Gamma^\theta_{\varphi\varphi}=-\sin\theta\cos\theta\,,
 \end{equation}
 \begin{equation}\label{christsymbolsthroughlagrangiansforsphere2}
    \Gamma^\varphi_{\theta\theta}=\Gamma^\varphi_{\varphi\varphi}=0,\,
    \Gamma^\varphi_{\theta\varphi}=\Gamma^\varphi_{\varphi\theta}={\rm cotan\, }\theta\,.
 \end{equation}
 (Compare with previous calculations for connection in subsections 2.2.1 and 2.3.4)


To find geodesics one have to solve second order differential equations \eqref{EL eqforsphere}:

One can see that the great circles: $\varphi=\varphi_0$, $\theta=\theta_0+t$ are solutions of
second order differential equations \eqref{EL eqforsphere} with initial conditions
\begin{equation}\label{initialconditions}
  \theta(t)\big\vert_{t=0}=\theta_0, \dot\theta(t)\big\vert_{t=0}=1,\quad
  \varphi(t)\big\vert_{t=0}=\varphi_0, \dot\theta(t)\big\vert_{t=0}=0\,.
\end{equation}
The rotation of the sphere is isometry, which does not change Levi-Civta connection.
Hence an arbitrary great circle is geodesic.

Prove that an arbitrary geodesic is an arc of great circle.
Let the curve $\theta=\theta(t),\varphi=\varphi(t)$,
$0\leq t\leq t_1$ be geodesic. Rotating the sphere
we can come to the curve $\theta=\theta'(t),\varphi=\varphi'(t)$, $0\leq t\leq t_1$
such that  velocity vector at the initial time is direccted along meridian, i.e.
initial conditions are
\begin{equation}\label{initialconditionsnew}
  \theta'(t)\big\vert_{t=0}=\theta_0, \dot\theta'(t)\big\vert_{t=0}=a,\quad
  \varphi'(t)\big\vert_{t=0}=\varphi_0, \dot\varphi'(t)\big\vert_{t=0}=0\,.
\end{equation}
(Compare with initial conditions \eqref{initialconditions})
Second order differential equations with boundary conditions for coordinates and velocities at $t=0$ have unique
solution. The solutions of second order differential equations \eqref{EL eqforsphere} with initial conditions
\eqref{initialconditionsnew} is a curve
                $\theta'(t)=\theta_0+at$, $\varphi'(t)=\varphi_0$. It is great circle.
                Hence initial curve the geodesic $\theta=\theta(t),\varphi=\varphi(t)$,
$0\leq t\leq t_1$ is an arc of great circle too.

We proved that all geodesics are great circles.


\m


2) {\it Lobachevsky plane.}


   Lagrangian of "free" particle on the Lobachevsky plane with metric $G={dx^2+dy^2\over y^2}$ is
      $$
   L={1\over 2}{\dot x^2+\dot y^2\over y^2}.
      $$
Euler-Lagrange equations are

  $$
  {\p L\over \p x}=0={d\over dt}{\p L\over \p \dot x}={d\over dt}\left({\dot x\over y^2}\right)=
            {{\buildrel \cdot\cdot\over x}\over y^2}-{2\dot x\dot y\over y^3}, {\rm i.e.}\quad
            {\buildrel \cdot\cdot\over x}-{2\dot x\dot y\over y}=0\,,
             $$
             $$
              {\p L\over \p y}=-{\dot x^2+\dot y^2\over y^3}=
              {d\over dt}{\p L\over \p \dot y}={d\over dt}\left({\dot y\over y^2}\right)=
            {{\buildrel \cdot\cdot\over y}\over y^2}-{2\dot y^2\over y^3}, {\rm i.e.}\quad
            {\buildrel \cdot\cdot\over y}+{\dot x^2\over y}-{\dot y^2\over y}=0\,.
            $$
Comparing these equations with equations for geodesics:
${\buildrel \cdot\cdot\over x^i}-\dot x^k\Gamma^i_{km}\dot x^m=0$
($i=1,2$, $x=x^1,y=x^2$) we come to
                $$
\Gamma^{x}_{xx}=0,
\Gamma^{x}_{xy}=\Gamma^{x}_{yx}=-{1\over y},\,
\Gamma^{x}_{yy}=0,\,\Gamma^{y}_{xx}={1\over y}, \Gamma^{y}_{xy}=\Gamma^{y}_{yx}=0, \Gamma^{y}_{yy}=-{1\over y}\,.
              \hbox{\finish}  $$
In  a similar way as for a sphere one can find geodesics on Lobachevsky plane.
First we note that vertical rays are geodesics.  Then using the inversions with centre on the absolute
one can see that arcs of the circles with centre at the absolute ($y=0$) are geodesics too.
(See homework 6).


\subsubsection {Variational principe and Euler-Lagrange equations }

   Here very briefly we will explain how Euler-Lagrange equations follow from variational principe.

\m
\def\confspace {\M^{^{\x_2,t_2}}_{_{\x_1,t_1}}}

Let $M$ be a manifold (not necessarily Riemannian) and $L=L(x^i,\dot x^i)$ be a Lagrangian on it.

   Denote my $\M^{^{\x_2,t_2}}_{_{\x_1,t_1}}$ the space of curves (paths)
such that they start at the point $\x_1$ at the "time" $t=t_1$ and end at the
point  $\x_2$ at the "time" $t=t_2$:
\begin{equation}\label{spacceofpaths}
\M^{^{\x_2,t_2}}_{_{\x_1,t_1}}=\{C\colon\,\, \x(t), t_1\leq t\leq t_2,\,\,\x(t_1)=\x_1, \x(t_2)=\x_2\}\,.
\end{equation}

  Consider the following functional $S$ on the space $\confspace$:
                \begin{equation}\label{actionfunctional1}
  S\left[\x(t)\right]=\int_{t_1}^{t_2}L\left(x^i(t), \dot x^i(t)\right)dt\,.
                \end{equation}
    for every curve $\x(t)\in \confspace$.

This functional is called {\it action} functional.

{\bf Theorem}  {\it Let functional $S$ attaints the minimal value on the path $\x_0(t)\in\confspace$, i.e.
                             \begin{equation}\label{minimalvalueofthefunctional}
                    \forall \x(t)\in \confspace\quad            S[\x_0(t)]\leq  S[\x(t)]\,.
                             \end{equation}
Then the path $\x_0(t)$ is a solution of Euler-Lagrange equations of the Lagrangian $L$:
            \begin{equation}\label{minimalfunctionsolution1}
{d\over dt}\left({\p L\over \p \dot x^i}\right)={\p L\over \p  x^i}\,\,
            {\rm if}\,\, \x(t)=\x_0(t)\,.
            \end{equation}
 }
 \m


 {\bf Remark} The path $\x(t)$ sometimes is called {\it extremal} of the action functional
\eqref{actionfunctional1}.

   We will use this Theorem to show that the geodesics are in some sense shortest curves
   \footnote{The statement of this Theorem is enough for our purposes.
  In fact in classical mechanics another more useful statement
  is used:
  the path $\x_0(t)$ is a solution of Euler-Lagrange equations of the Lagrangian $L$
  if and only if it is the stationary "point" of the action functional  \eqref{actionfunctional1}, i.e.
           \begin{equation}\label{stationarpath}
            S[\x_0(t)+\delta \x(t)]-S[\x_0(t)+\delta \x(t)]=0(\delta \x(t))
           \end{equation}
 for an arbitrary infinitesimal variation of the path $\x_0(t)$:
   $\delta \x_(t_1)=\delta \x_(t_2)=0$.}.

\subsection {Geodesics and shortest distance.}

 Many of you know that geodesics are in some sense shortest curves.
 We will give an exact meaning to this statement and prove it using variational principe:

Let $M$ be a Riemannian manifold.

{\bf Theorem}
{\it Let $\x_1$ and $\x_2$ be two points on $M$.
The shortest curve which joins these points is an arc of geodesic.

Let $C$ be a geodesic on $M$ and  $\x_1\in C$. Then for an arbitrary point  $\x_2\in C$ which is
close to the point $\x_1$ the arc of geodesic joining the points $\x_1,\x_2$ is a shortest curve between
these points\footnote{More precisely: for every point $\x_1\in C$ there exists a ball $B_{\delta}(\x_1)$
such that for an arbitrary point $\x_2\in C\cap B_{\delta}(\x_1)$
the arc of geodesic joining the points $\x_1,\x_2$ is a shortest curve between
these points.}
.}

\m

  This Theorem makes a bridge between two different approach to geodesic: the shortest disntance and
  parallel transport of velocity vector.

\m

Sketch a proof:


Consider the following two Lagrangians: Lagrangian of a "free "
particle $L_{\rm free}={g_{ik}(x)\dot x^i\dot x^k\over 2}$
and the length Lagrangian
         $$
      L_{\rm length}(x,\dot x)=\sqrt{g_{ik}(x)\dot x^i\dot \x^k}=\sqrt {2L_{\rm free}}\,.
         $$
   If $C\colon \,\,x^i(t), t_1\leq t\leq t_2$ is a curve on $M$ then
                  $$
             \hbox{Length of the curve $C$}=
                  $$
    \begin{equation}\label{lengthlagrangian}
    \int_{t_1}^{t_2}L_{\rm length}(x^i(t),\dot x^i(t))dt=
    \int_{t_1}^{t_2}\sqrt{g_{ik}(x(t))\dot x^i(t)\dot x^k(t)}dt\,.
    \end{equation}


    The proof of the Theorem follows from the following observation:

 {\it Observation}  Euler-Lagrange equations for the length  functional \eqref{lengthlagrangian}
 are equivalent to the  Euler-Lagrange equations for action functional \eqref{actionfunctional1}.
        This means that extremals of the length functional and action functionals coincide.

        \m

    Indeed it follows from this observation and  the variational principe that
     the shortest curves obey
     the Euler-Lagrange equations for the action functional.
     We showed before that Euler-Lagrange equations for action functional \eqref{actionfunctional1}
    define geodesics.   Hence the shortest curves are geodesics.

    \m

    One can  check the observation by direct calculation:  Calculate Euler-Lagrange equations for the Lagrangian
    $L_{\rm length}=\sqrt{g_{ik}(x)\dot x^i\dot x^k}=\sqrt {2L_{\rm free}}$:
                    $$
     {d\over dt}\left({\p L_{\rm length}\over \p \dot x^i}\right)-{\p L_{\rm length}\over \p  x^i}=
     {d\over dt}\left( {1\over \sqrt{g_{ik}\dot x^i\dot x^k}}g_{ik}\dot x^k\right)-
     {1\over 2\sqrt{g_{ik}\dot x^i\dot x^k}}{\p g_{km}\dot x^k\dot x^m\over  \p x^i}
                    $$
                    \begin{equation}\label{geodesiclagrequation4}
=    {d\over dt}\left( {1\over L_{\rm length}}{\p L_{\rm free}\over  \dot \p x^i}\right)-
     {1\over L_{\rm length}}{\p L_{\rm free}\over  \p x^i}=0\,.
                                         \end{equation}


To facilitate calculations note that the length functional \eqref{lengthlagrangian}
    is reparameterisation invariant.  Choose the natural parameter $s(t)$ or a parameter proportional
    to the natural parameter on the curve $x^i(t)$.  We come to $L_{\rm length}=const$ and
    it follows from  \eqref{geodesiclagrequation4} that
        $$
   {d\over dt}\left({\p L_{\rm length}\over \p \dot x^i}\right)-{\p L_{\rm length}\over \p  x^i}=
    {1\over L_{\rm length}}\left( {d\over dt}\left( {\p L_{\rm free}\over  \dot \p x^i}\right)-
     {\p L_{\rm free}\over  \p x^i}\right)=0\,.
            $$
 We prove that Euler-Lagrange equations for length and action Lagrangians coincide.\finish


\m

  In the Euclidean space straight lines are the shortest distances between two points. On the other hand
  their velocity vectors are constant.
We realise now that in general Riemannian manifold the role of geodesic is twofold also:
  they are locally shortest and have covariantly constant velocity vectors.

\m

\subsubsection {Again geodesics for sphere and Lobachevsky plane}

The fact that geodesics are shortest gives us another tool to calculate geodesics.

Consider again examples of sphere and Lobachevsky plane and find geodesics using the fact that they are shortest.
The fact that geodesics are locally the shortest curves

\m


Consider  again sphere in $\E^3$ with the radius $R$:
  Coordinates $\theta,\varphi$, induced Riemannian metrics (first quadratic form):
\begin{equation}\label{metricsonsphere5}
  G=R^2(d\theta^2+\sin^2\theta d\varphi^2)\,.
\end{equation}
Consider two arbitrary points $A$ and $B$ on  the sphere.
Let $(\theta_0,\varphi_0)$ be coordinates
  of the point $A$ and $(\theta_1,\varphi_1)$ be coordinates
  of the point $B$


 Let $C_{AB}$ be a curve which connects these points:
   $C_{AB}\colon \theta(t),\varphi(t)$ such that
    $\theta(t_0)=\theta_0, \theta(t_1)=\theta_1$,
    $\varphi(t_0)=\theta_0, \theta(t_1)=\theta_1$
  then:
\begin{equation}\label{lengthonsphere5}
  L_{C_{AB}}=\int R\sqrt {\theta_t^2+\sin^2\theta(t)\varphi_t^2}dt
\end{equation}

  Suppose that points $A$ and $B$
  have the same latitude, i.e.
  if $(\theta_0,\varphi_0)$ are coordinates
  of the point $A$ and $(\theta_1,\varphi_1)$ are coordinates
  of the point $B$ then $\varphi_0=\varphi_1$
  (if it is not the fact then we can come to this condition rotating the sphere)

Now it is easy to see that an arc of meridian, the  curve $\varphi=\varphi_0$ is geodesics:
Indeed consider an arbitrary curve $\theta (t),\varphi(t)$
which connects the points $A,B$:
$\theta(t_0)=\theta(t_1)=\theta_0$,
$\varphi(t_0)=\varphi(t_1)=\varphi_0$.
Compare its length with the length
of the meridian which connects the points $A$, $B$:
\begin{equation}\label{geodforsphereap}
\int_{t_0}^{t_1} R\sqrt {\theta_t^2+\sin^2\theta\varphi_t^2}dt\geq
R\int_{t_0}^{t_1} \sqrt {\theta_t^2}dt=R\int_{t_0}^{t_1} {\theta_t}dt=
R(\theta_1-\theta_0)
\end{equation}
Thus we see that  the great circle joining points $A,B$ is the shortest.
{\it The great circles on sphere are geodesics.}
 It corresponds to geometrical intuition:
     The geodesics on the sphere are the circles of intersection of the sphere
     with the plane which crosses the centre.


\bigskip

     {\it Geodesics on Lobachevsky plane}

     Riemannian metric on Lobachevsky plane:
            \begin{equation}\label{lobachmetric5}
               G={dx^2+dy^2\over y^2}
            \end{equation}



The length of the curve $\gamma\colon x=x(t, y=y(t))$ is equal to
                     $$
   L=\int\sqrt {{x_t^2+y_t^2\over y^2(t)}}dt
                        $$

In particularly the length of the vertical interval $[1,\vare]$ tends
to infinity if $\vare\to 0$:
                    $$
   L=\int\sqrt {{x_t^2+y_t^2\over y^2(t)}}dt=
   \int_\vare^1\sqrt {{1\over t^2}}dt=\log {1\over \vare}
                     $$
One can see that the distance from  every point to the line $y=0$ is
equal to infinity. This motivates the fact that the line $y=0$
is called {\it absolute}.

Consider two points $A=(x_0,y_0)$, $B=(x_1,y_1)$ on Lobachevsky plane.


It is easy to see that vertical lines are geodesics of Lobachevsky plane.

Namely let points $A, B$ are on the ray $x=x_0$.
Let  $C_{AB}$ be an arc of the ray $x=x_0$ which joins these points:
    $C_{AB}\colon x=x_0, y=y_0+t$
Then it is easy to see that the length of the curve $C_{AB}$ is less or equal than the length of the arbitrary curve
  $x=x(t),y=y(t)$ which joins these points: $x(t)\big\vert_{t=0}=x_0, y(t)\big\vert_{t=0}=y_0$,
$x(t)\big\vert_{t=t_1}=x_0, y(t)\big\vert_{t=t_1}=y_1$:
          $$
 \int_{0}^t\sqrt {x_t^2+y_t^2\over y^2(t)}dt\geq
 \int_{0}^t\sqrt {y_t^2\over y^2(t)}dt=\int_{y_0}^{y_1} {dt\over t}dt=\log {y_1\over y_0}=\hbox{length of $C_{AB}$}
          $$
Hence $C_{AB}$ is shortest. We prove that vertical rays are geodesics.

Consider now the case if $x_0\not=x_1$.
Find  geodesics which connects two points $A,B$ which are not on the same vertical ray.
Consider semicircle which passes
these two points such that its centre is on the absolute.
We prove that it is a geodesic.

\medskip
{\footnotesize
{\it Proof} Let coordinates of the centre of the circle
are $(a,0)$. Then consider polar coordinates $(r,\varphi)$:
\begin{equation}\label{semicircle}
  x=a+r\cos\varphi, y=r\sin\varphi
\end{equation}
In these polar coordinates $r$-coordinate of the semicircle is constant.

Find Lobachevsky metric  in these coordinates:
 $dx=-r\sin\varphi d\varphi+\cos\varphi dr$, $dy=r\cos\varphi d\varphi+\sin\varphi dr$,
 $dx^2+dy^2=dr^2+r^2d\varphi^2$. Hence:
\begin{equation}\label{lob2}
  G={dx^2+dy^2\over y^2}={dr^2+r^2d\varphi^2\over r^2\sin^2\varphi}=
={d\varphi^2\over \sin^2\varphi}+{dr^2\over r^2\sin^2\varphi}
\end{equation}
We see
that the length of the arbitrary curve which connects points $A,B$
is greater or equal to the length of the arc of the circle:
\begin{equation}\label{lengthmin1}
 L_{AB}=
 \int_{t_0}^{t_1}\sqrt
 {{\varphi_t^2\over \sin^2 \varphi}+{r_t^2\over r^2\sin^2\varphi}}dt\geq
\int_{t_0}^{t_1}\sqrt
 {{\varphi_t^2\over \sin^2 \varphi}}dt=
 \end{equation}
 $$
\int_{t_0}^{t_1}
 {{\varphi_t\over \sin \varphi}}dt=
\int_{\varphi_0}^{\varphi_1}
 {{d\varphi\over \sin \varphi}}=
 \log {\tan \varphi_1\over \tan \varphi_1}
$$

The proof is finished.}



\subsection {$^*$Integrals of motions and geodesics.}

We see how useful in Riemannian geometry to use the Lagrangian approach.

To solve and study solutions of Lagrangian equations (in particular geodesics which are solutions of Euler-Lagrange
equations for Lagrangian of free particle) it is very useful to use {integrals of motion}

\subsubsection {$^*$Integral of motion for arbitrary Lagrangian $L(x,\dot x)$}
   Let $L=L(x,\dot x)$ be a Lagrangian, the function of point and velocity vectors on manifold $M$
   (the function on tangent bundle $TM$).

   {\bf Definition}  We say that the function $F=F(q,\dot q)$ on $TM$ is {\it integral of motion}
   for Lagrangian $L=L(x,\dot x)$ if for any curve $q=q(t)$ which is the solution of Euler-Lagrange equations of motions
     the magnitude  $I(t)=F(x(t), \dot x(t))$ is preserved along this curve:
               \begin{equation}\label{integralofmotion}
               F(x(t), \dot x(t))=const\,\,\hbox{if $x(t)$  is a solution of Euler-Lagrange equations\eqref{ELequations1}.}
                 \end{equation}
In other words
               \begin{equation}\label{integralofmotion2}
               {d\over dt}
               \left(F(x(t), \dot x(t))\right)=0\,\,{\rm if}\,\,
               x^i(t)\colon\,\,   {d\over dt}\left({\p L\over \dot \p x^i}\right)-{\p L\over \p x^i}=0\,.
                 \end{equation}

               \subsubsection {$^*$Basic examples of Integrals of motion: Generalised momentum and Energy}

 Let $L(x^i,\dot x^i)$ does not depend on the coordinate $x^1$.
    $L=L(x^2,\dots,x^n, \dot x^1,\dot x^2,\dots, \dot x^n)$. Then  the function
                   $$
       F_1(x,\dot x)={\p L\over \p\dot x^1}
                   $$
is integral of motion. (In the case if $L(x^i,\dot x^i)$ does not depend on the coordinate $x^i$.
    the function $F_i(x,\dot x)={\p L\over \dot \p x^i}$ will be integral of motion.)


Proof is simple. Check the condition \eqref{integralofmotion2}:
Euler-Lagrange equations of motion are:
          $$
          {d\over dt}\left({\p L\over \dot \p x^i}\right)-{\p L\over \p x^i}=0\quad (i=1,2,\dots,n)
          $$
 We see that exactly first equation of motion is
               $$
       {d\over dt}\left({\p L\over \dot \p x^1}\right)= {d\over dt}F_1(q,\dot q)=0
       \quad \hbox{since ${\p L\over \p x^1}=0,$}\,.
               $$


 (if $L(x^i,\dot x^i)$ does not depend on the coordinate $x^i$
    then the function $F_i(x,\dot x)={\p L\over \dot x^1}$ is  integral of motion
    since $i-th$ equation is exactly the condition $\dot F_i=0$.)

    The integral of motion $F_i={\p L\over \dot \p x^i}$ is called sometimes {\it generalised momentum}.


    \m

    Another very important example of integral of motion is: energy.
      \begin{equation}\label{energy}
      E(x,\dot x)=\dot x^i {\p L\over \p \dot x^i}-L\,.
      \end{equation}
    One can check by direct calculation that it is indeed integral of motion.
    Using Euler Lagrange equations
    ${d\over dt}\left({\p L\over \dot \p x^i}\right)-{\p L\over \p x^i}$ we have:
           $$
     {d\over dt}E(x(t), \dot x(t))={d\over dt}\left(\dot x^i {\p L\over \p \dot x^i}-L\right)=
        {\p L\over \p \dot x^i}{d\dot x^i\over dt}+{d\over dt}\left({\p L\over \p\dot x^i}\right)\dot x^i-{dL\over dt}=
           $$
           $$
        {\p L\over \p \dot x^i}{d\dot x^i\over dt}+{\p L\over \p x^i}{dx^i\over dt}-{dL(x,\dot x)\over dt}=
        {dL(x,\dot x)\over dt}-{dL(x,\dot x)\over dt}=0\,.
           $$
\subsubsection {$^*$Integrals of motion for geodesics}

Apply the integral of motions for studying geodesics.

The Lagrangian of "free" particle  $L_{\rm free}={g_{ik}(x)\dot x^i\dot x^k\over 2}$.
  For Lagrangian of free particle solution of Euler-Lagrange equations of motions are geodesics.

  If $F=F(x,\dot x)$ is the integral of motion of the free Lagrangian $L_{\rm free}={g_{ik}(x)\dot x^i\dot x^k\over 2}$
  then the condition \eqref{integralofmotion} means that the magnitude
        $I(t)=F(x^i(t), \dot x^i(t))$ is preserved along the geodesics:
                  \begin{equation}\label{integralofmotionforgeodesic}
               I(t)=F(x^i(t), \dot x^i(t))=const, {\rm i.e.}\,
               {d\over dt}I(t)=0\,\,\,\hbox{if $x^i(t)$  is geodesic.}
                 \end{equation}

  Consider examples of integrals of motion for free Lagrangian, i.e. magnitudes which preserve along the geodesics:

  \m

  {\bf Example 1}  Note that for an arbitrary "free" Lagrangian Energy integral \eqref{energy}
   is an integral of motion:
                       $$
      E=\dot x^i {\p L\over \p \dot x^i}-L=
      \dot x^i {\p \left({g_{pq}(x)\dot x^p\dot x^q\over 2}\right)\over \p \dot x^i}-
      {g_{ik}(x)\dot x^i\dot x^k\over 2}=
                       $$
                    \begin{equation}\label{energy2}
   \dot x^ig_{iq}(x)\dot x^q-{g_{ik}(x)\dot x^i\dot x^k\over 2}={g_{ik}(x)\dot x^i\dot x^k\over 2}\,.
                    \end{equation}
   This is an integral of motion for an arbitrary Riemannian metric. It is preserved on an arbitrary geodesic
                         $$
            {dE(t)\over dt}={d\over dt}\left({1\over 2}
            g_{ik}(x(t))\dot x^i(t)\dot x^k(t)\right)=0\,.
                         $$
   In fact we already know this integral of motion:  Energy \eqref{energy2} is proportional to the
   square of the length of velocity vector:
   \begin{equation}\label{energyandspeed}
   |\v|=\sqrt {g_{ik}(x)\dot x^i\dot x^k}=\sqrt {2E}\,.
   \end{equation}
   We already proved that velocity vector is preserved
   along the geodesic (see the Proposition in the subsection 3.2.1 and its proof \eqref{preservationofthelength}.)


\m


  {\bf Example 2}  Consider Riemannian metric $G=adu^2+bdv^2$ (see also calculations in subsection 2.3.3)
   in the case if $a=a(u)$, $b=b(u)$, i.e. coefficients do not depend on the second coordinate  $v$:
                \begin{equation}\label{intofmotionwhenfirstcoordianteiscyclic}
G=a(u)du^2+b(u)dv^2,\,\, L_{\rm free}={1\over 2}\left(a(u)\dot u^2+b(u)\dot v^2\right)
                \end{equation}
 We see that Lagrangian does not depend on the second coordinate $v$ hence the magnitude
      \begin{equation}\label{cyclicintegra2}
        F={\p L_{\rm free}\over \p \dot v}=b(u)\dot v
      \end{equation}
  is preserved along geodesic. It
   is integral of motion because Euler-Lagrange equation for coordinate $v$ is
    $$
 {d\over dt}{\p L_{\rm free}\over \p\dot v}-{\p L_{\rm free}\over \p v}=
 {d\over dt}{\p L_{\rm free}\over \p\dot v}={d\over dt}F=0\,\,
 \hbox{ since ${\p L_{\rm free}\over \p v}=0$.}\,.
 $$


 \m

 In fact all revolution surfaces which we consider here (cylinder, cone, sphere,...) have Riemannian
 metric of this type.
 Indeed consider further examples.

 {\bf Example (sphere)}

 Sphere of the radius $R$ in $\E^3$. Riemannian metric:
$G=Rd\theta^2+R^2\sin^2\theta d\varphi^2$ and
$L_{\rm free}={1\over 2}\left(R^2\dot \theta^2+R^2\sin^2\theta\dot \varphi^2\right) $
It is the case \eqref{intofmotionwhenfirstcoordianteiscyclic} for $u=\theta,v=\varphi$, $b(u)=R^2\sin^2\theta$
                The integral of motion is
                 $$
           F={\p L_{\rm free}\over \p \dot\varphi}=R^2\sin^2\theta\dot\varphi
                 $$

 \m

{\bf Example (cone)}

Consider cone $\begin{cases}x=ah\cos\varphi\cr y=ah\sin\varphi \cr z=bh\cr\end{cases}$. Riemannian metric:
                          $$
                          G=d(ah\cos\varphi)^2+d(ah\sin\varphi)^2+(dbh)^2=
                          (a^2+b^2)dh^2+a^2h^2d\varphi^2\,.
                          $$
and free Lagrangian
                 $$
L_{\rm free}={(a^2+b^2)\dot h^2+a^2h^2\dot\varphi^2\over 2}\,.
                  $$
The integral of motion is
                 $$
           F={\p L_{\rm free}\over \p \dot\varphi}=a^2h^2\dot\varphi.
                 $$
  \m

{\bf Example (general surface of revolution)}

Consider  a surface  of revolution in $\E^3$:
                   \begin{equation}\label{surfaceoffrevolutiongeneral}
                    \r(h,\varphi)\colon \,\,
                  \begin{cases}
                  x=f(h)\cos\varphi\cr
                  y=f(h)\sin\varphi\cr
                  z=h
                   \end{cases}\qquad
                   (f(h)>0)
                       \end{equation}

(In the case $f(h)=R$ it is cylinder, in the case $f(h)=kh$ it is a cone, in the case $f(h)=\sqrt{R^2-h^2}$
it is a sphere, in the case $f(h)=\sqrt {R^2+h^2}$ it is one-sheeted hyperboloid, in the case
$z=\cos h$ it is catenoid,...)

For the surface of revolution \eqref{surfaceoffrevolutiongeneral}
                   $$
          G=d(f(h)\cos\varphi)^2+d(f(h)\sin\varphi)^2+(dh)^2=
          (f'(h)\cos\varphi dh-f(h)\sin\varphi d\varphi)^2+
                   $$
                   $$
     (f'(h)\sin\varphi dh+f(h)\cos\varphi d\varphi)^2+dh^2=(1+f'^2(h))dh^2+f^2(h)d\varphi^2\,.
                   $$
The "free" Lagrangian of the surface of revolution is
                 $$
L_{\rm free}={\left(1+f'^2(h)\right)\dot h^2+f^2(h)\dot\varphi^2\over 2}\,.
                  $$
and the integral of motion is
                 $$
           F={\p L_{\rm free}\over \p \dot\varphi}=f^2(h)\dot\varphi.
                 $$



\subsubsection {$^*$Using integral of  motions to calculate geodesics}

  Integrals of motions may be very useful to calculate geodesics.
  The equations for geodesics are second order differential equations.  If we know integrals of motions
  they help us to solve these equations.  Consider just an example.

    For Lobachevsky plane the free Lagrangian $L={\dot x^2+\dot y^2\over 2y^2}$. We already
    calculated geodesics in the subsection 3.3.4. Geodesics are solutions of second order Euler-Lagrange equations for
    the Lagrangian $L={\dot x^2+\dot y^2\over 2y^2}$ (see the subsection 3.3.4)
             $$
             \begin{cases}
    {\buildrel \cdot\cdot\over x}-{2\dot x\dot y\over y}=0\cr
    {\buildrel \cdot\cdot\over y}+{\dot x^2\over y}-{\dot y^2\over y}=0\cr
    \end{cases}
                          $$
  It is not so easy to solve these differential equations.

  For Lobachevsky plane we know two integrals of motions:
  \begin{equation}\label{energyforlobacchevskyplaneandsecondintegral}
    E=L={{\dot x^2+\dot y^2\over 2y^2}},\quad {\rm and}\,\,\,
     F={\p L\over \p \dot x}={\dot x\over y^2}\,.
  \end{equation}
         These both integrals preserve in time: if $x(t),y(t)$ is geodesics then
         $$
         \begin{cases}
         F={\dot x(t)\over y(t)^2}\cr
         E={{\dot x(t)^2+\dot y(t)^2\over 2y(t)^2}}=C_2\cr
         \end{cases}\Rightarrow
         \begin{cases}
         \dot x=C_1y^2\cr
         \dot y=\pm\sqrt {2C_2y^2-C_1^2y^4}\cr
         \end{cases}
                 $$
  These are first order differential equations.  It is much easier to solve these equations in general case
  than initial second order differential equations.



\section {Surfaces in $\E^3$}

 Now equipped by the knowledge of Riemannian geometry we consider surfaces in $\E^3$. We
  reconsider again conceptions of Shape (Weingarten) operator, Gaussian and mean curvatures,
  focusing attention on the the fact what properties are internal and what properties are external.


.
 \subsection{ Induced metric on surfaces.}

   Recall here again induced metric (see for detail subsection 1.4)

  If surface $M\colon\,\, \r=\r(u,v)$is embedded in $\E^3$ then induced Riemannian metric
  $G_M$ is defined by the formulae
  \begin{equation}\label{scalarproduct2}
    \langle\X,\Y\rangle=G_M(\X,\Y)=G(\X,\Y)\,,
  \end{equation}
  where $G$ is Euclidean metric in $\E^3$:
           $$
        G_M=dx^2+dy^2+dz^2\big\vert_{\r=\r(u,v)}=
        \sum_{i=1}^3(dx^i)^2\big\vert_{\r=\r(u,v)}=\sum_{i=1}^3\left({\p x^i\over \p u^\a}du^\a\right)^2
                $$
                $$=
        {\p x^i\over \p u^\a} {\p x^i\over\p  u^\beta}du^\a du^\beta\,,
           $$
        i.e.
            $$
      G_M=g_{\a\beta}du^\a ,\,\, {\rm where}\,\, g_{\a\beta}=
      {\p x^i\over \p u^\a} {\p x^i\over \p u^\beta}du^\a du^\beta\,.
           $$
 We use notations $x,y,z$ or $x^i$ ($i=1,2,3$) for Cartesian coordinates in $\E^3$,
 $u,v$ or $u^\a$ ($\a=1,2$) for coordinates on the surface. We usually   omit summation symbol
 over  dummy indices.    For coordinate tangent vectors
           $$
 \underbrace{\p\over \p u_\a}_{\hbox{Internal observer}} =
 \underbrace{\r_\a={\p x^i\over \p u^\a}{\p\over \p x^i}}_{\hbox{External observer}}
           $$
We have already plenty examples in the subsection 1.4. In particular for scalar product
\begin{equation}\label{scalarproduct4}
  g_{\a\beta}=\left\langle{\p\over u_\a},{\p\over u_\beta}\right\rangle=x^i\a x^i\beta\,.
  \langle \r_\a,\r_\beta\rangle\,.
\end{equation}



  \subsection {Induced connection}

  We already studied before induced connection on manifolds embedded in $\E^n$ (see the subsection 2.2).

  We do it again in a more detail for  surfaces in $\E^3$.

Let $\nabla^{\rm can.flat}$ be a canonical flat connection in Euclidean space $\E^3$, i.e. a connection such
that its  Christoffel symbols vanish in cartesian coordinates, i.e.
\begin{equation}\label{canonicalflatconnection}
    \nabla^{\rm can.flat}_\X \Y= \p_\X \Y= \X^m(x){\p \over x^m}Y^i(x){\p\over \p x^i}
\end{equation}
in  Cartesian coordinates. (Of course in arbitrary curvilinear coordinates Christoffel symbols
of canonical flat connection may have non-zero components, but in Cartesian coordinates Christoffel symbols vanish.
It is why we sometimes use notation  $\p_\X$ instead $\nabla^{\rm can.flat}_\X \Y$.)


   Induced connection $\nabla$ on the surface can be defined by the relation
             \begin{equation}\label{inducedconnection4}
             {\nabla}_\X\Y=\left( \nabla^{\rm can.flat}_\X \Y\right)_{\rm tangent}
             \end{equation}
          (See also the subsection (2.2))
One can show that this formula indeed defines the connection on $M$.


Calculate Christoffel symbols of this connection using the formulae above and the expressions
for basic vectors $\p_\a={\p\over \p u^\a}$,  $\p_\beta={\p\over \p u^\beta}$ for external observer.

We have
\begin{equation}\label{christoffelsymbols1forinducedconnection}
\nabla_\a\p_\beta=\Gamma^\gamma_{\a\beta}\p_\gamma=
\left( \nabla^{\rm can.flat}_{\p_\a} {\p_\beta}\right)_{\rm tangent}=
\left( \nabla^{\rm can.flat}_{\p_\a} {\r_\beta}\right)_{\rm tangent}
\end{equation}
i.e.
\begin{equation}\label{christoffelsymbols1forinducedconnection2}
 \Gamma^\gamma_{\a\beta}\r_\gamma=\left({\p^2\r \over \p u^\a\p u^\beta}\right)_{\rm tangent},
\end{equation}
Before continuing calculations note that we see already that the induced connection is {\it symmetric} one.

Now continue to calculate. Any vector attached at the surface can be decomposed as a sum of tangent vector and the vector orthogonal to the surface:
               $$
           {\p^2\r \over \p u^\a\p u^\beta}=\left({\p^2\r \over \p u^\a\p u^\beta}\right)_{\rm tangent}+\N
               $$
 Hence
 \begin{equation}\label{christoffelsymbols1forinducedconnection4}
 \Gamma^\gamma_{\a\beta}\r_\gamma=\left({\p^2\r \over \p u^\a\p u^\beta}\right)_{\rm tangent}=
 {\p^2\r \over \p u^\a\p u^\beta}-\N= \r_{\a\beta}-\N
\end{equation}
Take scalar product of  both parts of this equation on the tangent vector vector $\r_\pi$, we come to:
            $$
         \Gamma^\gamma_{\a\beta}\langle\r_\gamma, \r_\pi\rangle=
         \langle\r_{\a\beta},\r_\pi\rangle+\langle\N,\r_\pi\rangle\,.
            $$
We have that $\N\bot \r_\pi$, hence $\langle\N,\r_\pi\rangle=0$. Using the fact that
$\langle\r_\gamma, \r_\pi\rangle=g_{\gamma\pi}$ are entries of the matrix $||G_M||$ of the induced Riemannian metric
(see the equation \eqref{scalarproduct4}) we come to
               $$
\Gamma^\gamma_{\a\beta}g_{\gamma\pi}=\langle\r_{\a\beta},\r_\pi\rangle=x^i_{\a\beta}x^i_\pi\,.
               $$
Multiplying both sides of this equation on inverse matrix we come to :
               $$
\Gamma^\gamma_{\a\beta}g_{\gamma\pi}g^{\pi\rho}=\Gamma^\gamma_{\a\beta}\delta_\gamma^\rho=
\Gamma^\rho_{\a\beta}=x^i_{\a\beta}x^i_\pi g^{\pi\rho}\,.
               $$

We come to the explicit formula for the Christoffel symbols of the induced connection in terms of
parametric equations $x^i(u^\a)$ and induced metric $g_{\a\beta}$:
\begin{equation}\label{connectionintermsofinducedmetric}
\Gamma^\gamma_{\a\beta}=x^i_{\a\beta}x^i_\pi g^{\pi\gamma}\,.
\end{equation}

\subsubsection{Induced connection---the Levi Civita connection of induced metric.}

 Now you may ask a question: is there another  way to obtain induced connection (may be different one )
 on the manifold  $x^i=x^i(u^\a)$. Recall that one can consider Levi-Civita connection of
 Riemannian metric $G_M=g_{\a\beta}du^\a du^\beta$ with Christoffel symbols:
                   \begin{equation}\label{levi-civita12}
      \Gamma^\gamma_{\a\beta}={1\over 2}g^{\gamma\pi}
      \left({\p g_{\pi \a}\over \p u^\beta}+{\p g_{\pi \beta}\over \p u^\a}-
      {\p g_{\a\beta}\over \p u^\pi}\right)
                   \end{equation}
 (see the subsection about Levi-Civita connection)

 {\sl Natural question arises: may be these connections are the same?}----Yes, it is

              \m

    {\bf Proposition} {\it The connection $\nabla$ induced through the flat canonical connection on the surface
    $M$  coincides with Levi-Civita connection. In other words Christoffel symbols \eqref{scalarproduct4}
    coincide with Christoffel symbols \eqref{levi-civita12}.}.
    The proof of this Proposition follows from the  following claim

              \m
       {\bf Claim} The induced connection $\nabla$ is symmetric connection which is compatible with metric.


     Indeed we proved that Levi-Civita connection \eqref{levi-civita12} is a unique symmetric connection which is compatible with metric.
     Hence every connection which is compatible wit metric equals to Levi-Civita connection \eqref{levi-civita12}.


     It remains to prove the Claim.  When calculating Christoffel symbols of induced connection we already
     noted that its Christoffel symbols are symmetric. It remains to prove that induced connection is compatible
     with induced metric, i.e. covariant derivative preserves scalar product of tangent vectors.
     To prove this note that covariant derivative with respect to flat canonical metric preserves
 metric in Euclidean space:
                              $$
    \p_\X\langle \Y,\Z\rangle=\langle \nabla^{\rm can.flat}_\X\Y,\Z\rangle+\langle \Y,\nabla^{\rm can.flat}_\X\Z\rangle
                              $$
                 On the other hand for an arbitrary tangent vector $\R$
                 $$
   \nabla^{\rm can.flat}_\X \R= \left(\nabla^{\rm can.flat}_\X \R\right)_{\rm tangent}+\hbox {vector orthogonal to the surface}
                $$
    and $\left(\nabla^{\rm can.flat}_\X \R\right)_{\rm tangent}=\nabla_\X\R$, where $\nabla$ is the induced connection. Hence
                             $$
     \p_\X\langle \Y,\Z\rangle=\langle \nabla_\X\Y+\dots,\Z\rangle+\langle \Y,\nabla_\X\Z+\dots\rangle=
     \langle \nabla_\X\Y,\Z\rangle+\langle \Y,\nabla_\X\Z\rangle,
                             $$
 where we denote by $\dots$ vectors orthogonal to the surface.  Thus
 $\nabla$ preserves induced metric. The claim is proved.


 Note that we come to two different definitions for induced connections and prove that they same.
 The second connection is totally described in terms of metric on the surface, i.e. in terms of {\it Internal Observer}.
  The first connection
 is explicitly depended on the embedding functions. It is described in terms of {\it External Observer.}

 We continue to consider different geometrical structures form the point of view of External and Internal Observers.

 We will try to prove the remarkable theorem Gau\ss's {\it Theorema Egregium} that the Gaussian  curvature
 of the surface (which you already learned in the course of geometry)  in fact depends only on induced metric...

 \subsection {Weingarten (shape) operator and second quadratic form}

   \subsubsection {Recalling Weingarten operator}

 Continue to play with formulae \footnote{In some sense differential geometry it is when we write down
 the formulae expressing the geometrical facts, differentiate these formulae then reveal the geometrical meaning
  of the new obtained formulae e.t.c.}.


Recall the Weingarten (shape) operator which acts on tangent vectors:

  \begin{equation}\label{weingoperator1}
    S\X=-\p_\X \n\,,
  \end{equation}
 where we denote by $\n$-unit normal vector field at the points of the surface $M$: $\langle\n,\r_\a\rangle=0$,
 $\langle\n,\r_\a\rangle=1$.

\m

 Now we realise that the derivative $\p_\X \R$ of vector field with respect to another vector field
 is not a well-defined object: we need a connection.
 The formula $\p_\X \R$ in Cartesian coordinates,
 is nothing but the derivative with respect to flat canonical connection:
 If we work only in Cartesian coordinates we do not need to distinguish between
 $\p_\X \R$ and $\nabla^{\rm can.flat}_\X\R $. Sometimes with some abuse of notations we will use
 $\p_\X \R$ instead $\nabla^{\rm can.flat}_\X\R $, but never forget: this can be done only in Cartesian coordinates
 where Christoffel symbols of flat canonical connection vanish:
              $$
      \p_\X \R=\nabla^{\rm can.flat}_\X\R  \quad \hbox{in Cartesian coordinates}\,.
         $$

 So  the rigorous definition of Weingarten operator is
\begin{equation}\label{weingoperator2}
    S\X=-\nabla^{\rm can.flat}_\X \n\,,
  \end{equation}
  but we often use the former one (equation \eqref{weingoperator1})
   just remembering that this can be done only in Cartesian coordinates.

  Recall that the fact that Weingarten operator $S$ maps tangent vectors to tangent vectors follows from the property:
  $\langle\n,\X\rangle=0\Rightarrow \X$ is tangent to the surface.

  Indeed:
     $$
 0= \p_\X\langle\n,\n\rangle=2\langle\p_\X\n,\n\rangle=-2\langle S\X,\n\rangle=0\Rightarrow S\X\,\,
 \hbox {is tangent to the surface}
     $$
     \m
Recall also that normal unit vector is defined up to a sign, $\n\to -\n$. On the other hand if $\n$
is chosen then $\S$ is defined uniquely.


\subsubsection {Second quadratic form}

We define now the new object: {\it second quadratic form}

{\bf Definition}. For two tangent vectors $\X, \Y$
$A(\X,\Y)$ is defined such that
\begin{equation}\label{secondquadraticform1}
    \left(\nabla^{\rm can.flat}_\X \Y\right)_\bot=A(\X,\Y)\n
\end{equation}
i.e. we take orthogonal component of the derivative of $\Y$ with respect to $\X$.

This definition seems to be very vague: to evaluate covariant derivative we have to consider not a
vector $\Y$ at a given point
but  the vector field.
In fact one can see that  $A(\X,\Y)$ does depend only on the value of $\Y$ at the given point.

Indeed it follows from the definition of second quadratic form and from the properties of Weingarten operator that
        $$
         A(\X,\Y)=\left\langle\left(\nabla^{\rm can.flat}_\X \Y\right)_\bot,\n\right\rangle=
    \left\langle\nabla^{\rm can.flat}_\X \Y,\n\right\rangle=
        $$
\begin{equation}\label{properofsecquadrform4}
    \p_\X \langle\Y,\n\rangle-\left\langle \Y,\nabla^{\rm can.flat}_\X\n\right\rangle=\langle S(\X), \Y\rangle
\end{equation}



We proved that second quadratic form depends only on value of vector field $\Y$ at the given poit and we established
the relation between second quadratic form and Weingarten operator.

\m

{\bf Proposition} {\it The second quadratic form $A(\X,\Y)$ is symmetric bilinear form on tangent vectors $\X,\Y$
in a given point.}

       \begin{equation}\label{symmetricityofsecquadrform}
A\colon T_\pt M\otimes T_\pt M\to \R,\quad A(\X,\Y)=A(\Y,\X)=\langle S\X,\Y\rangle\,.
       \end{equation}

 In components
 \begin{equation}\label{compexpressforscquadrform}
    A=A_{\a\beta}du^\a du^\beta=\langle \r_{\a\beta}, \n\rangle={\p^2 x^i\over \p u^\a \p u^\beta}n^i\,.
 \end{equation}
and
\begin{equation}\label{compexpressforweingarten}
    S^\a_\beta=g^{\a\pi}A_{\pi\beta}=g^{\a\pi}x^i_{\pi\beta}n^i\,,
     \end{equation}
i.e.
            $$
         A=GS, S=G^{-1}A\,.
            $$

 {\bf Remark} The normal unit vector field is defined up to a sign.


\subsubsection {Gaussian and mean curvatures}

   Recall that Gaussian curvature
            $$
          K=\det S
            $$
and mean curvature
           $$
         H={\rm Tr \,}S
           $$
It is easy to see that  for Gaussian curvature
           $$
         K=\det S=\det(G^{-1}A)={\det A\over \det G}.
           $$

We know already the geometrical meaning of Gaussian and mean curvatures from the point of view of the External Observer:

Gaussian curvature $K$ equals to the product of principal curvatures, and mean curvartures equals to the sum of principal curvatures.

Now we ask a principal question: what bout internal observer, "aunt" living on the surface?

We will show that Gaussian curvature can be expressed in terms of induced Riemannian metric, i.e. it is an
 internal characteristic of the surface, invariant of isometries.

  It is not the case with mean curvature: cylinder is isometric to the plane but it have
   non-zero mean curvature.

\subsubsection {Examples of calculation of Weingarten operator,
Second quadratic forms, curvatures for cylinder, cone and sphere.}

\m

{\it Cylinder}

\m

  We already calculated induced Riemannian metric on the cylinder (see \eqref {formula forfirstformcyl}).

 Cylinder is given by the equation $x^2+y^2=R^2$. One can consider the following
parameterisation
 of this surface:
\begin{equation}\label{surface11}
  \r(h,\varphi)\colon\quad
  \begin{cases}
  x=R\cos\varphi\\
  y=R\sin\varphi\\
  z=h\\
  \end{cases}\,,\quad   \r_h=
  \begin{pmatrix}
        0\\
        0\\
        1\\
   \end{pmatrix}\,,
\quad
  \r_\varphi=\begin{pmatrix}
        -R\sin\varphi\\
        R\cos\varphi\\
          0\\
   \end{pmatrix}\,,
\end{equation}
$G_{cylinder}=\left(dx^2+dy^2+dz^2\right)\big\vert_{x=a\cos\varphi,y=a\sin\varphi,z=h}=$
        \begin{equation*}\label{firtsquadraticformcylinder(diff)11}
               =(-a\sin\varphi d\varphi)^2+(a\cos\varphi d\varphi)^2+dh^2=a^2d\varphi^2+dh^2\,,\quad
               ||g_{\a\beta}||=
  \begin{pmatrix}
   1 & 0 \\
   0& R^2 \\
   \end{pmatrix}\,.
        \end{equation*}

   Normal unit vector $\n=\pm \begin{pmatrix}
        \cos \varphi\cr
        \sin\varphi\cr
        0\cr
   \end{pmatrix}$. Choose $\n=\begin{pmatrix}
        \cos \varphi\cr
        \sin\varphi\cr
        0\cr
   \end{pmatrix}$. Weingarten operator
     \begin{equation*}\label{weingrartenforcylinder}
        S\p_h=-\nabla^{\rm can.flat}_{\r_h}\n=-\p_{\r_h} \n=-\p_h\begin{pmatrix}
        \cos \varphi\cr
        \sin\varphi\cr
        0\cr
   \end{pmatrix}=0\,,
     \end{equation*}
\begin{equation*}\label{weingrartenforcylinder}
        S\p_\varphi=-\nabla^{\rm can.flat}_{\r_\varphi}\n=-\p_{\r_\varphi} \n=-{\p_\varphi}
        \begin{pmatrix}
        \cos \varphi\cr
        \sin\varphi\cr
        0\cr
   \end{pmatrix}=\begin{pmatrix}
        \sin\varphi \cr
        -\cos\varphi\cr
        0\cr
   \end{pmatrix}=-{\p_\varphi\over R}.
    \end{equation*}
\begin{equation}\label{weingrartenforcylinder2}
           S
   \begin{pmatrix}
        \p_h \cr
        \p_\varphi\cr
   \end{pmatrix}=
   \begin{pmatrix}
         0\cr
        {-\p_\varphi\over R}\cr
   \end{pmatrix},\quad S=\begin{pmatrix}0&0\cr 0 &{-1\over R}\end{pmatrix}\,.
    \end{equation}
Calculate   second quadratic form:  $\r_{hh}=\p_h\r_h=
  \begin{pmatrix}
        0\\
        0\\
        0\\
   \end{pmatrix}\,$, $\r_{h\varphi}=\r_{\varphi h}=$
             $$
             \p_h
       \begin{pmatrix}
        -R\sin\varphi\\
        R\cos\varphi\\
          0\\
   \end{pmatrix}=0,\,\,
   \r_{\varphi \varphi}=\p_\varphi
       \begin{pmatrix}
        -R\sin\varphi\\
        R\cos\varphi\\
          0\\
   \end{pmatrix}=\begin{pmatrix}
        -R\cos\varphi\\
        -R\sin\varphi\\
          0\\
   \end{pmatrix}=-R\n\,.
                      $$
We have
                    \begin{equation}\label{secquadrformforcylinder}
            A_{\a\beta}=\langle\r_{\a\beta},\n\rangle,\,
              A=\begin{pmatrix}\langle\r_{hh},\n\rangle&\langle\r_{h\varphi},\n\rangle\cr
                               \langle\r_{\varphi h},\n\rangle &\langle\r_{\varphi\varphi},\n\rangle\cr
                                   \end{pmatrix}=
                                   \begin{pmatrix}0&0\cr
                                0&-R\cr
                                   \end{pmatrix},
                    \end{equation}
                    $$
       A=GS=\begin{pmatrix}1&0\cr
                                0&R^2\cr
                                   \end{pmatrix}
                                   \begin{pmatrix}0&0\cr
                                0&-{1\over R}\cr
                                   \end{pmatrix}=
                                   \begin{pmatrix}0&0\cr
                                0&-R\cr
                                   \end{pmatrix},
                    $$
For Gaussian and mean curvatures we have
     \begin{equation}\label{Gaussianforcylinder}
        K=\det S={\det A\over \det G}=\det
                              \begin{pmatrix}0&0\cr
                                0&-{1\over R}\cr
                                   \end{pmatrix}=0\,,
     \end{equation}
     and mean curvature
      \begin{equation}\label{meanianforcylinder}
        H={\rm Tr\,} S={\rm Tr\,}
                              \begin{pmatrix}0&0\cr
                                0&-{1\over R}\cr
                                   \end{pmatrix}=-{1\over R}\,.
     \end{equation}
Mean curvature is define up to a sign. If we change $\n\to-\n$ mean curvature $H\to {1\over R}$ and Gaussian curvature
will not change.


\bigskip

{\it Cone}

  We already calculated induced Riemannian metric on the cone (see \eqref{firtsquadraticformforconus(diff)}.

 Cone is given by the equation $x^2+y^2-k^2z^2=0$. One can consider the following
parameterisation
 of this surface:
\begin{equation}\label{surface111}
  \r(h,\varphi)\colon\quad
  \begin{cases}
  x=kh\cos\varphi\\
  y=kh\sin\varphi\\
  z=h\\
  \end{cases}\,,\quad   \r_h=
  \begin{pmatrix}
        k\cos\varphi\\
        k\sin\varphi\\
        1\\
   \end{pmatrix}\,,
\quad
  \r_\varphi=\begin{pmatrix}
        -kh\sin\varphi\\
        kh\cos\varphi\\
          0\\
   \end{pmatrix}\,,
\end{equation}
$G_{cone}=\left(dx^2+dy^2+dz^2\right)\big\vert_{x=kh\cos\varphi,y=kh\sin\varphi,z=h}=$
        \begin{equation*}\label{firtsquadraticformcylinder(diff)11}
               =(-kh\sin\varphi d\varphi+k\cos\varphi dh)^2+
               (kh\cos\varphi d\varphi+k\sin\varphi dh)^2+dh^2=
                      \end{equation*}
             $$
              k^2h^2d\varphi^2+(k^2+1)dh^2,\quad
               ||g_{\a\beta}||=
  \begin{pmatrix}
   k^2+1 & 0 \\
   0& k^2h^2 \\
   \end{pmatrix}\,.
          $$

  One can see that  $\N=
      \begin{pmatrix}
        \cos \varphi\cr
        \sin\varphi\cr
        -k\cr
   \end{pmatrix}$ is orthogonal to the surface: $\N\bot \r_h,\r_\varphi$. Hence
   normal unit vector $\n=\pm
     {1\over {\sqrt {1+k^2}}}
      \begin{pmatrix}
        \cos \varphi\cr
        \sin\varphi\cr
        -k\cr
   \end{pmatrix}$. Choose $\n=
   {1\over {\sqrt {1+k^2}}}
      \begin{pmatrix}
        \cos \varphi\cr
        \sin\varphi\cr
        -k\cr
   \end{pmatrix}$.
     Weingarten operator
     \begin{equation*}\label{weingrartenforcylinder}
        S\p_h=-\nabla^{\rm can.flat}_{\r_h}\n=-\p_{\r_h} \n=
        -\p_h
        \left(
        {1\over {\sqrt {1+k^2}}}
      \begin{pmatrix}
        \cos \varphi\cr
        \sin\varphi\cr
        -k\cr
   \end{pmatrix}
   \right)
        =0\,,
     \end{equation*}
\begin{equation*}\label{weingrartenforcylinder}
        S\p_\varphi=-\nabla^{\rm can.flat}_{\r_\varphi}\n=-\p_{\r_\varphi} \n=
        -\p_\varphi
        \left(
{1\over {\sqrt {1+k^2}}}
      \begin{pmatrix}
        \cos \varphi\cr
        \sin\varphi\cr
        -k\cr
   \end{pmatrix}
   \right)
       =
       \end{equation*}
       \begin{equation*}
       {1\over {\sqrt {1+k^2}}}
       \begin{pmatrix}
        \sin\varphi \cr
        -\cos\varphi\cr
        0\cr
   \end{pmatrix}=-{\p_\varphi\over kh\sqrt {k^2+1}}.
    \end{equation*}
\begin{equation}\label{weingrartenforcylinder2}
           S
   \begin{pmatrix}
        \p_h \cr
        \p_\varphi\cr
   \end{pmatrix}=
   \begin{pmatrix}
         0\cr
        -{\p_\varphi\over kh\sqrt {k^2+1}}\cr
   \end{pmatrix},\quad S=\begin{pmatrix}0&0\cr 0 &{-1\over kh\sqrt {k^2+1}}\end{pmatrix}\,.
    \end{equation}
Calculate   second quadratic form:  $\r_{hh}=\p_h\r_h=
  \begin{pmatrix}
        0\\
        0\\
        0\\
   \end{pmatrix}\,$, $\r_{h\varphi}=\r_{\varphi h}=$
             $$
             \p_h
       \begin{pmatrix}
        -kh\sin\varphi\\
        kh\cos\varphi\\
          0\\
   \end{pmatrix}=\begin{pmatrix}
        -k\sin\varphi\\
        k\cos\varphi\\
          0\\
   \end{pmatrix},\,\,
   \r_{\varphi \varphi}=\p_\varphi
       \begin{pmatrix}
        -kh\sin\varphi\\
        kh\cos\varphi\\
          0\\
   \end{pmatrix}=\begin{pmatrix}
        -kh\cos\varphi\\
        -kh\sin\varphi\\
          0\\
   \end{pmatrix}\,.
                      $$
We have
                    \begin{equation}\label{secquadrformforcylinder}
            A_{\a\beta}=\langle\r_{\a\beta},\n\rangle,\,
              A=\begin{pmatrix}\langle\r_{hh},\n\rangle&\langle\r_{h\varphi},\n\rangle\cr
                               \langle\r_{\varphi h},\n\rangle &\langle\r_{\varphi\varphi},\n\rangle\cr
                                   \end{pmatrix}=
                                   \begin{pmatrix}0&0\cr
                                0&-{kh\over {\sqrt {1+k^2}}}\cr
                                   \end{pmatrix},
                    \end{equation}
                    $$
       A=GS=\begin{pmatrix}k^2+1&0\cr
                                0&k^2h^2\cr
                                   \end{pmatrix}
                                   \begin{pmatrix}0&0\cr
                                0&{-1\over kh\sqrt {k^2+1}}\cr
                                   \end{pmatrix}=
                                   \begin{pmatrix}0&0\cr
                                0&{-kh\over \sqrt {k^2+1}}\cr
                                   \end{pmatrix},
                    $$
For Gaussian and mean curvatures we have
     \begin{equation}\label{Gaussianforcylinder}
        K=\det S={\det A\over \det G}=\det
                              \begin{pmatrix}0&0\cr
                                0&{-1\over kh\sqrt {k^2+1}}\cr
                                   \end{pmatrix}=0\,,
     \end{equation}
     and mean curvature
      \begin{equation}\label{meanianforcylinder}
        H={\rm Tr\,} S={\rm Tr\,}
                              \begin{pmatrix}0&0\cr
                                0&{-1\over kh\sqrt {k^2+1}}\cr
                                   \end{pmatrix}={-1\over kh\sqrt {k^2+1}}\,.
     \end{equation}
Mean curvature is define up to a sign. If we change $\n\to-\n$ mean curvature $H\to {1\over R}$ and Gaussian curvature
will not change.


\bigskip




{\it Sphere}


\medskip


  Sphere is given by the equation $x^2+y^2+z^2=a^2$. Consider the  parameterisation
 of sphere in spherical coordinates
\begin{equation}\label{surfacesphere11}
  \r(\theta,\varphi)\colon\quad
  \begin{cases}
  x=R\sin\theta\cos\varphi\\
  y=R\sin\theta\sin\varphi\\
  z=R\cos\theta\\
  \end{cases}
\end{equation}

\medskip

We already calculated induced Riemannian metric on the sphere (see \eqref{firtsquadraticformforsphere(diff)}).
 Recall that
    $$
  \r_\theta=\begin{pmatrix}
        R\cos\theta\cos\varphi\\
        R\cos\theta\sin\varphi\\
        -R\sin\theta\\
   \end{pmatrix}\,,
\quad
  \r_\varphi=\begin{pmatrix}
        -R\sin\theta\sin\varphi\\
        R\sin\theta\cos\varphi\\
          0\\
   \end{pmatrix}
  $$
 and
            $$
              G_{S^2}=\left(dx^2+dy^2+dz^2\right)\big\vert_{x=R\sin\theta\cos\varphi,y=R\sin\theta\sin\varphi,
              z=R\cos\theta}=
                      $$
                      $$
                      (R\cos\theta\cos\varphi d\theta-R\sin\theta\sin\varphi d\varphi)^2+
                      (R\cos\theta\sin\varphi d\theta+R\sin\theta\cos\varphi d\varphi)^2+
                      $$
                      $$
                      (-R\sin\theta d\theta)^2=
          R^2\cos^2\theta d\theta^2+R^2\sin^2\theta d\varphi^2+R^2\sin^2\theta d\theta^2=
                      $$
        \begin{equation*}\label{firtsquadraticformforsphere(diff)}
           \qquad
             =R^2d\theta^2+R^2\sin^2\theta d\varphi^2\,,\qquad
                        ||g_{\a\beta}||=
   \begin{pmatrix}
   R^2 & 0 \\
   0&  R^2\sin^2\theta \\
   \end{pmatrix}\,.
                       \end{equation*}

For the sphere $\r(\theta,\varphi)$ is orthogonal to the surface.
       Hence
   normal unit vector
   $   \n(\theta,\varphi)=\pm {\r(\theta,\varphi)\over R}=\pm
   \begin{pmatrix}
    \sin\theta\cos\varphi\cr
     \sin\theta\sin\varphi\cr
      \cos\theta\cr
   \end{pmatrix}$.
           Choose $\n= {\r\over R}=
   \begin{pmatrix}
    \sin\theta\cos\varphi\cr
     \sin\theta\sin\varphi\cr
      \cos\theta\cr
   \end{pmatrix}.
     $
     Weingarten operator
     \begin{equation*}\label{weingrartenforcylinder}
        S\p_\theta=-\nabla^{\rm can.flat}_{\r_\theta}\n=-\p_\theta \n=
          -\p_\theta\left({\r\over R}\right)=-{\r_\theta\over R}\,,
       \end{equation*}
       \begin{equation*}\label{weingrartenforcylinder}
       S\p_\varphi=-\nabla^{\rm can.flat}_{\r_\varphi}\n=-\p_\varphi \n=
          -\p_\varphi\left({\r\over R}\right)=-{\r_\varphi\over R}\,.
    \end{equation*}
\begin{equation}\label{weingrartenforcylinder2}
           S
   \begin{pmatrix}
        \p_\theta \cr
        \p_\varphi\cr
   \end{pmatrix}=
   \begin{pmatrix}
         -{\p_\theta\over R}\cr
        -{\p_\varphi\over R}\cr
   \end{pmatrix},\quad S=
        -
   \begin{pmatrix}{1\over R}&0\cr 0 &{1\over R}\end{pmatrix}\,.
    \end{equation}
For second quadratic form:  $\r_{\theta\theta}=\p_\theta\r_\theta=
  \begin{pmatrix}
        -R\sin\theta\cos\varphi\\
        -R\sin\theta\sin\varphi\\
        -R\cos\theta\\
   \end{pmatrix}\,$, $\r_{\theta\varphi}=\r_{\varphi \theta}=$
             $$
             \p_\theta
       \begin{pmatrix}
        -R\sin\theta\sin\varphi\\
        R\sin\theta\cos\varphi\\
          0\\
   \end{pmatrix}=\begin{pmatrix}
        -R\cos\theta\sin\varphi\\
        R\cos\theta\cos\varphi\\
          0\\
   \end{pmatrix},\,\,
   \r_{\varphi \varphi}=\p_\varphi
       \begin{pmatrix}
        -R\sin\theta\sin\varphi\\
          R\sin\theta\cos\varphi\\
          0\\
   \end{pmatrix}=\begin{pmatrix}
        -R\sin\theta\cos\varphi\\
        -R\sin\theta\sin\varphi\\
          0\\
   \end{pmatrix}\,.
                      $$
We have
                    \begin{equation}\label{secquadrformforcylinder}
            A_{\a\beta}=\langle\r_{\a\beta},\n\rangle,\,
              A=\begin{pmatrix}\langle\r_{\theta\theta},\n\rangle&\langle\r_{\theta\varphi},\n\rangle\cr
                               \langle\r_{\varphi \theta},\n\rangle &\langle\r_{\varphi\varphi},\n\rangle\cr
                                   \end{pmatrix}=
                                   \begin{pmatrix}-R&0\cr
                                0&-R\sin^2\theta\cr
                                   \end{pmatrix},
                    \end{equation}
                    $$
       A=GS=\begin{pmatrix}R^2&0\cr
                                0&R^2\sin^2\theta\cr
                                   \end{pmatrix}
                                   \begin{pmatrix}{-1\over R}&0\cr
                                0&{-1\over R}\cr
                                   \end{pmatrix}=
                                       -R
                                  \begin{pmatrix}1&0\cr
                                0&\sin^2\theta\cr
                                   \end{pmatrix},
                    $$
For Gaussian and mean curvatures we have
     \begin{equation}\label{Gaussianforcylinder}
        K=\det S={\det A\over \det G}=\det
                              \begin{pmatrix}-{1\over R}&0\cr
                                0&-{1\over R}\cr
                                   \end{pmatrix}={1\over R^2}\,,
     \end{equation}
     and mean curvature
      \begin{equation}\label{meanianforcylinder}
        H={\rm Tr\,} S={\rm Tr\,}
                              \begin{pmatrix}-{1\over R}&0\cr
                                0&-{1\over R}\cr
                                   \end{pmatrix}=-{2\over R}\,,
     \end{equation}
Mean curvature is define up to a sign. If we change $\n\to-\n$ mean curvature $H\to {1\over R}$ and Gaussian curvature
will not change.


We see that for the sphere  Gaussian curvature is not equal to zero, whilst for cylinder and cone Gaussian curvature
equals to zero.


\subsection {Parallel transport along closed curves on surfaces. Derivation formulae and {\it Theorema Egregium} }

\subsubsection {Formulation of the main result}

Let $M$ be a surface in  Euclidean space $\E^3$.
  Consider a closed curve $C$ on $M$,
  $M\colon \r=\r(u,v)$, $C\colon \r=\r(u(t,v(t)), 0\leq t\leq t_1, \, \x(0)=\x(t_1)$.
  ($u(t), v(t)$ are internal coordinates of the curve $C$.)

Consider the parallel transport of an arbitrary tangent $\X$ vector along the closed curve $C$:
          \begin{equation*}\label{paraltransportalongclosedcurve1}
\X(t)= \underbrace {X^\a(t){\p\over \p u_\a}\big\vert_{u^\a(t)}}_{\hbox{Internal observer}} =
 \underbrace{X^\a(t)\r_\a\big\vert_{\r(u(t),v(t))}}_{\hbox{External observer}}\,\,,\left(
 \r_\a={\p x^i\over \p u^\a}{\p\over \p x^i}\right)\,.
          \end{equation*}
          \begin{equation*}\label{paraltransportalongclosedcurve2}
\X(t)\colon\,\, {\nabla \X(t)\over dt}=0,\,\, 0\leq t\leq t_1\,,
          \end{equation*}
i.e.
                \begin{equation}\label{paraltransportalongclosedcurve3}
{d X^\a(t)\over dt}+
X^\beta(t)\Gamma^\a_{\beta\gamma}(u(t))
{du^\gamma(t)\over dt}=0,\,\, 0\leq t\leq t_1\,,
          \end{equation}
where $\nabla$ is the connection induced on the surface $M$ by canonical flat connection
(see \eqref{christoffelsymbols1forinducedconnection}), i.e.
the Levi-Civita connection of the induced Riemannian metric on the surface $M$
 and $\Gamma^\a_{\beta\gamma}$ its Christoffel symbols:
\begin{equation}\label{connectioninduced12}
          \Gamma^\gamma_{\a\beta}={1\over 2}g^{\gamma\pi}
      \left({\p g_{\pi \a}\over \p u^\beta}+{\p g_{\pi \beta}\over \p u^\a}-
      {\p g_{\a\beta}\over \p u^\pi}\right)\,,{\rm where}\,
      g_{\a\beta}=\langle\r_\a,\r_\beta\rangle={\p x^i\over \p u^\a} {\p x^i\over \p u^\beta}
     \end{equation}
 are components of induced Riemannian metric
 $G_M=g_{\a\beta}du^\a$    (see the subsection 4.2.1)


Let $\r(0)=\pt$ be a starting (and ending) point of the curve $C$: $\r(0)=\r(t_1)=\pt$.
The differential equation \eqref{paraltransportalongclosedcurve3} defines the linear operator
            \begin{equation}\label{linearoperatoroverclosedcurve}
            R_C\colon T_{\pt}M\longrightarrow T_{\pt}M
            \end{equation}
For any vector $\X\in T_{\pt}M$, its image the vector $R_C\X$ as the solution of the differential equation
\eqref{paraltransportalongclosedcurve3} with initial condition $\X(t)\big\vert_{t=0}=\X$.

On the other hand we know that parallel transport of the vector does not change its length (see
\eqref{lengthdoesnotchangeduringparalleltransport} in the subsection 3.2.1):
       \begin{equation}\label{linearoperatoroverclosedcurvescalarproduct}
            \langle\X,\X\rangle=\langle R_C\X,R_C\X\rangle
            \end{equation}
We see that $\R_C$ is an orthogonal operator in the $2$-dimensional vector space $T_{\pt}M$.
We know that orthogonal operator preserving orientation is the operator of rotation on some angle $\Phi$.

We see that if $\R_C$ preserves orientation\footnote{We consider the case if the operator $R_C$ preserves orientation.
In our considerations we consider only the case if the closed curve $C$ is a boundary of
a compact oriented domain $D\subset M$. In this case one can see that operator $R_C$ preserves an orientation.}
 then the action of operator $\R_C$ on vectors is rotation on the angle,
i.e. the result of parallel transport along closed curve is rotation on the angle
                               \begin{equation}\label{rotationontheangle}
                               \Delta\Phi=\Delta\Phi(C)
                               \end{equation}
 which depends on the curve.


The very beautiful question arises:  How to calculate this angle $\Delta\Phi(C)$


\m

{\bf Theorem}
{\it  Let $M$ be a surface in  Euclidean space $\E^3$.
  Let $C$ be a closed curve $C$ on $M$ such that $C$ is a boundary of a compact oriented domain $D\subset M$.
Consider the parallel transport of an arbitrary tangent vector along the closed curve $C$.
As a result of parallel transport along this closed curve any  tangent vector rotates through the angle

\begin{equation}\label{theoremofrotationonangle}
\Delta\Phi=\angle\left({\X, \R_C\X}\right)=\int_D K d\sigma\,,
             \end{equation}
where $K$ is the Gaussian curvature and $d\sigma=\sqrt {\det g}dudv$ is the area element induced by the
Riemannian metric on the surface $M$, i.e.  $d\sigma=\sqrt {\det g}dudv$.



}

\m
{\bf Remark} One can show that the angle of rotation does not depend on initial point of the curve.


\m


{\bf Example} Consider the closed curve, "latitude" $C_{\theta_0}\theta=\theta_0$ on the sphere of the radius $R$.
Calculations show that
               \begin{equation}\label{rotationforlatitude}
                \Delta\Phi(C_{\theta_0})=2\pi(1-\cos\theta_0)
               \end{equation}

(see the Homework 6). On the other hand the latitude $C_{\theta_0}$ is the boundary of the segment  $D$
 with area $2\pi RH$ where $H=R(1-\cos \theta_0)$. Hence
           $$
   \angle\left({\X, \R_C\X}\right)={2\pi RH\over R^2}={1\over R^2}\cdot \hbox{area of the segment}=\int_D  Kd\sigma
           $$
since Gaussian curvature is equal to $1\over R^2$

The  proof of this Theorem is the content of the subsections above.

Before show one of the remarkable corollaries of this Theorem.

Let  $D$ be a small domain around  a given point $\pt$, let $C$ its boundary and
$\Delta \Phi(D)$ the angle of rotation.   Denote by $S(D)$ an area of this domain.
Applying the Theorem for the case when area of the domain $D$ tends to zero we  we come to the statement that
            $$
      \hbox{if $S(D)\to 0$ then \,} \Delta\Phi(D)=\int_D Kd\sigma\to K(\pt)S(D),\,{\rm i.e.}
            $$

\begin{equation}\label{egregium1}
K(\pt)=\lim_{S(D)\to 0}{\Delta\Phi(D)\over S(D)},\,{\rm i.e.}
\end{equation}

Now notice that  left hand side od this equation defining Gaussian curvature $K(\pt)$ depends only on Riemannian
metric on the surface $C$. Indeed numerator of LHS is defined by the solution of differential equation
$\eqref{paraltransportalongclosedcurve3}$ which depends on Levi-Civita connection depending on
the induced Riemannian metric, and denominator is an area depending on Riemannian metric too.

Thus we come to very important

{\bf Corollary} {\it Gau\ss \,\, Egregium Theorema}

Gaussian curvature of the surface is invariant of isometries.


\bigskip


In next subsections we develop the technique which  itself is very interesting. One of the applications of this
technique is the proof of the Theorem \eqref{theoremofrotationonangle}.
This we will prove Theorema Egregium too.

Later in the fifth section we will give another proof of the Theorema  Egregium.

\subsubsection{ Derivation formulae}

Let $M$ be a surface embedded in  Euclidean space $\E^3$,
$M\colon \r=\r(u,v)$.

   Let ${\e,\f,\n}$ be three vector fields defined on the points of this surface
   such that they form an orthonormal basis at any point, so that the vectors $\e,\f$ are tangent to the surface
   and  the vector $\n$ is orthogonal to the surface\footnote{
   One can say that $\{\e,\f,\n\}$ is an orthonormal basis in $T_{\pt}\E^3$ at every point of surface  $\pt\in M$
such that $\{\e,\f\}$ is an orthonormal basis in $T_{\pt}\E^3$ at every point of surface  $\pt\in M$.}.
   Vector fields $\e,\f,\n$ are functions on the surface $M$:
                   $$
          \e=\e(u,v),\,\, \f=\f(u,v)\,, \n=\n(u,v)\,.
                   $$
 Consider $1$-forms $d\e,d\f,d\n$:
              $$
       d\e={\p \e\over \p u}du+{\p \e\over \p v}dv,\,
       d\f={\p \f\over \p u}du+{\p \f\over \p v}dv\,,
       d\n={\p \n\over \p u}du+{\p \n\over \p v}dv
              $$
 These $1$-forms take values in the vectors in $\E^3$, i.e. they are {\it vector valued} $1$-forms.
 Any vector in $\E^3$ attached at an arbitrary point of the surface
 can be expanded over the basis  $\{\e,\f,\n\}$. Thus vector valued $1$-forms $d\e,d\f,d\n$ can be expanded
 in a sum of $1$-forms with values in basic vectors  $\e,\f,\n$. E.g.
 for $d\e={\p \e\over \p u}du+{\p \e\over \p v}dv$ expanding   vectors
 ${\p \e\over \p u}$ and ${\p \e\over \p v}$ over basis vectors we come to
            $$
      {\p \e\over \p u}=A_1(u,v)\e+B_1(u,v)\f+C_1(u,v)\n,\,\,
      {\p \e\over \p v}=A_2(u,v)\e+B_2(u,v)\f+C_2(u,v)\n
            $$
 thus
          $$
   d\e={\p \e\over \p u}du+{\p \e\over \p v}dv=\left(A_1\e+B_1\f+C_1\n\right)du+\left(A_2\e+B_2\f+C_2\n\right)dv=
          $$
           \begin{equation}\label{one-forms4}
 =\underbrace{(A_1du+A_2dv)}_{M_{11}}\e+
 \underbrace{(B_1du+B_2dv)}_{M_{12}}\f+\underbrace{(C_1du+C_2dv)}_{M_{11}}\e,
           \end{equation}
 i.e.
            $$
       d\e=M_{11}\e+M_{12}\f+M_{13}\n,
            $$
where $M_{11}, M_{12}$ and $M_{13}$ are $1$-forms on the surface $M$ defined by the relation \eqref{one-forms4}.

In the same way we do the expansions of vector-valued $1$-forms $d\f$ and $d\n$ we come to
                 \begin{equation*}
                \begin{matrix}
                d\e=M_{11}\e+M_{12}\f+M_{13}\n\cr
                d\f=M_{21}\e+M_{22}\f+M_{23}\n\cr
                d\n=M_{31}\e+M_{32}\f+M_{33}\n\cr
                \end{matrix}
                    \end{equation*}
   This equation can be rewritten in the following way:
                  \begin{equation}\label{matrixequation1}
                       d
                  \begin{pmatrix}
                \e\cr
                \f\cr
                \n\cr
                \end{pmatrix}
                    =
                \begin{pmatrix}
                M_{11} &M_{12} &M_{13}\cr
                M_{21} &M_{22} &M_{23}\cr
                M_{31} &M_{32} &M_{33}\cr
                \end{pmatrix}
                      \begin{pmatrix}
                \e\cr
                \f\cr
                \n\cr
                \end{pmatrix}
                    \end{equation}

       \m

  {\it Proposition} {\it The matrix $M$ in the equation \eqref{matrixequation1} is antisymmetrical matrix, i.e.
                     \begin{equation}
                       \begin{matrix}
                    M_{11}=M_{22}=M_{33}=0\cr
                       M_{12}=-M_{21}=a\cr
                       M_{13}=-M_{31}= b\cr
                       M_{23}=-M_{32}=-b\cr
                       \end{matrix}
                       \end{equation}
i.e.



                      \begin{equation}\label{derivationformulae1}
                    d\begin{pmatrix}
                    \e\cr\f\cr\n\cr
                    \end{pmatrix}=
                    \begin{pmatrix}
                    0&a&b\cr -a&0&c\cr -b&-c&0\cr
                    \end{pmatrix}
                 \begin{pmatrix}
                    \e\cr\f\cr\n\cr
                    \end{pmatrix}\,,
                       \end{equation}
 where $a,b,c$ are $1$-forms on the surface $M$.}


 \m
Formulae \eqref{derivationformulae1} are called {\it derivation formulae}.


 Prove this Proposition. Recall that $\{\e,\f,\n\}$ is orthonormal basis, i.e. at every point of the surface
                  $$
  \langle\e,\e\rangle=\langle\f,\f\rangle=\langle\n,\n\rangle=1, \,{\rm and}\,\,
   \langle\e,\f\rangle=\langle\e,\n\rangle=\langle\f,\n\rangle=0
         $$
 Now using \eqref{matrixequation1} we have
         $$
  \langle\e,\e\rangle=1\Rightarrow d\langle\e,\e\rangle=0=2\langle\e,d\e\rangle
  =\langle\e,M_{11}\e+M_{12}\f+M_{13}\n\rangle=
               $$
               $$
               M_{11}\langle\e,\e\rangle+M_{12}\langle\e,\f\rangle+
  M_{13}\langle\e,\n\rangle=M_{11}\Rightarrow M_{11}=0\,.
              $$
 Analogously
           $$
     \langle\f,\f\rangle=1\Rightarrow d\langle\f,\f\rangle=0=2\langle\f,d\f\rangle
  =\langle\f,M_{21}\e+M_{22}\f+M_{23}\n\rangle=M_{22}\Rightarrow M_{22}=0\,,
           $$
            $$
             \langle\n,\n\rangle=1\Rightarrow d\langle\n,\n\rangle=0=2\langle\n,d\n\rangle
  =\langle\n,M_{31}\e+M_{32}\f+M_{33}\n\rangle=M_{33}. \Rightarrow M_{33}=0\,.
            $$
 We proved already that $M_{11}=M_{22}=M_{33}=0$. Now prove that
 $M_{12}=-M_{21}$, $M_{13}=-M_{31}$ and $M_{13}=-M_{31}$.
                       $$
  \langle\e,\f\rangle=0\Rightarrow d\langle\e,\f\rangle=0=\langle\e,d\f\rangle+\langle d\e,\f\rangle=
             $$
             $$
  \langle\e,M_{21}\e+M_{22}\f+M_{23}\n\rangle+\langle M_{11}\e+M_{12}\f+M_{13}\n,\f\rangle=
  M_{21}+M_{12}=0.
          $$
 Analogously
       $$
       \langle\e,\n\rangle=0\Rightarrow d\langle\e,\n\rangle=0=\langle\e,d\n\rangle+\langle d\e,\n\rangle=
             $$
             $$
  \langle\e,M_{31}\e+M_{32}\f+M_{33}\n\rangle+\langle M_{11}\e+M_{12}\f+M_{13}\n,\n\rangle=
  M_{31}+M_{13}=0
       $$
 and
          $$
          \langle\f,\n\rangle=0\Rightarrow d\langle\f,\n\rangle=0=\langle\f,d\n\rangle+\langle d\f,\n\rangle=
             $$
             $$
  \langle\f,M_{31}\e+M_{32}\f+M_{33}\n\rangle+\langle M_{21}\e+M_{22}\f+M_{23}\n,\n\rangle=
  M_{32}+M_{23}=0.
          $$

       {\bf Remark} This proof could be performed much more shortly in condensed notations. Derivation formulae
       \eqref{derivationformulae1} in condensed notations are
          \begin{equation}\label{derivationcondensed}
          d\e_i=M_{ik}\e_k
          \end{equation}
     Orthonormality condition means that $\langle\e_i,\e_k\rangle=\delta_{ik}$. Hence
        \begin{equation}
   d\langle\e_i,\e_k\rangle=0=\langle d\e_i,\e_k\rangle+\langle\e_i,d\e_k\rangle=
   \langle  M_{im}\e_m,\e_k\rangle+\langle\e_i,M_{kn}\e_n\rangle=M_{ik}+M_{ki}=0\hbox{\finish}
        \end{equation}
    Much shorter, is not it?


 \subsubsection  {Gauss condition (structure equations)}
    Derive the relations between $1$-forms $a,b$ and $c$ in derivation formulae.

    Recall that $a,b,c$ are $1$-forms, $\e,f,\n$ are vector valued functions ($0$-forms)
    and  $d\e,d\f,d\n$ are vector valued $1$-forms.
    (We use the simple identity that $ddf=0$ and the fact that for $1$-form $\w\wedge \w=0$.)
     We have from derivation formulae \eqref{derivationformulae1} that
         $$
     d^2\e=0= d(a\f+b\n)=da\f-a\wedge d\f+db\n-b\wedge d\n=
        $$
        $$
        da\, \f-a\wedge (-a \e+c\n)+db\,\n-b\wedge(-b\e-c\f)=
                       $$
                       $$
          (da+b\wedge c)\f+(a\wedge a+b\wedge b)\e+(db-a\wedge c)\n=
          (da+b\wedge c)\f+(db+c\wedge a)\n=0\,.
                      $$
We see that
                \begin{equation}\label{gausscondition1}
                    (da+b\wedge c)\f+(db+c\wedge a)\n=0
                \end{equation}
                Hence components of the left hand side equal to zero:
\begin{equation}\label{gausscondition2}
                    (da+b\wedge c)=0\,\,(db+c\wedge a)=0\,.
                \end{equation}
Analogously
              $$
                       d^2\f=0= d(-a\e+c\n)=-da\e+a\wedge d\e+dc\n-c\wedge d\n=
        $$
        $$
        -da\e+a\wedge (a \f+b\n)+dc\,\n-c\wedge(-b\e-c\f)=
                       $$
                       $$
          (-da+c\wedge b)\e+(dc+a\wedge b)\n=0\,.
              $$


 Hence we come to structure equations:

               \begin{equation}\label{firststructureformula}
               \begin{matrix}
                da+b\wedge c=0\cr
                 db+c\wedge a=0\cr
                 dc+a\wedge a=0\cr
                 \end{matrix}
               \end{equation}


          \subsubsection {Geometrical meaning of derivation formulae.
         Weingarten operator and second quadratic form in terms of derivation formulae. }




   Let $M$ be a surface in $\E^3$.

          Let ${\e,\f,\n}$ be three vector fields defined on the points of this surface
   such that they form an orthonormal basis at any point, so that the vectors $\e,\f$ are tangent to the surface
   and  the vector $\n$ is orthogonal to the surface.
   Note that in generally these vectors are not coordinate vectors.

     Describe  Riemannian geometry on the surface
     $M$ in terms of this basis and derivation formulae \eqref{derivationformulae1}.




    \m



    {\it Induced Riemannian metric}

    If $G$ is the Riemannian metric induced on the surface $M$ then since $\e,\f$ is orthonormal basis
    at every tangent space $T_\pt M$ then
        \begin{equation}\label{inducriemmetricinnonholonombasis}
        G(\e,\e)=G(\f,\f)=1,\,\,\,G(\e,\f)=G(\f,\e)=0
        \end{equation}
       The matrix of the Riemannian metric in the basis $\{\e,\f\}$ is
     \begin{equation}\label{inducedriemmetricinnonholonombasis2}
        G=
        \begin{pmatrix}
        1& 0\cr
        0& 1\cr
        \end{pmatrix}
        =
        \begin{pmatrix}
        b(\e)&  b(\f)\cr
        c(\e)& c(\f)\cr
        \end{pmatrix}
     \end{equation}


      \m


    {\it Induced connection}
    Let $\nabla$ be the connection induced by the canonical flat connection on the surface $M$.

    Then according equations \eqref{inducedconnection4} and derivation formulae \eqref{derivationformulae1}
    for every tangent vector $\X$
        \begin{equation}\label{inducedconnectionintermsofnonholonombasis1}
        \nabla_\X \e=\left(\p_\X\e\right)_{\rm tangent}=\left(d\e(\X)\right)_{\rm tangent}=
        \left(a(\X)\f+b(\X)\n\right)_{\rm tangent}=a(\X)\f\,.
        \end{equation}
and
    \begin{equation}\label{inducedconnectionintermsofnonholonombasis2}
        \nabla_\X \f=\left(\p_\X\f\right)_{\rm tangent}=\left(d\f(\X)\right)_{\rm tangent}=
        \left(-a(\X)\e+c(\X)\n\right)_{\rm tangent}=-a(\X)\e\,.
        \end{equation}
 In particular
              \begin{equation}\label{inducedconnectionintermsofnonholonombasis3}
              \begin{matrix}
        \nabla_\e \e=a(\e)\f  &   \nabla_\f \e=a(\f)\f\cr
        \nabla_\e \f=-a(\e)\e  &   \nabla_\f \f=-a(\f)\e\cr
        \end{matrix}
        \end{equation}
We know that the connection $\nabla$ is Levi-Civita connection of the induced Riemannian metric
\eqref{inducedconnectionintermsofnonholonombasis1} (see the subsection 4.2.1)\footnote{In
particular this implies that this is symmetric connection, i.e.
\begin{equation}\label{commutatorofvectorfields}
    \nabla\f\e-\nabla_\e\f-[\f,\e]=a(\f)\f+a(\e)\e-[\f,\e]=0\,.
\end{equation}
}.

\m
{\it Second Quadratic form}
  Let $A(\X,\Y)$ be second quadratic form. Then according to
  to \eqref{secondquadraticform1} and derivation formulae \eqref{derivationformulae1} we have
               $$
  A(\e,\e)=\langle\p_\e\e,\n\rangle=\langle d\e(\e),\n\rangle=\langle a(\e)\f+b(\e)\n,\n\rangle=b(\e)\,,
               $$
         $$
   A(\f,\e)=\langle\p_\f\e,\n\rangle=\langle d\e(\f),\n\rangle=\langle a(\f)\f+b(\f)\n,\n\rangle=b(\f)\,,
         $$
         $$
     A(\e,\f)=\langle\p_\e\f,\n\rangle=\langle d\f(\e),\n\rangle=\langle -a(\e)\f+c(\e)\n,\n\rangle=c(\e)\,,
           $$
           $$
           A(\f,\f)=\langle\p_\f\f,\n\rangle=\langle d\f(\f),\n\rangle=\langle -a(\f)\f+c(\f)\n,\n\rangle=c(\f)\,,
           $$
     The matrix of the second quadratic form in the basis $\{\e,\f\}$ is
     \begin{equation}\label{inducedquadrforminnonholonombasis}
        A=
        \begin{pmatrix}
        A(\e,\e)& A(\f,\e)\cr
        A(\e,\f)& A(\f,\f)\cr
        \end{pmatrix}
        =
        \begin{pmatrix}
        b(\e)&  b(\f)\cr
        c(\e)& c(\f)\cr
        \end{pmatrix}
     \end{equation}
 This is symmetrical matrix (see the subsection 4.3.2):
            \begin{equation}\label{symmericinnonholonom}
 A(\f,\e)=b(\f)=A(\e,\f)=c(\e)\,.
            \end{equation}

\m
{\it Weingarten operator}

Let $S$ be Weingarten operator: $S\X=-\p_\X\n$ (see the subsection 4.3.1). Then
it follows from derivation formulae that
                  $$
     S\X=-\p_\X\n=-d\n(\X)=-\left(-b(X)\e-c(\X)\f\right)=b(\X)\e+c(\X)f
                  $$
In particular
 \begin{equation*}\label{inducedweingartenoperator}
    S(\e)=b(\e)\e+c(\e)\f\,, S(\f)=b(\f)\e+c(\f)\f
\end{equation*}
and the matrix of the Weingarten operator in the basis $\{\e,\f\}$ is
\begin{equation}\label{inducedweingartenoperator}
S=
\begin{pmatrix}
b(\e)  &c(\e)\cr
b(\f)  &c(\f)\cr
\end{pmatrix}
\end{equation}

According to the condition \eqref{symmericinnonholonom}  the matrix $S$ is symmetrical.

The relations $A=GS, S=G^{-1}A$ for Weingarten operator, Riemannian metric and second quadratic form
are evidently obeyed for matrices of these operators in the basis ${\e,\f}$ where $G=1$, $A=S$.

\subsubsection{Gaussian and mean curvature in terms of derivation formulae}
Now we are equipped to express Gaussian and mean curvatures (see the subsection 4.3.3) in terms of derivation formulae.
  Using \eqref{inducedweingartenoperator} we have for Gaussian curvature
           \begin{equation}\label{gaussiancurvintermsofderformulae}
            K=\det S=b(\e)c(\f)-c(\e)b(\f)=(b\wedge c)(\e,\f)
           \end{equation}
 and for mean curvature
   \begin{equation}\label{meancurvintermsofderformulae}
            H={\rm Tr\,} S=b(\e)+c(\f)
           \end{equation}
 What next? We will  study in more detail formula \eqref{gaussiancurvintermsofderformulae} later.

 Now consider some examples of calculation derivation formulae, Weingarten operator, e.t..c. for
 some examples using derivation formulae.

 \subsubsection {Examples of calculations of derivation formulae for cylinder, cone and sphere }

 In the subsection 4.3.4 we calculated Weingarten oeprator, second quadratic form and curvatures
 for cylinder, cone and sphere.
 Now we do the same but in terms of derivation formulae.
\m

 {\it Cylinder}

\m



We have to define three vector fields
${\e,\f,\n}$ on the points of the cylinder $\r(h,\varphi)\colon\quad
  \begin{cases}
  x=R\cos\varphi\\
  y=R\sin\varphi\\
  z=h\\
  \end{cases}$
such that they form an orthonormal basis at any point, so that the vectors $\e,\f$ are tangent to the surface
   and  the vector $\n$ is orthogonal to the surface.

 We already calculated coordinate vector fields $\r_h,\r_\varphi$ and normal unit vector field
$\n$ (see \eqref{surface11} ):
 \begin{equation}\label{surface115}
    \r_h=
  \begin{pmatrix}
        0\\
        0\\
        1\\
   \end{pmatrix}\,,
\quad
  \r_\varphi=\begin{pmatrix}
        -R\sin\varphi\\
        R\cos\varphi\\
          0\\
   \end{pmatrix}\,,\quad
      \n=\begin{pmatrix}
        \cos \varphi\cr
        \sin\varphi\cr
        0\cr
   \end{pmatrix}\,.
\end{equation}

Vectors $\r_h,\r_\varphi$ and $\n$ are orthogonal to each other but not all of them have unit length. One can choose
           \begin{equation}\label{nonholonombasisfor cylinder}
        \e=\r_h=  \begin{pmatrix}
        0\cr
        0\cr
        1\cr
   \end{pmatrix},
   \f={\r_\varphi\over R}=\begin{pmatrix}
        -\sin\varphi\cr
        \cos\varphi\cr
          0\cr
   \end{pmatrix},\,\n=\begin{pmatrix}
        \cos \varphi\cr
        \sin\varphi\cr
        0\cr
   \end{pmatrix}
           \end{equation}
   These vectors form an orthonormal basis and $\e,\f$  form an orthonormal basis in tangent space.

   Derive for this basis derivation formulae \eqref{derivationformulae1}. For vector fields $\e,\f,\n$
   in \eqref{nonholonombasisfor cylinder} we have
                   $$
             d\e=0, d\f=d\begin{pmatrix}
        -\sin\varphi\cr
        \cos\varphi\cr
          0\cr
   \end{pmatrix}=\begin{pmatrix}
        -\cos\varphi\cr
        -\sin\varphi\cr
          0\cr
   \end{pmatrix}d\varphi=-\n d\varphi,
          $$
          $$
   d\n=d\begin{pmatrix}
        \cos \varphi\cr
        \sin\varphi\cr
        0\cr
   \end{pmatrix}=
   \begin{pmatrix}
        -\sin\varphi \cr
        \cos\varphi\cr
        0\cr
   \end{pmatrix}=\f d\varphi,
                   $$
   i.e.
   \begin{equation}\label{derivationformulaefor cylinder}
                    d\begin{pmatrix}
                    \e\cr\f\cr\n\cr
                    \end{pmatrix}=
                    \begin{pmatrix}
                    0&a&b\cr -a&0&c\cr -b&-c&0\cr
                    \end{pmatrix}
                 \begin{pmatrix}
                    \e\cr\f\cr\n\cr
                    \end{pmatrix}=
                     \begin{pmatrix}
                    0&0&0\cr 0&0&-d\varphi\cr 0&d\varphi&0\cr
                    \end{pmatrix}
                 \begin{pmatrix}
                    \e\cr\f\cr\n\cr
                    \end{pmatrix}\,,
                   \end{equation}
  i.e. in derivation formulae $a=b=0$, $c=-d\varphi$.

  The matrix of Weingarten operator in the basis  $\{\e,\f\}$ is
             $$
     S=
\begin{pmatrix}
b(\e)  &c(\e)\cr
b(\f)  &c(\f)\cr
\end{pmatrix}=
\begin{pmatrix}
0  &-d\varphi(\e)\cr
0  &-d\varphi(\f)\cr
\end{pmatrix}=
\begin{pmatrix}
0  &0\cr
0  &-{1\over R}\cr
\end{pmatrix}
       $$
    According to  \eqref{gaussiancurvintermsofderformulae}  and\eqref{meancurvintermsofderformulae}
                Gaussian curvature $K=b(\e)c(\f)-b(\e)c(\f)=0$ and
                mean curvature
                 $$
    H=b(\e)+c(\f)=-d\varphi(\f)=-d\varphi\left({\r_\varphi\over R}\right)=-{1\over R}
                 $$
  (Compare with calculations in the subsection 4.3.4)


\bigskip

{\it Cone}


For  cone:
          $$
  \r(h,\varphi)\colon\quad
  \begin{cases}
  x=kh\cos\varphi\\
  y=kh\sin\varphi\\
  z=h\\
  \end{cases}\,   \,,
   $$
        $$
        \r_h=
  \begin{pmatrix}
        k\cos\varphi\cr
        k\sin\varphi\cr
        1\cr
   \end{pmatrix},\quad
  \r_\varphi=\begin{pmatrix}
        -kh\sin\varphi\cr
        kh\cos\varphi\cr
          0\cr
   \end{pmatrix}\,,\quad
   \n=
   {1\over {\sqrt {1+k^2}}}
      \begin{pmatrix}
        \cos \varphi\cr
        \sin\varphi\cr
        -k\cr
   \end{pmatrix}
            $$
Tangent vectors  $\r_h,\r_\varphi$ are orthogonal to each other.
The length of the vector $\r_h$ equals to $\sqrt {1+k^2}$ and  the length of the vector $\r_\varphi$
equals to $kh$.
Hence we can choose orthonormal basis $\{\e,\f,\n\}$ such that vectors $\e,\f$ are unit vectors in the
directions of the vectors  $\r_h,\r_\varphi$:
               $$
               \e= {\r_h\over \sqrt {1+k^2}}=
               {1\over \sqrt {1+k^2}}
  \begin{pmatrix}
        k\cos\varphi\cr
        k\sin\varphi\cr
        1\cr
   \end{pmatrix},\,
  \f={\r_\varphi\over hk}=\begin{pmatrix}
        -\sin\varphi\cr
        \cos\varphi\cr
          0\cr
   \end{pmatrix},\,
   \n=
   {1\over {\sqrt {1+k^2}}}
      \begin{pmatrix}
        \cos \varphi\cr
        \sin\varphi\cr
        -k\cr
   \end{pmatrix}
               $$
Calculate $d\e,d\f$ and $d\n$:
           $$
      d\e=d\begin{pmatrix}
        k\cos\varphi\cr
        k\sin\varphi\cr
        1\cr
   \end{pmatrix}={kd\varphi\over \sqrt{1+k^2}}\begin{pmatrix}
        -\sin\varphi\cr
        \cos\varphi\cr
        0\cr
   \end{pmatrix}={ kd\varphi\over \sqrt{1+k^2}}\f,
           $$
           $$
   d\f=d\begin{pmatrix}
        -\sin\varphi\cr
        \cos\varphi\cr
          0\cr
   \end{pmatrix}=
     \begin{pmatrix}
        -\cos\varphi\cr
        -\sin\varphi\cr
          0\cr
   \end{pmatrix}d\varphi=
        $$
        $$
   {-k\over 1+k^2}\begin{pmatrix}
        k\cos\varphi\cr
        k\sin\varphi\cr
        1\cr
   \end{pmatrix}d\varphi-{d\varphi\over 1+k^2}
             \begin{pmatrix}
        \cos \varphi\cr
        \sin\varphi\cr
        -k\cr
   \end{pmatrix}={-kd\varphi\over \sqrt{1+k^2}}\e-{d\varphi\over \sqrt {1+k^2}}\n\,,
     $$
and
       $$
   d\n={1\over \sqrt {1+k^2}}d\begin{pmatrix}
        \cos \varphi\cr
        \sin\varphi\cr
        -k\cr
   \end{pmatrix}=
   {d\varphi\over \sqrt {1+k^2}}
   \begin{pmatrix}
        -\sin \varphi\cr
        \cos\varphi\cr
        0\cr
   \end{pmatrix}\,.
       $$
We come to
   \begin{equation}\label{derivationformulaefor cone}
                    d\begin{pmatrix}
                    \e\cr\f\cr\n\cr
                    \end{pmatrix}=
                    \begin{pmatrix}
                    0&a&b\cr -a&0&c\cr -b&-c&0\cr
                    \end{pmatrix}
                 \begin{pmatrix}
                    \e\cr\f\cr\n\cr
                    \end{pmatrix}=
                     \begin{pmatrix}
                    0&{kd\varphi\over \sqrt {1+k^2}}&0\cr
                     -{kd\varphi\over \sqrt {1+k^2}}&0&{-d\varphi\over \sqrt {1+k^2}}\cr
                      0&{d\varphi\over \sqrt {1+k^2}}&0\cr
                    \end{pmatrix}
                 \begin{pmatrix}
                    \e\cr\f\cr\n\cr
                    \end{pmatrix}\,,
                   \end{equation}
  i.e. in derivation formulae for $1$-forms $a={kd\varphi\over \sqrt {1+k^2}}$, $b=0$ and
  and $c=-{-d\varphi\over \sqrt {1+k^2}}$.

  The matrix of Weingarten operator in the basis  $\{\e,\f\}$ is
             $$
     S=
\begin{pmatrix}
b(\e)  &c(\e)\cr
b(\f)  &c(\f)\cr
\end{pmatrix}=
S=
\begin{pmatrix}
0  &{-d\varphi (\e)\over \sqrt {1+k^2}}\cr
0  &{-d\varphi (\f)\over \sqrt {1+k^2}}\cr
\end{pmatrix}=
\begin{pmatrix}
0  &0\cr
0  &{-1\over kh \sqrt {1+k^2}}\cr
\end{pmatrix}\,.
       $$
since $d\varphi(\f)=d\varphi\left({\r_\varphi\over kh}\right)={1\over kh}d\varphi(\p_\varphi)={1\over kh}$.


According to \eqref{gaussiancurvintermsofderformulae}, \eqref{meancurvintermsofderformulae}
                Gaussian curvature  $$
                K=b(\e)c(\f)-b(\e)c(\f)=0
                $$
and
                mean curvature
                 $$
    H=b(\e)+c(\f)=-d\varphi(\f)=-d\varphi\left({\r_\varphi\over R}\right)=-{1\over R}
                 $$
  (Compare with calculations in the subsection 4.3.4)



\bigskip




{\it Sphere}


\medskip


For sphere
\begin{equation}\label{surfacesphere11}
  \r(\theta,\varphi)\colon\quad
  \begin{cases}
  x=R\sin\theta\cos\varphi\\
  y=R\sin\theta\sin\varphi\\
  z=R\cos\theta\\
  \end{cases}
\end{equation}
    $$
  \r_\theta(\theta,\varphi)={\p \r\over \p \theta}=
  \begin{pmatrix}
        R\cos\theta\cos\varphi\cr
        R\cos\theta\sin\varphi\cr
        -R\sin\theta\cr
   \end{pmatrix}\,,
\quad
  \r_\varphi(\theta,\varphi)={\p \r\over \p \varphi}=
         \begin{pmatrix}
        -R\sin\theta\sin\varphi\cr
        R\sin\theta\cos\varphi\cr
          0\cr
   \end{pmatrix},
     $$
     $$
   \n(\theta,\varphi)={\r\over R}=
   \begin{pmatrix}
    \sin\theta\cos\varphi\cr
     \sin\theta\sin\varphi\cr
      \cos\theta\cr
   \end{pmatrix}.
  $$
Tangent vectors  $\r_\theta,\r_\varphi$ are orthogonal to each other.
The length of the vector $\r_\theta$ equals to $R$ and the
length of the vector $\r_\varphi$ equals to $R\sin\theta$.
Hence we can choose orthonormal basis $\{\e,\f,\n\}$ such that vectors $\e,\f$ are unit vectors in the
directions of the vectors  $\r_\theta,\r_\varphi$:
         $$
    \e(\theta,\varphi)={\r_\theta\over R}=
         \begin{pmatrix}
        \cos\theta\cos\varphi\cr
        \cos\theta\sin\varphi\cr
        -\sin\theta\cr
   \end{pmatrix},\,
   \f(\theta,\varphi)={\r_\varphi\over R\sin\theta}=
        \begin{pmatrix}
        -\sin\varphi\cr
        \cos\varphi\cr
          0\cr
   \end{pmatrix},\,
     \n(\theta,\varphi)={\r\over R}=
   \begin{pmatrix}
    \sin\theta\cos\varphi\cr
     \sin\theta\sin\varphi\cr
      \cos\theta\cr
   \end{pmatrix}.
         $$
Calculate $d\e, d\f$ and $d\n$:
            $$
        d\e=d
        \begin{pmatrix}
        \cos\theta\cos\varphi\cr
        \cos\theta\sin\varphi\cr
        -\sin\theta\cr
   \end{pmatrix}=
           $$
           $$
\begin{pmatrix}
        -\cos\theta\sin\varphi\cr
        \cos\theta\cos\varphi\cr
           0\cr
   \end{pmatrix}d\varphi-\begin{pmatrix}
        \cos\theta\cos\varphi\cr
        \cos\theta\sin\varphi\cr
        -\sin\theta\cr
   \end{pmatrix}=\cos\theta d\varphi \f-d\theta \n,
            $$
             $$
            d\f=d
            \begin{pmatrix}
        -\sin\varphi\cr
        \cos\varphi\cr
          0\cr
   \end{pmatrix}=
              -
   \begin{pmatrix}
        \cos\varphi\cr
        \sin\varphi\cr
          0\cr
   \end{pmatrix}d\varphi=
            $$
            $$
       -\cos\theta d\varphi \begin{pmatrix}
        \cos\theta\cos\varphi\cr
        \cos\theta\sin\varphi\cr
        -\sin\theta\cr
   \end{pmatrix}-\sin\theta d\varphi
   \begin{pmatrix}
    \sin\theta\cos\varphi\cr
     \sin\theta\sin\varphi\cr
      \cos\theta\cr
   \end{pmatrix}=
   -\cos\theta d\varphi \e-\sin\theta d\varphi \n\,,
            $$
             $$
             d\n=
   d\begin{pmatrix}
    \sin\theta\cos\varphi\cr
     \sin\theta\sin\varphi\cr
      \cos\theta\cr
   \end{pmatrix}=
   \begin{pmatrix}
    \cos\theta\cos\varphi\cr
     \cos\theta\sin\varphi\cr
      -\sin\theta\cr
   \end{pmatrix}d\theta+
   \begin{pmatrix}
    -\sin\theta\sin\varphi\cr
     \sin\theta\cos\varphi\cr
        0\cr
   \end{pmatrix}d\varphi
        $$
        $$
   =d\theta \e+\sin\theta d\varphi \f\,.
        $$
   i.e.
   \begin{equation}\label{derivationformulaefor cylinder}
                    d\begin{pmatrix}
                    \e\cr\f\cr\n\cr
                    \end{pmatrix}=
                    \begin{pmatrix}
                    0&a&b\cr -a&0&c\cr -b&-c&0\cr
                    \end{pmatrix}
                 \begin{pmatrix}
                    \e\cr\f\cr\n\cr
                    \end{pmatrix}=
                     \begin{pmatrix}
                    0&\cos\theta d\varphi& -d\theta\cr
                    -\cos\theta d\varphi&0&-\sin\theta d\varphi\cr
                     d\theta&\sin\theta d\varphi&0\cr
                    \end{pmatrix}
                 \begin{pmatrix}
                    \e\cr\f\cr\n\cr
                    \end{pmatrix}\,,
                   \end{equation}
  i.e. in derivation formulae $a=\cos\theta d\varphi$,
  $b=-d\theta$, $c=-\sin\theta d\varphi$.

  The matrix of Weingarten operator in the basis  $\{\e,\f\}$ is
             $$
     S=
\begin{pmatrix}
b(\e)  &c(\e)\cr
b(\f)  &c(\f)\cr
\end{pmatrix}=
\begin{pmatrix}
-d\theta (\e) &-\sin\theta d\varphi (\e)\cr
-d\theta (\f)  &-\sin\theta d\varphi(\f)\cr
\end{pmatrix}=
\begin{pmatrix}
{-1\over R}  &0\cr
0  &-{1\over R}\cr
\end{pmatrix}
       $$
       since $d\theta(\e)=d\theta\left({\p_\theta\over R}\right)={1\over R}d\theta(\p_\theta)={1\over R}$,
       $\,d\varphi(\e)=d\varphi\left({\p_\theta\over R}\right)={1\over R}d\varphi(\p_\theta)=0$.

    According to  \eqref{gaussiancurvintermsofderformulae}  and\eqref{meancurvintermsofderformulae}
                Gaussian curvature $$
                K=b(\e)c(\f)-b(\e)c(\f)={1\over R^2}
                                    $$
                                    and
                mean curvature
                 $$
    H=b(\e)+c(\f)=-{2\over R}
                 $$
    Notice that for calculation of Weingarten operator and curvatures we used only
    $1$-forms $b$ and $c$, i.e. the derivation equation for $d\n$, ($d\n=d\theta \e+\sin\theta d\varphi \f$).
  (Compare with calculations in the subsection 4.3.4)





 Mean curvature is define up to a sign. If we change $\n\to-\n$ mean curvature $H\to {1\over R}$ and Gaussian curvature
will not change.


We see that for the sphere  Gaussian curvature is not equal to zero, whilst for cylinder and cone Gaussian curvature
equals to zero.


\subsubsection { Proof of the Theorem of parallel transport along closed curve.}

We are ready now to prove the Theorem.  Recall that the Theorem states following:

If $C$ is a closed curve on a surface $M$ such that $C$ is a boundary of a compact oriented domain $D\subset M$,
then during the  parallel transport of an arbitrary tangent vector  along the closed curve $C$
the vector rotates through the angle
\begin{equation}\label{theoremofrotationonangle4}
\Delta\Phi=\angle\left({\X, \R_C\X}\right)=\int_D K d\sigma\,,
             \end{equation}
where $K$ is the Gaussian curvature and $d\sigma=\sqrt {\det g}dudv$ is the area element induced by the
Riemannian metric on the surface $M$, i.e.  $d\sigma=\sqrt {\det g}dudv$.

(see \eqref{theoremofrotationonangle}.


Recall that for derivation formulae
\eqref{derivationformulae1} we obtained structure equations
                  \begin{equation}\label{firststructureformula4}
               \begin{matrix}
                da+b\wedge c=0\cr
                 db+c\wedge a=0\cr
                 dc+a\wedge a=0\cr
                 \end{matrix}
               \end{equation}

We need to use only one of these equations, the equation
             \begin{equation}\label{gausscondition}
                da+b\wedge c=0\,.
             \end{equation}
This condition sometimes is called {\it Gau\ss \,\,condition}.



Let as always  $\{\e,\f,\n\}$ be an orthonormal basis in $T_{\pt}\E^3$ at every point of surface  $\pt\in M$
such that $\{\e,\f\}$ is an orthonormal basis in $T_{\pt}\E^3$ at every point of surface  $\pt\in M$.
Then the  Gau\ss\, condition \eqref{gausscondition} and equation \eqref{gaussiancurvintermsofderformulae} mean
 that  for Gaussian curvature on the surface $M$ can be expressed through the $2$-form $da$ and base vectors $\{\e,\f\}$:
              \begin{equation}\label{gaussiancurvatureintermsof form a}
                K=b\wedge c(\e,\f)=-da(\e,\f)
              \end{equation}
  We use this formula to prove the Theorem.



             Now calculate the parallel transport of an arbitrary tangent vector over the closed curve $C$
             on the surface $M$.



 Let $\r=\r(u,v)=\r(u\a)$ ($\a=1,2$, $(u,v)=(u^1,v^1)$) be an equation of the surface $M$.

Let $u^\a=u^\a(t)$ ($\a=1,2$) be the equation of the curve $C$.
Let $\X(t)$ be the parallel transport of vector field along the closed curve $C$,
i.e. $\X(t)$ is tangent to the surface $M$ at the point $u(t)$ of the curve $C$ and
vector field $\X(t)$ is covariantly constant along the curve:
      $$
    {\nabla \X(t)\over dt}=0
      $$
     To write this equation in components we usually  expanded the vector field
in the coordinate basis $\{\r_u=\p_u,\r_v=\p_v\}$ and used Christoffel symbols of the connection
  $\Gamma^\a_{\beta\gamma}\colon \nabla_\beta\p_\gamma=\Gamma^\a_{\beta\gamma}\p_\a$.

  Now we will do it in different way: {\it instead coordinate basis $\{\r_u=\p_u,\r_v=\p_v\}$ we will use
  the basis $\{\e,\f\}$.}   In the subsection 3.4.4 we obtained that the connection $\nabla$ has the following appearance
  in this basis
  \begin{equation}\label{indconforrotation}
    \nabla_\v\e=a(\v)\f,\,, \nabla_\v\f=-a(\v)\e
  \end{equation}
  (see the equations \eqref{inducedconnectionintermsofnonholonombasis1} and \eqref{inducedconnectionintermsofnonholonombasis2})

Let        $$
\X=\X(u(t))=X^1(t)\e(u(t))+X^2(t)\f(u(t))
           $$
Lbe an expansion of tangent vector field $\X(t)$ over basis $\{\e,\f\}$.
Let $\v$ be velocity vector of the curve $C$.
  Then the equation of parallel transport ${\nabla \X(t)\over dt}=0$
                 will have the following appearance:
                      $$
              {\nabla \X(t)\over dt}=0=\nabla_\v \left(X^1(t)\e(u(t))+X^2(t)\f(u(t))\right)=
                      $$
                      $$
                      {dX^1(t)\over dt}\e(u(t))+X^1(t)\nabla_\v\e(u(t))+
                      {dX^2(t)\over dt}\f(u(t))+X^2(t)\nabla_\v\f(u(t))=
                      $$
                      $$
                      {dX^1(t)\over dt}\e(u(t))+X^1(t)a(\v)\f(u(t))+
                      {dX^2(t)\over dt}\f(u(t))-X^2(t)a(\v)\e(u(t))=
                      $$
              $$
          \left({dX^1(t)\over dt}-X^2(t)a(\v)\right)\e(u(t))+
           \left({dX^2(t)\over dt}+X^1(t)a(\v)\right)\f(u(t))=0.
              $$
Thus we come to equation:
              $$
             \begin{cases}
             \dot X^1(t)-a(\v(t))X^2=0\cr
             \dot X^2(t)+a(\v(t))X^1=0\cr
             \end{cases}
              $$
There are many ways to solve this equation. It is very convenient to consider complex variable
             $$
           Z(t)=X^1(t)+iX^2(t)
             $$
We see that
             $$
         \dot Z(t)=\dot X^1(t)+i\dot X^2(t)=a(\v(t))X^2-ia(\v(t)X^1=-ia(\v)Z(t),
                 $$
i.e.
   \begin{equation}\label{complexrotation}
    {dZ(t)\over dt}=-ia(\v(t))Z(t)
   \end{equation}
The solution of this equation is:
           \begin{equation}\label{complexrotation}
      Z(t)=Z(0)e^{-i\int_0^t a(\v(\tau))d\tau}
   \end{equation}
   Calculate $\int_0^{t_1} a(\v(\tau))d\tau$ for closed curve $u(0)=u(t_1)$. Due to Stokes Theorem:
            $$
            \int_0^{t_1} a(\v(t))dt=\int_C a=\int_D da
            $$
   Hence using Gauss condition \eqref{gausscondition} we see that
      $$
      \int_0^{t_1} a(\v(t))dt=\int_C a=\int_D da=-\int_D b\wedge c
      $$

{\bf Claim}
            \begin{equation}\label{claimoncurvature}
            \int_D b\wedge c=-\int_D da=\int K d\sigma\,.
            \end{equation}

Theorem follows from this claim:
           \begin{equation}\label{complexrotation2}
      Z(t_1)=Z(0)e^{-i\int_C a}=Z(0)e^{i\int_D b\wedge C}
   \end{equation}
Denote the integral ${i\int_D b\wedge C}$ by $\Delta \Phi\colon \Delta \Phi={i\int_D b\wedge C}$. We have
           \begin{equation}\label{complexrotation2}
      Z(t_1)=X^1(t_1)+iX^2(t_1)=\left(X^1(0)+iX^2(0)\right)e^{i\Delta\Phi}=
   \end{equation}

\medskip

 It remains to prove the claim.  The induced volume form $d\sigma$ is $2$-form. Its value
 on two orthogonal unit vector $\e,\f$ equals to $1$:
              \begin{equation}\label{valueonorthonormalvectors}
                d\sigma (\e,\f)=1
              \end{equation}
(In coordinates $u,v$ volume form $d\sigma=\sqrt{\det g}du\wedge dv$).

The value of the form $b\wedge c$ on vectors $\{\e,\f\}$ equals to Gaussian curvature according to \eqref{gaussiancurvatureintermsof form a}.
We see that
             $$
        b\wedge c(\e,\f)=-da(\e,\f)=K d\sigma (\e,\f)
             $$
Hence $2$-forms $b\wedge c$, $-da$ and volume form $d\sigma$ coincide. Thus we prove \eqref{claimoncurvature}.


\section {Curvtature tensor}

\subsection {Definition}

Let $\X,\Y,\Z$ be an arbitrary vector fields on the manifold equipped with affine connection  $\nabla$.

Consider the following operation which assings to the vector fields $\X,\Y$ and $\Z$ the new vector field.
\begin{equation}\label{operationdefinig the tensor}
   {\cal R}(\X,\Y)\Z=
    \left(
    \nabla_\X\nabla_\Y-\nabla_\Y\nabla_\X-
   \nabla_{[\X,\Y]}
    \right)\Z
\end{equation}
This operation is obviously linear over the scalar coefficients and it is $C^{\infty}(M)$
with respect to vector fields $\X,\Y\,\Z$, i.e. for an arbitrary functions
$f,g,h$
        \begin{equation}\label{propertiesoflienarity}
            {\cal R}(f\X,g\Y)(h\Z)=fgh{\cal R}(\X,\Y)\Z,
        \end{equation}
        i.e. it defines
the tensor field of the type $\begin{pmatrix}1\cr 3\end{pmatrix}$: If $\X=X^i\p_i, \X=X^i\p_i,\X=X^i\p_i$
then according to \eqref{propertiesoflienarity}
           $$
        {\cal R}(\X,\Y)\Z={\cal R}(X^m\p_m,Y^n\p_n)(Z^r\p_r)=Z^rR^i_{rmn}X^mY^n
           $$
where we denote by $R^i_{rmn}$ the components of the tensor $\cal R$ in the coordinate basis ${\p_i}$
\begin{equation}\label{componentsofcurvaturetensor}
    R^i_{rmn}\p_i={\cal R}(\p_m,\p_n)\p_r
\end{equation}
Express components of the curvature tensor in terms of Christoffel symbols of the connection.
If $\nabla_m\p_n=\Gamma_{mn}^r\p_r$ then according to the \eqref{operationdefinig the tensor} we have:
                $$
        R^i_{rmn}\p_i={\cal R}(\p_m,\p_n)\p_r=\nabla_{\p_m}\nabla_{\p_n}\p_r-\nabla_{\p_n}\nabla_{\p_m}\p_r=
                    $$
                    $$
              \nabla_{\p_m}\left(\Gamma_{nr}^p\p_p\right)-\nabla_{\p_n}\left(\Gamma_{mr}^p\p_p\right)=
                $$
                \begin{equation}\label{curvatureincomponents}
                \p_m\Gamma_{nr}^i+\Gamma_{mp}^i\Gamma_{nr}^p-\left(m\leftrightarrow n\right)
                \end{equation}

 The proof of the property \eqref{propertiesoflienarity} can be given just  by straightforward calculations:
     Consider e.g. the case $f=g=1$, then

        $$
         {\cal R}(\X,\Y)(h\Z)=
    \nabla_{\X}\nabla_\Y(h\Z)-\nabla_{\Y}\nabla_\X(h\Z)-
   \nabla_{[\X,\Y]}(h\Z)=
        $$
        $$
     \nabla_\X\left(\p_\Y h\Z+h\nabla_\Y\Z\right)-\nabla_\Y\left(\p_\X h\Z+h\nabla_\X\Z\right)-
     \p_{[\X,\Y]}h \Z-h\nabla_{[\X,\Y]}\Z=
        $$
        $$
    \p_\X\p_\Y h\Z+\p_\Y h\nabla_\X\Z+
    \p_\X h\nabla_\Y\Z+
    h\nabla_\X\nabla_\Y \Z-
        $$
        $$
        \p_\Y\p_\X h\Z-\p_\X h\nabla_\Y\Z-
    \p_\Y h\nabla_\X\Z+
    h\nabla_\Y\nabla_\X \Z-
        $$
        $$
      \p_{[\X,\Y]}h \Z-h\nabla_{[\X,\Y]}\Z=
        $$
        $$
    h\left[\nabla_{\X}\nabla_\Y\Z-\nabla_{\Y}\nabla_\X\Z)-
   \nabla_{[\X,\Y]}\Z\right]+\left[\p_\X\p_\Y h-\p_\Y\p_\X h-\p_{[\X,\Y]h}\right]\Z=
        $$
        $$
h\nabla_{\X}\nabla_\Y\Z-\nabla_{\Y}\nabla_\X\Z)-
   \nabla_{[\X,\Y]}\Z=h{\cal R}(\X,\Y)\Z\,,
        $$
   since $\p_\X\p_\Y h-\p_\Y\p_\X h-\p_{[\X,\Y]}h=0$.




\subsection {Curvature of surfaces in $\E^3$.. {\it Theorema Egregium} egain}

  Express Riemannian curvature of surfaces in $\E^3$ in terms of derivation formulae \eqref{derivationformulae1}.

   Consider derivation formulae for the orthonormal basis $\{\e,\f,\n,\}$ adjusted to the surface $M$:
                      \begin{equation}\label{derivationformulae1}
                    d\begin{pmatrix}
                    \e\cr\f\cr\n\cr
                    \end{pmatrix}=
                    \begin{pmatrix}
                    0&a&b\cr -a&0&c\cr -b&-c&0\cr
                    \end{pmatrix}
                 \begin{pmatrix}
                    \e\cr\f\cr\n\cr
                    \end{pmatrix}\,,
                       \end{equation}

where as usual $\e,\f,\n$ vector fields of unit length which are  orthogonal
to each other and $\n$ is orthogonal to the surface $M$.  As we know the induced
connection on the surface $M$ is defined by the formulae \eqref{inducedconnectionintermsofnonholonombasis1}
and \eqref{inducedconnectionintermsofnonholonombasis2}:
   \begin{equation}\label{inducedconnectionintermsofnonholonombasis11}
        \nabla_\Y \e=\left(d\e(\Y)\right)_{\rm tangent}=a(\X)\f\,,
        \nabla_\Y \f=\left(d\f(\Y)\right)_{\rm tangent}=-a(\X)\e\,,
        \end{equation}

 According to the definition of curvature calculate
         $$
 R(\e,\f)\e=\nabla_{\e}\nabla_{\f}\e-\nabla_{\f}\nabla_{\e}\e-\nabla_{[\e,\f]}\e\,.
         $$
Note that since the induced connection is symmetrical connection then:
        \begin{equation}\label{commutatorofnonholonombasis}
 \nabla_\e \f-\nabla_\f \e- [\e,\f]=0\,.
\end{equation}
hence due to \eqref{inducedconnectionintermsofnonholonombasis11}

       \begin{equation}\label{commutatorofnonholonombasis}
 [\e,\f]=\nabla_\e \f-\nabla_\f \e=a(\e)\e+a(\f)\f
\end{equation}
Thus we see that  $R(\e,\f)\e=$
       $$
 \nabla_{\e}\nabla_{\f}\e-\nabla_{\f}\nabla_{\e}\e-\nabla_{[\e,\f]}\e=
       \nabla_\e\left(a(\f)\f\right)-\nabla_\f\left(a(\e)\f\right)-\nabla_{a(\e)\e+a(\f)\f}\e=
       $$
       $$
  \p_{\e}a(\f)\f+a(\f)\nabla_\e\f-\p_{\f}a(\e)\f-a(\e)\nabla_\f\f-a(\e)\nabla_\e\e-a(\f)\nabla_\f\e=
       $$
       $$
  \p_{\e}a(\f)\f-a(\f)a(\e)\e-\p_{\f}a(\e)\f+a(\e)a(\f)\e-a(\e)a(\e)\f-a(\f)a(\f)\f=
       $$
       $$
       \left[\p_{\e}a(\f)\f-\p_{\f}a(\e)\f-a\left((\e)\e-a(\f)\f\right)\right]\f=
       da(\e,\f)\f\,.
       $$
  Recall that we established in \ref{gaussiancurvatureintermsof form a} that for Gaussian curvature $K$
          $$
       K=b\wedge c(\e,\f)=-da(\e,\f)
          $$
       Hence we come to the relation:
       \begin{equation}\label{relationbetweengausscurvatureand riemtensor}
        R(\e,\f)\e=da(\e,\f)=-K\f\,.
       \end{equation}
       This means that
       $$
         R^2_{112}=-K
       $$
(in the basis ${\e,\f}$), i.e.
 the scalar curvature
   $$
R=2R_{1212}=2K
   $$
 We come to fundamental relation which claims that the Gaussian curvature  (the magnitude defined in terms of
 External observer) equals to the scalar curvature, the magnitude defined in terms of Internal observer.
   This gives us the straightforward proof of Theorema Egregium.






\end{document}



