

\magnification=1200 


\baselineskip=14pt
\def\vare {\varepsilon}
\def\A {{\bf A}}
\def\t {\tilde}
\def\a {\alpha}
\def\K {{\bf K}}
\def\N {{\bf N}}
\def\V {{\cal V}}
\def\s {{\sigma}}
\def\S {{\Sigma}}
\def\s {{\sigma}}
\def\p{\partial}
\def\vare{{\varepsilon}}
\def\Q {{\bf Q}}
\def\D {{\cal D}}
\def\G {{\Gamma}}
\def\C {{\bf C}}
\def\M {{\cal M}}
\def\Z {{\bf Z}}
\def\U  {{\cal U}}
\def\H {{\cal H}}
\def\R  {{\bf R}}
\def\S  {{\bf S}}
\def\E  {{\bf E}}
\def\l {\lambda}
\def\degree {{\bf {\rm degree}\,\,}}
\def \finish {${\,\,\vrule height1mm depth2mm width 8pt}$}
\def \m {\medskip}
\def\p {\partial}
\def\r {{\bf r}}
\def\pt {{\bf pt}}
\def\v {{\bf v}}
\def\n {{\bf n}}
\def\t {{\bf t}}
\def\b {{\bf b}}
\def\c {{\bf c }}
\def\e{{\bf e}}
\def\ac {{\bf a}}
\def \X   {{\bf X}}
\def \Y   {{\bf Y}}
\def \x   {{\bf x}}
\def \y   {{\bf y}}
\def \G{{\cal G}}

{\it  I wrote this on 22-nd  June 2017}

\m


\centerline {\bf Hidden hyperbolicity}

{\it In this \'etude we consider one geometrical problem,
which has two manifestations. It seems to be
the standard Euclidean problem,  but it
 possesses the hidden hyperbolicity.
We first explain how the example is arised, then consider
its solution in terms of Euclidean and Hyperbolic Geometry}

\m

\centerline {\it Source}

Consider a realisation of hyperbolic (Lobachevsky) plane
as upper half-plane of Euclidean plane. Let $C$ be a (usual Euclidean)
circle in this half-plane. It is also the hyperbolic circle/
   Let $O$ be a centre of $C$ considered 
as the Eucldean circle, and 
let $A$ be a centre of $C$ considered 
as the hyperbolic  circle.
 The fact that the point $A$ is the hyp[erbolic centre 
means that if two curves $\gamma,\gamma'$ are  
arbitrary hyperbolic geodesics
passing through the point $A$, and 
$L_\gamma,L'_{\gamma'} $ are points
of the intersections of these geodesics 
with circle $C$, 
$L_\gamma=\gamma\times C$,  
$L_{\gamma'}=\gamma'\times C$  
then the (hyperbolic) lengths of these geodesics 
coincide. In particular due to the lemma this implies
that all these geodesics intersect the circle $C$
under the angle $\pi\over 2$.

Now look at this picture from the point of view 
of Euclidean geometry. 
   All gedesics are half-circles
with centres on the absolute, the line $x=0$,
(except the geodesicis which are the vertical lines)
Angles are the same. If geodesic $\gamma$ is represented
by the half-circle with the centre at the point 
$K$, then the segment $KL_\gamma$, radius of this half-circle
has to be tangent to the circle $C$ in the case if this geodesic
intersects the circle $C$ under the angle $\pi\over 2$.
 Thus we come to conclusion
that for the arbitrary  point $K$ on the absolute
the (Euclidean) length of the tangent from the point $K$ to the
circle $C$ is equal to the (Euclidean) length of the segment $KA$.

\bigskip



Based on the considerations above we will
formulate the following problem of Eucldean geometry.

\smallskip

{\sl
  Let $C$ be a circle in the Euclidean 
plane and let $A$, be an arbitrary point on this plane.

 Consider the locus $M_{C,A}$ of the points $K$ such that the 
length of the tangent from $K$ to the circle
$C$ is equal to the length of the segment $KA$:
                     $$
   M_{C,A}=
\{K\colon\quad KA=
\hbox {lenght of the tangent from $K$ to the circle.}\}\,.
     \eqno (1)
                     $$

Find the locus  $M_{C,A}$.
}

\smallskip

This problem looks as standard geometrical question in Euclidean gepmetry.
 Temporary forgetting where this problem comes from, we will
discuss first solution of this problem just in terms of  Eucldean
geometry


\bigskip


{\sl Solution}

Let $A'$ be  a point which is inverse to the point
$A$ with respect to the circle $C$:
Points $O$ (centre of the circle), $A$ and $A"$
belong to the same ray $r_{OA}$ and
             $$
|OA|\cdot |OA'|=1\,.
   \eqno (2)
                 $$ 
(We suppose that  the circle $C$ has unit length.)


Consider the set of circles passing through the points
   $A$ and $A'$. One can see that every such a circle,
                     $$
 \hbox { intersects the circle $C$ under the right angle}
             \eqno(3)
                     $$
 This means that for centre $K$ of such circle, the
tangent to the circle $C$ is the radius of this circle.
   We see that the locus  $M_{C,A}$ $=$ the locus
 of the centres of circles passing through the points 
     $A$ and $A'$  $=$
  the locus of the points which are on the
same (Euclidean) distance from the points $A$ and $A'$.
    This is the line $l$ which is ortogonal to the
$AA'$ and 
passes through th emiddle point $P$ of the segment
  $AA'$ ($P\in AA', |AP|=|PA'|$. 

    So we come to 
                   $$
        M_{C,A}= \{l\colon \quad d(l,A)=d(l,A')\}\,.
                      \eqno (4)
                   $$
It remains to prove just relation (3).
  It can be checked straightforwardly.
  The following beautiful prove implies inversion.

    Let $L$ be an arbitrary circle passing through the points
       $A,A'$.  Let this circle intersects the circle $C$
at the point $M$. Triangle $AMA'$
remains intact under the inversion with respect to
the circle $C$: $A\leftrightarrow A', M\leftrightarrow M$
(see equation (1)).  Hence the inversion transforms the circle $L$ to 
the same circle.
If $\varphi$ is angle of intersection of $L$ with $C$,
then $\pi-\varphi$ is the angle of intersection of inversed
circle with $C$;ince inversion transforms the direction of
 arcs. We have $\varphi=\pi-\varphi$, i.e. $\varphi={\pi\over 2}$. 


\m

\bigskip


 \centerline {\it Meaning in hyperbolic geometry}

The considerations of the first paragraph
show that the fact that 
  $l=M_{C,A}$ has explanations in hyperbolicity.


Indeed the line $l$  divides the plane on
two half-planes. Consider the half-plane 
the  circle $C$ belogs to, as a model
of hyperbolic (Lobachevsky) plane
\footnote{$^*$}
         {
   Recall shrotly what is it. 
One can consider 
Cartesian coordinates $(x,y)$ such that
 the line $l$ is $y=0$ the half-plane 
is $y\geq 0$.
Then hyperbolic plane $H$ can be defined as 
as this half-plane with Riemannian metric
         $
       G={dx^2+dy^2\over y^2}
         $.
The geodesics of this metric 
(lines of hyperbolic plane) 
are vertical lines $x=a$
and upper half-circles with centre on the absolute
   $l$: $\cases{(x-a)^2+y^2=R^2\cr y>0 \cr}$.
The distance  
between two points $A_1=(x_1,y_1)$
and $A_2=(x_2,y_2)$, the length of the geodesic
passing via points $A_1,A_2$
 can be defined alternatively
by cross-ratio of the points
        $$
d(A_1,A_2)=\left|
  \log\left(A_1, A_2,A_0,A_\infty)\right)\right|=
\left|\log\left(
 {z_1-z_0\over z_1-z_\infty}:
 {z_2-z_0\over z_1-z_\infty}\right)\right|\,,
        $$  
where points $A_0,A_\infty$
are points of intersection of the half-circle with absolute, $z_1=x_1+iy_1$,$z_2=x_2+iy_1$.
E.g. for two points $A_1=(0,a_1)$, $A_2=(o,a_2)$
on the vertical line (this is geodesic)
 $$
d(A_1,A_2)=
    \left|\log
  \left|(A_1, A_2,0,\infty)\right)\right|=
            \left|
             \log
             \left(
 {ia_1-0\over ia_1-\infty}:
 {ia_2-0\over ia_1-\infty}\right)\right|=
            \left|
             \log
             \left(
 {a_1\over a_2}\right)\right|\,.
        $$ 
}.
The circle $C$ will be the circle in the hyperbolic plane also.

One of the points $A$ or $A'$ is in the circle $C$. 
WLOG suppose that this is a point $A$. 

 One can see traightfoewardly that the point $A$ is the centre
of the hyperbolic circle  $C$
\footnote{$^2$}
           {
   It is convenient to consider coordinates
$(x,y)$ such that the line $l$ is defined by $x=0$, and the vertical
ray $AP$ is $y=0$.
  Let point $a$ be on the distance $a$ from the Euclidean centre
of the circle $C$. Then the Euclidean centre of the circle has coordinates 
$O=\left(0,{1\over 2}\left(a+{1\over a}\right)\right)$,
and respectively 
                      $$
P=(0,0)\,,\quad
 A=\left(0,{1\over 2}\left(a+{1\over a}\right)-a\right)=
        \left(0,{1\over 2}\left({1\over a}-a\right)\right)\,.
                      $$
Recall that $P$ is the point of intersection of the vertical 
ray passing throught the point $A$
with absolute $l$.
Let  $N_{1.2}$ be points of the intersection of
the ray $AP$, then
                 $$
N_{1,2}=\left(0,{1\over 2}\left(a+{1\over a}\right)\pm 1\right)=
       \left(0, {1\over 2}\left(\sqrt {1\over a}\pm \sqrt {a}\right)^2\right)
                 $$
We see that  $|PA|$ is geometrical mean of $PN_1$, $PN_2$:
                             $$
|PN_1|\cdot |PN_2|={1\over 2}
      \left(\sqrt {1\over a}+\sqrt {a}\right)^2
   {1\over 2}\left(\sqrt {1\over a}- \sqrt {a}\right)^2=
   {1\over 4}\left({1\over a}-  {a}\right)^2=
   |PA|^2\,,
                             $$ 
i.e. the hyperbolical lengths $|AN_1|$ and $|AN_2|$ coincide.
}.
This immediately implies that the points of $l$ belong to the locus
   $M_{C,A}$.  Indeed  due to the lemma all the geodesics starting
at the point $A$, the centre of the circle, interesect 
the circle $C$ under the right angle, i.e. the points of $l$
belong to this locus.
 
\bye

