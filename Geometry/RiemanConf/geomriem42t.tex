\def\vare {\varepsilon}
\def\A {{\bf A}}
\def\t {\tilde}
\def\a {\alpha}
\def\K {{\bf K}}
\def\N {{\bf N}}
\def\V {{\cal V}}
\def\s {{\sigma}}
\def\S {{\bf S}}
\def\s {{\sigma}}
\def\p{\partial}
\def\vare{{\varepsilon}}
\def\Q {{\bf Q}}
\def\D {{\cal D}}
\def\G {{\Gamma}}
\def\C {{\bf C}}
\def\M {{\cal M}}
\def\Z {{\bf Z}}
\def\U  {{\cal U}}
\def\H {{\cal H}}
\def\R  {{\bf R}}
\def\E  {{\bf E}}
\def\l {\lambda}
\def\degree {{\bf {\rm degree}\,\,}}
\def \finish {${\,\,\vrule height1mm depth2mm width 8pt}$}
\def \m {\medskip}
\def\p {\partial}
\def\r {{\bf r}}
\def\v {{\bf v}}
\def\n {{\bf n}}
\def\t {{\bf t}}
\def\b {{\bf b}}
\def\e{{\bf e}}
\def\f{{\bf f}}
\def\ac {{\bf a}}
\def \X   {{\bf X}}
\def \Y   {{\bf Y}}
\def\diag {\rm diag\,\,}
\def\pt {{\bf p}}
\def\w {\omega}
\def\la{\langle}
\def\ra{\rangle}
\def\x{{\bf x}}

\documentclass[12pt]{article}
\usepackage{amsmath,amsthm}


\usepackage{amsmath,amssymb,amsfonts,amsthm}


\theoremstyle{theorem}
\newtheorem{thm}{Khimera}

\numberwithin{equation}{section}


%\title{Riemannian Geometry}
%\date{}
\begin{document}
%\maketitle


\tableofcontents







\section {Surfaces in $\E^3$}

 Now equipped by the knowledge of Riemannian geometry we consider surfaces in $\E^3$. We
  reconsider again conceptions of Shape (Weingarten) operator, Gaussian and mean curvatures,
  focusing attention on the the fact what properties are internal and what properties are external.


.
 \subsection{ Induced metric on surfaces.}

   Recall here again induced metric (see for detail subsection 1.4)

  If surface $M\colon\,\, \r=\r(u,v)$is embedded in $\E^3$ then induced Riemannian metric
  $G_M$ is defined by the formulae
  \begin{equation}\label{scalarproduct2}
    \langle\X,\Y\rangle=G_M(\X,\Y)=G(\X,\Y)\,,
  \end{equation}
  where $G$ is Euclidean metric in $\E^3$:
           $$
        G_M=dx^2+dy^2+dz^2\big\vert_{\r=\r(u,v)}=
        \sum_{i=1}^3(dx^i)^2\big\vert_{\r=\r(u,v)}=\sum_{i=1}^3\left({\p x^i\over \p u^\a}du^\a\right)^2
                $$
                $$=
        {\p x^i\over \p u^\a} {\p x^i\over\p  u^\beta}du^\a du^\beta\,,
           $$
        i.e.
            $$
      G_M=g_{\a\beta}du^\a ,\,\, {\rm where}\,\, g_{\a\beta}=
      {\p x^i\over \p u^\a} {\p x^i\over \p u^\beta}du^\a du^\beta\,.
           $$
 We use notations $x,y,z$ or $x^i$ ($i=1,2,3$) for Cartesian coordinates in $\E^3$,
 $u,v$ or $u^\a$ ($\a=1,2$) for coordinates on the surface. We usually   omit summation symbol
 over  dummy indices.    For coordinate tangent vectors
           $$
 \underbrace{\p\over \p u_\a}_{\hbox{Internal observer}} =
 \underbrace{\r_\a={\p x^i\over \p u^\a}{\p\over \p x^i}}_{\hbox{External observer}}
           $$
We have already plenty examples in the subsection 1.4. In particular for scalar product
\begin{equation}\label{scalarproduct4}
  g_{\a\beta}=\left\langle{\p\over u_\a},{\p\over u_\beta}\right\rangle=x^i\a x^i\beta\,.
  \langle \r_\a,\r_\beta\rangle\,.
\end{equation}



  \subsection {Induced connection}

  We already studied before induced connection on manifolds embedded in $\E^n$ (see the subsection 2.2).

  We do it again in a more detail for  surfaces in $\E^3$.

Let $\nabla^{\rm can.flat}$ be a canonical flat connection in Euclidean space $\E^3$, i.e. a connection such
that its  Christoffel symbols vanish in cartesian coordinates, i.e.
\begin{equation}\label{canonicalflatconnection}
    \nabla^{\rm can.flat}_\X \Y= \p_\X \Y= \X^m(x){\p \over x^m}Y^i(x){\p\over \p x^i}
\end{equation}
in  Cartesian coordinates. (Of course in arbitrary curvilinear coordinates Christoffel symbols
of canonical flat connection may have non-zero components, but in Cartesian coordinates Christoffel symbols vanish.
It is why we sometimes use notation  $\p_\X$ instead $\nabla^{\rm can.flat}_\X \Y$.)


   Induced connection $\nabla$ on the surface can be defined by the relation
             \begin{equation}\label{inducedconnection4}
             {\nabla}_\X\Y=\left( \nabla^{\rm can.flat}_\X \Y\right)_{\rm tangent}
             \end{equation}
          (See also the subsection (2.2))
One can show that this formula indeed defines the connection on $M$.


Calculate Christoffel symbols of this connection using the formulae above and the expressions
for basic vectors $\p_\a={\p\over \p u^\a}$,  $\p_\beta={\p\over \p u^\beta}$ for external observer.

We have
\begin{equation}\label{christoffelsymbols1forinducedconnection}
\nabla_\a\p_\beta=\Gamma^\gamma_{\a\beta}\p_\gamma=
\left( \nabla^{\rm can.flat}_{\p_\a} {\p_\beta}\right)_{\rm tangent}=
\left( \nabla^{\rm can.flat}_{\p_\a} {\r_\beta}\right)_{\rm tangent}
\end{equation}
i.e.
\begin{equation}\label{christoffelsymbols1forinducedconnection2}
 \Gamma^\gamma_{\a\beta}\r_\gamma=\left({\p^2\r \over \p u^\a\p u^\beta}\right)_{\rm tangent},
\end{equation}
Before continuing calculations note that we see already that the induced connection is {\it symmetric} one.

Now continue to calculate. Any vector attached at the surface can be decomposed as a sum of tangent vector and the vector orthogonal to the surface:
               $$
           {\p^2\r \over \p u^\a\p u^\beta}=\left({\p^2\r \over \p u^\a\p u^\beta}\right)_{\rm tangent}+\N
               $$
 Hence
 \begin{equation}\label{christoffelsymbols1forinducedconnection4}
 \Gamma^\gamma_{\a\beta}\r_\gamma=\left({\p^2\r \over \p u^\a\p u^\beta}\right)_{\rm tangent}=
 {\p^2\r \over \p u^\a\p u^\beta}-\N= \r_{\a\beta}-\N
\end{equation}
Take scalar product of  both parts of this equation on the tangent vector vector $\r_\pi$, we come to:
            $$
         \Gamma^\gamma_{\a\beta}\langle\r_\gamma, \r_\pi\rangle=
         \langle\r_{\a\beta},\r_\pi\rangle+\langle\N,\r_\pi\rangle\,.
            $$
We have that $\N\bot \r_\pi$, hence $\langle\N,\r_\pi\rangle=0$. Using the fact that
$\langle\r_\gamma, \r_\pi\rangle=g_{\gamma\pi}$ are entries of the matrix $||G_M||$ of the induced Riemannian metric
(see the equation \eqref{scalarproduct4}) we come to
               $$
\Gamma^\gamma_{\a\beta}g_{\gamma\pi}=\langle\r_{\a\beta},\r_\pi\rangle=x^i_{\a\beta}x^i_\pi\,.
               $$
Multiplying both sides of this equation on inverse matrix we come to :
               $$
\Gamma^\gamma_{\a\beta}g_{\gamma\pi}g^{\pi\rho}=\Gamma^\gamma_{\a\beta}\delta_\gamma^\rho=
\Gamma^\rho_{\a\beta}=x^i_{\a\beta}x^i_\pi g^{\pi\rho}\,.
               $$

We come to the explicit formula for the Christoffel symbols of the induced connection in terms of
parametric equations $x^i(u^\a)$ and induced metric $g_{\a\beta}$:
\begin{equation}\label{connectionintermsofinducedmetric}
\Gamma^\gamma_{\a\beta}=x^i_{\a\beta}x^i_\pi g^{\pi\gamma}\,.
\end{equation}

\subsubsection{Induced connection---the Levi Civita connection of induced metric.}

 Now you may ask a question: is there another  way to obtain induced connection (may be different one )
 on the manifold  $x^i=x^i(u^\a)$. Recall that one can consider Levi-Civita connection of
 Riemannian metric $G_M=g_{\a\beta}du^\a du^\beta$ with Christoffel symbols:
                   \begin{equation}\label{levi-civita12}
      \Gamma^\gamma_{\a\beta}={1\over 2}g^{\gamma\pi}
      \left({\p g_{\pi \a}\over \p u^\beta}+{\p g_{\pi \beta}\over \p u^\a}-
      {\p g_{\a\beta}\over \p u^\pi}\right)
                   \end{equation}
 (see the subsection about Levi-Civita connection)

 {\sl Natural question arises: may be these connections are the same?}----Yes, it is

              \m

    {\bf Proposition} {\it The connection $\nabla$ induced through the flat canonical connection on the surface
    $M$  coincides with Levi-Civita connection. In other words Christoffel symbols \eqref{scalarproduct4}
    coincide with Christoffel symbols \eqref{levi-civita12}.}.
    The proof of this Proposition follows from the  following claim

              \m
       {\bf Claim} The induced connection $\nabla$ is symmetric connection which is compatible with metric.


     Indeed we proved that Levi-Civita connection \eqref{levi-civita12} is a unique symmetric connection which is compatible with metric.
     Hence every connection which is compatible wit metric equals to Levi-Civita connection \eqref{levi-civita12}.


     It remains to prove the Claim.  When calculating Christoffel symbols of induced connection we already
     noted that its Christoffel symbols are symmetric. It remains to prove that induced connection is compatible
     with induced metric, i.e. covariant derivative preserves scalar product of tangent vectors.
     To prove this note that covariant derivative with respect to flat canonical metric preserves
 metric in Euclidean space:
                              $$
    \p_\X\langle \Y,\Z\rangle=\langle \nabla^{\rm can.flat}_\X\Y,\Z\rangle+\langle \Y,\nabla^{\rm can.flat}_\X\Z\rangle
                              $$
                 On the other hand for an arbitrary tangent vector $\R$
                 $$
   \nabla^{\rm can.flat}_\X \R= \left(\nabla^{\rm can.flat}_\X \R\right)_{\rm tangent}+\hbox {vector orthogonal to the surface}
                $$
    and $\left(\nabla^{\rm can.flat}_\X \R\right)_{\rm tangent}=\nabla_\X\R$, where $\nabla$ is the induced connection. Hence
                             $$
     \p_\X\langle \Y,\Z\rangle=\langle \nabla_\X\Y+\dots,\Z\rangle+\langle \Y,\nabla_\X\Z+\dots\rangle=
     \langle \nabla_\X\Y,\Z\rangle+\langle \Y,\nabla_\X\Z\rangle,
                             $$
 where we denote by $\dots$ vectors orthogonal to the surface.  Thus
 $\nabla$ preserves induced metric. The claim is proved.


 Note that we come to two different definitions for induced connections and prove that they same.
 The second connection is totally described in terms of metric on the surface, i.e. in terms of {\it Internal Observer}.
  The first connection
 is explicitly depended on the embedding functions. It is described in terms of {\it External Observer.}

 We continue to consider different geometrical structures form the point of view of External and Internal Observers.

 We will try to prove the remarkable theorem Gau\ss's {\it Theorema Egregium} that the Gaussian  curvature
 of the surface (which you already learned in the course of geometry)  in fact depends only on induced metric...

 \subsection {Weingarten (shape) operator and second quadratic form}

   \subsubsection {Recalling Weingarten operator}

 Continue to play with formulae \footnote{In some sense differential geometry it is when we write down
 the formulae expressing the geometrical facts, differentiate these formulae then reveal the geometrical meaning
  of the new obtained formulae e.t.c.}.


Recall the Weingarten (shape) operator which acts on tangent vectors:

  \begin{equation}\label{weingoperator1}
    S\X=-\p_\X \n\,,
  \end{equation}
 where we denote by $\n$-unit normal vector field at the points of the surface $M$: $\langle\n,\r_\a\rangle=0$,
 $\langle\n,\r_\a\rangle=1$.

\m

 Now we realise that the derivative $\p_\X \R$ of vector field with respect to another vector field
 is not a well-defined object: we need a connection.
 The formula $\p_\X \R$ in Cartesian coordinates,
 is nothing but the derivative with respect to flat canonical connection:
 If we work only in Cartesian coordinates we do not need to distinguish between
 $\p_\X \R$ and $\nabla^{\rm can.flat}_\X\R $. Sometimes with some abuse of notations we will use
 $\p_\X \R$ instead $\nabla^{\rm can.flat}_\X\R $, but never forget: this can be done only in Cartesian coordinates
 where Christoffel symbols of flat canonical connection vanish:
              $$
      \p_\X \R=\nabla^{\rm can.flat}_\X\R  \quad \hbox{in Cartesian coordinates}\,.
         $$

 So  the rigorous definition of Weingarten operator is
\begin{equation}\label{weingoperator2}
    S\X=-\nabla^{\rm can.flat}_\X \n\,,
  \end{equation}
  but we often use the former one (equation \eqref{weingoperator1})
   just remembering that this can be done only in Cartesian coordinates.

  Recall that the fact that Weingarten operator $S$ maps tangent vectors to tangent vectors follows from the property:
  $\langle\n,\X\rangle=0\Rightarrow \X$ is tangent to the surface.

  Indeed:
     $$
 0= \p_\X\langle\n,\n\rangle=2\langle\p_\X\n,\n\rangle=-2\langle S\X,\n\rangle=0\Rightarrow S\X\,\,
 \hbox {is tangent to the surface}
     $$
     \m
Recall also that normal unit vector is defined up to a sign, $\n\to -\n$. On the other hand if $\n$
is chosen then $\S$ is defined uniquely.


\subsubsection {Second quadratic form}

We define now the new object: {\it second quadratic form}

{\bf Definition}. For two tangent vectors $\X, \Y$
$A(\X,\Y)$ is defined such that
\begin{equation}\label{secondquadraticform1}
    \left(\nabla^{\rm can.flat}_\X \Y\right)_\bot=A(\X,\Y)\n
\end{equation}
i.e. we take orthogonal component of the derivative of $\Y$ with respect to $\X$.

This definition seems to be very vague: to evaluate covariant derivative we have to consider not a
vector $\Y$ at a given point
but  the vector field.
In fact one can see that  $A(\X,\Y)$ does depend only on the value of $\Y$ at the given point.

Indeed it follows from the definition of second quadratic form and from the properties of Weingarten operator that
        $$
         A(\X,\Y)=\left\langle\left(\nabla^{\rm can.flat}_\X \Y\right)_\bot,\n\right\rangle=
    \left\langle\nabla^{\rm can.flat}_\X \Y,\n\right\rangle=
        $$
\begin{equation}\label{properofsecquadrform4}
    \p_\X \langle\Y,\n\rangle-\left\langle \Y,\nabla^{\rm can.flat}_\X\n\right\rangle=\langle S(\X), \Y\rangle
\end{equation}



We proved that second quadratic form depends only on value of vector field $\Y$ at the given poit and we established
the relation between second quadratic form and Weingarten operator.

\m

{\bf Proposition} {\it The second quadratic form $A(\X,\Y)$ is symmetric bilinear form on tangent vectors $\X,\Y$
in a given point.}

       \begin{equation}\label{symmetricityofsecquadrform}
A\colon T_\pt M\otimes T_\pt M\to \R,\quad A(\X,\Y)=A(\Y,\X)=\langle S\X,\Y\rangle\,.
       \end{equation}

 In components
 \begin{equation}\label{compexpressforscquadrform}
    A=A_{\a\beta}du^\a du^\beta=\langle \r_{\a\beta}, \n\rangle={\p^2 x^i\over \p u^\a \p u^\beta}n^i\,.
 \end{equation}
and
\begin{equation}\label{compexpressforweingarten}
    S^\a_\beta=g^{\a\pi}A_{\pi\beta}=g^{\a\pi}x^i_{\pi\beta}n^i\,,
     \end{equation}
i.e.
            $$
         A=GS, S=G^{-1}A\,.
            $$

 {\bf Remark} The normal unit vector field is defined up to a sign.


\subsubsection {Gaussian and mean curvatures}

   Recall that Gaussian curvature
            $$
          K=\det S
            $$
and mean curvature
           $$
         H={\rm Tr \,}S
           $$
It is easy to see that  for Gaussian curvature
           $$
         K=\det S=\det(G^{-1}A)={\det A\over \det G}.
           $$

We know already the geometrical meaning of Gaussian and mean curvatures from the point of view of the External Observer:

Gaussian curvature $K$ equals to the product of principal curvatures, and mean curvartures equals to the sum of principal curvatures.

Now we ask a principal question: what bout internal observer, "aunt" living on the surface?

We will show that Gaussian curvature can be expressed in terms of induced Riemannian metric, i.e. it is an
 internal characteristic of the surface, invariant of isometries.

  It is not the case with mean curvature: cylinder is isometric to the plane but it have
   non-zero mean curvature.

\subsubsection {Examples of calculation of Weingarten operator,
Second quadratic forms, curvatures for cylinder, cone and sphere.}

\m

{\it Cylinder}

\m

  We already calculated induced Riemannian metric on the cylinder (see \eqref {formula forfirstformcyl}).

 Cylinder is given by the equation $x^2+y^2=R^2$. One can consider the following
parameterisation
 of this surface:
\begin{equation}\label{surface11}
  \r(h,\varphi)\colon\quad
  \begin{cases}
  x=R\cos\varphi\\
  y=R\sin\varphi\\
  z=h\\
  \end{cases}\,,\quad   \r_h=
  \begin{pmatrix}
        0\\
        0\\
        1\\
   \end{pmatrix}\,,
\quad
  \r_\varphi=\begin{pmatrix}
        -R\sin\varphi\\
        R\cos\varphi\\
          0\\
   \end{pmatrix}\,,
\end{equation}
$G_{cylinder}=\left(dx^2+dy^2+dz^2\right)\big\vert_{x=a\cos\varphi,y=a\sin\varphi,z=h}=$
        \begin{equation*}\label{firtsquadraticformcylinder(diff)11}
               =(-a\sin\varphi d\varphi)^2+(a\cos\varphi d\varphi)^2+dh^2=a^2d\varphi^2+dh^2\,,\quad
               ||g_{\a\beta}||=
  \begin{pmatrix}
   1 & 0 \\
   0& R^2 \\
   \end{pmatrix}\,.
        \end{equation*}

   Normal unit vector $\n=\pm \begin{pmatrix}
        \cos \varphi\cr
        \sin\varphi\cr
        0\cr
   \end{pmatrix}$. Choose $\n=\begin{pmatrix}
        \cos \varphi\cr
        \sin\varphi\cr
        0\cr
   \end{pmatrix}$. Weingarten operator
     \begin{equation*}\label{weingrartenforcylinder}
        S\p_h=-\nabla^{\rm can.flat}_{\r_h}\n=-\p_{\r_h} \n=-\p_h\begin{pmatrix}
        \cos \varphi\cr
        \sin\varphi\cr
        0\cr
   \end{pmatrix}=0\,,
     \end{equation*}
\begin{equation*}\label{weingrartenforcylinder}
        S\p_\varphi=-\nabla^{\rm can.flat}_{\r_\varphi}\n=-\p_{\r_\varphi} \n=-{\p_\varphi}
        \begin{pmatrix}
        \cos \varphi\cr
        \sin\varphi\cr
        0\cr
   \end{pmatrix}=\begin{pmatrix}
        \sin\varphi \cr
        -\cos\varphi\cr
        0\cr
   \end{pmatrix}=-{\p_\varphi\over R}.
    \end{equation*}
\begin{equation}\label{weingrartenforcylinder2}
           S
   \begin{pmatrix}
        \p_h \cr
        \p_\varphi\cr
   \end{pmatrix}=
   \begin{pmatrix}
         0\cr
        {-\p_\varphi\over R}\cr
   \end{pmatrix},\quad S=\begin{pmatrix}0&0\cr 0 &{-1\over R}\end{pmatrix}\,.
    \end{equation}
Calculate   second quadratic form:  $\r_{hh}=\p_h\r_h=
  \begin{pmatrix}
        0\\
        0\\
        0\\
   \end{pmatrix}\,$, $\r_{h\varphi}=\r_{\varphi h}=$
             $$
             \p_h
       \begin{pmatrix}
        -R\sin\varphi\\
        R\cos\varphi\\
          0\\
   \end{pmatrix}=0,\,\,
   \r_{\varphi \varphi}=\p_\varphi
       \begin{pmatrix}
        -R\sin\varphi\\
        R\cos\varphi\\
          0\\
   \end{pmatrix}=\begin{pmatrix}
        -R\cos\varphi\\
        -R\sin\varphi\\
          0\\
   \end{pmatrix}=-R\n\,.
                      $$
We have
                    \begin{equation}\label{secquadrformforcylinder}
            A_{\a\beta}=\langle\r_{\a\beta},\n\rangle,\,
              A=\begin{pmatrix}\langle\r_{hh},\n\rangle&\langle\r_{h\varphi},\n\rangle\cr
                               \langle\r_{\varphi h},\n\rangle &\langle\r_{\varphi\varphi},\n\rangle\cr
                                   \end{pmatrix}=
                                   \begin{pmatrix}0&0\cr
                                0&-R\cr
                                   \end{pmatrix},
                    \end{equation}
                    $$
       A=GS=\begin{pmatrix}1&0\cr
                                0&R^2\cr
                                   \end{pmatrix}
                                   \begin{pmatrix}0&0\cr
                                0&-{1\over R}\cr
                                   \end{pmatrix}=
                                   \begin{pmatrix}0&0\cr
                                0&-R\cr
                                   \end{pmatrix},
                    $$
For Gaussian and mean curvatures we have
     \begin{equation}\label{Gaussianforcylinder}
        K=\det S={\det A\over \det G}=\det
                              \begin{pmatrix}0&0\cr
                                0&-{1\over R}\cr
                                   \end{pmatrix}=0\,,
     \end{equation}
     and mean curvature
      \begin{equation}\label{meanianforcylinder}
        H={\rm Tr\,} S={\rm Tr\,}
                              \begin{pmatrix}0&0\cr
                                0&-{1\over R}\cr
                                   \end{pmatrix}=-{1\over R}\,.
     \end{equation}
Mean curvature is define up to a sign. If we change $\n\to-\n$ mean curvature $H\to {1\over R}$ and Gaussian curvature
will not change.


\bigskip

{\it Cone}

  We already calculated induced Riemannian metric on the cone (see \eqref{firtsquadraticformforconus(diff)}.

 Cone is given by the equation $x^2+y^2-k^2z^2=0$. One can consider the following
parameterisation
 of this surface:
\begin{equation}\label{surface111}
  \r(h,\varphi)\colon\quad
  \begin{cases}
  x=kh\cos\varphi\\
  y=kh\sin\varphi\\
  z=h\\
  \end{cases}\,,\quad   \r_h=
  \begin{pmatrix}
        k\cos\varphi\\
        k\sin\varphi\\
        1\\
   \end{pmatrix}\,,
\quad
  \r_\varphi=\begin{pmatrix}
        -kh\sin\varphi\\
        kh\cos\varphi\\
          0\\
   \end{pmatrix}\,,
\end{equation}
$G_{cone}=\left(dx^2+dy^2+dz^2\right)\big\vert_{x=kh\cos\varphi,y=kh\sin\varphi,z=h}=$
        \begin{equation*}\label{firtsquadraticformcylinder(diff)11}
               =(-kh\sin\varphi d\varphi+k\cos\varphi dh)^2+
               (kh\cos\varphi d\varphi+k\sin\varphi dh)^2+dh^2=
                      \end{equation*}
             $$
              k^2h^2d\varphi^2+(k^2+1)dh^2,\quad
               ||g_{\a\beta}||=
  \begin{pmatrix}
   k^2+1 & 0 \\
   0& k^2h^2 \\
   \end{pmatrix}\,.
          $$

  One can see that  $\N=
      \begin{pmatrix}
        \cos \varphi\cr
        \sin\varphi\cr
        -k\cr
   \end{pmatrix}$ is orthogonal to the surface: $\N\bot \r_h,\r_\varphi$. Hence
   normal unit vector $\n=\pm
     {1\over {\sqrt {1+k^2}}}
      \begin{pmatrix}
        \cos \varphi\cr
        \sin\varphi\cr
        -k\cr
   \end{pmatrix}$. Choose $\n=
   {1\over {\sqrt {1+k^2}}}
      \begin{pmatrix}
        \cos \varphi\cr
        \sin\varphi\cr
        -k\cr
   \end{pmatrix}$.
     Weingarten operator
     \begin{equation*}\label{weingrartenforcylinder}
        S\p_h=-\nabla^{\rm can.flat}_{\r_h}\n=-\p_{\r_h} \n=
        -\p_h
        \left(
        {1\over {\sqrt {1+k^2}}}
      \begin{pmatrix}
        \cos \varphi\cr
        \sin\varphi\cr
        -k\cr
   \end{pmatrix}
   \right)
        =0\,,
     \end{equation*}
\begin{equation*}\label{weingrartenforcylinder}
        S\p_\varphi=-\nabla^{\rm can.flat}_{\r_\varphi}\n=-\p_{\r_\varphi} \n=
        -\p_\varphi
        \left(
{1\over {\sqrt {1+k^2}}}
      \begin{pmatrix}
        \cos \varphi\cr
        \sin\varphi\cr
        -k\cr
   \end{pmatrix}
   \right)
       =
       \end{equation*}
       \begin{equation*}
       {1\over {\sqrt {1+k^2}}}
       \begin{pmatrix}
        \sin\varphi \cr
        -\cos\varphi\cr
        0\cr
   \end{pmatrix}=-{\p_\varphi\over kh\sqrt {k^2+1}}.
    \end{equation*}
\begin{equation}\label{weingrartenforcylinder2}
           S
   \begin{pmatrix}
        \p_h \cr
        \p_\varphi\cr
   \end{pmatrix}=
   \begin{pmatrix}
         0\cr
        -{\p_\varphi\over kh\sqrt {k^2+1}}\cr
   \end{pmatrix},\quad S=\begin{pmatrix}0&0\cr 0 &{-1\over kh\sqrt {k^2+1}}\end{pmatrix}\,.
    \end{equation}
Calculate   second quadratic form:  $\r_{hh}=\p_h\r_h=
  \begin{pmatrix}
        0\\
        0\\
        0\\
   \end{pmatrix}\,$, $\r_{h\varphi}=\r_{\varphi h}=$
             $$
             \p_h
       \begin{pmatrix}
        -kh\sin\varphi\\
        kh\cos\varphi\\
          0\\
   \end{pmatrix}=\begin{pmatrix}
        -k\sin\varphi\\
        k\cos\varphi\\
          0\\
   \end{pmatrix},\,\,
   \r_{\varphi \varphi}=\p_\varphi
       \begin{pmatrix}
        -kh\sin\varphi\\
        kh\cos\varphi\\
          0\\
   \end{pmatrix}=\begin{pmatrix}
        -kh\cos\varphi\\
        -kh\sin\varphi\\
          0\\
   \end{pmatrix}\,.
                      $$
We have
                    \begin{equation}\label{secquadrformforcylinder}
            A_{\a\beta}=\langle\r_{\a\beta},\n\rangle,\,
              A=\begin{pmatrix}\langle\r_{hh},\n\rangle&\langle\r_{h\varphi},\n\rangle\cr
                               \langle\r_{\varphi h},\n\rangle &\langle\r_{\varphi\varphi},\n\rangle\cr
                                   \end{pmatrix}=
                                   \begin{pmatrix}0&0\cr
                                0&-{kh\over {\sqrt {1+k^2}}}\cr
                                   \end{pmatrix},
                    \end{equation}
                    $$
       A=GS=\begin{pmatrix}k^2+1&0\cr
                                0&k^2h^2\cr
                                   \end{pmatrix}
                                   \begin{pmatrix}0&0\cr
                                0&{-1\over kh\sqrt {k^2+1}}\cr
                                   \end{pmatrix}=
                                   \begin{pmatrix}0&0\cr
                                0&{-kh\over \sqrt {k^2+1}}\cr
                                   \end{pmatrix},
                    $$
For Gaussian and mean curvatures we have
     \begin{equation}\label{Gaussianforcylinder}
        K=\det S={\det A\over \det G}=\det
                              \begin{pmatrix}0&0\cr
                                0&{-1\over kh\sqrt {k^2+1}}\cr
                                   \end{pmatrix}=0\,,
     \end{equation}
     and mean curvature
      \begin{equation}\label{meanianforcylinder}
        H={\rm Tr\,} S={\rm Tr\,}
                              \begin{pmatrix}0&0\cr
                                0&{-1\over kh\sqrt {k^2+1}}\cr
                                   \end{pmatrix}={-1\over kh\sqrt {k^2+1}}\,.
     \end{equation}
Mean curvature is define up to a sign. If we change $\n\to-\n$ mean curvature $H\to {1\over R}$ and Gaussian curvature
will not change.


\bigskip




{\it Sphere}


\medskip


  Sphere is given by the equation $x^2+y^2+z^2=a^2$. Consider the  parameterisation
 of sphere in spherical coordinates
\begin{equation}\label{surfacesphere11}
  \r(\theta,\varphi)\colon\quad
  \begin{cases}
  x=R\sin\theta\cos\varphi\\
  y=R\sin\theta\sin\varphi\\
  z=R\cos\theta\\
  \end{cases}
\end{equation}

\medskip

We already calculated induced Riemannian metric on the sphere (see \eqref{firtsquadraticformforsphere(diff)}).
 Recall that
    $$
  \r_\theta=\begin{pmatrix}
        R\cos\theta\cos\varphi\\
        R\cos\theta\sin\varphi\\
        -R\sin\theta\\
   \end{pmatrix}\,,
\quad
  \r_\varphi=\begin{pmatrix}
        -R\sin\theta\sin\varphi\\
        R\sin\theta\cos\varphi\\
          0\\
   \end{pmatrix}
  $$
 and
            $$
              G_{S^2}=\left(dx^2+dy^2+dz^2\right)\big\vert_{x=R\sin\theta\cos\varphi,y=R\sin\theta\sin\varphi,
              z=R\cos\theta}=
                      $$
                      $$
                      (R\cos\theta\cos\varphi d\theta-R\sin\theta\sin\varphi d\varphi)^2+
                      (R\cos\theta\sin\varphi d\theta+R\sin\theta\cos\varphi d\varphi)^2+
                      $$
                      $$
                      (-R\sin\theta d\theta)^2=
          R^2\cos^2\theta d\theta^2+R^2\sin^2\theta d\varphi^2+R^2\sin^2\theta d\theta^2=
                      $$
        \begin{equation*}\label{firtsquadraticformforsphere(diff)}
           \qquad
             =R^2d\theta^2+R^2\sin^2\theta d\varphi^2\,,\qquad
                        ||g_{\a\beta}||=
   \begin{pmatrix}
   R^2 & 0 \\
   0&  R^2\sin^2\theta \\
   \end{pmatrix}\,.
                       \end{equation*}

For the sphere $\r(\theta,\varphi)$ is orthogonal to the surface.
       Hence
   normal unit vector
   $   \n(\theta,\varphi)=\pm {\r(\theta,\varphi)\over R}=\pm
   \begin{pmatrix}
    \sin\theta\cos\varphi\cr
     \sin\theta\sin\varphi\cr
      \cos\theta\cr
   \end{pmatrix}$.
           Choose $\n= {\r\over R}=
   \begin{pmatrix}
    \sin\theta\cos\varphi\cr
     \sin\theta\sin\varphi\cr
      \cos\theta\cr
   \end{pmatrix}.
     $
     Weingarten operator
     \begin{equation*}\label{weingrartenforcylinder}
        S\p_\theta=-\nabla^{\rm can.flat}_{\r_\theta}\n=-\p_\theta \n=
          -\p_\theta\left({\r\over R}\right)=-{\r_\theta\over R}\,,
       \end{equation*}
       \begin{equation*}\label{weingrartenforcylinder}
       S\p_\varphi=-\nabla^{\rm can.flat}_{\r_\varphi}\n=-\p_\varphi \n=
          -\p_\varphi\left({\r\over R}\right)=-{\r_\varphi\over R}\,.
    \end{equation*}
\begin{equation}\label{weingrartenforcylinder2}
           S
   \begin{pmatrix}
        \p_\theta \cr
        \p_\varphi\cr
   \end{pmatrix}=
   \begin{pmatrix}
         -{\p_\theta\over R}\cr
        -{\p_\varphi\over R}\cr
   \end{pmatrix},\quad S=
        -
   \begin{pmatrix}{1\over R}&0\cr 0 &{1\over R}\end{pmatrix}\,.
    \end{equation}
For second quadratic form:  $\r_{\theta\theta}=\p_\theta\r_\theta=
  \begin{pmatrix}
        -R\sin\theta\cos\varphi\\
        -R\sin\theta\sin\varphi\\
        -R\cos\theta\\
   \end{pmatrix}\,$, $\r_{\theta\varphi}=\r_{\varphi \theta}=$
             $$
             \p_\theta
       \begin{pmatrix}
        -R\sin\theta\sin\varphi\\
        R\sin\theta\cos\varphi\\
          0\\
   \end{pmatrix}=\begin{pmatrix}
        -R\cos\theta\sin\varphi\\
        R\cos\theta\cos\varphi\\
          0\\
   \end{pmatrix},\,\,
   \r_{\varphi \varphi}=\p_\varphi
       \begin{pmatrix}
        -R\sin\theta\sin\varphi\\
          R\sin\theta\cos\varphi\\
          0\\
   \end{pmatrix}=\begin{pmatrix}
        -R\sin\theta\cos\varphi\\
        -R\sin\theta\sin\varphi\\
          0\\
   \end{pmatrix}\,.
                      $$
We have
                    \begin{equation}\label{secquadrformforcylinder}
            A_{\a\beta}=\langle\r_{\a\beta},\n\rangle,\,
              A=\begin{pmatrix}\langle\r_{\theta\theta},\n\rangle&\langle\r_{\theta\varphi},\n\rangle\cr
                               \langle\r_{\varphi \theta},\n\rangle &\langle\r_{\varphi\varphi},\n\rangle\cr
                                   \end{pmatrix}=
                                   \begin{pmatrix}-R&0\cr
                                0&-R\sin^2\theta\cr
                                   \end{pmatrix},
                    \end{equation}
                    $$
       A=GS=\begin{pmatrix}R^2&0\cr
                                0&R^2\sin^2\theta\cr
                                   \end{pmatrix}
                                   \begin{pmatrix}{-1\over R}&0\cr
                                0&{-1\over R}\cr
                                   \end{pmatrix}=
                                       -R
                                  \begin{pmatrix}1&0\cr
                                0&\sin^2\theta\cr
                                   \end{pmatrix},
                    $$
For Gaussian and mean curvatures we have
     \begin{equation}\label{Gaussianforcylinder}
        K=\det S={\det A\over \det G}=\det
                              \begin{pmatrix}-{1\over R}&0\cr
                                0&-{1\over R}\cr
                                   \end{pmatrix}={1\over R^2}\,,
     \end{equation}
     and mean curvature
      \begin{equation}\label{meanianforcylinder}
        H={\rm Tr\,} S={\rm Tr\,}
                              \begin{pmatrix}-{1\over R}&0\cr
                                0&-{1\over R}\cr
                                   \end{pmatrix}=-{2\over R}\,,
     \end{equation}
Mean curvature is define up to a sign. If we change $\n\to-\n$ mean curvature $H\to {1\over R}$ and Gaussian curvature
will not change.


We see that for the sphere  Gaussian curvature is not equal to zero, whilst for cylinder and cone Gaussian curvature
equals to zero.


\subsection {Parallel transport along closed curves on surfaces. Derivation formulae and {\it Theorema Egregium} }

\subsubsection {Formulation of the main result}

Let $M$ be a surface in  Euclidean space $\E^3$.
  Consider a closed curve $C$ on $M$,
  $M\colon \r=\r(u,v)$, $C\colon \r=\r(u(t,v(t)), 0\leq t\leq t_1, \, \x(0)=\x(t_1)$.
  ($u(t), v(t)$ are internal coordinates of the curve $C$.)

Consider the parallel transport of an arbitrary tangent $\X$ vector along the closed curve $C$:
          \begin{equation*}\label{paraltransportalongclosedcurve1}
\X(t)= \underbrace {X^\a(t){\p\over \p u_\a}\big\vert_{u^\a(t)}}_{\hbox{Internal observer}} =
 \underbrace{X^\a(t)\r_\a\big\vert_{\r(u(t),v(t))}}_{\hbox{External observer}}\,\,,\left(
 \r_\a={\p x^i\over \p u^\a}{\p\over \p x^i}\right)\,.
          \end{equation*}
          \begin{equation*}\label{paraltransportalongclosedcurve2}
\X(t)\colon\,\, {\nabla \X(t)\over dt}=0,\,\, 0\leq t\leq t_1\,,
          \end{equation*}
i.e.
                \begin{equation}\label{paraltransportalongclosedcurve3}
{d X^\a(t)\over dt}+
X^\beta(t)\Gamma^\a_{\beta\gamma}(u(t))
{du^\gamma(t)\over dt}=0,\,\, 0\leq t\leq t_1\,,
          \end{equation}
where $\nabla$ is the connection induced on the surface $M$ by canonical flat connection
(see \eqref{christoffelsymbols1forinducedconnection}), i.e.
the Levi-Civita connection of the induced Riemannian metric on the surface $M$
 and $\Gamma^\a_{\beta\gamma}$ its Christoffel symbols:
\begin{equation}\label{connectioninduced12}
          \Gamma^\gamma_{\a\beta}={1\over 2}g^{\gamma\pi}
      \left({\p g_{\pi \a}\over \p u^\beta}+{\p g_{\pi \beta}\over \p u^\a}-
      {\p g_{\a\beta}\over \p u^\pi}\right)\,,{\rm where}\,
      g_{\a\beta}=\langle\r_\a,\r_\beta\rangle={\p x^i\over \p u^\a} {\p x^i\over \p u^\beta}
     \end{equation}
 are components of induced Riemannian metric
 $G_M=g_{\a\beta}du^\a$    (see the subsection 4.2.1)


Let $\r(0)=\pt$ be a starting (and ending) point of the curve $C$: $\r(0)=\r(t_1)=\pt$.
The differential equation \eqref{paraltransportalongclosedcurve3} defines the linear operator
            \begin{equation}\label{linearoperatoroverclosedcurve}
            R_C\colon T_{\pt}M\longrightarrow T_{\pt}M
            \end{equation}
For any vector $\X\in T_{\pt}M$, its image the vector $R_C\X$ as the solution of the differential equation
\eqref{paraltransportalongclosedcurve3} with initial condition $\X(t)\big\vert_{t=0}=\X$.

On the other hand we know that parallel transport of the vector does not change its length (see
\eqref{lengthdoesnotchangeduringparalleltransport} in the subsection 3.2.1):
       \begin{equation}\label{linearoperatoroverclosedcurvescalarproduct}
            \langle\X,\X\rangle=\langle R_C\X,R_C\X\rangle
            \end{equation}
We see that $\R_C$ is an orthogonal operator in the $2$-dimensional vector space $T_{\pt}M$.
We know that orthogonal operator preserving orientation is the operator of rotation on some angle $\Phi$.

We see that if $\R_C$ preserves orientation\footnote{We consider the case if the operator $R_C$ preserves orientation.
In our considerations we consider only the case if the closed curve $C$ is a boundary of
a compact oriented domain $D\subset M$. In this case one can see that operator $R_C$ preserves an orientation.}
 then the action of operator $\R_C$ on vectors is rotation on the angle,
i.e. the result of parallel transport along closed curve is rotation on the angle
                               \begin{equation}\label{rotationontheangle}
                               \Delta\Phi=\Delta\Phi(C)
                               \end{equation}
 which depends on the curve.


The very beautiful question arises:  How to calculate this angle $\Delta\Phi(C)$


\m

{\bf Theorem}
{\it  Let $M$ be a surface in  Euclidean space $\E^3$.
  Let $C$ be a closed curve $C$ on $M$ such that $C$ is a boundary of a compact oriented domain $D\subset M$.
Consider the parallel transport of an arbitrary tangent vector along the closed curve $C$.
As a result of parallel transport along this closed curve any  tangent vector rotates through the angle

\begin{equation}\label{theoremofrotationonangle}
\Delta\Phi=\angle\left({\X, \R_C\X}\right)=\int_D K d\sigma\,,
             \end{equation}
where $K$ is the Gaussian curvature and $d\sigma=\sqrt {\det g}dudv$ is the area element induced by the
Riemannian metric on the surface $M$, i.e.  $d\sigma=\sqrt {\det g}dudv$.



}

\m
{\bf Remark} One can show that the angle of rotation does not depend on initial point of the curve.


\m


{\bf Example} Consider the closed curve, "latitude" $C_{\theta_0}\theta=\theta_0$ on the sphere of the radius $R$.
Calculations show that
               \begin{equation}\label{rotationforlatitude}
                \Delta\Phi(C_{\theta_0})=2\pi(1-\cos\theta_0)
               \end{equation}

(see the Homework 6). On the other hand the latitude $C_{\theta_0}$ is the boundary of the segment  $D$
 with area $2\pi RH$ where $H=R(1-\cos \theta_0)$. Hence
           $$
   \angle\left({\X, \R_C\X}\right)={2\pi RH\over R^2}={1\over R^2}\cdot \hbox{area of the segment}=\int_D  Kd\sigma
           $$
since Gaussian curvature is equal to $1\over R^2$

The  proof of this Theorem is the content of the subsections above.

Before show one of the remarkable corollaries of this Theorem.

Let  $D$ be a small domain around  a given point $\pt$, let $C$ its boundary and
$\Delta \Phi(D)$ the angle of rotation.   Denote by $S(D)$ an area of this domain.
Applying the Theorem for the case when area of the domain $D$ tends to zero we  we come to the statement that
            $$
      \hbox{if $S(D)\to 0$ then \,} \Delta\Phi(D)=\int_D Kd\sigma\to K(\pt)S(D),\,{\rm i.e.}
            $$

\begin{equation}\label{egregium1}
K(\pt)=\lim_{S(D)\to 0}{\Delta\Phi(D)\over S(D)},\,{\rm i.e.}
\end{equation}

Now notice that  left hand side od this equation defining Gaussian curvature $K(\pt)$ depends only on Riemannian
metric on the surface $C$. Indeed numerator of LHS is defined by the solution of differential equation
$\eqref{paraltransportalongclosedcurve3}$ which depends on Levi-Civita connection depending on
the induced Riemannian metric, and denominator is an area depending on Riemannian metric too.

Thus we come to very important

{\bf Corollary} {\it Gau\ss \,\, Egregium Theorema}

Gaussian curvature of the surface is invariant of isometries.


\bigskip


In next subsections we develop the technique which  itself is very interesting. One of the applications of this
technique is the proof of the Theorem \eqref{theoremofrotationonangle}.
This we will prove Theorema Egregium too.

Later in the fifth section we will give another proof of the Theorema  Egregium.

\subsubsection{ Derivation formulae}

Let $M$ be a surface embedded in  Euclidean space $\E^3$,
$M\colon \r=\r(u,v)$.

   Let ${\e,\f,\n}$ be three vector fields defined on the points of this surface
   such that they form an orthonormal basis at any point, so that the vectors $\e,\f$ are tangent to the surface
   and  the vector $\n$ is orthogonal to the surface\footnote{
   One can say that $\{\e,\f,\n\}$ is an orthonormal basis in $T_{\pt}\E^3$ at every point of surface  $\pt\in M$
such that $\{\e,\f\}$ is an orthonormal basis in $T_{\pt}\E^3$ at every point of surface  $\pt\in M$.}.
   Vector fields $\e,\f,\n$ are functions on the surface $M$:
                   $$
          \e=\e(u,v),\,\, \f=\f(u,v)\,, \n=\n(u,v)\,.
                   $$
 Consider $1$-forms $d\e,d\f,d\n$:
              $$
       d\e={\p \e\over \p u}du+{\p \e\over \p v}dv,\,
       d\f={\p \f\over \p u}du+{\p \f\over \p v}dv\,,
       d\n={\p \n\over \p u}du+{\p \n\over \p v}dv
              $$
 These $1$-forms take values in the vectors in $\E^3$, i.e. they are {\it vector valued} $1$-forms.
 Any vector in $\E^3$ attached at an arbitrary point of the surface
 can be expanded over the basis  $\{\e,\f,\n\}$. Thus vector valued $1$-forms $d\e,d\f,d\n$ can be expanded
 in a sum of $1$-forms with values in basic vectors  $\e,\f,\n$. E.g.
 for $d\e={\p \e\over \p u}du+{\p \e\over \p v}dv$ expanding   vectors
 ${\p \e\over \p u}$ and ${\p \e\over \p v}$ over basis vectors we come to
            $$
      {\p \e\over \p u}=A_1(u,v)\e+B_1(u,v)\f+C_1(u,v)\n,\,\,
      {\p \e\over \p v}=A_2(u,v)\e+B_2(u,v)\f+C_2(u,v)\n
            $$
 thus
          $$
   d\e={\p \e\over \p u}du+{\p \e\over \p v}dv=\left(A_1\e+B_1\f+C_1\n\right)du+\left(A_2\e+B_2\f+C_2\n\right)dv=
          $$
           \begin{equation}\label{one-forms4}
 =\underbrace{(A_1du+A_2dv)}_{M_{11}}\e+
 \underbrace{(B_1du+B_2dv)}_{M_{12}}\f+\underbrace{(C_1du+C_2dv)}_{M_{11}}\e,
           \end{equation}
 i.e.
            $$
       d\e=M_{11}\e+M_{12}\f+M_{13}\n,
            $$
where $M_{11}, M_{12}$ and $M_{13}$ are $1$-forms on the surface $M$ defined by the relation \eqref{one-forms4}.

In the same way we do the expansions of vector-valued $1$-forms $d\f$ and $d\n$ we come to
                 \begin{equation*}
                \begin{matrix}
                d\e=M_{11}\e+M_{12}\f+M_{13}\n\cr
                d\f=M_{21}\e+M_{22}\f+M_{23}\n\cr
                d\n=M_{31}\e+M_{32}\f+M_{33}\n\cr
                \end{matrix}
                    \end{equation*}
   This equation can be rewritten in the following way:
                  \begin{equation}\label{matrixequation1}
                       d
                  \begin{pmatrix}
                \e\cr
                \f\cr
                \n\cr
                \end{pmatrix}
                    =
                \begin{pmatrix}
                M_{11} &M_{12} &M_{13}\cr
                M_{21} &M_{22} &M_{23}\cr
                M_{31} &M_{32} &M_{33}\cr
                \end{pmatrix}
                      \begin{pmatrix}
                \e\cr
                \f\cr
                \n\cr
                \end{pmatrix}
                    \end{equation}

       \m

  {\it Proposition} {\it The matrix $M$ in the equation \eqref{matrixequation1} is antisymmetrical matrix, i.e.
                     \begin{equation}
                       \begin{matrix}
                    M_{11}=M_{22}=M_{33}=0\cr
                       M_{12}=-M_{21}=a\cr
                       M_{13}=-M_{31}= b\cr
                       M_{23}=-M_{32}=-b\cr
                       \end{matrix}
                       \end{equation}
i.e.



                      \begin{equation}\label{derivationformulae1}
                    d\begin{pmatrix}
                    \e\cr\f\cr\n\cr
                    \end{pmatrix}=
                    \begin{pmatrix}
                    0&a&b\cr -a&0&c\cr -b&-c&0\cr
                    \end{pmatrix}
                 \begin{pmatrix}
                    \e\cr\f\cr\n\cr
                    \end{pmatrix}\,,
                       \end{equation}
 where $a,b,c$ are $1$-forms on the surface $M$.}


 \m
Formulae \eqref{derivationformulae1} are called {\it derivation formulae}.


 Prove this Proposition. Recall that $\{\e,\f,\n\}$ is orthonormal basis, i.e. at every point of the surface
                  $$
  \langle\e,\e\rangle=\langle\f,\f\rangle=\langle\n,\n\rangle=1, \,{\rm and}\,\,
   \langle\e,\f\rangle=\langle\e,\n\rangle=\langle\f,\n\rangle=0
         $$
 Now using \eqref{matrixequation1} we have
         $$
  \langle\e,\e\rangle=1\Rightarrow d\langle\e,\e\rangle=0=2\langle\e,d\e\rangle
  =\langle\e,M_{11}\e+M_{12}\f+M_{13}\n\rangle=
               $$
               $$
               M_{11}\langle\e,\e\rangle+M_{12}\langle\e,\f\rangle+
  M_{13}\langle\e,\n\rangle=M_{11}\Rightarrow M_{11}=0\,.
              $$
 Analogously
           $$
     \langle\f,\f\rangle=1\Rightarrow d\langle\f,\f\rangle=0=2\langle\f,d\f\rangle
  =\langle\f,M_{21}\e+M_{22}\f+M_{23}\n\rangle=M_{22}\Rightarrow M_{22}=0\,,
           $$
            $$
             \langle\n,\n\rangle=1\Rightarrow d\langle\n,\n\rangle=0=2\langle\n,d\n\rangle
  =\langle\n,M_{31}\e+M_{32}\f+M_{33}\n\rangle=M_{33}. \Rightarrow M_{33}=0\,.
            $$
 We proved already that $M_{11}=M_{22}=M_{33}=0$. Now prove that
 $M_{12}=-M_{21}$, $M_{13}=-M_{31}$ and $M_{13}=-M_{31}$.
                       $$
  \langle\e,\f\rangle=0\Rightarrow d\langle\e,\f\rangle=0=\langle\e,d\f\rangle+\langle d\e,\f\rangle=
             $$
             $$
  \langle\e,M_{21}\e+M_{22}\f+M_{23}\n\rangle+\langle M_{11}\e+M_{12}\f+M_{13}\n,\f\rangle=
  M_{21}+M_{12}=0.
          $$
 Analogously
       $$
       \langle\e,\n\rangle=0\Rightarrow d\langle\e,\n\rangle=0=\langle\e,d\n\rangle+\langle d\e,\n\rangle=
             $$
             $$
  \langle\e,M_{31}\e+M_{32}\f+M_{33}\n\rangle+\langle M_{11}\e+M_{12}\f+M_{13}\n,\n\rangle=
  M_{31}+M_{13}=0
       $$
 and
          $$
          \langle\f,\n\rangle=0\Rightarrow d\langle\f,\n\rangle=0=\langle\f,d\n\rangle+\langle d\f,\n\rangle=
             $$
             $$
  \langle\f,M_{31}\e+M_{32}\f+M_{33}\n\rangle+\langle M_{21}\e+M_{22}\f+M_{23}\n,\n\rangle=
  M_{32}+M_{23}=0.
          $$

       {\bf Remark} This proof could be performed much more shortly in condensed notations. Derivation formulae
       \eqref{derivationformulae1} in condensed notations are
          \begin{equation}\label{derivationcondensed}
          d\e_i=M_{ik}\e_k
          \end{equation}
     Orthonormality condition means that $\langle\e_i,\e_k\rangle=\delta_{ik}$. Hence
        \begin{equation}
   d\langle\e_i,\e_k\rangle=0=\langle d\e_i,\e_k\rangle+\langle\e_i,d\e_k\rangle=
   \langle  M_{im}\e_m,\e_k\rangle+\langle\e_i,M_{kn}\e_n\rangle=M_{ik}+M_{ki}=0\hbox{\finish}
        \end{equation}
    Much shorter, is not it?


 \subsubsection  {Gauss condition (structure equations)}
    Derive the relations between $1$-forms $a,b$ and $c$ in derivation formulae.

    Recall that $a,b,c$ are $1$-forms, $\e,f,\n$ are vector valued functions ($0$-forms)
    and  $d\e,d\f,d\n$ are vector valued $1$-forms.
    (We use the simple identity that $ddf=0$ and the fact that for $1$-form $\w\wedge \w=0$.)
     We have from derivation formulae \eqref{derivationformulae1} that
         $$
     d^2\e=0= d(a\f+b\n)=da\f-a\wedge d\f+db\n-b\wedge d\n=
        $$
        $$
        da\, \f-a\wedge (-a \e+c\n)+db\,\n-b\wedge(-b\e-c\f)=
                       $$
                       $$
          (da+b\wedge c)\f+(a\wedge a+b\wedge b)\e+(db-a\wedge c)\n=
          (da+b\wedge c)\f+(db+c\wedge a)\n=0\,.
                      $$
We see that
                \begin{equation}\label{gausscondition1}
                    (da+b\wedge c)\f+(db+c\wedge a)\n=0
                \end{equation}
                Hence components of the left hand side equal to zero:
\begin{equation}\label{gausscondition2}
                    (da+b\wedge c)=0\,\,(db+c\wedge a)=0\,.
                \end{equation}
Analogously
              $$
                       d^2\f=0= d(-a\e+c\n)=-da\e+a\wedge d\e+dc\n-c\wedge d\n=
        $$
        $$
        -da\e+a\wedge (a \f+b\n)+dc\,\n-c\wedge(-b\e-c\f)=
                       $$
                       $$
          (-da+c\wedge b)\e+(dc+a\wedge b)\n=0\,.
              $$


 Hence we come to structure equations:

               \begin{equation}\label{firststructureformula}
               \begin{matrix}
                da+b\wedge c=0\cr
                 db+c\wedge a=0\cr
                 dc+a\wedge a=0\cr
                 \end{matrix}
               \end{equation}


          \subsubsection {Geometrical meaning of derivation formulae.
         Weingarten operator and second quadratic form in terms of derivation formulae. }




   Let $M$ be a surface in $\E^3$.

          Let ${\e,\f,\n}$ be three vector fields defined on the points of this surface
   such that they form an orthonormal basis at any point, so that the vectors $\e,\f$ are tangent to the surface
   and  the vector $\n$ is orthogonal to the surface.
   Note that in generally these vectors are not coordinate vectors.

     Describe  Riemannian geometry on the surface
     $M$ in terms of this basis and derivation formulae \eqref{derivationformulae1}.




    \m



    {\it Induced Riemannian metric}

    If $G$ is the Riemannian metric induced on the surface $M$ then since $\e,\f$ is orthonormal basis
    at every tangent space $T_\pt M$ then
        \begin{equation}\label{inducriemmetricinnonholonombasis}
        G(\e,\e)=G(\f,\f)=1,\,\,\,G(\e,\f)=G(\f,\e)=0
        \end{equation}
       The matrix of the Riemannian metric in the basis $\{\e,\f\}$ is
     \begin{equation}\label{inducedriemmetricinnonholonombasis2}
        G=
        \begin{pmatrix}
        1& 0\cr
        0& 1\cr
        \end{pmatrix}
        =
        \begin{pmatrix}
        b(\e)&  b(\f)\cr
        c(\e)& c(\f)\cr
        \end{pmatrix}
     \end{equation}


      \m


    {\it Induced connection}
    Let $\nabla$ be the connection induced by the canonical flat connection on the surface $M$.

    Then according equations \eqref{inducedconnection4} and derivation formulae \eqref{derivationformulae1}
    for every tangent vector $\X$
        \begin{equation}\label{inducedconnectionintermsofnonholonombasis1}
        \nabla_\X \e=\left(\p_\X\e\right)_{\rm tangent}=\left(d\e(\X)\right)_{\rm tangent}=
        \left(a(\X)\f+b(\X)\n\right)_{\rm tangent}=a(\X)\f\,.
        \end{equation}
and
    \begin{equation}\label{inducedconnectionintermsofnonholonombasis2}
        \nabla_\X \f=\left(\p_\X\f\right)_{\rm tangent}=\left(d\f(\X)\right)_{\rm tangent}=
        \left(-a(\X)\e+c(\X)\n\right)_{\rm tangent}=-a(\X)\e\,.
        \end{equation}
 In particular
              \begin{equation}\label{inducedconnectionintermsofnonholonombasis3}
              \begin{matrix}
        \nabla_\e \e=a(\e)\f  &   \nabla_\f \e=a(\f)\f\cr
        \nabla_\e \f=-a(\e)\e  &   \nabla_\f \f=-a(\f)\e\cr
        \end{matrix}
        \end{equation}
We know that the connection $\nabla$ is Levi-Civita connection of the induced Riemannian metric
\eqref{inducedconnectionintermsofnonholonombasis1} (see the subsection 4.2.1)\footnote{In
particular this implies that this is symmetric connection, i.e.
\begin{equation}\label{commutatorofvectorfields}
    \nabla\f\e-\nabla_\e\f-[\f,\e]=a(\f)\f+a(\e)\e-[\f,\e]=0\,.
\end{equation}
}.

\m
{\it Second Quadratic form}
  Let $A(\X,\Y)$ be second quadratic form. Then according to
  to \eqref{secondquadraticform1} and derivation formulae \eqref{derivationformulae1} we have
               $$
  A(\e,\e)=\langle\p_\e\e,\n\rangle=\langle d\e(\e),\n\rangle=\langle a(\e)\f+b(\e)\n,\n\rangle=b(\e)\,,
               $$
         $$
   A(\f,\e)=\langle\p_\f\e,\n\rangle=\langle d\e(\f),\n\rangle=\langle a(\f)\f+b(\f)\n,\n\rangle=b(\f)\,,
         $$
         $$
     A(\e,\f)=\langle\p_\e\f,\n\rangle=\langle d\f(\e),\n\rangle=\langle -a(\e)\f+c(\e)\n,\n\rangle=c(\e)\,,
           $$
           $$
           A(\f,\f)=\langle\p_\f\f,\n\rangle=\langle d\f(\f),\n\rangle=\langle -a(\f)\f+c(\f)\n,\n\rangle=c(\f)\,,
           $$
     The matrix of the second quadratic form in the basis $\{\e,\f\}$ is
     \begin{equation}\label{inducedquadrforminnonholonombasis}
        A=
        \begin{pmatrix}
        A(\e,\e)& A(\f,\e)\cr
        A(\e,\f)& A(\f,\f)\cr
        \end{pmatrix}
        =
        \begin{pmatrix}
        b(\e)&  b(\f)\cr
        c(\e)& c(\f)\cr
        \end{pmatrix}
     \end{equation}
 This is symmetrical matrix (see the subsection 4.3.2):
            \begin{equation}\label{symmericinnonholonom}
 A(\f,\e)=b(\f)=A(\e,\f)=c(\e)\,.
            \end{equation}

\m
{\it Weingarten operator}

Let $S$ be Weingarten operator: $S\X=-\p_\X\n$ (see the subsection 4.3.1). Then
it follows from derivation formulae that
                  $$
     S\X=-\p_\X\n=-d\n(\X)=-\left(-b(X)\e-c(\X)\f\right)=b(\X)\e+c(\X)f
                  $$
In particular
 \begin{equation*}\label{inducedweingartenoperator}
    S(\e)=b(\e)\e+c(\e)\f\,, S(\f)=b(\f)\e+c(\f)\f
\end{equation*}
and the matrix of the Weingarten operator in the basis $\{\e,\f\}$ is
\begin{equation}\label{inducedweingartenoperator}
S=
\begin{pmatrix}
b(\e)  &c(\e)\cr
b(\f)  &c(\f)\cr
\end{pmatrix}
\end{equation}

According to the condition \eqref{symmericinnonholonom}  the matrix $S$ is symmetrical.

The relations $A=GS, S=G^{-1}A$ for Weingarten operator, Riemannian metric and second quadratic form
are evidently obeyed for matrices of these operators in the basis ${\e,\f}$ where $G=1$, $A=S$.

\subsubsection{Gaussian and mean curvature in terms of derivation formulae}
Now we are equipped to express Gaussian and mean curvatures (see the subsection 4.3.3) in terms of derivation formulae.
  Using \eqref{inducedweingartenoperator} we have for Gaussian curvature
           \begin{equation}\label{gaussiancurvintermsofderformulae}
            K=\det S=b(\e)c(\f)-c(\e)b(\f)=(b\wedge c)(\e,\f)
           \end{equation}
 and for mean curvature
   \begin{equation}\label{meancurvintermsofderformulae}
            H={\rm Tr\,} S=b(\e)+c(\f)
           \end{equation}
 What next? We will  study in more detail formula \eqref{gaussiancurvintermsofderformulae} later.

 Now consider some examples of calculation derivation formulae, Weingarten operator, e.t..c. for
 some examples using derivation formulae.

 \subsubsection {Examples of calculations of derivation formulae for cylinder, cone and sphere }

 In the subsection 4.3.4 we calculated Weingarten oeprator, second quadratic form and curvatures
 for cylinder, cone and sphere.
 Now we do the same but in terms of derivation formulae.
\m

 {\it Cylinder}

\m



We have to define three vector fields
${\e,\f,\n}$ on the points of the cylinder $\r(h,\varphi)\colon\quad
  \begin{cases}
  x=R\cos\varphi\\
  y=R\sin\varphi\\
  z=h\\
  \end{cases}$
such that they form an orthonormal basis at any point, so that the vectors $\e,\f$ are tangent to the surface
   and  the vector $\n$ is orthogonal to the surface.

 We already calculated coordinate vector fields $\r_h,\r_\varphi$ and normal unit vector field
$\n$ (see \eqref{surface11} ):
 \begin{equation}\label{surface115}
    \r_h=
  \begin{pmatrix}
        0\\
        0\\
        1\\
   \end{pmatrix}\,,
\quad
  \r_\varphi=\begin{pmatrix}
        -R\sin\varphi\\
        R\cos\varphi\\
          0\\
   \end{pmatrix}\,,\quad
      \n=\begin{pmatrix}
        \cos \varphi\cr
        \sin\varphi\cr
        0\cr
   \end{pmatrix}\,.
\end{equation}

Vectors $\r_h,\r_\varphi$ and $\n$ are orthogonal to each other but not all of them have unit length. One can choose
           \begin{equation}\label{nonholonombasisfor cylinder}
        \e=\r_h=  \begin{pmatrix}
        0\cr
        0\cr
        1\cr
   \end{pmatrix},
   \f={\r_\varphi\over R}=\begin{pmatrix}
        -\sin\varphi\cr
        \cos\varphi\cr
          0\cr
   \end{pmatrix},\,\n=\begin{pmatrix}
        \cos \varphi\cr
        \sin\varphi\cr
        0\cr
   \end{pmatrix}
           \end{equation}
   These vectors form an orthonormal basis and $\e,\f$  form an orthonormal basis in tangent space.

   Derive for this basis derivation formulae \eqref{derivationformulae1}. For vector fields $\e,\f,\n$
   in \eqref{nonholonombasisfor cylinder} we have
                   $$
             d\e=0, d\f=d\begin{pmatrix}
        -\sin\varphi\cr
        \cos\varphi\cr
          0\cr
   \end{pmatrix}=\begin{pmatrix}
        -\cos\varphi\cr
        -\sin\varphi\cr
          0\cr
   \end{pmatrix}d\varphi=-\n d\varphi,
          $$
          $$
   d\n=d\begin{pmatrix}
        \cos \varphi\cr
        \sin\varphi\cr
        0\cr
   \end{pmatrix}=
   \begin{pmatrix}
        -\sin\varphi \cr
        \cos\varphi\cr
        0\cr
   \end{pmatrix}=\f d\varphi,
                   $$
   i.e.
   \begin{equation}\label{derivationformulaefor cylinder}
                    d\begin{pmatrix}
                    \e\cr\f\cr\n\cr
                    \end{pmatrix}=
                    \begin{pmatrix}
                    0&a&b\cr -a&0&c\cr -b&-c&0\cr
                    \end{pmatrix}
                 \begin{pmatrix}
                    \e\cr\f\cr\n\cr
                    \end{pmatrix}=
                     \begin{pmatrix}
                    0&0&0\cr 0&0&-d\varphi\cr 0&d\varphi&0\cr
                    \end{pmatrix}
                 \begin{pmatrix}
                    \e\cr\f\cr\n\cr
                    \end{pmatrix}\,,
                   \end{equation}
  i.e. in derivation formulae $a=b=0$, $c=-d\varphi$.

  The matrix of Weingarten operator in the basis  $\{\e,\f\}$ is
             $$
     S=
\begin{pmatrix}
b(\e)  &c(\e)\cr
b(\f)  &c(\f)\cr
\end{pmatrix}=
\begin{pmatrix}
0  &-d\varphi(\e)\cr
0  &-d\varphi(\f)\cr
\end{pmatrix}=
\begin{pmatrix}
0  &0\cr
0  &-{1\over R}\cr
\end{pmatrix}
       $$
    According to  \eqref{gaussiancurvintermsofderformulae}  and\eqref{meancurvintermsofderformulae}
                Gaussian curvature $K=b(\e)c(\f)-b(\e)c(\f)=0$ and
                mean curvature
                 $$
    H=b(\e)+c(\f)=-d\varphi(\f)=-d\varphi\left({\r_\varphi\over R}\right)=-{1\over R}
                 $$
  (Compare with calculations in the subsection 4.3.4)


\bigskip

{\it Cone}


For  cone:
          $$
  \r(h,\varphi)\colon\quad
  \begin{cases}
  x=kh\cos\varphi\\
  y=kh\sin\varphi\\
  z=h\\
  \end{cases}\,   \,,
   $$
        $$
        \r_h=
  \begin{pmatrix}
        k\cos\varphi\cr
        k\sin\varphi\cr
        1\cr
   \end{pmatrix},\quad
  \r_\varphi=\begin{pmatrix}
        -kh\sin\varphi\cr
        kh\cos\varphi\cr
          0\cr
   \end{pmatrix}\,,\quad
   \n=
   {1\over {\sqrt {1+k^2}}}
      \begin{pmatrix}
        \cos \varphi\cr
        \sin\varphi\cr
        -k\cr
   \end{pmatrix}
            $$
Tangent vectors  $\r_h,\r_\varphi$ are orthogonal to each other.
The length of the vector $\r_h$ equals to $\sqrt {1+k^2}$ and  the length of the vector $\r_\varphi$
equals to $kh$.
Hence we can choose orthonormal basis $\{\e,\f,\n\}$ such that vectors $\e,\f$ are unit vectors in the
directions of the vectors  $\r_h,\r_\varphi$:
               $$
               \e= {\r_h\over \sqrt {1+k^2}}=
               {1\over \sqrt {1+k^2}}
  \begin{pmatrix}
        k\cos\varphi\cr
        k\sin\varphi\cr
        1\cr
   \end{pmatrix},\,
  \f={\r_\varphi\over hk}=\begin{pmatrix}
        -\sin\varphi\cr
        \cos\varphi\cr
          0\cr
   \end{pmatrix},\,
   \n=
   {1\over {\sqrt {1+k^2}}}
      \begin{pmatrix}
        \cos \varphi\cr
        \sin\varphi\cr
        -k\cr
   \end{pmatrix}
               $$
Calculate $d\e,d\f$ and $d\n$:
           $$
      d\e=d\begin{pmatrix}
        k\cos\varphi\cr
        k\sin\varphi\cr
        1\cr
   \end{pmatrix}={kd\varphi\over \sqrt{1+k^2}}\begin{pmatrix}
        -\sin\varphi\cr
        \cos\varphi\cr
        0\cr
   \end{pmatrix}={ kd\varphi\over \sqrt{1+k^2}}\f,
           $$
           $$
   d\f=d\begin{pmatrix}
        -\sin\varphi\cr
        \cos\varphi\cr
          0\cr
   \end{pmatrix}=
     \begin{pmatrix}
        -\cos\varphi\cr
        -\sin\varphi\cr
          0\cr
   \end{pmatrix}d\varphi=
        $$
        $$
   {-k\over 1+k^2}\begin{pmatrix}
        k\cos\varphi\cr
        k\sin\varphi\cr
        1\cr
   \end{pmatrix}d\varphi-{d\varphi\over 1+k^2}
             \begin{pmatrix}
        \cos \varphi\cr
        \sin\varphi\cr
        -k\cr
   \end{pmatrix}={-kd\varphi\over \sqrt{1+k^2}}\e-{d\varphi\over \sqrt {1+k^2}}\n\,,
     $$
and
       $$
   d\n={1\over \sqrt {1+k^2}}d\begin{pmatrix}
        \cos \varphi\cr
        \sin\varphi\cr
        -k\cr
   \end{pmatrix}=
   {d\varphi\over \sqrt {1+k^2}}
   \begin{pmatrix}
        -\sin \varphi\cr
        \cos\varphi\cr
        0\cr
   \end{pmatrix}\,.
       $$
We come to
   \begin{equation}\label{derivationformulaefor cone}
                    d\begin{pmatrix}
                    \e\cr\f\cr\n\cr
                    \end{pmatrix}=
                    \begin{pmatrix}
                    0&a&b\cr -a&0&c\cr -b&-c&0\cr
                    \end{pmatrix}
                 \begin{pmatrix}
                    \e\cr\f\cr\n\cr
                    \end{pmatrix}=
                     \begin{pmatrix}
                    0&{kd\varphi\over \sqrt {1+k^2}}&0\cr
                     -{kd\varphi\over \sqrt {1+k^2}}&0&{-d\varphi\over \sqrt {1+k^2}}\cr
                      0&{d\varphi\over \sqrt {1+k^2}}&0\cr
                    \end{pmatrix}
                 \begin{pmatrix}
                    \e\cr\f\cr\n\cr
                    \end{pmatrix}\,,
                   \end{equation}
  i.e. in derivation formulae for $1$-forms $a={kd\varphi\over \sqrt {1+k^2}}$, $b=0$ and
  and $c=-{-d\varphi\over \sqrt {1+k^2}}$.

  The matrix of Weingarten operator in the basis  $\{\e,\f\}$ is
             $$
     S=
\begin{pmatrix}
b(\e)  &c(\e)\cr
b(\f)  &c(\f)\cr
\end{pmatrix}=
S=
\begin{pmatrix}
0  &{-d\varphi (\e)\over \sqrt {1+k^2}}\cr
0  &{-d\varphi (\f)\over \sqrt {1+k^2}}\cr
\end{pmatrix}=
\begin{pmatrix}
0  &0\cr
0  &{-1\over kh \sqrt {1+k^2}}\cr
\end{pmatrix}\,.
       $$
since $d\varphi(\f)=d\varphi\left({\r_\varphi\over kh}\right)={1\over kh}d\varphi(\p_\varphi)={1\over kh}$.


According to \eqref{gaussiancurvintermsofderformulae}, \eqref{meancurvintermsofderformulae}
                Gaussian curvature  $$
                K=b(\e)c(\f)-b(\e)c(\f)=0
                $$
and
                mean curvature
                 $$
    H=b(\e)+c(\f)=-d\varphi(\f)=-d\varphi\left({\r_\varphi\over R}\right)=-{1\over R}
                 $$
  (Compare with calculations in the subsection 4.3.4)



\bigskip




{\it Sphere}


\medskip


For sphere
\begin{equation}\label{surfacesphere11}
  \r(\theta,\varphi)\colon\quad
  \begin{cases}
  x=R\sin\theta\cos\varphi\\
  y=R\sin\theta\sin\varphi\\
  z=R\cos\theta\\
  \end{cases}
\end{equation}
    $$
  \r_\theta(\theta,\varphi)={\p \r\over \p \theta}=
  \begin{pmatrix}
        R\cos\theta\cos\varphi\cr
        R\cos\theta\sin\varphi\cr
        -R\sin\theta\cr
   \end{pmatrix}\,,
\quad
  \r_\varphi(\theta,\varphi)={\p \r\over \p \varphi}=
         \begin{pmatrix}
        -R\sin\theta\sin\varphi\cr
        R\sin\theta\cos\varphi\cr
          0\cr
   \end{pmatrix},
     $$
     $$
   \n(\theta,\varphi)={\r\over R}=
   \begin{pmatrix}
    \sin\theta\cos\varphi\cr
     \sin\theta\sin\varphi\cr
      \cos\theta\cr
   \end{pmatrix}.
  $$
Tangent vectors  $\r_\theta,\r_\varphi$ are orthogonal to each other.
The length of the vector $\r_\theta$ equals to $R$ and the
length of the vector $\r_\varphi$ equals to $R\sin\theta$.
Hence we can choose orthonormal basis $\{\e,\f,\n\}$ such that vectors $\e,\f$ are unit vectors in the
directions of the vectors  $\r_\theta,\r_\varphi$:
         $$
    \e(\theta,\varphi)={\r_\theta\over R}=
         \begin{pmatrix}
        \cos\theta\cos\varphi\cr
        \cos\theta\sin\varphi\cr
        -\sin\theta\cr
   \end{pmatrix},\,
   \f(\theta,\varphi)={\r_\varphi\over R\sin\theta}=
        \begin{pmatrix}
        -\sin\varphi\cr
        \cos\varphi\cr
          0\cr
   \end{pmatrix},\,
     \n(\theta,\varphi)={\r\over R}=
   \begin{pmatrix}
    \sin\theta\cos\varphi\cr
     \sin\theta\sin\varphi\cr
      \cos\theta\cr
   \end{pmatrix}.
         $$
Calculate $d\e, d\f$ and $d\n$:
            $$
        d\e=d
        \begin{pmatrix}
        \cos\theta\cos\varphi\cr
        \cos\theta\sin\varphi\cr
        -\sin\theta\cr
   \end{pmatrix}=
           $$
           $$
\begin{pmatrix}
        -\cos\theta\sin\varphi\cr
        \cos\theta\cos\varphi\cr
           0\cr
   \end{pmatrix}d\varphi-\begin{pmatrix}
        \cos\theta\cos\varphi\cr
        \cos\theta\sin\varphi\cr
        -\sin\theta\cr
   \end{pmatrix}=\cos\theta d\varphi \f-d\theta \n,
            $$
             $$
            d\f=d
            \begin{pmatrix}
        -\sin\varphi\cr
        \cos\varphi\cr
          0\cr
   \end{pmatrix}=
              -
   \begin{pmatrix}
        \cos\varphi\cr
        \sin\varphi\cr
          0\cr
   \end{pmatrix}d\varphi=
            $$
            $$
       -\cos\theta d\varphi \begin{pmatrix}
        \cos\theta\cos\varphi\cr
        \cos\theta\sin\varphi\cr
        -\sin\theta\cr
   \end{pmatrix}-\sin\theta d\varphi
   \begin{pmatrix}
    \sin\theta\cos\varphi\cr
     \sin\theta\sin\varphi\cr
      \cos\theta\cr
   \end{pmatrix}=
   -\cos\theta d\varphi \e-\sin\theta d\varphi \n\,,
            $$
             $$
             d\n=
   d\begin{pmatrix}
    \sin\theta\cos\varphi\cr
     \sin\theta\sin\varphi\cr
      \cos\theta\cr
   \end{pmatrix}=
   \begin{pmatrix}
    \cos\theta\cos\varphi\cr
     \cos\theta\sin\varphi\cr
      -\sin\theta\cr
   \end{pmatrix}d\theta+
   \begin{pmatrix}
    -\sin\theta\sin\varphi\cr
     \sin\theta\cos\varphi\cr
        0\cr
   \end{pmatrix}d\varphi
        $$
        $$
   =d\theta \e+\sin\theta d\varphi \f\,.
        $$
   i.e.
   \begin{equation}\label{derivationformulaefor cylinder}
                    d\begin{pmatrix}
                    \e\cr\f\cr\n\cr
                    \end{pmatrix}=
                    \begin{pmatrix}
                    0&a&b\cr -a&0&c\cr -b&-c&0\cr
                    \end{pmatrix}
                 \begin{pmatrix}
                    \e\cr\f\cr\n\cr
                    \end{pmatrix}=
                     \begin{pmatrix}
                    0&\cos\theta d\varphi& -d\theta\cr
                    -\cos\theta d\varphi&0&-\sin\theta d\varphi\cr
                     d\theta&\sin\theta d\varphi&0\cr
                    \end{pmatrix}
                 \begin{pmatrix}
                    \e\cr\f\cr\n\cr
                    \end{pmatrix}\,,
                   \end{equation}
  i.e. in derivation formulae $a=\cos\theta d\varphi$,
  $b=-d\theta$, $c=-\sin\theta d\varphi$.

  The matrix of Weingarten operator in the basis  $\{\e,\f\}$ is
             $$
     S=
\begin{pmatrix}
b(\e)  &c(\e)\cr
b(\f)  &c(\f)\cr
\end{pmatrix}=
\begin{pmatrix}
-d\theta (\e) &-\sin\theta d\varphi (\e)\cr
-d\theta (\f)  &-\sin\theta d\varphi(\f)\cr
\end{pmatrix}=
\begin{pmatrix}
{-1\over R}  &0\cr
0  &-{1\over R}\cr
\end{pmatrix}
       $$
       since $d\theta(\e)=d\theta\left({\p_\theta\over R}\right)={1\over R}d\theta(\p_\theta)={1\over R}$,
       $\,d\varphi(\e)=d\varphi\left({\p_\theta\over R}\right)={1\over R}d\varphi(\p_\theta)=0$.

    According to  \eqref{gaussiancurvintermsofderformulae}  and\eqref{meancurvintermsofderformulae}
                Gaussian curvature $$
                K=b(\e)c(\f)-b(\e)c(\f)={1\over R^2}
                                    $$
                                    and
                mean curvature
                 $$
    H=b(\e)+c(\f)=-{2\over R}
                 $$
    Notice that for calculation of Weingarten operator and curvatures we used only
    $1$-forms $b$ and $c$, i.e. the derivation equation for $d\n$, ($d\n=d\theta \e+\sin\theta d\varphi \f$).
  (Compare with calculations in the subsection 4.3.4)





 Mean curvature is define up to a sign. If we change $\n\to-\n$ mean curvature $H\to {1\over R}$ and Gaussian curvature
will not change.


We see that for the sphere  Gaussian curvature is not equal to zero, whilst for cylinder and cone Gaussian curvature
equals to zero.


\subsubsection { Proof of the Theorem of parallel transport along closed curve.}

We are ready now to prove the Theorem.  Recall that the Theorem states following:

If $C$ is a closed curve on a surface $M$ such that $C$ is a boundary of a compact oriented domain $D\subset M$,
then during the  parallel transport of an arbitrary tangent vector  along the closed curve $C$
the vector rotates through the angle
\begin{equation}\label{theoremofrotationonangle4}
\Delta\Phi=\angle\left({\X, \R_C\X}\right)=\int_D K d\sigma\,,
             \end{equation}
where $K$ is the Gaussian curvature and $d\sigma=\sqrt {\det g}dudv$ is the area element induced by the
Riemannian metric on the surface $M$, i.e.  $d\sigma=\sqrt {\det g}dudv$.

(see \eqref{theoremofrotationonangle}.


Recall that for derivation formulae
\eqref{derivationformulae1} we obtained structure equations
                  \begin{equation}\label{firststructureformula4}
               \begin{matrix}
                da+b\wedge c=0\cr
                 db+c\wedge a=0\cr
                 dc+a\wedge a=0\cr
                 \end{matrix}
               \end{equation}

We need to use only one of these equations, the equation
             \begin{equation}\label{gausscondition}
                da+b\wedge c=0\,.
             \end{equation}
This condition sometimes is called {\it Gau\ss \,\,condition}.



Let as always  $\{\e,\f,\n\}$ be an orthonormal basis in $T_{\pt}\E^3$ at every point of surface  $\pt\in M$
such that $\{\e,\f\}$ is an orthonormal basis in $T_{\pt}\E^3$ at every point of surface  $\pt\in M$.
Then the  Gau\ss\, condition \eqref{gausscondition} and equation \eqref{gaussiancurvintermsofderformulae} mean
 that  for Gaussian curvature on the surface $M$ can be expressed through the $2$-form $da$ and base vectors $\{\e,\f\}$:
              \begin{equation}\label{gaussiancurvatureintermsof form a}
                K=b\wedge c(\e,\f)=-da(\e,\f)
              \end{equation}
  We use this formula to prove the Theorem.



             Now calculate the parallel transport of an arbitrary tangent vector over the closed curve $C$
             on the surface $M$.



 Let $\r=\r(u,v)=\r(u\a)$ ($\a=1,2$, $(u,v)=(u^1,v^1)$) be an equation of the surface $M$.

Let $u^\a=u^\a(t)$ ($\a=1,2$) be the equation of the curve $C$.
Let $\X(t)$ be the parallel transport of vector field along the closed curve $C$,
i.e. $\X(t)$ is tangent to the surface $M$ at the point $u(t)$ of the curve $C$ and
vector field $\X(t)$ is covariantly constant along the curve:
      $$
    {\nabla \X(t)\over dt}=0
      $$
     To write this equation in components we usually  expanded the vector field
in the coordinate basis $\{\r_u=\p_u,\r_v=\p_v\}$ and used Christoffel symbols of the connection
  $\Gamma^\a_{\beta\gamma}\colon \nabla_\beta\p_\gamma=\Gamma^\a_{\beta\gamma}\p_\a$.

  Now we will do it in different way: {\it instead coordinate basis $\{\r_u=\p_u,\r_v=\p_v\}$ we will use
  the basis $\{\e,\f\}$.}   In the subsection 3.4.4 we obtained that the connection $\nabla$ has the following appearance
  in this basis
  \begin{equation}\label{indconforrotation}
    \nabla_\v\e=a(\v)\f,\,, \nabla_\v\f=-a(\v)\e
  \end{equation}
  (see the equations \eqref{inducedconnectionintermsofnonholonombasis1} and \eqref{inducedconnectionintermsofnonholonombasis2})

Let        $$
\X=\X(u(t))=X^1(t)\e(u(t))+X^2(t)\f(u(t))
           $$
Lbe an expansion of tangent vector field $\X(t)$ over basis $\{\e,\f\}$.
Let $\v$ be velocity vector of the curve $C$.
  Then the equation of parallel transport ${\nabla \X(t)\over dt}=0$
                 will have the following appearance:
                      $$
              {\nabla \X(t)\over dt}=0=\nabla_\v \left(X^1(t)\e(u(t))+X^2(t)\f(u(t))\right)=
                      $$
                      $$
                      {dX^1(t)\over dt}\e(u(t))+X^1(t)\nabla_\v\e(u(t))+
                      {dX^2(t)\over dt}\f(u(t))+X^2(t)\nabla_\v\f(u(t))=
                      $$
                      $$
                      {dX^1(t)\over dt}\e(u(t))+X^1(t)a(\v)\f(u(t))+
                      {dX^2(t)\over dt}\f(u(t))-X^2(t)a(\v)\e(u(t))=
                      $$
              $$
          \left({dX^1(t)\over dt}-X^2(t)a(\v)\right)\e(u(t))+
           \left({dX^2(t)\over dt}+X^1(t)a(\v)\right)\f(u(t))=0.
              $$
Thus we come to equation:
              $$
             \begin{cases}
             \dot X^1(t)-a(\v(t))X^2=0\cr
             \dot X^2(t)+a(\v(t))X^1=0\cr
             \end{cases}
              $$
There are many ways to solve this equation. It is very convenient to consider complex variable
             $$
           Z(t)=X^1(t)+iX^2(t)
             $$
We see that
             $$
         \dot Z(t)=\dot X^1(t)+i\dot X^2(t)=a(\v(t))X^2-ia(\v(t)X^1=-ia(\v)Z(t),
                 $$
i.e.
   \begin{equation}\label{complexrotation}
    {dZ(t)\over dt}=-ia(\v(t))Z(t)
   \end{equation}
The solution of this equation is:
           \begin{equation}\label{complexrotation}
      Z(t)=Z(0)e^{-i\int_0^t a(\v(\tau))d\tau}
   \end{equation}
   Calculate $\int_0^{t_1} a(\v(\tau))d\tau$ for closed curve $u(0)=u(t_1)$. Due to Stokes Theorem:
            $$
            \int_0^{t_1} a(\v(t))dt=\int_C a=\int_D da
            $$
   Hence using Gauss condition \eqref{gausscondition} we see that
      $$
      \int_0^{t_1} a(\v(t))dt=\int_C a=\int_D da=-\int_D b\wedge c
      $$

{\bf Claim}
            \begin{equation}\label{claimoncurvature}
            \int_D b\wedge c=-\int_D da=\int K d\sigma\,.
            \end{equation}

Theorem follows from this claim:
           \begin{equation}\label{complexrotation2}
      Z(t_1)=Z(0)e^{-i\int_C a}=Z(0)e^{i\int_D b\wedge C}
   \end{equation}
Denote the integral ${i\int_D b\wedge C}$ by $\Delta \Phi\colon \Delta \Phi={i\int_D b\wedge C}$. We have
           \begin{equation}\label{complexrotation2}
      Z(t_1)=X^1(t_1)+iX^2(t_1)=\left(X^1(0)+iX^2(0)\right)e^{i\Delta\Phi}=
   \end{equation}

\medskip

 It remains to prove the claim.  The induced volume form $d\sigma$ is $2$-form. Its value
 on two orthogonal unit vector $\e,\f$ equals to $1$:
              \begin{equation}\label{valueonorthonormalvectors}
                d\sigma (\e,\f)=1
              \end{equation}
(In coordinates $u,v$ volume form $d\sigma=\sqrt{\det g}du\wedge dv$).

The value of the form $b\wedge c$ on vectors $\{\e,\f\}$ equals to Gaussian curvature according to \eqref{gaussiancurvatureintermsof form a}.
We see that
             $$
        b\wedge c(\e,\f)=-da(\e,\f)=K d\sigma (\e,\f)
             $$
Hence $2$-forms $b\wedge c$, $-da$ and volume form $d\sigma$ coincide. Thus we prove \eqref{claimoncurvature}.


\section {Curvtature tensor}

\subsection {Definition}

Let $\X,\Y,\Z$ be an arbitrary vector fields on the manifold equipped with affine connection  $\nabla$.

Consider the following operation which assings to the vector fields $\X,\Y$ and $\Z$ the new vector field.
\begin{equation}\label{operationdefinig the tensor}
   {\cal R}(\X,\Y)\Z=
    \left(
    \nabla_\X\nabla_\Y-\nabla_\Y\nabla_\X-
   \nabla_{[\X,\Y]}
    \right)\Z
\end{equation}
This operation is obviously linear over the scalar coefficients and it is $C^{\infty}(M)$
with respect to vector fields $\X,\Y\,\Z$, i.e. for an arbitrary functions
$f,g,h$
        \begin{equation}\label{propertiesoflienarity}
            {\cal R}(f\X,g\Y)(h\Z)=fgh{\cal R}(\X,\Y)\Z,
        \end{equation}
        i.e. it defines
the tensor field of the type $\begin{pmatrix}1\cr 3\end{pmatrix}$: If $\X=X^i\p_i, \X=X^i\p_i,\X=X^i\p_i$
then according to \eqref{propertiesoflienarity}
           $$
        {\cal R}(\X,\Y)\Z={\cal R}(X^m\p_m,Y^n\p_n)(Z^r\p_r)=Z^rR^i_{rmn}X^mY^n
           $$
where we denote by $R^i_{rmn}$ the components of the tensor $\cal R$ in the coordinate basis ${\p_i}$
\begin{equation}\label{componentsofcurvaturetensor}
    R^i_{rmn}\p_i={\cal R}(\p_m,\p_n)\p_r
\end{equation}
Express components of the curvature tensor in terms of Christoffel symbols of the connection.
If $\nabla_m\p_n=\Gamma_{mn}^r\p_r$ then according to the \eqref{operationdefinig the tensor} we have:
                $$
        R^i_{rmn}\p_i={\cal R}(\p_m,\p_n)\p_r=\nabla_{\p_m}\nabla_{\p_n}\p_r-\nabla_{\p_n}\nabla_{\p_m}\p_r=
                    $$
                    $$
              \nabla_{\p_m}\left(\Gamma_{nr}^p\p_p\right)-\nabla_{\p_n}\left(\Gamma_{mr}^p\p_p\right)=
                $$
                \begin{equation}\label{curvatureincomponents}
                \p_m\Gamma_{nr}^i+\Gamma_{mp}^i\Gamma_{nr}^p-\left(m\leftrightarrow n\right)
                \end{equation}

 The proof of the property \eqref{propertiesoflienarity} can be given just  by straightforward calculations:
     Consider e.g. the case $f=g=1$, then

        $$
         {\cal R}(\X,\Y)(h\Z)=
    \nabla_{\X}\nabla_\Y(h\Z)-\nabla_{\Y}\nabla_\X(h\Z)-
   \nabla_{[\X,\Y]}(h\Z)=
        $$
        $$
     \nabla_\X\left(\p_\Y h\Z+h\nabla_\Y\Z\right)-\nabla_\Y\left(\p_\X h\Z+h\nabla_\X\Z\right)-
     \p_{[\X,\Y]}h \Z-h\nabla_{[\X,\Y]}\Z=
        $$
        $$
    \p_\X\p_\Y h\Z+\p_\Y h\nabla_\X\Z+
    \p_\X h\nabla_\Y\Z+
    h\nabla_\X\nabla_\Y \Z-
        $$
        $$
        \p_\Y\p_\X h\Z-\p_\X h\nabla_\Y\Z-
    \p_\Y h\nabla_\X\Z+
    h\nabla_\Y\nabla_\X \Z-
        $$
        $$
      \p_{[\X,\Y]}h \Z-h\nabla_{[\X,\Y]}\Z=
        $$
        $$
    h\left[\nabla_{\X}\nabla_\Y\Z-\nabla_{\Y}\nabla_\X\Z)-
   \nabla_{[\X,\Y]}\Z\right]+\left[\p_\X\p_\Y h-\p_\Y\p_\X h-\p_{[\X,\Y]h}\right]\Z=
        $$
        $$
h\nabla_{\X}\nabla_\Y\Z-\nabla_{\Y}\nabla_\X\Z)-
   \nabla_{[\X,\Y]}\Z=h{\cal R}(\X,\Y)\Z\,,
        $$
   since $\p_\X\p_\Y h-\p_\Y\p_\X h-\p_{[\X,\Y]}h=0$.




\subsection {Curvature of surfaces in $\E^3$.. {\it Theorema Egregium} egain}

  Express Riemannian curvature of surfaces in $\E^3$ in terms of derivation formulae \eqref{derivationformulae1}.

   Consider derivation formulae for the orthonormal basis $\{\e,\f,\n,\}$ adjusted to the surface $M$:
                      \begin{equation}\label{derivationformulae1}
                    d\begin{pmatrix}
                    \e\cr\f\cr\n\cr
                    \end{pmatrix}=
                    \begin{pmatrix}
                    0&a&b\cr -a&0&c\cr -b&-c&0\cr
                    \end{pmatrix}
                 \begin{pmatrix}
                    \e\cr\f\cr\n\cr
                    \end{pmatrix}\,,
                       \end{equation}

where as usual $\e,\f,\n$ vector fields of unit length which are  orthogonal
to each other and $\n$ is orthogonal to the surface $M$.  As we know the induced
connection on the surface $M$ is defined by the formulae \eqref{inducedconnectionintermsofnonholonombasis1}
and \eqref{inducedconnectionintermsofnonholonombasis2}:
   \begin{equation}\label{inducedconnectionintermsofnonholonombasis11}
        \nabla_\Y \e=\left(d\e(\Y)\right)_{\rm tangent}=a(\X)\f\,,
        \nabla_\Y \f=\left(d\f(\Y)\right)_{\rm tangent}=-a(\X)\e\,,
        \end{equation}

 According to the definition of curvature calculate
         $$
 R(\e,\f)\e=\nabla_{\e}\nabla_{\f}\e-\nabla_{\f}\nabla_{\e}\e-\nabla_{[\e,\f]}\e\,.
         $$
Note that since the induced connection is symmetrical connection then:
        \begin{equation}\label{commutatorofnonholonombasis}
 \nabla_\e \f-\nabla_\f \e- [\e,\f]=0\,.
\end{equation}
hence due to \eqref{inducedconnectionintermsofnonholonombasis11}

       \begin{equation}\label{commutatorofnonholonombasis}
 [\e,\f]=\nabla_\e \f-\nabla_\f \e=a(\e)\e+a(\f)\f
\end{equation}
Thus we see that  $R(\e,\f)\e=$
       $$
 \nabla_{\e}\nabla_{\f}\e-\nabla_{\f}\nabla_{\e}\e-\nabla_{[\e,\f]}\e=
       \nabla_\e\left(a(\f)\f\right)-\nabla_\f\left(a(\e)\f\right)-\nabla_{a(\e)\e+a(\f)\f}\e=
       $$
       $$
  \p_{\e}a(\f)\f+a(\f)\nabla_\e\f-\p_{\f}a(\e)\f-a(\e)\nabla_\f\f-a(\e)\nabla_\e\e-a(\f)\nabla_\f\e=
       $$
       $$
  \p_{\e}a(\f)\f-a(\f)a(\e)\e-\p_{\f}a(\e)\f+a(\e)a(\f)\e-a(\e)a(\e)\f-a(\f)a(\f)\f=
       $$
       $$
       \left[\p_{\e}a(\f)\f-\p_{\f}a(\e)\f-a\left((\e)\e-a(\f)\f\right)\right]\f=
       da(\e,\f)\f\,.
       $$
  Recall that we established in \ref{gaussiancurvatureintermsof form a} that for Gaussian curvature $K$
          $$
       K=b\wedge c(\e,\f)=-da(\e,\f)
          $$
       Hence we come to the relation:
       \begin{equation}\label{relationbetweengausscurvatureand riemtensor}
        R(\e,\f)\e=da(\e,\f)=-K\f\,.
       \end{equation}
       This means that
       $$
         R^2_{112}=-K
       $$
(in the basis ${\e,\f}$), i.e.
 the scalar curvature
   $$
R=2R_{1212}=2K
   $$
 We come to fundamental relation which claims that the Gaussian curvature  (the magnitude defined in terms of
 External observer) equals to the scalar curvature, the magnitude defined in terms of Internal observer.
   This gives us the straightforward proof of Theorema Egregium.





\section {Appendices}


 \subsection {Bianci identities}

Recall that the curvature is defines the linear operator depending
on the pair of vector fields: given vector fields $\X,\Y$ and $\Z$
\begin{equation}\label{operationdefinig the tensor1A}
   {\cal R}(\X,\Y)\Z=
    \left(
    \nabla_\X\nabla_\Y-\nabla_\Y\nabla_\X-
   \nabla_{[\X,\Y]}
    \right)\Z
\end{equation}
(see \eqref{operationdefinig the tensor})


   Consider
    ${\cal R}(\X,\Y)\Z+{\rm cycl.permutation}$:
    \begin{equation}\label{biancifirst}
{\cal R}(\X,\Y)\Z+{\rm
cycl.permutation}=\left(\nabla_\X\nabla_\Y-\nabla_\Y\nabla_\X-
   \nabla_{[\X,\Y]}\right)\Z+
\end{equation}
     $$
\left(\nabla_\Y\nabla_\Z-\nabla_\Z\nabla_\Y-
   \nabla_{[\Y,\Z]}\right)\X+\left(\nabla_\Z\nabla_\X-\nabla_\X\nabla_\Z-
   \nabla_{[\Z,\X]}\right)\Y=
     $$
     $$
     \nabla_\X\left(\nabla_\Y \Z-\nabla_\Z \Y-[\Y,\Z]\right)+
      \nabla_\Y\left(\nabla_\Z\X-\nabla_\X \Z-[\Z,\X]\right)+
       \nabla_\Z\left(\nabla_\X \Y-\nabla_\Y \X-[\X,\Y]\right)+
        $$
        $$
       \nabla_\X[\Y,\Z]+\nabla_\Y[\Z,\X]+\nabla_\Z[\X,\Y]-
       \nabla_{[\X,\Y]}\Z-\nabla_{[\Y,\Z]}\X-\nabla_{[\Z,\X]}\Y=
          $$
       $$
        \begin{matrix}
        \nabla_\X\left(\nabla_\Y \Z-\nabla_\Z \Y-[\Y,\Z]\right)+\cr
      \nabla_\Y\left(\nabla_\Z\X-\nabla_\X \Z-[\Z,\X]\right)+\cr
       \nabla_\Z\left(\nabla_\X \Y-\nabla_\Y \X-[\X,\Y]\right)+\cr
\left(\nabla_\X[\Y,\Z]-\nabla_{[\Y,\Z]\X}-\left[\X,\left[\Y,\Z\right]\right]\right)+\cr
\left(\nabla_\Y[\Z,\X]-\nabla_{[\Z,\X]\Y}-\left[\Y,\left[\Z,\X\right]\right]\right)+\cr
\left(\nabla_\Z[\X,\Y]-\nabla_{[\X,\Y]\Z}-\left[\Z,\left[\X,\Y\right]\right]\right)+\cr
\left[\X,\left[\Y,\Z\right]\right]+\left[\Y,\left[\Z,\X\right]\right]+\left[\Z,\left[\X,\Y\right]\right]=\cr
\end{matrix}
       $$
\begin{equation}\label{torsioncyclicpermutation}
    \left(\nabla_\X\left(T\left(\Y,\Z\right)\right)+{\rm cycl.permutation}\right)+
    \left(T\left(\X,[\Y,\Z]\right)+{\rm cycl.permutation}\right)
\end{equation}

One can deduce that right hand side equals to $\left(\nabla_\X
T\right)(\Y,\Z)+T(T,...)+{\rm cycl.permutation}$

In particular if torsion equals to zero then
     $$
     {\cal R}(\X,\Y)\Z+{\rm cycl.permutation}=0\,.
     $$


 \end{document}
