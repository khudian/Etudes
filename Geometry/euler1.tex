
\magnification=1200 \baselineskip=14pt
\def\vare {\varepsilon}
\def\A {{\bf A}}
\def\t {\tilde}
\def\a {\alpha}
\def\K {{\bf K}}
\def\N {{\bf N}}
\def\V {{\cal V}}
\def\s {{\sigma}}
\def\S {{\Sigma}}
\def\s {{\sigma}}
\def\p{\partial}
\def\vare{{\varepsilon}}
\def\Q {{\bf Q}}
\def\D {{\cal D}}
\def\G {{\Gamma}}
\def\C {{\bf C}}
\def\M {{\cal M}}
\def\Z {{\bf Z}}
\def\U  {{\cal U}}
\def\H {{\cal H}}
\def\R  {{\bf R}}
\def\E  {{\bf E}}
\def\l {\lambda}
\def\degree {{\bf {\rm degree}\,\,}}
\def \finish {${\,\,\vrule height1mm depth2mm width 8pt}$}
\def \m {\medskip}
\def\p {\partial}
\def\r {{\bf r}}
\def\v {{\bf v}}
\def\n {{\bf n}}
\def\t {{\bf t}}
\def\b {{\bf b}}
\def\e{{\bf e}}
\def\f{{\bf f}}
\def\g{{\bf g}}
\def\ac {{\bf a}}
\def \X   {{\bf X}}
\def \Y   {{\bf Y}}
\def \x   {{\bf x}}
\def \y   {{\bf y}}

  \centerline {\bf On one subset in $SO(3)$ and Euler Theorem}

{\it This text was written around 2006.  
Korchagina had a beautiful talk with very facinating 
title ``On the second generation of proof of 
Theorem on finite simple groups''. Sasha Borovik did some
interesting remarks about status of this Theorem and 
about relation between $\R P^3$ 
and $SO(3)$.}

\bigskip
          {\it around 2006       
   $\qquad\qquad\qquad\qquad\qquad$
                         The group $S0(3)$ has
                             fantom memories

       $\qquad\qquad\qquad\qquad\qquad\qquad\qquad\qquad\qquad\qquad$ 
        of lost operations}


{\sl $\qquad\qquad\qquad\qquad\qquad\qquad\qquad\qquad\qquad\qquad$
             (...initiated by remark of Sasha Borovik)}

\bigskip

It is well-well-known that $SO(3)\approx \R P^3$(as manifold) , 
but in fact  one can say much more:
 $S0(3)$ `knows' about structure of projective space.  In particular
the subspace  $\R P^2\subset \R P^3$ can be
canonically embedded  in $SO(3)$. It is the following
subset (not subgroup!) in  $SO(3)$ :
             $$
    L=\{A\colon\,\, A\in SO(3),\, \det(1+A)=0\}\,.
    \eqno (1)
             $$
Geometrically it means following: Operators $A\in L$
are orthogonal transformations
which are rotations around axis
on the angle $\pi$. 
It will be $\R P^2$\footnote{$^{1)}$}{
   Any rotation =axis+angle of rotation=point of the
ball of the radius $\pi$. Rotation on angle $\not=\pi$
is an interior of this ball. The boundary of this ball 
with fatorised antipodal points = $L$ } . 
This is not hard to see. 
  The subset $L$h naturally appears also in
another algebraic proof of Euler Euler Theorem.

 Recall first maybe the most beautiful proof of Euler Theorem (Coxeter Proof).
Let $`A \in SO(3)$, $\{\e,\f,\g\}$ be an orthogonal basis 
and  $\{\e,\f,\g\}$ be a new orthogonal basis,
$\e'=A\e,\f'=A\e, \g'=A\g $.
Consider a reflection $O_1$ which transforms $\e$ to $\e'$; the
invariant plane of this reflection is spanned by vectors $\e+\e'$ and 
$\e\times \e'$. Then consider 
the reflection $O_2$ which  transforms
$\f$ to $\f'$. The vector $\e'$ belongs to invariant plane of this 
reflection. Under composition of these
two reflections $f$ transforms to $f'$ too. Indeed 
in other case we consider third reflection with respect to the
plane $\e',\f'$ but composition of three reflections has determinant $-1$.)
Hence $A=O_2O_1$. 
The intersection of invariant planes of these reflection is an axis
of rotation.\finish

{\it  Now algebraic proof.} Let $A\in SO(3)$.
Note first that if there exists eigenvector $\n$ 
with eigenvalue $1$ then restriction of transformation
$A$ on the plane orthogonal to the vector $\n$ is nothing 
but orthogonal rotation of the plane $\a$.
(Indeed if $\alpha$ is a plane which is orthogonal to
$\n$, then it is obvious that $A$ maps $\alpha$ on $\alpha$,
it is orthogonal operator on $\alpha$ and $\det A\vert_\alpha=1$. ).  
Hence $A$ is rotation around axis $\n$. 
It remains to prove that such an  eigenvector $\n$ exists.
Consider characteristic polynomial
$P(z)=\det(z-A)$. It is cubic polynomial, and it has at least one 
real root. 
Let $\lambda$ be its real root and $\n$ be corresponding eigenvector.
Operator $A$ preserves scalar product. This implies that $\l=\pm 1$.
if $\lambda=1$, then everything is already proved (see above).
If $\lambda=:-1$, then $\det (1+L)=0$, i.e. $A\in L$
\footnote{$^{2)}$}{Of course in this case also one can prove the existence
of eigenvector with eigenvalue $1$, but it is much easier to do
it straightforwardly}. 
Let $\n$ be an eigenvector corresponding to the eigenvalue $\l=-1$: 
$A\n=-\n$. Consider a plane $\a$,
which is orthogonal to the eigenvector $\n$. 
Restriction of
$A$ on the plane $\a$ is orthogonal transformation with determinant $-1$,
i.e. the restriction of $A$ on the plane $\a$ is a reflection of the plane
with respect of a line $l\in\a$ ($A$ on $\a$ has eigenvectors  $\f,\g$
with eigenvalues $1,-1$ respectively, and 
$l$ is directed along vector $\f$.) 
We see that in this special case $A$ is a rotation around axis $\l$
on the angle $\pi$. Subset $L$ is in one-one correspondence with
axis= lines which go over origin, i.e. $L=\R P^2$.
\finish


\bye
