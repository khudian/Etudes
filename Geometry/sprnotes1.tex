\def\vare {\varepsilon}
\def\A {\bf A}
\def\t {\tilde}
\def\a {\alpha}
\def\K {{\bf K}}
\def\N {{\bf N}}
\def\V {{\cal V}}
\def\s {{\sigma}}
\def\S {{\Sigma}}
\def\s {{\sigma}}
\def\p{\partial}
\def\vare{{\varepsilon}}
\def\Q {{\bf Q}}
\def\D {{\cal D}}
\def\G {{\Gamma}}
\def\C {{\bf C}}
\def\M {{\cal M}}
\def\Z {{\bf Z}}
\def\U  {{\cal U}}
\def\H {{\cal H}}
\def\R  {{\bf R}}
\def\E  {{\bf E}}
\def\l {\lambda}
\def\degree {{\bf {\rm degree}\,\,}}
\def \finish {${\,\,\vrule height1mm depth2mm width 8pt}$}
\def \m {\medskip}
\def\p {\partial}
\def\r {{\bf r}}
\def\v {{\bf v}}
\def\n {{\bf n}}
\def\t {{\bf t}}
\def\b {{\bf b}}
\def\ac {{\bf a}}
\def \X   {{\bf X}}
\def \Y   {{\bf Y}}
\def\Tr {{\rm Tr\,}}


\documentclass[12pt]{article}
\usepackage{amsmath,amsthm}

\theoremstyle{theorem}
\newtheorem{thm}{Khimera}




\title{Notes on some basic formulae of Differential Geometry}

\begin{document}

\title{Notes on some basic formulae of Differential Geometry}



\tableofcontents

 \section {}

  In these notes we consider some formulae for hypersurfaces (surfaces of codimension $1$)
   in the fashion very easy generalisable to supercase.

Explain the idea on examples.

 Consider a surface of codimension $1$ in $(n+1)$-dimensional Euclidean space
  with cartesian coordinates $x^0,x^1,\dots, x^n$.
 How to write invariant objects of integration over
 hypersurfaces, without using our knowledge of differential
 geometry (first and second quadratic forms and so on...).
  Let an hypersurface be given by equation $G=0$.
 Consider an expression
  \begin{equation}\label{lengthdual}
     A(\p G)=\sqrt{\p_\mu G\p_\mu G}=\sqrt{G_0^2+\dots+G_n^2}
\end{equation}
   One can see that:
\begin {itemize}

  \item  For function $\tilde G=HG$ which defines the same surface
     \begin{equation}\label{idealdensity1}
  A(\tilde G)\vert_{G=0}=HA(G)\vert_{G=0}
     \end{equation}
  \item $A(\p G)$ is invariant with respect to isometries of ambient space
\begin{equation}\label{idealdensity2}
\end{equation}

\end {itemize}
  These conditions allows us to assign to $A$ in \eqref{lengthdual}
    the local functional $\Phi_C(A)$ on hypersurfaces ($n$-dimensional surfaces in $\E^{n+1}$)
   in the following way: If surface $C$ is given by the equation
   $G=0$ then
   \begin{equation}\label{integralofdualdensity}
   \Phi_C(A)=\int A(G)\delta(G)d^{n+1}x=\int \sqrt{\p_\mu G(x)\p_\mu G(x)}\delta(G(x))
   dx^0\dots  dx^{n+1}
     \end{equation}
 It is easy to see that the functional $\Phi_A(C)$ {\it does not depend on the choice of the
 function $G$ which defines the supersurface}:
              $$
     G\to G'=HG,\quad \int A(HG)\delta(HG)d^{n+1}x=\int A(G)\delta(G)d^{n+1}x
              $$
The formula (3) defines the functional on surfaces and this functional is
invariant with respect to isometries. To revive  the geometricl meaning of this functional
consider the special case when
\begin{equation}\label{specialcase}
    G(z)(z)=x^0-f(x^1,\dots,x^n)
\end{equation}
Then the functional  \eqref{integralofdualdensity} will have the appearance:
\begin{equation}\label{specialapp}
\Phi_C(A)=\int \sqrt
                 {
                 1+
    \left({\p f\over\p x^1}\right)^2+
                   \dots+
    \left({\p f\over\p x^1}\right)^2
                  }\delta
      \left(x^0-f(x^1,\dots,x^n)\right)
      dx^0 dx^1\dots dx^n
\end{equation}
\begin{equation}\label{specialapp2}
   =\int \sqrt
                 {
                 1+
    \left({\p f\over\p x^1}\right)^2+
                   \dots+
    \left({\p f\over\p x^1}\right)^2
                  } dx^1\dots dx^n
\end{equation}

One can see that it is an area of the surface: The integrand is nothing but
square root of determinant of the first quadratic form on the surface.



\subsection {Density}

The function $A$ depending on $G$ and its derivatives
\begin{equation}\label{dualdensity}
    A=A(G,\p_\mu G, \p^2_{\mu_{1}\mu_2}G,\dots,\p^k_{\mu_{1}\dots{\mu_k}}G)
\end{equation}
is called  {\it (dual) density of the weight $\sigma$ and rank $k$}
if it obeys the following condition:
                $$
    A=A\left(
    HG,\p_\mu\left(H G\right), \p^2_{\mu_{1}\mu_2}\left(H G\right),\dots,
    \p^k_{\mu_{1}\dots{\mu_k}\left(H G\right)}
    \right)
    \vert_{G=0}=
    $$
    \begin{equation}\label{densityweightdefinition}
    H^\sigma\cdot
    A\left(
    G,\p_\mu G, \p^2_{\mu_{1}\mu_2}G,\dots,\p^k_{\mu_{1}\dots{\mu_k}G}
    \right)
\end{equation}
where $H$ is an arbitrary? function.

The density of the weight $1$ defines functional on surfaces:
If $G=0$ defines surface, then the integral
\begin{equation}\label{surfacefunctional}
    \Phi_C(A)=\int A(
    G,\p_\mu G, \p^2_{\mu_{1}\mu_2}G,\dots,\p^k_{\mu_{1}\dots{\mu_k}G})\delta(G)d^{n+1}x
\end{equation}
defines the functional on the surface $C$ defined  by the equation $G=0$.

{\it Density of the weight $\sigma=1$ defines volume form on surfaces.}

E.g.  $A$ in \eqref{lengthdual} defines density of the weight $1$ and rank $1$.
It defines volume form induced by Euclidean metric on embedded surfaces.
We later call the density $A$ area density.

The density of the weight $0$ defines the function on the surfaces.
In principal any density of the weight $\sigma\not=0$ defines volume form
$A^\s$, but it can happen that
$A$ is nilpotent and the volume form is trivial. (E.g. invariant density
for surfaces in symplectic superspace.)

is density (dual density) of the rank $k$



\subsection {Formula for mean curvature}

One can see that conditions \eqref{idealdensity1},\eqref{idealdensity2}+ uniquely define
the area functional in the {\it class of local functionals depending on first derivatives.}
What if we consider functionals depending on higher derivatives?

\medskip



In this paragraph we consider the simplest possible density of the second rank
(i.e. density depending on first and second derivatives)
which is invariant under isometries.
 Consider
        \begin{equation}\label{zerothorder}
    A_0(G)=\p_\mu \p_\mu G=\Delta G
\end{equation}

 This expression is invariant with respect to isometries
 but the condition
\eqref{densityweightdefinition} is not satisfied:
 \begin{equation}\label{checkingthecondition1}
\Delta \tilde G\vert_{G=0}=\Delta(HG)\vert_{G=0}=H\Delta G+2\p_\mu H\p_\mu G
\end{equation}

The term $\p_\mu H\p_\mu G$ is an obstacle to the condition \eqref{densityweightdefinition}.
It survives only for $H\vert_{G=0}=consta$.

Consider another invariant in second derivatives:
                \begin{equation}\label{higherordersecondinvariant}
             I_1(G)=\p_\mu G\p_\nu G \p_\mu\p_\nu G
                 \end{equation}
It is easy to see that
\begin{equation}\label{checkingthecondition2}
I_1(HG)\vert_{G=0}=H^3I_1(G)\vert_{G=0}+2H^2\left(\p_\mu H\p_\mu G\right)
                 \left(\p_\nu G\p_\nu G\right)
\end{equation}
We already know (see \eqref{lengthdual}) that
\begin{equation}\label{checkingthecondition3}
\p_\nu (HG)\p_\nu (HG)\vert_{G=0}=H^2\p_\nu G\p_\nu G\,.
\end{equation}
Hence
\begin{equation}\label{checkingthecondition4}
{I_1(HG)\over \p_\nu (HG)\p_\nu (HG)}\vert_{G=0}=
   H{I_1(G)\over \p_\nu (G)\p_\nu (G)}+2\p_\mu H\p_\mu G
\end{equation}
Comparing with condition \eqref{checkingthecondition1} for function \eqref{zerothorder}
we see that for function
\begin{equation}\label{formulaformeancurvature}
    A(G)=\Delta G-{I_1(G)\over \p_\nu (G)\p_\nu (G)}=
    \Delta G-{\p_\mu G\p_\nu G \p_\mu\p_\nu G\over \p_\nu (G)\p_\nu (G)}
\end{equation}
the condition  \eqref{densityweightdefinition}  holds:
\begin{equation}\label{formulaformeancurvature}
    A(HG)\vert{G=0}=H\Delta G-H{I_1(G)\over \p_\nu (G)\p_\nu (G)}=
    HA(G)
\end{equation}
Thus we see that expression \eqref{formulaformeancurvature}
defines an invariant density of the second rank. Its weight is equal to $\s=1$.
Comparing with the area density $A_{\rm area}$ (see \eqref{lengthdual})
we come to the dnsity of the zeroth weight:
\begin{equation}\label{formulaformeancurvature2}
A_{\rm mean\,curv.}={1\over \sqrt {\p_\s G\p_\s G}}\left(
\Delta G-{\p_\mu G\p_\nu G \p_\mu\p_\nu G\over \p_\nu (G)\p_\nu (G)}
                     \right)
\end{equation}
(The notation $A_{\rm mean\,curv.}$ will be justified below)

\medskip


Thus we come to the functional on surfaces
The formula
\begin{equation}\label{meancurvaturefunctional}
\Phi_M(A)=\int \left
[\Delta G-{\p_\mu G\p_\nu G \p_\mu\p_\nu G\over \p_\nu (G)\p_\nu (G)}
        \right]\delta(G)dx^0 dx^1\dots dx^n
\end{equation}
defines the functional on surface
                  $M=\{x, G(x)=0\}$.
                  This functional due to
\eqref{formulaformeancurvature} does not depend on the function $G$
defining a surface $M$ and is invariant with respect to isometries.

There are not many functionals of this type.
We will see in the next section that
this is just functional of mean curvature.

\subsection {Mean curvature and variation of area}

We know that mean curvature functional is a
variation of area functional.
Show this. Namely we calculate Euler-Lagrange equations of area functional
and show that this lead to the density \eqref{formulaformeancurvature}.

It is standard calculations but here we obtain them in dual formalism:
it is much easier.

 Consider variation of area functional
\footnote{we denote by the same letter
"$\delta$" the symbol of variation and the symbol of generalised function.
 Sorry for confusion}. If $G\to G+\vare F$ then \eqref{integralofdualdensity}
\begin{equation}\label{variationofareafunctional}
    {\delta \left[\int \sqrt{\p_\mu G(x)\p_\mu G(x)}\delta(G(x))d^{n+1}x\right]}=
\end{equation}
                    $$
  \int
         {\vare\p_\mu F\p_\mu G\over \sqrt{\p_\mu G(x)\p_\mu G(x)}}\delta(G(x))d^{n+1}x+
         \int \sqrt{\p_\mu G(x)\p_\mu G(x)}\delta'(G(x))\vare F(x)d^{n+1}x
                    $$
                        $$
 =                     \int
                       \left\{
                         -\p_\mu
                         \left[
{\p_\mu G\over \sqrt{\p_\mu G(x)\p_\mu G(x)}}\delta(G(x))
                         \right]+
                         \sqrt{\p_\mu G(x)\p_\mu G(x)}\delta'(G(x))
                       \right\}
                       \vare F(x)d^{n+1}x=
                        $$
                        $$
                        \int
                       \left\{
                       -\p_\mu
                         \left[
{\p_\mu G\over \sqrt{\p_\mu G(x)\p_\mu G(x)}}\delta(G(x))
                         \right]\delta(G(x))
                       \right\}
                       \vare F(x)d^{n+1}x\,.
                        $$
The explicit expression for integrand will be:
                        $$
                        -\p_\mu
                         \left[
{\p_\mu G\over \sqrt{\p_\mu G(x)\p_\mu G(x)}}\delta(G(x))
                         \right]=
-{1\over \sqrt{\p_\mu G(x)\p_\mu G(x)}}
                            \left(
                          \Delta G-{\p_\mu G\p_\nu G \p_\mu\p_\nu G\over \p_\nu (G)\p_\nu (G)}
                            \right)=
                           $$
                           $$
                           -{A(G)\over \sqrt{\p_\mu G(x)\p_\mu G(x)}}
                           $$
where $A(G)$ mean curvature dual density.

Hence we come to the answer: if $G\to G+\vare F$ then
                     \begin{equation}\label{variationofareameandensity}
{\delta \left[\int \sqrt{\p_\mu G(x)\p_\mu G(x)}\delta(G(x))d^{n+1}x\right]}=
-\int  {A(G)\vare F\over \sqrt{\p_\mu G(x)\p_\mu G(x)}}\delta(G(x))d^{n+1}x
\end{equation}
\begin{equation}\label{stationarysurface}
    A(G)\vert_{G=0}\equiv 0
\end{equation}In particular the stationary surface for area functional will be:

It is just the condition that mean curvature is equal to zero at stationary surface
of area functional.

We see  that


 {\it The mean curvature density defines mean curvature volume form on surfaces
               \begin{equation}\label{meancurvaturevolumeform}
    A_{\rm mean\,curv}=\Delta G-{\p_\mu G\p_\nu G \p_\mu\p_\nu G\over \p_\nu (G)\p_\nu (G)}
                  \end{equation}
It is a density of the weight $\s=1$ and rank $k=2$ and

  The mean curvature itself  ---is the following
  density of the weight $\s=0$ and rank $k=2$:
             \begin{equation}\label{meancurvatureitself}
   H= {A_{\rm mean\,curv}\over A_{\rm length}}=
   {\Delta G\over \sqrt{\p_\mu G\p_\mu G}}-
   {\p_\mu G\p_\nu G \p_\mu\p_\nu G\over \left(\p_\nu (G)\p_\nu (G)\right)^{3/2}}
                  \end{equation}
}

{\bf Note} The dual density $A_{\rm mean\,curv}$ corresponds to the density
$H\sqrt {\det g_{ik}}$. It is instructive to do the following exercise:
Consider the sphere of the radius $a$: $x^2+y^2+z^2-a^2=0$. Calculate $A$.
It does not depend on $A$. Calculate $\int A\delta(\Phi)d^3 x$

\subsection {Operator-valued densities}

Now we try to write all invariants of rank $k=2$ on surfaces. It will be
symmetric functions of eigenvalues of dshape oeprator, i.e. traces of exterior product
of shape oeprator. E.g. mean curvature is nothing but a trace of shape operator.

It turns out that our dual shape operator is the summ of usual shape operator
and one-dimensional operator which is proportional to mean curvature.

In the way similar  how we constructed a mean curvature we will construct now
this operator: we will find {\it operator valued density}

Operator valued density $\hat M(G)$ takes vales in oeprators.
E.g. $\hat M(G)=\sqrt{\p_\mu G\p_\mu G}\delta^i_k$ is operator valued density
of the weight $\s=1$ and rank $k=1$.

In the previous subsection starting from the function $A_0=\Delta G$
we come to the mean curvature density (see \eqref{zerothorder}---\eqref{formulaformeancurvature})

Now we start from
\begin{equation}\label{zerothorderoperatordensity}
    A_0=\p_\mu\p_\nu G
\end{equation}


Sure it is not density but this expression gives rise to operator-valued density
in the same way as in  \eqref{zerothorder}---\eqref{formulaformeancurvature}


 {\bf Proposition 1}
 {\it The following formula
 \begin{equation}\label{operatorvalueddensity1}
    M_{\mu\nu}(G,\p G, \p^2G)=
   \p_\mu\p_\nu G+{\p_\mu G\p_\nu G\Delta G\over \p_\s G\p_\s G}
  -{\p_\mu G\p_\s G\p_\s\p_\nu G+\p_\nu G\p_\s G\p_\s\p_\mu G
  \over \p_\delta G\p_\delta G}
\end{equation}

(summation is assumed over numb indices)
 defines operator-valued density of the weight $\s=1$, rank $k=2$}

\medskip





 Proof: Direct Check

\bigskip

{\bf Remark} In Euclidean space we identify upper and down indices.

Later the expression: "operator $M_{ik}$" means that by default one of the indices
is risen by the metric $g_{ik}=\delta_{ik}$


\smallskip


Note that the trace of the operator-valued density  \eqref{operatorvalueddensity1}
is up to a coefficient
nothing but mean curvature-density\eqref{meancurvaturevolumeform}
                 $$
    \Tr \left(M_{\mu\nu}(G,\p G, \p^2G)\right)=
   \Delta G+{\p_\mu G\p_\mu G\Delta G\over \p_\s G\p_\s G}
  -{\p_\mu G\p_\s G\p_\s\p_\mu G+\p_\mu G\p_\s G\p_\s\p_\mu G
  \over \p_\delta G\p_\delta G}=
            $$
  \begin{equation}\label{traceofoperatorvalueddensity}
2\left(\Delta G-{\p_\mu G\p_\nu G \p_\mu\p_\nu G\over \p_\nu (G)\p_\nu (G)}\right)=
2 A_{\rm mean\,curv}
\end{equation}


It follows from Proposition1 and \eqref{lengthdual} that the
The following formula
\begin{equation}\label{operatorvalueddensity1}
                 {1\over \sqrt {\p_\mu G\p_\mu G}} \left(
   \p_\mu\p_\nu G+{\p_\mu G\p_\nu G\Delta G\over \p_\s G\p_\s G}
  -{\p_\mu G\p_\s G\p_\s\p_\nu G+\p_\nu G\p_\s G\p_\s\p_\mu G
  \over \p_\delta G\p_\delta G}\right)
\end{equation}
defines operator-valued density of the weight $\s=0$, rank $k=2$

We come to

\bigskip

{\bf Theorem}
{\it The following formula
 \begin{equation}\label{operatorvalueddensity1}
                       \det
                       \left(\delta_{\mu\nu}+
                 {z\over \sqrt {\p_\rho G\p_\rho G}} \left(
   \p_\mu\p_\nu G+{\p_\mu G\p_\nu G\Delta G\over \p_\s G\p_\s G}
  -{\p_\mu G\p_\s G\p_\s\p_\nu G+\p_\nu G\p_\s G\p_\s\p_\mu G
  \over \p_\delta G\p_\delta G}\right)\right)
\end{equation}
where $z$ is a formal parameter
defines density of the weight $\s=0$ and rank $k=2$.


This density is generating function for all invariant densities
}



\bigskip  The density is characteristic polynomial of dual shape operator.


Recall that shape operator $S$ for hypersurface
$\r=\r(\xi^k)$ ($k=1,\dots,n$) is a linear operator defined via
first quadratic form $g_{ik}=\left(\p_i{\bf r}, \p_k{\bf r}\right)$
and second quadratic
form  $A_{ik}=\left({\bf n}, \p^2_{ik}{\bf r}\right)$
                    \begin{equation}\label{usualshpaeoperator}
    S_{ik}=g^{ir}A_{rk}\left(\p_i{\bf r},
    \left({\bf n}, \p^2_{ik}{\bf r}\right)\p_r{\bf r}\right)^{-1}\left({\bf n}, \p^2_{ik}{\bf r}\right)
\end{equation}
(as usual we identify lower and upper indices)

Recall that mean curvature is equal to the trace of shape operator.

\bigskip

Now we clarify the relation of the operator $M_{\mu\nu}$ in the proposition 1
and shape operator. The fact that the trace of operator $M_{\mu\nu}$
is equal just twice to mean curvature is not ocasional.


{\bf Theorem}
{\it The operator \eqref{operatorvalueddensity1}
is direct sum of shape operator and scalar operator of multiplication on mean curvature.
Namely
let $C$ be hypersurface defined by the equation $G=0$,
 $p$ a point on it, $\bf n$ normal vector,
ad $S$-shape operator at the point $p$.  Then

\begin{equation}\label{directsum}
   {M_{\mu\nu}\over \sqrt {\p_\rho G \p_\rho G}}=
\left(
   \begin{array}{cc}
  S_{ik} & 0 \\
      0& H{\bf n} \\
   \end{array}
   \right)
\end{equation}

In particular

\begin{equation}\label{formulaefordulashape}
    \Tr S^{\rm dual}=\Tr S+H=2H,\quad \det S^{\rm dual}=H\det S=HK
\end{equation}
where $K$-gaussian curvature.


So shape operator is a "component" of operator \eqref{operatorvalueddensity1}.}

\bigskip
{\it Proof of the Theorem}.Calculate the action of the operator $S^{\rm dual}$
on arbitrary tangent vector and arbitrary normal vector.

Let ${\X}=X^\mu\p_mu$ is a tangent vector. Then  $X^\mu\p_\mu G=0$ if $G=0$.
Proof that the vector $S^{\rm dual}\X$ is tangent to the surface too:
                 $$
\p_\mu G{M_{\mu\nu}X^\nu\over \sqrt {\p_\rho G \p_\rho G}}=
                 $$
                  $$
                  {\p_\mu G\over \sqrt {\p_\mu G\p_\mu G}} \left(
   \p_\mu\p_\nu G+{\p_\mu G\p_\nu G\Delta G\over \p_\s G\p_\s G}
  -{\p_\mu G\p_\s G\p_\s\p_\nu G+\p_\nu G\p_\s G\p_\s\p_\mu G
  \over \p_\delta G\p_\delta G}\right)X^\nu
                  $$
                      $$
            {\p_\mu G\over \sqrt {\p_\mu G\p_\mu G}} \left(
   \p_\mu\p_\nu G X^\nu
  -{\p_\mu G\p_\s G\p_\s\p_\nu GX^\nu
  \over \p_\delta G\p_\delta G}\right)=
  {\p_\mu G \p_\mu\p_\nu G X^\nu-\p_\s G\p_\s\p_\nu GX^\nu
  \over \sqrt {\p_\mu G\p_\mu G}}
  =0$$
We proved that vectors tangent to surface remain tangent after action of
$S^{\rm dual}$. Now consider arbitrary vector orthogonal to surface. It
is proportional to $\p_mu G$. Calculate $S^{\rm dual}_{\mu\nu}\p_nu G$:
                $$
{M_{\mu\nu}{\p_\nu G}\over \sqrt {\p_\rho G \p_\rho G}}=
                 $$
                  $$
                  {\p_\nu G\over \sqrt {\p_\mu G\p_\mu G}} \left(
   \p_\mu\p_\nu G+{\p_\mu G\p_\nu G\Delta G\over \p_\s G\p_\s G}
  -{\p_\mu G\p_\s G\p_\s\p_\nu G+\p_\nu G\p_\s G\p_\s\p_\mu G
  \over \p_\delta G\p_\delta G}\right)=
                $$
              $$
  {\p_\mu G\over \sqrt {\p_\mu G\p_\mu G}} \left(
             \Delta G-
   {\p_\s G\p_\s\p_\nu G\p_\nu G
  \over \p_\delta G\p_\delta G}\right)=\p_\mu G H
              $$
              The proof is finished\finish


              \bigskip


 We can formally express the statement of the Theorem  in the following way:

Instead extracting the mode $H{\bf n}$ in the operator $S^{\rm dual}$
we {\bf add} again this mod but with changing parity: Consider the operator
\begin{equation}\label{operarotinextendedspace}
    \hat S=
\left(
\begin{array}{ccc}
  S^{\rm dual} & 0  \\
  0& H{\bf n}\\
\end{array}
\right)
=
\left(
\begin{array}{ccc}
  S_{ik} & 0 & 0 \\
  0& H{\bf n}& 0 \\
  0 & 0 &  H{\bf n} \\
\end{array}
\right)
\end{equation}
where the last dimension is odd. Then it is easy to see that
\begin{equation}\label{hostrelations}
    {\rm Ber\,}\left(1+z\hat S\right)=\left(1+zS_{ik}\right)
\end{equation}

\end{document}
