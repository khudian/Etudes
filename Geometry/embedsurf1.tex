\magnification=1200 \baselineskip=14pt
\def\vare {\varepsilon}
\def\A {{\bf A}}
\def\t {\tilde}
\def\a {\alpha}
\def\K {{\bf K}}
\def\N {{\bf N}}
\def\V {{\cal V}}
\def\s {{\sigma}}
\def\S {{\Sigma}}
\def\s {{\sigma}}
\def\p{\partial}
\def\vare{{\varepsilon}}
\def\Q {{\bf Q}}
\def\D {{\cal D}}
\def\G {{\Gamma}}
\def\C {{\bf C}}
\def\M {{\cal M}}
\def\Z {{\bf Z}}
\def\U  {{\cal U}}
\def\H {{\cal H}}
\def\R  {{\bf R}}
\def\E  {{\bf E}}
\def\l {\lambda}
\def\degree {{\bf {\rm degree}\,\,}}
\def \finish {${\,\,\vrule height1mm depth2mm width 8pt}$}
\def \m {\medskip}
\def\p {\partial}
\def\r {{\bf r}}
\def\v {{\bf v}}
\def\n {{\bf n}}
\def\t {{\bf t}}
\def\b {{\bf b}}
\def\e{{\bf e}}
\def\ac {{\bf a}}
\def \X   {{\bf X}}
\def \Y   {{\bf Y}}
\def \x   {{\bf x}}
\def \y   {{\bf y}}
\def\f {{\bf f}}
\def\pt {{\bf pt}}



   \centerline {\bf Differential geometry of surfaces embedded in Euclidean space}

\bigskip

\centerline {\it General stuff}

\m

({\tt This paragraph is almost the same as from the file "egregium1.tex}")
\m
 In this paragraph we consider relations between curvatures of tangent and normal bundle for an arbitrary
 $k$-dimensional surface in $n$-dimensional Euclidean space $\E^n$.

  Let $x^i(\xi^\a)$ ($i=1,\dots,n,\a=1,\dots,k$) be a
  parameterisation of the given surface $M$. The trivial
  (canonical) connection in $\E^n$ induces connection on surface and
  on normal bundle to the surface.
   Consider an orthonormal (in general non-holonomic) basis $\{\e_a,\n_p\}$, $a=1,\dots,k,p=1,\dots,n-k$
   on the points of the surface $M$ such that vectors $\e_1,\dots,\e_k$ are tangent to the surface
   and vectors $\n_1,\dots,\n_{n-k}$ are normal vectors:
          $$
        \e_a(\xi)=e_a^\a x^i_\a {\p\over \p x^i}\big\vert_{x^i_\a={\p x^i(\xi)\over \p \xi^\a}}
        ,\,\, (a=1,\dots,k),\,\,
        (\e_a,\e_b)=\delta_{ab},(\e_a,\n_p)=0,(\n_p,\n_q)=\delta_{pq}
          $$
  One  can consider  basis vectors $\{\e_a,\n_p\}$ as functions, $0$-forms on the points of the manifold $M$
  with values in the vectors of $\E^n$,
  $\{\e_a(\xi)=\e_a(x(\xi)),\n_p(\xi)=\n_p\left(x\left(\xi\right)\right)\}$.

 Consider $1$-forms $d\e_a$, $d\n_p$ on $M$ with values in vectors in $\E^n$. Expanding them over the basis
$\{\e_a,\n_p\}$ we come to the equations:
              $$
              d
              \pmatrix{\e_a\cr\n_p\cr}=
              \pmatrix {A_{ab} &L_{aq}\cr S_{pb} &T_{pq}}
              \pmatrix{\e_b\cr\n_q\cr}
              $$
            $$
            d\e_a=A_{ab}d\e_b+L_{aq}\n_q,\,\,\qquad d\n_p=S_{pb}\e_b+T_{pq}\n_q,
            \eqno (1.1)
            $$
where $(A_{ab},L_{aq},S_{pb},R_{pq})$ are one-forms.
The matrix  $\pmatrix {A_{ab} &L_{aq}\cr S_{pb} &T_{pq}}$ is antisymmetric since $(\e_a,\n_p)$ is an arthonormal
 basis at any point of $M$:
               $$
            A_{ab}=-A_{ba},\,\,   L_{ap}=-S_{pa},\,\,\, T_{pq}=-T_{qp}
            \eqno (1.2)
               $$

 The condition that $d^2\e_a=0, d^2 \n_b=0$ leads to identities:
                $$
        0=d^2 \pmatrix{\e_a\cr\n_p\cr}=
        \pmatrix {dA_{ab} &dL_{aq}\cr dS_{pb} &dT_{pq}}\pmatrix{\e_b\cr\n_q\cr}-
        \pmatrix {A_{ab} &L_{aq}\cr S_{pb} &T_{pq}}\pmatrix{d\e_b\cr d\n_q\cr}=
                $$
                $$
               \left(
        \pmatrix {dA_{ab} &dL_{aq}\cr dS_{pb} &dT_{pq}}-
        \pmatrix {A_{ac} &L_{ar}\cr S_{pc} &T_{pr}}
        \pmatrix {A_{cb} &L_{cq}\cr S_{rb} &T_{rq}}
        \right)
        \pmatrix{\e_b\cr\n_q\cr}
        \eqno (1.3)
                $$
Hence we have four conditions (structure equations)
             $$
             \matrix
             {
         dA_{ab}-A_{ac}\wedge A_{cb}-L_{ar}\wedge S_{rb}=0\cr
         d L_{aq}-A_{ac}\wedge L_{cq}-L_{ar}\wedge T_{rq}=0\cr
          dS_{pb}-S_{pc}\wedge A_{cb}-T_{pr}\wedge S_{rb}=0\cr
          dT_{pq}-S_{pc}\wedge L_{cq}-T_{pr}\wedge T_{rq}\cr=0
               }
               \eqno (1.4)
             $$
In a view of (2) we come to the following three conditions on one-forms $A_{ab}, S_{pb}, T_{pq}$ :
                    $$
                     \matrix
             {
         dA_{ab}-A_{ac}\wedge A_{cb}+S_{ra}\wedge S_{rb}=0\qquad\qquad &(*)\cr
          dS_{pb}-S_{pc}\wedge A_{cb}-T_{pr}\wedge S_{rb}=0\qquad\qquad &(**)\cr
          dT_{pq}+S_{pc}\wedge S_{qc}-T_{pr}\wedge T_{rq}=0\qquad\qquad &(***)\cr
               }
               \eqno (1.4a)
                    $$
Clarify geometric meaning of these conditions.

1. One-forms $\{A_{ab}\}$ in (1) define parallel transport of tangent vectors and
connection in the tangent bundle such that $\e_b$-component of covariant derivative of
vector field $\e_a(x(\xi))$
along tangent vector $\X$ is equal to  $A_{ab}(\X)$.
   For any section $u^a\e_a$  the one-form  with values in tangent vectors to $M$
                $$
                D(u^a\e_a)=du^a\e_a+u^aD\e_a=du^a\e_a+u^aA_{ab}\e_b,
                \eqno (1.5a)
                  $$
  where $D$ is covariant differential.  Respectively
                  $$
            \nabla_\X \left({u^a\e_a}\right)=D(u^a\e_a)\left(\X\right)=
            du^a\left(\X\right)\e_a+u^aD\e_a\left(\X\right)=
            \p_\X u^a \e_a+u^aA_{ab}(\X)\e_b\,.
            \eqno (1.5b)
                $$


 One can show that this is Levi-Chivita connection of the metrics $g_{\a\b}$ ($\e g\e^+=1$) ({\tt Show it!}).
       The action of $D^2$ defines curvature operator:
                $$
         D^2(u^a\e_a)=D\left(du^a\e_a+u^aA_{ab}\e_b\right)=-du^aD\e_a+d(u^aA_{ab})\e_b-u^aA_{ab}D\e_b=
            u^a{\cal R}^{\rm tang}_{ab}\e_b
               $$
where for curvature-form  ${\cal R}^{\rm tang}_{ab}$
             $$
{\cal R}^{\rm tang}_{ab}=dA_{ab}-A_{ac}\wedge A_{cb}
                      \eqno (1.6)
             $$
Hence the first condition in (3) means
           $$
   {\cal R}^{\rm tang}_{ab}=-S_{ra}\wedge S_{rb}
   \eqno (1.7a)
           $$



 Analogously we can consider connection and curvature of normal bundle.
 In particular we see that for curvature ${\cal R}^{\rm norm}_{pq}$ of normal bundle
            $$
            {\cal R}^{\rm norm}_{ab}=dT_{pq}-T_{pr}\wedge T_{rq}
            \eqno (1.7b)
            $$
and third condition means
          $$
  {\cal R}^{\rm norm}_{ab}=--S_{pc}\wedge S_{qc}
         \eqno (1.7c)
            $$
            We see that both curvatures of normal and tangent bundle are defined via the one forms
  $S_{pr}$.




\bigskip

\centerline {\bf Examples}

\m


First  very simple example

\m

{\bf Example}  Consider a sphere $S^2$ in $\E^3$:
                 $$
  \r(\theta,\varphi)\colon\qquad\cases{x=a\sin\theta\cos\varphi\cr
               y=a\sin\theta\sin\varphi\cr
               x=a\cos\theta\cr}
               \eqno (ex1.1)
                 $$
Tangent vectors are
           $$
          \r_\theta=a\pmatrix{\cos\theta\cos\varphi\cr
                   \cos\theta\sin\varphi\cr
               -\sin\theta\cr},\,\,\,
                \r_\varphi=a\pmatrix{-\sin\theta\sin\varphi\cr
                   \sin\theta\cos\varphi\cr
                        0\cr}
                        \eqno (ex1.2)
           $$
and unit normal vector is $\n=\pmatrix{\sin\theta\cos\varphi\cr
               \sin\theta\sin\varphi\cr
               \cos\theta\cr}$

Hence one can consider orthonormal moving frame  $\{\e_\theta,\e_\varphi,\n\}$, where
              $$
     \e_\theta=\r_\theta=\pmatrix{\cos\theta\cos\varphi\cr
                   \cos\theta\sin\varphi\cr
               -\sin\theta\cr},\quad
       \e_\varphi={\r_\varphi\over \sin\theta}=
       \pmatrix{-\sin\varphi\cr
                   \cos\varphi\cr
                        0\cr},\quad
                        \n=\pmatrix{\sin\theta\cos\varphi\cr
               \sin\theta\sin\varphi\cr
               \cos\theta\cr}\,.
              $$
We have
               $$
d\e_\theta=d\pmatrix{\cos\theta\cos\varphi\cr
                   \cos\theta\sin\varphi\cr
               -\sin\theta\cr}=\pmatrix{-\sin\theta\cos\varphi\cr
                   -\sin\theta\sin\varphi\cr
               -\cos\theta\cr}d\theta+\pmatrix{-\cos\theta\sin\varphi\cr
                   \cos\theta\cos\varphi\cr
                    0\cr}d\varphi=-d\theta\n+\cos\theta d\varphi \e_\varphi\,,
                              $$
                              $$
    d\e_\varphi=d\pmatrix{-\sin\varphi\cr
                   \cos\varphi\cr
                        0\cr}=\pmatrix{-\cos\varphi\cr
                   -\sin\varphi\cr
                        0\cr}d\varphi=\left(-\cos\theta\pmatrix{\cos\theta\cos\varphi\cr
                   \cos\theta\sin\varphi\cr
               -\sin\theta\cr}-\sin\theta
               \pmatrix{\sin\theta\cos\varphi\cr
               \sin\theta\sin\varphi\cr
               \cos\theta\cr}\right)d\varphi
                            $$
                            $$
                    =-\cos\theta d\varphi \e_\theta-\sin\theta d\varphi \n\,,
                            $$
  and
                           $$
  d\n=d\pmatrix{\sin\theta\cos\varphi\cr
               \sin\theta\sin\varphi\cr
               \cos\theta\cr}=\pmatrix{\cos\theta\cos\varphi\cr
               \cos\theta\sin\varphi\cr
               -\sin\theta\cr}d\theta+\pmatrix{-\sin\theta\sin\varphi\cr
               \sin\theta\cos\varphi\cr
               0\cr}d\varphi= d\theta \e_\theta+\sin\theta d\varphi \e_\varphi
                           $$
i.e.
                   $$
      d\pmatrix{\e_\theta\cr
               \e_\varphi\cr
               \n\cr}=
               \pmatrix
                  {
       &0 &\cos\theta d\varphi    &-d\theta\cr
       &-\cos\theta d\varphi &0   &-\sin\theta d\varphi\cr
      &d\theta   &\sin\theta d\varphi &0\cr         }
      \pmatrix{\e_\theta\cr
               \e_\varphi\cr
               \n\cr}\,,
                   $$
                   $$
         \pmatrix {A_{ab} &L_{aq}\cr S_{pb} &T_{pq}}=
          \pmatrix
                  {
       &0 &\cos\theta d\varphi    &-d\theta\cr
       &-\cos\theta d\varphi &0   &-\sin\theta d\varphi\cr
      &d\theta   &\sin\theta d\varphi &0\cr }
                                           $$

  One can see that $A=\sin\theta d\varphi$ defines connection. Curvature of this connection
   $R=dA=\sin\theta d\theta\wedge d\varphi$. These formulae are written in h=non-holonomic coordinates
   $\{\e,\f\}$  Rewriting them in holonomic coordinates $\{\r_\theta,\r_\varphi\}$ we come to connection
   and curvature defined my first quadratic form (riemanian inducd metric)....


   \bigskip


      \centerline {\bf Parallel transport of tangent vector along closed curve on two-dimensional surface in $\E^3$}.

      Let $M$ be two-dimensional surface $\r=\r(\xi^1,\xi^2)$ in $\E^3$ and let $C\colon \r=\r(t)$,
        $t_1\leq t\leq t_2, \r(t_1)=\r(t_2)$ be a closed curve on it.
        Let $\X(t)$ be a parallel transport of the vector $\X$ along this curve
  Then the angle between the initial vector $\X(t_1)$ and $\X(t_2)$ equals to
                          $$
                        \varphi=\int_D K d\sigma\,,
                        \eqno (3.1)
                          $$
           in the case if $C$ is a boundary of the domain $D$. Here
        $d\sigma$ is an area form on $M$ compatible with induced metric:
         $d\sigma=\sqrt{\det g}d\xi^1d\xi^2$, and $K$- is scalar curvature of the
         induced metrics  which is equal
          to Gaussian curvature\footnote{$^*$} {The fact that these external and internal cuyrvatures coincide is
          the famous Gau\ss's {\ it Theorema Egregium}.}.

\m

  Prove this theorem.

\m


    Consider moving orthonormal frame $\{\e_1,\e_2,\n\}$ on the surface $M$, where
    vector field $\n$ is normal to the surface. Thus vectors $\{\e_1,\e_2\}$  is non-holonomic basis
    of tangent spaces to  $M$. Act by differential on these vectors.   We have
                   $$
                   d
               \pmatrix{
        \e_1\cr \e_2\cr \n\cr
                   }=
               \pmatrix
                  {
       &0 &  A&  S_1\cr
       &-A &0   &S_2\cr
      &-S_1&-S_2&0\cr         }
      \pmatrix{\e_1\cr\e_2\cr\n\cr}\,,
      \eqno (3.2)
                   $$
where $A,S_1,S_2$ are $1$-forms. The condition $d^2=0$ implies
                              $$
                              \matrix
                                 {
    0=d^2\e_1=d(A\e_2+S_1\n)=dA \e_2-A\wedge(-A\e_1+S_2\n)+dS_1\n-S_1\wedge (-S_1\e_1-S_2\e_2)\cr
    0=d^2\e_2=d(-A\e_1+S_2\n)=-dA \e_1+A\wedge(A\e_2+S_1\n)+dS_2\n-S_2\wedge (-S_1\e_1-S_2\e_2)\cr
    0=d^2\n=d(-S_1\e_1-S_2\e_2)=-dS_1\e_1+S_1\wedge(A\e_2+S_1\n)-dS_2\e_2+S_2\wedge (-A\e_1+S_2\n)\cr
                                }
                                \eqno (3.3)
                              $$
                              i.e.,
                              $$
             dA+S_1\wedge S_2=0
                \eqno (3.4a)
                              $$
                              $$
                           \matrix
                                {
                              &dS_1&-&A\wedge S_2=&0\cr
                              &dS_2&+&A\wedge S_1=&0\cr
                                }
                                \eqno (3.4b)
                              $$
 Look carefully, very carefully on the equation (3.4a).
 Basis is not holonomic.  Due to this fact all formulae become more transparent.

 On the left hand side of the formula (3.4a) we have  two-form---curvature of the Levi-Civita connection
  on the right  hand we have determinant of shape operator. The formula (3.4a)
  {\bf states that curvature calculated by the internal observer on the surface
  $R=R(\Gamma(g))$ equals to Gaussian curvature.}

  Now calculate the angle of rotation during parallel transport. It is the same for all the vectors.
   Consider on the points of the curve $C$ the orthonormal basis $\{\ac(t),{\bf b}(t)\}$,
   $t_1\leq t<t_2$  which is a parallel transport of the basis $\{\e_1(t),\e_2(t)\}$
   at the initial point $t=t_1$
                   $$
    \ac(t_1)=\e_1(t_1),
    {\bf b}(t_1)=\e_2(t_1),\,\,
    D\ac(t)=D{\bf b}(t)\equiv 0 \,\,\hbox {along vectors $\v(t)={d\r(t)\over dt}$}\,,
                   $$
  where $\v(t)={d\r(t)\over dt}$ is velocity vector of the curve $C$

   The basis $\{\ac(t), {\bf b}(t)\}$ remains orthonormal basis during parallel transport, i.e.
                $$
                \{\ac(t), {\bf b}(t)\}=\{\e_1(t), \e_2(t)\}
                \pmatrix
                {&\cos\phi (t)  &\sin\phi(t)\cr
                -&\sin\phi (t)  &\cos\phi(t)\cr }
                $$
In particular
                 $$
           \ac(t)=\e_1(t)\cos\phi(t)+\e_2(t)\sin\phi(t), \hbox {with $\phi(t_1)=0$}
                 $$
Il faut calculer $\phi(t_2)$.

Now remember formulae (1.5)   (3.3) we come to
       $$
    0=\nabla_{\v(t)} \ac(t)=
    \nabla_{\v(t)}\left(\e_1(t)\cos\phi(t)+\e_2(t)\sin\phi(t)\right)=
         $$
         $$
    \left(D\e_1(t)(\v(t))\cos\phi(t)+\e_1(t){d\cos\phi(t)\over dt}\right)+
        \left(D\e_2(t)(\v(t))\sin\phi(t)+\e_2(t){d\sin\phi(t)\over dt}\right)=
       $$
       $$
  \left((A(\v(t))\e_2(t)\cos\phi(t)-\e_1(t)\sin\phi(t)\dot\phi(t)\right)+
  \left((-A(\v(t))\e_1(t)\sin\phi(t)+\e_2(t)\cos\phi(t)\dot\phi(t)\right)
       $$
  since $D\e_1=A\e_2, De_2=-A\e_1$.
       It follows from this relation that
                  $$
           \dot\phi(t)+A(\v(t))=0\,,
                 \eqno ()
                 $$
 We did it:
                  $$
         \phi(t_2)=\int_{t_1}^{t_2}\dot\phi(t)dt=-\int_C A=-\int_{\p D} A=-\int_D dA
                  $$
({\tt It has to be explained that $dA=K\sqrt {det g}$})
\bye