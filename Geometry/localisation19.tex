


    

\magnification=1200
\baselineskip=14pt

\def\A {{\bf A}} 
\def\B {{\cal B}}
\def\C {{\bf C}}
\def\CC {{\cal C}}
\def\Cl {{\tt \hbox{Cliff}}}
\def\E {{\bf E}}
\def\EE {{\cal E}}
\def\F {{\cal F}}
\def\FF {{\cal F}}
\def\G {\Gamma}
\def\GG {{\cal G}}
\def\H {{\bf H}}
\def\K {{\bf K}}
\def\L {{\cal L}}
\def\M {{\cal M}}
\def\N {{\bf N}}
\def\R {{\bf R}}
\def\Sb {{\bf S}}
\def\SS {{\cal S}}
\def\Tr {{\rm Tr\,}}
\def\V {{\cal V}}
\def\X {{\bf X}}
\def\XX {{\cal X}}
\def\Y {{\bf Y}}
\def\Z {{\bf Z}}

\def\a {\alpha}
\def\ac {{\bf a}}
\def\b {{\bf b}}
\def\bs {{\bf s}}
\def\c {{\bf c}}
\def\d {\delta}
\def\dist {{\tt \hbox{distance}}}
\def\e {{\bf e}}
\def\f {{\bf f}}
\def\finish {\blacksquare}
\def\g {{\bf g}}
\def\grad {{\rm grad\,}}
\def\h {\hbar}
\def\k {{\bf k}}
\def\l {{\bf l}}
\def\m {{\bf m}}
\def\n {{\bf n}}
\def\p {\partial}
\def\pb {{\bf p}}
\def\pt {{\bf pt}}
\def\q {{\bf q}}
\def\r {{\bf r}}
\def\s {\sigma}
\def\t {{\bf t}}
\def\tS {{\tilde \Sigma}}
\def\td {\tilde}
\def\v {{\bf v}}
\def\vare {\varepsilon}
\def\x {{\bf x}}
\def\y {{\bf y}}
\def\w {\omega}

\centerline {\bf Minimal formulation of localisation
principle}

 {\tt 10 October,   Cyprus---12 October David's
birthday.}

{\it 
   Here I try to formulate the statement on localisation
which seems to be very "rational". It is 
based on my notes in 2013, and these notes are based
on understanding of Belavin calculations based
on the classic paper of Schwarz}

  Let $F(x)=e^{iH(x)}$ be 
an arbitrary (non-zero valued)
function in  symplectic manifold  
$(M\,,\Omega)$.  Consider the integral
        $$
    \int_M e^{iH} dp dq\,,
      \eqno (1)
         $$
where $dpdq$ is Lioville measure in $M$:
        $$
  dpdq =\underbrace
     {
  \Omega\wedge\dots\wedge \Omega}
          _
      {\hbox{$n$-times}}\,,\quad \hbox
      {$2n$=dimension of $M$}\,.
         $$

{\bf Statement} The integral (1) can be localised
if there exist $D_{\bf H}$-invariant 
$1$-form $\omega$ which produces
symplectic structure on $M$
(i.e. $\tilde\Omega=d\omega$ is non-degenerate
$2$-form).

We denote by  ${\bf D}_H=K$ 
Hamiltonian vector field of Hamiltonian
$H$
       $$
   \Omega {\cal c}{\bf K}+dH\left(\bf K\right)=0\,,
\,, ({\bf K}=D_{H})\,,\quad
 {\L_\K}\omega=0\,.
        $$
In the case if $\{x_i\}$ are points where vector field
   ${\bf K}=D_{\bf H}$ vanishes,then
this vector field defines at every point the
linear operator $dK_i$\footnote{$^{*}$}
{the action of operator $dK_i$ on
an arbitrary tangent vector $\v\in T_{x_i}M$
can be defined by an equation
           $
          dK_i(\v)=[\tilde\v,K]
           $,
where $\tilde \v$ is an arbitrary vector field such that
its value at the point $x_i$ is equal to $\v$,
and $[\,,\,]$ is the commutator of vector fields.
The vanishing of ector $K$ at the point $x_i$ provides
the correctness of this definition.}
This localised integral is equal (up to coefficients
containing
 $\pi$) to
         $$
{1\over n!}\sum_{x_i}{e^{iH(x_i)}\over \sqrt {\det d K(x_i)}}
     \eqno (1a)
         $$
\medskip


Namely following Belavin \footnote{$^{**}$}
{I did the slight but important modification of his
calculations} consider a function
            $$
      Z(t)=
   \int_{\Pi TM} 
e^{i
      \left(
  {H+\Omega}+
 t \sqrt {\L_{\bf K}\omega} 
    \right)}dpdq d\xi
       =\int_{\Pi TM} e^{i
      \left(
  H+\Omega+
 t \left(h+\tilde \Omega\right) 
    \right)}dpdqd\xi\,,
        \eqno (2)
             $$
where  $\Pi TM$ is tangent bundle with reversed parity
of fibres,$dpdqd\xi$ is canonical measure 
on $\Pi TM$, 
$h$ is an `artifical' Hamiltonian   
defined by the new symplectic structure;
  this new artificial Hamiltonian
 produces the same Hamiltonian vector fied:
   $D_h=D_H=\bf K$.
        $$
h=\tilde\Omega {\cal c} {\bf K}\,,
\quad \tilde \Omega=d\omega\,.
          $$
We use notation $\sqrt {\L_\K}$
for equivariant differential:
        $$
\sqrt{\L_{\bf K}}\omega=
d_K \omega=d\omega+
\omega {\cal c} {\bf K}\,.
        $$
{\bf Remark} 
We denote by the same 
letter differential form
and the function on $\Pi TM$ 
corresponding to this form.
    

\medskip

{\bf Lemma} A function (2) {\tt does not depend} on $t$:



We use this lemma to reduce the calculation
of initial integral $Z(0)$
to the integral $Z(T)$, which for big $t$
can be expressed in terms of artificial HHamiltonian
   $h$.
  Lemma implies 
 that the initial integral
can be calculated in terms of an artificial hamiltonian,
and this integral may be caluclated using the stationary
phase method:
         $$
    \int e^{iH} dp dq=Z(0)=Z(\infty)\,.
         $$
i.e.
        $$
    \int_M e^{iH} dp dq
         =
\int_M e^{iH}\underbrace
     {
  \Omega\wedge\dots\wedge \Omega}
          _
      {\hbox{$n$-times}}=
       \int_{\Pi TM}
        e^{i\left(H+\Omega\right)}
       dpdqd\xi=
        $$
        $$
  \lim_{T\to \infty}
 \int_{\Pi TM} 
e^{i
      \left(
  {H+\Omega}+
 T \sqrt {\L_{\bf K}\omega} 
    \right)}dpdqd\xi
  =
    \lim_{T\to \infty}
       \int_{\Pi TM} e^{i
      \left(
  H+\Omega+
 T \left(h+\tilde \Omega\right) 
    \right)}dpdqd\xi\,,
   \eqno (3)
       $$
Now notice that for every $T$
 $$       
 \int_{\Pi TM} e^{i
      \left(
  H+\Omega+
 T \left(h+\tilde \Omega\right) 
    \right)}dpdqd\xi=
  \int_{M} e^{i
      \left(
 H+Th \right)}\sum_{k=0}^n 
{
T^k\tilde \Omega^k
   \wedge
\Omega^{n-k}
\over k!(n-k)!}
  $$ and for every term
$ \int_{\Pi TM} e^{i
      \left(
  H+\Omega+
 T \left(h+\tilde \Omega\right) 
    \right)}dpdqd\xi=
  \int_{M} e^{i
      \left(
 H+Th \right)}\sum_{k=0}^n 
{
\tilde \Omega^k
   \wedge
\Omega^{n-k}}
  $
    the integral is proportional to $t^{k-n}$,
i.e. it is tending to zero if $T\to \infty$ and $k\not
=n$. Hence  we have that in equation (3)
      $$
Z(0)=Z(\infty)=
\lim_{T\to \infty}       
 \int_{\Pi TM} e^{i
      \left(
  H+\Omega+
 T \left(h+\tilde \Omega\right) 
    \right)}dpdqd\xi=
\lim_{T\to \infty}       
  \int_{M} e^{i
      \left(
 H+Th \right)} 
{
T^n\tilde \Omega^n
\over n!}\,.
      $$ 

Notice that at the points $\{x_i\}$
 where vector field $\bf K$ vanishes,
artificial Hamiltonian 
$h=\omega {\cal c}\K$ and its first derivatives
vanish also:
    $$
     h\big\vert_{x=x_i}=
\omega_i K^i\big\vert_{x=x_i}=0\,, 
  {\p h\over \p x^k}\big\vert_{x=x_i}=
\tilde\Omega_{km}K^m\big\vert_{x=x_i}=0\,.
   \eqno (4)
    $$
Hence
calculating  the last integral using the stationary phase
method we come to
       $$
\lim_{T\to \infty}       
 \left[
         {T^n\over n!}
     \sum_{x_i}
         e^{iH(x_i)}  
\int_{M} e^{iT
      \left(
 h_{pq}(x_i)(x^p-x_i^p)(x^q-x_i^q)+
\dots\right)} 
\tilde \Omega^n
\right]=
       $$
       $$
     {1\over n!}
 \sum_{x_i}
         e^{iH(x_i)}
         {  
\sqrt
  {
\det \tilde \Omega
     \over
 \det 
Hessian \,\,{\rm of}\,\,h
    }
   }
 \big\vert_{x_i}\,.
    \eqno (5)
$$
Using equation (4) we see that
for Hessian  of $h$ at stationary points
     $$
  {\p^2 h\over \p x^p \p x^q}\big\vert_{x_i}=
  \tilde \Omega_{pr}{\p K^r\over \p
x^q}\big\vert_{x_i}\,,
       $$
thus at stationary points ${x_i}$
    $$      
 \det 
Hessian \,\,{\rm of}\,\,h
\big\vert_{x_i}
 =
\det\tilde\Omega\cdot \det dK_i.
      $$
THis means that  equation (5) implies the statement
(1a).


Finally we prove the lemma.

{\tt Proof of the lemma}
     $$
 {d Z(t)\over dt}=
 \int_{\Pi TM} it \sqrt {\L_{\bf K}}
      \omega
           e^{i
      \left(
  {H+\Omega}+
 t \sqrt {\L_{\bf K}\omega} 
    \right)}
    dpdqd\xi\,, 
            $$
and due to the $\bf K$ invariance of the form $\omega$
it is equal to 
$$
 {d Z(t)\over dt}=
 \int_{\Pi TM} it\sqrt {\L_{\bf K}}
    \left(
      \omega
           e^{i
      \left(
  {H+\Omega}+
 t \sqrt {\L_{\bf K}\omega} 
    \right)}
     \right)dpdqd\xi\,, 
       $$
One can see that the last integral vanishes
since $\sqrt {\L_\K}\sigma=d\sigma+\sigma 
{\cal c}\bf K$
Lemma is proved.

\medskip

\bye
