
\magnification=1200

\baselineskip=14pt

\def\l {\lambda}

       {\sl Dear Tian, yesterday we have  so interesting talk,
I absolutelyu forgot to speak little bit about Shur lemma.}

     \centerline {\bf Shur Lemma}
 
  It is the powerful tool in representation theory. The statement
and proof is very short and beautiful.

  Consider an arbitrary group $G$. We say
  that $T$ is linear representation of 
  the group $G$ in a vector space $V$ if
                     $$
  g\in G\mapsto T_g\colon\quad  V\to V \,\hbox {is linear map}\,\,
 {\rm and}\,\,     T_{g_1}\circ T_{g_2}=T_{g_1 g_2}\,.
                     $$
  We say that $U$ is invariant subspace of $V$
  if $T_g U\to U$ for every $g$.
   Sure, zero subspace and space $V$ itself are invariant subspaces.
  We call them trivial subspaces.

  We say that representation $[G,V]$ is {\it irreducible}
  if vector space $V$ does not possess non-trivial 
   invariant subspaces.
  
  Let $[G,M]$ and $[G,N]$ are two linear representations 
of a given group $G$ in vector spaces $M$ and $N$.

  We say that linear operator $F\colon M\to N$
is an interwinning operator if it commutes with an action
of group $G$:
                    $$
\forall g\in G\,, T_g\circ F=F\circ {\tilde T}_g\,. 
                    $$
   We say that two  representations are equivalent
   if there exist interwinning operator $F: M\to N$.


   

  {\bf Lemma} (Shur)
 
  Let ${G,M}$ be $[G,N]$ two {\it irreducible}
 linear representations  of a given group $G$ in finite-dimensional
vector spaces $M$ and $N$.  Then
  
1. These representations are not equivalent if  
$\dim M\not =\dim N$.   

2. If $M=N$ and basic field is algebraically closed
 then interwinning operator is identity operator
 (up to a scalar multiplier).

\medskip

   Shur lemma says that equivalent representations have the
same dimension, and intewinning operator is defined up to
a scalar operator.

\medskip

{\sl Proof}.  Let $[G,M]$ be $[G,N]$ two 
 linear representations  of a given group $G$ in vector 
spaces $M$ and $N$, and let $F\colon M\to N$ be an 
interwinning operator between these spaces.
  Then evidently kernel of $F$ and image
of $F$ are invariant subspaces.  This means
that kernel of $F$ is zero subspace in $M$ and 
image of $F$ is $N$, since these representations are
irreducible. Thus  $F$ is isomorphism of $M$ on $N$.

  Let $M=N$ and basic field be closed. 
     Suppose $F\colon M\to M$ is an interwinning operator.
  If $F\not 0$ then
   it  has at least one non-zero eigenvector
(with eigenvalue $\l$ which is a root of polynomial 
$\det (F-\l I)$).    Consider invariant 
subspace of $M$ which is a kernel
of interwinning operator $F-\l I$. 
It is not zero, since it contains the eigenvector.
 Hence it has to coincide with $M$, since representation 
is irreducible. Hence $F=\l I$. 
   


   $$ $$

   This lemma is a basis of representation theory.

   I just would like to give you one problem related with this lemma.

   {\bf Question}  Show that for an arbitrary $N$ there exist
 two $N\times N$ matrices $A,B$ such that for an arbitrary
     $N\times N$ matrix $C$, the condition
             $$
 [C,A]=CA-AC=[C,B]=CB-BC=0
             $$ 
  implies that $C$ is equal to scalar matrix $\l I$.



\bye
